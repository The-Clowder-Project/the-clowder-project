\input{preamble}

% OK, start here.
%
\usepackage{fontspec}
\let\hyperwhite\relax
\let\hyperred\relax
\newcommand{\hyperwhite}{\hypersetup{citecolor=white,filecolor=white,linkcolor=white,urlcolor=white}}
\newcommand{\hyperred}{%
\hypersetup{%
    citecolor=TitlingRed,%
    filecolor=TitlingRed,%
    linkcolor=TitlingRed,%
     urlcolor=TitlingRed%
}}
\let\ChapterRef\relax
\newcommand{\ChapterRef}[2]{#1}
\setcounter{tocdepth}{2}
%▓▓▓▓▓▓▓▓▓▓▓▓▓▓▓▓▓▓▓▓▓▓▓▓▓▓▓▓▓▓▓▓▓
%▓▓ ╔╦╗╦╔╦╗╦  ╔═╗  ╔═╗╔═╗╔╗╔╔╦╗ ▓▓
%▓▓  ║ ║ ║ ║  ║╣   ╠╣ ║ ║║║║ ║  ▓▓
%▓▓  ╩ ╩ ╩ ╩═╝╚═╝  ╚  ╚═╝╝╚╝ ╩  ▓▓
%▓▓▓▓▓▓▓▓▓▓▓▓▓▓▓▓▓▓▓▓▓▓▓▓▓▓▓▓▓▓▓▓▓
%\usepackage{titlesec}
%▓▓▓▓▓▓▓▓▓▓▓▓▓▓▓▓▓▓▓▓▓▓▓▓▓▓▓▓▓▓▓▓▓▓▓▓▓▓▓▓▓▓▓▓▓▓▓▓▓▓▓▓▓▓▓
%▓▓ ╔╦╗╔═╗╔╗ ╦  ╔═╗  ╔═╗╔═╗  ╔═╗╔═╗╔╗╔╔╦╗╔═╗╔╗╔╔╦╗╔═╗ ▓▓
%▓▓  ║ ╠═╣╠╩╗║  ║╣   ║ ║╠╣   ║  ║ ║║║║ ║ ║╣ ║║║ ║ ╚═╗ ▓▓
%▓▓  ╩ ╩ ╩╚═╝╩═╝╚═╝  ╚═╝╚    ╚═╝╚═╝╝╚╝ ╩ ╚═╝╝╚╝ ╩ ╚═╝ ▓▓
%▓▓▓▓▓▓▓▓▓▓▓▓▓▓▓▓▓▓▓▓▓▓▓▓▓▓▓▓▓▓▓▓▓▓▓▓▓▓▓▓▓▓▓▓▓▓▓▓▓▓▓▓▓▓▓
\newcommand{\ChapterTableOfContents}{%
    \begingroup
    \addfontfeature{Numbers={Lining,Monospaced}}
    \hypersetup{hidelinks}\tableofcontents%
    \endgroup
}%

\makeatletter
\newcommand \DotFill {\leavevmode \cleaders \hb@xt@ .33em{\hss .\hss }\hfill \kern \z@}
\makeatother

\definecolor{ToCGrey}{rgb}{0.4,0.4,0.4}
\definecolor{mainColor}{rgb}{0.82745098,0.18431373,0.18431373}
\usepackage{titletoc}
\titlecontents{part}
[0.0em]
{\addvspace{1pc}\color{TitlingRed}\large\bfseries\text{Part }}
{\bfseries\textcolor{TitlingRed}{\contentslabel{0.0em}}\hspace*{1.35em}}
{}
{\textcolor{TitlingRed}{{\hfill\bfseries\contentspage\nobreak}}}
[]
\titlecontents{section}
[0.0em]
{\addvspace{1pc}}
{\color{black}\bfseries\textcolor{TitlingRed}{\contentslabel{0.0em}}\hspace*{1.35em}}
{}
{\textcolor{black}{\textbf{\DotFill}{\bfseries\contentspage\nobreak}}}
[]
\titlecontents{subsection}
[0.0em]
{}
{\hspace*{1.35em}\color{ToCGrey}{\contentslabel{0.0em}}\hspace*{2.1em}}
{}
{{\textcolor{ToCGrey}\DotFill}\textcolor{ToCGrey}{\contentspage}\nobreak}
[]
\usepackage{marginnote}
\renewcommand*{\marginfont}{\normalfont}
\usepackage{inconsolata}
\setmonofont{inconsolata}%
\let\ChapterRef\relax
\newcommand{\ChapterRef}[2]{#1}
\AtBeginEnvironment{subappendices}{%%
    \section*{\huge Appendices}%
}%

\begin{document}

\title{Conditions on Relations}

\maketitle

\phantomsection
\label{section-phantom}

This chapter contains some material about reflexive, symmetric, transitive, equivalence, and apartness relations.
%Notably, we discuss and explore:
%\begin{enumerate}
%    \item Equivalence relations (\cref{section-equivalence-relations}) and quotient sets (\cref{section-quotients-by-equivalence-relations}).
%\end{enumerate}

\ChapterTableOfContents

\section{Functional and Total Relations}\label{section-functional-and-total-relations}
\subsection{Functional Relations}\label{subsection-functional-relations}
Let $A$ and $B$ be sets.
\begin{definition}{Functional Relations}{functional-relations}%
    A relation $R\colon A\rightproarrow B$ is \index[set-theory]{relation!functional}\textbf{functional} if, for each $a\in A$, the set $R(a)$ is either empty or a singleton.
\end{definition}
\begin{proposition}{Properties of Functional Relations}{properties-of-functional-relations}%
    Let $R\colon A\rightproarrow B$ be a relation.
    \begin{enumerate}
        \item\label{properties-of-functional-relations-characterisations}\SloganFont{Characterisations. }The following conditions are equivalent:
            \begin{enumerate}
                \item\label{properties-of-functional-relations-characterisations-1}The relation $R$ is functional.
                \item\label{properties-of-functional-relations-characterisations-2}We have $R\procirc R^{\dagger}\subset\chi_{B}$.
            \end{enumerate}
        %\item\label{properties-of-functional-relations-}\SloganFont{. }
    \end{enumerate}
\end{proposition}
\begin{Proof}{Proof of \cref{properties-of-functional-relations}}%
    \FirstProofBox{\cref{properties-of-functional-relations-characterisations}: Characterisations}%
    We claim that \cref{properties-of-functional-relations-characterisations-1,properties-of-functional-relations-characterisations-2} are indeed equivalent:
    \begin{itemize}
        \item\SloganFont{\cref{properties-of-functional-relations-characterisations-1}$\implies$\cref{properties-of-functional-relations-characterisations-2}: }Let $(b,b')\in B\times B$. We need to show that
            \[
                [R\procirc R^{\dagger}](b,b')%
                \preceq_{\TTV}%
                \chi_{B}(b,b'),%
            \]%
            i.e.\ that if there exists some $a\in A$ such that $b\sim_{R^{\dagger}}a$ and $a\sim_{R}b'$, then $b=b'$. But since $b\sim_{R^{\dagger}}a$ is the same as $a\sim_{R}b$, we have both $a\sim_{R}b$ and $a\sim_{R}b'$ at the same time, which implies $b=b'$ since $R$ is functional.
        \item\SloganFont{\cref{properties-of-functional-relations-characterisations-2}$\implies$\cref{properties-of-functional-relations-characterisations-1}: }Suppose that we have $a\sim_{R}b$ and $a\sim_{R}b'$ for $b,b'\in B$. We claim that $b=b'$:
            \begin{itemize}
                \item Since $a\sim_{R}b$, we have $b\sim_{R^{\dagger}}a$.
                \item Since $R\procirc R^{\dagger}\subset\chi_{B}$, we have
                    \[
                        [R\procirc R^{\dagger}](b,b')%
                        \preceq_{\TTV}%
                        \chi_{B}(b,b'),%
                    \]%
                    and since $b\sim_{R^{\dagger}}a$ and $a\sim_{R}b'$, it follows that $[R\procirc R^{\dagger}](b,b')=\true$, and thus $\chi_{B}(b,b')=\true$ as well, i.e.\ $b=b'$.
            \end{itemize}
    \end{itemize}
    This finishes the proof.
\end{Proof}
\subsection{Total Relations}\label{subsection-total-relations}
Let $A$ and $B$ be sets.
\begin{definition}{Total Relations}{total-relations}%
    A relation $R\colon A\rightproarrow B$ is \index[set-theory]{relation!total}\textbf{total} if, for each $a\in A$, we have $R(a)\neq\emptyset$.
\end{definition}
\begin{proposition}{Properties of Total Relations}{properties-of-total-relations}%
    Let $R\colon A\rightproarrow B$ be a relation.
    \begin{enumerate}
        \item\label{properties-of-total-relations-characterisations}\SloganFont{Characterisations. }The following conditions are equivalent:
            \begin{enumerate}
                \item\label{properties-of-total-relations-characterisations-1}The relation $R$ is total.
                \item\label{properties-of-total-relations-characterisations-2}We have $\chi_{A}\subset R^{\dagger}\procirc R$.
            \end{enumerate}
        %\item\label{properties-of-total-relations-}\SloganFont{. }
    \end{enumerate}
\end{proposition}
\begin{Proof}{Proof of \cref{properties-of-total-relations}}%
    \FirstProofBox{\cref{properties-of-total-relations-characterisations}: Characterisations}%
    We claim that \cref{properties-of-total-relations-characterisations-1,properties-of-total-relations-characterisations-2} are indeed equivalent:
    \begin{itemize}
        \item\SloganFont{\cref{properties-of-total-relations-characterisations-1}$\implies$\cref{properties-of-total-relations-characterisations-2}: }We have to show that, for each $(a,a')\in A$, we have
            \[
                \chi_{A}(a,a')%
                \preceq_{\TTV}%
                [R^{\dagger}\procirc R](a,a'),%
            \]%
            i.e.\ that if $a=a'$, then there exists some $b\in B$ such that $a\sim_{R}b$ and $b\sim_{R^{\dagger}}a'$ (i.e.\ $a\sim_{R}b$ again), which follows from the totality of $R$.
        \item\SloganFont{\cref{properties-of-total-relations-characterisations-2}$\implies$\cref{properties-of-total-relations-characterisations-1}: }Given $a\in A$, since $\chi_{A}\subset R^{\dagger}\procirc R$, we must have
            \[
                \{a\}%
                \subset%
                [R^{\dagger}\procirc R](a),%
            \]%
            implying that there must exist some $b\in B$ such that $a\sim_{R}b$ and $b\sim_{R^{\dagger}}a$ (i.e.\ $a\sim_{R}b$) and thus $R(a)\neq\emptyset$, as $b\in R(a)$.
    \end{itemize}
    This finishes the proof.
\end{Proof}
\section{Reflexive Relations}\label{section-reflexive-relations}
\subsection{Foundations}\label{subsection-reflexive-relations-foundations}
Let $A$ be a set.
\begin{definition}{Reflexive Relations}{reflexive-relations}%
    A \index[set-theory]{relation!reflexive}\textbf{reflexive relation} is equivalently:%
    %--- Begin Footnote ---%
    \footnote{%
        Note that since $\eRel(A,A)$ is posetal, reflexivity is a property of a relation, rather than extra structure.
        \par\vspace*{\TCBBoxCorrection}
    }%
    %---  End Footnote  ---%
    \begin{itemize}
        \item An $\E_{0}$-monoid in $(\NerveB(\eRel(A,A)),\chi_{A})$.
        \item A pointed object   in $(\eRel(A,A),\chi_{A})$.
    \end{itemize}
\end{definition}
\begin{remark}{Unwinding \cref{reflexive-relations}}{unwinding-reflexive-relations}%
    In detail, a relation $R$ on $A$ is \index[set-theory]{relation!reflexive}\textbf{reflexive} if we have an inclusion
    \[
        \eta_{R}%
        \colon%
        \chi_{A}%
        \subset%
        R%
    \]%
    of relations in $\eRel(A,A)$, i.e.\ if, for each $a\in A$, we have $a\sim_{R}a$.%
\end{remark}
\begin{definition}{The Po/Set of Reflexive Relations on a Set}{the-po-set-of-reflexive-relations-on-a-set}%
    Let $A$ be a set.
    \begin{enumerate}
        \item\label{the-po-set-of-reflexive-relations-on-a-set-1}The \index[set-theory]{relation!reflexive, set of}\textbf{set of reflexive relations on $A$} is the subset \index[notation]{RelreflA@$\Rel^{\rmrefl}(A,A)$}$\smash{\Rel^{\rmrefl}(A,A)}$ of $\Rel(A,A)$ spanned by the reflexive relations.
        \item\label{the-po-set-of-reflexive-relations-on-a-set-2}The \index[set-theory]{relation!reflexive, poset of}\textbf{poset of relations on $A$} is is the subposet \index[notation]{RelreflA@$\eRel^{\refl}(A,A)$}$\smash{\eRel^{\refl}(A,A)}$ of $\eRel(A,A)$ spanned by the reflexive relations.
    \end{enumerate}
\end{definition}
\begin{proposition}{Properties of Reflexive Relations}{properties-of-reflexive-relations}%
    Let $R$ and $S$ be relations on $A$.
    \begin{enumerate}
        \item\label{properties-of-reflexive-relations-interaction-with-inverses}\SloganFont{Interaction With Inverses. }If $R$ is reflexive, then so is $R^{\dagger}$.
        \item\label{properties-of-reflexive-relations-interaction-with-composition}\SloganFont{Interaction With Composition. }If $R$ and $S$ are reflexive, then so is $S\procirc R$.
        %\item\label{properties-of-reflexive-relations-}\SloganFont{. }
    \end{enumerate}
\end{proposition}
\begin{Proof}{Proof of \cref{properties-of-reflexive-relations}}%
    \FirstProofBox{\cref{properties-of-reflexive-relations-interaction-with-inverses}: Interaction With Inverses}%
    Clear.

    \ProofBox{\cref{properties-of-reflexive-relations-interaction-with-composition}: Interaction With Composition}%
    Clear.
\end{Proof}
\subsection{The Reflexive Closure of a Relation}\label{subsection-reflexive-relations-the-reflexive-closure-of-a-relation}
Let $R$ be a relation on $A$.
\begin{definition}{The Reflexive Closure of a Relation}{the-reflexive-closure-of-a-relation}%
    The \index[set-theory]{relation!reflexive closure of}\textbf{reflexive closure} of $\unsim_{R}$ is the relation \index[notation]{simreflR@$\unsim^{\rmrefl}_{R}$}$\smash{\unsim^{\rmrefl}_{R}}$%
    %--- Begin Footnote ---%
    \footnote{%
        \SloganFont{Further Notation: }Also written \index[notation]{Rrefl@$R^{\rmrefl}$}$R^{\rmrefl}$.
    } %
    %---  End Footnote  ---%
    satisfying the following universal property:%
    %--- Begin Footnote ---%
    \footnote{%
        \SloganFont{Slogan: }The reflexive closure of $R$ is the smallest reflexive relation containing $R$.
        \par\vspace*{\TCBBoxCorrection}
    }%
    %---  End Footnote  ---%
    \begin{itemize}
        \itemstar Given another reflexive relation $\unsim_{S}$ on $A$ such that $R\subset S$, there exists an inclusion $\smash{\unsim^{\rmrefl}_{R}}\subset\unsim_{S}$.
    \end{itemize}
\end{definition}
\begin{construction}{The Reflexive Closure of a Relation}{construction-of-the-reflexive-closure-of-a-relation}%
    Concretely, $\smash{\unsim^{\rmrefl}_{R}}$ is the free pointed object on $R$ in $(\eRel(A,A),\chi_{A})$%
    %--- Begin Footnote ---%
    \footnote{%
        Or, equivalently, the free $\E_{0}$-monoid on $R$ in $(\NerveB(\eRel(A,A)),\chi_{A})$.
        \par\vspace*{\TCBBoxCorrection}
    }, %
    %---  End Footnote  ---%
    being given by
    \begin{align*}
        R^{\rmrefl} &\defeq R\mathbin{\icoprod^{\eRel(A,A)}}\Delta_{A}\\
                    &=      R\cup\Delta_{A}\\
                    &=      \{(a,b)\in A\times A\ \middle|\ \text{we have $a\sim_{R}b$ or $a=b$}\}.%
    \end{align*}
\end{construction}
\begin{Proof}{Proof of \cref{construction-of-the-reflexive-closure-of-a-relation}}%
    Clear.
\end{Proof}
\begin{proposition}{Properties of the Reflexive Closure of a Relation}{properties-of-the-reflexive-closure-of-a-relation}%
    Let $R$ be a relation on $A$.
    \begin{enumerate}
        \item\label{properties-of-the-reflexive-closure-of-a-relation-adjointness}\SloganFont{Adjointness. }We have an adjunction
            \begin{webcompile}
                \Adjunction#(-)^{\rmrefl}#\Wasureru#\eRel(A,A)#\eRel^{\refl}(A,A),#
            \end{webcompile}
            witnessed by a bijection of sets%
            \[
                \eRel^{\refl}(R^{\rmrefl},S)
                \cong
                \eRel(R,S),
            \]%
            natural in $R\in\Obj(\eRel^{\refl}(A,A))$ and $S\in\Obj(\eRel(A,A))$.
        \item\label{properties-of-the-reflexive-closure-of-a-relation-the-reflexive-closure-of-a-reflexive-relation}\SloganFont{The Reflexive Closure of a Reflexive Relation. }If $R$ is reflexive, then $R^{\rmrefl}=R$.
        \item\label{properties-of-the-reflexive-closure-of-a-relation-idempotency}\SloganFont{Idempotency. }We have
            \[
                (R^{\rmrefl})^{\rmrefl}
                =
                R^{\rmrefl}.
            \]%
        \item\label{properties-of-the-reflexive-closure-of-a-relation-interaction-with-inverses}\SloganFont{Interaction With Inverses. }We have
            \begin{webcompile}
                \Big(R^{\dagger}\Big)^{\rmrefl}%
                =%
                \Big(R^{\rmrefl}\Big)^{\dagger},%
                \quad%
                \begin{tikzcd}[row sep={5.0*\the\DL,between origins}, column sep={8.0*\the\DL,between origins}, background color=backgroundColor, ampersand replacement=\&]
                    \Rel(A,A)
                    \arrow[r,"{(-)^{\rmrefl}}"]
                    \arrow[d,"{(-)^{\dagger}}"']
                    \&
                    \Rel(A,A)
                    \arrow[d,"{(-)^{\dagger}}"]
                    \\
                    \Rel(A,A)
                    \arrow[r,"{(-)^{\rmrefl}}"']
                    \&
                    \Rel(A,A)\mrp{.}
                \end{tikzcd}
            \end{webcompile}
        \item\label{properties-of-the-reflexive-closure-of-a-relation-interaction-with-composition}\SloganFont{Interaction With Composition. }We have
            \begin{webcompile}
                (S\procirc R)^{\rmrefl}%
                =%
                S^{\rmrefl}\procirc R^{\rmrefl},%
                \quad%
                \begin{tikzcd}[row sep={5.0*\the\DL,between origins}, column sep={9.0*\the\DL,between origins}, background color=backgroundColor, ampersand replacement=\&]
                    \Rel(A,A)\times\Rel(A,A)
                    \arrow[r,"\procirc"]
                    \arrow[d,"{(-)^{\rmrefl}}\times{(-)^{\rmrefl}}"']
                    \&
                    \Rel(A,A)
                    \arrow[d,"{(-)^{\rmrefl}}"]
                    \\
                    \Rel(A,A)\times\Rel(A,A)
                    \arrow[r,"\procirc"']
                    \&
                    \Rel(A,A)\mrp{.}
                \end{tikzcd}
            \end{webcompile}
        %\item\label{properties-of-the-reflexive-closure-of-a-relation-}\SloganFont{. }
    \end{enumerate}
\end{proposition}
\begin{Proof}{Proof of \cref{properties-of-the-reflexive-closure-of-a-relation}}%
    \FirstProofBox{\cref{properties-of-the-reflexive-closure-of-a-relation-adjointness}: Adjointness}%
    This is a rephrasing of the universal property of the reflexive closure of a relation, stated in \cref{the-reflexive-closure-of-a-relation}.

    \ProofBox{\cref{properties-of-the-reflexive-closure-of-a-relation-the-reflexive-closure-of-a-reflexive-relation}: The Reflexive Closure of a Reflexive Relation}%
    Clear.

    \ProofBox{\cref{properties-of-the-reflexive-closure-of-a-relation-idempotency}: Idempotency}%
    This follows from \cref{properties-of-the-reflexive-closure-of-a-relation-the-reflexive-closure-of-a-reflexive-relation}.

    \ProofBox{\cref{properties-of-the-reflexive-closure-of-a-relation-interaction-with-inverses}: Interaction With Inverses}%
    Clear.

    \ProofBox{\cref{properties-of-the-reflexive-closure-of-a-relation-interaction-with-composition}: Interaction With Composition}%
    This follows from \cref{properties-of-reflexive-relations-interaction-with-composition} of \cref{properties-of-reflexive-relations}.
\end{Proof}
\section{Symmetric Relations}\label{section-symmetric-relations}
\subsection{Foundations}\label{subsection-symmetric-relations-foundations}
Let $A$ be a set.
\begin{definition}{Symmetric Relations}{symmetric-relations}%
    A relation $R$ on $A$ is \index[set-theory]{relation!symmetric}\textbf{symmetric} if we have $R^{\dagger}=R$.
\end{definition}
\begin{remark}{Unwinding \cref{symmetric-relations}}{unwinding-symmetric-relations}%
    In detail, a relation $R$ is symmetric if it satisfies the following condition:
    \begin{itemize}
        \itemstar For each $a,b\in A$, if $a\sim_{R}b$, then $b\sim_{R}a$.
    \end{itemize}
\end{remark}
\begin{definition}{The Po/Set of Symmetric Relations on a Set}{the-po-set-of-symmetric-relations-on-a-set}%
    Let $A$ be a set.
    \begin{enumerate}
        \item\label{the-po-set-of-symmetric-relations-on-a-set-1}The \index[set-theory]{relation!symmetric, set of}\textbf{set of symmetric relations on $A$} is the subset \index[notation]{RelsymmA@$\Rel^{\rmsymm}(A,A)$}$\smash{\Rel^{\rmsymm}(A,A)}$ of $\Rel(A,A)$ spanned by the symmetric relations.
        \item\label{the-po-set-of-symmetric-relations-on-a-set-2}The \index[set-theory]{relation!poset of}\textbf{poset of relations on $A$} is is the subposet \index[notation]{RelsymmA@$\eRel^{\symm}(A,A)$}$\smash{\eRel^{\symm}(A,A)}$ of $\eRel(A,A)$ spanned by the symmetric relations.
    \end{enumerate}
\end{definition}
\begin{proposition}{Properties of Symmetric Relations}{properties-of-symmetric-relations}%
    Let $R$ and $S$ be relations on $A$.
    \begin{enumerate}
        \item\label{properties-of-symmetric-relations-interaction-with-inverses}\SloganFont{Interaction With Inverses. }If $R$ is symmetric, then so is $R^{\dagger}$.
        \item\label{properties-of-symmetric-relations-interaction-with-composition}\SloganFont{Interaction With Composition. }If $R$ and $S$ are symmetric, then so is $S\procirc R$.
        %\item\label{properties-of-symmetric-relations-}\SloganFont{. }
    \end{enumerate}
\end{proposition}
\begin{Proof}{Proof of \cref{properties-of-symmetric-relations}}%
    \FirstProofBox{\cref{properties-of-symmetric-relations-interaction-with-inverses}: Interaction With Inverses}%
    Clear.

    \ProofBox{\cref{properties-of-symmetric-relations-interaction-with-composition}: Interaction With Composition}%
    Clear.
\end{Proof}
\subsection{The Symmetric Closure of a Relation}\label{subsection-symmetric-relations-the-symmetric-closure-of-a-relation}
Let $R$ be a relation on $A$.
\begin{definition}{The Symmetric Closure of a Relation}{the-symmetric-closure-of-a-relation}%
    The \index[set-theory]{relation!symmetric closure of}\textbf{symmetric closure} of $\unsim_{R}$ is the relation \index[notation]{simsymmR@$\unsim^{\rmsymm}_{R}$}$\smash{\unsim^{\rmsymm}_{R}}$%
    %--- Begin Footnote ---%
    \footnote{%
        \SloganFont{Further Notation: }Also written \index[notation]{Rsymm@$R^{\rmsymm}$}$R^{\rmsymm}$.
    } %
    %---  End Footnote  ---%
    satisfying the following universal property:%
    %--- Begin Footnote ---%
    \footnote{%
        \SloganFont{Slogan: }The symmetric closure of $R$ is the smallest symmetric relation containing $R$.
        \par\vspace*{\TCBBoxCorrection}
    }%
    %---  End Footnote  ---%

    \begin{itemize}
        \itemstar Given another symmetric relation $\unsim_{S}$ on $A$ such that $R\subset S$, there exists an inclusion $\smash{\unsim^{\rmsymm}_{R}}\subset\unsim_{S}$.
    \end{itemize}
\end{definition}
\begin{construction}{The Symmetric Closure of a Relation}{construction-of-the-symmetric-closure-of-a-relation}%
    Concretely, $\smash{\unsim^{\rmsymm}_{R}}$ is the symmetric relation on $A$ defined by
    \begin{align*}
        R^{\rmsymm} &\defeq R\cup R^{\dagger}\\%
                    &=      \{(a,b)\in A\times A\ \middle|\ \text{we have $a\sim_{R}b$ or $b\sim_{R}a$}\}.%
    \end{align*}
\end{construction}
\begin{Proof}{Proof of \cref{construction-of-the-symmetric-closure-of-a-relation}}%
    Clear.
\end{Proof}
\begin{proposition}{Properties of the Symmetric Closure of a Relation}{properties-of-the-symmetric-closure-of-a-relation}%
    Let $R$ be a relation on $A$.
    \begin{enumerate}
        \item\label{properties-of-the-symmetric-closure-of-a-relation-adjointness}\SloganFont{Adjointness. }We have an adjunction
            \begin{webcompile}
                \Adjunction#(-)^{\rmsymm}#\Wasureru#\eRel(A,A)#\eRel^{\symm}(A,A),#
            \end{webcompile}
            witnessed by a bijection of sets%
            \[
                \eRel^{\symm}(R^{\rmsymm},S)
                \cong
                \eRel(R,S),
            \]%
            natural in $R\in\Obj(\eRel^{\symm}(A,A))$ and $S\in\Obj(\eRel(A,A))$.
        \item\label{properties-of-the-symmetric-closure-of-a-relation-the-symmetric-closure-of-a-symmetric-relation}\SloganFont{The Symmetric Closure of a Symmetric Relation. }If $R$ is symmetric, then $R^{\rmsymm}=R$.
        \item\label{properties-of-the-symmetric-closure-of-a-relation-idempotency}\SloganFont{Idempotency. }We have
            \[
                (R^{\rmsymm})^{\rmsymm}
                =
                R^{\rmsymm}.
            \]%
        \item\label{properties-of-the-symmetric-closure-of-a-relation-interaction-with-inverses}\SloganFont{Interaction With Inverses. }We have
            \begin{webcompile}%
                \Big(R^{\dagger}\Big)^{\rmsymm}%
                =%
                \Big(R^{\rmsymm}\Big)^{\dagger},%
                \quad
                \begin{tikzcd}[row sep={5.0*\the\DL,between origins}, column sep={8.5*\the\DL,between origins}, background color=backgroundColor, ampersand replacement=\&]
                    \Rel(A,A)
                    \arrow[r,"{(-)^{\rmsymm}}"]
                    \arrow[d,"{(-)^{\dagger}}"']
                    \&
                    \Rel(A,A)
                    \arrow[d,"{(-)^{\dagger}}"]
                    \\
                    \Rel(A,A)
                    \arrow[r,"{(-)^{\rmsymm}}"']
                    \&
                    \Rel(A,A)\mrp{.}
                \end{tikzcd}
            \end{webcompile}
        \item\label{properties-of-the-symmetric-closure-of-a-relation-interaction-with-composition}\SloganFont{Interaction With Composition. }We have
            \begin{webcompile}
                (S\procirc R)^{\rmsymm}%
                =%
                S^{\rmsymm}\procirc R^{\rmsymm},%
                \quad%
                \begin{tikzcd}[row sep={5.0*\the\DL,between origins}, column sep={9.0*\the\DL,between origins}, background color=backgroundColor, ampersand replacement=\&]
                    \Rel(A,A)\times\Rel(A,A)
                    \arrow[r,"\procirc"]
                    \arrow[d,"{(-)^{\rmsymm}}\times{(-)^{\rmsymm}}"']
                    \&
                    \Rel(A,A)
                    \arrow[d,"{(-)^{\rmsymm}}"]
                    \\
                    \Rel(A,A)\times\Rel(A,A)
                    \arrow[r,"\procirc"']
                    \&
                    \Rel(A,A)\mrp{.}
                \end{tikzcd}
            \end{webcompile}
        %\item\label{properties-of-the-symmetric-closure-of-a-relation-}\SloganFont{. }
    \end{enumerate}
\end{proposition}
\begin{Proof}{Proof of \cref{properties-of-the-symmetric-closure-of-a-relation}}%
    \FirstProofBox{\cref{properties-of-the-symmetric-closure-of-a-relation-adjointness}: Adjointness}%
    This is a rephrasing of the universal property of the symmetric closure of a relation, stated in \cref{the-symmetric-closure-of-a-relation}.

    \ProofBox{\cref{properties-of-the-symmetric-closure-of-a-relation-the-symmetric-closure-of-a-symmetric-relation}: The Symmetric Closure of a Symmetric Relation}%
    Clear.

    \ProofBox{\cref{properties-of-the-symmetric-closure-of-a-relation-idempotency}: Idempotency}%
    This follows from \cref{properties-of-the-symmetric-closure-of-a-relation-the-symmetric-closure-of-a-symmetric-relation}.

    \ProofBox{\cref{properties-of-the-symmetric-closure-of-a-relation-interaction-with-inverses}: Interaction With Inverses}%
    Clear.

    \ProofBox{\cref{properties-of-the-symmetric-closure-of-a-relation-interaction-with-composition}: Interaction With Composition}%
    This follows from \cref{properties-of-symmetric-relations-interaction-with-composition} of \cref{properties-of-symmetric-relations}.
\end{Proof}
\section{Transitive Relations}\label{section-transitive-relations}
\subsection{Foundations}\label{subsection-transitive-relations-foundations}
Let $A$ be a set.
\begin{definition}{Transitive Relations}{transitive-relations}%
    A \index[set-theory]{relation!transitive}\textbf{transitive relation} is equivalently:%
    %--- Begin Footnote ---%
    \footnote{%
        Note that since $\eRel(A,A)$ is posetal, transitivity is a property of a relation, rather than extra structure.
        \par\vspace*{\TCBBoxCorrection}
    }%
    %---  End Footnote  ---%
    \begin{itemize}
        \item A non-unital $\E_{1}$-monoid in $(\NerveB(\eRel(A,A)),\procirc)$.
        \item A non-unital monoid          in $(\eRel(A,A),\procirc)$.
    \end{itemize}
\end{definition}
\begin{remark}{Unwinding \cref{transitive-relations}}{unwinding-transitive-relations}%
    In detail, a relation $R$ on $A$ is \textbf{transitive} if we have an inclusion
    \[
        \mu_{R}%
        \colon%
        R\procirc R%
        \subset%
        R%
    \]%
    of relations in $\eRel(A,A)$, i.e.\ if, for each $a,c\in A$, the following condition is satisfied:
    \begin{itemize}
        \itemstar If there exists some $b\in A$ such that $a\sim_{R}b$ and $b\sim_{R}c$, then $a\sim_{R}c$.
    \end{itemize}
\end{remark}
\begin{definition}{The Po/Set of Transitive Relations on a Set}{the-po-set-of-transitive-relations-on-a-set}%
    Let $A$ be a set.
    \begin{enumerate}
        \item\label{the-po-set-of-transitive-relations-on-a-set-1}The \index[set-theory]{relation!transitive, set of}\textbf{set of transitive relations from $A$ to $B$} is the subset \index[notation]{ReltransA@$\Rel^{\rmtrans}(A)$}$\smash{\Rel^{\rmtrans}(A)}$ of $\Rel(A,A)$ spanned by the transitive relations.
        \item\label{the-po-set-of-transitive-relations-on-a-set-2}The \index[set-theory]{relation!poset of}\textbf{poset of relations from $A$ to $B$} is is the subposet \index[notation]{ReltransA@$\eRel^{\trans}(A)$}$\smash{\eRel^{\trans}(A)}$ of $\eRel(A,A)$ spanned by the transitive relations.
    \end{enumerate}
\end{definition}
\begin{proposition}{Properties of Transitive Relations}{properties-of-transitive-relations}%
    Let $R$ and $S$ be relations on $A$.
    \begin{enumerate}
        \item\label{properties-of-transitive-relations-interaction-with-inverses}\SloganFont{Interaction With Inverses. }If $R$ is transitive, then so is $R^{\dagger}$.
        \item\label{properties-of-transitive-relations-interaction-with-composition}\SloganFont{Interaction With Composition. }If $R$ and $S$ are transitive, then $S\procirc R$ \demph{may fail to be transitive}.
        %\item\label{properties-of-transitive-relations-}\SloganFont{. }
    \end{enumerate}
\end{proposition}
\begin{Proof}{Proof of \cref{properties-of-transitive-relations}}%
    \FirstProofBox{\cref{properties-of-transitive-relations-interaction-with-inverses}: Interaction With Inverses}%
    Clear.

    \ProofBox{\cref{properties-of-transitive-relations-interaction-with-composition}: Interaction With Composition}%
    See \cite{MSE2096272}.%
    %--- Begin Footnote ---%
    \footnote{%
        \SloganFont{Intuition: }Transitivity for $R$ and $S$ fails to imply that of $S\procirc R$ because the composition operation for relations intertwines $R$ and $S$ in an incompatible way:
        \begin{itemize}
            \item If $a\sim_{S\procirc R}c$ and $c\sim_{S\procirc r}e$, then:
                \begin{itemize}
                    \item There is some $b\in A$ such that:
                        \begin{itemize}
                            \item $a\sim_{R}b$;
                            \item $b\sim_{S}c$;
                        \end{itemize}
                    \item There is some $d\in A$ such that:
                        \begin{itemize}
                            \item $c\sim_{R}d$;
                            \item $d\sim_{S}e$.
                        \end{itemize}
                \end{itemize}
        \end{itemize}
        \par\vspace*{\TCBBoxCorrection}
    }%
    %---  End Footnote  ---%
\end{Proof}
\subsection{The Transitive Closure of a Relation}\label{subsection-transitive-relations-the-transitive-closure-of-a-relation}
Let $R$ be a relation on $A$.
\begin{definition}{The Transitive Closure of a Relation}{the-transitive-closure-of-a-relation}%
    The \index[set-theory]{relation!transitive closure of}\textbf{transitive closure} of $\unsim_{R}$ is the relation \index[notation]{simtransR@$\unsim^{\rmtrans}_{R}$}$\smash{\unsim^{\rmtrans}_{R}}$%
    %--- Begin Footnote ---%
    \footnote{%
        \SloganFont{Further Notation: }Also written \index[notation]{Rtrans@$R^{\rmtrans}$}$R^{\rmtrans}$.
    } %
    %---  End Footnote  ---%
    satisfying the following universal property:%
    %--- Begin Footnote ---%
    \footnote{%
        \SloganFont{Slogan: }The transitive closure of $R$ is the smallest transitive relation containing $R$.
        \par\vspace*{\TCBBoxCorrection}
    }%
    %---  End Footnote  ---%
    \begin{itemize}
        \itemstar Given another transitive relation $\unsim_{S}$ on $A$ such that $R\subset S$, there exists an inclusion $\smash{\unsim^{\rmtrans}_{R}}\subset\unsim_{S}$.
    \end{itemize}
\end{definition}
\begin{construction}{The Transitive Closure of a Relation}{construction-of-the-transitive-closure-of-a-relation}%
    Concretely, $\smash{\unsim^{\rmtrans}_{R}}$ is the free non-unital monoid on $R$ in $(\eRel(A,A),\procirc)$%
    %--- Begin Footnote ---%
    \footnote{%
        Or, equivalently, the free non-unital $\E_{1}$-monoid on $R$ in $(\NerveB(\eRel(A,A)),\procirc)$.
        \par\vspace*{\TCBBoxCorrection}
    }, %
    %---  End Footnote  ---%
    being given by
    \begin{align*}
        R^{\rmtrans} &\defeq \coprod_{n=1}^{\infty}R^{\procirc n}\\%
                     &\defeq \bigcup_{n=1}^{\infty}R^{\procirc n}\\%
                     &\defeq \{
                                 (a,b)\in A\times B%
                                 \ \middle|\ %
                                 \begin{aligned}
                                     &\text{there exists some $(x_{1},\ldots,x_{n})\in R^{\times n}$}\\
                                     &\text{such that $a\sim_{R}x_{1}\sim_{R}\cdots\sim_{R}x_{n}\sim_{R}b$}%
                                 \end{aligned}
                             \}.%
    \end{align*}
\end{construction}
\begin{Proof}{Proof of \cref{construction-of-the-transitive-closure-of-a-relation}}%
    Clear.
\end{Proof}
\begin{proposition}{Properties of the Transitive Closure of a Relation}{properties-of-the-transitive-closure-of-a-relation}%
    Let $R$ be a relation on $A$.
    \begin{enumerate}
        \item\label{properties-of-the-transitive-closure-of-a-relation-adjointness}\SloganFont{Adjointness. }We have an adjunction
            \begin{webcompile}
                \Adjunction#(-)^{\rmtrans}#\Wasureru#\eRel(A,A)#\eRel^{\trans}(A,A),#
            \end{webcompile}
            witnessed by a bijection of sets%
            \[
                \eRel^{\trans}(R^{\rmtrans},S)
                \cong
                \eRel(R,S),
            \]%
            natural in $R\in\Obj(\eRel^{\trans}(A,A))$ and $S\in\Obj(\eRel(A,B))$.
        \item\label{properties-of-the-transitive-closure-of-a-relation-the-transitive-closure-of-a-transitive-relation}\SloganFont{The Transitive Closure of a Transitive Relation. }If $R$ is transitive, then $R^{\rmtrans}=R$.
        \item\label{properties-of-the-transitive-closure-of-a-relation-idempotency}\SloganFont{Idempotency. }We have
            \[
                (R^{\rmtrans})^{\rmtrans}
                =
                R^{\rmtrans}.
            \]%
        \item\label{properties-of-the-transitive-closure-of-a-relation-interaction-with-inverses}\SloganFont{Interaction With Inverses. }We have
            \begin{webcompile}
                \Big(R^{\dagger}\Big)^{\rmtrans}%
                =%
                \Big(R^{\rmtrans}\Big)^{\dagger},%
                \quad%
                \begin{tikzcd}[row sep={5.0*\the\DL,between origins}, column sep={8.5*\the\DL,between origins}, background color=backgroundColor, ampersand replacement=\&]
                    \Rel(A,A)
                    \arrow[r,"{(-)^{\rmtrans}}"]
                    \arrow[d,"{(-)^{\dagger}}"']
                    \&
                    \Rel(A,A)
                    \arrow[d,"{(-)^{\dagger}}"]
                    \\
                    \Rel(A,A)
                    \arrow[r,"{(-)^{\rmtrans}}"']
                    \&
                    \Rel(A,A)\mrp{.}
                \end{tikzcd}
            \end{webcompile}
        \item\label{properties-of-the-transitive-closure-of-a-relation-interaction-with-composition}\SloganFont{Interaction With Composition. }We have
            \begin{webcompile}
                (S\procirc R)^{\rmtrans}%
                \nneq%
                S^{\rmtrans}\procirc R^{\rmtrans},%%
                \quad%
                \begin{tikzcd}[row sep={5.0*\the\DL,between origins}, column sep={9.0*\the\DL,between origins}, background color=backgroundColor, ampersand replacement=\&]
                    \Rel(A,A)\times\Rel(A,A)
                    \arrow[r,"\procirc"]
                    \arrow[d,"{(-)^{\rmtrans}}\times{(-)^{\rmtrans}}"']
                    \&
                    \Rel(A,A)
                    \arrow[d,"{(-)^{\rmtrans}}"]
                    \\
                    \Rel(A,A)\times\Rel(A,A)
                    \arrow[r,"\procirc"']
                    \&
                    \Rel(A,A)\mrp{.}
                    % Symbols
                    \arrow[from=2-2,to=1-2,"{\ncomm}"description,phantom,shorten=0.5*\the\DL,Rightarrow,xshift=-4.5*\the\DL]%
                \end{tikzcd}
            \end{webcompile}
        %\item\label{properties-of-the-transitive-closure-of-a-relation-}\SloganFont{. }
    \end{enumerate}
\end{proposition}
\begin{Proof}{Proof of \cref{properties-of-the-transitive-closure-of-a-relation}}%
    \FirstProofBox{\cref{properties-of-the-transitive-closure-of-a-relation-adjointness}: Adjointness}%
    This is a rephrasing of the universal property of the transitive closure of a relation, stated in \cref{the-transitive-closure-of-a-relation}.

    \ProofBox{\cref{properties-of-the-transitive-closure-of-a-relation-the-transitive-closure-of-a-transitive-relation}: The Transitive Closure of a Transitive Relation}%
    Clear.

    \ProofBox{\cref{properties-of-the-transitive-closure-of-a-relation-idempotency}: Idempotency}%
    This follows from \cref{properties-of-the-transitive-closure-of-a-relation-the-transitive-closure-of-a-transitive-relation}.

    \ProofBox{\cref{properties-of-the-transitive-closure-of-a-relation-interaction-with-inverses}: Interaction With Inverses}%
    We have
    \begin{align*}
        (R^{\dagger})^{\rmtrans} &= \bigcup_{n=1}^{\infty}(R^{\dagger})^{\procirc n}\\
                                 &= \bigcup_{n=1}^{\infty}(R^{\procirc n})^{\dagger}\\
                                 &= (\bigcup_{n=1}^{\infty}R^{\procirc n})^{\dagger}\\
                                 &= (R^{\rmtrans})^{\dagger},
    \end{align*}
    where we have used, respectively:
    \begin{itemize}
        \item \cref{construction-of-the-transitive-closure-of-a-relation}.
        \item \ChapterRef{\ChapterConstructionsWithRelations, \cref{constructions-with-relations:properties-of-composition-of-relations-interaction-with-inverses} of \cref{constructions-with-relations:properties-of-composition-of-relations}}{\cref{properties-of-composition-of-relations-interaction-with-inverses} of \cref{properties-of-composition-of-relations}}.
        \item \ChapterRef{\ChapterConstructionsWithRelations, \cref{constructions-with-relations:properties-of-unions-of-families-of-relations-interaction-with-inverses} of \cref{constructions-with-relations:properties-of-unions-of-families-of-relations}}{\cref{properties-of-unions-of-families-of-relations-interaction-with-inverses} of \cref{properties-of-unions-of-families-of-relations}}.
        \item \cref{construction-of-the-transitive-closure-of-a-relation}.
    \end{itemize}
    This finishes the proof.

    \ProofBox{\cref{properties-of-the-transitive-closure-of-a-relation-interaction-with-composition}: Interaction With Composition}%
    This follows from \cref{properties-of-transitive-relations-interaction-with-composition} of \cref{properties-of-transitive-relations}.
\end{Proof}
\section{Equivalence Relations}\label{section-equivalence-relations}
\subsection{Foundations}\label{subsection-equivalence-relations-foundations}
Let $A$ be a set.
\begin{definition}{Equivalence Relations}{equivalence-relations}%
    A relation $R$ is an \index[set-theory]{relation!equivalence relation}\textbf{equivalence relation} if it is reflexive, symmetric, and transitive.%
    %--- Begin Footnote ---%
    \footnote{%
        \SloganFont{Further Terminology: }If instead $R$ is just symmetric and transitive, then it is called a \index[set-theory]{relation!partial equivalence relation}\textbf{partial equivalence relation}.
        \par\vspace*{\TCBBoxCorrection}
    }%
    %---  End Footnote  ---%
\end{definition}
\begin{example}{The Kernel of a Function}{the-kernel-of-a-function}%
    The \index[set-theory]{equivalence relation!kernel}\index[set-theory]{function!kernel of}\textbf{kernel of a function $f\colon A\to B$} is the equivalence relation $\unsim_{\Ker(f)}$ on $A$ obtained by declaring $a\sim_{\Ker(f)}b$ \textiff $f(a)=f(b)$.%
    %--- Begin Footnote ---%
    \footnote{%
        The kernel $\Ker(f)\colon A\rightproarrow A$ of $f$ is the underlying functor of the monad induced by the adjunction $\Gr(f)\dashv f^{-1}\colon A\rightleftadjointarrows B$ in $\sfbfRel$ of \ChapterRef{\ChapterConstructionsWithRelations, \cref{constructions-with-relations:properties-of-graphs-of-functions-adjointness-inside-sfbfrel} of \cref{constructions-with-relations:properties-of-graphs-of-functions}}{\cref{properties-of-graphs-of-functions-adjointness-inside-sfbfrel} of \cref{properties-of-graphs-of-functions}}.
        \par\vspace*{\TCBBoxCorrection}
    }%
    %---  End Footnote  ---%
\end{example}
\begin{definition}{The Po/Set of Equivalence Relations on a Set}{the-po-set-of-equivalence-relations-on-a-set}%
    Let $A$ and $B$ be sets.
    \begin{enumerate}
        \item\label{the-po-set-of-equivalence-relations-on-a-set-1}The \index[set-theory]{relation!equivalence, set of}\textbf{set of equivalence relations from $A$ to $B$} is the subset \index[notation]{ReleqAB@$\Rel^{\rmeq}(A,B)$}$\smash{\Rel^{\rmeq}(A,B)}$ of $\Rel(A,B)$ spanned by the equivalence relations.
        \item\label{the-po-set-of-equivalence-relations-on-a-set-2}The \index[set-theory]{relation!poset of}\textbf{poset of relations from $A$ to $B$} is is the subposet \index[notation]{ReleqAB@$\eRel^{\eq}(A,B)$}$\smash{\eRel^{\eq}(A,B)}$ of $\eRel(A,B)$ spanned by the equivalence relations.
    \end{enumerate}
\end{definition}
\subsection{The Equivalence Closure of a Relation}\label{subsection-equivalence-relations-the-equivalence-closure-of-a-relation}
Let $R$ be a relation on $A$.
\begin{definition}{The Equivalence Closure of a Relation}{the-equivalence-closure-of-a-relation}%
    The \index[set-theory]{relation!equivalence closure of}\textbf{equivalence closure}%
    %--- Begin Footnote ---%
    \footnote{%
        \SloganFont{Further Terminology: }Also called the \textbf{equivalence relation associated to $\unsim_{R}$}.
    } %
    %---  End Footnote  ---%
    of $\unsim_{R}$ is the relation \index[notation]{simtransR@$\unsim^{\rmeq}_{R}$}$\smash{\unsim^{\rmeq}_{R}}$%
    %--- Begin Footnote ---%
    \footnote{%
        \SloganFont{Further Notation: }Also written \index[notation]{Req@$R^{\eq}$}$R^{\eq}$.
    } %
    %---  End Footnote  ---%
    satisfying the following universal property:%
    %--- Begin Footnote ---%
    \footnote{%
        \SloganFont{Slogan: }The equivalence closure of $R$ is the smallest equivalence relation containing $R$.
        \par\vspace*{\TCBBoxCorrection}
    }%
    %---  End Footnote  ---%
    \begin{itemize}
        \itemstar Given another equivalence relation $\unsim_{S}$ on $A$ such that $R\subset S$, there exists an inclusion $\smash{\unsim^{\rmeq}_{R}}\subset\unsim_{S}$.
    \end{itemize}
\end{definition}
\begin{construction}{The Equivalence Closure of a Relation}{construction-of-the-equivalence-closure-of-a-relation}%
    Concretely, $\smash{\unsim^{\rmeq}_{R}}$ is the equivalence relation on $A$ defined by
    \begin{align*}
        R^{\rmeq} &\defeq ((R^{\rmrefl})^{\rmsymm})^{\rmtrans}\\%
                  &=      ((R^{\rmsymm})^{\rmtrans})^{\rmrefl}\\%
                  &=      \{
                               (a,b)\in A\times B%
                               \ \middle|\ %
                               \begin{aligned}
                                   &\text{there exists $(x_{1},\ldots,x_{n})\in R^{\times n}$ satisfying at}\\[-2.5pt]
                                   &\text{least one of the following conditions:}\\[7.5pt]
                                   &\mspace{25mu}\rlap{\text{1.}}\mspace{22.5mu}\text{The following conditions are satisfied:}\\[7.5pt]%
                                   &\mspace{50mu}\rlap{\text{\norg(a\norg)}}\mspace{30mu}\text{We have $a\sim_{R}x_{1}$       or $x_{1}\sim_{R}a$;}\\
                                   &\mspace{50mu}\rlap{\text{\norg(b\norg)}}\mspace{30mu}\text{We have $x_{i}\sim_{R}x_{i+1}$ or $x_{i+1}\sim_{R}x_{i}$}\\[-2.5pt]
                                   &\mspace{81.25mu}\text{for each $1\leq i\leq n-1$;}\\
                                   &\mspace{50mu}\rlap{\text{\norg(c\norg)}}\mspace{30mu}\text{We have $b\sim_{R}x_{n}$       or $x_{n}\sim_{R}b$;}\\[7.5pt]%
                                   &\mspace{25mu}\rlap{\text{2.}}\mspace{22.5mu}\text{We have $a=b$.}%
                               \end{aligned}
                           \}.%
    \end{align*}
\end{construction}
\begin{Proof}{Proof of \cref{construction-of-the-equivalence-closure-of-a-relation}}%
    From the universal properties of the reflexive, symmetric, and transitive closures of a relation (\cref{the-reflexive-closure-of-a-relation,the-symmetric-closure-of-a-relation,the-transitive-closure-of-a-relation}), we see that it suffices to prove that:%
    \begin{enumerate}
        \item\label{proof-of-construction-of-the-equivalence-closure-of-a-relation-item-1}The symmetric  closure of a reflexive relation is still reflexive.
        \item\label{proof-of-construction-of-the-equivalence-closure-of-a-relation-item-2}The transitive closure of a symmetric relation is still symmetric.
    \end{enumerate}
    which are both clear.
\end{Proof}
\begin{proposition}{Properties of Equivalence Relations}{properties-of-the-equivalence-closure-of-a-relation}%
    Let $R$ be a relation on $A$.
    \begin{enumerate}
        \item\label{properties-of-the-equivalence-closure-of-a-relation-adjointness}\SloganFont{Adjointness. }We have an adjunction
            \begin{webcompile}
                \Adjunction#(-)^{\rmeq}#\Wasureru#\eRel(A,B)#\eRel^{\eq}(A,B),#
            \end{webcompile}
            witnessed by a bijection of sets%
            \[
                \eRel^{\eq}(R^{\rmeq},S)
                \cong
                \eRel(R,S),
            \]%
            natural in $R\in\Obj(\eRel^{\eq}(A,B))$ and $S\in\Obj(\eRel(A,B))$.
        \item\label{properties-of-the-equivalence-closure-of-a-relation-the-equivalence-closure-of-a-equivalence-relation}\SloganFont{The Equivalence Closure of an Equivalence Relation. }If $R$ is an equivalence relation, then $R^{\rmeq}=R$.
        \item\label{properties-of-the-equivalence-closure-of-a-relation-idempotency}\SloganFont{Idempotency. }We have
            \[
                (R^{\rmeq})^{\rmeq}
                =
                R^{\rmeq}.
            \]%
        %\item\label{properties-of-the-equivalence-closure-of-a-relation-}\SloganFont{. }
    \end{enumerate}
\end{proposition}
\begin{Proof}{Proof of \cref{properties-of-the-equivalence-closure-of-a-relation}}%
    \FirstProofBox{\cref{properties-of-the-equivalence-closure-of-a-relation-adjointness}: Adjointness}%
    This is a rephrasing of the universal property of the equivalence closure of a relation, stated in \cref{the-equivalence-closure-of-a-relation}.

    \ProofBox{\cref{properties-of-the-equivalence-closure-of-a-relation-the-equivalence-closure-of-a-equivalence-relation}: The Equivalence Closure of an Equivalence Relation}%
    Clear.

    \ProofBox{\cref{properties-of-the-equivalence-closure-of-a-relation-idempotency}: Idempotency}%
    This follows from \cref{properties-of-the-equivalence-closure-of-a-relation-the-equivalence-closure-of-a-equivalence-relation}.
\end{Proof}
\section{Quotients by Equivalence Relations}\label{section-quotients-by-equivalence-relations}
\subsection{Equivalence Classes}\label{subsection-quotients-by-equivalence-relations-equivalence-classes}
Let $A$ be a set, let $R$ be a relation on $A$, and let $a\in A$.
\begin{definition}{Equivalence Classes}{equivalence-classes}%
    The \index[set-theory]{equivalence class}\textbf{equivalence class associated to $a$} is the set \index[notation]{a@$[a]$}$[a]$ defined by%
    \begin{align*}
        [a] &\defeq \{x\in X\ \middle|\ x\sim_{R}a\}\\%
            &=      \{x\in X\ \middle|\ a\sim_{R}x\}\tag{since $R$ is symmetric}.%
    \end{align*}
\end{definition}
\subsection{Quotients of Sets by Equivalence Relations}\label{subsection-quotients-by-equivalence-relations-quotients-of-sets-by-equivalence-relations}
Let $A$ be a set and let $R$ be a relation on $A$.
\begin{definition}{Quotients of Sets by Equivalence Relations}{quotients-of-sets-by-equivalence-relations}%
    The \index[set-theory]{quotient!by an equivalence relation}\textbf{quotient of $X$ by $R$} is the set \index[notation]{XmodR@$X/\unsim_{R}$}$X/\unsim_{R}$ defined by%
    \[
        X/\unsim_{R}%
        \defeq%
        \{%
            [a]\in\mathcal{P}(X)%
            \ \middle|\ %
            a\in X%
        \}.%
    \]%
\end{definition}
\begin{remark}{Why Use \say{Equivalence} Relations for Quotient Sets}{why-use-equivalence-relations-for-quotient-sets}%
    The reason we define quotient sets for equivalence relations only is that each of the properties of being an equivalence relation---reflexivity, symmetry, and transitivity---ensures that the equivalences classes $[a]$ of $X$ under $R$ are well-behaved:
    \begin{itemize}
        \item\SloganFont{Reflexivity. }If $R$ is reflexive, then, for each $a\in X$, we have $a\in[a]$.
        \item\SloganFont{Symmetry. }The equivalence class $[a]$ of an element $a$ of $X$ is defined by
            \[
                [a]%
                \defeq%
                \{x\in X\ \middle|\ x\sim_{R}a\},%
            \]%
            but we could equally well define
            \[
                [a]'%
                \defeq%
                \{x\in X\ \middle|\ a\sim_{R}x\}%
            \]%
            instead. This is not a problem when $R$ is symmetric, as we then have $[a]=[a]'$.%
            %--- Begin Footnote ---%
            \footnote{%
                When categorifying equivalence relations, one finds that $[a]$ and $[a]'$ correspond to presheaves and copresheaves; see \ChapterRef{\ChapterConstructionsWithCategories, \cref{constructions-with-categories:equivalence-classes-of-objects-by-profunctors}}{\cref{equivalence-classes-of-objects-by-profunctors}}.
                \par\vspace*{\TCBBoxCorrection}
            }%
            %---  End Footnote  ---%
        \item\SloganFont{Transitivity. }If $R$ is transitive, then $[a]$ and $[b]$ are disjoint \textiff $a\nsim_{R}b$, and equal otherwise.
    \end{itemize}
\end{remark}
\begin{proposition}{Properties of Quotient Sets}{properties-of-quotient-sets}%
    Let $f\colon X\to Y$ be a function and let $R$ be a relation on $X$.
    \begin{enumerate}
        \item\label{properties-of-quotient-sets-as-a-coequaliser}\SloganFont{As a Coequaliser. }We have an isomorphism of sets%
            \[
                X/\unsim^{\rmeq}_{R}%
                \cong
                \CoEq(R\hookrightarrow X\times X\xrightrightarrows{\pr_{1}}{\pr_{2}}X),%
            \]%
            where $\unsim^{\rmeq}_{R}$ is the equivalence relation generated by $\unsim_{R}$.
        \item\label{properties-of-quotient-sets-as-a-pushout}\SloganFont{As a Pushout. }We have an isomorphism of sets%
            %--- Begin Footnote ---%
            \footnote{%
                Dually, we also have an isomorphism of sets
                \begin{webcompile}
                    \Eq(\pr_{1},\pr_{2})%
                    \cong
                    X\times_{X/\unsim^{\rmeq}_{R}}X,%
                    \quad
                    \begin{tikzcd}[row sep={5.0*\the\DL,between origins}, column sep={5.0*\the\DL,between origins}, background color=backgroundColor, ampersand replacement=\&]
                        \Eq(\pr_{1},\pr_{2})%
                        \arrow[r]
                        \arrow[d]
                        \arrow[rd,very near start,phantom,"\lrcorner"]
                        \&
                        X
                        \arrow[d]
                        \\
                        X
                        \arrow[r]
                        \&
                        X/\unsim^{\rmeq}_{R}\mrp{.}
                    \end{tikzcd}
                \end{webcompile}
                \par\vspace*{\TCBBoxCorrection}
            }%
            %---  End Footnote  ---%
            \begin{webcompile}
                X/\unsim^{\rmeq}_{R}%
                \cong
                X\ipushout{\Eq(\pr_{1},\pr_{2})}X,%
                \quad
                \begin{tikzcd}[row sep={5.0*\the\DL,between origins}, column sep={5.0*\the\DL,between origins}, background color=backgroundColor, ampersand replacement=\&]
                    X/\unsim^{\rmeq}_{R}
                    \arrow[from=r]
                    \arrow[from=d]
                    \arrow[rd,very near start,phantom,"\ulcorner"]
                    \&
                    X
                    \arrow[from=d]
                    \\
                    X
                    \arrow[from=r]
                    \&
                    \Eq(\pr_{1},\pr_{2})\mrp{.}
                \end{tikzcd}
            \end{webcompile}
            where $\unsim^{\rmeq}_{R}$ is the equivalence relation generated by $\unsim_{R}$.
        \item\label{properties-of-quotient-sets-the-first-isomorphism-theorem-for-sets}\SloganFont{The First Isomorphism Theorem for Sets. }We have an isomorphism of sets%
            %--- Begin Footnote ---%
            \footnote{%
                \SloganFont{Further Terminology: }The set $X/\unsim_{\Ker(f)}$ is often called the \index[set-theory]{coimage}\index[set-theory]{function!coimage of}\textbf{coimage of $f$}, and denoted by \index[notation]{CoImf@$\CoIm(f)$}$\CoIm(f)$.
            }%
            %---  End Footnote  ---%
            %--- Begin Footnote ---%
            \footnote{%
                In a sense this is a result relating the monad in $\sfbfRel$ induced by $f$ with the comonad in $\sfbfRel$ induced by $f$, as the kernel and image
                \begin{gather*}
                    \Ker(f)\colon X\rightproarrow X,\\
                    \Im(f)\subset Y
                \end{gather*}
                of $f$ are the underlying functors of (respectively) the induced monad and comonad of the adjunction
                \begin{webcompile}
                    \RelAdjunctionShortSize#4.0#\Gr(f)#f^{-1}#A#B#
                \end{webcompile}
                of \ChapterRef{\ChapterConstructionsWithRelations, \cref{constructions-with-relations:properties-of-graphs-of-functions-adjointness-inside-sfbfrel} of \cref{constructions-with-relations:properties-of-graphs-of-functions}}{\cref{properties-of-graphs-of-functions-adjointness-inside-sfbfrel} of \cref{properties-of-graphs-of-functions}}.
            }%
            %---  End Footnote  ---%
            \[
                X/\unsim_{\Ker(f)}
                \cong
                \Im(f).
            \]%
        \item\label{properties-of-quotient-sets-descending-functions-to-quotient-sets-1}\SloganFont{Descending Functions to Quotient Sets, \rmI. }Let $R$ be an equivalence relation on $X$. The following conditions are equivalent:
            \begin{enumerate}
                \item\label{properties-of-quotient-sets-descending-functions-to-quotient-sets-1-a}There exists a map
                    \[
                        \widebar{f}%
                        \colon%
                        X/\unsim_{R}%
                        \to%
                        Y%
                    \]%
                    making the diagram
                    \[
                        \begin{tikzcd}[row sep={5.0*\the\DL,between origins}, column sep={5.0*\the\DL,between origins}, background color=backgroundColor, ampersand replacement=\&]
                            X
                            \arrow[r,"f"]
                            \arrow[d,two heads,"q"']
                            \&
                            Y
                            \\
                            X/\unsim_{R}
                            \arrow[ru,"\widebar{f}"',"\exists",dotted]
                            \&
                        \end{tikzcd}
                    \]%
                    commute.
                \item\label{properties-of-quotient-sets-descending-functions-to-quotient-sets-1-b}We have $R\subset\Ker(f)$.
                \item\label{properties-of-quotient-sets-descending-functions-to-quotient-sets-1-c}For each $x,y\in X$, if $x\sim_{R}y$, then $f(x)=f(y)$.
            \end{enumerate}
        \item\label{properties-of-quotient-sets-descending-functions-to-quotient-sets-2}\SloganFont{Descending Functions to Quotient Sets, \rmII. }Let $R$ be an equivalence relation on $X$. If the conditions of \cref{properties-of-quotient-sets-descending-functions-to-quotient-sets-1} hold, then $\widebar{f}$ is the \emph{unique} map making the diagram
            \[
                \begin{tikzcd}[row sep={5.0*\the\DL,between origins}, column sep={5.0*\the\DL,between origins}, background color=backgroundColor, ampersand replacement=\&]
                    X
                    \arrow[r,"f"]
                    \arrow[d,two heads,"q"']
                    \&
                    Y
                    \\
                    X/\unsim_{R}
                    \arrow[ru,"\widebar{f}"',"\exists!",dashed]
                    \&
                \end{tikzcd}
            \]%
            commute.
        \item\label{properties-of-quotient-sets-descending-functions-to-quotient-sets-3}\SloganFont{Descending Functions to Quotient Sets, \rmIII. }Let $R$ be an equivalence relation on $X$. We have a bijection
            \[
                \Hom_{\Sets}(X/\unsim_{R},Y)%
                \cong%
                \Hom^{R}_{\Sets}(X,Y),%
            \]%
            natural in $X,Y\in\Obj(\Sets)$, given by the assignment $f\mapsto\widebar{f}$ of \cref{properties-of-quotient-sets-descending-functions-to-quotient-sets-1,properties-of-quotient-sets-descending-functions-to-quotient-sets-2}, where $\Hom^{R}_{\Sets}(X,Y)$ is the set defined by
            \[
                \Hom^{R}_{\Sets}(X,Y)%
                \defeq%
                \{%
                    f\in\Hom_{\Sets}(X,Y)%
                    \ \middle|\ %
                    \begin{aligned}
                        &\text{for each $x,y\in X$,}\\
                        &\text{if $x\sim_{R}y$, then}\\
                        &\text{$f(x)=f(y)$}%
                    \end{aligned}
                \}.%
            \]%
        \item\label{properties-of-quotient-sets-descending-functions-to-quotient-sets-4}\SloganFont{Descending Functions to Quotient Sets, \rmIV. }Let $R$ be an equivalence relation on $X$. If the conditions of \cref{properties-of-quotient-sets-descending-functions-to-quotient-sets-1} hold, then the following conditions are equivalent:
            \begin{enumerate}
                \item\label{properties-of-quotient-sets-descending-functions-to-quotient-sets-4-a}The map $\widebar{f}$ is an injection.
                \item\label{properties-of-quotient-sets-descending-functions-to-quotient-sets-4-b}We have $R=\Ker(f)$.
                \item\label{properties-of-quotient-sets-descending-functions-to-quotient-sets-4-c}For each $x,y\in X$, we have $x\sim_{R}y$ \textiff $f(x)=f(y)$.
            \end{enumerate}
        \item\label{properties-of-quotient-sets-descending-functions-to-quotient-sets-5}\SloganFont{Descending Functions to Quotient Sets, \rmV. }Let $R$ be an equivalence relation on $X$. If the conditions of \cref{properties-of-quotient-sets-descending-functions-to-quotient-sets-1} hold, then the following conditions are equivalent:
            \begin{enumerate}
                \item\label{properties-of-quotient-sets-descending-functions-to-quotient-sets-5-a}The map $f\colon X\to Y$ is surjective.
                \item\label{properties-of-quotient-sets-descending-functions-to-quotient-sets-5-b}The map $\widebar{f}\colon X/\unsim_{R}\to Y$ is surjective.
            \end{enumerate}
        \item\label{properties-of-quotient-sets-descending-functions-to-quotient-sets-6}\SloganFont{Descending Functions to Quotient Sets, \rmVI. }Let $R$ be a relation on $X$ and let $\unsim^{\rmeq}_{R}$ be the equivalence relation associated to $R$. The following conditions are equivalent:
            \begin{enumerate}
                \item\label{properties-of-quotient-sets-descending-functions-to-quotient-sets-6-a}The map $f$ satisfies the equivalent conditions of \cref{properties-of-quotient-sets-descending-functions-to-quotient-sets-1}:
                    \begin{itemize}
                        \item There exists a map
                            \[
                                \widebar{f}%
                                \colon%
                                X/\unsim^{\rmeq}_{R}%
                                \to%
                                Y%
                            \]%
                            making the diagram
                            \[
                                \begin{tikzcd}[row sep={5.0*\the\DL,between origins}, column sep={5.0*\the\DL,between origins}, background color=backgroundColor, ampersand replacement=\&]
                                    X
                                    \arrow[r,"f"]
                                    \arrow[d,two heads,"q"']
                                    \&
                                    Y
                                    \\
                                    X/\unsim^{\rmeq}_{R}
                                    \arrow[ru,"\widebar{f}"',"\exists",dotted]
                                    \&
                                \end{tikzcd}
                            \]%
                            commute.
                        \item For each $x,y\in X$, if $x\sim^{\rmeq}_{R}y$, then $f(x)=f(y)$.
                    \end{itemize}
                \item\label{properties-of-quotient-sets-descending-functions-to-quotient-sets-6-b}For each $x,y\in X$, if $x\sim_{R}y$, then $f(x)=f(y)$.
            \end{enumerate}
    \end{enumerate}
\end{proposition}
\begin{Proof}{Proof of \cref{properties-of-quotient-sets}}%
    \FirstProofBox{\cref{properties-of-quotient-sets-as-a-coequaliser}: As a Coequaliser}%
    Omitted.

    \ProofBox{\cref{properties-of-quotient-sets-as-a-pushout}: As a Pushout}%
    Omitted.

    \ProofBox{\cref{properties-of-quotient-sets-the-first-isomorphism-theorem-for-sets}: The First Isomorphism Theorem for Sets}%
    Clear.

    \ProofBox{\cref{properties-of-quotient-sets-descending-functions-to-quotient-sets-1}: Descending Functions to Quotient Sets, \rmI}%
    See \cite{proof-wiki:condition-for-mapping-from-quotient-set-to-be-well-defined}.

    \ProofBox{\cref{properties-of-quotient-sets-descending-functions-to-quotient-sets-2}: Descending Functions to Quotient Sets, \rmII}%
    See \cite{proof-wiki:mapping-from-quotient-set-when-defined-is-unique}.

    \ProofBox{\cref{properties-of-quotient-sets-descending-functions-to-quotient-sets-3}: Descending Functions to Quotient Sets, \rmIII}%
    This follows from \cref{properties-of-quotient-sets-descending-functions-to-quotient-sets-2,properties-of-quotient-sets-descending-functions-to-quotient-sets-3}.

    \ProofBox{\cref{properties-of-quotient-sets-descending-functions-to-quotient-sets-4}: Descending Functions to Quotient Sets, \rmIV}%
    See \cite{proof-wiki:condition-for-mapping-from-quotient-set-to-be-an-injection}.

    \ProofBox{\cref{properties-of-quotient-sets-descending-functions-to-quotient-sets-5}: Descending Functions to Quotient Sets, \rmV}%
    See \cite{proof-wiki:condition-for-mapping-from-quotient-set-to-be-a-surjection}.

    \ProofBox{\cref{properties-of-quotient-sets-descending-functions-to-quotient-sets-6}: Descending Functions to Quotient Sets, \rmVI}%
    The implication \cref{properties-of-quotient-sets-descending-functions-to-quotient-sets-5-a}$\implies$\cref{properties-of-quotient-sets-descending-functions-to-quotient-sets-5-b} is clear.

    Conversely, suppose that, for each $x,y\in X$, if $x\sim_{R}y$, then $f(x)=f(y)$. Spelling out the definition of the equivalence closure of $R$, we see that the condition $x\sim^{\rmeq}_{R}y$ unwinds to the following:
    \begin{itemize}
        \itemstar There exist $(x_{1},\ldots,x_{n})\in R^{\times n}$ satisfying at least one of the following conditions:%
            \begin{itemize}%
                \item The following conditions are satisfied:
                    \begin{itemize}
                        \item We have $x\sim_{R}x_{1}$       or $x_{1}\sim_{R}x$;
                        \item We have $x_{i}\sim_{R}x_{i+1}$ or $x_{i+1}\sim_{R}x_{i}$ for each $1\leq i\leq n-1$;
                        \item We have $y\sim_{R}x_{n}$       or $x_{n}\sim_{R}y$;
                    \end{itemize}
                \item We have $x=y$.%
            \end{itemize}%
    \end{itemize}
    Now, if $x=y$, then $f(x)=f(y)$ trivially; otherwise, we have
    \begin{align*}
        f(x)       &= f(x_{1}),\\
        f(x_{1})   &= f(x_{2}),\\
                   &\vdots     \\
        f(x_{n-1}) &= f(x_{n}),\\
        f(x_{n})   &= f(y),
    \end{align*}
    and $f(x)=f(y)$, as we wanted to show.
\end{Proof}
\begin{appendices}
\begin{multicols}{2}[\section{Other Chapters}]
\noindent
\textbf{Preliminaries}
\begin{enumerate}
\item \hyperref[introduction:section-phantom]{Introduction}
\end{enumerate}
\textbf{Sets}
\begin{enumerate}
\setcounter{enumi}{2}
\item \hyperref[sets:section-phantom]{Sets}
\item \hyperref[constructions-with-sets:section-phantom]{Constructions With Sets}
\item \hyperref[monoidal-structures-on-the-category-of-sets:section-phantom]{Monoidal Structures on the Category of Sets}
\item \hyperref[pointed-sets:section-phantom]{Pointed Sets}
\item \hyperref[tensor-products-of-pointed-sets:section-phantom]{Tensor Products of Pointed Sets}
\end{enumerate}
\textbf{Relations}
\begin{enumerate}
\setcounter{enumi}{6}
\item \hyperref[relations:section-phantom]{Relations}
\item \hyperref[constructions-with-relations:section-phantom]{Constructions With Relations}
\item \hyperref[conditions-on-relations:section-phantom]{Conditions on Relations}
\end{enumerate}
\textbf{Category Theory}
\begin{enumerate}
\setcounter{enumi}{9}
\item \hyperref[categories:section-phantom]{Categories}
\end{enumerate}
\textbf{Monoidal Categories}
\begin{enumerate}
\setcounter{enumi}{10}
\item \hyperref[constructions-with-monoidal-categories:section-phantom]{Constructions With Monoidal Categories}
\end{enumerate}
\textbf{Bicategories}
\begin{enumerate}
\setcounter{enumi}{11}
\item \hyperref[types-of-morphisms-in-bicategories:section-phantom]{Types of Morphisms in Bicategories}
\end{enumerate}
\textbf{Extra Part}
\begin{enumerate}
\setcounter{enumi}{12}
\item \hyperref[notes:section-phantom]{Notes}
\end{enumerate}
\end{multicols}

\end{appendices}
\end{document}
