\input{preamble}

% OK, start here.
%
\usepackage{fontspec}
\let\hyperwhite\relax
\let\hyperred\relax
\newcommand{\hyperwhite}{\hypersetup{citecolor=white,filecolor=white,linkcolor=white,urlcolor=white}}
\newcommand{\hyperred}{%
\hypersetup{%
    citecolor=TitlingRed,%
    filecolor=TitlingRed,%
    linkcolor=TitlingRed,%
     urlcolor=TitlingRed%
}}
\let\ChapterRef\relax
\newcommand{\ChapterRef}[2]{#1}
\setcounter{tocdepth}{2}
%▓▓▓▓▓▓▓▓▓▓▓▓▓▓▓▓▓▓▓▓▓▓▓▓▓▓▓▓▓▓▓▓▓
%▓▓ ╔╦╗╦╔╦╗╦  ╔═╗  ╔═╗╔═╗╔╗╔╔╦╗ ▓▓
%▓▓  ║ ║ ║ ║  ║╣   ╠╣ ║ ║║║║ ║  ▓▓
%▓▓  ╩ ╩ ╩ ╩═╝╚═╝  ╚  ╚═╝╝╚╝ ╩  ▓▓
%▓▓▓▓▓▓▓▓▓▓▓▓▓▓▓▓▓▓▓▓▓▓▓▓▓▓▓▓▓▓▓▓▓
%\usepackage{titlesec}
%▓▓▓▓▓▓▓▓▓▓▓▓▓▓▓▓▓▓▓▓▓▓▓▓▓▓▓▓▓▓▓▓▓▓▓▓▓▓▓▓▓▓▓▓▓▓▓▓▓▓▓▓▓▓▓
%▓▓ ╔╦╗╔═╗╔╗ ╦  ╔═╗  ╔═╗╔═╗  ╔═╗╔═╗╔╗╔╔╦╗╔═╗╔╗╔╔╦╗╔═╗ ▓▓
%▓▓  ║ ╠═╣╠╩╗║  ║╣   ║ ║╠╣   ║  ║ ║║║║ ║ ║╣ ║║║ ║ ╚═╗ ▓▓
%▓▓  ╩ ╩ ╩╚═╝╩═╝╚═╝  ╚═╝╚    ╚═╝╚═╝╝╚╝ ╩ ╚═╝╝╚╝ ╩ ╚═╝ ▓▓
%▓▓▓▓▓▓▓▓▓▓▓▓▓▓▓▓▓▓▓▓▓▓▓▓▓▓▓▓▓▓▓▓▓▓▓▓▓▓▓▓▓▓▓▓▓▓▓▓▓▓▓▓▓▓▓
\newcommand{\ChapterTableOfContents}{%
    \begingroup
    \addfontfeature{Numbers={Lining,Monospaced}}
    \hypersetup{hidelinks}\tableofcontents%
    \endgroup
}%

\makeatletter
\newcommand \DotFill {\leavevmode \cleaders \hb@xt@ .33em{\hss .\hss }\hfill \kern \z@}
\makeatother

\definecolor{ToCGrey}{rgb}{0.4,0.4,0.4}
\definecolor{mainColor}{rgb}{0.82745098,0.18431373,0.18431373}
\usepackage{titletoc}
\titlecontents{part}
[0.0em]
{\addvspace{1pc}\color{TitlingRed}\large\bfseries\text{Part }}
{\bfseries\textcolor{TitlingRed}{\contentslabel{0.0em}}\hspace*{1.35em}}
{}
{\textcolor{TitlingRed}{{\hfill\bfseries\contentspage\nobreak}}}
[]
\titlecontents{section}
[0.0em]
{\addvspace{1pc}}
{\color{black}\bfseries\textcolor{TitlingRed}{\contentslabel{0.0em}}\hspace*{1.35em}}
{}
{\textcolor{black}{\textbf{\DotFill}{\bfseries\contentspage\nobreak}}}
[]
\titlecontents{subsection}
[0.0em]
{}
{\hspace*{1.35em}\color{ToCGrey}{\contentslabel{0.0em}}\hspace*{2.1em}}
{}
{{\textcolor{ToCGrey}\DotFill}\textcolor{ToCGrey}{\contentspage}\nobreak}
[]
\usepackage{marginnote}
\renewcommand*{\marginfont}{\normalfont}
\usepackage{inconsolata}
\setmonofont{inconsolata}%
\let\ChapterRef\relax
\newcommand{\ChapterRef}[2]{#1}
\AtBeginEnvironment{subappendices}{%%
    \section*{\huge Appendices}%
}%

\begin{document}

\title{Presheaves and the Yoneda Lemma}

\maketitle

\phantomsection
\label{section-phantom}

This chapter contains some material about presheaves and the Yoneda lemma.

TODO:
\begin{enumerate}
    \item Adjointness of tensor product of functors
    \item Limit of category of elements (instead of colimit)
    \item Category of elements where objects are natural transformations $\SheafFont{F}\Rightarrow h_{X}$ instead of the other way around. Is this related to Isbell duality?
    \item Motivate the proof of the Yoneda lemma as in Martin's comment here: \url{https://mathoverflow.net/questions/130883/are-there-proofs-that-you-feel-you-did-not-understand-for-a-long-time#comment360113_131050}
    \item Add discussion of universal properties
    \item Add $h_{g\circ f}=h_{g}\circ h_{f}$ to properties of representable natural transformations
\end{enumerate}

\ChapterTableOfContents

\section{Presheaves}\label{section-presheaves}
\subsection{Foundations}\label{subsection-presheaves-foundations}
Let $\CatFont{C}$ be a category.
\begin{definition}{Presheaves on a Category}{presheaves-on-a-category}%
    A \index[categories]{presheaf}\textbf{presheaf on $\CatFont{C}$} is a functor $\SheafFont{F}\colon\smash{\CatFont{C}^{\op}}\to\Sets$.
\end{definition}
\begin{example}{Presheaves on One-Object Categories}{presheaves-on-one-object-categories}%
    Presheaves on the delooping $\B{A}$ of a monoid $A$ are precisely the left $A$-sets; see \ChapterRef{\ChapterMonoidActions, \cref{monoid-actions:left-a-sets-as-presheaves-on-ba}}{\cref{left-a-sets-as-presheaves-on-ba}}.
\end{example}
\begin{definition}{Morphisms of Presheaves}{morphisms-of-presheaves}%
    A \index[categories]{presheaf!morphism of}\textbf{morphism of presheaves} on $\CatFont{C}$ from $\SheafFont{F}$ to $\SheafFont{G}$ is a natural transformation $\alpha\colon\SheafFont{F}\Rightarrow\SheafFont{G}$.
\end{definition}
\begin{definition}{The Category of Presheaves on a Category}{the-category-of-presheaves-on-a-category}%
    The \index[set-theory]{presheaf!category of}\textbf{category of presheaves on $\CatFont{C}$} is the category \index[notation]{PShC@$\PSh(\CatFont{C})$}$\PSh(\CatFont{C})$%
    %--- Begin Footnote ---%
    \footnote{%
        \SloganFont{Further Notation: }Also written \index[notation]{Chat@$\widehat{\CatFont{C}}$}$\widehat{\CatFont{C}}$ in some parts of the literature.
        \par\vspace*{-1.75\baselineskip}
    } %
    %---  End Footnote  ---%
    defined by
    \[
        \PSh(\CatFont{C})
        \defeq
        \Fun\pig(\CatFont{C}^{\op},\Sets\pig).
    \]%
\end{definition}
\begin{remark}{Unwinding \cref{the-category-of-presheaves-on-a-category}}{unwinding-the-category-of-presheaves-on-a-category}%
    In detail, the \textbf{category of presheaves on $\CatFont{C}$} is the category $\PSh(\CatFont{C})$ where
    \begin{itemize}
        \item\SloganFont{Objects. }The objects of $\PSh(\CatFont{C})$ are presheaves on $\CatFont{C}$ as in \cref{presheaves-on-a-category}.
        \item\SloganFont{Morphisms. }The morphisms of $\PSh(\CatFont{C})$ are morphisms of presheaves as in \cref{morphisms-of-presheaves}, i.e.\ we have
            \[
                \Hom_{\PSh(\CatFont{C})}(\SheafFont{F},\SheafFont{G})%
                \defeq%
                \Nat(\SheafFont{F},\SheafFont{G})%
            \]%
            for each $\SheafFont{F},\SheafFont{G}\in\Obj(\PSh(\CatFont{C}))$.
        \item\SloganFont{Identities. }For each $\SheafFont{F}\in\Obj(\PSh(\CatFont{C}))$, the unit map
            \[
                \Unit^{\PSh(\CatFont{C})}_{\SheafFont{F}}%
                \colon%
                \pt%
                \to%
                \Nat(\SheafFont{F},\SheafFont{F})%
            \]%
            of $\PSh(\CatFont{C})$ at $\SheafFont{F}$ is defined by
            \[
                \id^{\PSh(\CatFont{C})}_{\SheafFont{F}}
                \defeq
                \id_{\SheafFont{F}},%
            \]%
            where $\id_{\SheafFont{F}}\colon\SheafFont{F}\Rightarrow\SheafFont{F}$ is the identity natural transformation of \ChapterRef{\ChapterCategories, \cref{categories:identity-natural-transformations}}{\cref{identity-natural-transformations}}.
        \item\SloganFont{Composition. }For each $\SheafFont{F},\SheafFont{G},\SheafFont{H}\in\Obj(\PSh(\CatFont{C}))$, the composition map
            \[
                \circ^{\PSh(\CatFont{C})}_{\SheafFont{F},\SheafFont{G},\SheafFont{H}}
                \colon
                \Nat(\SheafFont{G},\SheafFont{H})
                \times
                \Nat(\SheafFont{F},\SheafFont{G})
                \to
                \Nat(\SheafFont{F},\SheafFont{H})
            \]%
            of $\PSh(\CatFont{C})$ at $(\SheafFont{F},\SheafFont{G},\SheafFont{H})$ is defined by
            \[
                \beta\circ^{\PSh(\CatFont{C})}_{\SheafFont{F},\SheafFont{G},\SheafFont{H}}\alpha%
                \defeq
                \beta\circ\alpha,%
            \]%
            where $\beta\circ\alpha\colon\SheafFont{F}\Rightarrow\SheafFont{H}$ is the vertical composition of $\alpha$ and $\beta$ of \ChapterRef{\ChapterCategories, \cref{categories:vertical-composition-of-natural-transformations}}{\cref{vertical-composition-of-natural-transformations}}.
    \end{itemize}
\end{remark}
\subsection{Representable Presheaves}\label{subsection-representable-presheaves}
Let $\CatFont{C}$ be a category.
\begin{definition}{Representable Presheaves}{representable-presheaves}%
    Let $A\in\Obj(\CatFont{C})$.
    \begin{enumerate}
        \item\label{representable-presheaves-representable-presheaves-associated-to-an-object}The \index[categories]{presheaf!representable associated to an object}\textbf{representable presheaf associated to $A$} is the presheaf \index[notation]{hA@$h_{A}$}
            \[
                h_{A}%
                \colon%
                \CatFont{C}^{\op}%
                \to%
                \Sets%
            \]%
            where
            \begin{itemize}
                \item\SloganFont{Action on Objects. }For each $X\in\Obj(\CatFont{C})$, we have
                    \[
                        h_{A}(X)
                        \defeq
                        \Hom_{\CatFont{C}}(X,A).
                    \]%
                \item\SloganFont{Action on Morphisms. }For each $X,Y\in\Obj(\CatFont{C})$, the action on morphisms
                    \[
                        h_{A|X,Y}%
                        \colon%
                        \Hom_{\CatFont{C}}(X,Y)%
                        \to%
                        \Hom_{\Sets}(h_{A}(Y),h_{A}(X))%
                    \]%
                    of $h_{A}$ at $(X,Y)$ is given by sending a morphism
                    \[
                        f%
                        \colon%
                        X%
                        \to%
                        Y%
                    \]%
                    of $\CatFont{C}$ to the map of sets
                    \[
                        h_{A}(f)
                        \colon
                        \underbrace{h_{A}(Y)}_{\defeq\Hom_{\CatFont{C}}(Y,A)}
                        \to
                        \underbrace{h_{A}(X)}_{\defeq\Hom_{\CatFont{C}}(X,A)}
                    \]%
                    defined by
                    \[
                        h_{A}(f)
                        \defeq
                        f^{*},
                    \]%
                    where $f^{*}$ is the precomposition by $f$ morphism of \ChapterRef{\ChapterCategories, \cref{categories:precomposition-and-postcomposition-functions-precomposition} of \cref{categories:precomposition-and-postcomposition-functions}}{\cref{precomposition-and-postcomposition-functions-precomposition} of \cref{precomposition-and-postcomposition-functions}}.
            \end{itemize}
        \item\label{representable-presheaves-representing-objects-for-presheaves}A \index[categories]{presheaf!representing object of}\textbf{representing object} for a presheaf $\SheafFont{F}\colon\CatFont{C}^{\op}\to\Sets$ on $\CatFont{C}$ is an object $A$ of $\CatFont{C}$ such that we have $\SheafFont{F}\cong h_{A}$.
        \item\label{representable-presheaves-representable-presheaves}A presheaf $\SheafFont{F}\colon\CatFont{C}^{\op}\to\Sets$ on $\CatFont{C}$ is \index[categories]{presheaf!representable}\textbf{representable} if $\SheafFont{F}$ admits a representing object.
    \end{enumerate}
\end{definition}
\begin{example}{Representable Presheaves on One-Object Categories}{representable-presheaves-on-one-object-categories}%
    The representable presheaf on the delooping $\B{A}$ of a monoid $A$ associated to the unique object $\bullet$ of $\B{A}$ is the left regular representation of $A$ of \ChapterRef{\ChapterMonoidActions, \cref{monoid-actions:the-left-regular-representation-of-a-monoid}}{\cref{the-left-regular-representation-of-a-monoid}}.
\end{example}
\begin{proposition}{Uniqueness of Representing Objects Up to Isomorphism}{uniqueness-of-representing-objects-up-to-isomorphism}%
    Let $\SheafFont{F}\colon\CatFont{C}^{\op}\to\Sets$ be a presheaf. If there exist $A,B\in\Obj(\CatFont{C})$ such that we have natural isomorphisms
    \begin{align*}
        h_{A} &\cong\SheafFont{F},\\
        h_{B} &\cong\SheafFont{F},
    \end{align*}
    then $A\cong B$.
\end{proposition}
\begin{Proof}{Proof of \cref{uniqueness-of-representing-objects-up-to-isomorphism}}%
    By composing the isomorphisms $h_{A}\cong\SheafFont{F}\cong h_{B}$, we get a natural isomorphism $h_{A}\cong h_{B}$. By \cref{properties-of-the-yoneda-embedding-preservation-and-reflection-of-isomorphisms} of \cref{properties-of-the-yoneda-embedding}, we have $A\cong B$.
\end{Proof}
\subsection{Representable Natural Transformations}\label{subsection-representable-natural-transformations}
Let $\CatFont{C}$ be a category, let $A,B\in\Obj(\CatFont{C})$, and let $f\colon A\to B$ be a morphism of $\CatFont{C}$.
\begin{definition}{Representable Natural Transformations}{representable-natural-transformations}%
    The \index[categories]{representable natural transformation}\textbf{representable natural transformation associated to $f$} is the natural transformation
    \[
        h_{f}%
        \colon%
        h_{A}%
        \Rightarrow%
        h_{B}%
    \]%
    consisting of the collection
    \[
        \{
            h_{f|X}
            \colon
            \underbrace{h_{A}(X)}_{\defeq\Hom_{\CatFont{C}}(X,A)}
            \to
            \underbrace{h_{B}(X)}_{\defeq\Hom_{\CatFont{C}}(X,B)}
        \}_{X\in\Obj(\CatFont{C})}
    \]%
    with
    \[
        h_{f|X}
        \defeq
        f_{*},
    \]%
    where $f_{*}$ is the postcomposition by $f$ morphism of \ChapterRef{\ChapterCategories, \cref{categories:precomposition-and-postcomposition-functions-postcomposition} of \cref{categories:precomposition-and-postcomposition-functions}}{\cref{precomposition-and-postcomposition-functions-postcomposition} of \cref{precomposition-and-postcomposition-functions}}.
\end{definition}
\subsection{The Yoneda Embedding}\label{subsection-the-yoneda-embedding}
\begin{definition}{The Yoneda Embedding}{the-yoneda-embedding}%
    The \index[categories]{Yoneda embedding}\textbf{Yoneda embedding of $\CatFont{C}$}%
    %--- Begin Footnote ---%
    \footnote{%
        \SloganFont{Further Terminology: }Also called the \textbf{covariant Yoneda embedding} to distinguish it from the contravariant Yoneda embedding of \cref{the-contravariant-yoneda-lemma}.
    } %
    %---  End Footnote  ---%
    is the functor\index[notation]{yo@$\yo$}%
    %--- Begin Footnote ---%
    \footnote{%
        \SloganFont{Further Notation: }Also written $h_{(-)}$, or simply $\yo$.
        \par\vspace*{-1.75\baselineskip}
    } %
    %---  End Footnote  ---%
    \[
        \yo_{\CatFont{C}}%
        \colon%
        \CatFont{C}%
        \longhookrightarrow%
        \PSh(\CatFont{C})%
    \]%
    where
    \begin{itemize}
        \item\SloganFont{Action on Objects. }For each $A\in\Obj(\CatFont{C})$, we have
            \[
                \yo_{\CatFont{C}}(A)
                \defeq
                h_{A}.
            \]%
        \item\SloganFont{Action on Morphisms. }For each $A,B\in\Obj(\CatFont{C})$, the action on morphisms
            \[
                \yo_{\CatFont{C}|A,B}%
                \colon%
                \Hom_{\CatFont{C}}(A,B)%
                \to%
                \Nat(h_{A},h_{B})%
            \]%
            of $\yo_{\CatFont{C}}$ at $(A,B)$ is given by
            \[
                \yo_{\CatFont{C}|A,B}(f)%
                \defeq%
                h_{f}%
            \]%
            for each $f\in\Hom_{\CatFont{C}}(A,B)$, where $h_{f}$ is the representable natural transformation associated to $f$ of \cref{representable-natural-transformations}.
    \end{itemize}
\end{definition}
\begin{remark}{On the Usage of $\yo$ to Denote the Yoneda Embedding}{on-the-usage-of-yo-to-denote-the-yoneda-embedding}%
    The notation $\yo$ for the Yoneda embedding was first introduced in \cite{oplax-natural-transformations-twisted-quantum-field-theories-and-even-higher-morita-categories}. The symbol $\yo$ is the \href{https://en.wikipedia.org/wiki/Yo_(kana%29}{hiragana for \emph{yo}}, and comes from \say{Yoneda} in Nobuo Yoneda (\Japanese{米田信夫}).

    It is pronounced \emph{yo} but without letting the \say{o} in \emph{yo} sound like an o-u diphthong:
    \begin{itemize}
            % BEGIN RAW HTML %
            <li>Audio: <audio id="audioPlayerYo" src="/static/sounds/yo.mp3" type="audio/mp3"></audio><span class="dark-svg"><button id="playButtonYo"></button></span>
            % BEGIN LATEX HTML %
        \item Audio: see \url{https://topological-modular-forms.github.io/the-clowder-project/static/sounds/yo.mp3}
            % END RAW HTML %
        \item IPA transcription: [\IPA{jo̞}].
    \end{itemize}
\end{remark}
\begin{proposition}{Properties of the Yoneda Embedding}{properties-of-the-yoneda-embedding}%
    Let $\CatFont{C}$ be a category.
    \begin{enumerate}
        \item\label{properties-of-the-yoneda-embedding-fully-faithfulness}\SloganFont{Fully Faithfulness. }The Yoneda embedding
            \[
                \yo_{\CatFont{C}}%
                \colon%
                \CatFont{C}%
                \hookrightarrow%
                \PSh(\CatFont{C})%
            \]%
            is fully faithful.
        \item\label{properties-of-the-yoneda-embedding-preservation-and-reflection-of-isomorphisms}\SloganFont{Preservation and Reflection of Isomorphisms. }The Yoneda embedding
            \[
                \yo_{\CatFont{C}}%
                \colon%
                \CatFont{C}%
                \hookrightarrow%
                \PSh(\CatFont{C})%
            \]%
            preserves and reflects isomorphisms, i.e.\ given $A,B\in\Obj(\CatFont{C})$, the following conditions are equivalent:
            \begin{enumerate}
                \item We have $A\cong B$.
                \item We have $h_{A}\cong h_{B}$.
            \end{enumerate}
        \item\label{properties-of-the-yoneda-embedding-density}\SloganFont{Density. }The Yoneda embedding
            \[
                \yo_{\CatFont{C}}%
                \colon%
                \CatFont{C}%
                \hookrightarrow%
                \PSh(\CatFont{C})%
            \]%
            is dense.
        \item\label{properties-of-the-yoneda-embedding-interaction-with-density-comonads}\SloganFont{Interaction With Density Comonads. }We have
            \begin{webcompile}
                \Ran_{\yo}(\yo)%
                \cong%
                \id_{\PSh(\CatFont{C})},%
                \qquad
                \begin{tikzcd}[row sep={5.0*\the\DL,between origins}, column sep={5.0*\the\DL,between origins}, background color=backgroundColor, ampersand replacement=\&]
                    \&
                    {\PSh(\CatFont{C})}
                    \arrow[d, "\Lan_{\yo}(\yo)",dashed]
                    \\
                    \CatFont{C}
                    \arrow[ru, "\yo_{\CatFont{C}}"]
                    \arrow[r, "\yo_{\CatFont{C}}"'{name=F}]
                    \&
                    {\PSh(\CatFont{C})}\mrp{.}
                    % 2-Arrows
                    \arrow[from=F,to=1-2,Rightarrow,shorten=0.75em,pos=0.4,start anchor={[xshift=-0.1*\the\DL]},end anchor={[xshift=0.45*\the\DL]}]
                \end{tikzcd}
            \end{webcompile}
        \item\label{properties-of-the-yoneda-embedding-interaction-with-codensity-monads}\SloganFont{Interaction With Codensity Monads. }We have
            \[
                \Ran_{\yo}(\yo)%
                \cong%
                \IsbellSpec\circ\IsbellO,%
            \]%
            where $\IsbellSpec$ and $\IsbellO$ are the functors of \cref{the-isbell-duality-adjunction}.
        %\item\label{properties-of-the-yoneda-embedding-}\SloganFont{. }
    \end{enumerate}
\end{proposition}
\begin{Proof}{Proof of \cref{properties-of-the-yoneda-embedding}}%
    \FirstProofBox{\cref{properties-of-the-yoneda-embedding-fully-faithfulness}: Fully Faithfulness}%
    Let $A,B\in\Obj(\CatFont{C})$. Applying the Yoneda lemma (\cref{the-yoneda-lemma}) to the functor $h_{B}$ (i.e.\ in the case $\SheafFont{F}=h_{B}$), we have
    \[%
        \Hom_{\CatFont{C}}(A,B)%
        \cong%
        \Nat(h_{A},h_{B}),%
    \]%
    and the natural isomorphism
    \[
        \xi_{A,B}%
        \colon%
        \SheafFont{h}_{B}(A)%
        \Rightarrow%
        \Nat(h_{A},h_{B})%
    \]%
    witnessing this bijection is given by
    \begin{align*}
        \xi_{A,B}(g)_{X} &\defeq h^{X}_{g}\\
                         &\defeq g_{*}
    \end{align*}
    for each $X\in\Obj(\CatFont{C})$ and each $g\in h^{X}_{B}$, i.e.\ we have $\xi_{A,B}=\yo_{\CatFont{C}|A,B}$. Thus $\yo_{\CatFont{C}}$ is fully faithful.

    \ProofBox{\cref{properties-of-the-yoneda-embedding-preservation-and-reflection-of-isomorphisms}: Preservation and Reflection of Isomorphisms}%
    This follows from \ChapterRef{\ChapterCategories, \cref{categories:elementary-properties-of-functors-preservation-of-isomorphisms} of \cref{categories:elementary-properties-of-functors} and \cref{categories:properties-of-fully-faithful-functors-conservativity} of \cref{categories:properties-of-fully-faithful-functors}}{\cref{elementary-properties-of-functors-preservation-of-isomorphisms} of \cref{elementary-properties-of-functors} and \cref{properties-of-fully-faithful-functors-conservativity} of \cref{properties-of-fully-faithful-functors}}.

    \ProofBox{\cref{properties-of-the-yoneda-embedding-density}: Density}%
    Omitted.

    \ProofBox{\cref{properties-of-the-yoneda-embedding-interaction-with-density-comonads}: Interaction With Density Comonads}%
    Omitted.

    \ProofBox{\cref{properties-of-the-yoneda-embedding-interaction-with-codensity-monads}: Interaction With Codensity Monads}%
    Omitted.
\end{Proof}
\subsection{The Yoneda Lemma}\label{subsection-the-yoneda-lemma}
Let $\SheafFont{F}\colon\smash{\CatFont{C}^{\op}}\to\Sets$ be a presheaf on $\CatFont{C}$.
\begin{theorem}{The Yoneda Lemma}{the-yoneda-lemma}%
    \index[categories]{Yoneda lemma}We have a bijection
    \[
        \Nat(h_{A},\SheafFont{F})
        \cong
        \SheafFont{F}(A),
    \]%
    natural in $A\in\Obj(\CatFont{C})$, determining a natural isomorphism of functors
    \[
        \Nat(h_{(-)},\SheafFont{F})
        \cong
        \SheafFont{F}.
    \]%
\end{theorem}
\begin{Proof}{Proof of \cref{the-yoneda-lemma}}%
    \FirstProofBox{The Transformation $\ev\colon\Nat(h_{(-)},\SheafFont{F})\Rightarrow\SheafFont{F}$}%
    Let
    \[
        \ev%
        \colon%
        \Nat(h_{(-)},\SheafFont{F})%
        \Rightarrow%
        \SheafFont{F}
    \]%
    be the transformation consisting of the collection
    \[
        \{%
            \ev_{A}%
            \colon%
            \Nat(h_{A},\SheafFont{F})%
            \to%
            \SheafFont{F}(A)%
        \}_{A\in\Obj(\CatFont{C})}%
    \]%
    with
    \[
        \ev_{A}(\alpha)%
        =%
        \alpha_{A}(\id_{A})%
    \]%
    for each $\alpha\in\Nat(h_{A},\SheafFont{F})$, where $\alpha_{A}$ is the component
    \[
        \alpha_{A}%
        \colon%
        \Hom_{\CatFont{C}}(A,A)%
        \to%
        \SheafFont{F}(A)%
    \]%
    of $\alpha$ at $A$.

    \ProofBox{The Transformation $\xi\colon\SheafFont{F}\Rightarrow\Nat(h_{(-)},\SheafFont{F})$}%
    Let
    \[
        \xi%
        \colon%
        \SheafFont{F}%
        \Rightarrow%
        \Nat(h_{(-)},\SheafFont{F})%
    \]%
    be the transformation consisting of the collection
    \[
        \{%
            \xi_{A}%
            \colon%
            \SheafFont{F}(A)%
            \to%
            \Nat(h_{A},\SheafFont{F})%
        \}_{A\in\Obj(\CatFont{C})},%
    \]
    where $\xi_{A}$ is the map sending an element $\phi\in\SheafFont{F}(A)$ to the transformation
    \[%
        \xi_{A}(\phi)%
        \colon%
        h_{A}%
        \Rightarrow%
        \SheafFont{F}%
    \]
    (which we will show is natural in a bit) consisting of the collection
    \[%
        \{
            \xi_{A}(\phi)_{X}%
            \colon%
            h_{A}(X)%
            \to%
            \SheafFont{F}(X)%
        \}_{X\in\Obj(\CatFont{C})},%
    \]%
    with
    \[
        \xi_{A}(\phi)_{X}(f)%
        \defeq%
        [\SheafFont{F}(f)](\phi)%
    \]%
    for each $f\in h_{A}(X)$, where
    \[
        \SheafFont{F}(f)%
        \colon%
        \SheafFont{F}(A)%
        \to%
        \SheafFont{F}(X)%
    \]%
    is the image of $f$ by $\SheafFont{F}$.

    \ProofBox{Naturality of $\xi_{A}(\phi)\colon h_{A}\Rightarrow\SheafFont{F}$}%
    The transformation
    \[%
        \xi_{A}(\phi)%
        \colon%
        h_{A}%
        \Rightarrow%
        \SheafFont{F}%
    \]
    is indeed natural, as the diagram
    \[
        \begin{tikzcd}[row sep={5.0*\the\DL,between origins}, column sep={6.0*\the\DL,between origins}, background color=backgroundColor, ampersand replacement=\&]
            h^{Y}_{A}
            \arrow[r,"f^{*}"]
            \arrow[d,"{\xi_{A}(\phi)_{Y}}"']
            \&
            h^{X}_{A}
            \arrow[d,"{\xi_{A}(\phi)_{X}}"]
            \\
            \SheafFont{F}(Y)
            \arrow[r,"\SheafFont{F}(f)"']
            \&
            \SheafFont{F}(X)
        \end{tikzcd}
    \]%
    commutes for each morphism $f\colon X\to Y$ of $\CatFont{C}$, acting on elements as
    \begin{webcompile}
        \begin{tikzcd}[row sep={5.0*\the\DL,between origins}, column sep={10.0*\the\DL,between origins}, background color=backgroundColor, ampersand replacement=\&]
            h
            \arrow[d,mapsto]
            \&
            \\
            [\SheafFont{F}(h)](\phi)
            \arrow[r,mapsto]
            \&
            [\SheafFont{F}(f)]([\SheafFont{F}(h)](\phi))
        \end{tikzcd}
        \quad
        \begin{tikzcd}[row sep={5.0*\the\DL,between origins}, column sep={5.0*\the\DL,between origins}, background color=backgroundColor, ampersand replacement=\&]
            h
            \arrow[r,mapsto]
            \&
            h\circ f
            \arrow[d,mapsto]
            \\
            \&
            [\SheafFont{F}(h\circ f)(\phi)]\mrp{,}
        \end{tikzcd}
    \end{webcompile}
    where we have
    \[
        [\SheafFont{F}(f)]([\SheafFont{F}(h)](\phi))%
        =%
        [\SheafFont{F}(h\circ f)(\phi)]%
    \]%
    by the functoriality of $\SheafFont{F}$.

    \ProofBox{Naturality of $\ev\colon\Nat(h_{(-)},\SheafFont{F})\Rightarrow\SheafFont{F}$}%
    Let $f\colon X\to Y$ be a morphism of $\CatFont{C}$. We claim the naturality diagram
    \[
        \begin{tikzcd}[row sep={5.0*\the\DL,between origins}, column sep={8.5*\the\DL,between origins}, background color=backgroundColor, ampersand replacement=\&]
            \Nat(h_{Y},\SheafFont{F})%
            \arrow[r,"{(h_{f})^{*}}"]
            \arrow[d,"\ev_{Y}"']
            \&
            \Nat(h_{X},\SheafFont{F})%
            \arrow[d,"\ev_{X}"]
            \\
            \SheafFont{F}(Y)
            \arrow[r,"{\SheafFont{F}(f)}"']
            \&
            \SheafFont{F}(X)
        \end{tikzcd}
    \]%
    for $\ev$ at $f$, acting on elements as
    \begin{webcompile}
        \begin{tikzcd}[row sep={5.0*\the\DL,between origins}, column sep={9.0*\the\DL,between origins}, background color=backgroundColor, ampersand replacement=\&]
            \alpha
            \arrow[d,mapsto]
            \&
            \\
            {\alpha_{Y}(\id_{Y})}
            \arrow[r,mapsto]
            \&
            {[\SheafFont{F}(f)](\alpha_{Y}(\id_{Y}))}
        \end{tikzcd}
        \qquad
        \begin{tikzcd}[row sep={5.0*\the\DL,between origins}, column sep={5.0*\the\DL,between origins}, background color=backgroundColor, ampersand replacement=\&]
            \alpha
            \arrow[r,mapsto]
            \&
            \alpha\circ h_{f}
            \arrow[d,mapsto]
            \\
            \&
            {[\alpha\circ h_{f}]_{X}(\id_{X})}\mrp{,}
        \end{tikzcd}
    \end{webcompile}
    commutes. Indeed:
    \begin{enumerate}
        \item We have
            \begin{align*}
                [\alpha\circ h_{f}]_{X}(\id_{X}) &\defeq [\alpha_{X}\circ h_{f|X}](\id_{X})\\
                                                 &\defeq [\alpha_{X}\circ f_{*}](\id_{X})\\
                                                 &\defeq \alpha_{X}(f_{*}(\id_{X}))\\
                                                 &\defeq \alpha_{X}(f).
            \end{align*}
        \item Applying the naturality diagram
            \[
                \begin{tikzcd}[row sep={5.0*\the\DL,between origins}, column sep={6.0*\the\DL,between origins}, background color=backgroundColor, ampersand replacement=\&]
                    h^{Y}_{Y}
                    \arrow[r,"{f^{*}}"]
                    \arrow[d,"\alpha_{Y}"']
                    \&
                    h^{X}_{Y}
                    \arrow[d,"\alpha_{X}"]
                    \\
                    \SheafFont{F}(Y)
                    \arrow[r,"{\SheafFont{F}(f)}"']
                    \&
                    \SheafFont{F}(X)
                \end{tikzcd}
            \]%
            of $\alpha\colon h_{Y}\Rightarrow\SheafFont{F}$ at $f\colon X\to Y$ to the element $\id_{Y}$ of $h^{Y}_{Y}$, we have
            \begin{webcompile}
                \begin{tikzcd}[row sep={5.0*\the\DL,between origins}, column sep={9.0*\the\DL,between origins}, background color=backgroundColor, ampersand replacement=\&]
                    \id_{Y}
                    \arrow[d,mapsto]
                    \&
                    \\
                    {\alpha_{Y}(\id_{Y})}
                    \arrow[r,mapsto]
                    \&
                    {[\SheafFont{F}(f)](\alpha_{Y}(\id_{Y}))}
                \end{tikzcd}
                \qquad
                \begin{tikzcd}[row sep={5.0*\the\DL,between origins}, column sep={5.0*\the\DL,between origins}, background color=backgroundColor, ampersand replacement=\&]
                    \id_{Y}
                    \arrow[r,mapsto]
                    \&
                    f
                    \arrow[d,mapsto]
                    \\
                    \&
                    {\alpha_{X}(f)}\mrp{,}
                \end{tikzcd}
            \end{webcompile}
            showing that we have
            \[
                [\SheafFont{F}(f)](\alpha_{Y}(\id_{Y}))
                =
                \alpha_{X}(f)\mrp{.}
            \]%
    \end{enumerate}
    Thus the naturality diagram for $\ev$ at $f$ commutes, and $\ev$ is natural.

    \ProofBox{Naturality of $\xi\colon\SheafFont{F}\Rightarrow\Nat(h_{(-)},\SheafFont{F})$}%
    Let $f\colon X\to Y$ be a morphism of $\CatFont{C}$. We claim the naturality diagram
    \[
        \begin{tikzcd}[row sep={5.0*\the\DL,between origins}, column sep={8.5*\the\DL,between origins}, background color=backgroundColor, ampersand replacement=\&]
            \SheafFont{F}(Y)
            \arrow[r,"{\SheafFont{F}(f)}"]
            \arrow[d,"\xi_{Y}"']
            \&
            \SheafFont{F}(X)
            \arrow[d,"\xi_{X}"]
            \\
            \Nat(h_{Y},\SheafFont{F})%
            \arrow[r,"{(h_{f})^{*}}"']
            \&
            \Nat(h_{X},\SheafFont{F})%
        \end{tikzcd}
    \]%
    for $\xi$ at $f$, acting on elements as
    \begin{webcompile}
        \begin{tikzcd}[row sep={5.0*\the\DL,between origins}, column sep={7.0*\the\DL,between origins}, background color=backgroundColor, ampersand replacement=\&]
            \phi
            \arrow[d,mapsto]
            \&
            \\
            {\xi_{Y}(\phi)}
            \arrow[r,mapsto]
            \&
            {\xi_{Y}(\phi)\circ h_{f}}
        \end{tikzcd}
        \qquad
        \begin{tikzcd}[row sep={5.0*\the\DL,between origins}, column sep={6.0*\the\DL,between origins}, background color=backgroundColor, ampersand replacement=\&]
            \phi
            \arrow[r,mapsto]
            \&
            [\SheafFont{F}(f)](\phi)
            \arrow[d,mapsto]
            \\
            \&
            {\xi_{X}([\SheafFont{F}(f)](\phi))}
        \end{tikzcd}
    \end{webcompile}
    commutes. Indeed, for each $X\in\Obj(\CatFont{C})$ and each $g\in h^{A}_{X}$, we have
    \begin{align*}
        [\xi_{Y}(\phi)\circ h_{f}]_{X}(g) &\defeq [\xi_{Y}(\phi)_{X}\circ h_{f|X}](g)\\
                                          &\defeq [\xi_{Y}(\phi)_{X}\circ f_{*}](g)\\
                                          &\defeq \xi_{Y}(\phi)_{X}(f_{*}(g))\\
                                          &\defeq \xi_{Y}(\phi)_{X}(f\circ g)\\
                                          &\defeq [\SheafFont{F}(f\circ g)](\phi)
    \end{align*}
    and
    \begin{align*}
        [\xi_{X}([\SheafFont{F}(f)](\phi))]_{X}(g) &\defeq \SheafFont{F}(g)([\SheafFont{F}(f)](\phi))\\
                                                   &=      [\SheafFont{F}(f\circ g)](\phi),
    \end{align*}
    where we have used the functoriality of $\SheafFont{F}$. Thus $\xi_{Y}(\phi)\circ h_{f}$ and $\xi_{X}([\SheafFont{F}(f)](\phi))$ are equal, and the naturality diagram for $\xi$ at $f$ above commutes, showing $\xi$ to be natural.

    \ProofBox{Invertibility \rmI: $\ev\circ\xi=\id_{\SheafFont{F}}$}%
    We claim that $\ev\circ\xi=\id_{\SheafFont{F}}$, i.e.\ that we have
    \[
        (\ev\circ\xi)_{A}%
        =%
        \id_{\SheafFont{F}(A)}%
    \]%
    for each $A\in\Obj(\CatFont{C})$. Indeed, we have
    \begin{align*}
        [\ev\circ\xi]_{A}(\phi) &\defeq [\ev_{A}\circ\xi_{A}](\phi)\\
                                &\defeq \ev_{A}(\xi_{A}(\phi))\\
                                &\defeq \xi_{A}(\phi)_{A}(\id_{A})\\
                                &\defeq [\SheafFont{F}(\id_{A})](\phi)\\
                                &=      [\id_{\SheafFont{F}(A)}](\phi)
    \end{align*}
    for each $\phi\in\SheafFont{F}(A)$.

    \ProofBox{Invertibility \rmII: $\xi\circ\ev=\id_{\Nat(h_{(-)},\SheafFont{F})}$}%
    We claim that $\xi\circ\ev=\id_{\Nat(h_{(-)},\SheafFont{F})}$, i.e.\ that we have
    \[
        (\xi\circ\ev)_{A}%
        =%
        \id_{\Nat(h_{A},\SheafFont{F})}%
    \]%
    for each $A\in\Obj(\CatFont{C})$. Indeed:
    \begin{enumerate}
        \item We have
            \begin{align*}
                [\xi\circ\ev]_{A}(\alpha) &\defeq [\xi_{A}\circ\ev_{A}](\alpha)\\
                                          &\defeq \xi_{A}(\ev_{A}(\alpha))\\
                                          &\defeq \xi_{A}(\alpha_{A}(\id_{A}))\\
            \end{align*}
            for each $\alpha\in\Nat(h_{A},\SheafFont{F})$.
        \item For each $X\in\Obj(\CatFont{C})$, we have
            \[
                \xi_{A}(\alpha_{A}(\id_{A}))_{X}%
                =%
                \alpha_{X},%
            \]%
            since we have
            \begin{align*}
                \xi_{A}(\alpha_{A}(\id_{A}))_{X}(f) &\defeq  [\SheafFont{F}(f)](\alpha_{A}(\id_{A}))\\%
                                                    &\eqstar \alpha_{X}(f)%
            \end{align*}
            for each $f\in h_{A}(X)$, where the equality marked with $(\dagger)$ follows from the commutativity of the naturality diagram
            \[
                \begin{tikzcd}[row sep={5.0*\the\DL,between origins}, column sep={6.0*\the\DL,between origins}, background color=backgroundColor, ampersand replacement=\&]
                    h^{A}_{A}
                    \arrow[r,"f_{*}"]%
                    \arrow[d,"\alpha_{A}"']%
                    \&
                    h^{A}_{X}
                    \arrow[d,"\alpha_{X}"]%
                    \\
                    \SheafFont{F}(A)
                    \arrow[r,"{\SheafFont{F}(f)}"']
                    \&
                    \SheafFont{F}(X)
                \end{tikzcd}
            \]%
            of $\alpha$ at $f\colon A\to X$, which acts on $\id_{A}$ as
            \[
                \begin{tikzcd}[row sep={5.0*\the\DL,between origins}, column sep={11.0*\the\DL,between origins}, background color=backgroundColor, ampersand replacement=\&]
                    \id_{A}
                    \arrow[r,mapsto]%
                    \arrow[d,mapsto]%
                    \&
                    f
                    \arrow[d,mapsto]%
                    \\
                    \alpha_{A}(\id_{A})
                    \arrow[r,mapsto]
                    \&
                    {[\SheafFont{F}(f)](\alpha_{A}(\id_{A}))=\alpha_{X}(f)}\mrp{.}
                \end{tikzcd}
            \]%
    \end{enumerate}
    This finishes the proof.
\end{Proof}%
\subsection{Properties of Categories of Presheaves}\label{subsection-properties-of-categories-of-presheaves}
\begin{proposition}{Properties of Categories of Presheaves}{properties-of-categories-of-presheaves}%
    Let $\CatFont{C}$ be a category.
    \begin{enumerate}
        \item\label{properties-of-categories-of-presheaves-functoriality}\SloganFont{Functoriality. }The assignment $\CatFont{C}\mapsto\PSh(\CatFont{C})$ defines a functor
            \[
                \PSh%
                \colon%
                \Cats%
                \to%
                \Cats%
            \]%
            up to some set-theoretic considerations.%
            %--- Begin Footnote ---%
            \footnote{%
                For instance:
                \begin{itemize}
                    \item The $\Cats$ in the source of $\PSh$ could be small categories, and then the $\Cats$ in the right would be locally small categories.
                    \item The $\Cats$ in the source of $\PSh$ could be locally small categories, and then the $\Cats$ on the right would be large categories.
                \end{itemize}
                In general, one can systematise and formalise this using Grothendieck universes.
                \par\vspace*{-1.75\baselineskip}
            }%
            %---  End Footnote  ---%
        \item\label{properties-of-categories-of-presheaves-interaction-with-slice-categories}\SloganFont{Interaction With Slice Categories. }Let $X\in\Obj(\CatFont{C})$. We have an equivalence of categories
            \[
                \PSh(\CatFont{C}_{/X})%
                \eqcong%
                \PSh(\CatFont{C})_{/h_{X}}.%
            \]%
        \item\label{properties-of-categories-of-presheaves-interaction-with-categories-of-elements}\SloganFont{Interaction With Categories of Elements. }Let $\SheafFont{F}\in\Obj(\PSh(\CatFont{C}))$. We have an equivalence of categories
            \[
                \PSh(\catEl{\CatFont{C}}{\SheafFont{F}})%
                \eqcong%
                \PSh(\CatFont{C})_{/\SheafFont{F}}.%
            \]%
        %\item\label{properties-of-categories-of-presheaves-}\SloganFont{. }
    \end{enumerate}
\end{proposition}
\begin{Proof}{Proof of \cref{properties-of-categories-of-presheaves}}%
    \FirstProofBox{\cref{properties-of-categories-of-presheaves-functoriality}: Functoriality}%
    Omitted.

    \ProofBox{\cref{properties-of-categories-of-presheaves-interaction-with-slice-categories}: Interaction With Slice Categories}%
    Omitted.

    \ProofBox{\cref{properties-of-categories-of-presheaves-interaction-with-categories-of-elements}: Interaction With Categories of Elements}%
    Omitted.
\end{Proof}
\section{Copresheaves}\label{section-copresheaves}
\subsection{Foundations}\label{subsection-copresheaves-foundations}
Let $\CatFont{C}$ be a category.
\begin{definition}{Copresheaves on a Category}{copresheaves-on-a-category}%
    A \index[categories]{copresheaf}\textbf{copresheaf on $\CatFont{C}$} is a functor $F\colon\CatFont{C}\to\Sets$.
\end{definition}
\begin{example}{Copresheaves on One-Object Categories}{copresheaves-on-one-object-categories}%
    Copresheaves on the delooping $\B{A}$ of a monoid $A$ are precisely the right $A$-sets; see \ChapterRef{\ChapterMonoidActions, \cref{monoid-actions:right-a-sets-as-copresheaves-on-ba}}{\cref{right-a-sets-as-copresheaves-on-ba}}.
\end{example}
\begin{definition}{Morphisms of Copresheaves}{morphisms-of-copresheaves}%
    A \index[categories]{copresheaf!morphism of}\textbf{morphism of copresheaves} on $\CatFont{C}$ from $F$ to $G$ is a natural transformation $\alpha\colon F\Rightarrow G$.
\end{definition}
\begin{definition}{The Category of Copresheaves on a Category}{the-category-of-copresheaves-on-a-category}%
    The \index[set-theory]{copresheaf!category of}\textbf{category of copresheaves on $\CatFont{C}$} is the category \index[notation]{CoPShC@$\CoPSh(\CatFont{C})$}$\CoPSh(\CatFont{C})$ defined by
    \[
        \CoPSh(\CatFont{C})
        \defeq
        \Fun(\CatFont{C},\Sets).
    \]%
\end{definition}
\begin{remark}{Unwinding \cref{the-category-of-copresheaves-on-a-category}}{unwinding-the-category-of-copresheaves-on-a-category}%
    In detail, the \textbf{category of copresheaves on $\CatFont{C}$} is the category $\CoPSh(\CatFont{C})$ where
    \begin{itemize}
        \item\SloganFont{Objects. }The objects of $\CoPSh(\CatFont{C})$ are copresheaves on $\CatFont{C}$ as in \cref{copresheaves-on-a-category}.
        \item\SloganFont{Morphisms. }The morphisms of $\CoPSh(\CatFont{C})$ are morphisms of copresheaves as in \cref{morphisms-of-copresheaves}, i.e.\ we have
            \[
                \Hom_{\CoPSh(\CatFont{C})}(F,G)%
                \defeq%
                \Nat(F,G)%
            \]%
            for each $F,G\in\Obj(\CoPSh(\CatFont{C}))$.
        \item\SloganFont{Identities. }For each $F\in\Obj(\CoPSh(\CatFont{C}))$, the unit map
            \[
                \Unit^{\CoPSh(\CatFont{C})}_{F}%
                \colon%
                \pt%
                \to%
                \Nat(F,F)%
            \]%
            of $\CoPSh(\CatFont{C})$ at $F$ is defined by
            \[
                \id^{\CoPSh(\CatFont{C})}_{F}
                \defeq
                \id_{F},%
            \]%
            where $\id_{F}\colon F\Rightarrow F$ is the identity natural transformation of \ChapterRef{\ChapterCategories, \cref{categories:identity-natural-transformations}}{\cref{identity-natural-transformations}}.
        \item\SloganFont{Composition. }For each $F,G,H\in\Obj(\CoPSh(\CatFont{C}))$, the composition map
            \[
                \circ^{\CoPSh(\CatFont{C})}_{F,G,H}
                \colon
                \Nat(G,H)
                \times
                \Nat(F,G)
                \to
                \Nat(F,H)
            \]%
            of $\CoPSh(\CatFont{C})$ at $(F,G,H)$ is defined by
            \[
                \beta\circ^{\CoPSh(\CatFont{C})}_{F,G,H}\alpha%
                \defeq
                \beta\circ\alpha,%
            \]%
            where $\beta\circ\alpha\colon F\Rightarrow H$ is the vertical composition of $\alpha$ and $\beta$ of \ChapterRef{\ChapterCategories, \cref{categories:vertical-composition-of-natural-transformations}}{\cref{vertical-composition-of-natural-transformations}}.
    \end{itemize}
\end{remark}
\subsection{Corepresentable Copresheaves}\label{subsection-corepresentable-copresheaves}
Let $\CatFont{C}$ be a category.
\begin{definition}{Corepresentable Copresheaves}{corepresentable-copresheaves}%
    Let $A\in\Obj(\CatFont{C})$.
    \begin{enumerate}
        \item\label{corepresentable-copresheaves-corepresentable-copresheaves-associated-to-an-object}The \index[categories]{copresheaf!corepresentable associated to an object}\textbf{corepresentable copresheaf associated to $A$} is the copresheaf \index[notation]{hA@$h^{A}$}
            \[
                h^{A}%
                \colon%
                \CatFont{C}%
                \to%
                \Sets%
            \]%
            where
            \begin{itemize}
                \item\SloganFont{Action on Objects. }For each $X\in\Obj(\CatFont{C})$, we have
                    \[
                        h^{A}(X)
                        \defeq
                        \Hom_{\CatFont{C}}(A,X).
                    \]%
                \item\SloganFont{Action on Morphisms. }For each $X,Y\in\Obj(\CatFont{C})$, the action on morphisms
                    \[
                        h^{A}_{X,Y}%
                        \colon%
                        \Hom_{\CatFont{C}}(X,Y)%
                        \to%
                        \Hom_{\Sets}(h^{A}(X),h^{A}(Y))%
                    \]%
                    of $h^{A}$ at $(X,Y)$ is given by sending a morphism
                    \[
                        f%
                        \colon%
                        X%
                        \to%
                        Y%
                    \]%
                    of $\CatFont{C}$ to the map of sets
                    \[
                        h^{A}(f)
                        \colon
                        \underbrace{h^{A}(X)}_{\defeq\Hom_{\CatFont{C}}(A,X)}
                        \to
                        \underbrace{h^{A}(Y)}_{\defeq\Hom_{\CatFont{C}}(A,Y)}
                    \]%
                    defined by
                    \[
                        h^{A}(f)
                        \defeq
                        f_{*},
                    \]%
                    where $f_{*}$ is the postcomposition by $f$ morphism of \ChapterRef{\ChapterCategories, \cref{categories:precomposition-and-postcomposition-functions-postcomposition} of \cref{categories:precomposition-and-postcomposition-functions}}{\cref{precomposition-and-postcomposition-functions-postcomposition} of \cref{precomposition-and-postcomposition-functions}}.
            \end{itemize}
        \item\label{corepresentable-copresheaves-corepresenting-objects-for-copresheaves}A \index[categories]{copresheaf!corepresenting object of}\textbf{corepresenting object} for a copresheaf $F\colon\CatFont{C}\to\Sets$ on $\CatFont{C}$ is an object $A$ of $\CatFont{C}$ such that we have $F\cong h^{A}$.
        \item\label{corepresentable-copresheaves-corepresentable-copresheaves}A copresheaf $F\colon\CatFont{C}^{\op}\to\Sets$ on $\CatFont{C}$ is \index[categories]{copresheaf!corepresentable}\textbf{corepresentable} if $F$ admits a corepresenting object.
    \end{enumerate}
\end{definition}
\begin{example}{Corepresentable Copresheaves on One-Object Categories}{corepresentable-copresheaves-on-one-object-categories}%
    The corepresentable copresheaf on the delooping $\B{A}$ of a monoid $A$ associated to the unique object $\bullet$ of $\B{A}$ is the right regular representation of $A$ of \ChapterRef{\ChapterMonoidActions, \cref{monoid-actions:the-right-regular-representation-of-a-monoid}}{\cref{the-right-regular-representation-of-a-monoid}}.
\end{example}
\begin{proposition}{Uniqueness of Corepresenting Objects Up to Isomorphism}{uniqueness-of-corepresenting-objects-up-to-isomorphism}%
    Let $F\colon\CatFont{C}\to\Sets$ be a copresheaf. If there exist $A,B\in\Obj(\CatFont{C})$ such that we have natural isomorphisms
    \begin{align*}
        h^{A} &\cong F,\\
        h^{B} &\cong F,
    \end{align*}
    then $A\cong B$.
\end{proposition}
\begin{Proof}{Proof of \cref{uniqueness-of-corepresenting-objects-up-to-isomorphism}}%
    By composing the isomorphisms $h^{A}\cong F\cong h^{B}$, we get a natural isomorphism $h^{A}\cong h^{B}$. By \cref{properties-of-the-contravariant-yoneda-embedding-preservation-and-reflection-of-isomorphisms} of \cref{properties-of-the-contravariant-yoneda-embedding}, we have $A\cong B$.
\end{Proof}
\subsection{Corepresentable Natural Transformations}\label{subsection-corepresentable-natural-transformations}
Let $\CatFont{C}$ be a category, let $A,B\in\Obj(\CatFont{C})$, and let $f\colon A\to B$ be a morphism of $\CatFont{C}$.
\begin{definition}{Corepresentable Natural Transformations}{corepresentable-natural-transformations}%
    The \index[categories]{corepresentable natural transformation}\textbf{corepresentable natural transformation associated to $f$} is the natural transformation
    \[
        h^{f}%
        \colon%
        h^{B}%
        \Rightarrow%
        h^{A}%
    \]%
    consisting of the collection
    \[
        \{
            h^{f}_{X}
            \colon
            \underbrace{h^{B}(X)}_{\defeq\Hom_{\CatFont{C}}(B,X)}
            \to
            \underbrace{h^{A}(X)}_{\defeq\Hom_{\CatFont{C}}(A,X)}
        \}_{X\in\Obj(\CatFont{C})}
    \]%
    with
    \[
        h^{f}_{X}
        \defeq
        f^{*},
    \]%
    where $f_{*}$ is the precomposition by $f$ morphism of \ChapterRef{\ChapterCategories, \cref{categories:precomposition-and-postcomposition-functions-precomposition} of \cref{categories:precomposition-and-postcomposition-functions}}{\cref{precomposition-and-postcomposition-functions-precomposition} of \cref{precomposition-and-postcomposition-functions}}.
\end{definition}
\subsection{The Contravariant Yoneda Embedding}\label{subsection-the-contravariant-yoneda-embedding}
\begin{definition}{The Contravariant Yoneda Embedding}{the-contravariant-yoneda-embedding}%
    The \index[categories]{Yoneda embedding!contravariant}\textbf{contravariant Yoneda embedding of $\CatFont{C}$} is the functor\index[notation]{yo@$\coyo$}%
    %--- Begin Footnote ---%
    \footnote{%
        \SloganFont{Further Notation: }Also written $h^{(-)}$, or simply $\coyo$.
        \par\vspace*{-1.75\baselineskip}
    } %
    %---  End Footnote  ---%
    \[
        \coyo_{\CatFont{C}}%
        \colon%
        \CatFont{C}^{\op}%
        \longhookrightarrow%
        \CoPSh(\CatFont{C})%
    \]%
    where
    \begin{itemize}
        \item\SloganFont{Action on Objects. }For each $A\in\Obj(\CatFont{C})$, we have
            \[
                \coyo_{\CatFont{C}}(A)
                \defeq
                h^{A}.
            \]%
        \item\SloganFont{Action on Morphisms. }For each $A,B\in\Obj(\CatFont{C})$, the action on morphisms
            \[
                \coyo_{\CatFont{C}|A,B}%
                \colon%
                \Hom_{\CatFont{C}}(A,B)%
                \to%
                \Nat(h^{B},h^{A})%
            \]%
            of $\coyo_{\CatFont{C}}$ at $(A,B)$ is given by
            \[
                \coyo_{\CatFont{C}|A,B}(f)%
                \defeq%
                h^{f}%
            \]%
            for each $f\in\Hom_{\CatFont{C}}(A,B)$, where $h^{f}$ is the corepresentable natural transformation associated to $f$ of \cref{corepresentable-natural-transformations}.
    \end{itemize}
\end{definition}
\begin{proposition}{Properties of the Contravariant Yoneda Embedding}{properties-of-the-contravariant-yoneda-embedding}%
    Let $\CatFont{C}$ be a category.
    \begin{enumerate}
        \item\label{properties-of-the-contravariant-yoneda-embedding-fully-faithfulness}\SloganFont{Fully Faithfulness. }The contravariant Yoneda embedding
            \[
                \coyo_{\CatFont{C}}%
                \colon%
                \CatFont{C}^{\op}%
                \hookrightarrow%
                \CoPSh(\CatFont{C})%
            \]%
            is fully faithful.
        \item\label{properties-of-the-contravariant-yoneda-embedding-preservation-and-reflection-of-isomorphisms}\SloganFont{Preservation and Reflection of Isomorphisms. }The contravariant Yoneda embedding
            \[
                \coyo_{\CatFont{C}}%
                \colon%
                \CatFont{C}^{\op}%
                \hookrightarrow%
                \CoPSh(\CatFont{C})%
            \]%
            preserves and reflects isomorphisms, i.e.\ given $A,B\in\Obj(\CatFont{C})$, the following conditions are equivalent:
            \begin{enumerate}
                \item We have $A\cong B$.
                \item We have $h^{A}\cong h^{B}$.
            \end{enumerate}
        %\item\label{properties-of-the-contravariant-yoneda-embedding-}\SloganFont{. }
    \end{enumerate}
\end{proposition}
\begin{Proof}{Proof of \cref{properties-of-the-contravariant-yoneda-embedding}}%
    \FirstProofBox{\cref{properties-of-the-contravariant-yoneda-embedding-fully-faithfulness}: Fully Faithfulness}%
    The proof is dual to that of \cref{properties-of-the-yoneda-embedding-fully-faithfulness} of \cref{properties-of-the-yoneda-embedding}, and is therefore omitted.

    \ProofBox{\cref{properties-of-the-contravariant-yoneda-embedding-preservation-and-reflection-of-isomorphisms}: Preservation and Reflection of Isomorphisms}%
    This follows from \ChapterRef{\ChapterCategories, \cref{categories:elementary-properties-of-functors-preservation-of-isomorphisms} of \cref{categories:elementary-properties-of-functors} and \cref{categories:properties-of-fully-faithful-functors-conservativity} of \cref{categories:properties-of-fully-faithful-functors}}{\cref{elementary-properties-of-functors-preservation-of-isomorphisms} of \cref{elementary-properties-of-functors} and \cref{properties-of-fully-faithful-functors-conservativity} of \cref{properties-of-fully-faithful-functors}}.
\end{Proof}
\subsection{The Contravariant Yoneda Lemma}\label{subsection-the-contravariant-yoneda-lemma}
Let $F\colon\CatFont{C}\to\Sets$ be a copresheaf on $\CatFont{C}$.
\begin{theorem}{The Contravariant Yoneda Lemma}{the-contravariant-yoneda-lemma}%
    \index[categories]{Yoneda lemma!contravariant}We have a bijection
    \[
        \Nat(h^{A},F)
        \cong
        F(A),
    \]%
    natural in $A\in\Obj(\CatFont{C})$, determining a natural isomorphism of functors
    \[
        \Nat(h^{(-)},F)
        \cong
        F.
    \]%
\end{theorem}
\begin{Proof}{Proof of \cref{the-contravariant-yoneda-lemma}}%
    The proof is dual to that of \cref{the-yoneda-lemma}, and is therefore omitted.
\end{Proof}%
\section{Restricted Yoneda Embeddings and Yoneda Extensions}\label{section-restricted-yoneda-embeddings-and-yoneda-extensions}
\subsection{Foundations}\label{subsection-restricted-yoneda-embeddings-foundations}
let $F\colon\CatFont{C}\to\CatFont{D}$ be a functor.
\begin{definition}{The Restricted Yoneda Embedding Associated to a Functor}{the-restricted-yoneda-embedding-associated-to-a-functor}%
    The \index[categories]{Yoneda embedding!restricted}\index[categories]{restricted Yoneda embedding}\textbf{restricted Yoneda embedding associated to $F$} is the functor\index[notation]{yoF@$\yo_{F}$}
    \[
        \yo_{F}%
        \colon%
        \CatFont{D}%
        \longhookrightarrow%
        \PSh(\CatFont{C})%
    \]%
    defined as the composition
    \[
        \CatFont{D}
        \xhookrightarrow{\yo_{\CatFont{D}}}
        \PSh(\CatFont{D})
        \xrightarrow{F^{\op,*}}
        \PSh(\CatFont{C}).
    \]%
\end{definition}
\begin{remark}{Unwinding \cref{the-restricted-yoneda-embedding-associated-to-a-functor}}{unwinding-the-restricted-yoneda-embedding-associated-to-a-functor}%
    In detail, the \textbf{restricted Yoneda embedding associated to $F$} is the functor
    \[
        \yo_{F}%
        \colon%
        \CatFont{D}%
        \longhookrightarrow%
        \PSh(\CatFont{C})%
    \]%
    where
    \begin{itemize}
        \item\SloganFont{Action on Objects. }For each $A\in\Obj(\CatFont{D})$, we have
            \begin{align*}
                \yo_{F}(A) &\defeq h_{A}\circ F^{\op}\\
                           &\defeq h^{F(-)}_{A}.
            \end{align*}
        \item\SloganFont{Action on Morphisms. }For each $A,B\in\Obj(\CatFont{D})$, the action on morphisms
            \[
                \yo_{F|A,B}%
                \colon%
                \Hom_{\CatFont{D}}(A,B)%
                \to%
                \Nat(h^{F(-)}_{A},h^{F(-)}_{B})%
            \]%
            of $\yo_{F}$ at $(A,B)$ is given by
            \begin{align*}
                \yo_{F|A,B}(f) &\defeq h^{F(-)}_{f}\\%
                               &\defeq h_{f}\star\id_{F^{\op}}%
            \end{align*}
            for each $f\in\Hom_{\CatFont{D}}(A,B)$, where $h_{f}$ is the representable natural transformation associated to $f$ of \cref{representable-natural-transformations}.
    \end{itemize}
\end{remark}
\begin{example}{Examples of Restricted Yoneda Embeddings}{examples-of-restricted-yoneda-embeddings}%
    Here are some examples of restricted Yoneda embeddings.
    \begin{enumerate}
        \item\label{examples-of-restricted-yoneda-embeddings-the-nerve-functor}\SloganFont{The Nerve Functor. }Let
            \[
                \iota%
                \colon%
                \SimplexCategory%
                \hookrightarrow%
                \Cats%
            \]%
            be the functor given by $[n]\to\OrdinalCategoryN{n}$. Then the restricted Yoneda embedding
            \[
                \yo_{\iota}%
                \colon%
                \Cats%
                \to%
                \underbrace{\PSh(\SimplexCategory)}_{\defeq\sSets}%
            \]%
            of $\iota$ is given by the nerve functor $\NerveB$ of \ChapterRef{\ChapterSimplicialObjects, \cref{simplicial-objects:the-nerve-functor}}{\cref{the-nerve-functor}}.
        \item\label{examples-of-restricted-yoneda-embeddings-the-singular-complex-of-a-topological-space}\SloganFont{The Singular Simplicial Set Associated to a Topological Space. }Let
            \[
                \iota%
                \colon%
                \SimplexCategory%
                \hookrightarrow%
                \Top%
            \]%
            be the functor given by $[n]\to\abs{\Delta^{n}}$. Then the restricted Yoneda embedding
            \[
                \yo_{\iota}%
                \colon%
                \Top%
                \to%
                \underbrace{\PSh(\SimplexCategory)}_{\defeq\sSets}%
            \]%
            of $\iota$ is given by the singular simplicial set functor $\Sing_{\bullet}$ of \ChapterRef{\ChapterSimplicialHomotopyTheory, \cref{simplicial-homotopy-theory:the-singular-simplicial-set-functor}}{\cref{the-singular-simplicial-set-functor}}.
        \item\label{examples-of-restricted-yoneda-embeddings-the-coherent-nerve-functor}\SloganFont{The Coherent Nerve Functor. }Let
            \[
                \iota%
                \colon%
                \SimplexCategory%
                \hookrightarrow%
                \sCats%
            \]%
            be the functor given by $[n]\to\Path(\Delta^{n})$, where $\Path(\Delta^{n})$ is the simplicial category of \ChapterRef{\ChapterSimplicialCategories, \cref{simplicial-categories-the-path-simplicial-category-of-delta-n}}{\cref{the-path-simplicial-category-of-delta-n}}. Then the restricted Yoneda embedding
            \[
                \yo_{\iota}%
                \colon%
                \sCats%
                \to%
                \underbrace{\PSh(\SimplexCategory)}_{\defeq\sSets}%
            \]%
            of $\iota$ is given by the coherent nerve functor $\sNerveB$ of \ChapterRef{\ChapterSimplicialCategories, \cref{simplicial-categories:the-homotopy-coherent-nerve-functor}}{\cref{the-homotopy-coherent-nerve-functor}}.
        \item\label{examples-of-restricted-yoneda-embeddings-kan-s-ex-functor}\SloganFont{Kan's $\Ex$ Functor. }Let
            \[
                \sd%
                \colon%
                \SimplexCategory%
                \hookrightarrow%
                \sSets%
            \]%
            be the functor given by $[n]\to\Sd(\Delta^{n})$, where $\Sd(\Delta^{n})$ is the barycentric subdivision of $\Delta^{n}$ of \cref{TODO}. Then the restricted Yoneda embedding
            \[
                \yo_{\sd}%
                \colon%
                \sSets%
                \to%
                \underbrace{\PSh(\SimplexCategory)}_{\defeq\sSets}%
            \]%
            of $\sd$ is given by Kan's $\Ex$ functor of \cref{TODO}.
    \end{enumerate}
\end{example}
\begin{proposition}{Properties of the Restricted Yoneda Embedding}{properties-of-the-restricted-yoneda-embedding}%
    let $F\colon\CatFont{C}\to\CatFont{D}$ be a functor.
    \begin{enumerate}
        \item\label{properties-of-the-restricted-yoneda-embedding-interaction-with-fully-faithfulness}\SloganFont{Interaction With Fully Faithfulness. }The following conditions are equivalent:
            \begin{enumerate}
                \item\label{properties-of-the-restricted-yoneda-embedding-interaction-with-fully-faithfulness-1}The restricted Yoneda embedding $\yo_{F}$ is fully faithful.
                \item\label{properties-of-the-restricted-yoneda-embedding-interaction-with-fully-faithfulness-2}The functor $F$ is dense (\ChapterRef{\ChapterLimitsAndColimits, \cref{limits-and-colimits:dense-functors}}{\cref{dense-functors}}).
            \end{enumerate}
        \item\label{properties-of-the-restricted-yoneda-embedding-as-a-left-kan-extension}\SloganFont{As a Left Kan Extension. }We have a natural isomorphism of functors
            \begin{webcompile}
                \yo_{F}%
                \cong%
                \Lan_{F}(\yo),%
                \quad
                \begin{tikzcd}[row sep={5.0*\the\DL,between origins}, column sep={5.0*\the\DL,between origins}, background color=backgroundColor, ampersand replacement=\&]
                    \&
                    \CatFont{D}
                    \arrow[d, "\yo_{F}",dashed]
                    \\
                    \CatFont{C}
                    \arrow[ru, "F"]
                    \arrow[r, "\yo_{\CatFont{C}}"'{name=F}]
                    \&
                    {\PSh(\CatFont{C})}\mrp{.}
                    % 2-Arrows
                    \arrow[from=F,to=1-2,Rightarrow,shorten=0.75em,pos=0.4,start anchor={[xshift=-0.15*\the\DL]},end anchor={[xshift=0.45*\the\DL]}]
                \end{tikzcd}
            \end{webcompile}
        %\item\label{properties-of-the-restricted-yoneda-embedding-}\SloganFont{. }
    \end{enumerate}
\end{proposition}
\begin{Proof}{Proof of \cref{properties-of-the-restricted-yoneda-embedding}}%
    \FirstProofBox{\cref{properties-of-the-restricted-yoneda-embedding-interaction-with-fully-faithfulness}: Interaction With Fully Faithfulness}%
    Omitted.

    \ProofBox{\cref{properties-of-the-restricted-yoneda-embedding-as-a-left-kan-extension}: As a Left Kan Extension}%
    Omitted.
\end{Proof}
\subsection{The Yoneda Extension Functor}\label{subsection-the-yoneda-extension-functor}
Let $F\colon\CatFont{C}\to\CatFont{D}$ be a functor with $\CatFont{C}$ small and $\CatFont{D}$ cocomplete.
\begin{definition}{The Yoneda Extension Functor}{the-yoneda-extension-functor}%
    The \index[categories]{Yoneda extension}\textbf{Yoneda extension functor associated to $F$} is the left Kan extension
    \begin{webcompile}
        \Lan_{\yo}(F)%
        \colon%
        \PSh(\CatFont{C})%
        \to%
        \CatFont{D},%
        \quad
        \begin{tikzcd}[row sep={5.0*\the\DL,between origins}, column sep={5.0*\the\DL,between origins}, background color=backgroundColor, ampersand replacement=\&]
            \&
            {\PSh(\CatFont{C})}
            \arrow[d, "\Lan_{\yo}(F)",dashed]
            \\
            \CatFont{C}
            \arrow[ru, "\yo_{\CatFont{C}}"]
            \arrow[r, "F"'{name=F}]
            \&
            \CatFont{D}\mrp{.}
            % 2-Arrows
            \arrow[from=F,to=1-2,Rightarrow,shorten=0.75em,pos=0.4,start anchor={[xshift=-0.15*\the\DL]},end anchor={[xshift=0.2*\the\DL]}]
        \end{tikzcd}
    \end{webcompile}
\end{definition}
\begin{example}{Examples of Yoneda Extensions}{examples-of-yoneda-extensions}%
    Here are some examples of Yoneda extensions.
    \begin{enumerate}
        \item\label{examples-of-yoneda-extensions-the-homotopy-category-functor}\SloganFont{The Homotopy Category Functor. }Let
            \[
                \iota%
                \colon%
                \SimplexCategory%
                \hookrightarrow%
                \Cats%
            \]%
            be the functor given by $[n]\to\OrdinalCategoryN{n}$. Then the Yoneda extension
            \[
                \Lan_{\yo}(\iota)%
                \colon%
                \underbrace{\PSh(\SimplexCategory)}_{\defeq\sSets}%
                \to%
                \Cats%
            \]%
            of $\iota$ is given by the homotopy category functor $\Ho$ of \ChapterRef{\ChapterSimplicialObjects, \cref{simplicial-objects:the-homotopy-category-functor}}{\cref{the-homotopy-category-functor}}.
        \item\label{examples-of-yoneda-extensions-the-geometric-realisation-functor}\SloganFont{The Geometric Realisation Functor. }Let
            \[
                \iota%
                \colon%
                \SimplexCategory%
                \hookrightarrow%
                \Top%
            \]%
            be the functor given by $[n]\to\abs{\Delta^{n}}$. Then the Yoneda extension
            \[
                \Lan_{\yo}(\iota)%
                \colon%
                \underbrace{\PSh(\SimplexCategory)}_{\defeq\sSets}%
                \to%
                \Top%
            \]%
            of $\iota$ is given by the geometric realisation functor $\abs{-}$ of \ChapterRef{\ChapterSimplicialHomotopyTheory, \cref{simplicial-homotopy-theory:the-geometric-realisation-functor}}{\cref{the-geometric-realisation-functor}}.
        \item\label{examples-of-yoneda-extensions-the-path-simplicial-category-functor}\SloganFont{The Path Simplicial Category Functor. }Let
            \[
                \iota%
                \colon%
                \SimplexCategory%
                \hookrightarrow%
                \sCats%
            \]%
            be the functor given by $[n]\to\Path(\Delta^{n})$, where $\Path(\Delta^{n})$ is the simplicial category of \ChapterRef{\ChapterSimplicialCategories, \cref{simplicial-categories-the-path-simplicial-category-of-delta-n}}{\cref{the-path-simplicial-category-of-delta-n}}. Then the Yoneda extension
            \[
                \Lan_{\yo}(\iota)%
                \colon%
                \underbrace{\PSh(\SimplexCategory)}_{\defeq\sSets}%
                \to%
                \sCats%
            \]%
            of $\iota$ is given by the path simplicial category functor $\Path$ of \ChapterRef{\ChapterSimplicialCategories, \cref{simplicial-categories:the-path-simplicial-category-functor}}{\cref{the-path-simplicial-category-functor}}.
        \item\label{examples-of-yoneda-extensions-the-barycenteric-subdivision-functor}\SloganFont{The Barycentric Subdivision Functor. }Let
            \[
                \sd%
                \colon%
                \SimplexCategory%
                \hookrightarrow%
                \sSets%
            \]%
            be the functor given by $[n]\to\Sd(\Delta^{n})$, where $\Sd(\Delta^{n})$ is the barycentric subdivision of $\Delta^{n}$ of \cref{TODO}. Then the Yoneda extension
            \[
                \Lan_{\yo}(\sd)%
                \colon%
                \underbrace{\PSh(\SimplexCategory)}_{\defeq\sSets}%
                \to%
                \sSets%
            \]%
            of $\sd$ is given by the barycentric subdivision functor $\Sd$ of \cref{TODO}.
    \end{enumerate}
\end{example}
\begin{proposition}{Properties of Yoneda Extensions}{properties-of-yoneda-extensions}%
    Let $F\colon\CatFont{C}\to\CatFont{D}$ be a functor with $\CatFont{C}$ small and $\CatFont{D}$ cocomplete.
    \begin{enumerate}
        \item\label{properties-of-yoneda-extensions-functoriality}\SloganFont{Functoriality. }The assignment $F\mapsto\Lan_{\yo}(F)$ defines a functor
            \[
                \Lan_{\yo}%
                \colon%
                \Fun(\CatFont{C},\CatFont{D})%
                \to%
                \Fun(\PSh(\CatFont{C}),\CatFont{D}).%
            \]%
        \item\label{properties-of-yoneda-extensions-adjointness}\SloganFont{Adjointness. }We have an adjunction%
            %--- Begin Footnote ---%
            \footnote{%
                Applying \cref{properties-of-the-restricted-yoneda-embedding-as-a-left-kan-extension} of \cref{properties-of-the-restricted-yoneda-embedding}, we see that this adjunction has the form $\Lan_{\yo}(F)\dashv\Lan_{F}(\yo)$.
                \par\vspace*{-1.75\baselineskip}
            }%
            %---  End Footnote  ---%
            \begin{webcompile}
                \Adjunction#\Lan_{\yo}(F)#\yo_{F}#\PSh(\CatFont{C})#\CatFont{D},#
            \end{webcompile}%
            witnessed by a bijection
            \[
                \Hom_{\CatFont{D}}([\Lan_{\yo}(F)](\SheafFont{F}),D)%
                \cong
                \Nat(\SheafFont{F},\yo_{F}(D)),
            \]%
            natural in $\SheafFont{F}\in\Obj(\PSh(\CatFont{C}))$ and $D\in\Obj(\CatFont{D})$.
        \item\label{properties-of-yoneda-extensions-interaction-with-the-yoneda-embedding}\SloganFont{Interaction With the Yoneda Embedding. }We have a natural isomorphism of functors
            \begin{webcompile}
                \Lan_{\yo}(F)\circ\yo_{\CatFont{C}}%
                \cong%
                F,%
                \quad
                \begin{tikzcd}[row sep={5.0*\the\DL,between origins}, column sep={5.0*\the\DL,between origins}, background color=backgroundColor, ampersand replacement=\&]
                    \&
                    {\PSh(\CatFont{C})}
                    \arrow[d, "\Lan_{\yo}(F)",dashed]
                    \\
                    \CatFont{C}
                    \arrow[ru, "\yo_{\CatFont{C}}"]
                    \arrow[r, "F"'{name=F}]
                    \&
                    \CatFont{D}\mrp{.}
                    % 2-Arrows
                    \arrow[from=F,to=1-2,Leftrightarrow,dashed,shorten=0.5em,pos=0.4,shift right=0.1*\the\DL]
                \end{tikzcd}
            \end{webcompile}
        \item\label{properties-of-yoneda-extensions-as-a-coend}\SloganFont{As a Coend. }We have
            \begin{align*}
                [\Lan_{\yo}(F)](\SheafFont{F}) &\cong \int^{A\in\CatFont{C}}\Nat(h_{A},\SheafFont{F})\odot F(A)\\
                                               &\cong \int^{A\in\CatFont{C}}\SheafFont{F}(A)\odot F(A)
            \end{align*}
            for each $\SheafFont{F}\in\Obj(\PSh(\CatFont{C}))$.
        \item\label{properties-of-yoneda-extensions-interaction-with-tensors-of-presheaves-with-functors}\SloganFont{Interaction With Tensors of Presheaves With Functors. }We have a natural isomorphism
            \[
                \Lan_{\yo}(F)%
                \cong
                (-)\odot_{\CatFont{C}}F,%
            \]%
            natural in $F\in\Obj(\Fun(\CatFont{C},\CatFont{D}))$.
        \item\label{properties-of-the-yoneda-extensions-interaction-with-finite-limits}\SloganFont{Interaction With Finite Limits. }Let $F\colon\CatFont{C}\to\Sets$ be a functor. The following conditions are equivalent:
            \begin{enumerate}
                \item\label{properties-of-the-restricted-yoneda-embedding-interaction-with-finite-limits-1}The functor $F$ preserves finite limits.
                \item\label{properties-of-the-restricted-yoneda-embedding-interaction-with-finite-limits-2}The functor $\Lan_{\yo}(F)$ preserves finite limits.
                \item\label{properties-of-the-restricted-yoneda-embedding-interaction-with-finite-limits-3}The category of elements $\catEl{\CatFont{C}}{F}$ of $F$ is cofiltered.
            \end{enumerate}
        %\item\label{properties-of-yoneda-extensions-}\SloganFont{. }
    \end{enumerate}
\end{proposition}
\begin{Proof}{Proof of \cref{properties-of-yoneda-extensions}}%
    \FirstProofBox{\cref{properties-of-yoneda-extensions-functoriality}: Functoriality}%
    This follows from \ChapterRef{\ChapterKanExtensions, \cref{categories:properties-of-kan-extensions-functoriality} of \cref{categories:properties-of-kan-extensions}}{\cref{properties-of-kan-extensions-functoriality} of \cref{properties-of-kan-extensions}}.

    \ProofBox{\cref{properties-of-yoneda-extensions-adjointness}: Adjointness}%
    Omitted.

    \ProofBox{\cref{properties-of-yoneda-extensions-interaction-with-the-yoneda-embedding}: Interaction With the Yoneda Embedding}%
    This follows from \ChapterRef{\ChapterKanExtensions, \cref{categories:properties-of-kan-extensions-interaction-with-fully-faithfulness} of \cref{categories:properties-of-kan-extensions}}{\cref{properties-of-kan-extensions-interaction-with-fully-faithfulness} of \cref{properties-of-kan-extensions}}.

    \ProofBox{\cref{properties-of-yoneda-extensions-as-a-coend}: As a Coend}%
    This follows from \ChapterRef{\ChapterKanExtensions, \cref{categories:properties-of-kan-extensions-as-co-ends} of \cref{categories:properties-of-kan-extensions}}{\cref{properties-of-kan-extensions-as-co-ends} of \cref{properties-of-kan-extensions}} and \cref{the-yoneda-lemma}.

    \ProofBox{\cref{properties-of-yoneda-extensions-interaction-with-tensors-of-presheaves-with-functors}: Interaction With Tensors of Presheaves With Functors}%
    This follows from \cref{properties-of-yoneda-extensions-as-a-coend}.

    \ProofBox{\cref{properties-of-the-yoneda-extensions-interaction-with-finite-limits}: Interaction With Finite Limits}%
    See \cite[Proposition 3.2.15]{coend-calculus}.
\end{Proof}
\section{Functor Tensor Products}\label{section-functor-tensor-products}
\subsection{The Tensor Product of Presheaves With Copresheaves}\label{subsection-the-tensor-product-of-presheaves-with-copresheaves}
Let $\CatFont{C}$ be a category, let $\SheafFont{F}\colon\CatFont{C}^{\op}\to\Sets$ be a presheaf on $\CatFont{C}$, and let $G\colon\CatFont{C}\to\Sets$ be a copresheaf on $\CatFont{C}$.
\begin{definition}{The Tensor Product of Presheaves With Copresheaves}{the-tensor-product-of-presheaves-with-copresheaves}%
    The \index[categories]{tensor product of functors}\index[categories]{functor tensor product}\textbf{tensor product} of $\SheafFont{F}$ with $G$ is the set \index[notation]{FboxtimesCG@$\SheafFont{F}\boxtimes_{\CatFont{C}}G$}$\SheafFont{F}\boxtimes_{\CatFont{C}}G$%
    %--- Begin Footnote ---%
    \footnote{%
        \SloganFont{Further Notation: }Also written simply \index[notation]{FboxtimesG@$\SheafFont{F}\boxtimes G$}$\SheafFont{F}\boxtimes G$.
        \par\vspace*{-1.75\baselineskip}
    } %
    %---  End Footnote  ---%
    defined by
    \[
        \SheafFont{F}\boxtimes_{\CatFont{C}}G%
        \defeq%
        \int^{A\in\CatFont{C}}\SheafFont{F}(A)\times G(A).%
    \]%
\end{definition}
\begin{remark}{Unwinding \cref{the-tensor-product-of-presheaves-with-copresheaves}}{unwinding-the-tensor-product-of-presheaves-with-copresheaves}%
    In other words, the tensor product of $\SheafFont{F}$ with $G$ is the set $\SheafFont{F}\boxtimes_{\CatFont{C}}G$ defined as the coend of the functor
    \[
        \CatFont{C}^{\op}\times\CatFont{C}%
        \xrightarrow{\SheafFont{F}\times G}%
        \Sets\times\Sets%
        \xrightarrow{\times}%
        \Sets,%
    \]%
    which is equivalently the composition
    \begin{webcompile}
        \mathord{\times}\circ(\SheafFont{F}\times G)%
        \cong%
        \SheafFont{F}\procirc F,%
        \mkern27mu%
        \begin{tikzcd}[row sep={5.0*\the\DL,between origins}, column sep={5.0*\the\DL,between origins}, background color=backgroundColor, ampersand replacement=\&]
            \CatFont{C}
            \arrow[r,"F",mid vert]
            \arrow[rd,"{\mathord{\times}\circ(\SheafFont{F}\times G)}"',mid vert]
            \&
            \PunctualCategory
            \arrow[d,"\SheafFont{F}",mid vert]
            \\
            \&
            \CatFont{C}
        \end{tikzcd}
    \end{webcompile}
    in $\Prof$.
\end{remark}
\begin{example}{The Tensor Product of Presheaves With Copresheaves on One Object Categories}{the-tensor-product-of-presheaves-with-copresheaves-on-one-object-categories}%
\end{example}
\begin{proposition}{Properties of Tensor Products of Presheaves With Copresheaves}{properties-of-tensor-products-of-presheaves-with-copresheaves}%
    Let $\CatFont{C}$ be a category.
    \begin{enumerate}
        \item\label{properties-of-tensor-products-of-presheaves-with-copresheaves-functoriality}\SloganFont{Functoriality. }The assignments $\SheafFont{F},G,(\SheafFont{F},G)\mapsto\SheafFont{F}\boxtimes_{\CatFont{C}}G$ define functors
            \[
                \BifunctorialityPeriod{\SheafFont{F}\boxtimes_{\CatFont{C}}-}{-\boxtimes_{\CatFont{C}}G}{-_{1}\boxtimes_{\CatFont{C}}-_{2}}{\PSh(\CatFont{C})}{\CoPSh(\CatFont{C})}{\PSh(\CatFont{C})\times\CoPSh(\CatFont{C})}{\Sets}%
            \]%
        \item\label{properties-of-tensor-products-of-presheaves-with-copresheaves-as-a-composition-of-profunctors}\SloganFont{As a Composition of Profunctors. }Let $\CatFont{C}$ be a category and let:
            \begin{itemize}
                \item $\SheafFont{F}\colon\PunctualCategory\rightproarrow\CatFont{C}$ be a presheaf on $\CatFont{C}$, viewed as a profunctor.
                \item $F\colon\CatFont{C}\rightproarrow\PunctualCategory$             be a copresheaf on $\CatFont{C}$, viewed as a profunctor.
            \end{itemize}
            We have a natural isomorphism of profunctors
            \begin{webcompile}
                \SheafFont{F}\boxtimes_{\CatFont{C}}F%
                \cong%
                F\procirc\SheafFont{F},%
                \mkern27mu%
                \begin{tikzcd}[row sep={5.0*\the\DL,between origins}, column sep={5.0*\the\DL,between origins}, background color=backgroundColor, ampersand replacement=\&]
                    \&
                    \CatFont{C}
                    \arrow[d,"F",mid vert]
                    \\
                    \PunctualCategory
                    \arrow[ru,"\SheafFont{F}",mid vert]
                    \arrow[r,"\SheafFont{F}\boxtimes_{\CatFont{C}}F"'{name=F},mid vert]
                    \&
                    \PunctualCategory\mrp{,}
                    \arrow[from=F,to=1-2,Leftrightarrow,shorten=0.75em,pos=0.4,start anchor={[xshift=-0.4*\the\DL]},end anchor={[xshift=0.175*\the\DL]},densely dashed]
                \end{tikzcd}
            \end{webcompile}
            natural in $\SheafFont{F}\in\Obj(\PSh(\CatFont{C}))$ and $F\in\Obj(\CoPSh(\CatFont{C}))$.
        \item\label{properties-of-tensor-products-of-presheaves-with-copresheaves-interaction-with-representable-presheaves}\SloganFont{Interaction With Representable Presheaves. }Let $\SheafFont{F}$ be a presheaf on $\CatFont{C}$. We have a bijection of sets
            \[
                \SheafFont{F}\boxtimes_{\CatFont{C}}h^{X}%
                \cong
                \SheafFont{F}(X),%
            \]%
            natural in $X\in\Obj(\CatFont{C})$, giving a natural isomorphism of functors
            \begin{webcompile}
                \SheafFont{F}\boxtimes_{\CatFont{C}}h^{(-)}%
                \cong
                \SheafFont{F},%
                \mkern27mu%
                \begin{tikzcd}[row sep={5.0*\the\DL,between origins}, column sep={5.0*\the\DL,between origins}, background color=backgroundColor, ampersand replacement=\&]
                    \&
                    \CoPSh(\CatFont{C})
                    \arrow[d,"\SheafFont{F}\boxtimes_{\CatFont{C}}-"]
                    \\
                    \CatFont{C}^{\op}
                    \arrow[ru,"\coyo_{\CatFont{C}}"]
                    \arrow[r,"\SheafFont{F}"'{name=F}]
                    \&
                    \Sets\mrp{.}
                    \arrow[from=F,to=1-2,Leftrightarrow,shorten=0.75em,pos=0.4,start anchor={[xshift=-0.15*\the\DL]},end anchor={[xshift=0.2*\the\DL]},densely dashed]
                \end{tikzcd}
            \end{webcompile}
        \item\label{properties-of-tensor-products-of-presheaves-with-copresheaves-interaction-with-corepresentable-copresheaves}\SloganFont{Interaction With Corepresentable Copresheaves. }Let $G$ be a copresheaf on $\CatFont{C}$. We have a bijection of sets
            \[
                h_{X}\boxtimes_{\CatFont{C}}G%
                \cong
                G(X),%
            \]%
            natural in $X\in\Obj(\CatFont{C})$, giving a natural isomorphism of functors
            \begin{webcompile}
                h_{(-)}\boxtimes_{\CatFont{C}}G%
                \cong
                G,%
                \mkern27mu%
                \begin{tikzcd}[row sep={5.0*\the\DL,between origins}, column sep={5.0*\the\DL,between origins}, background color=backgroundColor, ampersand replacement=\&]
                    \&
                    \PSh(\CatFont{C})
                    \arrow[d,"-\boxtimes_{\CatFont{C}}G"]
                    \\
                    \CatFont{C}
                    \arrow[ru,"\yo_{\CatFont{C}}"]
                    \arrow[r,"G"'{name=F}]
                    \&
                    \Sets\mrp{.}
                    \arrow[from=F,to=1-2,Leftrightarrow,shorten=0.75em,pos=0.4,start anchor={[xshift=-0.15*\the\DL]},end anchor={[xshift=0.3*\the\DL]},densely dashed]
                \end{tikzcd}
            \end{webcompile}
        \item\label{properties-of-tensor-products-of-presheaves-with-copresheaves-interaction-with-yoneda-extensions}\SloganFont{Interaction With Yoneda Extensions. }Let $G\colon\CatFont{C}\to\Sets$ be a copresheaf on $\CatFont{C}$. We have a natural isomorphism
            \begin{webcompile}
                \Lan_{\yo}(G)%
                \cong
                (-)\boxtimes_{\CatFont{C}}G,%
                \quad
                \begin{tikzcd}[row sep={5.0*\the\DL,between origins}, column sep={5.0*\the\DL,between origins}, background color=backgroundColor, ampersand replacement=\&]
                    \&
                    {\PSh(\CatFont{C})}
                    \arrow[d, "{(-)\boxtimes_{\CatFont{C}}G}",dashed]
                    \\
                    \CatFont{C}
                    \arrow[ru, "\yo_{\CatFont{C}}"]
                    \arrow[r, "G"'{name=F}]
                    \&
                    \Sets\mrp{,}
                    % 2-Arrows
                    \arrow[from=F,to=1-2,Rightarrow,shorten=0.75em,pos=0.4,start anchor={[xshift=-0.15*\the\DL]},end anchor={[xshift=0.3*\the\DL]}]
                \end{tikzcd}
            \end{webcompile}
            natural in $G\in\Obj(\CoPSh(\CatFont{C}))$.
        \item\label{properties-of-tensor-products-of-presheaves-with-copresheaves-interaction-with-contravariant-yoneda-extensions}\SloganFont{Interaction With Contravariant Yoneda Extensions. }Let $\SheafFont{F}\colon\CatFont{C}^{\op}\to\Sets$ be a presheaf on $\CatFont{C}$. We have a natural isomorphism
            \begin{webcompile}
                \Lan_{\coyo}(\SheafFont{F})%
                \cong
                \SheafFont{F}\boxtimes_{\CatFont{C}}(-),%
                \quad
                \begin{tikzcd}[row sep={5.0*\the\DL,between origins}, column sep={5.0*\the\DL,between origins}, background color=backgroundColor, ampersand replacement=\&]
                    \&
                    {\CoPSh(\CatFont{C})}
                    \arrow[d, "{\SheafFont{F}\boxtimes_{\CatFont{C}}(-)}",dashed]
                    \\
                    \CatFont{C}^{\op}
                    \arrow[ru, "\coyo_{\CatFont{C}}"]
                    \arrow[r, "\SheafFont{F}"'{name=F}]
                    \&
                    \Sets\mrp{,}
                    % 2-Arrows
                    \arrow[from=F,to=1-2,Rightarrow,shorten=0.75em,pos=0.4,start anchor={[xshift=-0.15*\the\DL]},end anchor={[xshift=0.2*\the\DL]}]
                \end{tikzcd}
            \end{webcompile}
            natural in $\SheafFont{F}\in\Obj(\PSh(\CatFont{C}))$.
        %\item\label{properties-of-tensor-products-of-presheaves-with-copresheaves-}\SloganFont{. }
    \end{enumerate}
\end{proposition}
\begin{Proof}{Proof of \cref{properties-of-tensor-products-of-presheaves-with-copresheaves}}%
    \FirstProofBox{\cref{properties-of-tensor-products-of-presheaves-with-copresheaves-functoriality}: Functoriality}%
    Omitted.

    \ProofBox{\cref{properties-of-tensor-products-of-presheaves-with-copresheaves-as-a-composition-of-profunctors}: As a Composition of Profunctors}%
    Clear.

    \ProofBox{\cref{properties-of-tensor-products-of-presheaves-with-copresheaves-interaction-with-representable-presheaves}: Interaction With Representable Presheaves}%
    This follows from \cref{TODO}.

    \ProofBox{\cref{properties-of-tensor-products-of-presheaves-with-copresheaves-interaction-with-corepresentable-copresheaves}: Interaction With Corepresentable Copresheaves}%
    This follows from \cref{TODO}.

    \ProofBox{\cref{properties-of-tensor-products-of-presheaves-with-copresheaves-interaction-with-yoneda-extensions}: Interaction With Yoneda Extensions}%
    This is a special case of \cref{properties-of-yoneda-extensions-interaction-with-tensors-of-presheaves-with-functors} of \cref{properties-of-yoneda-extensions}.

    \ProofBox{\cref{properties-of-tensor-products-of-presheaves-with-copresheaves-interaction-with-contravariant-yoneda-extensions}: Interaction With Contravariant Yoneda Extensions}%
    This is a special case of \cref{properties-of-contravariant-yoneda-extensions-interaction-with-tensors-of-copresheaves-with-functors} of \cref{properties-of-contravariant-yoneda-extensions}.
\end{Proof}
\subsection{The Tensor of a Presheaf With a Functor}\label{subsection-the-tensor-of-a-presheaf-with-a-functor}
Let $\CatFont{C}$ be a category, let $\CatFont{D}$ be a category with coproducts, let $\SheafFont{F}\colon\CatFont{C}^{\op}\to\Sets$ be a presheaf on $\CatFont{C}$, and let $G\colon\CatFont{C}\to\SheafFont{D}$ be a functor.
\begin{definition}{The Tensor of a Presheaf With a Functor}{the-tensor-of-a-presheaf-with-a-functor}%
    The \index[categories]{tensor of a presheaf with a functor}\textbf{tensor} of $\SheafFont{F}$ with $G$ is the object \index[notation]{FodotCG@$\SheafFont{F}\odot_{\CatFont{C}}G$}$\SheafFont{F}\odot_{\CatFont{C}}G$%
    %--- Begin Footnote ---%
    \footnote{%
        \SloganFont{Further Notation: }Also written simply \index[notation]{FodotG@$\SheafFont{F}\odot G$}$\SheafFont{F}\odot G$.
        \par\vspace*{-1.75\baselineskip}
    } %
    %---  End Footnote  ---%
    of $\CatFont{D}$ defined by
    \[
        \SheafFont{F}\odot_{\CatFont{C}}G%
        \defeq%
        \int^{A\in\CatFont{C}}\SheafFont{F}(A)\odot G(A).%
    \]%
\end{definition}
\begin{remark}{Unwinding \cref{the-tensor-of-a-presheaf-with-a-functor}}{unwinding-the-tensor-of-a-presheaf-with-a-functor}%
    In other words, the tensor of $\SheafFont{F}$ with $G$ is the object $\SheafFont{F}\odot_{\CatFont{C}}G$ of $\CatFont{D}$ defined as the coend of the functor
    \[
        \CatFont{C}^{\op}\times\CatFont{C}%
        \xrightarrow{\SheafFont{F}\times G}%
        \Sets\times\CatFont{D}%
        \xrightarrow{\odot}%
        \CatFont{D}.%
    \]%
\end{remark}
\begin{proposition}{Properties of Tensors of Presheaves With Functors}{properties-of-tensors-of-presheaves-with-functors}%
    Let $\CatFont{C}$ be a category.
    \begin{enumerate}
        \item\label{properties-of-tensors-of-presheaves-with-functors-functoriality}\SloganFont{Functoriality. }The assignments $\SheafFont{F},G,(\SheafFont{F},G)\mapsto\SheafFont{F}\odot_{\CatFont{C}}G$ define functors
            \[
                \BifunctorialityPeriod{\SheafFont{F}\odot_{\CatFont{C}}-}{-\odot_{\CatFont{C}}G}{-_{1}\odot_{\CatFont{C}}-_{2}}{\PSh(\CatFont{C})}{\Fun(\CatFont{C},\CatFont{D})}{\PSh(\CatFont{C})\times\Fun(\CatFont{C},\CatFont{D})}{\CatFont{D}}%
            \]%
        \item\label{properties-of-tensors-of-presheaves-with-functors-}\SloganFont{Interaction With Corepresentable Copresheaves. }We have an isomorphism
            \[
                h_{X}\odot_{\CatFont{C}}G%
                \cong
                G(X),%
            \]%
            natural in $X\in\Obj(\CatFont{C})$, giving a natural isomorphism of functors
            \[
                h_{(-)}\odot_{\CatFont{C}}G%
                \cong
                G.%
            \]%
        \item\label{properties-of-tensors-of-presheaves-with-functors-interaction-with-yoneda-extensions}\SloganFont{Interaction With Yoneda Extensions. }We have a natural isomorphism
            \[
                \Lan_{\yo}(G)%
                \cong
                (-)\odot_{\CatFont{C}}G,%
            \]%
            natural in $G\in\Obj(\Fun(\CatFont{C},\CatFont{D}))$.
        %\item\label{properties-of-tensors-of-presheaves-with-functors-}\SloganFont{. }
    \end{enumerate}
\end{proposition}
\begin{Proof}{Proof of \cref{properties-of-tensors-of-presheaves-with-functors}}%
    \FirstProofBox{\cref{properties-of-tensors-of-presheaves-with-functors-functoriality}: Functoriality}%
    Omitted.

    \ProofBox{\cref{properties-of-tensors-of-presheaves-with-functors-}: Interaction With Corepresentable Copresheaves}%
    This follows from \cref{TODO}.

    \ProofBox{\cref{properties-of-tensors-of-presheaves-with-functors-interaction-with-yoneda-extensions}: Interaction With Yoneda Extensions}%
    This is a repetition of \cref{properties-of-yoneda-extensions-interaction-with-tensors-of-presheaves-with-functors} of \cref{properties-of-yoneda-extensions}, and is proved there.
\end{Proof}
\subsection{The Tensor of a Copresheaf With a Functor}\label{subsection-the-tensor-of-a-copresheaf-with-a-functor}
Let $\CatFont{C}$ be a category, let $\CatFont{D}$ be a category with coproducts, let $F\colon\CatFont{C}\to\Sets$ be a copresheaf on $\CatFont{C}$, and let $G\colon\CatFont{C}^{\op}\to\SheafFont{D}$ be a functor.
\begin{definition}{The Tensor of a Copresheaf With a Functor}{the-tensor-of-a-copresheaf-with-a-functor}%
    The \index[categories]{tensor of a copresheaf with a functor}\textbf{tensor} of $F$ with $G$ is the set \index[notation]{FodotCG@$F\odot_{\CatFont{C}}G$}$F\odot_{\CatFont{C}}G$%
    %--- Begin Footnote ---%
    \footnote{%
        \SloganFont{Further Notation: }Also written simply \index[notation]{FodotG@$F\odot G$}$F\odot G$.
        \par\vspace*{-1.75\baselineskip}
    } %
    %---  End Footnote  ---%
    defined by
    \[
        F\odot_{\CatFont{C}}G%
        \defeq%
        \int^{A\in\CatFont{C}}F(A)\odot G(A).%
    \]%
\end{definition}
\begin{remark}{Unwinding \cref{the-tensor-of-a-copresheaf-with-a-functor}}{unwinding-the-tensor-of-a-copresheaf-with-a-functor}%
    In other words, the tensor of $F$ with $G$ is the object $F\odot_{\CatFont{C}}G$ of $\CatFont{D}$ defined as the coend of the functor
    \[
        \CatFont{C}^{\op}\times\CatFont{C}%
        \rightisoarrow%
        \CatFont{C}\times\CatFont{C}^{\op}%
        \xrightarrow{F\times G}%
        \Sets\times\CatFont{D}%
        \xrightarrow{\odot}%
        \CatFont{D}.%
    \]%
\end{remark}
\begin{proposition}{Properties of Tensors of Copresheaves With Functors}{properties-of-tensors-of-copresheaves-with-functors}%
    Let $\CatFont{C}$ be a category.
    \begin{enumerate}
        \item\label{properties-of-tensors-of-copresheaves-with-functors-functoriality}\SloganFont{Functoriality. }The assignments $F,G,(F,G)\mapsto F\odot_{\CatFont{C}}G$ define functors
            \[
                \BifunctorialityPeriod{F\odot_{\CatFont{C}}-}{-\odot_{\CatFont{C}}\SheafFont{G}}{-_{1}\odot_{\CatFont{C}}-_{2}}{\CoPSh(\CatFont{C})}{\Fun(\CatFont{C}^{\op},\CatFont{D})}{\Fun(\CatFont{C}^{\op},\CatFont{D})\times\CoPSh(\CatFont{C})}{\CatFont{D}}%
            \]%
        \item\label{properties-of-tensors-of-copresheaves-with-functors-}\SloganFont{Interaction With Corepresentable Copresheaves. }We have an isomorphism
            \[
                h^{X}\odot_{\CatFont{C}}G%
                \cong
                G(X),%
            \]%
            natural in $X\in\Obj(\CatFont{C})$, giving a natural isomorphism of functors
            \[
                h^{(-)}\odot_{\CatFont{C}}G%
                \cong
                G.%
            \]%
        \item\label{properties-of-tensors-of-copresheaves-with-functors-interaction-with-contravariant-yoneda-extensions}\SloganFont{Interaction With Contravariant Yoneda Extensions. }We have a natural isomorphism
            \[
                \Lan_{\coyo}(G)%
                \cong
                G\odot_{\CatFont{C}}(-),%
            \]%
            natural in $G\in\Obj(\Fun(\CatFont{C}^{\op},\CatFont{D}))$.
        %\item\label{properties-of-tensors-of-copresheaves-with-functors-}\SloganFont{. }
    \end{enumerate}
\end{proposition}
\begin{Proof}{Proof of \cref{properties-of-tensors-of-copresheaves-with-functors}}%
    \FirstProofBox{\cref{properties-of-tensors-of-copresheaves-with-functors-functoriality}: Functoriality}%
    Omitted.

    \ProofBox{\cref{properties-of-tensors-of-copresheaves-with-functors-}: Interaction With Representable Presheaves}%
    This follows from \cref{TODO}.

    \ProofBox{\cref{properties-of-tensors-of-copresheaves-with-functors-}: Interaction With Corepresentable Copresheaves}%
    This follows from \cref{TODO}.

    \ProofBox{\cref{properties-of-tensors-of-copresheaves-with-functors-interaction-with-yoneda-extensions}: Interaction With Yoneda Extensions}%
    Omitted.

    \ProofBox{\cref{properties-of-tensors-of-copresheaves-with-functors-interaction-with-contravariant-yoneda-extensions}: Interaction With Contravariant Yoneda Extensions}%
    Omitted.
\end{Proof}
\begin{appendices}
\begin{multicols}{2}[\section{Other Chapters}]
\noindent
\textbf{Preliminaries}
\begin{enumerate}
\item \hyperref[introduction:section-phantom]{Introduction}
\end{enumerate}
\textbf{Sets}
\begin{enumerate}
\setcounter{enumi}{2}
\item \hyperref[sets:section-phantom]{Sets}
\item \hyperref[constructions-with-sets:section-phantom]{Constructions With Sets}
\item \hyperref[monoidal-structures-on-the-category-of-sets:section-phantom]{Monoidal Structures on the Category of Sets}
\item \hyperref[pointed-sets:section-phantom]{Pointed Sets}
\item \hyperref[tensor-products-of-pointed-sets:section-phantom]{Tensor Products of Pointed Sets}
\end{enumerate}
\textbf{Relations}
\begin{enumerate}
\setcounter{enumi}{6}
\item \hyperref[relations:section-phantom]{Relations}
\item \hyperref[constructions-with-relations:section-phantom]{Constructions With Relations}
\item \hyperref[conditions-on-relations:section-phantom]{Conditions on Relations}
\end{enumerate}
\textbf{Category Theory}
\begin{enumerate}
\setcounter{enumi}{9}
\item \hyperref[categories:section-phantom]{Categories}
\end{enumerate}
\textbf{Monoidal Categories}
\begin{enumerate}
\setcounter{enumi}{10}
\item \hyperref[constructions-with-monoidal-categories:section-phantom]{Constructions With Monoidal Categories}
\end{enumerate}
\textbf{Bicategories}
\begin{enumerate}
\setcounter{enumi}{11}
\item \hyperref[types-of-morphisms-in-bicategories:section-phantom]{Types of Morphisms in Bicategories}
\end{enumerate}
\textbf{Extra Part}
\begin{enumerate}
\setcounter{enumi}{12}
\item \hyperref[notes:section-phantom]{Notes}
\end{enumerate}
\end{multicols}

\end{appendices}
\end{document}
