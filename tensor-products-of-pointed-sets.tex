\input{preamble}

% OK, start here.
%
\usepackage{fontspec}
\let\hyperwhite\relax
\let\hyperred\relax
\newcommand{\hyperwhite}{\hypersetup{citecolor=white,filecolor=white,linkcolor=white,urlcolor=white}}
\newcommand{\hyperred}{%
\hypersetup{%
    citecolor=TitlingRed,%
    filecolor=TitlingRed,%
    linkcolor=TitlingRed,%
     urlcolor=TitlingRed%
}}
\let\ChapterRef\relax
\newcommand{\ChapterRef}[2]{#1}
\setcounter{tocdepth}{2}
%▓▓▓▓▓▓▓▓▓▓▓▓▓▓▓▓▓▓▓▓▓▓▓▓▓▓▓▓▓▓▓▓▓
%▓▓ ╔╦╗╦╔╦╗╦  ╔═╗  ╔═╗╔═╗╔╗╔╔╦╗ ▓▓
%▓▓  ║ ║ ║ ║  ║╣   ╠╣ ║ ║║║║ ║  ▓▓
%▓▓  ╩ ╩ ╩ ╩═╝╚═╝  ╚  ╚═╝╝╚╝ ╩  ▓▓
%▓▓▓▓▓▓▓▓▓▓▓▓▓▓▓▓▓▓▓▓▓▓▓▓▓▓▓▓▓▓▓▓▓
%\usepackage{titlesec}
%▓▓▓▓▓▓▓▓▓▓▓▓▓▓▓▓▓▓▓▓▓▓▓▓▓▓▓▓▓▓▓▓▓▓▓▓▓▓▓▓▓▓▓▓▓▓▓▓▓▓▓▓▓▓▓
%▓▓ ╔╦╗╔═╗╔╗ ╦  ╔═╗  ╔═╗╔═╗  ╔═╗╔═╗╔╗╔╔╦╗╔═╗╔╗╔╔╦╗╔═╗ ▓▓
%▓▓  ║ ╠═╣╠╩╗║  ║╣   ║ ║╠╣   ║  ║ ║║║║ ║ ║╣ ║║║ ║ ╚═╗ ▓▓
%▓▓  ╩ ╩ ╩╚═╝╩═╝╚═╝  ╚═╝╚    ╚═╝╚═╝╝╚╝ ╩ ╚═╝╝╚╝ ╩ ╚═╝ ▓▓
%▓▓▓▓▓▓▓▓▓▓▓▓▓▓▓▓▓▓▓▓▓▓▓▓▓▓▓▓▓▓▓▓▓▓▓▓▓▓▓▓▓▓▓▓▓▓▓▓▓▓▓▓▓▓▓
\newcommand{\ChapterTableOfContents}{%
    \begingroup
    \addfontfeature{Numbers={Lining,Monospaced}}
    \hypersetup{hidelinks}\tableofcontents%
    \endgroup
}%

\makeatletter
\newcommand \DotFill {\leavevmode \cleaders \hb@xt@ .33em{\hss .\hss }\hfill \kern \z@}
\makeatother

\definecolor{ToCGrey}{rgb}{0.4,0.4,0.4}
\definecolor{mainColor}{rgb}{0.82745098,0.18431373,0.18431373}
\usepackage{titletoc}
\titlecontents{part}
[0.0em]
{\addvspace{1pc}\color{TitlingRed}\large\bfseries\text{Part }}
{\bfseries\textcolor{TitlingRed}{\contentslabel{0.0em}}\hspace*{1.35em}}
{}
{\textcolor{TitlingRed}{{\hfill\bfseries\contentspage\nobreak}}}
[]
\titlecontents{section}
[0.0em]
{\addvspace{1pc}}
{\color{black}\bfseries\textcolor{TitlingRed}{\contentslabel{0.0em}}\hspace*{1.35em}}
{}
{\textcolor{black}{\textbf{\DotFill}{\bfseries\contentspage\nobreak}}}
[]
\titlecontents{subsection}
[0.0em]
{}
{\hspace*{1.35em}\color{ToCGrey}{\contentslabel{0.0em}}\hspace*{2.1em}}
{}
{{\textcolor{ToCGrey}\DotFill}\textcolor{ToCGrey}{\contentspage}\nobreak}
[]
\usepackage{marginnote}
\renewcommand*{\marginfont}{\normalfont}
\usepackage{inconsolata}
\setmonofont{inconsolata}%
\let\ChapterRef\relax
\newcommand{\ChapterRef}[2]{#1}
\AtBeginEnvironment{subappendices}{%%
    \section*{\huge Appendices}%
}%

\begin{document}

\title{Tensor Products of Pointed Sets}

\maketitle

\phantomsection
\label{section-phantom}

In this chapter we introduce, construct, and study tensor products of pointed sets. The most well-known among these is the \emph{smash product of pointed sets}
\[
    \wedge%
    \colon%
    \Sets_{*}\times\Sets_{*}%
    \to%
    \Sets_{*},
\]%
introduced in \cref{subsection-smash-products-of-pointed-sets-foundations}, defined via a universal property as inducing a bijection between the following data:
\begin{itemize}
    \item Pointed maps $f\colon X\wedge Y\to Z$.
    \item Maps of sets $f\colon X\times Y\to Z$ satisfying
        \begin{align*}
            f(x_{0},y) &= z_{0},\\
            f(x,y_{0}) &= z_{0}
        \end{align*}
        for each $x\in X$ and each $y\in Y$.
\end{itemize}
As it turns out, however, dropping either of the \emph{bilinearity} conditions
\begin{align*}
    f(x_{0},y) &= z_{0},\\
    f(x,y_{0}) &= z_{0}
\end{align*}
while retaining the other leads to two other tensor products of pointed sets,
\begin{align*}
    \lhd &\colon \Sets_{*}\times\Sets_{*} \to \Sets_{*},\\
    \rhd &\colon \Sets_{*}\times\Sets_{*} \to \Sets_{*},
\end{align*}
called the \emph{left} and \emph{right tensor products of pointed sets}. In contrast to $\wedge$, which turns out to endow $\Sets_{*}$ with a monoidal category structure (\cref{the-monoidal-structure-on-pointed-sets-associated-to-the-smash-product-of-pointed-sets}), these do not admit invertible associators and unitors, but do endow $\Sets_{*}$ with the structure of a skew monoidal category, however (\cref{the-left-skew-monoidal-structure-on-pointed-sets-associated-to-the-left-tensor-product-of-pointed-sets,the-right-skew-monoidal-structure-on-pointed-sets-associated-to-the-right-tensor-product-of-pointed-sets}).

Finally, in addition to the tensor products $\lhd$, $\rhd$, and $\wedge$, we also have a \say{tensor product} of the form
\[
    \odot%
    \colon%
    \Sets\times\Sets_{*}%
    \to%
    \Sets_{*},%
\]%
called the \emph{tensor} of sets with pointed sets. All in all, these tensor products assemble into a family of functors of the form
\begin{align*}
    \otimes_{k,\ell} &\colon \Mon_{\E_{k}}(\Sets)\times\Mon_{\E_{\ell}}(\Sets) \to \Mon_{\E_{k+\ell}}(\Sets),\\%
    \lhd_{i,k}       &\colon \Mon_{\E_{k}}(\Sets)\times\Mon_{\E_{k}}(\Sets)    \to \Mon_{\E_{k}}(\Sets),\\%
    \rhd_{i,k}       &\colon \Mon_{\E_{k}}(\Sets)\times\Mon_{\E_{k}}(\Sets)    \to \Mon_{\E_{k}}(\Sets),%
\end{align*}
where $k,\ell,i\in\N$ with $i\leq k-1$. Together with the Cartesian product $\times$ of $\Sets$, the tensor products studied in this chapter form the cases:
\begin{itemize}
    \item $(k,\ell)=(-1,-1)$ for the Cartesian product of $\Sets$;
    \item $(k,\ell)=(0,-1)$ and $(-1,0)$ for the tensor of sets with pointed sets of \cref{tensors-of-pointed-sets-by-sets};
    \item $(i,k)=(-1,0)$ for the left and right tensor products of pointed sets of \cref{section-the-left-tensor-product-of-pointed-sets,section-the-right-tensor-product-of-pointed-sets};
    \item $(k,\ell)=(-1,-1)$ for the smash product of pointed sets of \cref{section-the-smash-product-of-pointed-sets}.
\end{itemize}
In this chapter, we will carefully define and study bilinearity for pointed sets, as well as all the tensor products described above. Then, in \ChapterTensorProductsOfMonoids, we will extend these to tensor products involving also monoids and commutative monoids, which will end up covering all cases up to $k,\ell\leq2$, and hence \emph{all} cases since $\E_{k}$-monoids on $\Sets$ are the same as $\E_{2}$-monoids on $\Sets$ when $k\geq2$.

\ChapterTableOfContents

\section{Bilinear Morphisms of Pointed Sets}\label{section-bilinear-morphisms-of-pointed-sets}
\subsection{Left Bilinear Morphisms of Pointed Sets}\label{subsection-left-bilinear-morphisms-of-pointed-sets}
Let $(X,x_{0})$, $(Y,y_{0})$, and $(Z,z_{0})$ be pointed sets.
\begin{definition}{Left Bilinear Morphisms of pointed sets}{left-bilinear-morphisms-of-pointed-sets}%
    A \index[set-theory]{bilinear morphism of pointed sets!left}\textbf{left bilinear morphism of pointed sets from $(X\times Y,(x_{0},y_{0}))$ to $(Z,z_{0})$} is a map of sets
    \[
        f
        \colon
        X\times Y
        \to
        Z
    \]%
    satisfying the following condition:%
    %--- Begin Footnote ---%
    \footnote{%
        \SloganFont{Slogan: }The map $f$ is left bilinear if it preserves basepoints in its first argument.
    }%
    %---  End Footnote  ---%
    %--- Begin Footnote ---%
    \footnote{%
        Succinctly, $f$ is bilinear if we have
        \[
            f(x_{0},y)
            =
            z_{0}
        \]%
        for each $y\in Y$.
        \par\vspace*{\TCBBoxCorrection}
    }%
    %---  End Footnote  ---%
    \begin{itemize}
        \itemstar\SloganFont{Left Unital Bilinearity. }The diagram
            \[
                \begin{tikzcd}[row sep={0*\the\DL,between origins}, column sep={0*\the\DL,between origins}, background color=backgroundColor, ampersand replacement=\&]
                    \&[0.30901699437\TwoCm]
                    \&[0.5\TwoCm]
                    \pt\times\pt
                    \&[0.5\TwoCm]
                    \&[0.30901699437\TwoCm]
                    \\[0.58778525229\TwoCm]
                    \pt\times Y
                    \&[0.30901699437\TwoCm]
                    \&[0.5\TwoCm]
                    \&[0.5\TwoCm]
                    \&[0.30901699437\TwoCm]
                    \pt
                    \\[0.95105651629\TwoCm]
                    \&[0.30901699437\TwoCm]
                    X\times Y
                    \&[0.5\TwoCm]
                    \&[0.5\TwoCm]
                    Z
                    \&[0.30901699437\TwoCm]
                    % 1-Arrows
                    % Left Boundary
                    \arrow[from=2-1,to=1-3,"\id_{\pt}\times\epsilon_{Y}",pos=0.45]%
                    \arrow[from=1-3,to=2-5,bigisoarrow]%
                    \arrow[from=2-5,to=3-4,"{[z_{0}]}",pos=0.375]%
                    % Right Boundary
                    \arrow[from=2-1,to=3-2,"{[x_{0}]\times\id_{Y}}"',pos=0.375]%
                    \arrow[from=3-2,to=3-4,"f"',pos=0.45]%
                \end{tikzcd}
            \]%
            commutes, i.e.\ for each $y\in Y$, we have
            \[
                f(x_{0},y)
                =
                z_{0}.
            \]%
    \end{itemize}
\end{definition}
\begin{definition}{The Set of Left Bilinear Morphisms of Pointed Sets}{the-set-of-left-bilinear-morphisms-of-pointed-sets}%
    The \index[set-theory]{set of bilinear morphisms of pointed sets!left}\textbf{set of left bilinear morphisms of pointed sets from $(X\times Y,(x_{0},y_{0}))$ to $(Z,z_{0})$} is the set \index[notation]{HomotimesLSetsstarXtimesYZ@$\Hom^{\otimes,\rmL}_{\Sets_{*}}(X\times Y,Z)$}$\smash{\Hom^{\otimes,\rmL}_{\Sets_{*}}(X\times Y,Z)}$ defined by
    \[
        \Hom^{\otimes,\rmL}_{\Sets_{*}}(X\times Y,Z)
        \defeq
        \{%
            f\in\Hom_{\Sets}(X\times Y,Z)%
            \ \middle|\ %
            \text{$f$ is left bilinear}%
        \}.%
    \]%
\end{definition}
\subsection{Right Bilinear Morphisms of Pointed Sets}\label{subsection-right-bilinear-morphisms-of-pointed-sets}
Let $(X,x_{0})$, $(Y,y_{0})$, and $(Z,z_{0})$ be pointed sets.
\begin{definition}{Right Bilinear Morphisms of pointed sets}{right-bilinear-morphisms-of-pointed-sets}%
    A \index[set-theory]{bilinear morphism of pointed sets!right}\textbf{right bilinear morphism of pointed sets from $(X\times Y,(x_{0},y_{0}))$ to $(Z,z_{0})$} is a map of sets
    \[
        f
        \colon
        X\times Y
        \to
        Z
    \]%
    satisfying the following condition:%
    %--- Begin Footnote ---%
    \footnote{%
        \SloganFont{Slogan: }The map $f$ is right bilinear if it preserves basepoints in its second argument.
    }%
    %---  End Footnote  ---%
    %--- Begin Footnote ---%
    \footnote{%
        Succinctly, $f$ is bilinear if we have
        \[
            f(x,y_{0})
            =
            z_{0}
        \]%
        for each $x\in X$.
        \par\vspace*{\TCBBoxCorrection}
    }%
    %---  End Footnote  ---%
    \begin{itemize}
        \itemstar\SloganFont{Right Unital Bilinearity. }The diagram
            \[
                \begin{tikzcd}[row sep={0*\the\DL,between origins}, column sep={0*\the\DL,between origins}, background color=backgroundColor, ampersand replacement=\&]
                    \&[0.30901699437\TwoCm]
                    \&[0.5\TwoCm]
                    \pt\times\pt
                    \&[0.5\TwoCm]
                    \&[0.30901699437\TwoCm]
                    \\[0.58778525229\TwoCm]
                    X\times\pt
                    \&[0.30901699437\TwoCm]
                    \&[0.5\TwoCm]
                    \&[0.5\TwoCm]
                    \&[0.30901699437\TwoCm]
                    \pt
                    \\[0.95105651629\TwoCm]
                    \&[0.30901699437\TwoCm]
                    X\times Y
                    \&[0.5\TwoCm]
                    \&[0.5\TwoCm]
                    Z
                    \&[0.30901699437\TwoCm]
                    % 1-Arrows
                    % Left Boundary
                    \arrow[from=2-1,to=1-3,"\epsilon_{X}\times\id_{\pt}",pos=0.45]%
                    \arrow[from=1-3,to=2-5,bigisoarrow]%
                    \arrow[from=2-5,to=3-4,"{[z_{0}]}",pos=0.375]%
                    % Right Boundary
                    \arrow[from=2-1,to=3-2,"{\id_{X}\times[y_{0}]}"',pos=0.375]%
                    \arrow[from=3-2,to=3-4,"f"',pos=0.45]%
                \end{tikzcd}
            \]%
            commutes, i.e.\ for each $x\in X$, we have
            \[
                f(x,y_{0})
                =
                z_{0}.
            \]%
    \end{itemize}
\end{definition}
\begin{definition}{The Set of Right Bilinear Morphisms of Pointed Sets}{the-set-of-right-bilinear-morphisms-of-pointed-sets}%
    The \index[set-theory]{set of bilinear morphisms of pointed sets!right}\textbf{set of right bilinear morphisms of pointed sets from $(X\times Y,(x_{0},y_{0}))$ to $(Z,z_{0})$} is the set \index[notation]{HomotimesRSetsstarXtimesYZ@$\Hom^{\otimes,\rmR}_{\Sets_{*}}(X\times Y,Z)$}$\smash{\Hom^{\otimes,\rmR}_{\Sets_{*}}(X\times Y,Z)}$ defined by
    \[
        \Hom^{\otimes,\rmR}_{\Sets_{*}}(X\times Y,Z)
        \defeq
        \{%
            f\in\Hom_{\Sets}(X\times Y,Z)%
            \ \middle|\ %
            \text{$f$ is right bilinear}%
        \}.%
    \]%
\end{definition}
\subsection{Bilinear Morphisms of Pointed Sets}\label{subsection-bilinear-morphisms-of-pointed-sets}
Let $(X,x_{0})$, $(Y,y_{0})$, and $(Z,z_{0})$ be pointed sets.
\begin{definition}{Bilinear Morphisms of pointed sets}{bilinear-morphisms-of-pointed-sets}%
    A \index[set-theory]{bilinear morphism!of pointed sets}\textbf{bilinear morphism of pointed sets from $(X\times Y,(x_{0},y_{0}))$ to $(Z,z_{0})$} is a map of sets
    \[
        f
        \colon
        X\times Y
        \to
        Z
    \]%
    that is both left bilinear and right bilinear.%
\end{definition}
\begin{remark}{Unwinding \cref{bilinear-morphisms-of-pointed-sets}}{unwinding-bilinear-morphisms-of-pointed-sets}%
    In detail, a \textbf{bilinear morphism of pointed sets from $(X\times Y,(x_{0},y_{0}))$ to $(Z,z_{0})$} is a map of sets
    \[
        f
        \colon
        (X\times Y,(x_{0},y_{0}))
        \to
        (Z,z_{0})
    \]%
    satisfying the following conditions:%
    %--- Begin Footnote ---%
    \footnote{%
        \SloganFont{Slogan: }The map $f$ is bilinear if it preserves basepoints in each argument.
    }%
    %---  End Footnote  ---%
    %--- Begin Footnote ---%
    \footnote{%
        Succinctly, $f$ is bilinear if we have
        \begin{align*}
            f(x_{0},y) &= z_{0},\\
            f(x,y_{0}) &= z_{0}
        \end{align*}
        for each $x\in X$ and each $y\in Y$.
        \par\vspace*{\TCBBoxCorrection}
    }%
    %---  End Footnote  ---%
    \begin{enumerate}
        \item\label{unwinding-bilinear-morphisms-of-pointed-sets-left-unital-bilinearity}\SloganFont{Left Unital Bilinearity. }The diagram
            \[
                \begin{tikzcd}[row sep={0*\the\DL,between origins}, column sep={0*\the\DL,between origins}, background color=backgroundColor, ampersand replacement=\&]
                    \&[0.30901699437\TwoCm]
                    \&[0.5\TwoCm]
                    \pt\times\pt
                    \&[0.5\TwoCm]
                    \&[0.30901699437\TwoCm]
                    \\[0.58778525229\TwoCm]
                    \pt\times Y
                    \&[0.30901699437\TwoCm]
                    \&[0.5\TwoCm]
                    \&[0.5\TwoCm]
                    \&[0.30901699437\TwoCm]
                    \pt
                    \\[0.95105651629\TwoCm]
                    \&[0.30901699437\TwoCm]
                    X\times Y
                    \&[0.5\TwoCm]
                    \&[0.5\TwoCm]
                    Z
                    \&[0.30901699437\TwoCm]
                    % 1-Arrows
                    % Left Boundary
                    \arrow[from=2-1,to=1-3,"\id_{\pt}\times\epsilon_{Y}",pos=0.45]%
                    \arrow[from=1-3,to=2-5,bigisoarrow]%
                    \arrow[from=2-5,to=3-4,"{[z_{0}]}",pos=0.375]%
                    % Right Boundary
                    \arrow[from=2-1,to=3-2,"{[x_{0}]\times\id_{Y}}"',pos=0.375]%
                    \arrow[from=3-2,to=3-4,"f"',pos=0.45]%
                \end{tikzcd}
            \]%
            commutes, i.e.\ for each $y\in Y$, we have
            \[
                f(x_{0},y)
                =
                z_{0}.
            \]%
        \item\label{unwinding-bilinear-morphisms-of-pointed-sets-right-unital-bilinearity}\SloganFont{Right Unital Bilinearity. }The diagram
            \[
                \begin{tikzcd}[row sep={0*\the\DL,between origins}, column sep={0*\the\DL,between origins}, background color=backgroundColor, ampersand replacement=\&]
                    \&[0.30901699437\TwoCm]
                    \&[0.5\TwoCm]
                    \pt\times\pt
                    \&[0.5\TwoCm]
                    \&[0.30901699437\TwoCm]
                    \\[0.58778525229\TwoCm]
                    X\times\pt
                    \&[0.30901699437\TwoCm]
                    \&[0.5\TwoCm]
                    \&[0.5\TwoCm]
                    \&[0.30901699437\TwoCm]
                    \pt
                    \\[0.95105651629\TwoCm]
                    \&[0.30901699437\TwoCm]
                    X\times Y
                    \&[0.5\TwoCm]
                    \&[0.5\TwoCm]
                    Z
                    \&[0.30901699437\TwoCm]
                    % 1-Arrows
                    % Left Boundary
                    \arrow[from=2-1,to=1-3,"\epsilon_{X}\times\id_{\pt}",pos=0.45]%
                    \arrow[from=1-3,to=2-5,bigisoarrow]%
                    \arrow[from=2-5,to=3-4,"{[z_{0}]}",pos=0.375]%
                    % Right Boundary
                    \arrow[from=2-1,to=3-2,"{\id_{X}\times[y_{0}]}"',pos=0.375]%
                    \arrow[from=3-2,to=3-4,"f"',pos=0.45]%
                \end{tikzcd}
            \]%
            commutes, i.e.\ for each $x\in X$, we have
            \[
                f(x,y_{0})
                =
                z_{0}.
            \]%
    \end{enumerate}
\end{remark}
\begin{definition}{The Set of Bilinear Morphisms of Pointed Sets}{the-set-of-bilinear-morphisms-of-pointed-sets}%
    The \index[set-theory]{set of bilinear morphisms of pointed sets}\textbf{set of bilinear morphisms of pointed sets from $(X\times Y,(x_{0},y_{0}))$ to $(Z,z_{0})$} is the set \index[notation]{HomotimesSetsstarXtimesYZ@$\Hom^{\otimes}_{\Sets_{*}}(X\times Y,Z)$}$\smash{\Hom^{\otimes}_{\Sets_{*}}(X\times Y,Z)}$ defined by
    \[
        \Hom^{\otimes}_{\Sets_{*}}(X\times Y,Z)
        \defeq
        \{%
            f\in\Hom_{\Sets}(X\times Y,Z)%
            \ \middle|\ %
            \text{$f$ is bilinear}%
        \}.%
    \]%
\end{definition}
\section{Tensors and Cotensors of Pointed Sets by Sets}\label{section-tensor-and-cotensors-of-pointed-sets-by-sets}
\subsection{Tensors of Pointed Sets by Sets}\label{subsection-tensors-by-sets}
Let $(X,x_{0})$ be a pointed set and let $A$ be a set.%
\begin{definition}{Tensors of Pointed Sets by Sets}{tensors-of-pointed-sets-by-sets}%
    The \index[set-theory]{pointed set!tensor by a set}\textbf{tensor of $(X,x_{0})$ by $A$}%
    %--- Begin Footnote ---%
    \footnote{%
        \SloganFont{Further Terminology: }Also called the \index[set-theory]{pointed set!copower by a set|see {pointed set, tensor by a set}}\textbf{copower of $(X,x_{0})$ by $A$}.%
    } %
    %---  End Footnote  ---%
    is the tensor \index[notation]{AodotXxzero@$A\odot(X,x_{0})$}$A\odot(X,x_{0})$%
    %--- Begin Footnote ---%
    \footnote{%
        \SloganFont{Further Notation: }Often written \index[notation]{AodotX@$A\odot X$}$A\odot X$ for simplicity.
        \par\vspace*{\TCBBoxCorrection}
    } %
    %---  End Footnote  ---%
    of $(X,x_{0})$ by $A$ as in \ChapterRef{\ChapterLimitsAndColimits, \cref{weighted-limits-and-colimits:tensors}}{\cref{tensors}}.
\end{definition}
\begin{remark}{Unwinding \cref{tensors-of-pointed-sets-by-sets}}{unwinding-tensors-of-pointed-sets-by-sets}%
    In detail, the \textbf{tensor of $(X,x_{0})$ by $A$} is the pointed set \index[notation]{AodotXxzero@$A\odot(X,x_{0})$}$A\odot(X,x_{0})$ satisfying the following universal property:
    \begin{itemize}
        \itemstar We have a bijection
        \[%
             \Sets_{*}(A\odot X,K)%
             \cong%
             \Sets(A,\Sets_{*}(X,K)),%
        \]%
        natural in $(K,k_{0})\in\Obj(\Sets_{*})$.
    \end{itemize}
    This universal property is in turn equivalent to the following one:
    \begin{itemize}
        \itemstar We have a bijection
            \[
                \Sets_{*}(A\odot X,K)
                \cong
                \Sets^{\otimes}_{\E_{0}}(A\times X,K),
            \]%
            natural in $(K,k_{0})\in\Obj(\Sets_{*})$, where $\Sets^{\otimes}_{\E_{0}}(A\times X,K)$ is the set defined by
            \[
                \Sets^{\otimes}_{\E_{0}}(A\times X,K)
                \defeq
                \{%
                    f\in\Sets(A\times X,K)%
                    \ \middle|\ %
                    \begin{aligned}
                        &\text{for each $a\in A$, we}\\
                        &\text{have $f(a,x_{0})=k_{0}$}%
                    \end{aligned}
                \}.
            \]%
    \end{itemize}
\end{remark}
\begin{Proof}{Proof of the Equivalence in \cref{unwinding-tensors-of-pointed-sets-by-sets}}%
    We claim that we have a bijection
    \[
        \Sets(A,\Sets_{*}(X,K))%
        \cong%
        \Sets^{\otimes}_{\E_{0}}(A\times X,K)%
    \]%
    natural in $(K,k_{0})\in\Obj(\Sets_{*})$. Indeed, this bijection is a restriction of the bijection
    \[
        \Sets(A,\Sets(X,K))%
        \cong
        \Sets(A\times X,K)%
    \]%
    of \ChapterRef{\ChapterConstructionsWithSets, \cref{constructions-with-sets:properties-of-products-of-sets-adjointness-1} of \cref{constructions-with-sets:properties-of-products-of-sets}}{\cref{properties-of-products-of-sets-adjointness-1} of \cref{properties-of-products-of-sets}}:%
    \begin{itemize}
        \item A map
            \begin{webcompile}
                \phantom{\xi\colon}
                \begin{tikzcd}[row sep=0.0*\the\DL, column sep=1.0*\the\DL, background color=backgroundColor, ampersand replacement=\&]
                    \mathllap{\xi\colon}A%
                    \arrow[r]
                    \&
                    \Sets_{*}(X,K)\mrp{,}%
                    \\
                    a
                    \arrow[r, mapsto]
                    \&
                    {(\xi_{a}\colon X\to K)\mrp{,}}
                \end{tikzcd}
            \end{webcompile}
            in $\Sets(A,\Sets_{*}(X,K))$ gets sent to the map
            \[
                \xi^{\dagger}%
                \colon%
                A\times X%
                \to%
                K%
            \]%
            defined by
            \[
                \xi^{\dagger}(a,x)%
                \defeq%
                \xi_{a}(x)%
            \]%
            for each $(a,x)\in A\times X$, which indeed lies in $\Sets^{\otimes}_{\E_{0}}(A\times X,K)$, as we have%
            \begin{align*}
                \xi^{\dagger}(a,x_{0}) &\defeq \xi_{a}(x_{0})\\%
                                       &\defeq k_{0}%
            \end{align*}
            for each $a\in A$, where we have used that $\xi_{a}\in\Sets_{*}(X,K)$ is a morphism of pointed sets.
        \item Conversely, a map
            \[
                \xi%
                \colon%
                A\times X%
                \to%
                K%
            \]%
            in $\Sets^{\otimes}_{\E_{0}}(A\times X,K)$ gets sent to the map
            \begin{webcompile}
                \phantom{\xi^{\dagger}\colon}
                \begin{tikzcd}[row sep=0.0*\the\DL, column sep=1.0*\the\DL, background color=backgroundColor, ampersand replacement=\&]
                    \mathllap{\xi^{\dagger}\colon}A%
                    \arrow[r]
                    \&
                    \Sets_{*}(X,K)\mrp{,}%
                    \\
                    a
                    \arrow[r, mapsto]
                    \&
                    {(\xi^{\dagger}_{a}\colon X\to K)\mrp{,}}
                \end{tikzcd}
            \end{webcompile}
            where
            \[
                \xi^{\dagger}_{a}%
                \colon%
                X
                \to%
                K%
            \]%
            is the map defined by
            \[
                \xi^{\dagger}_{a}(x)%
                \defeq%
                \xi(a,x)%
            \]%
            for each $x\in X$, and indeed lies in $\Sets_{*}(X,K)$, as we have%
            \begin{align*}
                \xi^{\dagger}_{a}(x_{0}) &\defeq \xi(a,x_{0})\\%
                                         &\defeq k_{0}.%
            \end{align*}
    \end{itemize}
    This finishes the proof.
\end{Proof}
\begin{construction}{Construction of Tensors of Pointed Sets by Sets}{construction-of-tensors-of-pointed-sets-by-sets}%
    Concretely, the \textbf{tensor of $(X,x_{0})$ by $A$} is the pointed set $A\odot(X,x_{0})$ consisting of:
    \begin{itemize}
        \item\SloganFont{The Underlying Set. }The set $A\odot X$ given by
            \[
                A\odot X%
                \cong%
                \bigvee_{a\in A}(X,x_{0}),%
            \]%
            where $\bigvee_{a\in A}(X,x_{0})$ is the wedge product of the $A$-indexed family $((X,x_{0}))_{a\in A}$ of \ChapterRef{\ChapterPointedSets, \cref{pointed-sets:coproducts-of-families-of-pointed-sets}}{\cref{coproducts-of-families-of-pointed-sets}}.
        \item\SloganFont{The Basepoint. }The point $[(a,x_{0})]=[(a',x_{0})]$ of $\bigvee_{a\in A}(X,x_{0})$.%
    \end{itemize}
\end{construction}
\begin{Proof}{Proof of \cref{construction-of-tensors-of-pointed-sets-by-sets}}%
    (Proven below in a bit.)
\end{Proof}
\begin{notation}{Elements of Tensors of Pointed Sets by Sets}{elements-of-tensors-of-pointed-sets-by-sets}%
    We write \index[notation]{aodotx@$a\odot x$}$a\odot x$ for the element $[(a,x)]$ of
    \begin{align*}
        A\odot X &\cong  \bigvee_{a\in A}(X,x_{0})\\%
                 &\defeq (\coprod_{i\in I}X_{i})/\unsim.%
    \end{align*}
\end{notation}
\begin{remark}{Basepoints of Tensors of Pointed Sets by Sets}{basepoints-of-tensors-of-pointed-sets-by-sets}%
    Taking the tensor of any element of $A$ with the basepoint $x_{0}$ of $X$ leads to the same element in $A\odot X$, i.e.\ we have
    \[%
        a\odot x_{0}%
        =%
        a'\odot x_{0},%
    \]%
    for each $a,a'\in A$. This is due to the equivalence relation $\unsim$ on
    \[
        \bigvee_{a\in A}(X,x_{0})%
        \defeq%
        \coprod_{a\in A}X/\unsim%
    \]%
    identifying $(a,x_{0})$ with $(a',x_{0})$, so that the equivalence class $a\odot x_{0}$ is independent from the choice of $a\in A$.
\end{remark}
\begin{Proof}{Proof of \cref{construction-of-tensors-of-pointed-sets-by-sets}}%
    We claim we have a bijection
    \[
        \Sets_{*}(A\odot X,K)%
        \cong%
        \Sets(A,\Sets_{*}(X,K))%
    \]%
    natural in $(K,k_{0})\in\Obj(\Sets_{*})$.%
    \begin{enumerate}
        \item\label{proof-of-construction-of-tensors-of-pointed-sets-by-sets-1}\SloganFont{Map \rmI. }We define a map
            \[
                \Phi_{K}%
                \colon%
                \Sets_{*}(A\odot X,K)
                \to
                \Sets(A,\Sets_{*}(X,K))%
            \]%
            by sending a morphism of pointed sets
            \[
                \xi%
                \colon%
                (A\odot X,a\odot x_{0})%
                \to%
                (K,k_{0})%
            \]%
            to the map of sets%
            \begin{webcompile}
                \phantom{\xi^{\dagger}\colon}
                \begin{tikzcd}[row sep=0.0*\the\DL, column sep=1.0*\the\DL, background color=backgroundColor, ampersand replacement=\&]
                    \mathllap{\xi^{\dagger}\colon}A%
                    \arrow[r]
                    \&
                    \Sets_{*}(X,K)\mrp{,}%
                    \\
                    a
                    \arrow[r, mapsto]
                    \&
                    {(\xi_{a}\colon X\to K)\mrp{,}}
                \end{tikzcd}
            \end{webcompile}
            where
            \[
                \xi_{a}%
                \colon%
                (X,x_{0})%
                \to%
                (K,k_{0})%
            \]%
            is the morphism of pointed sets defined by
            \[
                \xi_{a}(x)%
                \defeq
                \xi(a\odot x)%
            \]%
            for each $x\in X$. Note that we have
            \begin{align*}
                \xi_{a}(x_{0}) &\defeq \xi(a\odot x_{0})\\%
                               &=      k_{0},%
            \end{align*}
            so that $\xi_{a}$ is indeed a morphism of pointed sets, where we have used that $\xi$ is a morphism of pointed sets.
        \item\label{proof-of-construction-of-tensors-of-pointed-sets-by-sets-2}\SloganFont{Map \rmII. }We define a map
            \[
                \Psi_{K}%
                \colon%
                \Sets(A,\Sets_{*}(X,K))%
                \to
                \Sets_{*}(A\odot X,K)
            \]%
            given by sending a map
            \begin{webcompile}
                \phantom{\xi\colon}
                \begin{tikzcd}[row sep=0.0*\the\DL, column sep=1.0*\the\DL, background color=backgroundColor, ampersand replacement=\&]
                    \mathllap{\xi\colon}A%
                    \arrow[r]
                    \&
                    \Sets_{*}(X,K)\mrp{,}%
                    \\
                    a
                    \arrow[r, mapsto]
                    \&
                    {(\xi_{a}\colon X\to K)\mrp{,}}
                \end{tikzcd}
            \end{webcompile}
            to the morphism of pointed sets
            \[
                \xi^{\dagger}%
                \colon%
                (A\odot X,a\odot x_{0})%
                \to%
                (K,k_{0})%
            \]%
            defined by
            \[
                \xi^{\dagger}(a\odot x)%
                \defeq%
                \xi_{a}(x)%
            \]%
            for each $a\odot x\in A\odot X$. Note that $\xi^{\dagger}$ is indeed a morphism of pointed sets, as we have
            \begin{align*}
                \xi^{\dagger}(a\odot x_{0}) &\defeq \xi_{a}(x_{0})\\%
                                           &=      k_{0},
            \end{align*}
            where we have used that $\xi(a)\in\Sets_{*}(X,K)$ is a morphism of pointed sets.
        \item\label{proof-of-construction-of-tensors-of-pointed-sets-by-sets-3}\SloganFont{Invertibility \rmI. }We claim that
            \[
                \Psi_{K}\circ\Phi_{K}%
                =%
                \id_{\Sets_{*}(A\odot X,K)}.%
            \]%
            Indeed, given a morphism of pointed sets
            \[
                \xi%
                \colon%
                (A\odot X,a\odot x_{0})%
                \to%
                (K,k_{0}),%
            \]%
            we have
            \begin{align*}
                [\Psi_{K}\circ\Phi_{K}](\xi) &= \Psi_{K}(\Phi_{K}(\xi))\\%
                                             &= \Psi_{K}(\llbracket a\mapsto\llbracket x\mapsto\xi(a\odot x)\rrbracket\rrbracket)\\%
                                             &= \Psi_{K}(\llbracket a'\mapsto\llbracket x'\mapsto\xi(a'\odot x')\rrbracket\rrbracket)\\%
                                             &= \llbracket a\odot x\mapsto\ev_{x}(\ev_{a}(\llbracket a'\mapsto\llbracket x'\mapsto\xi(a'\odot x')\rrbracket\rrbracket))\rrbracket\\%
                                             &= \llbracket a\odot x\mapsto\ev_{x}(\llbracket x'\mapsto\xi(a\odot x')\rrbracket)\rrbracket\\%
                                             &= \llbracket a\odot x\mapsto\xi(a\odot x)\rrbracket\\%
                                             &= \xi.%
            \end{align*}
        \item\label{proof-of-construction-of-tensors-of-pointed-sets-by-sets-4}\SloganFont{Invertibility \rmII. }We claim that
            \[
                \Phi_{K}\circ\Psi_{K}%
                =%
                \id_{\Sets(A,\Sets_{*}(X,K))}.%
            \]%
            Indeed, given a morphism $\xi\colon A\to\Sets_{*}(X,K)$, we have
            \begin{align*}
                [\Phi_{K}\circ\Psi_{K}](\xi) &= \Phi_{K}(\Psi_{K}(\xi))\\%
                                             &= \Phi_{K}(\llbracket a\odot x\mapsto\xi_{a}(x)\rrbracket)\\%
                                             &= \llbracket a\mapsto\llbracket x\mapsto\xi_{a}(x)\rrbracket\rrbracket\\
                                             &= \llbracket a\mapsto\xi(a)\rrbracket\\
                                             &= \xi.%
            \end{align*}
        \item\label{proof-of-construction-of-tensors-of-pointed-sets-by-sets-5}\SloganFont{Naturality of $\Phi$. }We need to show that, given a morphism of pointed sets
            \[
                \phi%
                \colon%
                (K,k_{0})%
                \to%
                (K',k'_{0}),%
            \]%
            the diagram
            \[
                \begin{tikzcd}[row sep={5.0*\the\DL,between origins}, column sep={12.0*\the\DL,between origins}, background color=backgroundColor, ampersand replacement=\&]
                    \Sets_{*}(A\odot X,K)
                    \arrow[r,"\Phi_{K}"]
                    \arrow[d,"{\phi_{*}}"']
                    \&
                    \Sets(A,\Sets_{*}(X,K))%
                    \arrow[d,"{(\phi_{*})_{*}}"]
                    \\
                    \Sets_{*}(A\odot X,K')
                    \arrow[r,"\Phi_{K'}"']
                    \&
                    \Sets(A,\Sets_{*}(X,K'))%
                \end{tikzcd}
            \]%
            commutes. Indeed, given a morphism of pointed sets
            \[
                \xi%
                \colon%
                (A\odot X,a\odot x_{0})%
                \to%
                (K,k_{0}),%
            \]%
            we have
            \begin{align*}
                [\Phi_{K'}\circ\phi_{*}](\xi) &= \Phi_{K'}(\phi_{*}(\xi))\\
                                              &= \Phi_{K'}(\phi\circ\xi)\\
                                              &= (\phi\circ\xi)^{\dagger}\\
                                              &= \llbracket a\mapsto\phi\circ\xi(a\odot-)\rrbracket\\
                                              &= \llbracket a\mapsto\phi_{*}(\xi(a\odot-))\rrbracket\\
                                              &= (\phi_{*})_{*}(\llbracket a\mapsto\xi(a\odot-\rrbracket))\\
                                              &= (\phi_{*})_{*}(\Phi_{K}(\xi))\\
                                              &= [(\phi_{*})_{*}\circ\Phi_{K}](\xi).
            \end{align*}
        \item\label{proof-of-construction-of-tensors-of-pointed-sets-by-sets-6}\SloganFont{Naturality of $\Psi$. }Since $\Phi$ is natural and $\Phi$ is a componentwise inverse to $\Psi$, it follows from \ChapterRef{\ChapterCategories, \cref{categories:properties-of-natural-isomorphisms-componentwise-inverses-of-natural-transformations-assemble-into-natural-transformations} of \cref{categories:properties-of-natural-isomorphisms}}{\cref{properties-of-natural-isomorphisms-componentwise-inverses-of-natural-transformations-assemble-into-natural-transformations} of \cref{properties-of-natural-isomorphisms}} that $\Psi$ is also natural.
    \end{enumerate}
    This finishes the proof.
\end{Proof}
\begin{proposition}{Properties of Tensors of Pointed Sets by Sets}{properties-of-tensors-of-pointed-sets-by-sets}%
    Let $(X,x_{0})$ be a pointed set and let $A$ be a set.%
    \begin{enumerate}
        \item\label{properties-of-tensors-of-pointed-sets-by-sets-functoriality}\SloganFont{Functoriality. }The assignments $A,(X,x_{0}),(A,(X,x_{0}))$ define functors
            \[
                \BifunctorialityPeriod{A\odot-}{-\odot X}{-_{1}\odot-_{2}}{\Sets\mrp{{}_{*}}}{\Sets}{\Sets\times\Sets_{*}}{\Sets_{*}}%
            \]%
            In particular, given:
            \begin{itemize}
                \item A map of sets $f\colon A\to B$;
                \item A pointed map $\phi\colon(X,x_{0})\to(Y,y_{0})$;
            \end{itemize}
            the induced map
            \[
                f\odot\phi%
                \colon%
                A\odot X%
                \to%
                B\odot Y%
            \]%
            is given by
            \[
                [f\odot\phi](a\odot x)%
                \defeq%
                f(a)\odot\phi(x)%
            \]%
            for each $a\odot x\in A\odot X$.
        \item\label{properties-of-tensors-of-pointed-sets-by-sets-adjointness-1}\SloganFont{Adjointness \rmI. }We have an adjunction
            \begin{webcompile}
                \Adjunction#-\odot X#\Sets_{*}(X,-)#\Sets#\Sets_{*},#
            \end{webcompile}
            witnessed by a bijection
            \[
                \Sets_{*}(A\odot X,K)%
                \cong%
                \Sets(A,\Sets_{*}(X,K)),%
            \]%
            natural in $A\in\Obj(\Sets)$ and $X,Y\in\Obj(\Sets_{*})$.
        \item\label{properties-of-tensors-of-pointed-sets-by-sets-adjointness-2}\SloganFont{Adjointness \rmII. }We have an adjunctions
            \begin{webcompile}
                \Adjunction#A\odot -#A\pitchfork-#\Sets_{*}#\Sets_{*},#
            \end{webcompile}
            witnessed by a bijection
            \[
                \Hom_{\Sets_{*}}(A\odot X,Y)%
                \cong%
                \Hom_{\Sets_{*}}(X,A\pitchfork Y),%
            \]%
            natural in $A\in\Obj(\Sets)$ and $X,Y\in\Obj(\Sets_{*})$.
        \item\label{properties-of-tensors-of-pointed-sets-by-sets-as-a-weighted-colimit}\SloganFont{As a Weighted Colimit. }We have
            \[
                A\odot X%
                \cong%
                \wcolim^{[A]}(X),
            \]%
            where in the right hand side we write:
            \begin{itemize}
                \item $A$ for the functor $A\colon\pt\to\Sets$ picking $A\in\Obj(\Sets)$;
                \item $X$ for the functor $X\colon\pt\to\Sets_{*}$ picking $(X,x_{0})\in\Obj(\Sets_{*})$.
            \end{itemize}
        \item\label{properties-of-tensors-of-pointed-sets-by-sets-iterated-tensors}\SloganFont{Iterated Tensors. }We have an isomorphism of pointed sets
            \[
                A\odot(B\odot X)%
                \cong%
                (A\times B)\odot X,%
            \]%
            natural in $A,B\in\Obj(\Sets)$ and $(X,x_{0})\in\Obj(\Sets_{*})$.
        \item\label{properties-of-tensors-of-pointed-sets-by-sets-interaction-with-homs}\SloganFont{Interaction With Homs. }We have a natural isomorphism
            \[%
                \Sets_{*}(A\odot X,-)%
                \cong %
                A\pitchfork\Sets_{*}(X,-).%
            \]%
        \item\label{properties-of-tensors-of-pointed-sets-by-sets-the-tensor-evaluation-map}\SloganFont{The Tensor Evaluation Map. }For each $X,Y\in\Obj(\Sets_{*})$, we have a map
            \[
                \ev^{\odot}_{X,Y}%
                \colon%
                \Sets_{*}(X,Y)\odot X%
                \to%
                Y,%
            \]%
            natural in $X,Y\in\Obj(\Sets_{*})$, and given by
            \[%
                \ev^{\odot}_{X,Y}(f\odot x)%
                \defeq%
                f(x)%
            \]%
            for each $f\odot x\in\Sets_{*}(X,Y)\odot X$.%
        \item\label{properties-of-tensors-of-pointed-sets-by-sets-the-tensor-coevaluation-map}\SloganFont{The Tensor Coevaluation Map. }For each $A\in\Obj(\Sets)$ and each $X\in\Obj(\Sets_{*})$, we have a map
            \[
                \coev^{\odot}_{A,X}%
                \colon%
                A%
                \to%
                \Sets_{*}(X,A\odot X),%
            \]%
            natural in $A\in\Obj(\Sets)$ and $X\in\Obj(\Sets_{*})$, and given by
            \[%
                \coev^{\odot}_{A,X}(a)%
                \defeq%
                \llbracket x\mapsto a\odot x\rrbracket%
            \]%
            for each $a\in A$.%
        %\item\label{properties-of-tensors-of-pointed-sets-by-sets-}\SloganFont{. }
    \end{enumerate}
\end{proposition}
\begin{Proof}{Proof of \cref{properties-of-tensors-of-pointed-sets-by-sets}}%
    \FirstProofBox{\cref{properties-of-tensors-of-pointed-sets-by-sets-functoriality}: Functoriality}%
    This is the special case of \ChapterRef{\ChapterLimitsAndColimits, \cref{limits-and-colimits:properties-of-tensors-functoriality} of \cref{limits-and-colimits:properties-of-tensors}}{\cref{properties-of-tensors-functoriality} of \cref{properties-of-tensors}} for $\CatFont{C}=\Sets_{*}$.

    \ProofBox{\cref{properties-of-tensors-of-pointed-sets-by-sets-adjointness-1}: Adjointness \rmI}%
    This is simply a rephrasing of \cref{tensors-of-pointed-sets-by-sets}.

    \ProofBox{\cref{properties-of-tensors-of-pointed-sets-by-sets-adjointness-2}: : Adjointness \rmII}%
    This is the special case of \ChapterRef{\ChapterLimitsAndColimits, \cref{limits-and-colimits:properties-of-tensors-adjointness-2} of \cref{limits-and-colimits:properties-of-tensors}}{\cref{properties-of-tensors-adjointness-2} of \cref{properties-of-tensors}} for $\CatFont{C}=\Sets_{*}$.

    \ProofBox{\cref{properties-of-tensors-of-pointed-sets-by-sets-as-a-weighted-colimit}: As a Weighted Colimit}%
    This is the special case of \ChapterRef{\ChapterLimitsAndColimits, \cref{limits-and-colimits:properties-of-tensors-as-a-weighted-colimit} of \cref{limits-and-colimits:properties-of-tensors}}{\cref{properties-of-tensors-as-a-weighted-colimit} of \cref{properties-of-tensors}} for $\CatFont{C}=\Sets_{*}$.

    \ProofBox{\cref{properties-of-tensors-of-pointed-sets-by-sets-iterated-tensors}: Iterated Tensors}%
    This is the special case of \ChapterRef{\ChapterLimitsAndColimits, \cref{limits-and-colimits:properties-of-tensors-iterated-tensors} of \cref{limits-and-colimits:properties-of-tensors}}{\cref{properties-of-tensors-iterated-tensors} of \cref{properties-of-tensors}} for $\CatFont{C}=\Sets_{*}$.

    \ProofBox{\cref{properties-of-tensors-of-pointed-sets-by-sets-interaction-with-homs}: Interaction With Homs}%
    This is the special case of \ChapterRef{\ChapterLimitsAndColimits, \cref{limits-and-colimits:properties-of-tensors-interaction-with-homs} of \cref{limits-and-colimits:properties-of-tensors}}{\cref{properties-of-tensors-interaction-with-homs} of \cref{properties-of-tensors}} for $\CatFont{C}=\Sets_{*}$.

    \ProofBox{\cref{properties-of-tensors-of-pointed-sets-by-sets-the-tensor-evaluation-map}: The Tensor Evaluation Map}%
    This is the special case of \ChapterRef{\ChapterLimitsAndColimits, \cref{limits-and-colimits:properties-of-tensors-the-tensor-evaluation-map} of \cref{limits-and-colimits:properties-of-tensors}}{\cref{properties-of-tensors-the-tensor-evaluation-map} of \cref{properties-of-tensors}} for $\CatFont{C}=\Sets_{*}$.

    \ProofBox{\cref{properties-of-tensors-of-pointed-sets-by-sets-the-tensor-coevaluation-map}: The Tensor Coevaluation Map}%
    This is the special case of \ChapterRef{\ChapterLimitsAndColimits, \cref{limits-and-colimits:properties-of-tensors-the-tensor-coevaluation-map} of \cref{limits-and-colimits:properties-of-tensors}}{\cref{properties-of-tensors-the-tensor-coevaluation-map} of \cref{properties-of-tensors}} for $\CatFont{C}=\Sets_{*}$.
\end{Proof}
\subsection{Cotensors of Pointed Sets by Sets}\label{subsection-cotensors-by-sets}
Let $(X,x_{0})$ be a pointed set and let $A$ be a set.%
\begin{definition}{Cotensors of Pointed Sets by Sets}{cotensors-of-pointed-sets-by-sets}%
    The \index[set-theory]{pointed set!cotensor by a set}\textbf{cotensor of $(X,x_{0})$ by $A$}%
    %--- Begin Footnote ---%
    \footnote{%
        \SloganFont{Further Terminology: }Also called the \index[set-theory]{pointed set!power by a set|see {pointed set, cotensor by a set}}\textbf{power of $(X,x_{0})$ by $A$}.%
    } %
    %---  End Footnote  ---%
    is the cotensor \index[notation]{ApitchforkXxzero@$A\pitchfork(X,x_{0})$}$A\pitchfork(X,x_{0})$%
    %--- Begin Footnote ---%
    \footnote{%
        \SloganFont{Further Notation: }Often written \index[notation]{ApitchforkX@$A\pitchfork X$}$A\pitchfork X$ for simplicity.
        \par\vspace*{\TCBBoxCorrection}
    } %
    %---  End Footnote  ---%
    of $(X,x_{0})$ by $A$ as in \ChapterRef{\ChapterLimitsAndColimits, \cref{weighted-limits-and-colimits:cotensors}}{\cref{cotensors}}.
\end{definition}
\begin{remark}{Unwinding \cref{cotensors-of-pointed-sets-by-sets}}{unwinding-cotensors-of-pointed-sets-by-sets}%
    In detail, the \textbf{cotensor of $(X,x_{0})$ by $A$} is the pointed set $A\pitchfork(X,x_{0})$ satisfying the following universal property:
    \begin{itemize}
        \itemstar We have a bijection
        \[%
             \Sets_{*}(K,A\pitchfork X)%
             \cong%
             \Sets(A,\Sets_{*}(K,X)),%
        \]%
        natural in $(K,k_{0})\in\Obj(\Sets_{*})$.
    \end{itemize}%
    This universal property is in turn equivalent to the following one:
    \begin{itemize}
        \itemstar We have a bijection
        \[
            \Sets_{*}(K,A\pitchfork X)
            \cong
            \Sets^{\otimes}_{\E_{0}}(A\times K,X),
        \]%
        natural in $(K,k_{0})\in\Obj(\Sets_{*})$, where $\Sets^{\otimes}_{\E_{0}}(A\times K,X)$ is the set defined by
        \[
            \Sets^{\otimes}_{\E_{0}}(A\times K,X)
            \defeq
            \{%
                f\in\Sets(A\times K,X)%
                \ \middle|\ %
                \begin{aligned}
                    &\text{for each $a\in A$, we}\\
                    &\text{have $f(a,k_{0})=x_{0}$}%
                \end{aligned}
            \}.
        \]%
    \end{itemize}%
\end{remark}
\begin{Proof}{Proof of the Equivalence in \cref{unwinding-cotensors-of-pointed-sets-by-sets}}%
    This follows from the bijection
    \[
        \Sets(A,\Sets_{*}(K,X))%
        \cong%
        \Sets^{\otimes}_{\E_{0}}(A\times K,X),%
    \]%
    natural in $(K,k_{0})\in\Obj(\Sets_{*})$ constructed in the proof of \cref{unwinding-tensors-of-pointed-sets-by-sets}.
\end{Proof}
\begin{construction}{Construction of Cotensors of Pointed Sets by Sets}{construction-of-cotensors-of-pointed-sets-by-sets}%
    Concretely, the \textbf{cotensor of $(X,x_{0})$ by $A$} is the pointed set $A\pitchfork(X,x_{0})$ consisting of:
    \begin{itemize}
        \item\SloganFont{The Underlying Set. }The set $A\pitchfork X$ given by
            \[
                A\pitchfork X%
                \cong%
                \bigwedge_{a\in A}(X,x_{0}),%
            \]%
            where $\bigwedge_{a\in A}(X,x_{0})$ is the smash product of the $A$-indexed family $((X,x_{0}))_{a\in A}$ of \cref{the-smash-product-of-a-family-of-pointed-sets}.
        \item\SloganFont{The Basepoint. }The point $[(x_{0})_{a\in A}]=[(x_{0},x_{0},x_{0},\ldots)]$ of $\bigwedge_{a\in A}(X,x_{0})$.
    \end{itemize}
\end{construction}
\begin{Proof}{Proof of \cref{construction-of-cotensors-of-pointed-sets-by-sets}}%
    We claim we have a bijection
    \[
         \Sets_{*}(K,A\pitchfork X)%
         \cong%
         \Sets(A,\Sets_{*}(K,X)),%
    \]%
    natural in $(K,k_{0})\in\Obj(\Sets_{*})$.%
    \begin{enumerate}
        \item\label{proof-of-construction-of-cotensors-of-pointed-sets-by-sets-1}\SloganFont{Map \rmI. }We define a map
            \[
                \Phi_{K}%
                \colon%
                \Sets_{*}(K,A\pitchfork X)%
                \to%
                \Sets(A,\Sets_{*}(K,X)),%
            \]%
            by sending a morphism of pointed sets
            \[
                \xi%
                \colon%
                (K,k_{0})%
                \to%
                (A\pitchfork X,[(x_{0})_{a\in A}])%
            \]%
            to the map of sets%
            \begin{webcompile}
                \phantom{\xi^{\dagger}\colon}
                \begin{tikzcd}[row sep=0.0*\the\DL, column sep=1.0*\the\DL, background color=backgroundColor, ampersand replacement=\&]
                    \mathllap{\xi^{\dagger}\colon}A%
                    \arrow[r]
                    \&
                    \Sets_{*}(K,X)\mrp{,}%
                    \\
                    a
                    \arrow[r, mapsto]
                    \&
                    {(\xi_{a}\colon K\to X)\mrp{,}}
                \end{tikzcd}
            \end{webcompile}
            where
            \[
                \xi_{a}%
                \colon%
                (K,k_{0})%
                \to%
                (X,x_{0})%
            \]%
            is the morphism of pointed sets defined by
            \[
                \xi_{a}(k)%
                =%
                \begin{cases}
                    x^{k}_{a} &\text{if $\xi(k)\neq[(x_{0})_{a\in A}]$,}\\%
                    x_{0}     &\text{if $\xi(k)=[(x_{0})_{a\in A}]$}%
                \end{cases}
            \]%
            for each $k\in K$, where $x^{k}_{a}$ is the $a$th component of $\xi(k)=[(x^{k}_{a})_{a\in A}]$. Note that:
            \begin{enumerate}
                \item\label{proof-of-construction-of-cotensors-of-pointed-sets-by-sets-1a}The definition of $\xi_{a}(k)$ is independent of the choice of equivalence class. Indeed, suppose we have
                    \begin{align*}
                        \xi(k) &= [(x^{k}_{a})_{a\in A}]\\
                               &= [(y^{k}_{a})_{a\in A}]
                    \end{align*}
                    with $x^{k}_{a}\neq y^{k}_{a}$ for some $a\in A$. Then there exist $a_{x},a_{y}\in A$ such that $x^{k}_{a_{x}}=y^{k}_{a_{y}}=x_{0}$. The equivalence relation $\unsim$ on $\prod_{a\in A}X$ then forces
                    \begin{align*}
                        [(x^{k}_{a})_{a\in A}] &= [(x_{0})_{a\in A}],\\
                        [(y^{k}_{a})_{a\in A}] &= [(x_{0})_{a\in A}],
                    \end{align*}
                    however, and $\xi_{a}(k)$ is defined to be $x_{0}$ in this case.
                \item\label{proof-of-construction-of-cotensors-of-pointed-sets-by-sets-1b}The map $\xi_{a}$ is indeed a morphism of pointed sets, as we have
                    \[
                        \xi_{a}(k_{0})%
                        =%
                        x_{0}
                    \]%
                    since $\xi(k_{0})=[(x_{0})_{a\in A}]$ as $\xi$ is a morphism of pointed sets and $\xi_{a}(k_{0})$, defined to be the $a$th component of $[(x_{0})_{a\in A}]$, is equal to $x_{0}$.
            \end{enumerate}
        \item\label{proof-of-construction-of-cotensors-of-pointed-sets-by-sets-2}\SloganFont{Map \rmII. }We define a map
            \[
                \Psi_{K}%
                \colon%
                \Sets(A,\Sets_{*}(K,X))%
                \to%
                \Sets_{*}(K,A\pitchfork X),%
            \]%
            given by sending a map
            \begin{webcompile}
                \phantom{\xi\colon}
                \begin{tikzcd}[row sep=0.0*\the\DL, column sep=1.0*\the\DL, background color=backgroundColor, ampersand replacement=\&]
                    \mathllap{\xi\colon}A%
                    \arrow[r]
                    \&
                    \Sets_{*}(K,X)\mrp{,}%
                    \\
                    a
                    \arrow[r, mapsto]
                    \&
                    {(\xi_{a}\colon K\to X)\mrp{,}}
                \end{tikzcd}
            \end{webcompile}
            to the morphism of pointed sets
            \[
                \xi^{\dagger}%
                \colon%
                (K,k_{0})%
                \to%
                (A\pitchfork X,[(x_{0})_{a\in A}])%
            \]%
            defined by
            \[
                \xi^{\dagger}(k)%
                \defeq%
                [(\xi_{a}(k))_{a\in A}]%
            \]%
            for each $k\in K$. Note that $\xi^{\dagger}$ is indeed a morphism of pointed sets, as we have
            \begin{align*}
                \xi^{\dagger}(k_{0}) &\defeq [(\xi_{a}(k_{0}))_{a\in A}]\\%
                                     &=      x_{0},
            \end{align*}
            where we have used that $\xi_{a}\in\Sets_{*}(K,X)$ is a morphism of pointed sets for each $a\in A$.
        \item\label{proof-of-construction-of-cotensors-of-pointed-sets-by-sets-3}\SloganFont{Invertibility \rmI. }We claim that
            \[
                \Psi_{K}\circ\Phi_{K}%
                =%
                \id_{\Sets_{*}(K,A\pitchfork X)}.%
            \]%
            Indeed, given a morphism of pointed sets
            \[
                \xi%
                \colon%
                (K,k_{0})%
                \to%
                (A\pitchfork X,[(x_{0})_{a\in A}])%
            \]%
            we have
            \begin{align*}
                [\Psi_{K}\circ\Phi_{K}](\xi) &= \Psi_{K}(\Phi_{K}(\xi))\\%
                                             &= \Psi_{K}(\llbracket a\mapsto\xi_{a}\rrbracket)\\%
                                             &= \Psi_{K}(\llbracket a'\mapsto\xi_{a'}\rrbracket)\\%
                                             &= \llbracket k\mapsto[(\ev_{a}(\llbracket a'\mapsto\xi_{a'}(k)\rrbracket))_{a\in A}]\rrbracket\\%
                                             &= \llbracket k\mapsto[(\xi_{a}(k))_{a\in A}]\rrbracket.%
            \end{align*}
            Now, we have two cases:
            \begin{enumerate}
                \item\label{proof-of-construction-of-cotensors-of-pointed-sets-by-sets-3a}If $\xi(k)=[(x_{0})_{a\in A}]$, we have
                    \begin{align*}
                        [\Psi_{K}\circ\Phi_{K}](\xi) &= \llbracket k\mapsto[(\xi_{a}(k))_{a\in A}]\rrbracket\\%
                                                     &= \llbracket k\mapsto[(x_{0})_{a\in A}]\rrbracket\\%
                                                     &= \llbracket k\mapsto\xi(k)\rrbracket\\%
                                                     &= \xi.%
                    \end{align*}
                \item\label{proof-of-construction-of-cotensors-of-pointed-sets-by-sets-3b}If $\xi(k)\neq [(x_{0})_{a\in A}]$ and $\xi(k)=[(x^{k}_{a})_{a\in A}]$ instead, we have
                    \begin{align*}
                        [\Psi_{K}\circ\Phi_{K}](\xi) &= \llbracket k\mapsto[(\xi_{a}(k))_{a\in A}]\rrbracket\\%
                                                     &= \llbracket k\mapsto[(x^{k}_{a})_{a\in A}]\rrbracket\\%
                                                     &= \llbracket k\mapsto\xi(k)\rrbracket\\%
                                                     &= \xi.%
                    \end{align*}
            \end{enumerate}
            In both cases, we have $[\Psi_{K}\circ\Phi_{K}](\xi)=\xi$, and thus we are done.
        \item\label{proof-of-construction-of-cotensors-of-pointed-sets-by-sets-4}\SloganFont{Invertibility \rmII. }We claim that
            \[
                \Phi_{K}\circ\Psi_{K}%
                =%
                \id_{\Sets(A,\Sets_{*}(K,X))}.%
            \]%
            Indeed, given a morphism $\xi\colon A\to\Sets_{*}(K,X)$, we have
            \begin{align*}
                [\Phi_{K}\circ\Psi_{K}](\xi) &= \Phi_{K}(\Psi_{K}(\xi))\\%
                                             &= \Phi_{K}(\llbracket k\mapsto[(\xi_{a}(k))_{a\in A}]\rrbracket)\\%
                                             &= \llbracket a\mapsto\llbracket k\mapsto\xi_{a}(k)\rrbracket\rrbracket\\%
                                             &= \xi%
            \end{align*}
        \item\label{proof-of-construction-of-cotensors-of-pointed-sets-by-sets-5}\SloganFont{Naturality of $\Psi$. }We need to show that, given a morphism of pointed sets
            \[
                \phi%
                \colon%
                (K,k_{0})%
                \to%
                (K',k'_{0}),%
            \]%
            the diagram
            \[
                \begin{tikzcd}[row sep={5.0*\the\DL,between origins}, column sep={12.0*\the\DL,between origins}, background color=backgroundColor, ampersand replacement=\&]
                    \Sets(A,\Sets_{*}(K',X))%
                    \arrow[r,"\Psi_{K'}"]
                    \arrow[d,"{(\phi^{*})_{*}}"']
                    \&
                    \Sets_{*}(K',A\pitchfork X)%
                    \arrow[d,"\phi^{*}"]
                    \\
                    \Sets(A,\Sets_{*}(K,X))%
                    \arrow[r,"\Psi_{K}"']
                    \&
                    \Sets_{*}(K,A\pitchfork X)%
                \end{tikzcd}
            \]%
            commutes. Indeed, given a map of sets
            \begin{webcompile}
                \phantom{\xi\colon}
                \begin{tikzcd}[row sep=0.0*\the\DL, column sep=1.0*\the\DL, background color=backgroundColor, ampersand replacement=\&]
                    \mathllap{\xi\colon}A%
                    \arrow[r]
                    \&
                    \Sets_{*}(K',X)\mrp{,}%
                    \\
                    a
                    \arrow[r, mapsto]
                    \&
                    {(\xi_{a}\colon K'\to X)\mrp{,}}
                \end{tikzcd}
            \end{webcompile}
            we have
            \begin{align*}
                [\Psi_{K}\circ(\phi^{*})_{*}](\xi) &= \Psi_{K}((\phi^{*})_{*}(\xi))\\
                                                   &= \Psi_{K}((\phi^{*})_{*}(\llbracket a\mapsto\xi_{a}\rrbracket))\\
                                                   &= \Psi_{K}((\llbracket a\mapsto\phi^{*}(\xi_{a})\rrbracket))\\
                                                   &= \Psi_{K}((\llbracket a\mapsto\llbracket k\mapsto\xi_{a}(\phi(k))\rrbracket\rrbracket))\\
                                                   &= \llbracket k\mapsto[(\xi_{a}(\phi(k)))_{a\in A}]\rrbracket\\
                                                   &= \phi^{*}(\llbracket k'\mapsto[(\xi_{a}(k'))_{a\in A}]\rrbracket)\\
                                                   &= \phi^{*}(\Psi_{K'}(\xi))\\
                                                   &= [\phi^{*}\circ\Psi_{K'}](\xi).
            \end{align*}
        \item\label{proof-of-construction-of-cotensors-of-pointed-sets-by-sets-6}\SloganFont{Naturality of $\Phi$. }Since $\Psi$ is natural and $\Psi$ is a componentwise inverse to $\Phi$, it follows from \ChapterRef{\ChapterCategories, \cref{categories:properties-of-natural-isomorphisms-componentwise-inverses-of-natural-transformations-assemble-into-natural-transformations} of \cref{categories:properties-of-natural-isomorphisms}}{\cref{properties-of-natural-isomorphisms-componentwise-inverses-of-natural-transformations-assemble-into-natural-transformations} of \cref{properties-of-natural-isomorphisms}} that $\Phi$ is also natural.
    \end{enumerate}
    This finishes the proof.
\end{Proof}
\begin{proposition}{Properties of Cotensors of Pointed Sets by Sets}{properties-of-cotensors-of-pointed-sets-by-sets}%
    Let $(X,x_{0})$ be a pointed set and let $A$ be a set.%
    \begin{enumerate}
        \item\label{properties-of-cotensors-of-pointed-sets-by-sets-functoriality}\SloganFont{Functoriality. }The assignments $A,(X,x_{0}),(A,(X,x_{0}))$ define functors
            \[
                \BifunctorialityPeriod{A\pitchfork-}{-\pitchfork X}{-_{1}\pitchfork-_{2}}{\Sets\mrp{{}_{*}}}{\Sets^{\mrp{\op}}}{\Sets^{\op}\times\Sets_{*}}{\Sets_{*}}%
            \]%
            In particular, given:
            \begin{itemize}
                \item A map of sets $f\colon A\to B$;
                \item A pointed map $\phi\colon(X,x_{0})\to(Y,y_{0})$;
            \end{itemize}
            the induced map
            \[
                f\odot\phi%
                \colon%
                A\pitchfork X%
                \to%
                B\pitchfork Y%
            \]%
            is given by
            \[
                [f\odot\phi]([(x_{a})_{a\in A}])%
                \defeq%
                [(\phi(x_{f(a)}))_{a\in A}]%
            \]%
            for each $[(x_{a})_{a\in A}]\in A\pitchfork X$.
        \item\label{properties-of-cotensors-of-pointed-sets-by-sets-adjointness-1}\SloganFont{Adjointness \rmI. }We have an adjunction
            \begin{webcompile}
                \Adjunction#-\pitchfork X#\Sets_{*}(-,X)#\Sets^{\op}#\Sets_{*},#
            \end{webcompile}
            witnessed by a bijection
            \[
                \Sets^{\op}_{*}(A\pitchfork X,K)%
                \cong%
                \Sets(A,\Sets_{*}(K,X)),%
            \]%
            i.e.\ by a bijection
            \[
                \Sets_{*}(K,A\pitchfork X)%
                \cong%
                \Sets(A,\Sets_{*}(K,X)),%
            \]%
            natural in $A\in\Obj(\Sets)$ and $X,Y\in\Obj(\Sets_{*})$.
        \item\label{properties-of-cotensors-of-pointed-sets-by-sets-adjointness-2}\SloganFont{Adjointness \rmII. }We have an adjunctions
            \begin{webcompile}
                \Adjunction#A\odot -#A\pitchfork-#\Sets_{*}#\Sets_{*},#
            \end{webcompile}
            witnessed by a bijection
            \[
                \Hom_{\Sets_{*}}(A\odot X,Y)%
                \cong%
                \Hom_{\Sets_{*}}(X,A\pitchfork Y),%
            \]%
            natural in $A\in\Obj(\Sets)$ and $X,Y\in\Obj(\Sets_{*})$.
        \item\label{properties-of-cotensors-of-pointed-sets-by-sets-as-a-weighted-limit}\SloganFont{As a Weighted Limit. }We have
            \[
                A\pitchfork X%
                \cong%
                \wlim^{[A]}(X),
            \]%
            where in the right hand side we write:
            \begin{itemize}
                \item $A$ for the functor $A\colon\pt\to\Sets$ picking $A\in\Obj(\Sets)$;
                \item $X$ for the functor $X\colon\pt\to\Sets_{*}$ picking $(X,x_{0})\in\Obj(\Sets_{*})$.
            \end{itemize}
        \item\label{properties-of-cotensors-of-pointed-sets-by-sets-iterated-cotensors}\SloganFont{Iterated Cotensors. }We have an isomorphism of pointed sets
            \[
                A\pitchfork(B\pitchfork X)%
                \cong%
                (A\times B)\pitchfork X,%
            \]%
            natural in $A,B\in\Obj(\Sets)$ and $(X,x_{0})\in\Obj(\Sets_{*})$.
        \item\label{properties-of-cotensors-of-pointed-sets-by-sets-commutativity-with-homs}\SloganFont{Commutativity With Homs. }We have natural isomorphisms
            \begin{align*}
                A\pitchfork\Sets_{*}(X,-) &\cong \Sets_{*}(A\odot X,-),\\
                A\pitchfork\Sets_{*}(-,Y) &\cong \Sets_{*}(-,A\pitchfork Y).
            \end{align*}
        \item\label{properties-of-cotensors-of-pointed-sets-by-sets-the-cotensor-evaluation-map}\SloganFont{The Cotensor Evaluation Map. }For each $X,Y\in\Obj(\Sets_{*})$, we have a map
            \[
                \ev^{\pitchfork}_{X,Y}%
                \colon%
                X%
                \to%
                \Sets_{*}(X,Y)\pitchfork Y,%
            \]%
            natural in $X,Y\in\Obj(\Sets_{*})$, and given by
            \[%
                \ev^{\pitchfork}_{X,Y}(x)%
                \defeq%
                [(f(x))_{f\in\Sets_{*}(X,Y)}]%
            \]%
            for each $x\in X$.%
        \item\label{properties-of-cotensors-of-pointed-sets-by-sets-the-cotensor-coevaluation-map}\SloganFont{The Cotensor Coevaluation Map. }For each $X\in\Obj(\Sets_{*})$ and each $A\in\Obj(\Sets)$, we have a map
            \[
                \coev^{\pitchfork}_{A,X}%
                \colon%
                A%
                \to%
                \Sets_{*}(A\pitchfork X,X),%
            \]%
            natural in $X\in\Obj(\Sets_{*})$ and $A\in\Obj(\Sets)$, and given by
            \[%
                \coev^{\pitchfork}_{A,X}(a)%
                \defeq%
                \llbracket[(x_{b})_{b\in A}]\mapsto x_{a}\rrbracket%
            \]%
            for each $a\in A$.%
        %\item\label{properties-of-cotensors-of-pointed-sets-by-sets-}\SloganFont{. }
    \end{enumerate}
\end{proposition}
\begin{Proof}{Proof of \cref{properties-of-cotensors-of-pointed-sets-by-sets}}%
    \FirstProofBox{\cref{properties-of-cotensors-of-pointed-sets-by-sets-functoriality}: Functoriality}%
    This is the special case of \ChapterRef{\ChapterLimitsAndColimits, \cref{limits-and-colimits:properties-of-cotensors-functoriality} of \cref{limits-and-colimits:properties-of-cotensors}}{\cref{properties-of-cotensors-functoriality} of \cref{properties-of-cotensors}} for $\CatFont{C}=\Sets_{*}$.

    \ProofBox{\cref{properties-of-cotensors-of-pointed-sets-by-sets-adjointness-1}: Adjointness \rmI}%
    This is simply a rephrasing of \cref{cotensors-of-pointed-sets-by-sets}.

    \ProofBox{\cref{properties-of-cotensors-of-pointed-sets-by-sets-adjointness-2}: : Adjointness \rmII}%
    This is the special case of \ChapterRef{\ChapterLimitsAndColimits, \cref{limits-and-colimits:properties-of-cotensors-adjointness-2} of \cref{limits-and-colimits:properties-of-cotensors}}{\cref{properties-of-cotensors-adjointness-2} of \cref{properties-of-cotensors}} for $\CatFont{C}=\Sets_{*}$.

    \ProofBox{\cref{properties-of-cotensors-of-pointed-sets-by-sets-as-a-weighted-limit}: As a Weighted Limit}%
    This is the special case of \ChapterRef{\ChapterLimitsAndColimits, \cref{limits-and-colimits:properties-of-cotensors-as-a-weighted-limit} of \cref{limits-and-colimits:properties-of-cotensors}}{\cref{properties-of-cotensors-as-a-weighted-limit} of \cref{properties-of-cotensors}} for $\CatFont{C}=\Sets_{*}$.

    \ProofBox{\cref{properties-of-cotensors-of-pointed-sets-by-sets-iterated-cotensors}: Iterated Cotensors}%
    This is the special case of \ChapterRef{\ChapterLimitsAndColimits, \cref{limits-and-colimits:properties-of-cotensors-iterated-cotensors} of \cref{limits-and-colimits:properties-of-cotensors}}{\cref{properties-of-cotensors-iterated-cotensors} of \cref{properties-of-cotensors}} for $\CatFont{C}=\Sets_{*}$.

    \ProofBox{\cref{properties-of-cotensors-of-pointed-sets-by-sets-commutativity-with-homs}: Commutativity With Homs}%
    This is the special case of \ChapterRef{\ChapterLimitsAndColimits, \cref{limits-and-colimits:properties-of-cotensors-commutativity-with-homs} of \cref{limits-and-colimits:properties-of-cotensors}}{\cref{properties-of-cotensors-commutativity-with-homs} of \cref{properties-of-cotensors}} for $\CatFont{C}=\Sets_{*}$.

    \ProofBox{\cref{properties-of-cotensors-of-pointed-sets-by-sets-the-cotensor-evaluation-map}: The Cotensor Evaluation Map}%
    This is the special case of \ChapterRef{\ChapterLimitsAndColimits, \cref{limits-and-colimits:properties-of-cotensors-the-cotensor-evaluation-map} of \cref{limits-and-colimits:properties-of-cotensors}}{\cref{properties-of-cotensors-the-cotensor-evaluation-map} of \cref{properties-of-cotensors}} for $\CatFont{C}=\Sets_{*}$.

    \ProofBox{\cref{properties-of-cotensors-of-pointed-sets-by-sets-the-cotensor-coevaluation-map}: The Cotensor Coevaluation Map}%
    This is the special case of \ChapterRef{\ChapterLimitsAndColimits, \cref{limits-and-colimits:properties-of-cotensors-the-cotensor-coevaluation-map} of \cref{limits-and-colimits:properties-of-cotensors}}{\cref{properties-of-cotensors-the-cotensor-coevaluation-map} of \cref{properties-of-cotensors}} for $\CatFont{C}=\Sets_{*}$.
\end{Proof}
\section{The Left Tensor Product of Pointed Sets}\label{section-the-left-tensor-product-of-pointed-sets}
\subsection{Foundations}\label{subsection-the-left-tensor-product-of-pointed-sets-foundations}
Let $(X,x_{0})$ and $(Y,y_{0})$ be pointed sets.
\begin{definition}{The Left Tensor Product of Pointed Sets}{the-left-tensor-product-of-pointed-sets}%
    The \index[set-theory]{pointed set!left tensor product}\textbf{left tensor product of pointed sets} is the functor\index[notation]{lhd@$\lhd$}%
    %--- Begin Footnote ---%
    \footnote{%
        \SloganFont{Further Notation: }Also written \index[notation]{lhdSetsstar@$\lhd_{\Sets_{*}}$}$\lhd_{\Sets_{*}}$.
        \par\vspace*{\TCBBoxCorrection}
    }%
    %---  End Footnote  ---%
    \[
        \lhd%
        \colon%
        \Sets_{*}\times\Sets_{*}%
        \to%
        \Sets_{*}%
    \]%
    defined as the composition
    \[
        \Sets_{*}\times\Sets_{*}%
        \xlongrightarrow{\sfid\times\Wasureru}%
        \Sets_{*}\times\Sets%
        \xlongrightarrow{\bfbeta^{\TwoCategoryOfCategories}_{\Sets_{*},\Sets}}%
        \Sets\times\Sets_{*}%
        \xlongrightarrow{\odot}%
        \Sets_{*},%
    \]%
    where:
    \begin{itemize}
        \item $\Wasureru\colon\Sets_{*}\to\Sets$ is the forgetful functor from pointed sets to sets.
        \item ${\bfbeta^{\TwoCategoryOfCategories}_{\Sets_{*},\Sets}}\colon\Sets_{*}\times\Sets\isorightarrow\Sets\times\Sets_{*}$ is the braiding of $\TwoCategoryOfCategories$, i.e.\ the functor witnessing the isomorphism%
            \[
                \Sets_{*}\times\Sets%
                \cong%
                \Sets\times\Sets_{*}.%
            \]%
        \item $\odot\colon\Sets\times\Sets_{*}\to\Sets_{*}$ is the tensor functor of \cref{properties-of-tensors-of-pointed-sets-by-sets-functoriality} of \cref{properties-of-tensors-of-pointed-sets-by-sets}.%
    \end{itemize}
\end{definition}
\begin{remark}{Unwinding \cref{the-left-tensor-product-of-pointed-sets}: Universal Property \rmI}{unwinding-the-left-tensor-product-of-pointed-sets-1-universal-property-1}%
    The left tensor product of pointed sets satisfies the following natural bijection:%
    \[%
        \Sets_{*}(X\lhd Y,Z)%
        \cong%
        \Hom^{\otimes,\rmL}_{\Sets_{*}}(X\times Y,Z).
    \]%
    That is to say, the following data are in natural bijection:
    \begin{enumerate}
        \item\label{unwinding-the-left-tensor-product-of-pointed-sets-1-universal-property-1-item-1}Pointed maps $f\colon X\lhd Y\to Z$.
        \item\label{unwinding-the-left-tensor-product-of-pointed-sets-1-universal-property-1-item-2}Maps of sets $f\colon X\times Y\to Z$ satisfying $f(x_{0},y)=z_{0}$ for each $y\in Y$.
    \end{enumerate}
\end{remark}
\begin{remark}{Unwinding \cref{the-left-tensor-product-of-pointed-sets}: Universal Property \rmII}{unwinding-the-left-tensor-product-of-pointed-sets-2-universal-property-2}%
    The left tensor product of pointed sets may be described as follows:
    \begin{itemize}
        \item The left tensor product of $(X,x_{0})$ and $(Y,y_{0})$ is the pair $((X\lhd Y,x_{0}\lhd y_{0}),\iota)$ consisting of
            \begin{itemize}
                \item A pointed set $(X\lhd Y,x_{0}\lhd y_{0})$;
                \item A left bilinear morphism of pointed sets $\iota\colon(X\times Y,(x_{0},y_{0}))\to X\lhd Y$;
            \end{itemize}
            satisfying the following universal property:
            \begin{itemize}
                \itemstar Given another such pair $((Z,z_{0}),f)$ consisting of
                    \begin{itemize}
                        \item A pointed set $(Z,z_{0})$;
                        \item A left bilinear morphism of pointed sets $f\colon(X\times Y,(x_{0},y_{0}))\to X\lhd Y$;
                    \end{itemize}
                    there exists a unique morphism of pointed sets $X\lhd Y\uearrow Z$ making the diagram
                    \[
                        \begin{tikzcd}[row sep={5.0*\the\DL,between origins}, column sep={5.0*\the\DL,between origins}, background color=backgroundColor, ampersand replacement=\&]
                            \&
                            X\lhd Y
                            \arrow[d,"\exists!",densely dashed]
                            \\
                            X\times Y
                            \arrow[r,"f"']
                            \arrow[ru,"\iota"]
                            \&
                            Z
                        \end{tikzcd}
                    \]%
                    commute.%
                \end{itemize}
    \end{itemize}
\end{remark}
\begin{construction}{The Left Tensor Product of Pointed Sets}{constructions-of-the-left-tensor-product-of-pointed-sets}%
    In detail, the \textbf{left tensor product of $(X,x_{0})$ and $(Y,y_{0})$} is the pointed set $(X\lhd Y,[x_{0}])$ consisting of:%
    \begin{itemize}
        \item\SloganFont{The Underlying Set. }The set $X\lhd Y$ defined by
            \begin{align*}
                X\lhd Y &\defeq \abs{Y}\odot X\\
                        &\cong  \bigvee_{y\in Y}(X,x_{0}),
            \end{align*}
            where $\abs{Y}$ denotes the underlying set of $(Y,y_{0})$.
        \item\SloganFont{The Underlying Basepoint. }The point $[(y_{0},x_{0})]$ of $\bigvee_{y\in Y}(X,x_{0})$, which is equal to $[(y,x_{0})]$ for any $y\in Y$.
    \end{itemize}
\end{construction}
\begin{Proof}{Proof of \cref{constructions-of-the-left-tensor-product-of-pointed-sets}}%
    Since $\bigvee_{y\in Y}(X,x_{0})$ is defined as the quotient of $\coprod_{y\in Y}X$ by the equivalence relation $R$ generated by declaring $(y,x)\sim(y',x')$ if $x=x'=x_{0}$, we have, by \ChapterRef{\ChapterConditionsOnRelations, \cref{conditions-on-relations:unwinding-the-quotient-of-a-set-by-an-equivalence-relation}}{\cref{unwinding-the-quotient-of-a-set-by-an-equivalence-relation}}, a natural bijection
    \[
        \Sets_{*}(X\lhd Y,Z)
        \cong
        \Hom^{R}_{\Sets}(\coprod_{y\in Y}X,Z),%
    \]%
    where $\Hom^{R}_{\Sets}(X\times Y,Z)$ is the set
    \begin{envsmallsize}
        \[
            \Hom^{R}_{\Sets}(\coprod_{y\in Y}X,Z)%
            \defeq%
            \{%
                f\in\Hom_{\Sets}(\coprod_{y\in Y}X,Z)%
                \ \middle|\ %
                \begin{aligned}
                    &\text{for each $x,y\in X$, if}\\
                    &\text{$(y,x)\sim_{R}(y',x')$, then}\\
                    &\text{$f(y,x)=f(y',x')$}%
                \end{aligned}
            \}.%
        \]%
    \end{envsmallsize}
    However, the condition $(y,x)\sim_{R}(y',x')$ only holds when:
    \begin{enumerate}
        \item\label{proof-of-constructions-of-the-left-tensor-product-of-pointed-sets-1}We have $x=x'$ and $y=y'$.
        \item\label{proof-of-constructions-of-the-left-tensor-product-of-pointed-sets-2}We have $x=x'=x_{0}$.
    \end{enumerate}
    So, given $f\in\Hom_{\Sets}(\coprod_{y\in Y}X,Z)$ with a corresponding $\widebar{f}\colon X\lhd Y\to Z$, the latter case above implies
    \begin{align*}
        f([(y,x_{0})]) &= f([(y',x_{0})])\\
                       &= f([(y_{0},x_{0})]),
    \end{align*}
    and since $\widebar{f}\colon X\lhd Y\to Z$ is a pointed map, we have
    \begin{align*}
        f([(y_{0},x_{0})]) &= \widebar{f}([(y_{0},x_{0})])\\
                           &= z_{0}.
    \end{align*}
    Thus the elements $f$ in $\Hom^{R}_{\Sets}(X\times Y,Z)$ are precisely those functions $f\colon X\times Y\to Z$ satisfying the equality
    \[
        f(x_{0},y)%
        =%
        z_{0}
    \]%
    for each $y\in Y$, giving an equality
    \[
        \Hom^{R}_{\Sets}(X\times Y,Z)%
        =%
        \Hom^{\otimes,\rmL}_{\Sets_{*}}(X\times Y,Z)%
    \]%
    of sets, which when composed with our earlier isomorphism
    \[
        \Sets_{*}(X\lhd Y,Z)
        \cong
        \Hom^{R}_{\Sets}(X\times Y,Z),%
    \]%
    gives our desired natural bijection, finishing the proof.
\end{Proof}
\begin{notation}{Elements of Left Tensor Products of Pointed Sets}{elements-of-left-tensor-products-of-pointed-sets}%
    We write%
    %--- Begin Footnote ---%
    \footnote{%
        \SloganFont{Further Notation: }Also written \index[notation]{xlefty@$x\lhd_{\Sets_{*}}y$}$x\lhd_{\Sets_{*}}y$.
        \par\vspace*{\TCBBoxCorrection}
    } %
    %---  End Footnote  ---%
    \index[notation]{xlhdy@$x\lhd y$}$x\lhd y$ for the element $[(y,x)]$ of
    \[
        X\lhd Y%
        \cong%
        \abs{Y}\odot X.%
    \]%
\end{notation}
\begin{remark}{Basepoints of Left Tensor Products of Pointed Sets}{basepoints-of-left-tensor-products-of-pointed-sets}%
    Employing the notation introduced in \cref{elements-of-left-tensor-products-of-pointed-sets}, we have
    \[
        x_{0}\lhd y_{0}%
        =%
        x_{0}\lhd y%
    \]%
    for each $y\in Y$, and
    \[
        x_{0}\lhd y%
        =%
        x_{0}\lhd y'%
    \]%
    for each $y,y'\in Y$.
\end{remark}
\begin{proposition}{Properties of Left Tensor Products of Pointed Sets}{properties-of-left-tensor-products-of-pointed-sets}%
    Let $(X,x_{0})$ and $(Y,y_{0})$ be pointed sets.
    \begin{enumerate}
        \item\label{properties-of-left-tensor-products-of-pointed-sets-functoriality}\SloganFont{Functoriality. }The assignments $X,Y,(X,Y)\mapsto X\lhd Y$ define functors
            \[
                \BifunctorialityPeriod{X\lhd-}{-\lhd Y}{-_{1}\lhd-_{2}}{\Sets_{*}}{\Sets_{*}}{\Sets_{*}\times\Sets_{*}}{\Sets_{*}}%
            \]%
            In particular, given pointed maps
            \begin{align*}
                f &\colon (X,x_{0}) \to (A,a_{0}),\\
                g &\colon (Y,y_{0}) \to (B,b_{0}),
            \end{align*}
            the induced map
            \[
                f\lhd g%
                \colon%
                X\lhd Y%
                \to%
                A\lhd B%
            \]%
            is given by
            \[
                [f\lhd g](x\lhd y)%
                \defeq%
                f(x)\lhd g(y)%
            \]%
            for each $x\lhd y\in X\lhd Y$.
        \item\label{properties-of-left-tensor-products-of-pointed-sets-adjointness-1}\SloganFont{Adjointness \rmI. }We have an adjunction%
            \begin{webcompile}
                \Adjunction#-\lhd Y#[Y,-]^{\lhd}_{\Sets_{*}}#\Sets_{*}#\Sets_{*},#
            \end{webcompile}
            witnessed by a bijection of sets
            \[
                \Hom_{\Sets_{*}}(X\lhd Y,Z)%
                \cong
                \Hom_{\Sets_{*}}(X,[Y,Z]^{\lhd}_{\Sets_{*}})%
            \]%
            natural in $(X,x_{0}),(Y,y_{0}),(Z,z_{0})\in\Obj(\Sets_{*})$, where $[X,Y]^{\lhd}_{\Sets_{*}}$ is the pointed set of \cref{the-left-internal-hom-of-pointed-sets}.
        \item\label{properties-of-left-tensor-products-of-pointed-sets-adjointness-2}\SloganFont{Adjointness \rmII. }The functor
            \[
                X\lhd-%
                \colon%
                \Sets_{*}%
                \to%
                \Sets_{*}%
            \]%
            does not admit a right adjoint.%
        \item\label{properties-of-left-tensor-products-of-pointed-sets-adjointness-3}\SloganFont{Adjointness \rmIII. }We have a $\Wasureru$-relative adjunction
            \begin{webcompile}
                \RelativeAdjunction#\Wasureru#X\lhd -#\Sets_{*}(X,-)#\Sets_{*}#\Sets_{*},#
            \end{webcompile}
            witnessed by a bijection of sets
            \[
                \Hom_{\Sets_{*}}(X\lhd Y,Z)%
                \cong
                \Hom_{\Sets}(|Y|,\Sets_{*}(X,Z))%
            \]%
            natural in $(X,x_{0}),(Y,y_{0}),(Z,z_{0})\in\Obj(\Sets_{*})$.
        %\item\label{properties-of-left-tensor-products-of-pointed-sets-}\SloganFont{. }%
    \end{enumerate}
\end{proposition}
\begin{Proof}{Proof of \cref{properties-of-left-tensor-products-of-pointed-sets}}%
    \FirstProofBox{\cref{properties-of-left-tensor-products-of-pointed-sets-functoriality}: Functoriality}%
    This follows from the definition of $\lhd$ as a composition of functors (\cref{the-left-tensor-product-of-pointed-sets}).

    \ProofBox{\cref{properties-of-left-tensor-products-of-pointed-sets-adjointness-1}: Adjointness \rmI}%
    This follows from \cref{properties-of-tensors-of-pointed-sets-by-sets-adjointness-2} of \cref{properties-of-tensors-of-pointed-sets-by-sets}.

    \ProofBox{\cref{properties-of-left-tensor-products-of-pointed-sets-adjointness-2}: Adjointness \rmII}%
    For $X\lhd -$ to admit a right adjoint would require it to preserve colimits by \ChapterRef{\ChapterAdjunctionsAndTheYonedaLemma, \cref{adjunctions-and-the-yoneda-lemma:properties-of-adjunctions-interaction-with-co-limits} of \cref{adjunctions-and-the-yoneda-lemma:properties-of-adjunctions}}{\cref{properties-of-adjunctions-interaction-with-co-limits} of \cref{properties-of-adjunctions}}. However, we have
    \begin{align*}
        X\lhd\pt &\defeq |\pt|\odot X\\
                 &\cong  X\\
                 &\ncong \pt,
    \end{align*}
    and thus we see that $X\lhd -$ does not have a right adjoint.

    \ProofBox{\cref{properties-of-left-tensor-products-of-pointed-sets-adjointness-3}: Adjointness \rmIII}%
    This follows from \cref{properties-of-tensors-of-pointed-sets-by-sets-adjointness-1} of \cref{properties-of-tensors-of-pointed-sets-by-sets}.
\end{Proof}
\begin{remark}{On the Failure of $X\lhd-$ To Be a Left Adjoint}{on-the-failure-of-xlhd-to-be-a-left-adjoint}%
    Here is some intuition on why $X\lhd-$ fails to be a left adjoint. \cref{properties-of-left-tensor-products-of-pointed-sets-adjointness-3} of \cref{properties-of-left-tensor-products-of-pointed-sets} states that we have a natural bijection
    \[
        \Hom_{\Sets_{*}}(X\lhd Y,Z)%
        \cong
        \Hom_{\Sets}(|Y|,\Sets_{*}(X,Z)),%
    \]%
    so it would be reasonable to wonder whether a natural bijection of the form
   \[
        \Hom_{\Sets_{*}}(X\lhd Y,Z)%
        \cong
        \Hom_{\Sets_{*}}(Y,\eSets_{*}(X,Z)),%
    \]%
    also holds, which would give $X\lhd-\dashv\eSets_{*}(X,-)$. However, such a bijection would require every map
    \[
        f%
        \colon%
        X\lhd Y%
        \to%
        Z%
    \]%
    to satisfy
    \[
        f(x\lhd y_{0})%
        =%
        z_{0}%
    \]%
    for each $x\in X$, whereas we are imposing such a basepoint preservation condition only for elements of the form $x_{0}\lhd y$. Thus $\eSets_{*}(X,-)$ can't be a right adjoint for $X\lhd-$, and as shown by \cref{properties-of-left-tensor-products-of-pointed-sets-adjointness-2} of \cref{properties-of-left-tensor-products-of-pointed-sets}, no functor can.%
    %--- Begin Footnote ---%
    \footnote{%
        The functor $\eSets_{*}(X,-)$ is instead right adjoint to $X\wedge-$, the smash product of pointed sets of \cref{smash-products-of-pointed-sets}. See \cref{properties-of-smash-products-of-pointed-sets-adjointness} of \cref{properties-of-smash-products-of-pointed-sets}.
        \par\vspace*{\TCBBoxCorrection}
    }%
    %---  End Footnote  ---%
\end{remark}
\subsection{The Left Internal Hom of Pointed Sets}\label{subsection-the-left-internal-hom-of-pointed-sets}
Let $(X,x_{0})$ and $(Y,y_{0})$ be pointed sets.
\begin{definition}{The Left Internal Hom of Pointed Sets}{the-left-internal-hom-of-pointed-sets}%
    The \index[set-theory]{pointed set!left internal Hom of}\textbf{left internal Hom}%
    %--- Begin Footnote ---%
    \footnote{%
        For a proof that $[-,-]^{\lhd}_{\Sets_{*}}$ is indeed the left internal Hom of $\Sets_{*}$ with respect to the left tensor product of pointed sets, see \cref{properties-of-left-tensor-products-of-pointed-sets-adjointness-1} of \cref{properties-of-left-tensor-products-of-pointed-sets}.
        \par\vspace*{\TCBBoxCorrection}
    } %
    %---  End Footnote  ---%
    \textbf{of pointed sets} is the functor\index[notation]{lhdSetsstar@$[-,-]^{\lhd}_{\Sets_{*}}$}%
    \[
        [-,-]^{\lhd}_{\Sets_{*}}%
        \colon%
        \Sets^{\op}_{*}\times\Sets_{*}%
        \to%
        \Sets_{*}%
    \]%
    defined as the composition
    \[
        \Sets^{\op}_{*}\times\Sets_{*}%
        \xlongrightarrow{\Wasureru\times\sfid}%
        \Sets^{\op}\times\Sets_{*}%
        \xlongrightarrow{\pitchfork}%
        \Sets_{*},%
    \]%
    where:
    \begin{itemize}
        \item $\Wasureru\colon\Sets_{*}\to\Sets$ is the forgetful functor from pointed sets to sets.
        \item $\pitchfork\colon\Sets^{\op}\times\Sets_{*}\to\Sets_{*}$ is the cotensor functor of \cref{properties-of-cotensors-of-pointed-sets-by-sets-functoriality} of \cref{properties-of-cotensors-of-pointed-sets-by-sets}.%
    \end{itemize}
\end{definition}
\begin{remark}{Unwinding \cref{the-left-internal-hom-of-pointed-sets}, \rmI: Universal Property}{unwinding-the-left-internal-hom-of-pointed-sets-1-universal-property}%
    The left internal Hom of pointed sets satisfies the following universal property:%
    \[%
        \Sets_{*}(X\lhd Y,Z)%
        \cong%
        \Sets_{*}(X,[Y,Z]^{\lhd}_{\Sets_{*}})%
    \]%
    That is to say, the following data are in bijection:
    \begin{enumerate}
        \item\label{unwinding-the-left-internal-hom-of-pointed-sets-1-universal-property-item-1}Pointed maps $f\colon X\lhd Y\to Z$.
        \item\label{unwinding-the-left-internal-hom-of-pointed-sets-1-universal-property-item-2}Pointed maps $f\colon X\to[Y,Z]^{\lhd}_{\Sets_{*}}$.
    \end{enumerate}
\end{remark}
\begin{remark}{Unwinding \cref{the-left-internal-hom-of-pointed-sets}, \rmII: Explicit Description}{unwinding-the-left-internal-hom-of-pointed-sets-1-explicit-description}%
    In detail, the \textbf{left internal Hom of $(X,x_{0})$ and $(Y,y_{0})$} is the pointed set $\smash{([X,Y]^{\lhd}_{\Sets_{*}},[(y_{0})_{x\in X}])}$ consisting of:%
    \begin{itemize}
        \item\SloganFont{The Underlying Set. }The set $[X,Y]^{\lhd}_{\Sets_{*}}$ defined by
            \begin{align*}
                [X,Y]^{\lhd}_{\Sets_{*}} &\defeq \abs{X}\pitchfork Y\\
                                         &\cong  \bigwedge_{x\in X}(Y,y_{0}),
            \end{align*}
            where $\abs{X}$ denotes the underlying set of $(X,x_{0})$.
        \item\SloganFont{The Underlying Basepoint. }The point $[(y_{0})_{x\in X}]$ of $\bigwedge_{x\in X}(Y,y_{0})$.
    \end{itemize}
\end{remark}
\begin{proposition}{Properties of Left Internal Homs of Pointed Sets}{properties-of-left-internal-homs-of-pointed-sets}%
    Let $(X,x_{0})$ and $(Y,y_{0})$ be pointed sets.
    \begin{enumerate}
        \item\label{properties-of-left-internal-homs-of-pointed-sets-functoriality}\SloganFont{Functoriality. }The assignments $X,Y,(X,Y)\mapsto[X,Y]^{\lhd}_{\Sets_{*}}$ define functors
            \[
                \BifunctorialityPeriod{[X,-]^{\lhd}_{\Sets_{*}}}{{[-,Y]^{\lhd}_{\Sets_{*}}}}{{[-_{1},-_{2}]^{\lhd}_{\Sets_{*}}}}{\Sets_{*}}{\Sets^{\mrp{\op}}_{*}}{\Sets^{\op}_{*}\times\Sets_{*}}{\Sets_{*}}%
            \]%
            In particular, given pointed maps
            \begin{align*}
                f &\colon (X,x_{0}) \to (A,a_{0}),\\
                g &\colon (Y,y_{0}) \to (B,b_{0}),
            \end{align*}
            the induced map
            \[
                [f,g]^{\lhd}_{\Sets_{*}}%
                \colon%
                [A,Y]^{\lhd}_{\Sets_{*}}%
                \to%
                [X,B]^{\lhd}_{\Sets_{*}}%
            \]%
            is given by
            \[
                [f,g]^{\lhd}_{\Sets_{*}}([(y_{a})_{a\in A}])%
                \defeq%
                [(g(y_{f(x)}))_{x\in X}]%
            \]%
            for each $[(y_{a})_{a\in A}]\in[A,Y]^{\lhd}_{\Sets_{*}}$.
        \item\label{properties-of-left-internal-homs-of-pointed-sets-adjointness-1}\SloganFont{Adjointness \rmI. }We have an adjunction%
            \begin{webcompile}
                \Adjunction#-\lhd Y#[Y,-]^{\lhd}_{\Sets_{*}}#\Sets_{*}#\Sets_{*},#
            \end{webcompile}
            witnessed by a bijection of sets
            \[
                \Hom_{\Sets_{*}}(X\lhd Y,Z)%
                \cong
                \Hom_{\Sets_{*}}(X,[Y,Z]^{\lhd}_{\Sets_{*}})%
            \]%
            natural in $(X,x_{0}),(Y,y_{0}),(Z,z_{0})\in\Obj(\Sets_{*})$
        \item\label{properties-of-left-internal-homs-of-pointed-sets-adjointness-2}\SloganFont{Adjointness \rmII. }The functor
            \[
                X\lhd-%
                \colon%
                \Sets_{*}%
                \to%
                \Sets_{*}%
            \]%
            does not admit a right adjoint.%
        %\item\label{properties-of-left-internal-homs-of-pointed-sets-}\SloganFont{. }%
    \end{enumerate}
\end{proposition}
\begin{Proof}{Proof of \cref{properties-of-left-internal-homs-of-pointed-sets}}%
    \FirstProofBox{\cref{properties-of-left-internal-homs-of-pointed-sets-functoriality}: Functoriality}%
    This follows from the definition of $[-,-]^{\lhd}_{\Sets_{*}}$ as a composition of functors (\cref{the-left-internal-hom-of-pointed-sets}).

    \ProofBox{\cref{properties-of-left-internal-homs-of-pointed-sets-adjointness-1}: Adjointness \rmI}%
    This is a repetition of \cref{properties-of-left-tensor-products-of-pointed-sets-adjointness-1} of \cref{properties-of-left-tensor-products-of-pointed-sets}, and is proved there.

    \ProofBox{\cref{properties-of-left-internal-homs-of-pointed-sets-adjointness-2}: Adjointness \rmII}%
    This is a repetition of \cref{properties-of-left-tensor-products-of-pointed-sets-adjointness-2} of \cref{properties-of-left-tensor-products-of-pointed-sets}, and is proved there.
\end{Proof}
\subsection{The Left Skew Unit}\label{subsection-the-left-skew-unit-of-the-left-tensor-product-of-pointed-sets}
\begin{definition}{The Left Skew Unit of $\lhd$}{the-left-skew-unit-of-the-left-tensor-product-of-pointed-sets}%
    The \index[set-theory]{left tensor product of pointed sets!left skew unit of}\textbf{left skew unit of the left tensor product of pointed sets} is the functor
    \[
        \Unit^{\Sets_{*},\lhd}
        \colon
        \PunctualCategory
        \to
        \Sets_{*}
    \]
    defined by
    \[
        \Unit^{\lhd}_{\Sets_{*}}%
        \defeq%
        S^{0}.
    \]%
\end{definition}
\subsection{The Left Skew Associator}\label{subsection-the-left-tensor-product-of-pointed-sets-the-left-skew-associator}
\begin{definition}{The Left Skew Associator of $\lhd$}{the-left-skew-associator-of-the-left-tensor-product-of-pointed-sets}%
    The \index[set-theory]{left tensor product of pointed sets!skew associator}\textbf{skew associator of the left tensor product of pointed sets} is the natural transformation
    \[
        \alpha^{\Sets_{*},\lhd}%
        \colon%
        {\lhd}\circ{({\lhd}\times\id_{\Sets_{*}})}%
        \Longrightarrow%
        {\lhd}\circ{(\id_{\Sets_{*}}\times{\lhd})}\circ{\bfalpha^{\Cats}_{\Sets_{*},\Sets_{*},\Sets_{*}}}%
    \]
    as in the diagram
    \[
        \begin{tikzcd}[row sep={0*\the\DL,between origins}, column sep={0*\the\DL,between origins}, background color=backgroundColor, ampersand replacement=\&]
            \&[0.30901699437\TwoCmPlusHalf]
            \&[0.5\TwoCmPlusHalf]
            \Sets_{*}\times(\Sets_{*}\times\Sets_{*})
            \&[0.5\TwoCmPlusHalf]
            \&[0.30901699437\TwoCmPlusHalf]
            \\[0.58778525229\TwoCmPlusHalf]
            (\Sets_{*}\times\Sets_{*})\times\Sets_{*}
            \&[0.30901699437\TwoCmPlusHalf]
            \&[0.5\TwoCmPlusHalf]
            \&[0.5\TwoCmPlusHalf]
            \&[0.30901699437\TwoCmPlusHalf]
            \Sets_{*}\times\Sets_{*}
            \\[0.95105651629\TwoCmPlusHalf]
            \&[0.30901699437\TwoCmPlusHalf]
            \Sets_{*}\times\Sets_{*}
            \&[0.5\TwoCmPlusHalf]
            \&[0.5\TwoCmPlusHalf]
            \Sets_{*}\mrp{,}
            \&[0.30901699437\TwoCmPlusHalf]
            % 1-Arrows
            % Left Boundary
            \arrow[from=2-1,to=1-3,"\bfalpha^{\Cats}_{\Sets_{*},\Sets_{*},\Sets_{*}}"{pos=0.3},isoarrowprime]%
            \arrow[from=1-3,to=2-5,"{\sfid\times{\lhd}}"{pos=0.6},""{name=2}]%
            \arrow[from=2-5,to=3-4,"\lhd"{pos=0.425}]%
            % Right Boundary
            \arrow[from=2-1,to=3-2,"{{\lhd}\times\sfid}"'{pos=0.425}]%
            \arrow[from=3-2,to=3-4,"\lhd"']%
            % 2-Arrows
            \arrow[from=3-2,to=2,"\alpha^{\Sets_{*},\lhd}"{description,pos=0.5},Rightarrow,shorten <= 0.5*\the\DL,shorten >= 1*\the\DL]%
        \end{tikzcd}
    \]%
    whose component
    \[
        \alpha^{\Sets_{*},\lhd}_{X,Y,Z}
        \colon
        (X\lhd Y)\lhd Z
        \to
        X\lhd(Y\lhd Z)
    \]%
    at $(X,x_{0}),(Y,y_{0}),(Z,z_{0})\in\Obj(\Sets_{*})$ is given by
    \begin{align*}
        (X\lhd Y)\lhd Z &\defeq |Z|\odot(X\lhd Y)\\
                        &\defeq |Z|\odot(|Y|\odot X)\\
                        &\cong  \bigvee_{z\in Z}|Y|\odot X\\
                        &\cong  \bigvee_{z\in Z}(\bigvee_{y\in Y}X)\\
                        &\to    \bigvee_{[(z,y)]\in\bigvee_{z\in Z}Y}X\\
                        &\cong  \bigvee_{[(z,y)]\in|Z|\odot Y}X\\
                        &\cong  ||Z|\odot Y|\odot X\\
                        &\defeq |Y\lhd Z|\odot X\\
                        &\defeq X\lhd(Y\lhd Z),%
    \end{align*}
    where the map%
    \[
        \bigvee_{z\in Z}(\bigvee_{y\in Y}X)%
        \to%
        \bigvee_{(z,y)\in\bigvee_{z\in Z}Y}X%
    \]%
    is given by $[(z,[(y,x)])]\mapsto[([(z,y)],x)]$.
\end{definition}
\begin{Proof}{Proof of \cref{the-left-skew-associator-of-the-left-tensor-product-of-pointed-sets}}%
    (Proven below in a bit.)
\end{Proof}
\begin{remark}{Unwinding \cref{the-left-skew-associator-of-the-left-tensor-product-of-pointed-sets}}{unwinding-the-left-skew-associator-of-the-left-tensor-product-of-pointed-sets}%
    Unwinding the notation for elements, we have
    \begin{align*}
        [(z,[(y,x)])] &\defeq [(z,x\lhd y)]\\
                      &\defeq (x\lhd y)\lhd z
    \end{align*}
    and
    \begin{align*}
        [([(z,y)],x)] &\defeq [(y\lhd z,x)]\\
                      &\defeq x\lhd(y\lhd z).
    \end{align*}
    So, in other words, $\alpha^{\Sets_{*},\lhd}_{X,Y,Z}$ acts on elements via
    \[
        \alpha^{\Sets_{*},\lhd}_{X,Y,Z}((x\lhd y)\lhd z)
        \defeq
        x\lhd(y\lhd z)
    \]%
    for each $(x\lhd y)\lhd z\in(X\lhd Y)\lhd Z$.
\end{remark}
\begin{remark}{Non-Invertibility of the Skew Associator of $\lhd$}{non-invertibility-of-the-left-skew-associator-of-the-left-tensor-product-of-pointed-sets}%
    Taking $y=y_{0}$, we see that the morphism $\smash{\alpha^{\Sets_{*},\lhd}_{X,Y,Z}}$ acts on elements as
    \[
        \alpha^{\Sets_{*},\lhd}_{X,Y,Z}((x\lhd y_{0})\lhd z)%
        \defeq%
        x\lhd(y_{0}\lhd z).%
    \]%
    However, by the definition of $\lhd$, we have $y_{0}\lhd z=y_{0}\lhd z'$ for all $z,z'\in Z$, preventing $\alpha^{\Sets_{*},\lhd}_{X,Y,Z}$ from being non-invertible.
\end{remark}
\begin{Proof}{Proof of \cref{the-left-skew-associator-of-the-left-tensor-product-of-pointed-sets}}%
    Firstly, note that, given $(X,x_{0}),(Y,y_{0}),(Z,z_{0})\in\Obj(\Sets_{*})$, the map
    \[
        \alpha^{\Sets_{*},\lhd}_{X,Y,Z}
        \colon
        (X\lhd Y)\lhd Z
        \to%
        X\lhd(Y\lhd Z)
    \]%
    is indeed a morphism of pointed sets, as we have
    \[
        \alpha^{\Sets_{*},\lhd}_{X,Y,Z}((x_{0}\lhd y_{0})\lhd z_{0})%
        =%
        x_{0}\lhd(y_{0}\lhd z_{0}).%
    \]%
    Next, we claim that $\alpha^{\Sets_{*},\lhd}$ is a natural transformation. We need to show that, given morphisms of pointed sets
    \begin{align*}
        f &\colon (X,x_{0}) \to (X',x'_{0}),\\
        g &\colon (Y,y_{0}) \to (Y',y'_{0}),\\
        h &\colon (Z,z_{0}) \to (Z',z'_{0})
    \end{align*}
    the diagram
    \[
        \begin{tikzcd}[row sep={5.0*\the\DL,between origins}, column sep={11.0*\the\DL,between origins}, background color=backgroundColor, ampersand replacement=\&]
            (X\lhd Y)\lhd Z
            \arrow[r,"{(f\lhd g)\lhd h}"]
            \arrow[d,"\alpha^{\Sets_{*},\lhd}_{X,Y,Z}"']
            \&
            (X'\lhd Y')\lhd Z'
            \arrow[d,"\alpha^{\Sets_{*},\lhd}_{X',Y',Z'}"]
            \\
            X\lhd(Y\lhd Z)
            \arrow[r,"{f\lhd(g\lhd h)}"']
            \&
            X'\lhd(Y'\lhd Z')
        \end{tikzcd}
    \]%
    commutes. Indeed, this diagram acts on elements as
    \[
        \begin{tikzcd}[row sep={5.0*\the\DL,between origins}, column sep={11.0*\the\DL,between origins}, background color=backgroundColor, ampersand replacement=\&]
            (x\lhd y)\lhd z
            \arrow[r,mapsto]
            \arrow[d,mapsto]
            \&
            (f(x)\lhd g(y))\lhd h(z)
            \arrow[d,mapsto]
            \\
            x\lhd(y\lhd z)
            \arrow[r,mapsto]
            \&
            f(x)\lhd(g(y)\lhd h(z))
        \end{tikzcd}
    \]%
    and hence indeed commutes, showing $\alpha^{\Sets_{*},\lhd}$ to be a natural transformation. This finishes the proof.
\end{Proof}
\subsection{The Left Skew Left Unitor}\label{subsection-the-left-tensor-product-of-pointed-sets-the-left-skew-left-unitor}
\begin{definition}{The Left Skew Left Unitor of $\lhd$}{the-left-skew-left-unitor-of-the-left-tensor-product-of-pointed-sets}%
    The \index[set-theory]{left tensor product of pointed sets!skew left unitor}\textbf{skew left unitor of the left tensor product of pointed sets} is the natural transformation
    \begin{webcompile}
        \LUnitor^{\Sets_{*},\lhd}%
        \colon%
        {\lhd}\circ{(\Unit^{\Sets_{*}}\times\id_{\Sets_{*}})}%
        \Longrightisoarrow%
        \bfLUnitor^{\TwoCategoryOfCategories}_{\Sets_{*}}%
        \begin{tikzcd}[row sep={10.0*\the\DL,between origins}, column sep={10.0*\the\DL,between origins}, background color=backgroundColor, ampersand replacement=\&]
            \PunctualCategory\times\Sets_{*}
            \arrow[r, "\Unit^{\Sets_{*}}\times\sfid"]
            \arrow[rd, dashed,"\bfLUnitor^{\TwoCategoryOfCategories}_{\Sets_{*}}"'{name=1,pos=0.475},bend right=30]
            \&
            \Sets_{*}\times\Sets_{*}
            \arrow[d, "\lhd"]
            \\
            {}
            \&
            \Sets_{*}\mathrlap{,}
            % 2-Arrows
            \arrow[Rightarrow,from=1-2,to=1,shorten >=1.0*\the\DL,shorten <=1.0*\the\DL,"\LUnitor^{\Sets_{*},\lhd}"description]
        \end{tikzcd}
    \end{webcompile}%
    whose component
    \[
        \LUnitor^{\Sets_{*},\lhd}_{X}
        \colon
        S^{0}\lhd X
        \to
        X
    \]%
    at $(X,x_{0})\in\Obj(\Sets_{*})$ is given by the composition%
    \begin{align*}
        S^{0}\lhd X &\cong            |X|\odot S^{0}\\
                    &\cong            \bigvee_{x\in X}S^{0}\\
                    &\shortrightarrow X,
    \end{align*}
    where $\bigvee_{x\in X}S^{0}\to X$ is the map given by
    \begin{align*}
        [(x,0)] &\mapsto x_{0},\\
        [(x,1)] &\mapsto x
    \end{align*}
    for each $x\in X$.
\end{definition}
\begin{Proof}{Proof of \cref{the-left-skew-left-unitor-of-the-left-tensor-product-of-pointed-sets}}%
    (Proven below in a bit.)
\end{Proof}
\begin{remark}{Unwinding \cref{the-left-skew-left-unitor-of-the-left-tensor-product-of-pointed-sets}}{unwinding-the-left-skew-left-unitor-of-the-left-tensor-product-of-pointed-sets}%
    In other words, $\smash{\LUnitor^{\Sets_{*},\lhd}_{X}}$ acts on elements as
    \begin{align*}
        \LUnitor^{\Sets_{*},\lhd}_{X}(0\lhd x) &\defeq x_{0},\\%
        \LUnitor^{\Sets_{*},\lhd}_{X}(1\lhd x) &\defeq x%
    \end{align*}
    for each $1\lhd x\in S^{0}\lhd X$.
\end{remark}
\begin{remark}{Non-Invertibility of the Skew Left Unitor of $\lhd$}{non-invertibility-of-the-left-skew-left-unitor-of-the-left-tensor-product-of-pointed-sets}%
    The morphism $\smash{\LUnitor^{\Sets_{*},\lhd}_{X}}$ is almost invertible, with its would-be-inverse
    \[
        \phi_{X}%
        \colon%
        X%
        \to%
        S^{0}\lhd X%
    \]%
    given by
    \[
        \phi_{X}(x)%
        \defeq%
        1\lhd x
    \]%
    for each $x\in X$. Indeed, we have
    \begin{align*}
        [\LUnitor^{\Sets_{*},\lhd}_{X}\circ\phi](x) &= \LUnitor^{\Sets_{*},\lhd}_{X}(\phi(x))\\
                                                    &= \LUnitor^{\Sets_{*},\lhd}_{X}(1\lhd x)\\
                                                    &= x\\
                                                    &= [\id_{X}](x)
    \end{align*}
    so that
    \[
        \LUnitor^{\Sets_{*},\lhd}_{X}\circ\phi%
        =%
        \id_{X}%
    \]%
    and
    \begin{align*}
        [\phi\circ\LUnitor^{\Sets_{*},\lhd}_{X}](1\lhd x) &= \phi(\LUnitor^{\Sets_{*},\lhd}_{X}(1\lhd x))\\
                                                          &= \phi(x)\\
                                                          &= 1\lhd x\\
                                                          &= [\id_{S^{0}\lhd X}](1\lhd x),
    \end{align*}
    but
    \begin{align*}
        [\phi\circ\LUnitor^{\Sets_{*},\lhd}_{X}](0\lhd x) &= \phi(\LUnitor^{\Sets_{*},\lhd}_{X}(0\lhd x))\\
                                                          &= \phi(x_{0})\\
                                                          &= 1\lhd x_{0},
    \end{align*}
    where $0\lhd x\neq1\lhd x_{0}$. Thus
    \[
        \phi\circ\LUnitor^{\Sets_{*},\lhd}_{X}%
        \eqquestion%
        \id_{S^{0}\lhd X}%
    \]%
    holds for all elements in $S^{0}\lhd X$ except one.
\end{remark}
\begin{Proof}{Proof of \cref{the-left-skew-left-unitor-of-the-left-tensor-product-of-pointed-sets}}%
    Firstly, note that, given $(X,x_{0})\in\Obj(\Sets_{*})$, the map
    \[
        \LUnitor^{\Sets_{*},\lhd}_{X}
        \colon
        S^{0}\lhd X
        \to
        X
    \]%
    is indeed a morphism of pointed sets, as we have
    \[
        \LUnitor^{\Sets_{*},\lhd}_{X}(0\lhd x_{0})%
        =%
        x_{0}.%
    \]%
    Next, we claim that $\LUnitor^{\Sets_{*},\lhd}$ is a natural transformation. We need to show that, given a morphism of pointed sets
    \[
        f%
        \colon%
        (X,x_{0})%
        \to%
        (Y,y_{0}),%
    \]%
    the diagram
    \[
        \begin{tikzcd}[row sep={5.0*\the\DL,between origins}, column sep={7.0*\the\DL,between origins}, background color=backgroundColor, ampersand replacement=\&]
            S^{0}\lhd X
            \arrow[r,"{\id_{S^{0}}\lhd f}"]
            \arrow[d,"\LUnitor^{\Sets_{*},\lhd}_{X}"']
            \&
            S^{0}\lhd Y
            \arrow[d,"\LUnitor^{\Sets_{*},\lhd}_{Y}"]
            \\
            X
            \arrow[r,"f"']
            \&
            Y
        \end{tikzcd}
    \]%
    commutes. Indeed, this diagram acts on elements as
    \begin{webcompile}
        \begin{tikzcd}[row sep={5.0*\the\DL,between origins}, column sep={6.0*\the\DL,between origins}, background color=backgroundColor, ampersand replacement=\&]
            0\lhd x
            \arrow[d,mapsto]
            \&
            \\
            x_{0}
            \arrow[r,mapsto]
            \&
            f(x_{0})
        \end{tikzcd}
        \quad
        \begin{tikzcd}[row sep={5.0*\the\DL,between origins}, column sep={6.0*\the\DL,between origins}, background color=backgroundColor, ampersand replacement=\&]
            0\lhd x
            \arrow[r,mapsto]
            \&
            0\lhd f(x)
            \arrow[d,mapsto]
            \\
            \&
            y_{0}
        \end{tikzcd}
    \end{webcompile}
    and
    \[
        \begin{tikzcd}[row sep={5.0*\the\DL,between origins}, column sep={6.0*\the\DL,between origins}, background color=backgroundColor, ampersand replacement=\&]
            1\lhd x
            \arrow[r,mapsto]
            \arrow[d,mapsto]
            \&
            1\lhd f(x)
            \arrow[d,mapsto]
            \\
            x
            \arrow[r,mapsto]
            \&
            f(x)
        \end{tikzcd}
    \]%
    and hence indeed commutes, showing $\LUnitor^{\Sets_{*},\lhd}$ to be a natural transformation. This finishes the proof.
\end{Proof}
\subsection{The Left Skew Right Unitor}\label{subsection-the-left-tensor-product-of-pointed-sets-the-left-skew-right-unitor}
\begin{definition}{The Left Skew Right Unitor of $\lhd$}{the-left-skew-right-unitor-of-the-left-tensor-product-of-pointed-sets}%
    The \index[set-theory]{left tensor product of pointed sets!skew right unitor}\textbf{skew right unitor of the left tensor product of pointed sets} is the natural transformation
    \begin{webcompile}
        \RUnitor^{\Sets_{*},\lhd}%
        \colon%
        \bfRUnitor^{\TwoCategoryOfCategories}_{\Sets_{*}}%
        \Longrightisoarrow%
        {\lhd}\circ{({\sfid}\times{\Unit^{\Sets_{*}}})},%
        \begin{tikzcd}[row sep={10.0*\the\DL,between origins}, column sep={10.0*\the\DL,between origins}, background color=backgroundColor, ampersand replacement=\&]
            \Sets_{*}\times\PunctualCategory
            \arrow[r, "\sfid\times\Unit^{\Sets_{*}}"]
            \arrow[rd, dashed,"\bfRUnitor^{\TwoCategoryOfCategories}_{\Sets_{*}}"'{name=1,pos=0.475},bend right=30]
            \&
            \Sets_{*}\times\Sets_{*}
            \arrow[d, "\lhd"]
            \\
            {}
            \&
            \Sets_{*}\mathrlap{,}
            % 2-Arrows
            \arrow[Rightarrow,from=1,to=1-2,shorten >=1.0*\the\DL,shorten <=1.0*\the\DL,"\RUnitor^{\Sets_{*},\lhd}"description]
        \end{tikzcd}
    \end{webcompile}%
    whose component
    \[
        \RUnitor^{\Sets_{*},\lhd}_{X}
        \colon
        X
        \to
        X\lhd S^{0}%
    \]%
    at $(X,x_{0})\in\Obj(\Sets_{*})$ is given by the composition%
    \begin{align*}
        X &\shortrightarrow X\vee X\\
          &\cong            |S^{0}|\odot X\\
          &\cong            X\lhd S^{0},
    \end{align*}
    where $X\to X\vee X$ is the map sending $X$ to the second factor of $X$ in $X\vee X$.
\end{definition}
\begin{Proof}{Proof of \cref{the-left-skew-right-unitor-of-the-left-tensor-product-of-pointed-sets}}%
    (Proven below in a bit.)
\end{Proof}
\begin{remark}{Unwinding \cref{the-left-skew-right-unitor-of-the-left-tensor-product-of-pointed-sets}}{unwinding-the-left-skew-right-unitor-of-the-left-tensor-product-of-pointed-sets}%
    In other words, $\smash{\RUnitor^{\Sets_{*},\lhd}_{X}}$ acts on elements as
    \[%
        \RUnitor^{\Sets_{*},\lhd}_{X}(x)%
        \defeq%
        [(1,x)]%
    \]%
    i.e.\ by
    \[%
        \RUnitor^{\Sets_{*},\lhd}_{X}(x)%
        \defeq%
        x\lhd 1%
    \]%
    for each $x\in X$.
\end{remark}
\begin{remark}{Non-Invertibility of the Skew Right Unitor of $\lhd$}{non-invertibility-of-the-left-skew-right-unitor-of-the-left-tensor-product-of-pointed-sets}%
    The morphism $\smash{\RUnitor^{\Sets_{*},\lhd}_{X}}$ is non-invertible, as it is non-surjective when viewed as a map of sets, since the elements $x\lhd 0$ of $X\lhd S^{0}$ with $x\neq x_{0}$ are outside the image of $\RUnitor^{\Sets_{*},\lhd}_{X}$, which sends $x$ to $x\lhd 1$.
\end{remark}
\begin{Proof}{Proof of \cref{the-left-skew-right-unitor-of-the-left-tensor-product-of-pointed-sets}}%
    Firstly, note that, given $(X,x_{0})\in\Obj(\Sets_{*})$, the map
    \[
        \RUnitor^{\Sets_{*},\lhd}_{X}
        \colon
        X
        \to
        X\lhd S^{0}%
    \]%
    is indeed a morphism of pointed sets as we have
    \begin{align*}
        \RUnitor^{\Sets_{*},\lhd}_{X}(x_{0}) &= x_{0}\lhd1\\%
                                             &= x_{0}\lhd0.%
    \end{align*}
    Next, we claim that $\RUnitor^{\Sets_{*},\lhd}$ is a natural transformation. We need to show that, given a morphism of pointed sets
    \[
        f%
        \colon%
        (X,x_{0})%
        \to%
        (Y,y_{0}),%
    \]%
    the diagram
    \[
        \begin{tikzcd}[row sep={5.0*\the\DL,between origins}, column sep={7.5*\the\DL,between origins}, background color=backgroundColor, ampersand replacement=\&]
            X
            \arrow[r,"f"]
            \arrow[d,"\RUnitor^{\Sets_{*},\lhd}_{X}"']
            \&
            Y
            \arrow[d,"\RUnitor^{\Sets_{*},\lhd}_{Y}"]
            \\
            X\lhd S^{0}
            \arrow[r,"{f\lhd\id_{S^{0}}}"']
            \&
            Y\lhd S^{0}
        \end{tikzcd}
    \]%
    commutes. Indeed, this diagram acts on elements as
    \[
        \begin{tikzcd}[row sep={5.0*\the\DL,between origins}, column sep={6.0*\the\DL,between origins}, background color=backgroundColor, ampersand replacement=\&]
            x
            \arrow[r,mapsto]
            \arrow[d,mapsto]
            \&
            f(x)
            \arrow[d,mapsto]
            \\
            x\lhd 0
            \arrow[r,mapsto]
            \&
            f(x)\lhd0
        \end{tikzcd}
    \]%
    and hence indeed commutes, showing $\RUnitor^{\Sets_{*},\lhd}$ to be a natural transformation. This finishes the proof.
\end{Proof}
\subsection{The Diagonal}\label{subsection-the-left-tensor-product-of-pointed-sets-the-diagonal}
\begin{definition}{The Diagonal of $\lhd$}{the-diagonal-of-the-left-tensor-product-of-pointed-sets}%
    The \index[set-theory]{left tensor product of pointed sets!diagonal}\textbf{diagonal of the left tensor product of pointed sets} is the natural transformation
    \begin{webcompile}
        \Delta^{\lhd}%
        \colon%
        \id_{\Sets_{*}}%
        \Longrightarrow%
        {\lhd}\circ{\Delta^{\TwoCategoryOfCategories}_{\Sets_{*}}},%
        \qquad%
        \begin{tikzcd}[row sep={5.0*\the\DL,between origins}, column sep={4.0*\the\DL,between origins}, background color=backgroundColor, ampersand replacement=\&]
            \Sets_{*}
            \arrow[rr,"\id_{\Sets_{*}}"{name=1},bend left=10]
            \arrow[rd,"\Delta^{\TwoCategoryOfCategories}_{\Sets_{*}}"'{pos=0.3},bend right=10]
            \&
            \&
            \Sets_{*}
            \\
            \&
            \Sets_{*}\times\Sets_{*}\mrp{,}
            \arrow[ru,"\lhd"'{pos=0.55},bend right=10]
            \&
            % 2-Arrows
            \arrow[from=1,to=2-2,"\Delta^{\lhd}"description,shorten <= 0.5*\the\DL,shorten >= 0.25*\the\DL,Rightarrow]%
        \end{tikzcd}
    \end{webcompile}%
    whose component
    \[
        \Delta^{\lhd}_{X}%
        \colon%
        (X,x_{0})%
        \to%
        (X\lhd X,x_{0}\lhd x_{0})%
    \]%
    at $(X,x_{0})\in\Obj(\Sets_{*})$ is given by
    \[
        \Delta^{\lhd}_{X}(x)%
        \defeq%
        x\lhd x%
    \]%
    for each $x\in X$.
\end{definition}
\begin{Proof}{Proof of \cref{the-diagonal-of-the-left-tensor-product-of-pointed-sets}}%
    \ProofBox{Being a Morphism of Pointed Sets}%
    We have
    \[
        \Delta^{\lhd}_{X}(x_{0})%
        \defeq%
        x_{0}\lhd x_{0},%
    \]%
    and thus $\Delta^{\lhd}_{X}$ is a morphism of pointed sets.

    \ProofBox{Naturality}%
    We need to show that, given a morphism of pointed sets
    \[
        f%
        \colon%
        (X,x_{0})%
        \to%
        (Y,y_{0}),%
    \]%
    the diagram
    \[
        \begin{tikzcd}[row sep={4.5*\the\DL,between origins}, column sep={6.5*\the\DL,between origins}, background color=backgroundColor, ampersand replacement=\&]
            X
            \arrow[r,"f"]
            \arrow[d,"\Delta^{\lhd}_{X}"']
            \&
            Y
            \arrow[d,"\Delta^{\lhd}_{Y}"]
            \\
            X\lhd X
            \arrow[r,"f\lhd f"']
            \&
            Y\lhd Y
        \end{tikzcd}
    \]%
    commutes. Indeed, this diagram acts on elements as
    \[
        \begin{tikzcd}[row sep={5.0*\the\DL,between origins}, column sep={7.0*\the\DL,between origins}, background color=backgroundColor, ampersand replacement=\&]
            x
            \arrow[r,mapsto]
            \arrow[d,mapsto]
            \&
            f(x)
            \arrow[d,mapsto]
            \\
            x\lhd x
            \arrow[r,mapsto]
            \&
            f(x)\lhd f(x)
        \end{tikzcd}
    \]%
    and hence indeed commutes, showing $\Delta^{\lhd}$ to be natural.
\end{Proof}
\subsection{The Left Skew Monoidal Structure on Pointed Sets Associated to $\lhd$}\label{subsection-the-left-skew-monoidal-structure-on-pointed-sets-associated-to-the-left-tensor-product-of-pointed-sets}
\begin{proposition}{The Left Skew Monoidal Structure on Pointed Sets Associated to $\lhd$}{the-left-skew-monoidal-structure-on-pointed-sets-associated-to-the-left-tensor-product-of-pointed-sets}%
    The category $\Sets_{*}$ admits a left-closed left skew monoidal category structure consisting of:%
    \begin{itemize}
        \item\SloganFont{The Underlying Category. }The category $\Sets_{*}$ of pointed sets.
        \item\SloganFont{The Left Skew Monoidal Product. }The left tensor product functor
            \[
                \lhd%
                \colon%
                \Sets_{*}\times\Sets_{*}%
                \to%
                \Sets_{*}
            \]%
            of \cref{the-left-tensor-product-of-pointed-sets}.
        \item\SloganFont{The Left Internal Skew Hom. }The left internal Hom functor
            \[
                [-,-]^{\lhd}_{\Sets_{*}}%
                \colon%
                \Sets^{\op}_{*}\times\Sets_{*}%
                \to%
                \Sets_{*}%
            \]%
            of \cref{the-left-internal-hom-of-pointed-sets}.
        \item\SloganFont{The Left Skew Monoidal Unit. }The functor
            \[
                \Unit^{\Sets_{*},\lhd}
                \colon
                \PunctualCategory
                \to
                \Sets_{*}
            \]
            of \cref{the-left-skew-unit-of-the-left-tensor-product-of-pointed-sets}.
        \item\SloganFont{The Left Skew Associators. }The natural transformation
            \[
                \alpha^{\Sets_{*},\lhd}%
                \colon%
                {\lhd}\circ{({\lhd}\times\id_{\Sets_{*}})}%
                \Longrightarrow%
                {\lhd}\circ{(\id_{\Sets_{*}}\times{\lhd})}\circ{\bfalpha^{\Cats}_{\Sets_{*},\Sets_{*},\Sets_{*}}}%
            \]
            of \cref{the-left-skew-associator-of-the-left-tensor-product-of-pointed-sets}.
        \item\SloganFont{The Left Skew Left Unitors. }The natural transformation
            \[
                \LUnitor^{\Sets_{*},\lhd}%
                \colon%
                {\lhd}\circ{(\Unit^{\Sets_{*}}\times\id_{\Sets_{*}})}%
                \Longrightisoarrow%
                \bfLUnitor^{\TwoCategoryOfCategories}_{\Sets_{*}}%
            \]
            of \cref{the-left-skew-left-unitor-of-the-left-tensor-product-of-pointed-sets}.
        \item\SloganFont{The Left Skew Right Unitors. }The natural transformation
            \[
                \RUnitor^{\Sets_{*},\lhd}%
                \colon%
                \bfRUnitor^{\TwoCategoryOfCategories}_{\Sets_{*}}%
                \Longrightisoarrow%
                {\lhd}\circ{({\sfid}\times{\Unit^{\Sets_{*}}})}%
            \]
            of \cref{the-left-skew-right-unitor-of-the-left-tensor-product-of-pointed-sets}.
    \end{itemize}
\end{proposition}
\begin{Proof}{Proof of \cref{the-left-skew-monoidal-structure-on-pointed-sets-associated-to-the-left-tensor-product-of-pointed-sets}}%
    \ProofBox{The Pentagon Identity}%
    Let $(W,w_{0})$, $(X,x_{0})$, $(Y,y_{0})$ and $(Z,z_{0})$ be pointed sets. We have to show that the diagram
    \[
        \begin{tikzcd}[row sep={0*\the\DL,between origins}, column sep={0*\the\DL,between origins}, background color=backgroundColor, ampersand replacement=\&]
            \&[0.30901699437\FourCmPlusHalf]
            \&[0.5\FourCmPlusHalf]
            (W\lhd(X\lhd Y))\lhd Z
            \&[0.5\FourCmPlusHalf]
            \&[0.30901699437\FourCmPlusHalf]
            \\[0.58778525229\FourCmPlusHalf]
            ((W\lhd X)\lhd Y)\lhd Z
            \&[0.30901699437\FourCmPlusHalf]
            \&[0.5\FourCmPlusHalf]
            \&[0.5\FourCmPlusHalf]
            \&[0.30901699437\FourCmPlusHalf]
            W\lhd((X\lhd Y)\lhd Z)
            \\[0.95105651629\FourCmPlusHalf]
            \&[0.30901699437\FourCmPlusHalf]
            (W\lhd X)\lhd(Y\lhd Z)
            \&[0.5\FourCmPlusHalf]
            \&[0.5\FourCmPlusHalf]
            W\lhd(X\lhd(Y\lhd Z))
            \&[0.30901699437\FourCmPlusHalf]
            % 1-Arrows
            % Left Boundary
            \arrow[from=2-1,to=1-3,"\alpha^{\Sets_{*},\lhd}_{W,X,Y}\lhd\id_{Z}"{pos=0.4125}]%
            \arrow[from=1-3,to=2-5,"\alpha^{\Sets_{*},\lhd}_{W,X\lhd Y,Z}"{pos=0.6}]%
            \arrow[from=2-5,to=3-4,"\id_{W}\lhd\alpha^{\Sets_{*},\lhd}_{X,Y,Z}"{pos=0.425}]%
            % Right Boundary
            \arrow[from=2-1,to=3-2,"\alpha^{\Sets_{*},\lhd}_{W\lhd X,Y,Z}"'{pos=0.425}]%
            \arrow[from=3-2,to=3-4,"\alpha^{\Sets_{*},\lhd}_{W,X,Y\lhd Z}"']%
        \end{tikzcd}
    \]%
    commutes. Indeed, this diagram acts on elements as
    \[
        \begin{tikzcd}[row sep={0*\the\DL,between origins}, column sep={0*\the\DL,between origins}, background color=backgroundColor, ampersand replacement=\&]
            \&[0.30901699437\FourCm]
            \&[0.5\FourCm]
            (w\lhd(x\lhd y))\lhd z
            \&[0.5\FourCm]
            \&[0.30901699437\FourCm]
            \\[0.58778525229\FourCm]
            ((w\lhd x)\lhd y)\lhd z
            \&[0.30901699437\FourCm]
            \&[0.5\FourCm]
            \&[0.5\FourCm]
            \&[0.30901699437\FourCm]
            w\lhd((x\lhd y)\lhd z)
            \\[0.95105651629\FourCm]
            \&[0.30901699437\FourCm]
            (w\lhd x)\lhd(y\lhd z)
            \&[0.5\FourCm]
            \&[0.5\FourCm]
            w\lhd(x\lhd(y\lhd z))
            \&[0.30901699437\FourCm]
            % 1-Arrows
            % Left Boundary
            \arrow[from=2-1,to=1-3,mapsto]
            \arrow[from=1-3,to=2-5,mapsto]
            \arrow[from=2-5,to=3-4,mapsto]
            % Right Boundary
            \arrow[from=2-1,to=3-2,mapsto]
            \arrow[from=3-2,to=3-4,mapsto]
        \end{tikzcd}
    \]%
    and thus we see that the pentagon identity is satisfied.

    \ProofBox{The Left Skew Left Triangle Identity}%
    Let $(X,x_{0})$ and $(Y,y_{0})$ be pointed sets. We have to show that the diagram
    \[
        \begin{tikzcd}[row sep={5.0*\the\DL,between origins}, column sep={10.0*\the\DL,between origins}, background color=backgroundColor, ampersand replacement=\&]
            (S^{0}\lhd X)\lhd Y
            \arrow[r,"\alpha^{\Sets_{*},\lhd}_{S^{0},X,Y}"]
            \arrow[rd,"\LUnitor^{\Sets_{*},\lhd}_{X}\lhd\id_{Y}"']
            \&
            S^{0}\lhd(X\lhd Y)
            \arrow[d,"\LUnitor^{\Sets_{*},\lhd}_{X\lhd Y}"]
            \\
            \&
            X\lhd Y
        \end{tikzcd}
    \]%
    commutes. Indeed, this diagram acts on elements as
    \[
        \begin{tikzcd}[row sep={5.0*\the\DL,between origins}, column sep={8.0*\the\DL,between origins}, background color=backgroundColor, ampersand replacement=\&]
            (0\lhd x)\lhd y
            \arrow[r,mapsto]
            \arrow[rd,mapsto]
            \&
            0\lhd(x\lhd y)
            \arrow[d,mapsto]
            \\
            \&
            x_{0}\lhd y=x_{0}\lhd y_{0}%
        \end{tikzcd}
    \]%
    and
    \[
        \begin{tikzcd}[row sep={5.0*\the\DL,between origins}, column sep={8.0*\the\DL,between origins}, background color=backgroundColor, ampersand replacement=\&]
            (1\lhd x)\lhd y
            \arrow[r,mapsto]
            \arrow[rd,mapsto]
            \&
            1\lhd(x\lhd y)
            \arrow[d,mapsto]
            \\
            \&
            x\lhd y
        \end{tikzcd}
    \]%
    and hence indeed commutes. Thus the left skew triangle identity is satisfied.

    \ProofBox{The Left Skew Right Triangle Identity}%
    Let $(X,x_{0})$ and $(Y,y_{0})$ be pointed sets. We have to show that the diagram
    \[
        \begin{tikzcd}[row sep={5.0*\the\DL,between origins}, column sep={10.5*\the\DL,between origins}, background color=backgroundColor, ampersand replacement=\&]
            X\lhd Y
            \arrow[rd,"\id_{X}\lhd\RUnitor^{\Sets_{*},\lhd}_{Y}"]
            \arrow[d,"\RUnitor^{\Sets_{*},\lhd}_{X\lhd Y}"']
            \&
            \\
            (X\lhd Y)\lhd S^{0}%
            \arrow[r,"\alpha^{\Sets_{*},\lhd}_{X,Y,S^{0}}"']
            \&
            X\lhd(Y\lhd S^{0})
        \end{tikzcd}
    \]%
    commutes. Indeed, this diagram acts on elements as
    \[
        \begin{tikzcd}[row sep={5.0*\the\DL,between origins}, column sep={8.0*\the\DL,between origins}, background color=backgroundColor, ampersand replacement=\&]
            x\lhd y
            \arrow[rd,mapsto]
            \arrow[d,mapsto]
            \&
            \\
            (x\lhd y)\lhd 1%
            \arrow[r,mapsto]
            \&
            x\lhd(y\lhd 1)
        \end{tikzcd}
    \]%
    and hence indeed commutes. Thus the right skew triangle identity is satisfied.

    \ProofBox{The Left Skew Middle Triangle Identity}%
    Let $(X,x_{0})$ and $(Y,y_{0})$ be pointed sets. We have to show that the diagram
    \[
        \begin{tikzcd}[row sep={5.0*\the\DL,between origins}, column sep={10.0*\the\DL,between origins}, background color=backgroundColor, ampersand replacement=\&]
            X\lhd Y
            \arrow[r,Equals]
            \arrow[d,"\RUnitor^{\Sets_{*},\lhd}_{X}\lhd\id_{Y}"']
            \&
            X\lhd Y
            \arrow[from=d,"\id_{A}\lhd\LUnitor^{\Sets_{*},\lhd}_{Y}"']
            \\
            (X\lhd S^{0})\lhd Y
            \arrow[r,"\alpha^{\Sets_{*},\lhd}_{X,S^{0},Y}"']
            \&
            X\lhd(S^{0}\lhd Y)
        \end{tikzcd}
    \]%
    commutes. Indeed, this diagram acts on elements as
    \[
        \begin{tikzcd}[row sep={5.0*\the\DL,between origins}, column sep={9.0*\the\DL,between origins}, background color=backgroundColor, ampersand replacement=\&]
            x\lhd y
            \arrow[r,mapsto]
            \arrow[d,mapsto]
            \&
            x\lhd y
            \arrow[from=d,mapsto]
            \\
            (x\lhd 1)\lhd y
            \arrow[r,mapsto]
            \&
            x\lhd(1\lhd y)
        \end{tikzcd}
    \]%
    and hence indeed commutes. Thus the right skew triangle identity is satisfied.

    \ProofBox{The Zig-Zag Identity}%
    We have to show that the diagram
    \[
        \begin{tikzcd}[row sep={6.0*\the\DL,between origins}, column sep={6.0*\the\DL,between origins}, background color=backgroundColor, ampersand replacement=\&]
            S^{0}
            \arrow[r,"\RUnitor^{\Sets_{*},\lhd}_{S^{0}}"]
            \arrow[rd,Equals]
            \&
            S^{0}\lhd S^{0}
            \arrow[d,"\LUnitor^{\Sets_{*},\lhd}_{S^{0}}"]
            \\
            \&
            S^{0}
        \end{tikzcd}
    \]%
    commutes. Indeed, this diagram acts on elements as
    \[
        \begin{tikzcd}[row sep={4.0*\the\DL,between origins}, column sep={4.0*\the\DL,between origins}, background color=backgroundColor, ampersand replacement=\&]
            0
            \arrow[r,mapsto]
            \arrow[rd,mapsto]
            \&
            0\lhd 1
            \arrow[d,mapsto]
            \\
            \&
            0
        \end{tikzcd}
    \]%
    and
    \[
        \begin{tikzcd}[row sep={4.0*\the\DL,between origins}, column sep={4.0*\the\DL,between origins}, background color=backgroundColor, ampersand replacement=\&]
            1
            \arrow[r,mapsto]
            \arrow[rd,mapsto]
            \&
            1\lhd 1
            \arrow[d,mapsto]
            \\
            \&
            1
        \end{tikzcd}
    \]%
    and hence indeed commutes. Thus the zig-zag identity is satisfied.

    \ProofBox{Left Skew Monoidal Left-Closedness}%
    This follows from \cref{properties-of-left-tensor-products-of-pointed-sets-adjointness-1} of \cref{properties-of-left-tensor-products-of-pointed-sets}.
\end{Proof}
\subsection{Monoids With Respect to the Left Tensor Product of Pointed Sets}\label{subsection-monoids-with-respect-to-the-left-tensor-product-of-pointed-sets}
\begin{proposition}{Monoids With Respect to $\lhd$}{monoids-with-respect-to-the-left-tensor-product-of-pointed-sets}%
    The category of monoids on $\smash{(\Sets_{*},\lhd,S^{0})}$ is isomorphic to the category of \say{monoids with left zero}%
    %--- Begin Footnote ---%
    \footnote{%
        A monoid with left zero is defined similarly as the monoids with zero of \ChapterMonoidsWithZero. Succinctly, they are monoids $(A,\mu_{A},\eta_{A})$ with a special element $0_{A}$ satisfying
        \[
            0_{A}a%
            =%
            0_{A}%
        \]%
        for each $a\in A$.
        \par\vspace*{\TCBBoxCorrection}
    } %
    %---  End Footnote  ---%
    and morphisms between them.
\end{proposition}
\begin{Proof}{Proof of \cref{monoids-with-respect-to-the-left-tensor-product-of-pointed-sets}}%
    \ProofBox{Monoids on $(\Sets_{*},\lhd,S^{0})$}%
    A monoid on $(\Sets_{*},\lhd,S^{0})$ consists of:
    \begin{itemize}
        \item\SloganFont{The Underlying Object. }A pointed set $(A,0_{A})$.
        \item\SloganFont{The Multiplication Morphism. }A morphism of pointed sets
            \[
                \mu_{A}%
                \colon%
                A\lhd A%
                \to%
                A,%
            \]%
            determining a left bilinear morphism of pointed sets
            \[
                \begin{tikzcd}[row sep=0.0*\the\DL, column sep=3.0*\the\DL, background color=backgroundColor, ampersand replacement=\&]
                    A\times A
                    \arrow[r]
                    \&
                    A
                    \\
                    (a,b)%
                    \arrow[r, mapsto]
                    \&
                    ab\mrp{.}%
                \end{tikzcd}
            \]%
        \item\SloganFont{The Unit Morphism. }A morphism of pointed sets
            \[
                \eta_{A}%
                \colon%
                S^{0}%
                \to%
                A%
            \]%
            picking an element $1_{A}$ of $A$.
    \end{itemize}
    satisfying the following conditions:
    \begin{enumerate}
        \item\label{proof-of-monoids-with-respect-to-the-left-tensor-product-of-pointed-sets-associativity}\SloganFont{Associativity. }The diagram
            \[
                \begin{tikzcd}[row sep={0*\the\DL,between origins}, column sep={0*\the\DL,between origins}, background color=backgroundColor, ampersand replacement=\&]
                    \&[0.30901699437\TwoCm]
                    \&[0.5\TwoCm]
                    A\lhd(A\lhd A)
                    \&[0.5\TwoCm]
                    \&[0.30901699437\TwoCm]
                    \\[0.58778525229\TwoCm]
                    (A\lhd A)\lhd A
                    \&[0.30901699437\TwoCm]
                    \&[0.5\TwoCm]
                    \&[0.5\TwoCm]
                    \&[0.30901699437\TwoCm]
                    A\lhd A
                    \\[0.95105651629\TwoCm]
                    \&[0.30901699437\TwoCm]
                    A\lhd A
                    \&[0.5\TwoCm]
                    \&[0.5\TwoCm]
                    A
                    \&[0.30901699437\TwoCm]
                    % 1-Arrows
                    % Left Boundary
                    \arrow[from=2-1,to=1-3,"\alpha^{\Sets_{*},\lhd}_{A,A,A}"{pos=0.4125}]%
                    \arrow[from=1-3,to=2-5,"\id_{A}\lhd\mu_{A}"{pos=0.6}]%
                    \arrow[from=2-5,to=3-4,"\mu_{A}"{pos=0.425}]%
                    % Right Boundary
                    \arrow[from=2-1,to=3-2,"\mu_{A}\lhd\id_{A}"'{pos=0.425}]%
                    \arrow[from=3-2,to=3-4,"\mu_{A}"']%
                \end{tikzcd}
            \]%
        \item\label{proof-of-monoids-with-respect-to-the-left-tensor-product-of-pointed-sets-left-unitality}\SloganFont{Left Unitality. }The diagram
            \[
                \begin{tikzcd}[row sep={5.0*\the\DL,between origins}, column sep={7.0*\the\DL,between origins}, background color=backgroundColor, ampersand replacement=\&]
                    S^{0}\lhd A
                    \arrow[r,"\eta_{A}\times\id_{A}"]
                    \arrow[rd,"\LUnitor^{\Sets_{*},\lhd}_{A}"']
                    \&
                    A\lhd A
                    \arrow[d,"\mu_{A}"]
                    \\
                    \&
                    A
                \end{tikzcd}
            \]%
            commutes.
        \item\label{proof-of-monoids-with-respect-to-the-left-tensor-product-of-pointed-sets-right-unitality}\SloganFont{Right Unitality. }The diagram
            \[
                \begin{tikzcd}[row sep={5.0*\the\DL,between origins}, column sep={6.0*\the\DL,between origins}, background color=backgroundColor, ampersand replacement=\&]
                    A
                    \arrow[r,"\RUnitor^{\Sets_{*},\lhd}_{A}"]
                    \arrow[d,Equals]
                    \&
                    A\lhd S^{0}
                    \arrow[d,"\id_{A}\times\eta_{A}"]
                    \\
                    A
                    \&
                    A\lhd A
                    \arrow[l,"\mu_{A}"]
                \end{tikzcd}
            \]%
            commutes.
    \end{enumerate}
    Being a left-bilinear morphism of pointed sets, the multiplication map satisfies
    \[
        0_{A}a%
        =%
        0_{A}%
    \]%
    for each $a\in A$. Now, the associativity, left unitality, and right unitality conditions act on elements as follows:
    \begin{enumerate}
        \item\label{proof-of-monoids-with-respect-to-the-left-tensor-product-of-pointed-sets-associativity-2}\SloganFont{Associativity. }The associativity condition acts as
            \begin{webcompile}
                \begin{tikzcd}[row sep={0*\the\DL,between origins}, column sep={0*\the\DL,between origins}, background color=backgroundColor, ampersand replacement=\&,cramped]
                    \&[0.30901699437\TwoCm]
                    \&[0.5\TwoCm]
                    \&[0.5\TwoCm]
                    \&[0.30901699437\TwoCm]
                    \\[0.58778525229\TwoCm]
                    (a\lhd b)\lhd c
                    \&[0.30901699437\TwoCm]
                    \&[0.5\TwoCm]
                    \&[0.5\TwoCm]
                    \&[0.30901699437\TwoCm]
                    \\[0.95105651629\TwoCm]
                    \&[0.30901699437\TwoCm]
                    ab\lhd c
                    \&[0.5\TwoCm]
                    \&[0.5\TwoCm]
                    (ab)c
                    \&[0.30901699437\TwoCm]
                    % 1-Arrows
                    % Right Boundary
                    \arrow[from=2-1,to=3-2,mapsto]
                    \arrow[from=3-2,to=3-4,mapsto]
                \end{tikzcd}
                \quad
                \begin{tikzcd}[row sep={0*\the\DL,between origins}, column sep={0*\the\DL,between origins}, background color=backgroundColor, ampersand replacement=\&,cramped]
                    \&[0.30901699437\TwoCm]
                    \&[0.5\TwoCm]
                    a\lhd(b\lhd c)
                    \&[0.5\TwoCm]
                    \&[0.30901699437\TwoCm]
                    \\[0.58778525229\TwoCm]
                    (a\lhd b)\lhd c
                    \&[0.30901699437\TwoCm]
                    \&[0.5\TwoCm]
                    \&[0.5\TwoCm]
                    \&[0.30901699437\TwoCm]
                    a\lhd bc
                    \\[0.95105651629\TwoCm]
                    \&[0.30901699437\TwoCm]
                    \&[0.5\TwoCm]
                    \&[0.5\TwoCm]
                    a(bc)
                    \&[0.30901699437\TwoCm]
                    % 1-Arrows
                    % Left Boundary
                    \arrow[from=2-1,to=1-3,mapsto]
                    \arrow[from=1-3,to=2-5,mapsto]
                    \arrow[from=2-5,to=3-4,mapsto]
                \end{tikzcd}
            \end{webcompile}
            This gives
            \[
                (ab)c%
                =%
                a(bc)%
            \]%
            for each $a,b,c\in A$.
        \item\label{proof-of-monoids-with-respect-to-the-left-tensor-product-of-pointed-sets-left-unitality-2}\SloganFont{Left Unitality. }The left unitality condition acts:
            \begin{enumerate}
                \item\label{proof-of-monoids-with-respect-to-the-left-tensor-product-of-pointed-sets-left-unitality-2-a}On $0\lhd a$ as
                    \begin{webcompile}
                        \begin{tikzcd}[row sep={5.0*\the\DL,between origins}, column sep={5.0*\the\DL,between origins}, background color=backgroundColor, ampersand replacement=\&]
                            0\lhd a
                            \arrow[rd,mapsto]
                            \&
                            \\
                            \&
                            0_{A}
                        \end{tikzcd}
                        \quad
                        \begin{tikzcd}[row sep={5.0*\the\DL,between origins}, column sep={5.0*\the\DL,between origins}, background color=backgroundColor, ampersand replacement=\&]
                            0\lhd a
                            \arrow[r,mapsto]
                            \arrow[rd,mapsto]
                            \&
                            0_{A}\lhd a
                            \arrow[d,mapsto]
                            \\
                            \&
                            0_{A}a\mrp{.}
                        \end{tikzcd}
                    \end{webcompile}
                \item\label{proof-of-monoids-with-respect-to-the-left-tensor-product-of-pointed-sets-left-unitality-2-b}On $1\lhd a$ as
                    \begin{webcompile}
                        \begin{tikzcd}[row sep={5.0*\the\DL,between origins}, column sep={5.0*\the\DL,between origins}, background color=backgroundColor, ampersand replacement=\&]
                            1\lhd a
                            \arrow[rd,mapsto]
                            \&
                            \\
                            \&
                            a
                        \end{tikzcd}
                        \quad
                        \begin{tikzcd}[row sep={5.0*\the\DL,between origins}, column sep={5.0*\the\DL,between origins}, background color=backgroundColor, ampersand replacement=\&]
                            1\lhd a
                            \arrow[r,mapsto]
                            \arrow[rd,mapsto]
                            \&
                            1_{A}\lhd a
                            \arrow[d,mapsto]
                            \\
                            \&
                            1_{A}a\mrp{.}
                        \end{tikzcd}
                    \end{webcompile}
            \end{enumerate}
            This gives
            \begin{align*}
                1_{A}a &= a,\\
                0_{A}a &= 0_{A}
            \end{align*}
            for each $a\in A$.
        \item\label{proof-of-monoids-with-respect-to-the-left-tensor-product-of-pointed-sets-right-unitality-2}\SloganFont{Right Unitality. }The right unitality condition acts as
            \begin{webcompile}
                \begin{tikzcd}[row sep={5.0*\the\DL,between origins}, column sep={5.0*\the\DL,between origins}, background color=backgroundColor, ampersand replacement=\&]
                    a
                    \arrow[d,mapsto]
                    \&
                    \\
                    a
                    \&
                \end{tikzcd}
                \begin{tikzcd}[row sep={5.0*\the\DL,between origins}, column sep={5.0*\the\DL,between origins}, background color=backgroundColor, ampersand replacement=\&]
                    a
                    \arrow[r,mapsto]
                    \&
                    a\lhd1
                    \arrow[d,mapsto]
                    \\
                    a1_{A}
                    \&
                    a\lhd1_{A}
                    \arrow[l,mapsto]
                \end{tikzcd}
            \end{webcompile}
            This gives
            \[
                a1_{A}%
                =%
                a%
            \]%
            for each $a\in A$.
    \end{enumerate}
    Thus we see that monoids with respect to $\lhd$ are exactly monoids with left zero.

    \ProofBox{Morphisms of Monoids on $(\Sets_{*},\lhd,S^{0})$}%
    A morphism of monoids on $(\Sets_{*},\lhd,S^{0})$ from $(A,\mu_{A},\eta_{A},0_{A})$ to $(B,\mu_{B},\eta_{B},0_{B})$ is a morphism of pointed sets
    \[
        f%
        \colon%
        (A,0_{A})%
        \to%
        (B,0_{B})%
    \]%
    satisfying the following conditions:
    \begin{enumerate}
        \item\label{proof-of-monoids-with-respect-to-the-left-tensor-product-of-pointed-sets-compatibility-with-the-multiplication-morphisms}\SloganFont{Compatibility With the Multiplication Morphisms. }The diagram
            \[
                \begin{tikzcd}[row sep={5.0*\the\DL,between origins}, column sep={6.5*\the\DL,between origins}, background color=backgroundColor, ampersand replacement=\&]
                    A\lhd A
                    \arrow[r,"f\lhd f"]
                    \arrow[d,"\mu_{A}"']
                    \&
                    B\lhd B
                    \arrow[d,"\mu_{B}"]
                    \\
                    A
                    \arrow[r,"f"']
                    \&
                    B
                \end{tikzcd}
            \]%
            commutes.
        \item\label{proof-of-monoids-with-respect-to-the-left-tensor-product-of-pointed-sets-compatibility-with-the-unit-morphisms}\SloganFont{Compatibility With the Unit Morphisms. }The diagram
            \[
                \begin{tikzcd}[row sep={5.0*\the\DL,between origins}, column sep={5.0*\the\DL,between origins}, background color=backgroundColor, ampersand replacement=\&]
                    S^{0}
                    \arrow[r,"\eta_{A}"]
                    \arrow[rd,"\eta_{B}"']
                    \&
                    A
                    \arrow[d,"f"]
                    \\
                    \&
                    B
                \end{tikzcd}
            \]%
            commutes.
    \end{enumerate}
    These act on elements as
    \begin{webcompile}
        \begin{tikzcd}[row sep={5.0*\the\DL,between origins}, column sep={6.5*\the\DL,between origins}, background color=backgroundColor, ampersand replacement=\&]
            a\lhd b
            \arrow[d,mapsto]
            \&
            \\
            ab
            \arrow[r,mapsto]
            \&
            f(ab)
        \end{tikzcd}
        \quad
        \begin{tikzcd}[row sep={5.0*\the\DL,between origins}, column sep={6.5*\the\DL,between origins}, background color=backgroundColor, ampersand replacement=\&]
            a\lhd b
            \arrow[r,mapsto]
            \&
            f(a)\lhd f(b)
            \arrow[d,mapsto]
            \\
            \&
            f(a)f(b)
        \end{tikzcd}
    \end{webcompile}
    and
    \begin{webcompile}
        \begin{tikzcd}[row sep={5.0*\the\DL,between origins}, column sep={5.0*\the\DL,between origins}, background color=backgroundColor, ampersand replacement=\&]
            0
            \arrow[rd,mapsto]
            \&
            \\
            \&
            0_{B}
        \end{tikzcd}
        \quad
        \begin{tikzcd}[row sep={5.0*\the\DL,between origins}, column sep={5.0*\the\DL,between origins}, background color=backgroundColor, ampersand replacement=\&]
            0
            \arrow[r,mapsto]
            \&
            0_{A}
            \arrow[d,mapsto]
            \\
            \&
            f(0_{A})
        \end{tikzcd}
    \end{webcompile}
    and
    \begin{webcompile}
        \begin{tikzcd}[row sep={5.0*\the\DL,between origins}, column sep={5.0*\the\DL,between origins}, background color=backgroundColor, ampersand replacement=\&]
            1
            \arrow[rd,mapsto]
            \&
            \\
            \&
            1_{B}
        \end{tikzcd}
        \quad
        \begin{tikzcd}[row sep={5.0*\the\DL,between origins}, column sep={5.0*\the\DL,between origins}, background color=backgroundColor, ampersand replacement=\&]
            1
            \arrow[r,mapsto]
            \&
            1_{A}
            \arrow[d,mapsto]
            \\
            \&
            f(1_{A})
        \end{tikzcd}
    \end{webcompile}
    giving
    \begin{gather*}
        f(ab)    = f(a)f(b),\\
        \begin{aligned}
            f(0_{A}) &= 0_{B},\\
            f(1_{A}) &= 1_{B},
        \end{aligned}
    \end{gather*}
    for each $a,b\in A$, which is exactly a morphism of monoids with left zero.

    \ProofBox{Identities and Composition}%
    Similarly, the identities and composition of $\Mon(\Sets_{*},\lhd,S^{0})$ can be easily seen to agree with those of monoids with left zero, which finishes the proof.
\end{Proof}
\section{The Right Tensor Product of Pointed Sets}\label{section-the-right-tensor-product-of-pointed-sets}
\subsection{Foundations}\label{subsection-the-right-tensor-product-of-pointed-sets-foundations}
Let $(X,x_{0})$ and $(Y,y_{0})$ be pointed sets.
\begin{definition}{The Right Tensor Product of Pointed Sets}{the-right-tensor-product-of-pointed-sets}%
    The \index[set-theory]{pointed set!right tensor product}\textbf{right tensor product of pointed sets} is the functor\index[notation]{rhd@$\rhd$}%
    %--- Begin Footnote ---%
    \footnote{%
        \SloganFont{Further Notation: }Also written \index[notation]{rhdSetsstar@$\rhd_{\Sets_{*}}$}$\rhd_{\Sets_{*}}$.
        \par\vspace*{\TCBBoxCorrection}
    }%
    %---  End Footnote  ---%
    \[
        \rhd%
        \colon%
        \Sets_{*}\times\Sets_{*}%
        \to%
        \Sets_{*}%
    \]%
    defined as the composition
    \[
        \Sets_{*}\times\Sets_{*}%
        \xlongrightarrow{\Wasureru\times\sfid}%
        \Sets\times\Sets_{*}%
        \xlongrightarrow{\odot}%
        \Sets_{*},%
    \]%
    where:
    \begin{itemize}
        \item $\Wasureru\colon\Sets_{*}\to\Sets$ is the forgetful functor from pointed sets to sets.
        \item $\odot\colon\Sets\times\Sets_{*}\to\Sets_{*}$ is the tensor functor of \cref{properties-of-tensors-of-pointed-sets-by-sets-functoriality} of \cref{properties-of-tensors-of-pointed-sets-by-sets}.%
    \end{itemize}
\end{definition}
\begin{remark}{Unwinding \cref{the-right-tensor-product-of-pointed-sets}: Universal Property \rmI}{unwinding-the-right-tensor-product-of-pointed-sets-universal-property-1}%
    The right tensor product of pointed sets satisfies the following natural bijection:%
    \[%
        \Sets_{*}(X\rhd Y,Z)%
        \cong%
        \Hom^{\otimes,\rmR}_{\Sets_{*}}(X\times Y,Z).
    \]%
    That is to say, the following data are in natural bijection:
    \begin{enumerate}
        \item\label{unwinding-the-right-tensor-product-of-pointed-sets-universal-property-1-item-1}Pointed maps $f\colon X\rhd Y\to Z$.
        \item\label{unwinding-the-right-tensor-product-of-pointed-sets-universal-property-1-item-2}Maps of sets $f\colon X\times Y\to Z$ satisfying $f(x,y_{0})=z_{0}$ for each $x\in X$.
    \end{enumerate}
\end{remark}
\begin{remark}{Unwinding \cref{the-right-tensor-product-of-pointed-sets}: Universal Property \rmII}{unwinding-the-right-tensor-product-of-pointed-sets-2-universal-property-2}%
    The right tensor product of pointed sets may be described as follows:
    \begin{itemize}
        \item The right tensor product of $(X,x_{0})$ and $(Y,y_{0})$ is the pair $((X\rhd Y,x_{0}\rhd y_{0}),\iota)$ consisting of
            \begin{itemize}
                \item A pointed set $(X\rhd Y,x_{0}\rhd y_{0})$;
                \item A right bilinear morphism of pointed sets $\iota\colon(X\times Y,(x_{0},y_{0}))\to X\rhd Y$;
            \end{itemize}
            satisfying the following universal property:
            \begin{itemize}
                \itemstar Given another such pair $((Z,z_{0}),f)$ consisting of
                    \begin{itemize}
                        \item A pointed set $(Z,z_{0})$;
                        \item A right bilinear morphism of pointed sets $f\colon(X\times Y,(x_{0},y_{0}))\to X\rhd Y$;
                    \end{itemize}
                    there exists a unique morphism of pointed sets $X\rhd Y\uearrow Z$ making the diagram
                    \[
                        \begin{tikzcd}[row sep={5.0*\the\DL,between origins}, column sep={5.0*\the\DL,between origins}, background color=backgroundColor, ampersand replacement=\&]
                            \&
                            X\rhd Y
                            \arrow[d,"\exists!",densely dashed]
                            \\
                            X\times Y
                            \arrow[r,"f"']
                            \arrow[ru,"\iota"]
                            \&
                            Z
                        \end{tikzcd}
                    \]%
                    commute.%
                \end{itemize}
    \end{itemize}
\end{remark}
\begin{construction}{The Right Tensor Product of Pointed Sets}{constructions-of-the-right-tensor-product-of-pointed-sets}%
    In detail, the \textbf{right tensor product of $(X,x_{0})$ and $(Y,y_{0})$} is the pointed set $(X\rhd Y,[y_{0}])$ consisting of:
    \begin{itemize}
        \item\SloganFont{The Underlying Set. }The set $X\rhd Y$ defined by
            \begin{align*}
                X\rhd Y &\defeq \abs{X}\odot Y\\
                        &\cong  \bigvee_{x\in X}(Y,y_{0}),
            \end{align*}
            where $\abs{X}$ denotes the underlying set of $(X,x_{0})$.
        \item\SloganFont{The Underlying Basepoint. }The point $[(x_{0},y_{0})]$ of $\bigvee_{x\in X}(Y,y_{0})$, which is equal to $[(x,y_{0})]$ for any $x\in X$.
    \end{itemize}
\end{construction}
\begin{Proof}{Proof of \cref{constructions-of-the-right-tensor-product-of-pointed-sets}}%
    Since $\bigvee_{y\in Y}(X,x_{0})$ is defined as the quotient of $\coprod_{x\in X}Y$ by the equivalence relation $R$ generated by declaring $(x,y)\sim(x',y')$ if $y=y'=y_{0}$, we have, by \ChapterRef{\ChapterConditionsOnRelations, \cref{conditions-on-relations:unwinding-the-quotient-of-a-set-by-an-equivalence-relation}}{\cref{unwinding-the-quotient-of-a-set-by-an-equivalence-relation}}, a natural bijection
    \[
        \Sets_{*}(X\rhd Y,Z)
        \cong
        \Hom^{R}_{\Sets}(\coprod_{X\in X}Y,Z),%
    \]%
    where $\Hom^{R}_{\Sets}(X\times Y,Z)$ is the set
    \begin{envsmallsize}
        \[
            \Hom^{R}_{\Sets}(\coprod_{x\in X}Y,Z)%
            \defeq%
            \{%
                f\in\Hom_{\Sets}(\coprod_{x\in X}Y,Z)%
                \ \middle|\ %
                \begin{aligned}
                    &\text{for each $x,y\in X$, if}\\
                    &\text{$(x,y)\sim_{R}(x',y')$, then}\\
                    &\text{$f(x,y)=f(x',y')$}%
                \end{aligned}
            \}.%
        \]%
    \end{envsmallsize}
    However, the condition $(x,y)\sim_{R}(x',y')$ only holds when:
    \begin{enumerate}
        \item\label{proof-of-constructions-of-the-right-tensor-product-of-pointed-sets-1}We have $x=x'$ and $y=y'$.
        \item\label{proof-of-constructions-of-the-right-tensor-product-of-pointed-sets-2}We have $y=y'=y_{0}$.
    \end{enumerate}
    So, given $f\in\Hom_{\Sets}(\coprod_{x\in X}Y,Z)$ with a corresponding $\widebar{f}\colon X\rhd Y\to Z$, the latter case above implies
    \begin{align*}
        f([(x,y_{0})]) &= f([(x',y_{0})])\\
                       &= f([(x_{0},y_{0})]),
    \end{align*}
    and since $\widebar{f}\colon X\rhd Y\to Z$ is a pointed map, we have
    \begin{align*}
        f([(x_{0},y_{0})]) &= \widebar{f}([(x_{0},y_{0})])\\
                           &= z_{0}.
    \end{align*}
    Thus the elements $f$ in $\Hom^{R}_{\Sets}(X\times Y,Z)$ are precisely those functions $f\colon X\times Y\to Z$ satisfying the equality
    \[
        f(x,y_{0})%
        =%
        z_{0}
    \]%
    for each $y\in Y$, giving an equality
    \[
        \Hom^{R}_{\Sets}(X\times Y,Z)%
        =%
        \Hom^{\otimes,\rmR}_{\Sets_{*}}(X\times Y,Z)%
    \]%
    of sets, which when composed with our earlier isomorphism
    \[
        \Sets_{*}(X\rhd Y,Z)
        \cong
        \Hom^{R}_{\Sets}(X\times Y,Z),%
    \]%
    gives our desired natural bijection, finishing the proof.
\end{Proof}
\begin{notation}{Elements of Right Tensor Products of Pointed Sets}{elements-of-right-tensor-products-of-pointed-sets}%
    We write%
    %--- Begin Footnote ---%
    \footnote{%
        \SloganFont{Further Notation: }Also written \index[notation]{xrighty@$x\rhd_{\Sets_{*}}y$}$x\rhd_{\Sets_{*}}y$.
        \par\vspace*{\TCBBoxCorrection}
    } %
    %---  End Footnote  ---%
    \index[notation]{xrhdy@$x\rhd y$}$x\rhd y$ for the element $[(x,y)]$ of
    \[
        X\rhd Y%
        \cong%
        \abs{X}\odot Y.%
    \]%
\end{notation}
\begin{remark}{Basepoints of Right Tensor Products of Pointed Sets}{basepoints-of-right-tensor-products-of-pointed-sets}%
    Employing the notation introduced in \cref{elements-of-right-tensor-products-of-pointed-sets}, we have
    \[
        x_{0}\rhd y_{0}%
        =%
        x\rhd y_{0}%
    \]%
    for each $x\in X$, and
    \[
        x\rhd y_{0}%
        =%
        x'\rhd y_{0}%
    \]%
    for each $x,x'\in X$.
\end{remark}
\begin{proposition}{Properties of Right Tensor Products of Pointed Sets}{properties-of-right-tensor-products-of-pointed-sets}%
    Let $(X,x_{0})$ and $(Y,y_{0})$ be pointed sets.
    \begin{enumerate}
        \item\label{properties-of-right-tensor-products-of-pointed-sets-functoriality}\SloganFont{Functoriality. }The assignments $X,Y,(X,Y)\mapsto X\rhd Y$ define functors
            \[
                \BifunctorialityPeriod{X\rhd-}{-\rhd Y}{-_{1}\rhd-_{2}}{\Sets_{*}}{\Sets_{*}}{\Sets_{*}\times\Sets_{*}}{\Sets_{*}}%
            \]%
            In particular, given pointed maps
            \begin{align*}
                f &\colon (X,x_{0}) \to (A,a_{0}),\\
                g &\colon (Y,y_{0}) \to (B,b_{0}),
            \end{align*}
            the induced map
            \[
                f\rhd g%
                \colon%
                X\rhd Y%
                \to%
                A\rhd B%
            \]%
            is given by
            \[
                [f\rhd g](x\rhd y)%
                \defeq%
                f(x)\rhd g(y)%
            \]%
            for each $x\rhd y\in X\rhd Y$.
        \item\label{properties-of-right-tensor-products-of-pointed-sets-adjointness-1}\SloganFont{Adjointness \rmI. }We have an adjunction%
            \begin{webcompile}
                \Adjunction#X\rhd -#[X,-]^{\rhd}_{\Sets_{*}}#\Sets_{*}#\Sets_{*},#
            \end{webcompile}
            witnessed by a bijection of sets
            \[
                \Hom_{\Sets_{*}}(X\rhd Y,Z)%
                \cong
                \Hom_{\Sets_{*}}(Y,[X,Z]^{\rhd}_{\Sets_{*}})%
            \]%
            natural in $(X,x_{0}),(Y,y_{0}),(Z,z_{0})\in\Obj(\Sets_{*})$, where $[X,Y]^{\rhd}_{\Sets_{*}}$ is the pointed set of \cref{the-right-internal-hom-of-pointed-sets}.
        \item\label{properties-of-right-tensor-products-of-pointed-sets-adjointness-2}\SloganFont{Adjointness \rmII. }The functor
            \[
                -\rhd Y%
                \colon%
                \Sets_{*}%
                \to%
                \Sets_{*}%
            \]%
            does not admit a right adjoint.%
        \item\label{properties-of-right-tensor-products-of-pointed-sets-adjointness-3}\SloganFont{Adjointness \rmIII. }We have a $\Wasureru$-relative adjunction
            \begin{webcompile}
                \RelativeAdjunction#\Wasureru#-\rhd Y#\Sets_{*}(Y,-)#\Sets_{*}#\Sets_{*},#
            \end{webcompile}
            witnessed by a bijection of sets
            \[
                \Hom_{\Sets_{*}}(X\rhd Y,Z)%
                \cong
                \Hom_{\Sets}(|X|,\Sets_{*}(Y,Z))%
            \]%
            natural in $(X,x_{0}),(Y,y_{0}),(Z,z_{0})\in\Obj(\Sets_{*})$.
        %\item\label{properties-of-right-tensor-products-of-pointed-sets-}\SloganFont{. }%
    \end{enumerate}
\end{proposition}
\begin{Proof}{Proof of \cref{properties-of-right-tensor-products-of-pointed-sets}}%
    \FirstProofBox{\cref{properties-of-right-tensor-products-of-pointed-sets-functoriality}: Functoriality}%
    This follows from the definition of $\rhd$ as a composition of functors (\cref{the-right-tensor-product-of-pointed-sets}).

    \ProofBox{\cref{properties-of-right-tensor-products-of-pointed-sets-adjointness-1}: Adjointness \rmI}%
    This follows from \cref{properties-of-tensors-of-pointed-sets-by-sets-adjointness-2} of \cref{properties-of-tensors-of-pointed-sets-by-sets}.

    \ProofBox{\cref{properties-of-right-tensor-products-of-pointed-sets-adjointness-2}: Adjointness \rmII}%
    For $-\rhd Y$ to admit a right adjoint would require it to preserve colimits by \ChapterRef{\ChapterAdjunctionsAndTheYonedaLemma, \cref{adjunctions-and-the-yoneda-lemma:properties-of-adjunctions-interaction-with-co-limits} of \cref{adjunctions-and-the-yoneda-lemma:properties-of-adjunctions}}{\cref{properties-of-adjunctions-interaction-with-co-limits} of \cref{properties-of-adjunctions}}. However, we have
    \begin{align*}
        \pt\rhd X &\defeq |\pt|\odot X\\
                  &\cong  X\\
                  &\ncong \pt,
    \end{align*}
    and thus we see that $-\rhd Y$ does not have a right adjoint.

    \ProofBox{\cref{properties-of-right-tensor-products-of-pointed-sets-adjointness-3}: Adjointness \rmIII}%
    This follows from \cref{properties-of-tensors-of-pointed-sets-by-sets-adjointness-1} of \cref{properties-of-tensors-of-pointed-sets-by-sets}.
\end{Proof}
\begin{remark}{On the Failure of $-\rhd Y$ To Be a Left Adjoint}{on-the-failure-of-xrhd-to-be-a-left-adjoint}%
    Here is some intuition on why $-\rhd Y$ fails to be a left adjoint. \cref{properties-of-left-tensor-products-of-pointed-sets-adjointness-3} of \cref{properties-of-left-tensor-products-of-pointed-sets} states that we have a natural bijection
    \[
        \Hom_{\Sets_{*}}(X\rhd Y,Z)%
        \cong
        \Hom_{\Sets}(|X|,\Sets_{*}(Y,Z)),%
    \]%
    so it would be reasonable to wonder whether a natural bijection of the form
   \[
        \Hom_{\Sets_{*}}(X\rhd Y,Z)%
        \cong
        \Hom_{\Sets_{*}}(X,\eSets_{*}(Y,Z)),%
    \]%
    also holds, which would give $-\rhd Y\dashv\eSets_{*}(Y,-)$. However, such a bijection would require every map
    \[
        f%
        \colon%
        X\rhd Y%
        \to%
        Z%
    \]%
    to satisfy
    \[
        f(x_{0}\rhd y)%
        =%
        z_{0}%
    \]%
    for each $x\in X$, whereas we are imposing such a basepoint preservation condition only for elements of the form $x\rhd y_{0}$. Thus $\eSets_{*}(Y,-)$ can't be a right adjoint for $-\rhd Y$, and as shown by \cref{properties-of-right-tensor-products-of-pointed-sets-adjointness-2} of \cref{properties-of-right-tensor-products-of-pointed-sets}, no functor can.%
    %--- Begin Footnote ---%
    \footnote{%
        The functor $\eSets_{*}(Y,-)$ is instead right adjoint to $-\wedge Y$, the smash product of pointed sets of \cref{smash-products-of-pointed-sets}. See \cref{properties-of-smash-products-of-pointed-sets-adjointness} of \cref{properties-of-smash-products-of-pointed-sets}.
        \par\vspace*{\TCBBoxCorrection}
    }%
    %---  End Footnote  ---%
\end{remark}
\subsection{The Right Internal Hom of Pointed Sets}\label{subsection-the-right-internal-hom-of-pointed-sets}
Let $(X,x_{0})$ and $(Y,y_{0})$ be pointed sets.
\begin{definition}{The Right Internal Hom of Pointed Sets}{the-right-internal-hom-of-pointed-sets}%
    The \index[set-theory]{pointed set!right internal Hom of}\textbf{right internal Hom}%
    %--- Begin Footnote ---%
    \footnote{%
        For a proof that $[-,-]^{\rhd}_{\Sets_{*}}$ is indeed the right internal Hom of $\Sets_{*}$ with respect to the right tensor product of pointed sets, see \cref{properties-of-right-tensor-products-of-pointed-sets-adjointness-1} of \cref{properties-of-right-tensor-products-of-pointed-sets}.
        \par\vspace*{\TCBBoxCorrection}
    } %
    %---  End Footnote  ---%
    \textbf{of pointed sets} is the functor\index[notation]{rhdSetsstar@$[-,-]^{\rhd}_{\Sets_{*}}$}%
    \[
        [-,-]^{\rhd}_{\Sets_{*}}%
        \colon%
        \Sets^{\op}_{*}\times\Sets_{*}%
        \to%
        \Sets_{*}%
    \]%
    defined as the composition
    \[
        \Sets^{\op}_{*}\times\Sets_{*}%
        \xlongrightarrow{\Wasureru\times\sfid}%
        \Sets^{\op}\times\Sets_{*}%
        \xlongrightarrow{\pitchfork}%
        \Sets_{*},%
    \]%
    where:
    \begin{itemize}
        \item $\Wasureru\colon\Sets_{*}\to\Sets$ is the forgetful functor from pointed sets to sets.
        \item $\pitchfork\colon\Sets^{\op}\times\Sets_{*}\to\Sets_{*}$ is the cotensor functor of \cref{properties-of-cotensors-of-pointed-sets-by-sets-functoriality} of \cref{properties-of-cotensors-of-pointed-sets-by-sets}.%
    \end{itemize}
\end{definition}
\begin{remark}{Unwinding \cref{the-right-internal-hom-of-pointed-sets}, \rmI: Comparison With $\smash{[-,-]^{\lhd}_{\Sets_{*}}}$}{unwinding-the-right-internal-hom-of-pointed-sets-1-comparison-with-the-left-internal-hom-of-pointed-sets}%
    We have
    \[
        [-,-]^{\lhd}_{\Sets_{*}}%
        =%
        [-,-]^{\rhd}_{\Sets_{*}}.%
    \]%
\end{remark}
\begin{remark}{Unwinding \cref{the-right-internal-hom-of-pointed-sets}, \rmII: Universal Property}{unwinding-the-right-internal-hom-of-pointed-sets-2-universal-property}%
    The right internal Hom of pointed sets satisfies the following universal property:%
    \[%
        \Sets_{*}(X\rhd Y,Z)%
        \cong%
        \Sets_{*}(Y,[X,Z]^{\rhd}_{\Sets_{*}})%
    \]%
    That is to say, the following data are in bijection:
    \begin{enumerate}
        \item\label{unwinding-the-right-internal-hom-of-pointed-sets-2-universal-property-item-1}Pointed maps $f\colon X\rhd Y\to Z$.
        \item\label{unwinding-the-right-internal-hom-of-pointed-sets-2-universal-property-item-2}Pointed maps $f\colon Y\to[X,Z]^{\rhd}_{\Sets_{*}}$.
    \end{enumerate}
\end{remark}
\begin{remark}{Unwinding \cref{the-right-internal-hom-of-pointed-sets}, \rmIII: Explicit Description}{unwinding-the-right-internal-hom-of-pointed-sets-3-explicit-description}%
    In detail, the \textbf{right internal Hom of $(X,x_{0})$ and $(Y,y_{0})$} is the pointed set $\smash{([X,Y]^{\rhd}_{\Sets_{*}},[(y_{0})_{x\in X}])}$ consisting of:%
    \begin{itemize}
        \item\SloganFont{The Underlying Set. }The set $[X,Y]^{\rhd}_{\Sets_{*}}$ defined by
            \begin{align*}
                [X,Y]^{\rhd}_{\Sets_{*}} &\defeq \abs{X}\pitchfork Y\\
                                         &\cong  \bigwedge_{x\in X}(Y,y_{0}),
            \end{align*}
            where $\abs{X}$ denotes the underlying set of $(X,x_{0})$.
        \item\SloganFont{The Underlying Basepoint. }The point $[(y_{0})_{x\in X}]$ of $\bigwedge_{x\in X}(Y,y_{0})$.
    \end{itemize}
\end{remark}
\begin{proposition}{Properties of Right Internal Homs of Pointed Sets}{properties-of-right-internal-homs-of-pointed-sets}%
    Let $(X,x_{0})$ and $(Y,y_{0})$ be pointed sets.
    \begin{enumerate}
        \item\label{properties-of-right-internal-homs-of-pointed-sets-functoriality}\SloganFont{Functoriality. }The assignments $X,Y,(X,Y)\mapsto[X,Y]^{\rhd}_{\Sets_{*}}$ define functors
            \[
                \BifunctorialityPeriod{[X,-]^{\rhd}_{\Sets_{*}}}{{[-,Y]^{\rhd}_{\Sets_{*}}}}{{[-_{1},-_{2}]^{\rhd}_{\Sets_{*}}}}{\Sets_{*}}{\Sets^{\mrp{\op}}_{*}}{\Sets^{\op}_{*}\times\Sets_{*}}{\Sets_{*}}%
            \]%
            In particular, given pointed maps
            \begin{align*}
                f &\colon (X,x_{0}) \to (A,a_{0}),\\
                g &\colon (Y,y_{0}) \to (B,b_{0}),
            \end{align*}
            the induced map
            \[
                [f,g]^{\rhd}_{\Sets_{*}}%
                \colon%
                [A,Y]^{\rhd}_{\Sets_{*}}%
                \to%
                [X,B]^{\rhd}_{\Sets_{*}}%
            \]%
            is given by
            \[
                [f,g]^{\rhd}_{\Sets_{*}}([(y_{a})_{a\in A}])%
                \defeq%
                [(g(y_{f(x)}))_{x\in X}]%
            \]%
            for each $[(y_{a})_{a\in A}]\in[A,Y]^{\rhd}_{\Sets_{*}}$.
        \item\label{properties-of-right-internal-homs-of-pointed-sets-adjointness-1}\SloganFont{Adjointness \rmI. }We have an adjunction%
            \begin{webcompile}
                \Adjunction#X\rhd -#[X,-]^{\rhd}_{\Sets_{*}}#\Sets_{*}#\Sets_{*},#
            \end{webcompile}
            witnessed by a bijection of sets
            \[
                \Hom_{\Sets_{*}}(X\rhd Y,Z)%
                \cong
                \Hom_{\Sets_{*}}(Y,[X,Z]^{\rhd}_{\Sets_{*}})%
            \]%
            natural in $(X,x_{0}),(Y,y_{0}),(Z,z_{0})\in\Obj(\Sets_{*})$, where $[X,Y]^{\rhd}_{\Sets_{*}}$ is the pointed set of \cref{the-right-internal-hom-of-pointed-sets}.
        \item\label{properties-of-right-internal-homs-of-pointed-sets-adjointness-2}\SloganFont{Adjointness \rmII. }The functor
            \[
                -\rhd Y%
                \colon%
                \Sets_{*}%
                \to%
                \Sets_{*}%
            \]%
            does not admit a right adjoint.%
        %\item\label{properties-of-right-internal-homs-of-pointed-sets-}\SloganFont{. }%
    \end{enumerate}
\end{proposition}
\begin{Proof}{Proof of \cref{properties-of-right-internal-homs-of-pointed-sets}}%
    \FirstProofBox{\cref{properties-of-right-internal-homs-of-pointed-sets-functoriality}: Functoriality}%
    This follows from the definition of $[-,-]^{\rhd}_{\Sets_{*}}$ as a composition of functors (\cref{the-right-internal-hom-of-pointed-sets}).

    \ProofBox{\cref{properties-of-right-internal-homs-of-pointed-sets-adjointness-1}: Adjointness \rmI}%
    This is a repetition of \cref{properties-of-right-tensor-products-of-pointed-sets-adjointness-1} of \cref{properties-of-right-tensor-products-of-pointed-sets}, and is proved there.

    \ProofBox{\cref{properties-of-right-internal-homs-of-pointed-sets-adjointness-2}: Adjointness \rmII}%
    This is a repetition of \cref{properties-of-right-tensor-products-of-pointed-sets-adjointness-2} of \cref{properties-of-right-tensor-products-of-pointed-sets}, and is proved there.
\end{Proof}
\subsection{The Right Skew Unit}\label{subsection-the-right-skew-unit-of-the-right-tensor-product-of-pointed-sets}
\begin{definition}{The Right Skew Unit of $\rhd$}{the-right-skew-unit-of-the-right-tensor-product-of-pointed-sets}%
    The \index[set-theory]{right tensor product of pointed sets!right skew unit of}\textbf{right skew unit of the right tensor product of pointed sets} is the functor
    \[
        \Unit^{\Sets_{*},\rhd}
        \colon
        \PunctualCategory
        \to
        \Sets_{*}
    \]
    defined by
    \[
        \Unit^{\rhd}_{\Sets_{*}}%
        \defeq%
        S^{0}.
    \]%
\end{definition}
\subsection{The Right Skew Associator}\label{subsection-the-right-tensor-product-of-pointed-sets-the-right-skew-associator}
\begin{definition}{The Right Skew Associator of $\rhd$}{the-right-skew-associator-of-the-right-tensor-product-of-pointed-sets}%
    The \index[set-theory]{right tensor product of pointed sets!skew associator}\textbf{skew associator of the right tensor product of pointed sets} is the natural transformation
    \[
        \alpha^{\Sets_{*},\rhd}
        \colon
        {\rhd}\circ{(\id_{\Sets_{*}}\times{\rhd})}
        \Longrightarrow%
        {\rhd}\circ{({\rhd}\times\id_{\Sets_{*}})}\circ{\bfalpha^{\Cats,-1}_{\Sets_{*},\Sets_{*},\Sets_{*}}}%
    \]
    as in the diagram
    \[
        \begin{tikzcd}[row sep={0*\the\DL,between origins}, column sep={0*\the\DL,between origins}, background color=backgroundColor, ampersand replacement=\&]
            \&[0.30901699437\TwoCmPlusHalf]
            \&[0.5\TwoCmPlusHalf]
            (\Sets_{*}\times\Sets_{*})\times\Sets_{*}
            \&[0.5\TwoCmPlusHalf]
            \&[0.30901699437\TwoCmPlusHalf]
            \\[0.58778525229\TwoCmPlusHalf]
            \Sets_{*}\times(\Sets_{*}\times\Sets_{*})
            \&[0.30901699437\TwoCmPlusHalf]
            \&[0.5\TwoCmPlusHalf]
            \&[0.5\TwoCmPlusHalf]
            \&[0.30901699437\TwoCmPlusHalf]
            \Sets_{*}\times\Sets_{*}
            \\[0.95105651629\TwoCmPlusHalf]
            \&[0.30901699437\TwoCmPlusHalf]
            \Sets_{*}\times\Sets_{*}
            \&[0.5\TwoCmPlusHalf]
            \&[0.5\TwoCmPlusHalf]
            \Sets_{*}\mrp{,}
            \&[0.30901699437\TwoCmPlusHalf]
            % 1-Arrows
            % Left Boundary
            \arrow[from=2-1,to=1-3,"\bfalpha^{\Cats,-1}_{\Sets_{*},\Sets_{*},\Sets_{*}}"{pos=0.3},isoarrowprime]%
            \arrow[from=1-3,to=2-5,"{{\rhd}\times\sfid}"{pos=0.6},""{name=2}]%
            \arrow[from=2-5,to=3-4,"\rhd"{pos=0.425}]%
            % Right Boundary
            \arrow[from=2-1,to=3-2,"{\sfid\times{\rhd}}"'{pos=0.425}]%
            \arrow[from=3-2,to=3-4,"\rhd"']%
            % 2-Arrows
            \arrow[from=3-2,to=2,"\alpha^{\Sets_{*},\rhd}"{description,pos=0.5},Rightarrow,shorten <= 0.5*\the\DL,shorten >= 1*\the\DL]%
        \end{tikzcd}
    \]%
    whose component
    \[
        \alpha^{\Sets_{*},\rhd}_{X,Y,Z}
        \colon
        X\rhd(Y\rhd Z)
        \to
        (X\rhd Y)\rhd Z
    \]%
    at $(X,x_{0}),(Y,y_{0}),(Z,z_{0})\in\Obj(\Sets_{*})$ is given by
    \begin{align*}
        X\rhd(Y\rhd Z) &\defeq |X|\odot(Y\rhd Z)\\
                       &\defeq |X|\odot(|Y|\odot Z)\\
                       &\cong  \bigvee_{x\in X}(|Y|\odot Z)\\
                       &\cong  \bigvee_{x\in X}(\bigvee_{y\in Y}Z)\\
                       &\to    \bigvee_{[(x,y)]\in\bigvee_{x\in X}Y}Z\\
                       &\cong  \bigvee_{[(x,y)]\in|X|\odot Y}Z\\%
                       &\cong  ||X|\odot Y|\odot Z\\%
                       &\defeq |X\rhd Y|\odot Z\\%
                       &\defeq (X\rhd Y)\rhd Z,%
    \end{align*}
    where the map%
    \[
        \bigvee_{x\in X}(\bigvee_{y\in Y}Z)%
        \to%
        \bigvee_{[(x,y)]\in\bigvee_{x\in X}Y}Z%
    \]%
    is given by $[(x,[(y,z)])]\mapsto[([(x,y)],z)]$.
\end{definition}
\begin{Proof}{Proof of \cref{the-right-skew-associator-of-the-right-tensor-product-of-pointed-sets}}%
    (Proven below in a bit.)
\end{Proof}
\begin{remark}{Unwinding \cref{the-right-skew-associator-of-the-right-tensor-product-of-pointed-sets}}{unwinding-the-right-skew-associator-of-the-right-tensor-product-of-pointed-sets}%
    Unwinding the notation for elements, we have
    \begin{align*}
        [(x,[(y,z)])] &\defeq [(x,y\rhd z)]\\
                      &\defeq x\rhd(y\rhd z)
    \end{align*}
    and
    \begin{align*}
        [([(x,y)],z)] &\defeq [(x\rhd y,z)]\\
                      &\defeq (x\rhd y)\rhd z.
    \end{align*}
    So, in other words, $\alpha^{\Sets_{*},\rhd}_{X,Y,Z}$ acts on elements via
    \[
        \alpha^{\Sets_{*},\rhd}_{X,Y,Z}(x\rhd(y\rhd z))
        \defeq
        (x\rhd y)\rhd z
    \]%
    for each $x\rhd(y\rhd z)\in X\rhd(Y\rhd Z)$.
\end{remark}
\begin{remark}{Non-Invertibility of the Skew Associator of $\rhd$}{non-invertibility-of-the-right-skew-associator-of-the-right-tensor-product-of-pointed-sets}%
    Taking $y=y_{0}$, we see that the morphism $\smash{\alpha^{\Sets_{*},\rhd}_{X,Y,Z}}$ acts on elements as
    \[
        \alpha^{\Sets_{*},\rhd}_{X,Y,Z}(x\rhd(y_{0}\rhd z))%
        \defeq%
        (x\rhd y_{0})\rhd z.%
    \]%
    However, by the definition of $\rhd$, we have $x\rhd y_{0}=x'\rhd y_{0}$ for all $x,x'\in X$, preventing $\alpha^{\Sets_{*},\rhd}_{X,Y,Z}$ from being non-invertible.
\end{remark}
\begin{Proof}{Proof of \cref{the-right-skew-associator-of-the-right-tensor-product-of-pointed-sets}}%
    Firstly, note that, given $(X,x_{0}),(Y,y_{0}),(Z,z_{0})\in\Obj(\Sets_{*})$, the map
    \[
        \alpha^{\Sets_{*},\rhd}_{X,Y,Z}
        \colon
        X\rhd(Y\rhd Z)
        \to%
        (X\rhd Y)\rhd Z
    \]%
    is indeed a morphism of pointed sets, as we have
    \[
        \alpha^{\Sets_{*},\rhd}_{X,Y,Z}(x_{0}\rhd(y_{0}\rhd z_{0}))%
        =%
        (x_{0}\rhd y_{0})\rhd z_{0}.%
    \]%
    Next, we claim that $\alpha^{\Sets_{*},\rhd}$ is a natural transformation. We need to show that, given morphisms of pointed sets
    \begin{align*}
        f &\colon (X,x_{0}) \to (X',x'_{0}),\\
        g &\colon (Y,y_{0}) \to (Y',y'_{0}),\\
        h &\colon (Z,z_{0}) \to (Z',z'_{0})
    \end{align*}
    the diagram
    \[
        \begin{tikzcd}[row sep={5.0*\the\DL,between origins}, column sep={11.0*\the\DL,between origins}, background color=backgroundColor, ampersand replacement=\&]
            X\rhd(Y\rhd Z)
            \arrow[r,"{f\rhd(g\rhd h)}"]
            \arrow[d,"\alpha^{\Sets_{*},\rhd}_{X,Y,Z}"']
            \&
            X'\rhd(Y'\rhd Z')
            \arrow[d,"\alpha^{\Sets_{*},\rhd}_{X',Y',Z'}"]
            \\
            (X\rhd Y)\rhd Z
            \arrow[r,"{(f\rhd g)\rhd h}"']
            \&
            (X'\rhd Y')\rhd Z'
        \end{tikzcd}
    \]%
    commutes. Indeed, this diagram acts on elements as
    \[
        \begin{tikzcd}[row sep={5.0*\the\DL,between origins}, column sep={11.0*\the\DL,between origins}, background color=backgroundColor, ampersand replacement=\&]
            x\rhd(y\rhd z)
            \arrow[r,mapsto]
            \arrow[d,mapsto]
            \&
            f(x)\rhd(g(y)\rhd h(z))
            \arrow[d,mapsto]
            \\
            (x\rhd y)\rhd z
            \arrow[r,mapsto]
            \&
            (f(x)\rhd g(y))\rhd h(z)
        \end{tikzcd}
    \]%
    and hence indeed commutes, showing $\alpha^{\Sets_{*},\rhd}$ to be a natural transformation. This finishes the proof.
\end{Proof}
\subsection{The Right Skew Left Unitor}\label{subsection-the-right-tensor-product-of-pointed-sets-the-right-skew-left-unitor}
\begin{definition}{The Right Skew Left Unitor of $\rhd$}{the-right-skew-left-unitor-of-the-right-tensor-product-of-pointed-sets}%
    The \index[set-theory]{right tensor product of pointed sets!skew left unitor}\textbf{skew left unitor of the right tensor product of pointed sets} is the natural transformation
    \begin{webcompile}
        \LUnitor^{\Sets_{*},\rhd}%
        \colon%
        \bfLUnitor^{\TwoCategoryOfCategories}_{\Sets_{*}}
        \Longrightisoarrow
        {\rhd}\circ{(\Unit^{\Sets_{*}}\times\id_{\Sets_{*}})}
        \begin{tikzcd}[row sep={10.0*\the\DL,between origins}, column sep={10.0*\the\DL,between origins}, background color=backgroundColor, ampersand replacement=\&]
            \PunctualCategory\times\Sets_{*}
            \arrow[r, "\Unit^{\Sets_{*}}\times\sfid"]
            \arrow[rd, dashed,"\bfLUnitor^{\TwoCategoryOfCategories}_{\Sets_{*}}"'{name=1,pos=0.475},bend right=30]
            \&
            \Sets_{*}\times\Sets_{*}
            \arrow[d, "\rhd"]
            \\
            {}
            \&
            \Sets_{*}\mathrlap{,}
            % 2-Arrows
            \arrow[Rightarrow,from=1,to=1-2,shorten >=1.0*\the\DL,shorten <=1.0*\the\DL,"\LUnitor^{\Sets_{*},\rhd}"description]
        \end{tikzcd}
    \end{webcompile}%
    whose component
    \[
        \LUnitor^{\Sets_{*},\rhd}_{X}
        \colon
        X
        \to
        S^{0}\rhd X
    \]%
    at $(X,x_{0})\in\Obj(\Sets_{*})$ is given by the composition%
    \begin{align*}
        X &\shortrightarrow X\vee X\\
          &\cong            |S^{0}|\odot X\\
          &\cong            S^{0}\rhd X,
    \end{align*}
    where $X\to X\vee X$ is the map sending $X$ to the second factor of $X$ in $X\vee X$.
\end{definition}
\begin{Proof}{Proof of \cref{the-right-skew-left-unitor-of-the-right-tensor-product-of-pointed-sets}}%
    (Proven below in a bit.)
\end{Proof}
\begin{remark}{Unwinding \cref{the-right-skew-left-unitor-of-the-right-tensor-product-of-pointed-sets}}{unwinding-the-right-skew-left-unitor-of-the-right-tensor-product-of-pointed-sets}%
    In other words, $\smash{\LUnitor^{\Sets_{*},\rhd}_{X}}$ acts on elements as
    \[%
        \LUnitor^{\Sets_{*},\rhd}_{X}(x)%
        \defeq%
        [(1,x)]%
    \]%
    i.e.\ by
    \[%
        \LUnitor^{\Sets_{*},\rhd}_{X}(x)%
        \defeq%
        1\rhd x%
    \]%
    for each $x\in X$.
\end{remark}
\begin{remark}{Non-Invertibility of the Skew Left Unitor of $\rhd$}{non-invertibility-of-the-right-skew-left-unitor-of-the-right-tensor-product-of-pointed-sets}%
    The morphism $\smash{\LUnitor^{\Sets_{*},\rhd}_{X}}$ is non-invertible, as it is non-surjective when viewed as a map of sets, since the elements $0\rhd x$ of $S^{0}\rhd X$ with $x\neq x_{0}$ are outside the image of $\LUnitor^{\Sets_{*},\rhd}_{X}$, which sends $x$ to $1\rhd x$.
\end{remark}
\begin{Proof}{Proof of \cref{the-right-skew-left-unitor-of-the-right-tensor-product-of-pointed-sets}}%
    Firstly, note that, given $(X,x_{0})\in\Obj(\Sets_{*})$, the map
    \[
        \LUnitor^{\Sets_{*},\rhd}_{X}
        \colon
        X
        \to
        S^{0}\rhd X
    \]%
    is indeed a morphism of pointed sets, as we have
    \begin{align*}
        \LUnitor^{\Sets_{*},\rhd}_{X}(x_{0}) &= 1\rhd x_{0}\\%
                                             &= 0\rhd x_{0}.%
    \end{align*}
    Next, we claim that $\LUnitor^{\Sets_{*},\rhd}$ is a natural transformation. We need to show that, given a morphism of pointed sets
    \[
        f%
        \colon%
        (X,x_{0})%
        \to%
        (Y,y_{0}),%
    \]%
    the diagram
    \[
        \begin{tikzcd}[row sep={5.0*\the\DL,between origins}, column sep={7.5*\the\DL,between origins}, background color=backgroundColor, ampersand replacement=\&]
            X
            \arrow[r,"f"]
            \arrow[d,"\LUnitor^{\Sets_{*},\rhd}_{X}"']
            \&
            Y
            \arrow[d,"\LUnitor^{\Sets_{*},\rhd}_{Y}"]
            \\
            S^{0}\rhd X
            \arrow[r,"{\id_{S^{0}}\rhd f}"']
            \&
            S^{0}\rhd Y
        \end{tikzcd}
    \]%
    commutes. Indeed, this diagram acts on elements as
    \[
        \begin{tikzcd}[row sep={5.0*\the\DL,between origins}, column sep={6.0*\the\DL,between origins}, background color=backgroundColor, ampersand replacement=\&]
            x
            \arrow[r,mapsto]
            \arrow[d,mapsto]
            \&
            f(x)
            \arrow[d,mapsto]
            \\
            1\rhd x
            \arrow[r,mapsto]
            \&
            1\rhd f(x)
        \end{tikzcd}
    \]%
    and hence indeed commutes, showing $\LUnitor^{\Sets_{*},\rhd}$ to be a natural transformation. This finishes the proof.
\end{Proof}
\subsection{The Right Skew Right Unitor}\label{subsection-the-right-tensor-product-of-pointed-sets-the-right-skew-right-unitor}
\begin{definition}{The Right Skew Right Unitor of $\rhd$}{the-right-skew-right-unitor-of-the-right-tensor-product-of-pointed-sets}%
    The \index[set-theory]{right tensor product of pointed sets!skew right unitor}\textbf{skew right unitor of the right tensor product of pointed sets} is the natural transformation
    \begin{webcompile}
        \RUnitor^{\Sets_{*},\rhd}%
        \colon%
        {\rhd}\circ{({\sfid}\times{\Unit^{\Sets_{*}}})}%
        \Longrightisoarrow%
        \bfRUnitor^{\TwoCategoryOfCategories}_{\Sets_{*}},%
        \begin{tikzcd}[row sep={10.0*\the\DL,between origins}, column sep={10.0*\the\DL,between origins}, background color=backgroundColor, ampersand replacement=\&]
            \Sets_{*}\times\PunctualCategory
            \arrow[r, "\sfid\times\Unit^{\Sets_{*}}"]
            \arrow[rd, dashed,"\bfRUnitor^{\TwoCategoryOfCategories}_{\Sets_{*}}"'{name=1,pos=0.475},bend right=30]
            \&
            \Sets_{*}\times\Sets_{*}
            \arrow[d, "\rhd"]
            \\
            {}
            \&
            \Sets_{*}\mathrlap{,}
            % 2-Arrows
            \arrow[Rightarrow,from=1-2,to=1,shorten >=1.0*\the\DL,shorten <=1.0*\the\DL,"\RUnitor^{\Sets_{*},\rhd}"description]
        \end{tikzcd}
    \end{webcompile}%
    whose component
    \[
        \RUnitor^{\Sets_{*},\rhd}_{X}
        \colon
        X\rhd S^{0}%
        \to
        X
    \]%
    at $(X,x_{0})\in\Obj(\Sets_{*})$ is given by the composition%
    \begin{align*}
        X\rhd S^{0} &\cong            |X|\odot S^{0}\\
                    &\cong            \bigvee_{x\in X}S^{0}\\
                    &\shortrightarrow X,
    \end{align*}
    where $\bigvee_{x\in X}S^{0}\to X$ is the map given by
    \begin{align*}
        [(x,0)] &\mapsto x_{0},\\
        [(x,1)] &\mapsto x
    \end{align*}
    for each $x\in X$.
\end{definition}
\begin{Proof}{Proof of \cref{the-right-skew-right-unitor-of-the-right-tensor-product-of-pointed-sets}}%
    (Proven below in a bit.)
\end{Proof}
\begin{remark}{Unwinding \cref{the-right-skew-right-unitor-of-the-right-tensor-product-of-pointed-sets}}{unwinding-the-right-skew-right-unitor-of-the-right-tensor-product-of-pointed-sets}%
    In other words, $\smash{\RUnitor^{\Sets_{*},\rhd}_{X}}$ acts on elements as
    \begin{align*}
        \RUnitor^{\Sets_{*},\rhd}_{X}(x\rhd 0) &\defeq x_{0},\\%
        \RUnitor^{\Sets_{*},\rhd}_{X}(x\rhd 1) &\defeq x%
    \end{align*}
    for each $x\rhd 1\in X\rhd S^{0}$.
\end{remark}
\begin{remark}{Non-Invertibility of the Skew Right Unitor of $\rhd$}{non-invertibility-of-the-right-skew-right-unitor-of-the-right-tensor-product-of-pointed-sets}%
    The morphism $\smash{\RUnitor^{\Sets_{*},\rhd}_{X}}$ is almost invertible, with its would-be-inverse
    \[
        \phi_{X}%
        \colon%
        X%
        \to%
        X\rhd S^{0}%
    \]%
    given by
    \[
        \phi_{X}(x)%
        \defeq%
        x\rhd 1
    \]%
    for each $x\in X$. Indeed, we have
    \begin{align*}
        [\RUnitor^{\Sets_{*},\rhd}_{X}\circ\phi](x) &= \RUnitor^{\Sets_{*},\rhd}_{X}(\phi(x))\\
                                                    &= \RUnitor^{\Sets_{*},\rhd}_{X}(x\rhd 1)\\
                                                    &= x\\
                                                    &= [\id_{X}](x)
    \end{align*}
    so that
    \[
        \RUnitor^{\Sets_{*},\rhd}_{X}\circ\phi%
        =%
        \id_{X}%
    \]%
    and
    \begin{align*}
        [\phi\circ\RUnitor^{\Sets_{*},\rhd}_{X}](x\rhd 1) &= \phi(\RUnitor^{\Sets_{*},\rhd}_{X}(x\rhd 1))\\
                                                          &= \phi(x)\\
                                                          &= x\rhd 1\\
                                                          &= [\id_{X\rhd S^{0}}](x\rhd 1),
    \end{align*}
    but
    \begin{align*}
        [\phi\circ\RUnitor^{\Sets_{*},\rhd}_{X}](x\rhd 0) &= \phi(\RUnitor^{\Sets_{*},\rhd}_{X}(x\rhd 0))\\
                                                          &= \phi(x_{0})\\
                                                          &= 1\rhd x_{0},
    \end{align*}
    where $x\rhd 0\neq1\rhd x_{0}$. Thus
    \[
        \phi\circ\RUnitor^{\Sets_{*},\rhd}_{X}%
        \eqquestion%
        \id_{X\rhd S^{0}}%
    \]%
    holds for all elements in $X\rhd S^{0}$ except one.
\end{remark}
\begin{Proof}{Proof of \cref{the-right-skew-right-unitor-of-the-right-tensor-product-of-pointed-sets}}%
    Firstly, note that, given $(X,x_{0})\in\Obj(\Sets_{*})$, the map
    \[
        \RUnitor^{\Sets_{*},\rhd}_{X}
        \colon
        X\rhd S^{0}%
        \to
        X
    \]%
    is indeed a morphism of pointed sets as we have
    \[
        \RUnitor^{\Sets_{*},\rhd}_{X}(x_{0}\rhd0)%
        =%
        x_{0}.%
    \]%
    Next, we claim that $\RUnitor^{\Sets_{*},\rhd}$ is a natural transformation. We need to show that, given a morphism of pointed sets
    \[
        f%
        \colon%
        (X,x_{0})%
        \to%
        (Y,y_{0}),%
    \]%
    the diagram
    \[
        \begin{tikzcd}[row sep={5.0*\the\DL,between origins}, column sep={7.5*\the\DL,between origins}, background color=backgroundColor, ampersand replacement=\&]
            X\rhd S^{0}
            \arrow[r,"{f\rhd\id_{S^{0}}}"]
            \arrow[d,"\RUnitor^{\Sets_{*},\rhd}_{X}"']
            \&
            Y\rhd S^{0}
            \arrow[d,"\RUnitor^{\Sets_{*},\rhd}_{Y}"]
            \\
            X
            \arrow[r,"f"']
            \&
            Y
        \end{tikzcd}
    \]%
    commutes. Indeed, this diagram acts on elements as
    \begin{webcompile}
        \begin{tikzcd}[row sep={5.0*\the\DL,between origins}, column sep={6.0*\the\DL,between origins}, background color=backgroundColor, ampersand replacement=\&]
            x\rhd 0
            \arrow[d,mapsto]
            \&
            \\
            x_{0}
            \arrow[r,mapsto]
            \&
            f(x_{0})
        \end{tikzcd}
        \quad
        \begin{tikzcd}[row sep={5.0*\the\DL,between origins}, column sep={6.0*\the\DL,between origins}, background color=backgroundColor, ampersand replacement=\&]
            x\rhd 0
            \arrow[r,mapsto]
            \&
            f(x)\rhd0
            \arrow[d,mapsto]
            \\
            \&
            y_{0}
        \end{tikzcd}
    \end{webcompile}
    and
    \[
        \begin{tikzcd}[row sep={5.0*\the\DL,between origins}, column sep={6.0*\the\DL,between origins}, background color=backgroundColor, ampersand replacement=\&]
            x\rhd1
            \arrow[r,mapsto]
            \arrow[d,mapsto]
            \&
            f(x)\rhd1
            \arrow[d,mapsto]
            \\
            x
            \arrow[r,mapsto]
            \&
            f(x)
        \end{tikzcd}
    \]%
    and hence indeed commutes, showing $\RUnitor^{\Sets_{*},\rhd}$ to be a natural transformation. This finishes the proof.
\end{Proof}
\subsection{The Diagonal}\label{subsection-the-right-tensor-product-of-pointed-sets-the-diagonal}
\begin{definition}{The Diagonal of $\rhd$}{the-diagonal-of-the-right-tensor-product-of-pointed-sets}%
    The \index[set-theory]{right tensor product of pointed sets!diagonal}\textbf{diagonal of the right tensor product of pointed sets} is the natural transformation
    \begin{webcompile}
        \Delta^{\rhd}%
        \colon%
        \id_{\Sets_{*}}%
        \Longrightarrow%
        {\rhd}\circ{\Delta^{\TwoCategoryOfCategories}_{\Sets_{*}}},%
        \qquad%
        \begin{tikzcd}[row sep={5.0*\the\DL,between origins}, column sep={4.0*\the\DL,between origins}, background color=backgroundColor, ampersand replacement=\&]
            \Sets_{*}
            \arrow[rr,"\id_{\Sets_{*}}"{name=1},bend left=10]
            \arrow[rd,"\Delta^{\TwoCategoryOfCategories}_{\Sets_{*}}"'{pos=0.3},bend right=10]
            \&
            \&
            \Sets_{*}
            \\
            \&
            \Sets_{*}\times\Sets_{*}\mrp{,}
            \arrow[ru,"\rhd"'{pos=0.55},bend right=10]
            \&
            % 2-Arrows
            \arrow[from=1,to=2-2,"\Delta^{\rhd}"description,shorten <= 0.5*\the\DL,shorten >= 0.25*\the\DL,Rightarrow]%
        \end{tikzcd}
    \end{webcompile}%
    whose component
    \[
        \Delta^{\rhd}_{X}%
        \colon%
        (X,x_{0})%
        \to%
        (X\rhd X,x_{0}\rhd x_{0})%
    \]%
    at $(X,x_{0})\in\Obj(\Sets_{*})$ is given by
    \[
        \Delta^{\rhd}_{X}(x)%
        \defeq%
        x\rhd x%
    \]%
    for each $x\in X$.
\end{definition}
\begin{Proof}{Proof of \cref{the-diagonal-of-the-right-tensor-product-of-pointed-sets}}%
    \ProofBox{Being a Morphism of Pointed Sets}%
    We have
    \[
        \Delta^{\rhd}_{X}(x_{0})%
        \defeq%
        x_{0}\rhd x_{0},%
    \]%
    and thus $\Delta^{\rhd}_{X}$ is a morphism of pointed sets.

    \ProofBox{Naturality}%
    We need to show that, given a morphism of pointed sets
    \[
        f%
        \colon%
        (X,x_{0})%
        \to%
        (Y,y_{0}),%
    \]%
    the diagram
    \[
        \begin{tikzcd}[row sep={4.5*\the\DL,between origins}, column sep={6.5*\the\DL,between origins}, background color=backgroundColor, ampersand replacement=\&]
            X
            \arrow[r,"f"]
            \arrow[d,"\Delta^{\rhd}_{X}"']
            \&
            Y
            \arrow[d,"\Delta^{\rhd}_{Y}"]
            \\
            X\rhd X
            \arrow[r,"f\rhd f"']
            \&
            Y\rhd Y
        \end{tikzcd}
    \]%
    commutes. Indeed, this diagram acts on elements as
    \[
        \begin{tikzcd}[row sep={5.0*\the\DL,between origins}, column sep={7.0*\the\DL,between origins}, background color=backgroundColor, ampersand replacement=\&]
            x
            \arrow[r,mapsto]
            \arrow[d,mapsto]
            \&
            f(x)
            \arrow[d,mapsto]
            \\
            x\rhd x
            \arrow[r,mapsto]
            \&
            f(x)\rhd f(x)
        \end{tikzcd}
    \]%
    and hence indeed commutes, showing $\Delta^{\rhd}$ to be natural.
\end{Proof}
\subsection{The Right Skew Monoidal Structure on Pointed Sets Associated to $\rhd$}\label{subsection-the-right-skew-monoidal-structure-on-pointed-sets-associated-to-the-right-tensor-product-of-pointed-sets-of-pointed-sets}
\begin{proposition}{The Right Skew Monoidal Structure on Pointed Sets Associated to $\rhd$}{the-right-skew-monoidal-structure-on-pointed-sets-associated-to-the-right-tensor-product-of-pointed-sets}%
    The category $\Sets_{*}$ admits a right-closed right skew monoidal category structure consisting of:
    \begin{itemize}
        \item\SloganFont{The Underlying Category. }The category $\Sets_{*}$ of pointed sets.
        \item\SloganFont{The Right Skew Monoidal Product. }The right tensor product functor
            \[
                \rhd%
                \colon%
                \Sets_{*}\times\Sets_{*}%
                \to%
                \Sets_{*}
            \]%
            of \cref{the-right-tensor-product-of-pointed-sets}.
        \item\SloganFont{The Right Internal Skew Hom. }The right internal Hom functor
            \[
                [-,-]^{\rhd}_{\Sets_{*}}%
                \colon%
                \Sets^{\op}_{*}\times\Sets_{*}%
                \to%
                \Sets_{*}%
            \]%
            of \cref{the-right-internal-hom-of-pointed-sets}.
        \item\SloganFont{The Right Skew Monoidal Unit. }The functor
            \[
                \Unit^{\Sets_{*},\rhd}
                \colon
                \PunctualCategory
                \to
                \Sets_{*}
            \]
            of \cref{the-right-skew-unit-of-the-right-tensor-product-of-pointed-sets}.
        \item\SloganFont{The Right Skew Associators. }The natural transformation
            \[
                \alpha^{\Sets_{*},\rhd}
                \colon
                {\rhd}\circ{(\id_{\Sets_{*}}\times{\rhd})}
                \Longrightarrow%
                {\rhd}\circ{({\rhd}\times\id_{\Sets_{*}})}\circ{\bfalpha^{\Cats,-1}_{\Sets_{*},\Sets_{*},\Sets_{*}}}%
            \]
            of \cref{the-right-skew-associator-of-the-right-tensor-product-of-pointed-sets}.
        \item\SloganFont{The Right Skew Left Unitors. }The natural transformation
            \[
                \LUnitor^{\Sets_{*},\rhd}%
                \colon%
                \bfLUnitor^{\TwoCategoryOfCategories}_{\Sets_{*}}
                \Longrightisoarrow
                {\rhd}\circ{(\Unit^{\Sets_{*}}\times\id_{\Sets_{*}})}
            \]
            of \cref{the-right-skew-left-unitor-of-the-right-tensor-product-of-pointed-sets}.
        \item\SloganFont{The Right Skew Right Unitors. }The natural transformation
            \[
                \RUnitor^{\Sets_{*},\rhd}%
                \colon%
                {\rhd}\circ{({\sfid}\times{\Unit^{\Sets_{*}}})}%
                \Longrightisoarrow%
                \bfRUnitor^{\TwoCategoryOfCategories}_{\Sets_{*}}%
            \]
            of \cref{the-right-skew-right-unitor-of-the-right-tensor-product-of-pointed-sets}.
    \end{itemize}
\end{proposition}
\begin{Proof}{Proof of \cref{the-right-skew-monoidal-structure-on-pointed-sets-associated-to-the-right-tensor-product-of-pointed-sets}}%
    \ProofBox{The Pentagon Identity}%
    Let $(W,w_{0})$, $(X,x_{0})$, $(Y,y_{0})$ and $(Z,z_{0})$ be pointed sets. We have to show that the diagram
    \[
        \begin{tikzcd}[row sep={0*\the\DL,between origins}, column sep={0*\the\DL,between origins}, background color=backgroundColor, ampersand replacement=\&]
            \&[0.30901699437\FourCmPlusHalf]
            \&[0.5\FourCmPlusHalf]
            W\rhd((X\rhd Y)\rhd Z)
            \&[0.5\FourCmPlusHalf]
            \&[0.30901699437\FourCmPlusHalf]
            \\[0.58778525229\FourCmPlusHalf]
            W\rhd(X\rhd(Y\rhd Z))
            \&[0.30901699437\FourCmPlusHalf]
            \&[0.5\FourCmPlusHalf]
            \&[0.5\FourCmPlusHalf]
            \&[0.30901699437\FourCmPlusHalf]
            (W\rhd(X\rhd Y))\rhd Z
            \\[0.95105651629\FourCmPlusHalf]
            \&[0.30901699437\FourCmPlusHalf]
            (W\rhd X)\rhd(Y\rhd Z)
            \&[0.5\FourCmPlusHalf]
            \&[0.5\FourCmPlusHalf]
            ((W\rhd X)\rhd Y)\rhd Z
            \&[0.30901699437\FourCmPlusHalf]
            % 1-Arrows
            % Left Boundary
            \arrow[from=2-1,to=1-3,"\alpha^{\Sets_{*},\rhd}_{W,X,Y}\rhd\id_{Z}"{pos=0.4125}]%
            \arrow[from=1-3,to=2-5,"\alpha^{\Sets_{*},\rhd}_{W,X\rhd Y,Z}"{pos=0.6}]%
            \arrow[from=2-5,to=3-4,"\id_{W}\rhd\alpha^{\Sets_{*},\rhd}_{X,Y,Z}"{pos=0.425}]%
            % Right Boundary
            \arrow[from=2-1,to=3-2,"\alpha^{\Sets_{*},\rhd}_{W\rhd X,Y,Z}"'{pos=0.425}]%
            \arrow[from=3-2,to=3-4,"\alpha^{\Sets_{*},\rhd}_{W,X,Y\rhd Z}"']%
        \end{tikzcd}
    \]%
    commutes. Indeed, this diagram acts on elements as
    \[
        \begin{tikzcd}[row sep={0*\the\DL,between origins}, column sep={0*\the\DL,between origins}, background color=backgroundColor, ampersand replacement=\&]
            \&[0.30901699437\FourCmPlusHalf]
            \&[0.5\FourCmPlusHalf]
            w\rhd((x\rhd y)\rhd z)
            \&[0.5\FourCmPlusHalf]
            \&[0.30901699437\FourCmPlusHalf]
            \\[0.58778525229\FourCmPlusHalf]
            w\rhd(x\rhd(y\rhd z))
            \&[0.30901699437\FourCmPlusHalf]
            \&[0.5\FourCmPlusHalf]
            \&[0.5\FourCmPlusHalf]
            \&[0.30901699437\FourCmPlusHalf]
            (w\rhd(x\rhd y))\rhd z
            \\[0.95105651629\FourCmPlusHalf]
            \&[0.30901699437\FourCmPlusHalf]
            (w\rhd x)\rhd(y\rhd z)
            \&[0.5\FourCmPlusHalf]
            \&[0.5\FourCmPlusHalf]
            ((w\rhd x)\rhd y)\rhd z
            \&[0.30901699437\FourCmPlusHalf]
            % 1-Arrows
            % Left Boundary
            \arrow[from=2-1,to=1-3,mapsto]
            \arrow[from=1-3,to=2-5,mapsto]
            \arrow[from=2-5,to=3-4,mapsto]
            % Right Boundary
            \arrow[from=2-1,to=3-2,mapsto]
            \arrow[from=3-2,to=3-4,mapsto]
        \end{tikzcd}
    \]%
    and thus we see that the pentagon identity is satisfied.

    \ProofBox{The Right Skew Left Triangle Identity}%
    Let $(X,x_{0})$ and $(Y,y_{0})$ be pointed sets. We have to show that the diagram
    \[
        \begin{tikzcd}[row sep={5.0*\the\DL,between origins}, column sep={10.0*\the\DL,between origins}, background color=backgroundColor, ampersand replacement=\&]
            X\rhd Y
            \arrow[d,"\LUnitor^{\Sets_{*},\rhd}_{X\rhd Y}"']
            \arrow[rd,"\LUnitor^{\Sets_{*},\rhd}_{X}\rhd\id_{Y}"]
            \&
            \\
            S^{0}\rhd(X\rhd Y)
            \arrow[r,"\alpha^{\Sets_{*},\rhd}_{S^{0},X,Y}"']
            \&
            (S^{0}\rhd X)\rhd Y
        \end{tikzcd}
    \]%
    commutes. Indeed, this diagram acts on elements as
    \[
        \begin{tikzcd}[row sep={5.0*\the\DL,between origins}, column sep={8.0*\the\DL,between origins}, background color=backgroundColor, ampersand replacement=\&]
            x\rhd y
            \arrow[d,mapsto]
            \arrow[rd,mapsto]
            \&
            \\
            1\rhd(x\rhd y)
            \arrow[r,mapsto]
            \&
            (1\rhd x)\rhd y
        \end{tikzcd}
    \]%
    and hence indeed commutes. Thus the left skew triangle identity is satisfied.

    \ProofBox{The Right Skew Right Triangle Identity}%
    Let $(X,x_{0})$ and $(Y,y_{0})$ be pointed sets. We have to show that the diagram
    \[
        \begin{tikzcd}[row sep={5.0*\the\DL,between origins}, column sep={11.5*\the\DL,between origins}, background color=backgroundColor, ampersand replacement=\&]
            X\rhd(Y\rhd S^{0})
            \arrow[rd,"\alpha^{\Sets_{*},\rhd}_{S^{0},X,Y}"']
            \arrow[r,"\id_{X}\rhd\RUnitor^{\Sets_{*},\rhd}_{Y}"]
            \&
            (X\rhd Y)\rhd S^{0}%
            \arrow[d,"\RUnitor^{\Sets_{*},\rhd}_{X\rhd Y}"]
            \\
            \&
            X\rhd Y
        \end{tikzcd}
    \]%
    commutes. Indeed, this diagram acts on elements as
    \[
        \begin{tikzcd}[row sep={5.0*\the\DL,between origins}, column sep={8.0*\the\DL,between origins}, background color=backgroundColor, ampersand replacement=\&]
            x\rhd(y\rhd0)
            \arrow[r,mapsto]
            \arrow[rd,mapsto]
            \&
            (x\rhd y)\rhd 0%
            \arrow[d,mapsto]
            \\
            \&
            x\rhd y_{0}=x_{0}\rhd y_{0}
        \end{tikzcd}
    \]%
    and
    \[
        \begin{tikzcd}[row sep={5.0*\the\DL,between origins}, column sep={8.0*\the\DL,between origins}, background color=backgroundColor, ampersand replacement=\&]
            x\rhd(y\rhd1)
            \arrow[r,mapsto]
            \arrow[rd,mapsto]
            \&
            (x\rhd y)\rhd 1%
            \arrow[d,mapsto]
            \\
            \&
            x\rhd y
        \end{tikzcd}
    \]%
    and hence indeed commutes. Thus the right skew triangle identity is satisfied.

    \ProofBox{The Right Skew Middle Triangle Identity}%
    Let $(X,x_{0})$ and $(Y,y_{0})$ be pointed sets. We have to show that the diagram
    \[
        \begin{tikzcd}[row sep={5.0*\the\DL,between origins}, column sep={10.0*\the\DL,between origins}, background color=backgroundColor, ampersand replacement=\&]
            X\rhd Y
            \arrow[r,Equals]
            \arrow[d,"\id_{X}\rhd\LUnitor^{\Sets_{*},\rhd}_{Y}"']
            \&
            X\rhd Y
            \arrow[from=d,"\RUnitor^{\Sets_{*},\rhd}_{X}\rhd\id_{Y}"']
            \\
            X\rhd(S^{0}\rhd Y)
            \arrow[r,"\alpha^{\Sets_{*},\rhd}_{X,S^{0},Y}"']
            \&
            (X\rhd S^{0})\rhd Y
        \end{tikzcd}
    \]%
    commutes. Indeed, this diagram acts on elements as
    \[
        \begin{tikzcd}[row sep={5.0*\the\DL,between origins}, column sep={9.0*\the\DL,between origins}, background color=backgroundColor, ampersand replacement=\&]
            x\rhd y
            \arrow[r,mapsto]
            \arrow[d,mapsto]
            \&
            x\rhd y
            \arrow[from=d,mapsto]
            \\
            x\rhd(1\rhd y)
            \arrow[r,mapsto]
            \&
            (x\rhd 1)\rhd y
        \end{tikzcd}
    \]%
    and hence indeed commutes. Thus the right skew triangle identity is satisfied.

    \ProofBox{The Zig-Zag Identity}%
    We have to show that the diagram
    \[
        \begin{tikzcd}[row sep={6.0*\the\DL,between origins}, column sep={6.0*\the\DL,between origins}, background color=backgroundColor, ampersand replacement=\&]
            S^{0}
            \arrow[r,"\LUnitor^{\Sets_{*},\rhd}_{S^{0}}"]
            \arrow[rd,Equals]
            \&
            S^{0}\rhd S^{0}
            \arrow[d,"\RUnitor^{\Sets_{*},\rhd}_{S^{0}}"]
            \\
            \&
            S^{0}
        \end{tikzcd}
    \]%
    commutes. Indeed, this diagram acts on elements as
    \[
        \begin{tikzcd}[row sep={4.0*\the\DL,between origins}, column sep={4.0*\the\DL,between origins}, background color=backgroundColor, ampersand replacement=\&]
            0
            \arrow[r,mapsto]
            \arrow[rd,mapsto]
            \&
            1\rhd 0
            \arrow[d,mapsto]
            \\
            \&
            0
        \end{tikzcd}
    \]%
    and
    \[
        \begin{tikzcd}[row sep={4.0*\the\DL,between origins}, column sep={4.0*\the\DL,between origins}, background color=backgroundColor, ampersand replacement=\&]
            1
            \arrow[r,mapsto]
            \arrow[rd,mapsto]
            \&
            1\rhd 1
            \arrow[d,mapsto]
            \\
            \&
            1
        \end{tikzcd}
    \]%
    and hence indeed commutes. Thus the zig-zag identity is satisfied.

    \ProofBox{Right Skew Monoidal Right-Closedness}%
    This follows from \cref{properties-of-right-tensor-products-of-pointed-sets-adjointness-1} of \cref{properties-of-right-tensor-products-of-pointed-sets}.
\end{Proof}
\subsection{Monoids With Respect to the Right Tensor Product of Pointed Sets}\label{subsection-monoids-with-respect-to-the-right-tensor-product-of-pointed-sets}
\begin{proposition}{Monoids With Respect to $\rhd$}{monoids-with-respect-to-the-right-tensor-product-of-pointed-sets}%
    The category of monoids on $\smash{(\Sets_{*},\rhd,S^{0})}$ is isomorphic to the category of \say{monoids with right zero}%
    %--- Begin Footnote ---%
    \footnote{%
        A monoid with right zero is defined similarly as the monoids with zero of \ChapterMonoidsWithZero. Succinctly, they are monoids $(A,\mu_{A},\eta_{A})$ with a special element $0_{A}$ satisfying
        \[
            0_{A}a%
            =%
            0_{A}%
        \]%
        for each $a\in A$.
        \par\vspace*{\TCBBoxCorrection}
    } %
    %---  End Footnote  ---%
    and morphisms between them.
\end{proposition}
\begin{Proof}{Proof of \cref{monoids-with-respect-to-the-right-tensor-product-of-pointed-sets}}%
    \ProofBox{Monoids on $(\Sets_{*},\rhd,S^{0})$}%
    A monoid on $(\Sets_{*},\rhd,S^{0})$ consists of:
    \begin{itemize}
        \item\SloganFont{The Underlying Object. }A pointed set $(A,0_{A})$.
        \item\SloganFont{The Multiplication Morphism. }A morphism of pointed sets
            \[
                \mu_{A}%
                \colon%
                A\rhd A%
                \to%
                A,%
            \]%
            determining a right bilinear morphism of pointed sets
            \[
                \begin{tikzcd}[row sep=0.0*\the\DL, column sep=3.0*\the\DL, background color=backgroundColor, ampersand replacement=\&]
                    A\times A
                    \arrow[r]
                    \&
                    A
                    \\
                    (a,b)%
                    \arrow[r, mapsto]
                    \&
                    ab\mrp{.}%
                \end{tikzcd}
            \]%
        \item\SloganFont{The Unit Morphism. }A morphism of pointed sets
            \[
                \eta_{A}%
                \colon%
                S^{0}%
                \to%
                A%
            \]%
            picking an element $1_{A}$ of $A$.
    \end{itemize}
    satisfying the following conditions:
    \begin{enumerate}
        \item\label{proof-of-monoids-with-respect-to-the-right-tensor-product-of-pointed-sets-associativity}\SloganFont{Associativity. }The diagram
            \[
                \begin{tikzcd}[row sep={0*\the\DL,between origins}, column sep={0*\the\DL,between origins}, background color=backgroundColor, ampersand replacement=\&]
                    \&[0.30901699437\TwoCm]
                    \&[0.5\TwoCm]
                    A\rhd(A\rhd A)
                    \&[0.5\TwoCm]
                    \&[0.30901699437\TwoCm]
                    \\[0.58778525229\TwoCm]
                    (A\rhd A)\rhd A
                    \&[0.30901699437\TwoCm]
                    \&[0.5\TwoCm]
                    \&[0.5\TwoCm]
                    \&[0.30901699437\TwoCm]
                    A\rhd A
                    \\[0.95105651629\TwoCm]
                    \&[0.30901699437\TwoCm]
                    A\rhd A
                    \&[0.5\TwoCm]
                    \&[0.5\TwoCm]
                    A
                    \&[0.30901699437\TwoCm]
                    % 1-Arrows
                    % Left Boundary
                    \arrow[from=2-1,to=1-3,"\alpha^{\Sets_{*},\rhd}_{A,A,A}"{pos=0.4125}]%
                    \arrow[from=1-3,to=2-5,"\id_{A}\rhd\mu_{A}"{pos=0.6}]%
                    \arrow[from=2-5,to=3-4,"\mu_{A}"{pos=0.425}]%
                    % Right Boundary
                    \arrow[from=2-1,to=3-2,"\mu_{A}\rhd\id_{A}"'{pos=0.425}]%
                    \arrow[from=3-2,to=3-4,"\mu_{A}"']%
                \end{tikzcd}
            \]%
        \item\label{proof-of-monoids-with-respect-to-the-right-tensor-product-of-pointed-sets-left-unitality}\SloganFont{Left Unitality. }The diagram
            \[
                \begin{tikzcd}[row sep={5.0*\the\DL,between origins}, column sep={6.0*\the\DL,between origins}, background color=backgroundColor, ampersand replacement=\&]
                    A
                    \arrow[r,"\LUnitor^{\Sets_{*},\rhd}_{A}"]
                    \arrow[d,Equals]
                    \&
                    S^{0}\rhd A
                    \arrow[d,"\eta_{A}\times\id_{A}"]
                    \\
                    A
                    \&
                    A\rhd A
                    \arrow[l,"\mu_{A}"]
                \end{tikzcd}
            \]%
            commutes.
        \item\label{proof-of-monoids-with-respect-to-the-right-tensor-product-of-pointed-sets-right-unitality}\SloganFont{Right Unitality. }The diagram
            \[
                \begin{tikzcd}[row sep={5.0*\the\DL,between origins}, column sep={7.0*\the\DL,between origins}, background color=backgroundColor, ampersand replacement=\&]
                    A\rhd S^{0}
                    \arrow[r,"\id_{A}\times\eta_{A}"]
                    \arrow[rd,"\RUnitor^{\Sets_{*},\rhd}_{A}"']
                    \&
                    A\rhd A
                    \arrow[d,"\mu_{A}"]
                    \\
                    \&
                    A
                \end{tikzcd}
            \]%
            commutes.
    \end{enumerate}
    Being a right-bilinear morphism of pointed sets, the multiplication map satisfies
    \[
        0_{A}a%
        =%
        0_{A}%
    \]%
    for each $a\in A$. Now, the associativity, left unitality, and right unitality conditions act on elements as follows:
    \begin{enumerate}
        \item\label{proof-of-monoids-with-respect-to-the-right-tensor-product-of-pointed-sets-associativity-1}\SloganFont{Associativity. }The associativity condition acts as
            \begin{webcompile}
                \begin{tikzcd}[row sep={0*\the\DL,between origins}, column sep={0*\the\DL,between origins}, background color=backgroundColor, ampersand replacement=\&,cramped]
                    \&[0.30901699437\TwoCm]
                    \&[0.5\TwoCm]
                    \&[0.5\TwoCm]
                    \&[0.30901699437\TwoCm]
                    \\[0.58778525229\TwoCm]
                    (a\rhd b)\rhd c
                    \&[0.30901699437\TwoCm]
                    \&[0.5\TwoCm]
                    \&[0.5\TwoCm]
                    \&[0.30901699437\TwoCm]
                    \\[0.95105651629\TwoCm]
                    \&[0.30901699437\TwoCm]
                    ab\rhd c
                    \&[0.5\TwoCm]
                    \&[0.5\TwoCm]
                    (ab)c
                    \&[0.30901699437\TwoCm]
                    % 1-Arrows
                    % Right Boundary
                    \arrow[from=2-1,to=3-2,mapsto]
                    \arrow[from=3-2,to=3-4,mapsto]
                \end{tikzcd}
                \quad
                \begin{tikzcd}[row sep={0*\the\DL,between origins}, column sep={0*\the\DL,between origins}, background color=backgroundColor, ampersand replacement=\&,cramped]
                    \&[0.30901699437\TwoCm]
                    \&[0.5\TwoCm]
                    a\rhd(b\rhd c)
                    \&[0.5\TwoCm]
                    \&[0.30901699437\TwoCm]
                    \\[0.58778525229\TwoCm]
                    (a\rhd b)\rhd c
                    \&[0.30901699437\TwoCm]
                    \&[0.5\TwoCm]
                    \&[0.5\TwoCm]
                    \&[0.30901699437\TwoCm]
                    a\rhd bc
                    \\[0.95105651629\TwoCm]
                    \&[0.30901699437\TwoCm]
                    \&[0.5\TwoCm]
                    \&[0.5\TwoCm]
                    a(bc)
                    \&[0.30901699437\TwoCm]
                    % 1-Arrows
                    % Left Boundary
                    \arrow[from=2-1,to=1-3,mapsto]
                    \arrow[from=1-3,to=2-5,mapsto]
                    \arrow[from=2-5,to=3-4,mapsto]
                \end{tikzcd}
            \end{webcompile}
            This gives
            \[
                (ab)c%
                =%
                a(bc)%
            \]%
            for each $a,b,c\in A$.
        \item\label{proof-of-monoids-with-respect-to-the-right-tensor-product-of-pointed-sets-left-unitality-2}\SloganFont{Left Unitality. }The left unitality condition acts as
            \begin{webcompile}
                \begin{tikzcd}[row sep={5.0*\the\DL,between origins}, column sep={5.0*\the\DL,between origins}, background color=backgroundColor, ampersand replacement=\&]
                    a
                    \arrow[d,mapsto]
                    \&
                    \\
                    a
                    \&
                \end{tikzcd}
                \begin{tikzcd}[row sep={5.0*\the\DL,between origins}, column sep={5.0*\the\DL,between origins}, background color=backgroundColor, ampersand replacement=\&]
                    a
                    \arrow[r,mapsto]
                    \&
                    1\rhd a
                    \arrow[d,mapsto]
                    \\
                    1_{A}a
                    \&
                    1_{A}\rhd a
                    \arrow[l,mapsto]
                \end{tikzcd}
            \end{webcompile}
            This gives
            \[
                1_{A}a%
                =%
                a%
            \]%
            for each $a\in A$.
        \item\label{proof-of-monoids-with-respect-to-the-right-tensor-product-of-pointed-sets-right-unitality-2}\SloganFont{Right Unitality. }The right unitality condition acts:
            \begin{enumerate}
                \item\label{proof-of-monoids-with-respect-to-the-right-tensor-product-of-pointed-sets-right-unitality-2-a}On $1\rhd 0$ as
                    \begin{webcompile}
                        \begin{tikzcd}[row sep={5.0*\the\DL,between origins}, column sep={5.0*\the\DL,between origins}, background color=backgroundColor, ampersand replacement=\&]
                            1\rhd 0
                            \arrow[rd,mapsto]
                            \&
                            \\
                            \&
                            0_{A}
                        \end{tikzcd}
                        \quad
                        \begin{tikzcd}[row sep={5.0*\the\DL,between origins}, column sep={5.0*\the\DL,between origins}, background color=backgroundColor, ampersand replacement=\&]
                            a\rhd 0
                            \arrow[r,mapsto]
                            \arrow[rd,mapsto]
                            \&
                            a\rhd 0_{A}
                            \arrow[d,mapsto]
                            \\
                            \&
                            a0_{A}\mrp{.}
                        \end{tikzcd}
                    \end{webcompile}
                \item\label{proof-of-monoids-with-respect-to-the-right-tensor-product-of-pointed-sets-right-unitality-2-b}On $a\rhd 1$ as
                    \begin{webcompile}
                        \begin{tikzcd}[row sep={5.0*\the\DL,between origins}, column sep={5.0*\the\DL,between origins}, background color=backgroundColor, ampersand replacement=\&]
                            a\rhd 1
                            \arrow[rd,mapsto]
                            \&
                            \\
                            \&
                            a
                        \end{tikzcd}
                        \quad
                        \begin{tikzcd}[row sep={5.0*\the\DL,between origins}, column sep={5.0*\the\DL,between origins}, background color=backgroundColor, ampersand replacement=\&]
                            a\rhd 1
                            \arrow[r,mapsto]
                            \arrow[rd,mapsto]
                            \&
                            a\rhd 1_{A}
                            \arrow[d,mapsto]
                            \\
                            \&
                            a1_{A}\mrp{.}
                        \end{tikzcd}
                    \end{webcompile}
            \end{enumerate}
            This gives
            \begin{align*}
                a1_{A} &= a,\\
                a0_{A} &= 0_{A}
            \end{align*}
            for each $a\in A$.
    \end{enumerate}
    Thus we see that monoids with respect to $\rhd$ are exactly monoids with right zero.

    \ProofBox{Morphisms of Monoids on $(\Sets_{*},\rhd,S^{0})$}%
    A morphism of monoids on $(\Sets_{*},\rhd,S^{0})$ from $(A,\mu_{A},\eta_{A},0_{A})$ to $(B,\mu_{B},\eta_{B},0_{B})$ is a morphism of pointed sets
    \[
        f%
        \colon%
        (A,0_{A})%
        \to%
        (B,0_{B})%
    \]%
    satisfying the following conditions:
    \begin{enumerate}
        \item\label{proof-of-monoids-with-respect-to-the-right-tensor-product-of-pointed-sets-compatibility-with-the-multiplication-morphisms}\SloganFont{Compatibility With the Multiplication Morphisms. }The diagram
            \[
                \begin{tikzcd}[row sep={5.0*\the\DL,between origins}, column sep={6.5*\the\DL,between origins}, background color=backgroundColor, ampersand replacement=\&]
                    A\rhd A
                    \arrow[r,"f\rhd f"]
                    \arrow[d,"\mu_{A}"']
                    \&
                    B\rhd B
                    \arrow[d,"\mu_{B}"]
                    \\
                    A
                    \arrow[r,"f"']
                    \&
                    B
                \end{tikzcd}
            \]%
            commutes.
        \item\label{proof-of-monoids-with-respect-to-the-right-tensor-product-of-pointed-sets-compatibility-with-the-unit-morphisms}\SloganFont{Compatibility With the Unit Morphisms. }The diagram
            \[
                \begin{tikzcd}[row sep={5.0*\the\DL,between origins}, column sep={5.0*\the\DL,between origins}, background color=backgroundColor, ampersand replacement=\&]
                    S^{0}
                    \arrow[r,"\eta_{A}"]
                    \arrow[rd,"\eta_{B}"']
                    \&
                    A
                    \arrow[d,"f"]
                    \\
                    \&
                    B
                \end{tikzcd}
            \]%
            commutes.
    \end{enumerate}
    These act on elements as
    \begin{webcompile}
        \begin{tikzcd}[row sep={5.0*\the\DL,between origins}, column sep={6.5*\the\DL,between origins}, background color=backgroundColor, ampersand replacement=\&]
            a\rhd b
            \arrow[d,mapsto]
            \&
            \\
            ab
            \arrow[r,mapsto]
            \&
            f(ab)
        \end{tikzcd}
        \quad
        \begin{tikzcd}[row sep={5.0*\the\DL,between origins}, column sep={6.5*\the\DL,between origins}, background color=backgroundColor, ampersand replacement=\&]
            a\rhd b
            \arrow[r,mapsto]
            \&
            f(a)\rhd f(b)
            \arrow[d,mapsto]
            \\
            \&
            f(a)f(b)
        \end{tikzcd}
    \end{webcompile}
    and
    \begin{webcompile}
        \begin{tikzcd}[row sep={5.0*\the\DL,between origins}, column sep={5.0*\the\DL,between origins}, background color=backgroundColor, ampersand replacement=\&]
            0
            \arrow[rd,mapsto]
            \&
            \\
            \&
            0_{B}
        \end{tikzcd}
        \quad
        \begin{tikzcd}[row sep={5.0*\the\DL,between origins}, column sep={5.0*\the\DL,between origins}, background color=backgroundColor, ampersand replacement=\&]
            0
            \arrow[r,mapsto]
            \&
            0_{A}
            \arrow[d,mapsto]
            \\
            \&
            f(0_{A})
        \end{tikzcd}
    \end{webcompile}
    and
    \begin{webcompile}
        \begin{tikzcd}[row sep={5.0*\the\DL,between origins}, column sep={5.0*\the\DL,between origins}, background color=backgroundColor, ampersand replacement=\&]
            1
            \arrow[rd,mapsto]
            \&
            \\
            \&
            1_{B}
        \end{tikzcd}
        \quad
        \begin{tikzcd}[row sep={5.0*\the\DL,between origins}, column sep={5.0*\the\DL,between origins}, background color=backgroundColor, ampersand replacement=\&]
            1
            \arrow[r,mapsto]
            \&
            1_{A}
            \arrow[d,mapsto]
            \\
            \&
            f(1_{A})
        \end{tikzcd}
    \end{webcompile}
    giving
    \begin{gather*}
        f(ab)    = f(a)f(b),\\
        \begin{aligned}
            f(0_{A}) &= 0_{B},\\
            f(1_{A}) &= 1_{B},
        \end{aligned}
    \end{gather*}
    for each $a,b\in A$, which is exactly a morphism of monoids with right zero.

    \ProofBox{Identities and Composition}%
    Similarly, the identities and composition of $\Mon(\Sets_{*},\rhd,S^{0})$ can be easily seen to agree with those of monoids with right zero, which finishes the proof.
\end{Proof}
\section{The Smash Product of Pointed Sets}\label{section-the-smash-product-of-pointed-sets}
\subsection{Foundations}\label{subsection-smash-products-of-pointed-sets-foundations}
Let $(X,x_{0})$ and $(Y,y_{0})$ be pointed sets.
\begin{definition}{Smash Products of Pointed Sets}{smash-products-of-pointed-sets}%
    The \index[set-theory]{smash product!of pointed sets}\textbf{smash product of $(X,x_{0})$ and $(Y,y_{0})$}%
    %--- Begin Footnote ---%
    \footnote{%
        \SloganFont{Further Terminology: }In the context of monoids with zero as models for $\F_{1}$-algebras, the smash product $X\wedge Y$ is also called the \textbf{tensor product of $\F_{1}$-modules of $(X,x_{0})$ and $(Y,y_{0})$} or the \textbf{tensor product of $(X,x_{0})$ and $(Y,y_{0})$ over $\F_{1}$}.
    } %
    %---  End Footnote  ---%
    is the pointed set \index[notation]{XwedgeY@$X\wedge Y$}$X\wedge Y$%
    %--- Begin Footnote ---%
    \footnote{%
        \SloganFont{Further Notation: }In the context of monoids with zero as models for $\F_{1}$-algebras, the smash product $X\wedge Y$ is also denoted \index[notation]{XotimesF1Y@$X\otimes_{\F_{1}}Y$}$X\otimes_{\F_{1}}Y$.
        \par\vspace*{\TCBBoxCorrection}
    } %
    %---  End Footnote  ---%
    satisfying the bijection
    \[
        \Sets_{*}(X\wedge Y,Z)
        \cong
        \Hom^{\otimes}_{\Sets_{*}}(X\times Y,Z),
    \]%
    naturally in $(X,x_{0}),(Y,y_{0}),(Z,z_{0})\in\Obj(\Sets_{*})$.
\end{definition}
\begin{remark}{Unwinding \cref{smash-products-of-pointed-sets}: The Universal Property \rmI}{unwinding-smash-products-of-pointed-sets-the-universal-property-1}%
    That is to say, the smash product of pointed sets is defined so as to induce a bijection between the following data:
    \begin{itemize}
        \item Pointed maps $f\colon X\wedge Y\to Z$.
        \item Maps of sets $f\colon X\times Y\to Z$ satisfying
            \begin{align*}
                f(x_{0},y) &= z_{0},\\
                f(x,y_{0}) &= z_{0}
            \end{align*}
            for each $x\in X$ and each $y\in Y$.
    \end{itemize}
\end{remark}
\begin{remark}{Unwinding \cref{smash-products-of-pointed-sets}: The Universal Property \rmII}{unwinding-smash-products-of-pointed-sets-the-universal-property-2}%
    The smash product of pointed sets may be described as follows:
    \begin{itemize}
        \item The smash product of $(X,x_{0})$ and $(Y,y_{0})$ is the pair $((X\wedge Y,x_{0}\wedge y_{0}),\iota)$ consisting of
            \begin{itemize}
                \item A pointed set $(X\wedge Y,x_{0}\wedge y_{0})$;
                \item A bilinear morphism of pointed sets $\iota\colon(X\times Y,(x_{0},y_{0}))\to X\wedge Y$;
            \end{itemize}
            satisfying the following universal property:

            \begin{itemize}
                \itemstar Given another such pair $((Z,z_{0}),f)$ consisting of
                \begin{itemize}
                    \item A pointed set $(Z,z_{0})$;
                    \item A bilinear morphism of pointed sets $f\colon(X\times Y,(x_{0},y_{0}))\to X\wedge Y$;
                \end{itemize}
                there exists a unique morphism of pointed sets $X\wedge Y\uearrow Z$ making the diagram
                \[
                    \begin{tikzcd}[row sep={5.0*\the\DL,between origins}, column sep={5.0*\the\DL,between origins}, background color=backgroundColor, ampersand replacement=\&]
                        \&
                        X\wedge Y
                        \arrow[d,"\exists!",densely dashed]
                        \\
                        X\times Y
                        \arrow[r,"f"']
                        \arrow[ru,"\iota"]
                        \&
                        Z
                    \end{tikzcd}
                \]%
                commute.%
            \end{itemize}
    \end{itemize}
\end{remark}
\begin{construction}{Smash Products of Pointed Sets}{construction-of-smash-products-of-pointed-sets}%
    Concretely, the smash product of $(X,x_{0})$ and $(Y,y_{0})$ is the pointed set $(X\wedge Y,x_{0}\wedge y_{0})$ consisting of:%
    \begin{itemize}
        \item\SloganFont{The Underlying Set. }The set $X\wedge Y$ defined by
            \[
                X\wedge Y%
                \cong%
                (X\times Y)/\unsim_{R},%
            \]%
            where $\unsim_{R}$ is the equivalence relation on $X\times Y$ obtained by declaring
            \begin{align*}
                (x_{0},y) &\sim_{R} (x_{0},y'),\\
                (x,y_{0}) &\sim_{R} (x',y_{0})
            \end{align*}
            for each $x,x'\in X$ and each $y,y'\in Y$.
        \item\SloganFont{The Basepoint. }The element $[(x_{0},y_{0})]$ of $X\wedge Y$ given by the equivalence class of $(x_{0},y_{0})$ under the equivalence relation $\unsim$ on $X\times Y$.
    \end{itemize}
\end{construction}
\begin{Proof}{Proof of \cref{construction-of-smash-products-of-pointed-sets}}%
    By \ChapterRef{\ChapterConditionsOnRelations, \cref{conditions-on-relations:unwinding-the-quotient-of-a-set-by-an-equivalence-relation}}{\cref{unwinding-the-quotient-of-a-set-by-an-equivalence-relation}}, we have a natural bijection
    \[
        \Sets_{*}(X\wedge Y,Z)
        \cong
        \Hom^{R}_{\Sets}(X\times Y,Z),%
    \]%
    where $\Hom^{R}_{\Sets}(X\times Y,Z)$ is the set
    \begin{envsmallsize}
        \[
            \Hom^{R}_{\Sets}(X\times Y,Z)%
            \defeq%
            \{%
                f\in\Hom_{\Sets}(X\times Y,Z)%
                \ \middle|\ %
                \begin{aligned}
                    &\text{for each $x,y\in X$, if}\\
                    &\text{$(x,y)\sim_{R}(x',y')$, then}\\
                    &\text{$f(x,y)=f(x',y')$}%
                \end{aligned}
            \}.%
        \]%
    \end{envsmallsize}
    However, the condition $(x,y)\sim_{R}(x',y')$ only holds when:
    \begin{enumerate}
        \item\label{proof-of-construction-of-smash-products-of-pointed-sets-1}We have $x=x'$ and $y=y'$.
        \item\label{proof-of-construction-of-smash-products-of-pointed-sets-2}The following conditions are satisfied:
            \begin{enumerate}
                \item\label{proof-of-construction-of-smash-products-of-pointed-sets-2-a}We have $x=x_{0}$ or $y=y_{0}$.
                \item\label{proof-of-construction-of-smash-products-of-pointed-sets-2-b}We have $x'=x_{0}$ or $y'=y_{0}$.
            \end{enumerate}
    \end{enumerate}
    So, given $f\in\Hom_{\Sets}(X\times Y,Z)$ with a corresponding $\widebar{f}\colon X\wedge Y\to Z$, the latter case above implies
    \begin{align*}
        f(x_{0},y) &= f(x,y_{0})\\
                   &= f(x_{0},y_{0}),
    \end{align*}
    and since $\widebar{f}\colon X\wedge Y\to Z$ is a pointed map, we have
    \begin{align*}
        f(x_{0},y_{0}) &= \widebar{f}(x_{0},y_{0})\\
                       &= z_{0}.
    \end{align*}
    Thus the elements $f$ in $\Hom^{R}_{\Sets}(X\times Y,Z)$ are precisely those functions $f\colon X\times Y\to Z$ satisfying the equalities
    \begin{align*}
        f(x_{0},y) &= z_{0},\\
        f(x,y_{0}) &= z_{0}
    \end{align*}
    for each $x\in X$ and each $y\in Y$, giving an equality
    \[
        \Hom^{R}_{\Sets}(X\times Y,Z)%
        =%
        \Hom^{\otimes}_{\Sets_{*}}(X\times Y,Z)%
    \]%
    of sets, which when composed with our earlier isomorphism
    \[
        \Sets_{*}(X\wedge Y,Z)
        \cong
        \Hom^{R}_{\Sets}(X\times Y,Z),%
    \]%
    gives our desired natural bijection, finishing the proof.
\end{Proof}
\begin{remark}{On the Construction of the Smash Product of Pointed Sets}{on-the-construction-of-the-smash-product-of-pointed-sets}%
    It is also somewhat common to write
    \[
        X\wedge Y%
        \defeq%
        \frac{X\times Y}{X\vee Y},%
    \]%
    identifying $X\vee Y$ with the subspace $(\{x_{0}\}\times Y)\cup(X\times\{y_{0}\})$ of $X\times Y$, and having the quotient be defined by declaring $(x,y)\sim(x',y')$ \textiff we have $(x,y),(x',y')\in X\vee Y$.
\end{remark}
\begin{construction}{A Second Construction of the Smash Product of Pointed Sets}{a-second-construction-of-the-smash-product-of-pointed-sets}%
    Alternatively, the smash product of $(X,x_{0})$ and $(Y,y_{0})$ may be constructed as the pointed set $X\wedge Y$ given by
    \begin{align*}
        X\wedge Y &\cong \bigvee_{x\in X^{-}}Y\\%
                  &\cong \bigvee_{y\in Y^{-}}X.%
    \end{align*}
\end{construction}
\begin{Proof}{Proof of \cref{a-second-construction-of-the-smash-product-of-pointed-sets}}%
    Indeed, since $X\cong\bigvee_{x\in X^{-}}S^{0}$, we have
    \begin{align*}
        X\wedge Y &\cong (\bigvee_{x\in X^{-}}S^{0})\wedge Y\\
                  &\cong \bigvee_{x\in X^{-}}S^{0}\wedge Y\\
                  &\cong \bigvee_{x\in X^{-}}Y,
    \end{align*}
    where we have used that $\wedge$ preserves colimits in both variables by \cref{TODO} for the second isomorphism above, since it has right adjoints in both variables by \cref{properties-of-smash-products-of-pointed-sets-adjointness}.

    A similar proof applies to the isomorphism $X\wedge Y\cong\bigvee_{y\in Y^{-}}X$.
\end{Proof}
\begin{notation}{Elements of Smash Products of Pointed Sets}{elements-of-smash-products-of-pointed-sets}%
    We write \index[notation]{xsmashy@$x\wedge y$}$x\wedge y$ for the element $[(x,y)]$ of $X\wedge Y\cong X\times Y/\unsim$.%
\end{notation}
\begin{remark}{Basepoints of Smash Products of Pointed Sets}{basepoints-of-smash-products-of-pointed-sets}%
    Employing the notation introduced in \cref{elements-of-smash-products-of-pointed-sets}, we have
    \begin{align*}
        x_{0}\wedge y_{0} &= x\wedge y_{0},\\%
                          &= x_{0}\wedge y%
    \end{align*}
    for each $x\in X$ and each $y\in Y$, and
    \begin{align*}
        x\wedge y_{0} &= x'\wedge y_{0},\\%
        x_{0}\wedge y &= x_{0}\wedge y'%
    \end{align*}
    for each $x,x'\in X$ and each $y,y'\in Y$.
\end{remark}
\begin{example}{Examples of Smash Products of Pointed Sets}{examples-of-smash-products-of-pointed-sets}%
    Here are some examples of smash products of pointed sets.
    \begin{enumerate}
        \item\label{examples-of-smash-products-of-pointed-sets-smashing-with-pt}\SloganFont{Smashing With $\pt$. }For any pointed set $X$, we have isomorphisms of pointed sets
            \begin{align*}
                \pt\wedge X &\cong \pt,\\
                X\wedge\pt  &\cong \pt.
            \end{align*}
        \item\label{examples-of-smash-products-of-pointed-sets-smashing-with-s-zero}\SloganFont{Smashing With $S^{0}$. }For any pointed set $X$, we have isomorphisms of pointed sets
            \begin{align*}
                S^{0}\wedge X &\cong X,\\
                X\wedge S^{0} &\cong X.
            \end{align*}
    \end{enumerate}
\end{example}
\begin{proposition}{Properties of Smash Products of Pointed Sets}{properties-of-smash-products-of-pointed-sets}%
    Let $(X,x_{0})$ and $(Y,y_{0})$ be pointed sets.
    \begin{enumerate}
        \item\label{properties-of-smash-products-of-pointed-sets-functoriality}\SloganFont{Functoriality. }The assignments $X,Y,(X,Y)\mapsto X\wedge Y$ define functors
            \[
                \BifunctorialityPeriod{X\wedge-}{-\wedge Y}{-_{1}\wedge-_{2}}{\Sets_{*}}{\Sets_{*}}{\Sets_{*}\times\Sets_{*}}{\Sets_{*}}%
            \]%
            In particular, given pointed maps
            \begin{align*}
                f &\colon (X,x_{0}) \to (A,a_{0}),\\
                g &\colon (Y,y_{0}) \to (B,b_{0}),
            \end{align*}
            the induced map
            \[
                f\wedge g%
                \colon%
                X\wedge Y%
                \to%
                A\wedge B%
            \]%
            is given by
            \[
                [f\wedge g](x\wedge y)%
                \defeq%
                f(x)\wedge g(y)%
            \]%
            for each $x\wedge y\in X\wedge Y$.
        \item\label{properties-of-smash-products-of-pointed-sets-adjointness}\SloganFont{Adjointness. }We have adjunctions
            \begin{webcompile}
                \begin{gathered}
                    \Adjunction#X\wedge-#{\eSets_{*}(X,-)}#\Sets_{*}#\Sets_{*},#\\
                    \Adjunction#-\wedge Y#{\eSets_{*}(Y,-)}#\Sets_{*}#\Sets_{*},#
                \end{gathered}
            \end{webcompile}%
            witnessed by bijections
            \begin{align*}
                \Hom_{\Sets_{*}}(X\wedge Y,Z) &\cong \Hom_{\Sets_{*}}(X,\eSets_{*}(Y,Z)),\\
                \Hom_{\Sets_{*}}(X\wedge Y,Z) &\cong \Hom_{\Sets_{*}}(X,\eSets_{*}(A,Z)),
            \end{align*}
            natural in $(X,x_{0}),(Y,y_{0}),(Z,z_{0})\in\Obj(\Sets_{*})$.
        \item\label{properties-of-smash-products-of-pointed-sets-enriched-adjointness}\SloganFont{Enriched Adjointness. }We have $\Sets_{*}$-enriched adjunctions
            \begin{webcompile}
                \begin{gathered}
                    \Adjunction#X\wedge-#{\eSets_{*}(X,-)}#\eSets_{*}#\eSets_{*},#\\
                    \Adjunction#-\wedge Y#{\eSets_{*}(Y,-)}#\eSets_{*}#\eSets_{*},#
                \end{gathered}
            \end{webcompile}%
            witnessed by isomorphisms of pointed sets
            \begin{align*}
                \eSets_{*}(X\wedge Y,Z) &\cong \eSets_{*}(X,\eSets_{*}(Y,Z)),\\
                \eSets_{*}(X\wedge Y,Z) &\cong \eSets_{*}(X,\eSets_{*}(A,Z)),
            \end{align*}
            natural in $(X,x_{0}),(Y,y_{0}),(Z,z_{0})\in\Obj(\eSets_{*})$.
        \item\label{properties-of-smash-products-of-pointed-sets-as-a-pushout}\SloganFont{As a Pushout. }We have an isomorphism
            \begin{webcompile}
                X\wedge Y%
                \cong%
                \pt\iipushout{X\vee Y}(X\times Y),%
                \qquad
                \begin{tikzcd}[row sep={5.0*\the\DL,between origins}, column sep={5.0*\the\DL,between origins}, background color=backgroundColor, ampersand replacement=\&]
                    X\wedge Y
                    \arrow[from=r]
                    \arrow[from=d]
                    \arrow[rd,very near start,phantom,"\ulcorner"]
                    \&
                    X\times Y
                    \arrow[from=d,hook,"\iota"']
                    \\
                    \pt
                    \arrow[from=r,"!"]
                    \&
                    X\vee Y\mrp{,}
                \end{tikzcd}
            \end{webcompile}%
            natural in $X,Y\in\Obj(\Sets_{*})$, where the pushout is taken in $\Sets$, and the embedding $\iota\colon X\vee Y\hookrightarrow X\times Y$ is defined following \cref{on-the-construction-of-the-smash-product-of-pointed-sets}.
        \item\label{properties-of-smash-products-of-pointed-sets-distributivity-over-wedge-sums}\SloganFont{Distributivity Over Wedge Sums. }We have isomorphisms of pointed sets
            \begin{align*}
                X\wedge(Y\vee Z)  &\cong (X\wedge Y)\vee(X\wedge Z),\\
                (X\vee Y)\wedge Z &\cong (X\wedge Z)\vee(Y\wedge Z),
            \end{align*}
            natural in $(X,x_{0}),(Y,y_{0}),(Z,z_{0})\in\Obj(\Sets_{*})$.
    \end{enumerate}
\end{proposition}
\begin{Proof}{Proof of \cref{properties-of-smash-products-of-pointed-sets}}%
    \FirstProofBox{\cref{properties-of-smash-products-of-pointed-sets-functoriality}: Functoriality}%
    The map $f\wedge g$ comes from \ChapterRef{\ChapterConditionsOnRelations, \cref{conditions-on-relations:properties-of-quotient-sets-descending-functions-to-quotient-sets-1} of \cref{conditions-on-relations:properties-of-quotient-sets}}{\cref{properties-of-quotient-sets-descending-functions-to-quotient-sets-1} of \cref{properties-of-quotient-sets}} via the map
    \[
        f\wedge g%
        \colon%
        X\times Y%
        \to%
        A\wedge B%
    \]%
    sending $(x,y)$ to $f(x)\wedge g(y)$, which we need to show satisfies
    \[
        [f\wedge g](x,y)%
        =%
        [f\wedge g](x',y')%
    \]%
    for each $(x,y),(x',y')\in X\times Y$ with $(x,y)\sim_{R}(x',y')$, where $\unsim_{R}$ is the relation constructing $X\wedge Y$ as
    \[
        X\wedge Y%
        \cong%
        (X\times Y)/\unsim_{R}%
    \]%
    in \cref{construction-of-smash-products-of-pointed-sets}. The condition defining $\unsim$ is that at least one of the following conditions is satisfied:
    \begin{enumerate}
        \item\label{proof-of-properties-of-smash-products-of-pointed-sets-functoriality-1}We have $x=x'$ and $y=y'$;
        \item\label{proof-of-properties-of-smash-products-of-pointed-sets-functoriality-2}Both of the following conditions are satisfied:
            \begin{enumerate}
                \item\label{proof-of-properties-of-smash-products-of-pointed-sets-functoriality-2-a}We have $x=x_{0}$ or $y=y_{0}$.
                \item\label{proof-of-properties-of-smash-products-of-pointed-sets-functoriality-2-b}We have $x'=x_{0}$ or $y'=y_{0}$.
            \end{enumerate}
    \end{enumerate}
    We have five cases:
    \begin{enumerate}
        \item\label{proof-of-properties-of-smash-products-of-pointed-sets-functoriality-3}In the first case, we clearly have
            \[
                [f\wedge g](x,y)%
                =%
                [f\wedge g](x',y')%
            \]%
            since $x=x'$ and $y=y'$.
        \item\label{proof-of-properties-of-smash-products-of-pointed-sets-functoriality-4}If $x=x_{0}$ and $x'=x_{0}$, we have 
            \begin{align*}
                [f\wedge g](x_{0},y) &\defeq f(x_{0})\wedge g(y)\\%
                                     &=      a_{0}\wedge g(y)\\%
                                     &=      a_{0}\wedge g(y')\\%
                                     &=      f(x_{0})\wedge g(y')\\%
                                     &\defeq [f\wedge g](x_{0},y').%
            \end{align*}
        \item\label{proof-of-properties-of-smash-products-of-pointed-sets-functoriality-5}If $x=x_{0}$ and $y'=y_{0}$, we have 
            \begin{align*}
                [f\wedge g](x_{0},y) &\defeq f(x_{0})\wedge g(y)\\%
                                     &=      a_{0}\wedge g(y)\\%
                                     &=      a_{0}\wedge b_{0}\\%
                                     &=      f(x')\wedge b_{0}\\%
                                     &=      f(x')\wedge g(y_{0})\\%
                                     &\defeq [f\wedge g](x',y_{0}).%
            \end{align*}
        \item\label{proof-of-properties-of-smash-products-of-pointed-sets-functoriality-6}If $y=y_{0}$ and $x'=x_{0}$, we have 
            \begin{align*}
                [f\wedge g](x,y_{0}) &\defeq f(x)\wedge g(y_{0})\\%
                                     &=      f(x)\wedge b_{0}\\%
                                     &=      a_{0}\wedge b_{0}\\%
                                     &=      a_{0}\wedge g(y')\\%
                                     &=      f(x_{0})\wedge g(y')\\%
                                     &\defeq [f\wedge g](x_{0},y').%
            \end{align*}
        \item\label{proof-of-properties-of-smash-products-of-pointed-sets-functoriality-7}If $y=y_{0}$ and $y'=y_{0}$, we have 
            \begin{align*}
                [f\wedge g](x,y_{0}) &\defeq f(x)\wedge g(y_{0})\\%
                                     &=      f(x)\wedge b_{0}\\%
                                     &=      f(x')\wedge b_{0}\\%
                                     &=      f(x)\wedge g(y_{0})\\%
                                     &\defeq [f\wedge g](x',y_{0}).%
            \end{align*}
    \end{enumerate}
    Thus $f\wedge g$ is well-defined. Next, we claim that $\wedge$ preserves identities and composition:
    \begin{itemize}
        \item\SloganFont{Preservation of Identities. }We have
            \begin{align*}
                [\id_{X}\wedge\id_{Y}](x\wedge y) &\defeq \id_{X}(x)\wedge\id_{Y}(y)\\
                                                  &=      x\wedge y\\
                                                  &=      [\id_{X\wedge Y}](x\wedge y)
            \end{align*}
            for each $x\wedge y\in X\wedge Y$, and thus
            \[
                \id_{X}\wedge\id_{Y}%
                =%
                \id_{X\wedge Y}.%
            \]%
        \item\SloganFont{Preservation of Composition. }Given pointed maps
            \begin{align*}
                f &\colon (X,x_{0})   \to (X',x'_{0}),\\
                h &\colon (X',x'_{0}) \to (X'',x''_{0}),\\
                g &\colon (Y,y_{0}) \to (Y',y'_{0}),\\
                k &\colon (Y',y'_{0}) \to (Y^{\prime\prime},y^{\prime\prime}_{0}),
            \end{align*}
            we have
            \begin{align*}
                [(h\circ f)\wedge(k\circ g)](x\wedge y) &\defeq h(f(x))\wedge k(g(y))\\
                                                        &\defeq [h\wedge k](f(x)\wedge g(y))\\
                                                        &\defeq [h\wedge k]([f\wedge g](x\wedge y))\\
                                                        &\defeq [(h\wedge k)\circ(f\wedge g)](x\wedge y)
            \end{align*}
            for each $x\wedge y\in X\wedge Y$, and thus
            \[
                (h\circ f)\wedge(k\circ g)%
                =%
                (h\wedge k)\circ(f\wedge g).%
            \]%
    \end{itemize}
    This finishes the proof.

    \ProofBox{\cref{properties-of-smash-products-of-pointed-sets-adjointness}: Adjointness}%
    We prove only the adjunction $-\wedge Y\dashv\eSets_{*}(Y,-)$, witnessed by a natural bijection
    \[
        \Hom_{\Sets_{*}}(X\wedge Y,Z)%
        \cong%
        \Hom_{\Sets_{*}}(X,\eSets_{*}(Y,Z)),%
    \]%
    as the proof of the adjunction $X\wedge-\dashv\eSets_{*}(X,-)$ is similar. We claim we have a bijection
    \[
        \Hom^{\otimes}_{\Sets_{*}}(X\times Y,Z)%
        \cong%
        \Hom_{\Sets_{*}}(X,\eSets_{*}(Y,Z))%
    \]%
    natural in $(X,x_{0}),(Y,y_{0}),(Z,z_{0})\in\Obj(\Sets_{*})$, impliying the desired adjunction. Indeed, this bijection is a restriction of the bijection
    \[
        \Sets(X\times Y,Z)%
        \cong
        \Sets(X,\Sets(Y,Z))%
    \]%
    of \ChapterRef{\ChapterConstructionsWithSets, \cref{constructions-with-sets:properties-of-products-of-sets-adjointness-1} of \cref{constructions-with-sets:properties-of-products-of-sets}}{\cref{properties-of-products-of-sets-adjointness-1} of \cref{properties-of-products-of-sets}}:%
    \begin{itemize}
        \item A map
            \[
                \xi%
                \colon%
                X\times Y%
                \to%
                Z%
            \]%
            in $\Hom^{\otimes}_{\Sets_{*}}(X\times Y,Z)$ gets sent to the pointed map
            \begin{webcompile}
                \phantom{\xi^{\dagger}\colon}
                \begin{tikzcd}[row sep=0.0*\the\DL, column sep=1.0*\the\DL, background color=backgroundColor, ampersand replacement=\&]
                    {\mathllap{\xi^{\dagger}\colon}(X,x_{0})}%
                    \arrow[r]
                    \&
                    {(\eSets_{*}(Y,Z),\Delta_{z_{0}})\mrp{,}}%
                    \\
                    x
                    \arrow[r, mapsto]
                    \&
                    {(\xi^{\dagger}_{x}\colon Y\to Z)\mrp{,}}
                \end{tikzcd}
            \end{webcompile}
            where $\xi^{\dagger}_{x}\colon Y\to Z$ is the map defined by
            \[
                \xi^{\dagger}_{x}(y)%
                \defeq%
                \xi(x,y)%
            \]%
            for each $y\in Y$, where:
            \begin{itemize}
                \item The map $\xi^{\dagger}$ is indeed pointed, as we have
                    \begin{align*}
                        \xi^{\dagger}_{x_{0}}(y) &\defeq \xi(x_{0},y)\\%
                                                 &\defeq z_{0}%
                    \end{align*}
                    for each $y\in Y$. Thus $\xi^{\dagger}_{x_{0}}=\Delta_{z_{0}}$ and $\xi^{\dagger}$ is pointed.
                \item The map $\xi^{\dagger}_{x}$ indeed lies in $\eSets_{*}(Y,Z)$, as we have%
                    \begin{align*}
                        \xi^{\dagger}_{x}(y_{0}) &\defeq \xi(x,y_{0})\\%
                                                 &\defeq z_{0}.%
                    \end{align*}
            \end{itemize}
        \item Conversely, a map
            \begin{webcompile}
                \phantom{\xi\colon}
                \begin{tikzcd}[row sep=0.0*\the\DL, column sep=1.0*\the\DL, background color=backgroundColor, ampersand replacement=\&]
                    {\mathllap{\xi\colon}(X,x_{0})}%
                    \arrow[r]
                    \&
                    {(\eSets_{*}(Y,Z),\Delta_{z_{0}})\mrp{,}}%
                    \\
                    x
                    \arrow[r, mapsto]
                    \&
                    {(\xi_{x}\colon Y\to Z)\mrp{,}}
                \end{tikzcd}
            \end{webcompile}
            in $\Hom_{\Sets_{*}}(X,\eSets_{*}(Y,Z))$ gets sent to the map
            \[
                \xi^{\dagger}%
                \colon%
                X\times Y%
                \to%
                Z%
            \]%
            defined by
            \[
                \xi^{\dagger}(x,y)%
                \defeq%
                \xi_{x}(y)%
            \]%
            for each $(x,y)\in X\times Y$, which indeed lies in $\Hom^{\otimes}_{\Sets_{*}}(X\times Y,Z)$, as:
            \begin{itemize}
                \item\SloganFont{Left Bilinearity. }We have
                    \begin{align*}
                        \xi^{\dagger}(x_{0},y) &\defeq \xi_{x_{0}}(y)\\%
                                               &\defeq \Delta_{z_{0}}(y)\\%
                                               &\defeq z_{0}%
                    \end{align*}
                    for each $y\in Y$, since $\xi_{x_{0}}=\Delta_{z_{0}}$ as $\xi$ is assumed to be a pointed map.
                \item\SloganFont{Right Bilinearity. }We have
                    \begin{align*}
                        \xi^{\dagger}(x,y_{0}) &\defeq \xi_{x}(y_{0})\\%
                                               &\defeq z_{0}%
                    \end{align*}
                    for each $x\in X$, since $\xi_{x}\in\eSets_{*}(Y,Z)$ is a morphism of pointed sets.
            \end{itemize}
    \end{itemize}
    This finishes the proof.

    \ProofBox{\cref{properties-of-smash-products-of-pointed-sets-enriched-adjointness}: Enriched Adjointness}%
    This follows from \cref{properties-of-smash-products-of-pointed-sets-adjointness} and \ChapterRef{\ChapterMonoidalCategories, \cref{monoidal-categories:properties-of-closed-monoidal-categories-internalising-the-tensor-internal-hom-adjunction} of \cref{monoidal-categories:properties-of-closed-monoidal-categories}}{\cref{properties-of-closed-monoidal-categories-internalising-the-tensor-internal-hom-adjunction} of \cref{properties-of-closed-monoidal-categories}}.

    \ProofBox{\cref{properties-of-smash-products-of-pointed-sets-as-a-pushout}: As a Pushout}%
    Following the description of \ChapterRef{\ChapterConstructionsWithSets, \cref{constructions-with-sets:unwinding-pushouts-of-sets}}{\cref{unwinding-pushouts-of-sets}}, we have
    \[
        \pt\icoprod_{X\vee Y}(X\times Y)%
        \cong%
        (\pt\times(X\times Y))/\unsim,%
    \]%
    where $\unsim$ identifies the elemenet $\point$ in $\pt$ with all elements of the form $(x_{0},y)$ and $(x,y_{0})$ in $X\times Y$. Thus \ChapterRef{\ChapterConditionsOnRelations, \cref{conditions-on-relations:properties-of-quotient-sets-descending-functions-to-quotient-sets-1} of \cref{conditions-on-relations:properties-of-quotient-sets}}{\cref{properties-of-quotient-sets-descending-functions-to-quotient-sets-1} of \cref{properties-of-quotient-sets}} coupled with \cref{basepoints-of-smash-products-of-pointed-sets} then gives us a well-defined map
    \[
        \pt\icoprod_{X\vee Y}(X\times Y)%
        \to%
        X\wedge Y%
    \]%
    via $[(\point,(x,y))]\mapsto x\wedge y$, with inverse 
    \[
        X\wedge Y%
        \to%
        \pt\icoprod_{X\vee Y}(X\times Y)%
    \]%
    given by $x\wedge y\mapsto[(\point,(x,y))]$.

    \ProofBox{\cref{properties-of-smash-products-of-pointed-sets-distributivity-over-wedge-sums}: Distributivity Over Wedge Sums}%
    This follows from \cref{the-monoidal-structure-on-pointed-sets-associated-to-the-smash-product-of-pointed-sets}, \ChapterRef{\ChapterMonoidalCategories, \cref{monoidal-categories:properties-of-closed-monoidal-categories-interaction-with-coproducts} of \cref{monoidal-categories:properties-of-closed-monoidal-categories}}{\cref{properties-of-closed-monoidal-categories-interaction-with-coproducts} of \cref{properties-of-closed-monoidal-categories}}, and the fact that $\vee$ is the coproduct in $\Sets_{*}$ (\ChapterRef{\ChapterPointedSets, \cref{pointed-sets:coproducts-of-pointed-sets}}{\cref{coproducts-of-pointed-sets}}).
\end{Proof}
\subsection{The Internal Hom of Pointed Sets}\label{subsection-the-internal-hom-of-pointed-sets}
Let $(X,x_{0})$ and $(Y,y_{0})$ be pointed sets.
\begin{definition}{The Internal Hom of Pointed Sets}{the-internal-hom-of-pointed-sets}%
    The \index[set-theory]{pointed set!of morphisms of Fonemodules@of morphisms of pointed sets}\index[set-theory]{pointed set!internal Hom of}\textbf{internal Hom}%
    %--- Begin Footnote ---%
    \footnote{%
        For a proof that $\eSets_{*}$ is indeed the internal Hom of $\Sets_{*}$ with respect to the smash product of pointed sets, see \cref{properties-of-smash-products-of-pointed-sets-adjointness} of \cref{properties-of-smash-products-of-pointed-sets}.
    } %
    %---  End Footnote  ---%
    \textbf{of pointed sets from $(X,x_{0})$ to $(Y,y_{0})$} is the pointed set \index[notation]{SetsstarXxzeroYyzero@$\eSets_{*}((X,x_{0}),(Y,y_{0}))$}$\eSets_{*}((X,x_{0}),(Y,y_{0}))$%
    %--- Begin Footnote ---%
    \footnote{%
        \SloganFont{Further Notation: }Also written \index[notation]{HomSetsstarXY@$\eHom_{\eSets_{*}}(X,Y)$}$\eHom_{\eSets_{*}}(X,Y)$.
        \par\vspace*{\TCBBoxCorrection}
    } %
    %---  End Footnote  ---%
    consisting of:
    \begin{itemize}
        \item\SloganFont{The Underlying Set. }The set $\Sets_{*}((X,x_{0}),(Y,y_{0}))$ of morphisms of pointed sets from $(X,x_{0})$ to $(Y,y_{0})$.
        \item\SloganFont{The Basepoint. }The element%
            \[%
                \Delta_{y_{0}}%
                \colon%
                (X,x_{0})%
                \to%
                (Y,y_{0})%
            \]%
            of $\Sets_{*}((X,x_{0}),(Y,y_{0}))$ given by
            \[
                \Delta_{y_{0}}(x)%
                \defeq%
                y_{0}%
            \]%
            for each $x\in X$.
    \end{itemize}
\end{definition}
\begin{proposition}{Properties of the Internal Hom of Pointed Sets}{properties-of-the-internal-hom-of-pointed-sets}%
    Let $(X,x_{0})$ and $(Y,y_{0})$ be pointed sets.
    \begin{enumerate}
        \item\label{properties-of-the-internal-hom-of-pointed-sets-functoriality}\SloganFont{Functoriality. }The assignments $X,Y,(X,Y)\mapsto\eSets_{*}(X,Y)$ define functors
            \[
                \BifunctorialityPeriod{\eSets_{*}(X,-)}{\eSets_{*}(-,Y)}{\eSets_{*}(-_{1},-_{2})}{\Sets_{*}}{\Sets^{\mrp{\op}}_{*}}{\Sets^{\op}_{*}\times\Sets_{*}}{\Sets_{*}}%
            \]%
            In particular, given pointed maps
            \begin{align*}
                f &\colon (X,x_{0}) \to (A,a_{0}),\\
                g &\colon (Y,y_{0}) \to (B,b_{0}),
            \end{align*}
            the induced map
            \[
                \eSets_{*}(f,g)%
                \colon%
                \eSets_{*}(A,Y)%
                \to%
                \eSets_{*}(X,B)%
            \]%
            is given by
            \[
                [\eSets_{*}(f,g)](\phi)%
                \defeq%
                g\circ\phi\circ f%
            \]%
            for each $\phi\in\eSets_{*}(A,Y)$.
        \item\label{properties-of-the-internal-hom-of-pointed-sets-adjointness}\SloganFont{Adjointness. }We have adjunctions
            \begin{webcompile}
                \begin{gathered}
                    \Adjunction#X\wedge-#{\eSets_{*}(X,-)}#\Sets_{*}#\Sets_{*},#\\
                    \Adjunction#-\wedge Y#{\eSets_{*}(Y,-)}#\Sets_{*}#\Sets_{*},#
                \end{gathered}
            \end{webcompile}%
            witnessed by bijections
            \begin{align*}
                \Hom_{\Sets_{*}}(X\wedge Y,Z) &\cong \Hom_{\Sets_{*}}(X,\eSets_{*}(Y,Z)),\\
                \Hom_{\Sets_{*}}(X\wedge Y,Z) &\cong \Hom_{\Sets_{*}}(X,\eSets_{*}(A,Z)),
            \end{align*}
            natural in $(X,x_{0}),(Y,y_{0}),(Z,z_{0})\in\Obj(\Sets_{*})$.
        \item\label{properties-of-the-internal-hom-of-pointed-sets-enriched-adjointness}\SloganFont{Enriched Adjointness. }We have $\Sets_{*}$-enriched adjunctions
            \begin{webcompile}
                \begin{gathered}
                    \Adjunction#X\wedge-#{\eSets_{*}(X,-)}#\eSets_{*}#\eSets_{*},#\\
                    \Adjunction#-\wedge Y#{\eSets_{*}(Y,-)}#\eSets_{*}#\eSets_{*},#
                \end{gathered}
            \end{webcompile}%
            witnessed by isomorphisms of pointed sets
            \begin{align*}
                \eSets_{*}(X\wedge Y,Z) &\cong \eSets_{*}(X,\eSets_{*}(Y,Z)),\\
                \eSets_{*}(X\wedge Y,Z) &\cong \eSets_{*}(X,\eSets_{*}(A,Z)),
            \end{align*}
            natural in $(X,x_{0}),(Y,y_{0}),(Z,z_{0})\in\Obj(\eSets_{*})$.
        %\item\label{properties-of-the-internal-hom-of-pointed-sets-}\SloganFont{. }
    \end{enumerate}
\end{proposition}
\begin{Proof}{Proof of \cref{properties-of-the-internal-hom-of-pointed-sets}}%
    \FirstProofBox{\cref{properties-of-the-internal-hom-of-pointed-sets-functoriality}: Functoriality}%
    This follows from \ChapterRef{\ChapterConstructionsWithSets, \cref{constructions-with-sets:properties-of-sets-of-maps-functoriality} of \cref{constructions-with-sets:properties-of-sets-of-maps}}{\cref{properties-of-sets-of-maps-functoriality} of \cref{properties-of-sets-of-maps}} and from the equalities
    \begin{align*}
        g\circ\Delta_{y_{0}}  &= \Delta_{z_{0}},\\
        \Delta_{y_{0}}\circ f &= \Delta_{y_{0}}
    \end{align*}
    for morphisms $f\colon(K,k_{0})\to(X,x_{0})$ and $g\colon(Y,y_{0})\to(Z,z_{0})$, which guarantee pre- and postcomposition by morphisms of pointed sets to also be morphisms of pointed sets.

    \ProofBox{\cref{properties-of-the-internal-hom-of-pointed-sets-adjointness}: Adjointness}%
    This is a repetition of \cref{properties-of-smash-products-of-pointed-sets-adjointness} of \cref{properties-of-smash-products-of-pointed-sets}, and is proved there.

    \ProofBox{\cref{properties-of-the-internal-hom-of-pointed-sets-enriched-adjointness}: Enriched Adjointness}%
    This is a repetition of \cref{properties-of-smash-products-of-pointed-sets-enriched-adjointness} of \cref{properties-of-smash-products-of-pointed-sets}, and is proved there.
\end{Proof}
\subsection{The Monoidal Unit}\label{subsection-the-monoidal-unit-of-the-smash-product-of-pointed-sets}
\begin{definition}{The Monoidal Unit of $\wedge$}{the-monoidal-unit-of-the-smash-product-of-pointed-sets}%
    The \index[set-theory]{smash product of pointed sets!monoidal unit of}\textbf{monoidal unit of the smash product of pointed sets} is the functor
    \[
        \Unit^{\Sets_{*}}
        \colon
        \PunctualCategory
        \to
        \Sets_{*}
    \]
    defined by
    \[
        \Unit_{\Sets_{*}}%
        \defeq%
        S^{0}.
    \]%
\end{definition}
\subsection{The Associator}\label{subsection-the-smash-product-of-pointed-sets-the-associator}
\begin{definition}{The Associator of $\wedge$}{the-associator-of-the-smash-product-of-pointed-sets}%
    The \index[set-theory]{smash product of pointed sets!associator}\textbf{associator of the smash product of pointed sets} is the natural isomorphism
    \[
        \alpha^{\Sets_{*}}
        \colon
        {\wedge}\circ{({\wedge}\times\id_{\Sets_{*}})}
        \Longrightisoarrow
        {\wedge}\circ{(\id_{\Sets_{*}}\times{\wedge})}\circ{\bfalpha^{\Cats}_{\Sets_{*},\Sets_{*},\Sets_{*}}},
    \]
    as in the diagram
    \[
        \begin{tikzcd}[row sep={0*\the\DL,between origins}, column sep={0*\the\DL,between origins}, background color=backgroundColor, ampersand replacement=\&]
            \&[0.30901699437\TwoCmPlusHalf]
            \&[0.5\TwoCmPlusHalf]
            \Sets_{*}\times(\Sets_{*}\times\Sets_{*})
            \&[0.5\TwoCmPlusHalf]
            \&[0.30901699437\TwoCmPlusHalf]
            \\[0.58778525229\TwoCmPlusHalf]
            (\Sets_{*}\times\Sets_{*})\times\Sets_{*}
            \&[0.30901699437\TwoCmPlusHalf]
            \&[0.5\TwoCmPlusHalf]
            \&[0.5\TwoCmPlusHalf]
            \&[0.30901699437\TwoCmPlusHalf]
            \Sets_{*}\times\Sets_{*}
            \\[0.95105651629\TwoCmPlusHalf]
            \&[0.30901699437\TwoCmPlusHalf]
            \Sets_{*}\times\Sets_{*}
            \&[0.5\TwoCmPlusHalf]
            \&[0.5\TwoCmPlusHalf]
            \Sets_{*}\mrp{,}
            \&[0.30901699437\TwoCmPlusHalf]
            % 1-Arrows
            % Left Boundary
            \arrow[from=2-1,to=1-3,"\bfalpha^{\Cats}_{\Sets_{*},\Sets_{*},\Sets_{*}}"{pos=0.3},isoarrowprime]%
            \arrow[from=1-3,to=2-5,"{\sfid\times{\wedge}}"{pos=0.6},""{name=2}]%
            \arrow[from=2-5,to=3-4,"\wedge"{pos=0.425}]%
            % Right Boundary
            \arrow[from=2-1,to=3-2,"{{\wedge}\times\sfid}"'{pos=0.425}]%
            \arrow[from=3-2,to=3-4,"\wedge"']%
            % 2-Arrows
            \arrow[from=3-2,to=2,"\alpha^{\Sets_{*}}"{description,pos=0.5},Rightarrow,shorten <= 0.5*\the\DL,shorten >= 1*\the\DL]%
        \end{tikzcd}
    \]%
    whose component
    \[
        \alpha^{\Sets_{*}}_{X,Y,Z}
        \colon
        (X\wedge Y)\wedge Z
        \isorightarrow
        X\wedge(Y\wedge Z)
    \]%
    at $(X,x_{0}),(Y,y_{0}),(Z,z_{0})\in\Obj(\Sets_{*})$ is given by
    \[
        \alpha^{\Sets_{*}}_{X,Y,Z}((x\wedge y)\wedge z)%
        \defeq%
        x\wedge(y\wedge z)
    \]%
    for each $(x\wedge y)\wedge z\in(X\wedge Y)\wedge Z$.
\end{definition}
\begin{Proof}{Proof of \cref{the-associator-of-the-smash-product-of-pointed-sets}}%
    \ProofBox{Well-Definedness}%
    Let $[((x,y),z)]=[((x',y'),z')]$ be an element in $(X\wedge Y)\wedge Z$. Then either:
    \begin{enumerate}
        \item\label{proof-of-the-associator-of-the-smash-product-of-pointed-sets-1}We have $x=x'$, $y=y'$, and $z=z'$.
        \item\label{proof-of-the-associator-of-the-smash-product-of-pointed-sets-2}Both of the following conditions are satisfied:
            \begin{enumerate}
                \item\label{proof-of-the-associator-of-the-smash-product-of-pointed-sets-2-a}We have $x=x_{0}$ or $y=y_{0}$ or $z=z_{0}$.
                \item\label{proof-of-the-associator-of-the-smash-product-of-pointed-sets-2-b}We have $x'=x_{0}$ or $y'=y_{0}$ or $z'=z_{0}$.
            \end{enumerate}
    \end{enumerate}
    In the first case, $\alpha^{\Sets_{*}}_{X,Y,Z}$ clearly sends both elements to the same element in $X\wedge(Y\wedge Z)$. Meanwhile, in the latter case both elements are equal to the basepoint $(x_{0}\wedge y_{0})\wedge z_{0}$ of $(X\wedge Y)\wedge Z$, which gets sent to the basepoint $x_{0}\wedge(y_{0}\wedge z_{0})$ of $X\wedge(Y\wedge Z)$.

    \ProofBox{Being a Morphism of Pointed Sets}%
    As just mentioned, we have
    \[
        \alpha^{\Sets_{*}}_{X,Y,Z}((x_{0}\wedge y_{0})\wedge z_{0})%
        \defeq%
        x_{0}\wedge(y_{0}\wedge z_{0}),%
    \]%
    and thus $\alpha^{\Sets_{*}}_{X,Y,Z}$ is a morphism of pointed sets.

    \ProofBox{Invertibility}%
    The inverse of $\alpha^{\Sets_{*}}_{X,Y,Z}$ is given by the morphism
    \[
        \alpha^{\Sets_{*},-1}_{X,Y,Z}
        \colon
        X\wedge(Y\wedge Z)
        \isorightarrow
        (X\wedge Y)\wedge Z
    \]%
    defined by
    \[
        \alpha^{\Sets_{*},-1}_{X,Y,Z}(x\wedge(y\wedge z))%
        \defeq%
        (x\wedge y)\wedge z
    \]%
    for each $x\wedge(y\wedge z)\in X\wedge(Y\wedge Z)$.

    \ProofBox{Naturality}%
    We need to show that, given morphisms of pointed sets
    \begin{align*}
        f &\colon (X,x_{0}) \to (X',x'_{0}),\\
        g &\colon (Y,y_{0}) \to (Y',y'_{0}),\\
        h &\colon (Z,z_{0}) \to (Z',z'_{0})
    \end{align*}
    the diagram
    \[
        \begin{tikzcd}[row sep={5.0*\the\DL,between origins}, column sep={11.0*\the\DL,between origins}, background color=backgroundColor, ampersand replacement=\&]
            (X\wedge Y)\wedge Z
            \arrow[r,"{(f\wedge g)\wedge h}"]
            \arrow[d,"\alpha^{\Sets_{*}}_{X,Y,Z}"']
            \&
            (X'\wedge Y')\wedge Z'
            \arrow[d,"\alpha^{\Sets_{*}}_{X',Y',Z'}"]
            \\
            X\wedge(Y\wedge Z)
            \arrow[r,"{f\wedge(g\wedge h)}"']
            \&
            X'\wedge(Y'\wedge Z')
        \end{tikzcd}
    \]%
    commutes. Indeed, this diagram acts on elements as
    \[
        \begin{tikzcd}[row sep={5.0*\the\DL,between origins}, column sep={11.0*\the\DL,between origins}, background color=backgroundColor, ampersand replacement=\&]
            (x\wedge y)\wedge z
            \arrow[r,mapsto]
            \arrow[d,mapsto]
            \&
            (f(x)\wedge g(y))\wedge h(z)
            \arrow[d,mapsto]
            \\
            x\wedge(y\wedge z)
            \arrow[r,mapsto]
            \&
            f(x)\wedge(g(y)\wedge h(z))
        \end{tikzcd}
    \]%
    and hence indeed commutes, showing $\alpha^{\Sets_{*}}$ to be a natural transformation.

    \ProofBox{Being a Natural Isomorphism}%
    Since $\alpha^{\Sets_{*}}$ is natural and $\alpha^{\Sets_{*},-1}$ is a componentwise inverse to $\alpha^{\Sets_{*}}$, it follows from \ChapterRef{\ChapterCategories, \cref{categories:properties-of-natural-isomorphisms-componentwise-inverses-of-natural-transformations-assemble-into-natural-transformations} of \cref{categories:properties-of-natural-isomorphisms}}{\cref{properties-of-natural-isomorphisms-componentwise-inverses-of-natural-transformations-assemble-into-natural-transformations} of \cref{properties-of-natural-isomorphisms}} that $\alpha^{\Sets_{*},-1}$ is also natural. Thus $\alpha^{\Sets_{*}}$ is a natural isomorphism.
\end{Proof}
\subsection{The Left Unitor}\label{subsection-the-smash-product-of-pointed-sets-the-left-unitor}
\begin{definition}{The Left Unitor of $\wedge$}{the-left-unitor-of-the-smash-product-of-pointed-sets}%
    The \index[set-theory]{smash product of pointed sets!left unitor}\textbf{left unitor of the smash product of pointed sets} is the natural isomorphism
    \begin{webcompile}
        \LUnitor^{\Sets_{*}}%
        \colon%
        {\wedge}\circ{(\Unit^{\Sets_{*}}\times\id_{\Sets_{*}})}
        \Longrightisoarrow
        \bfLUnitor^{\TwoCategoryOfCategories}_{\Sets_{*}}
        \begin{tikzcd}[row sep={10.0*\the\DL,between origins}, column sep={10.0*\the\DL,between origins}, background color=backgroundColor, ampersand replacement=\&]
            \PunctualCategory\times\Sets_{*}
            \arrow[r, "\Unit^{\Sets_{*}}\times\sfid"]
            \arrow[rd, dashed,"\bfLUnitor^{\TwoCategoryOfCategories}_{\Sets_{*}}"'{name=1,pos=0.475},bend right=30]
            \&
            \Sets_{*}\times\Sets_{*}
            \arrow[d, "\wedge"]
            \\
            {}
            \&
            \Sets_{*}\mathrlap{,}
            % 2-Arrows
            \arrow[Rightarrow,from=1-2,to=1,shorten >=1.0*\the\DL,shorten <=1.0*\the\DL,"\LUnitor^{\Sets_{*}}"description]
        \end{tikzcd}
    \end{webcompile}%
    whose component
    \[
        \LUnitor^{\Sets_{*}}_{X}
        \colon
        S^{0}\wedge X
        \isorightarrow
        X
    \]%
    at $X\in\Obj(\Sets_{*})$ is given by
    \begin{align*}
        0\wedge x &\mapsto x_{0},\\
        1\wedge x &\mapsto x
    \end{align*}
    for each $x\in X$.
\end{definition}
\begin{Proof}{Proof of \cref{the-left-unitor-of-the-smash-product-of-pointed-sets}}%
    \ProofBox{Well-Definedness}%
    Let $[(x,y)]=[(x',y')]$ be an element in $S^{0}\wedge X$. Then either:
    \begin{enumerate}
        \item\label{proof-of-the-left-unitor-of-the-smash-product-of-pointed-sets-1}We have $x=x'$ and $y=y'$.
        \item\label{proof-of-the-left-unitor-of-the-smash-product-of-pointed-sets-2}Both of the following conditions are satisfied:
            \begin{enumerate}
                \item\label{proof-of-the-left-unitor-of-the-smash-product-of-pointed-sets-2-a}We have $x=0$ or $y=x_{0}$.
                \item\label{proof-of-the-left-unitor-of-the-smash-product-of-pointed-sets-2-b}We have $x'=0$ or $y'=x_{0}$.
            \end{enumerate}
    \end{enumerate}
    In the first case, $\LUnitor^{\Sets_{*}}_{X}$ clearly sends both elements to the same element in $X$. Meanwhile, in the latter case both elements are equal to the basepoint $0\wedge x_{0}$ of $S^{0}\wedge X$, which gets sent to the basepoint $x_{0}$ of $X$.

    \ProofBox{Being a Morphism of Pointed Sets}%
    As just mentioned, we have
    \[
        \LUnitor^{\Sets_{*}}_{X}(0\wedge x_{0})%
        \defeq%
        x_{0},%
    \]%
    and thus $\LUnitor^{\Sets_{*}}_{X}$ is a morphism of pointed sets.

    \ProofBox{Invertibility}%
    The inverse of $\LUnitor^{\Sets_{*}}_{X}$ is the morphism
    \[
        \LUnitor^{\Sets_{*},-1}_{X}%
        \colon%
        X%
        \isorightarrow%
        S^{0}\wedge X%
    \]%
    defined by
    \[
        \LUnitor^{\Sets_{*},-1}_{X}(x)%
        \defeq%
        1\wedge x
    \]%
    for each $x\in X$. Indeed:
    \begin{enumerate}
        \item\label{proof-of-the-left-unitor-of-the-smash-product-of-pointed-sets-3}\SloganFont{Invertibility \rmI. }We have
            \begin{align*}
                [\LUnitor^{\Sets_{*},-1}_{X}\circ\LUnitor^{\Sets_{*}}_{X}](0\wedge x) &= \LUnitor^{\Sets_{*},-1}_{X}(\LUnitor^{\Sets_{*}}_{X}(0\wedge x))\\
                                                                                      &= \LUnitor^{\Sets_{*},-1}_{X}(x_{0})\\
                                                                                      &= 1\wedge x_{0}\\
                                                                                      &= 0\wedge x,
            \end{align*}
            and 
            \begin{align*}
                [\LUnitor^{\Sets_{*},-1}_{X}\circ\LUnitor^{\Sets_{*}}_{X}](1\wedge x) &= \LUnitor^{\Sets_{*},-1}_{X}(\LUnitor^{\Sets_{*}}_{X}(1\wedge x))\\
                                                                                      &= \LUnitor^{\Sets_{*},-1}_{X}(x)\\
                                                                                      &= 1\wedge x
            \end{align*}
            for each $x\in X$, and thus we have
            \[
                \LUnitor^{\Sets_{*},-1}_{X}%
                \circ%
                \LUnitor^{\Sets_{*}}_{X}%
                =%
                \id_{S^{0}\wedge X}.%
            \]%
        \item\label{proof-of-the-left-unitor-of-the-smash-product-of-pointed-sets-4}\SloganFont{Invertibility \rmII. }We have
            \begin{align*}
                [\LUnitor^{\Sets_{*}}_{X}\circ\LUnitor^{\Sets_{*},-1}_{X}](x) &= \LUnitor^{\Sets_{*}}_{X}(\LUnitor^{\Sets_{*},-1}_{X}(x))\\
                                                                              &= \LUnitor^{\Sets_{*},-1}_{X}(1\wedge x)\\
                                                                              &= x
            \end{align*}
            for each $x\in X$, and thus we have
            \[
                \LUnitor^{\Sets_{*}}_{X}%
                \circ%
                \LUnitor^{\Sets_{*},-1}_{X}%
                =%
                \id_{X}.%
            \]%
    \end{enumerate}
    This shows $\LUnitor^{\Sets_{*}}_{X}$ to be invertible.

    \ProofBox{Naturality}%
    We need to show that, given a morphism of pointed sets
    \[
        f%
        \colon%
        (X,x_{0})%
        \to%
        (Y,y_{0}),%
    \]%
    the diagram
    \[
        \begin{tikzcd}[row sep={5.0*\the\DL,between origins}, column sep={7.0*\the\DL,between origins}, background color=backgroundColor, ampersand replacement=\&]
            S^{0}\wedge X
            \arrow[r,"{\id_{S^{0}}\wedge f}"]
            \arrow[d,"\LUnitor^{\Sets_{*}}_{X}"']
            \&
            S^{0}\wedge Y
            \arrow[d,"\LUnitor^{\Sets_{*}}_{Y}"]
            \\
            X
            \arrow[r,"f"']
            \&
            Y
        \end{tikzcd}
    \]%
    commutes. Indeed, this diagram acts on elements as
    \begin{webcompile}
        \begin{tikzcd}[row sep={5.0*\the\DL,between origins}, column sep={6.0*\the\DL,between origins}, background color=backgroundColor, ampersand replacement=\&]
            0\wedge x
            \arrow[d,mapsto]
            \&
            \\
            x_{0}
            \arrow[r,mapsto]
            \&
            f(x_{0})
        \end{tikzcd}
        \quad
        \begin{tikzcd}[row sep={5.0*\the\DL,between origins}, column sep={6.0*\the\DL,between origins}, background color=backgroundColor, ampersand replacement=\&]
            0\wedge x
            \arrow[r,mapsto]
            \&
            0\wedge f(x)
            \arrow[d,mapsto]
            \\
            \&
            y_{0}
        \end{tikzcd}
    \end{webcompile}
    and
    \[
        \begin{tikzcd}[row sep={5.0*\the\DL,between origins}, column sep={6.0*\the\DL,between origins}, background color=backgroundColor, ampersand replacement=\&]
            1\wedge x
            \arrow[r,mapsto]
            \arrow[d,mapsto]
            \&
            1\wedge f(x)
            \arrow[d,mapsto]
            \\
            x
            \arrow[r,mapsto]
            \&
            f(x)
        \end{tikzcd}
    \]%
    and hence indeed commutes, showing $\LUnitor^{\Sets_{*}}$ to be a natural transformation.

    \ProofBox{Being a Natural Isomorphism}%
    Since $\LUnitor^{\Sets_{*}}$ is natural and $\LUnitor^{\Sets_{*},-1}$ is a componentwise inverse to $\LUnitor^{\Sets_{*}}$, it follows from \ChapterRef{\ChapterCategories, \cref{categories:properties-of-natural-isomorphisms-componentwise-inverses-of-natural-transformations-assemble-into-natural-transformations} of \cref{categories:properties-of-natural-isomorphisms}}{\cref{properties-of-natural-isomorphisms-componentwise-inverses-of-natural-transformations-assemble-into-natural-transformations} of \cref{properties-of-natural-isomorphisms}} that $\LUnitor^{\Sets_{*},-1}$ is also natural. Thus $\LUnitor^{\Sets_{*}}$ is a natural isomorphism.
\end{Proof}
\subsection{The Right Unitor}\label{subsection-the-smash-product-of-pointed-sets-the-right-unitor}
\begin{definition}{The Right Unitor of $\wedge$}{the-right-unitor-of-the-smash-product-of-pointed-sets}%
    The \index[set-theory]{smash product of pointed sets!right unitor}\textbf{right unitor of the smash product of pointed sets} is the natural isomorphism
    \begin{webcompile}
        \RUnitor^{\Sets_{*}}%
        \colon%
        {\wedge}\circ{({\sfid}\times{\Unit^{\Sets_{*}}})}%
        \Longrightisoarrow%
        \bfRUnitor^{\TwoCategoryOfCategories}_{\Sets_{*}},%
        \begin{tikzcd}[row sep={10.0*\the\DL,between origins}, column sep={10.0*\the\DL,between origins}, background color=backgroundColor, ampersand replacement=\&]
            \Sets_{*}\times\PunctualCategory
            \arrow[r, "\sfid\times\Unit^{\Sets_{*}}"]
            \arrow[rd, dashed,"\bfRUnitor^{\TwoCategoryOfCategories}_{\Sets_{*}}"'{name=1,pos=0.475},bend right=30]
            \&
            \Sets_{*}\times\Sets_{*}
            \arrow[d, "\wedge"]
            \\
            {}
            \&
            \Sets_{*}\mathrlap{,}
            % 2-Arrows
            \arrow[Rightarrow,from=1-2,to=1,shorten >=1.0*\the\DL,shorten <=1.0*\the\DL,"\RUnitor^{\Sets_{*}}"description]
        \end{tikzcd}
    \end{webcompile}%
    whose component
    \[
        \RUnitor^{\Sets_{*}}_{X}
        \colon
        X\wedge S^{0}
        \isorightarrow
        X
    \]%
    at $X\in\Obj(\Sets_{*})$ is given by
    \begin{align*}
        x\wedge 0 &\mapsto x_{0},\\
        x\wedge 1 &\mapsto x
    \end{align*}
    for each $x\in X$.
\end{definition}
\begin{Proof}{Proof of \cref{the-right-unitor-of-the-smash-product-of-pointed-sets}}%
    \ProofBox{Well-Definedness}%
    Let $[(x,y)]=[(x',y')]$ be an element in $X\wedge S^{0}$. Then either:
    \begin{enumerate}
        \item\label{proof-of-the-right-unitor-of-the-smash-product-of-pointed-sets-1}We have $x=x'$ and $y=y'$.
        \item\label{proof-of-the-right-unitor-of-the-smash-product-of-pointed-sets-2}Both of the following conditions are satisfied:
            \begin{enumerate}
                \item\label{proof-of-the-right-unitor-of-the-smash-product-of-pointed-sets-2-a}We have $x=x_{0}$ or $y=0$.
                \item\label{proof-of-the-right-unitor-of-the-smash-product-of-pointed-sets-2-b}We have $x'=x_{0}$ or $y'=0$.
            \end{enumerate}
    \end{enumerate}
    In the first case, $\RUnitor^{\Sets_{*}}_{X}$ clearly sends both elements to the same element in $X$. Meanwhile, in the latter case both elements are equal to the basepoint $x_{0}\wedge 0$ of $X\wedge S^{0}$, which gets sent to the basepoint $x_{0}$ of $X$.

    \ProofBox{Being a Morphism of Pointed Sets}%
    As just mentioned, we have
    \[
        \RUnitor^{\Sets_{*}}_{X}(x_{0}\wedge 0)%
        \defeq%
        x_{0},%
    \]%
    and thus $\RUnitor^{\Sets_{*}}_{X}$ is a morphism of pointed sets.

    \ProofBox{Invertibility}%
    The inverse of $\RUnitor^{\Sets_{*}}_{X}$ is the morphism
    \[
        \RUnitor^{\Sets_{*},-1}_{X}%
        \colon%
        X%
        \isorightarrow%
        X\wedge S^{0}%
    \]%
    defined by
    \[
        \RUnitor^{\Sets_{*},-1}_{X}(x)%
        \defeq%
        x\wedge 1
    \]%
    for each $x\in X$. Indeed:
    \begin{enumerate}
        \item\label{proof-of-the-right-unitor-of-the-smash-product-of-pointed-sets-3}\SloganFont{Invertibility \rmI. }We have
            \begin{align*}
                [\RUnitor^{\Sets_{*},-1}_{X}\circ\RUnitor^{\Sets_{*}}_{X}](x\wedge 0) &= \RUnitor^{\Sets_{*},-1}_{X}(\RUnitor^{\Sets_{*}}_{X}(x\wedge 0))\\
                                                                                      &= \RUnitor^{\Sets_{*},-1}_{X}(x_{0})\\
                                                                                      &= x_{0}\wedge 1\\
                                                                                      &= x\wedge 0,
            \end{align*}
            and 
            \begin{align*}
                [\RUnitor^{\Sets_{*},-1}_{X}\circ\RUnitor^{\Sets_{*}}_{X}](x\wedge 1) &= \RUnitor^{\Sets_{*},-1}_{X}(\RUnitor^{\Sets_{*}}_{X}(x\wedge 1))\\
                                                                                      &= \RUnitor^{\Sets_{*},-1}_{X}(x)\\
                                                                                      &= x\wedge 1
            \end{align*}
            for each $x\in X$, and thus we have
            \[
                \RUnitor^{\Sets_{*},-1}_{X}%
                \circ%
                \RUnitor^{\Sets_{*}}_{X}%
                =%
                \id_{X\wedge S^{0}}.%
            \]%
        \item\label{proof-of-the-right-unitor-of-the-smash-product-of-pointed-sets-4}\SloganFont{Invertibility \rmII. }We have
            \begin{align*}
                [\RUnitor^{\Sets_{*}}_{X}\circ\RUnitor^{\Sets_{*},-1}_{X}](x) &= \RUnitor^{\Sets_{*}}_{X}(\RUnitor^{\Sets_{*},-1}_{X}(x))\\
                                                                              &= \RUnitor^{\Sets_{*},-1}_{X}(x\wedge 1)\\
                                                                              &= x
            \end{align*}
            for each $x\in X$, and thus we have
            \[
                \RUnitor^{\Sets_{*}}_{X}%
                \circ%
                \RUnitor^{\Sets_{*},-1}_{X}%
                =%
                \id_{X}.%
            \]%
    \end{enumerate}
    This shows $\RUnitor^{\Sets_{*}}_{X}$ to be invertible.

    \ProofBox{Naturality}%
    We need to show that, given a morphism of pointed sets
    \[
        f%
        \colon%
        (X,x_{0})%
        \to%
        (Y,y_{0}),%
    \]%
    the diagram
    \[
        \begin{tikzcd}[row sep={5.0*\the\DL,between origins}, column sep={7.0*\the\DL,between origins}, background color=backgroundColor, ampersand replacement=\&]
            X\wedge S^{0}
            \arrow[r,"{f\wedge\id_{S^{0}}}"]
            \arrow[d,"\RUnitor^{\Sets_{*}}_{X}"']
            \&
            Y\wedge S^{0}
            \arrow[d,"\RUnitor^{\Sets_{*}}_{Y}"]
            \\
            X
            \arrow[r,"f"']
            \&
            Y
        \end{tikzcd}
    \]%
    commutes. Indeed, this diagram acts on elements as
    \begin{webcompile}
        \begin{tikzcd}[row sep={5.0*\the\DL,between origins}, column sep={6.0*\the\DL,between origins}, background color=backgroundColor, ampersand replacement=\&]
            x\wedge 0
            \arrow[d,mapsto]
            \&
            \\
            x_{0}
            \arrow[r,mapsto]
            \&
            f(x_{0})
        \end{tikzcd}
        \quad
        \begin{tikzcd}[row sep={5.0*\the\DL,between origins}, column sep={6.0*\the\DL,between origins}, background color=backgroundColor, ampersand replacement=\&]
            x\wedge 0
            \arrow[r,mapsto]
            \&
            f(x)\wedge 0
            \arrow[d,mapsto]
            \\
            \&
            y_{0}
        \end{tikzcd}
    \end{webcompile}
    and
    \[
        \begin{tikzcd}[row sep={5.0*\the\DL,between origins}, column sep={6.0*\the\DL,between origins}, background color=backgroundColor, ampersand replacement=\&]
            x\wedge 1
            \arrow[r,mapsto]
            \arrow[d,mapsto]
            \&
            f(x)\wedge 1
            \arrow[d,mapsto]
            \\
            x
            \arrow[r,mapsto]
            \&
            f(x)
        \end{tikzcd}
    \]%
    and hence indeed commutes, showing $\RUnitor^{\Sets_{*}}$ to be a natural transformation.

    \ProofBox{Being a Natural Isomorphism}%
    Since $\RUnitor^{\Sets_{*}}$ is natural and $\RUnitor^{\Sets_{*},-1}$ is a componentwise inverse to $\RUnitor^{\Sets_{*}}$, it follows from \ChapterRef{\ChapterCategories, \cref{categories:properties-of-natural-isomorphisms-componentwise-inverses-of-natural-transformations-assemble-into-natural-transformations} of \cref{categories:properties-of-natural-isomorphisms}}{\cref{properties-of-natural-isomorphisms-componentwise-inverses-of-natural-transformations-assemble-into-natural-transformations} of \cref{properties-of-natural-isomorphisms}} that $\RUnitor^{\Sets_{*},-1}$ is also natural. Thus $\RUnitor^{\Sets_{*}}$ is a natural isomorphism.
\end{Proof}
\subsection{The Symmetry}\label{subsection-the-smash-product-of-pointed-sets-the-symmetry}
\begin{definition}{The Symmetry of $\wedge$}{the-symmetry-of-the-smash-product-of-pointed-sets}%
    The \index[set-theory]{smash product of pointed sets!symmetry}\textbf{symmetry of the smash product of pointed sets} is the natural isomorphism
    \begin{webcompile}
        \sigma^{\Sets_{*}}
        \colon
        {\wedge}
        \Longrightisoarrow
        {\wedge}\circ{\bfsigma^{\TwoCategoryOfCategories}_{\Sets_{*},\Sets_{*}}},
        \qquad
        \begin{tikzcd}[row sep={5.0*\the\DL,between origins}, column sep={4.0*\the\DL,between origins}, background color=backgroundColor, ampersand replacement=\&]
            \Sets_{*}\times\Sets_{*}
            \arrow[rr,"\wedge"{name=1},pos=0.5]
            \arrow[rd,"\bfsigma^{\TwoCategoryOfCategories}_{\Sets_{*},\Sets_{*}}"'{pos=0.25},bend right=15]
            \&
            {}
            \&
            \Sets_{*}\mrp{,}
            \\
            \&
            \Sets_{*}\times\Sets_{*}
            \arrow[ru,"\wedge"'{pos=0.525},bend right=15]
            \&
            % 2-arrows
            \arrow[from=1-2,to=2-2,"\sigma^{\Sets_{*}}"{description,pos=0.425},shorten <= 0.0*\the\DL,shorten >=0.25*\the\DL,Rightarrow]%
        \end{tikzcd}
    \end{webcompile}%
    whose component
    \[
        \sigma^{\Sets_{*}}_{X,Y}
        \colon
        X\wedge Y
        \isorightarrow
        Y\wedge X
    \]%
    at $X,Y\in\Obj(\Sets_{*})$ is defined by
    \[
        \sigma^{\Sets_{*}}_{X,Y}(x\wedge y)%
        \defeq%
        y\wedge x%
    \]%
    for each $x\wedge y\in X\wedge Y$.
\end{definition}
\begin{Proof}{Proof of \cref{the-symmetry-of-the-smash-product-of-pointed-sets}}%
    \ProofBox{Well-Definedness}%
    Let $[(x,y)]=[(x',y')]$ be an element in $X\wedge Y$. Then either:
    \begin{enumerate}
        \item\label{proof-of-the-symmetry-of-the-smash-product-of-pointed-sets-1}We have $x=x'$ and $y=y'$.
        \item\label{proof-of-the-symmetry-of-the-smash-product-of-pointed-sets-2}Both of the following conditions are satisfied:
            \begin{enumerate}
                \item\label{proof-of-the-symmetry-of-the-smash-product-of-pointed-sets-2-a}We have $x=x_{0}$ or $y=y_{0}$.
                \item\label{proof-of-the-symmetry-of-the-smash-product-of-pointed-sets-2-b}We have $x'=x_{0}$ or $y'=y_{0}$.
            \end{enumerate}
    \end{enumerate}
    In the first case, $\sigma^{\Sets_{*}}_{X}$ clearly sends both elements to the same element in $X$. Meanwhile, in the latter case both elements are equal to the basepoint $x_{0}\wedge y_{0}$ of $X\wedge Y$, which gets sent to the basepoint $y_{0}\wedge x_{0}$ of $Y\wedge X$.

    \ProofBox{Being a Morphism of Pointed Sets}%
    As just mentioned, we have
    \[
        \sigma^{\Sets_{*}}_{X}(x_{0}\wedge y_{0})%
        \defeq%
        y_{0}\wedge x_{0},%
    \]%
    and thus $\sigma^{\Sets_{*}}_{X}$ is a morphism of pointed sets.

    \ProofBox{Invertibility}%
    The inverse of $\sigma^{\Sets_{*}}_{X,Y}$ is given by the morphism
    \[
        \sigma^{\Sets_{*},-1}_{X,Y}
        \colon
        Y\wedge X
        \isorightarrow
        X\wedge Y
    \]%
    defined by
    \[
        \sigma^{\Sets_{*},-1}_{X,Y}(y\wedge x)%
        \defeq%
        x\wedge y
    \]%
    for each $y\wedge x\in Y\wedge X$.

    \ProofBox{Naturality}%
    We need to show that, given morphisms of pointed sets
    \begin{align*}
        f &\colon (X,x_{0}) \to (A,a_{0}),\\%
        g &\colon (Y,y_{0}) \to (B,b_{0})%
    \end{align*}
    the diagram
    \[
        \begin{tikzcd}[row sep={5.0*\the\DL,between origins}, column sep={6.0*\the\DL,between origins}, background color=backgroundColor, ampersand replacement=\&]
            X\wedge Y
            \arrow[r,"f\wedge g"]
            \arrow[d,"\sigma^{\Sets_{*}}_{X,Y}"']
            \&
            A\wedge B
            \arrow[d,"\sigma^{\Sets_{*}}_{A,B}"]
            \\
            Y\wedge X
            \arrow[r,"g\wedge f"']
            \&
            B\wedge A
        \end{tikzcd}
    \]%
    commutes. Indeed, this diagram acts on elements as
    \[
        \begin{tikzcd}[row sep={5.0*\the\DL,between origins}, column sep={7.0*\the\DL,between origins}, background color=backgroundColor, ampersand replacement=\&]
            x\wedge y
            \arrow[r,mapsto]
            \arrow[d,mapsto]
            \&
            f(x)\wedge g(y)
            \arrow[d,mapsto]
            \\
            y\wedge x
            \arrow[r,mapsto]
            \&
            g(y)\wedge f(x)
        \end{tikzcd}
    \]%
    and hence indeed commutes, showing $\sigma^{\Sets_{*}}$ to be a natural transformation.

    \ProofBox{Being a Natural Isomorphism}%
    Since $\sigma^{\Sets_{*}}$ is natural and $\sigma^{\Sets_{*},-1}$ is a componentwise inverse to $\sigma^{\Sets_{*}}$, it follows from \ChapterRef{\ChapterCategories, \cref{categories:properties-of-natural-isomorphisms-componentwise-inverses-of-natural-transformations-assemble-into-natural-transformations} of \cref{categories:properties-of-natural-isomorphisms}}{\cref{properties-of-natural-isomorphisms-componentwise-inverses-of-natural-transformations-assemble-into-natural-transformations} of \cref{properties-of-natural-isomorphisms}} that $\sigma^{\Sets_{*},-1}$ is also natural. Thus $\sigma^{\Sets_{*}}$ is a natural isomorphism.
\end{Proof}
\subsection{The Diagonal}\label{subsection-the-smash-product-of-pointed-sets-the-diagonal}
\begin{definition}{The Diagonal of $\wedge$}{the-diagonal-of-the-smash-product-of-pointed-sets}%
    The \index[set-theory]{smash product of pointed sets!diagonal}\textbf{diagonal of the smash product of pointed sets} is the natural transformation
    \begin{webcompile}
        \Delta^{\wedge}%
        \colon%
        \id_{\Sets_{*}}%
        \Longrightarrow%
        {\wedge}\circ{\Delta^{\TwoCategoryOfCategories}_{\Sets_{*}}},%
        \qquad%
        \begin{tikzcd}[row sep={5.0*\the\DL,between origins}, column sep={4.0*\the\DL,between origins}, background color=backgroundColor, ampersand replacement=\&]
            \Sets_{*}
            \arrow[rr,"\id_{\Sets_{*}}"{name=1},bend left=10]
            \arrow[rd,"\Delta^{\TwoCategoryOfCategories}_{\Sets_{*}}"'{pos=0.3},bend right=10]
            \&
            \&
            \Sets_{*}
            \\
            \&
            \Sets_{*}\wedge\Sets_{*}\mrp{,}
            \arrow[ru,"\wedge"'{pos=0.55},bend right=10]
            \&
            % 2-Arrows
            \arrow[from=1,to=2-2,"\Delta^{\wedge}"description,shorten <= 0.5*\the\DL,shorten >= 0.25*\the\DL,Rightarrow]%
        \end{tikzcd}
    \end{webcompile}%
    whose component
    \[
        \Delta^{\wedge}_{X}%
        \colon%
        (X,x_{0})%
        \to%
        (X\wedge X,x_{0}\wedge x_{0})%
    \]%
    at $(X,x_{0})\in\Obj(\Sets_{*})$ is given by the composition
    \begin{webcompile}
        \begin{tikzcd}[row sep={2.0*\the\DL,between origins}, column sep={4.0*\the\DL,between origins}, background color=backgroundColor, ampersand replacement=\&,cramped]
            (X,x_{0})
            \arrow[r,"\Delta^{\wedge}_{X}"]
            \&
            \mrp{(X\times X,(x_{0},x_{0}))}
            \\
            \phantom{(X,x_{0})}
            \arrow[r,two heads]
            \&
            \mrp{((X\times X)/\unsim,[(x_{0},x_{0})])}
            \\
            \phantom{(X,x_{0})}
            \arrow[r,"\mathrm{def}",equals]
            \&
            \mrp{(X\wedge X,x_{0}\wedge x_{0})}
        \end{tikzcd}
        \phantom{((X\times X)/\unsim,[(x_{0},x_{0})])}
    \end{webcompile}%
    in $\Sets_{*}$, and thus by
    \[
        \Delta^{\wedge}_{X}(x)%
        \defeq%
        x\wedge x%
    \]%
    for each $x\in X$.
\end{definition}
\begin{Proof}{Proof of \cref{the-diagonal-of-the-smash-product-of-pointed-sets}}%
    \ProofBox{Being a Morphism of Pointed Sets}%
    We have
    \[
        \Delta^{\wedge}_{X}(x_{0})%
        \defeq%
        x_{0}\wedge x_{0},%
    \]%
    and thus $\Delta^{\wedge}_{X}$ is a morphism of pointed sets.

    \ProofBox{Naturality}%
    We need to show that, given a morphism of pointed sets
    \[
        f%
        \colon%
        (X,x_{0})%
        \to%
        (Y,y_{0}),%
    \]%
    the diagram
    \[
        \begin{tikzcd}[row sep={4.5*\the\DL,between origins}, column sep={6.5*\the\DL,between origins}, background color=backgroundColor, ampersand replacement=\&]
            X
            \arrow[r,"f"]
            \arrow[d,"\Delta^{\wedge}_{X}"']
            \&
            Y
            \arrow[d,"\Delta^{\wedge}_{Y}"]
            \\
            X\wedge X
            \arrow[r,"f\wedge f"']
            \&
            Y\wedge Y
        \end{tikzcd}
    \]%
    commutes. Indeed, this diagram acts on elements as
    \[
        \begin{tikzcd}[row sep={5.0*\the\DL,between origins}, column sep={7.0*\the\DL,between origins}, background color=backgroundColor, ampersand replacement=\&]
            x
            \arrow[r,mapsto]
            \arrow[d,mapsto]
            \&
            f(x)
            \arrow[d,mapsto]
            \\
            x\wedge x
            \arrow[r,mapsto]
            \&
            f(x)\wedge f(x)
        \end{tikzcd}
    \]%
    and hence indeed commutes, showing $\Delta^{\wedge}$ to be natural.
\end{Proof}
\begin{proposition}{Properties of the Diagonal of $\wedge$}{properties-of-the-diagonal-of-the-smash-product-of-pointed-sets}%
    Let $(X,x_{0})\in\Obj(\Sets_{*})$.
    \begin{enumerate}
        \item\label{properties-of-the-diagonal-of-the-smash-product-of-pointed-sets-monoidality}\SloganFont{Monoidality. }The diagonal
            \[
                \Delta^{\wedge}%
                \colon%
                \id_{\Sets_{*}}%
                \Longrightarrow%
                {\wedge}\circ{\Delta^{\TwoCategoryOfCategories}_{\Sets_{*}}},%
            \]%
            of the smash product of pointed sets is a monoidal natural transformation:
            \begin{enumerate}
                \item\label{properties-of-the-diagonal-of-the-smash-product-of-pointed-sets-monoidality-compatibility-with-strong-monoidality-constraints}\SloganFont{Compatibility With Strong Monoidality Constraints. }For each $(X,x_{0}),(Y,y_{0})\in\Obj(\Sets_{*})$, the diagram
                    \[
                        \begin{tikzcd}[row sep={5.0*\the\DL,between origins}, column sep={10.0*\the\DL,between origins}, background color=backgroundColor, ampersand replacement=\&]
                            X\wedge Y
                            \arrow[r,"\Delta^{\wedge}_{X}\wedge\Delta^{\wedge}_{Y}"]
                            \arrow[rd,"\Delta^{\wedge}_{X\wedge Y}"']
                            \&
                            (X\wedge X)\wedge(Y\wedge Y)
                            \arrow[d,isoarrow]
                            \\
                            \&
                            (X\wedge Y)\wedge(X\wedge Y)
                        \end{tikzcd}
                    \]%
                    commutes.
                \item\label{properties-of-the-diagonal-of-the-smash-product-of-pointed-sets-monoidality-compatibility-with-strong-unitality-constraints}\SloganFont{Compatibility With Strong Unitality Constraints. }The diagrams
                    \begin{webcompile}
                        \begin{tikzcd}[row sep={5.0*\the\DL,between origins}, column sep={5.0*\the\DL,between origins}, background color=backgroundColor, ampersand replacement=\&]
                            S^{0}
                            \arrow[r,"\Delta^{\wedge}_{S^{0}}"]
                            \arrow[rd,Equals]
                            \&
                            S^{0}\wedge S^{0}
                            \arrow[d,"\LUnitor^{\Sets_{*}}_{S^{0}}"]
                            \\
                            \&
                            S^{0}
                        \end{tikzcd}
                        \quad
                        \begin{tikzcd}[row sep={5.0*\the\DL,between origins}, column sep={5.0*\the\DL,between origins}, background color=backgroundColor, ampersand replacement=\&]
                            S^{0}
                            \arrow[r,"\Delta^{\wedge}_{S^{0}}"]
                            \arrow[rd,Equals]
                            \&
                            S^{0}\wedge S^{0}
                            \arrow[d,"\RUnitor^{\Sets_{*}}_{S^{0}}"]
                            \\
                            \&
                            S^{0}
                        \end{tikzcd}
                    \end{webcompile}
                    commute, i.e.\ we have
                    \begin{align*}
                        \Delta^{\wedge}_{S^{0}} &= \LUnitor^{\Sets_{*},-1}_{S^{0}}\\
                                                &= \RUnitor^{\Sets_{*},-1}_{S^{0}},
                    \end{align*}
                    where we recall that the equalities
                    \begin{align*}
                        \LUnitor^{\Sets_{*}}_{S^{0}}    &= \RUnitor^{\Sets_{*}}_{S^{0}},\\%
                        \LUnitor^{\Sets_{*},-1}_{S^{0}} &= \RUnitor^{\Sets_{*},-1}_{S^{0}}%
                    \end{align*}
                    are always true in any monoidal category by \ChapterRef{\ChapterMonoidalCategories, \cref{monoidal-categories:properties-of-monoidal-categories-coherence-for-left-and-right-unitors-of-the-monoidal-unit} of \cref{monoidal-categories:properties-of-monoidal-categories}}{\cref{properties-of-monoidal-categories-coherence-for-left-and-right-unitors-of-the-monoidal-unit} of \cref{properties-of-monoidal-categories}}.
            \end{enumerate}
        \item\label{properties-of-the-diagonal-of-the-smash-product-of-pointed-sets-the-diagonal-of-the-unit}\SloganFont{The Diagonal of the Unit. }The component
            \[
                \Delta^{\wedge}_{S^{0}}
                \colon
                S^{0}
                \isorightarrow
                S^{0}\wedge S^{0}
            \]%
            of $\Delta^{\wedge}$ at $S^{0}$ is an isomorphism.
        %\item\label{properties-of-the-diagonal-of-the-smash-product-of-pointed-sets-}\SloganFont{. }
    \end{enumerate}
\end{proposition}
\begin{Proof}{Proof of \cref{properties-of-the-diagonal-of-the-smash-product-of-pointed-sets}}%
    \ProofBox{\cref{properties-of-the-diagonal-of-the-smash-product-of-pointed-sets-monoidality}: Monoidality}%
    We claim that $\Delta^{\wedge}$ is indeed monoidal:
    \begin{enumerate}
        \item\label{proof-of-properties-of-the-diagonal-of-the-smash-product-of-pointed-sets-monoidality-1}\SloganFont{\cref{properties-of-the-diagonal-of-the-smash-product-of-pointed-sets-monoidality-compatibility-with-strong-monoidality-constraints}: Compatibility With Strong Monoidality Constraints: }We need to show that the diagram
            \[
                \begin{tikzcd}[row sep={5.0*\the\DL,between origins}, column sep={10.0*\the\DL,between origins}, background color=backgroundColor, ampersand replacement=\&]
                    X\wedge Y
                    \arrow[r,"\Delta^{\wedge}_{X}\wedge\Delta^{\wedge}_{Y}"]
                    \arrow[rd,"\Delta^{\wedge}_{X\wedge Y}"']
                    \&
                    (X\wedge X)\wedge(Y\wedge Y)
                    \arrow[d,isoarrow]
                    \\
                    \&
                    (X\wedge Y)\wedge(X\wedge Y)
                \end{tikzcd}
            \]%
            commutes. Indeed, this diagram acts on elements as
            \[
                \begin{tikzcd}[row sep={5.0*\the\DL,between origins}, column sep={9.0*\the\DL,between origins}, background color=backgroundColor, ampersand replacement=\&]
                    x\wedge y
                    \arrow[r,mapsto]
                    \arrow[rd,mapsto]
                    \&
                    (x\wedge x)\wedge(y\wedge y)
                    \arrow[d,mapsto]
                    \\
                    \&
                    (x\wedge y)\wedge(x\wedge y)
                \end{tikzcd}
            \]%
            and hence indeed commutes.
        \item\label{proof-of-properties-of-the-diagonal-of-the-smash-product-of-pointed-sets-monoidality-2}\SloganFont{\cref{properties-of-the-diagonal-of-the-smash-product-of-pointed-sets-monoidality-compatibility-with-strong-unitality-constraints}: Compatibility With Strong Unitality Constraints: }As shown in the proof of \cref{the-left-unitor-of-the-smash-product-of-pointed-sets}, the inverse of the left unitor of $\Sets_{*}$ with respect to to the smash product of pointed sets at $(X,x_{0})\in\Obj(\Sets_{*})$ is given by
            \[
                \LUnitor^{\Sets_{*},-1}_{X}(x)%
                \defeq%
                1\wedge x
            \]%
            for each $x\in X$, so when $X=S^{0}$, we have
            \begin{align*}
                \LUnitor^{\Sets_{*},-1}_{S^{0}}(0) &\defeq 1\wedge0,\\
                \LUnitor^{\Sets_{*},-1}_{S^{0}}(1) &\defeq 1\wedge1.
            \end{align*}
            But since $1\wedge0=0\wedge0$ and
            \begin{align*}
                \Delta^{\wedge}_{S^{0}}(0) &\defeq 0\wedge0,\\
                \Delta^{\wedge}_{S^{0}}(1) &\defeq 1\wedge1,
            \end{align*}
            it follows that we indeed have $\Delta^{\wedge}_{S^{0}}=\LUnitor^{\Sets_{*},-1}_{S^{0}}$.
    \end{enumerate}
    This finishes the proof.

    \ProofBox{\cref{properties-of-the-diagonal-of-the-smash-product-of-pointed-sets-the-diagonal-of-the-unit}: The Diagonal of the Unit}%
    This follows from \cref{properties-of-the-diagonal-of-the-smash-product-of-pointed-sets-monoidality} and the invertibility of the left/right unitor of $\Sets_{*}$ with respect to $\wedge$, proved in the proof of \cref{the-left-unitor-of-the-smash-product-of-pointed-sets} for the left unitor or the proof of \cref{the-right-unitor-of-the-smash-product-of-pointed-sets} for the right unitor.
\end{Proof}
\subsection{The Monoidal Structure on Pointed Sets Associated to $\wedge$}\label{subsection-the-monoidal-structure-on-pointed-sets-associated-to-the-smash-product-of-pointed-sets}
\begin{proposition}{The Monoidal Structure on Pointed Sets Associated to $\wedge$}{the-monoidal-structure-on-pointed-sets-associated-to-the-smash-product-of-pointed-sets}%
    The category $\Sets_{*}$ admits a closed monoidal category with diagonals structure consisting of:%
    \begin{itemize}
        \item\SloganFont{The Underlying Category. }The category $\Sets_{*}$ of pointed sets.
        \item\SloganFont{The Monoidal Product. }The smash product functor
            \[
                \wedge%
                \colon%
                \Sets_{*}\times\Sets_{*}%
                \to%
                \Sets_{*}
            \]%
            of \cref{properties-of-smash-products-of-pointed-sets-functoriality} of \cref{properties-of-smash-products-of-pointed-sets}.
        \item\SloganFont{The Internal Hom. }The internal Hom functor
            \[
                \eSets_{*}%
                \colon%
                \Sets^{\op}_{*}\times\Sets_{*}%
                \to%
                \Sets_{*}%
            \]%
            of \cref{properties-of-the-internal-hom-of-pointed-sets-functoriality} of \cref{properties-of-the-internal-hom-of-pointed-sets}.
        \item\SloganFont{The Monoidal Unit. }The functor
            \[
                \Unit^{\Sets_{*}}
                \colon
                \PunctualCategory
                \to
                \Sets_{*}
            \]
            of \cref{the-monoidal-unit-of-the-smash-product-of-pointed-sets}.
        \item\SloganFont{The Associators. }The natural isomorphism
            \[
                \alpha^{\Sets_{*}}
                \colon
                {\wedge}\circ{({\wedge}\times\id_{\Sets_{*}})}
                \Longrightisoarrow
                {\wedge}\circ{(\id_{\Sets_{*}}\times{\wedge})}\circ{\bfalpha^{\Cats}_{\Sets_{*},\Sets_{*},\Sets_{*}}}
            \]
            of \cref{the-associator-of-the-smash-product-of-pointed-sets}.
        \item\SloganFont{The Left Unitors. }The natural isomorphism
            \[
                \LUnitor^{\Sets_{*}}%
                \colon%
                {\wedge}\circ{(\Unit^{\Sets_{*}}\times\id_{\Sets_{*}})}
                \Longrightisoarrow
                \bfLUnitor^{\TwoCategoryOfCategories}_{\Sets_{*}}
            \]
            of \cref{the-left-unitor-of-the-smash-product-of-pointed-sets}.
        \item\SloganFont{The Right Unitors. }The natural isomorphism
            \[
                \RUnitor^{\Sets_{*}}%
                \colon%
                {\wedge}\circ{({\sfid}\times{\Unit^{\Sets_{*}}})}%
                \Longrightisoarrow%
                \bfRUnitor^{\TwoCategoryOfCategories}_{\Sets_{*}}%
            \]
            of \cref{the-right-unitor-of-the-smash-product-of-pointed-sets}.
        \item\SloganFont{The Symmetry. }The natural isomorphism
            \[
                \sigma^{\Sets_{*}}
                \colon
                {\wedge}
                \Longrightisoarrow
                {\wedge}\circ{\bfsigma^{\TwoCategoryOfCategories}_{\Sets_{*},\Sets_{*}}}
            \]
            of \cref{the-symmetry-of-the-smash-product-of-pointed-sets}.
        \item\SloganFont{The Diagonals. }The monoidal natural transformation
            \[
                \Delta^{\wedge}%
                \colon%
                \id_{\Sets_{*}}%
                \Longrightarrow%
                \wedge\circ\Delta^{\TwoCategoryOfCategories}_{\Sets_{*}}%
            \]
            of \cref{the-diagonal-of-the-smash-product-of-pointed-sets}.
    \end{itemize}
\end{proposition}
\begin{Proof}{Proof of \cref{the-monoidal-structure-on-pointed-sets-associated-to-the-smash-product-of-pointed-sets}}%
    \FirstProofBox{The Pentagon Identity}%
    Let $(W,w_{0})$, $(X,x_{0})$, $(Y,y_{0})$ and $(Z,z_{0})$ be pointed sets. We have to show that the diagram
    \[
        \begin{tikzcd}[row sep={0*\the\DL,between origins}, column sep={0*\the\DL,between origins}, background color=backgroundColor, ampersand replacement=\&]
            \&[0.30901699437\FourCmPlusHalf]
            \&[0.5\FourCmPlusHalf]
            (W\wedge(X\wedge Y))\wedge Z
            \&[0.5\FourCmPlusHalf]
            \&[0.30901699437\FourCmPlusHalf]
            \\[0.58778525229\FourCmPlusHalf]
            ((W\wedge X)\wedge Y)\wedge Z
            \&[0.30901699437\FourCmPlusHalf]
            \&[0.5\FourCmPlusHalf]
            \&[0.5\FourCmPlusHalf]
            \&[0.30901699437\FourCmPlusHalf]
            W\wedge((X\wedge Y)\wedge Z)
            \\[0.95105651629\FourCmPlusHalf]
            \&[0.30901699437\FourCmPlusHalf]
            (W\wedge X)\wedge(Y\wedge Z)
            \&[0.5\FourCmPlusHalf]
            \&[0.5\FourCmPlusHalf]
            W\wedge(X\wedge(Y\wedge Z))
            \&[0.30901699437\FourCmPlusHalf]
            % 1-Arrows
            % Left Boundary
            \arrow[from=2-1,to=1-3,"\alpha^{\Sets_{*}}_{W,X,Y}\wedge\id_{Z}"{pos=0.4125}]%
            \arrow[from=1-3,to=2-5,"\alpha^{\Sets_{*}}_{W,X\wedge Y,Z}"{pos=0.6}]%
            \arrow[from=2-5,to=3-4,"\id_{W}\wedge\alpha^{\Sets_{*}}_{X,Y,Z}"{pos=0.425}]%
            % Right Boundary
            \arrow[from=2-1,to=3-2,"\alpha^{\Sets_{*}}_{W\wedge X,Y,Z}"'{pos=0.425}]%
            \arrow[from=3-2,to=3-4,"\alpha^{\Sets_{*}}_{W,X,Y\wedge Z}"']%
        \end{tikzcd}
    \]%
    commutes. Indeed, this diagram acts on elements as
    \[
        \begin{tikzcd}[row sep={0*\the\DL,between origins}, column sep={0*\the\DL,between origins}, background color=backgroundColor, ampersand replacement=\&]
            \&[0.30901699437\FourCm]
            \&[0.5\FourCm]
            (w\wedge(x\wedge y))\wedge z
            \&[0.5\FourCm]
            \&[0.30901699437\FourCm]
            \\[0.58778525229\FourCm]
            ((w\wedge x)\wedge y)\wedge z
            \&[0.30901699437\FourCm]
            \&[0.5\FourCm]
            \&[0.5\FourCm]
            \&[0.30901699437\FourCm]
            w\wedge((x\wedge y)\wedge z)
            \\[0.95105651629\FourCm]
            \&[0.30901699437\FourCm]
            (w\wedge x)\wedge(y\wedge z)
            \&[0.5\FourCm]
            \&[0.5\FourCm]
            w\wedge(x\wedge(y\wedge z))
            \&[0.30901699437\FourCm]
            % 1-Arrows
            % Left Boundary
            \arrow[from=2-1,to=1-3,mapsto]
            \arrow[from=1-3,to=2-5,mapsto]
            \arrow[from=2-5,to=3-4,mapsto]
            % Right Boundary
            \arrow[from=2-1,to=3-2,mapsto]
            \arrow[from=3-2,to=3-4,mapsto]
        \end{tikzcd}
    \]%
    and thus we see that the pentagon identity is satisfied.

    \ProofBox{The Triangle Identity}%
    Let $(X,x_{0})$ and $(Y,y_{0})$ be pointed sets. We have to show that the diagram
    \[
        \begin{tikzcd}[row sep={5.0*\the\DL,between origins}, column sep={5.0*\the\DL,between origins}, background color=backgroundColor, ampersand replacement=\&]
            (X\wedge S^{0})\wedge Y
            \arrow[rr, "\alpha^{\Sets_{*}}_{X,S^{0},Y}"]
            \arrow[rd, "\RUnitor^{\Sets_{*}}_{X}\wedge\id_{Y}"']
            \&
            \&
            X\wedge(S^{0}\wedge Y)
            \arrow[ld, "\id_{X}\wedge\LUnitor^{\Sets_{*}}_{Y}"]
            \\
            \&
            X\wedge Y
            \&
        \end{tikzcd}
    \]%
    commutes. Indeed, this diagram acts on elements as
    \begin{webcompile}
        \begin{tikzcd}[row sep={5.0*\the\DL,between origins}, column sep={5.0*\the\DL,between origins}, background color=backgroundColor, ampersand replacement=\&]
            (x\wedge 0)\wedge y
            \arrow[rd,mapsto]
            \&
            \&
            \\
            \&
            x_{0}\wedge y
            \&
        \end{tikzcd}
        \quad
        \begin{tikzcd}[row sep={5.0*\the\DL,between origins}, column sep={5.0*\the\DL,between origins}, background color=backgroundColor, ampersand replacement=\&]
            (x\wedge 0)\wedge y
            \arrow[rr,mapsto]
            \&
            \&
            x\wedge(0\wedge y)
            \arrow[ld,mapsto]
            \\
            \&
            x\wedge y_{0}
            \&
        \end{tikzcd}
    \end{webcompile}
    and
    \[
        \begin{tikzcd}[row sep={5.0*\the\DL,between origins}, column sep={5.0*\the\DL,between origins}, background color=backgroundColor, ampersand replacement=\&]
            (x\wedge 1)\wedge y
            \arrow[rr,mapsto]
            \arrow[rd,mapsto]
            \&
            \&
            x\wedge(1\wedge y)
            \arrow[ld,mapsto]
            \\
            \&
            x\wedge y\mrp{,}
            \&
        \end{tikzcd}
    \]%
    and thus we see that the triangle identity is satisfied.

    \ProofBox{The Left Hexagon Identity}%
    Let $(X,x_{0})$, $(Y,y_{0})$, and $(Z,z_{0})$ be pointed sets. We have to show that the diagram
    \[
        \begin{tikzcd}[row sep={0.0*\the\DL,between origins}, column sep={0.0*\the\DL,between origins}, background color=backgroundColor, ampersand replacement=\&]
            \&[0.86602540378\TwoCmPlusHalf]
            (X\wedge Y)\wedge Z
            \arrow[ld,"\alpha^{\Sets_{*}}_{X,Y,Z}"']
            \arrow[rd,"\beta^{\Sets_{*}}_{X,Y}\wedge\id_{Z}"]
            \&[0.86602540378\TwoCmPlusHalf]
            \\[0.5\TwoCmPlusHalf]
            X\wedge(Y\wedge Z)
            \arrow[d,"\beta^{\Sets_{*}}_{X,Y\wedge Z}"']
            \&[0.86602540378\TwoCmPlusHalf]
            \&[0.86602540378\TwoCmPlusHalf]
            (Y\wedge X)\wedge Z
            \arrow[d,"\alpha^{\Sets_{*}}_{Y,X,Z}"]
            \\[\TwoCmPlusHalf]
            (Y\wedge Z)\wedge X
            \arrow[rd,"\alpha^{\Sets_{*}}_{Y,Z,X}"']
            \&[0.86602540378\TwoCmPlusHalf]
            \&[0.86602540378\TwoCmPlusHalf]
            Y\wedge(X\wedge Z)
            \arrow[ld,"\id_{Y}\wedge\beta^{\Sets_{*}}_{X,Z}"]
            \\[0.5\TwoCmPlusHalf]
            \&[0.86602540378\TwoCmPlusHalf]
            Y\wedge(Z\wedge X)
            \&[0.86602540378\TwoCmPlusHalf]
        \end{tikzcd}
    \]%
    commutes. Indeed, this diagram acts on elements as
    \[
        \begin{tikzcd}[row sep={0.0*\the\DL,between origins}, column sep={0.0*\the\DL,between origins}, background color=backgroundColor, ampersand replacement=\&]
            \&[0.86602540378\TwoCmPlusHalf]
            (x\wedge y)\wedge z
            \arrow[ld,mapsto]
            \arrow[rd,mapsto]
            \&[0.86602540378\TwoCmPlusHalf]
            \\[0.5\TwoCmPlusHalf]
            x\wedge(y\wedge z)
            \arrow[d,mapsto]
            \&[0.86602540378\TwoCmPlusHalf]
            \&[0.86602540378\TwoCmPlusHalf]
            (y\wedge x)\wedge z
            \arrow[d,mapsto]
            \\[\TwoCmPlusHalf]
            (y\wedge z)\wedge x
            \arrow[rd,mapsto]
            \&[0.86602540378\TwoCmPlusHalf]
            \&[0.86602540378\TwoCmPlusHalf]
            y\wedge(x\wedge z)
            \arrow[ld,mapsto]
            \\[0.5\TwoCmPlusHalf]
            \&[0.86602540378\TwoCmPlusHalf]
            y\wedge(z\wedge x)
            \&[0.86602540378\TwoCmPlusHalf]
        \end{tikzcd}
    \]%
    and thus we see that the left hexagon identity is satisfied.

    \ProofBox{The Right Hexagon Identity}%
    Let $(X,x_{0})$, $(Y,y_{0})$, and $(Z,z_{0})$ be pointed sets. We have to show that the diagram
    \[
        \begin{tikzcd}[row sep={0.0*\the\DL,between origins}, column sep={0.0*\the\DL,between origins}, background color=backgroundColor, ampersand replacement=\&]
            \&[0.86602540378\TwoCmPlusHalf]
            X\wedge(Y\wedge Z)
            \arrow[ld,"(\alpha^{\Sets_{*}}_{X,Y,Z})^{-1}"']
            \arrow[rd,"\id_{X}\wedge\beta^{\Sets_{*}}_{Y,Z}"]
            \&[0.86602540378\TwoCmPlusHalf]
            \\[0.5\TwoCmPlusHalf]
            (X\wedge Y)\wedge Z
            \arrow[d,"\beta^{\Sets_{*}}_{X\wedge Y,Z}"']
            \&[0.86602540378\TwoCmPlusHalf]
            \&[0.86602540378\TwoCmPlusHalf]
            X\wedge(Z\wedge Y)
            \arrow[d,"(\alpha^{\Sets_{*}}_{X,Z,Y})^{-1}"]
            \\[\TwoCmPlusHalf]
            Z\wedge(X\wedge Y)
            \arrow[rd,"(\alpha^{\Sets_{*}}_{Z,X,Y})^{-1}"']
            \&[0.86602540378\TwoCmPlusHalf]
            \&[0.86602540378\TwoCmPlusHalf]
            (X\wedge Z)\wedge Y
            \arrow[ld,"\beta^{\Sets_{*}}_{X,Z}\wedge\id_{Y}"]
            \\[0.5\TwoCmPlusHalf]
            \&[0.86602540378\TwoCmPlusHalf]
            (Z\wedge X)\wedge Y
            \&[0.86602540378\TwoCmPlusHalf]
        \end{tikzcd}
    \]%
    commutes. Indeed, this diagram acts on elements as
    \[
        \begin{tikzcd}[row sep={0.0*\the\DL,between origins}, column sep={0.0*\the\DL,between origins}, background color=backgroundColor, ampersand replacement=\&]
            \&[0.86602540378\TwoCmPlusHalf]
            x\wedge(y\wedge z)
            \arrow[ld,mapsto]
            \arrow[rd,mapsto]
            \&[0.86602540378\TwoCmPlusHalf]
            \\[0.5\TwoCmPlusHalf]
            (x\wedge y)\wedge z
            \arrow[d,mapsto]
            \&[0.86602540378\TwoCmPlusHalf]
            \&[0.86602540378\TwoCmPlusHalf]
            x\wedge(z\wedge y)
            \arrow[d,mapsto]
            \\[\TwoCmPlusHalf]
            z\wedge(x\wedge y)
            \arrow[rd,mapsto]
            \&[0.86602540378\TwoCmPlusHalf]
            \&[0.86602540378\TwoCmPlusHalf]
            (x\wedge z)\wedge y
            \arrow[ld,mapsto]
            \\[0.5\TwoCmPlusHalf]
            \&[0.86602540378\TwoCmPlusHalf]
            (z\wedge x)\wedge y
            \&[0.86602540378\TwoCmPlusHalf]
        \end{tikzcd}
    \]%
    and thus we see that the right hexagon identity is satisfied.

    \ProofBox{Monoidal Closedness}%
    This follows from \cref{properties-of-smash-products-of-pointed-sets-adjointness} of \cref{properties-of-smash-products-of-pointed-sets}.

    \ProofBox{Existence of Monoidal Diagonals}%
    This follows from \cref{properties-of-the-diagonal-of-the-smash-product-of-pointed-sets-monoidality,properties-of-the-diagonal-of-the-smash-product-of-pointed-sets-the-diagonal-of-the-unit} of \cref{properties-of-the-diagonal-of-the-smash-product-of-pointed-sets}.
\end{Proof}
\subsection{The Universal Property of $(\Sets_{*},\wedge,S^{0})$}\label{subsection-the-universal-property-of-the-smash-product-of-pointed-sets}
\begin{theorem}{The Universal Property of $(\Sets_{*},\wedge,S^{0})$}{the-universal-property-of-sets-star-smash-s-zero}%
    The symmetric monoidal structure on the category $\Sets_{*}$ of \cref{the-monoidal-structure-on-pointed-sets-associated-to-the-smash-product-of-pointed-sets} is uniquely determined by the following requirements:
    \begin{enumerate}
        \item\label{the-universal-property-of-sets-star-smash-s-zero-existence-of-an-internal-hom}\SloganFont{Existence of an Internal Hom. }The tensor product
            \[
                \otimes_{\Sets_{*}}%
                \colon
                \Sets_{*}\times\Sets_{*}
                \to
                \Sets_{*}
            \]%
            of $\Sets_{*}$ admits an internal Hom $[-_{1},-_{2}]_{\Sets_{*}}$.
        \item\label{the-universal-property-of-sets-star-smash-s-zero-the-unit-object-is-s-zero}\SloganFont{The Unit Object Is $S^{0}$. }We have $\Unit_{\Sets_{*}}\cong S^{0}$.
    \end{enumerate}
    More precisely, the full subcategory of the category $\ModuliCategory^{\cld}_{\E_{\infty}}(\Sets_{*})$ of \cref{TODO} spanned by the closed symmetric monoidal categories $\left(\phantom{\mrp{\LUnitor^{\Sets_{*}}}}\Sets_{*}\right.$, $\otimes_{\Sets_{*}}$, $[-_{1},-_{2}]_{\Sets_{*}}$, $\Unit_{\Sets_{*}}$, $\LUnitor^{\Sets_{*}}$, $\RUnitor^{\Sets_{*}}$, $\left.\sigma^{\Sets_{*}}\right)$ satisfying \cref{the-universal-property-of-sets-star-smash-s-zero-existence-of-an-internal-hom,the-universal-property-of-sets-star-smash-s-zero-the-unit-object-is-s-zero} is contractible (i.e.\ equivalent to the punctual category).
\end{theorem}
\begin{Proof}{Proof of \cref{the-universal-property-of-sets-star-smash-s-zero}}%
    \FirstProofBox{Unwinding the Statement}%
    Let $(\Sets_{*},\otimes_{\Sets_{*}},[-_{1},-_{2}]_{\Sets_{*}},\Unit_{\Sets_{*}},\LUnitor',\RUnitor',\sigma')$ be a closed symmetric monoidal category satisfying \cref{the-universal-property-of-sets-star-smash-s-zero-existence-of-an-internal-hom,the-universal-property-of-sets-star-smash-s-zero-the-unit-object-is-s-zero}. We need to show that the identity functor
    \[
        \id_{\Sets_{*}}%
        \colon%
        \Sets_{*}%
        \to%
        \Sets_{*}%
    \]%
    admits a \emph{unique} closed symmetric monoidal functor structure
    \[%
        \begin{array}{cccc}
            \phantom{\id^{\otimes}_{\Sets_{*}}}\mlp{\id^{\otimes}_{\Sets_{*}}}       \colon\mkern-10mu & X\otimes_{\Sets_{*}}Y \mkern-10mu&{}\mathbin{\isorightarrow}&\mkern-10mu{}X\wedge Y,\\
            \phantom{\id^{\otimes}_{\Sets_{*}}}\mlp{\id^{\Hom}_{\Sets_{*}}}          \colon\mkern-10mu & [X,Y]_{\Sets_{*}}     \mkern-10mu&{}\mathbin{\isorightarrow}&\mkern-10mu{}\Sets_{*}(X,Y),\\
            \phantom{\id^{\otimes}_{\Sets_{*}}}\mlp{\id^{\otimes}_{\Unit|\Sets_{*}}} \colon\mkern-10mu & \Unit_{\Sets_{*}}     \mkern-10mu&{}\mathbin{\isorightarrow}&\mkern-10mu{}S^{0},
        \end{array}
    \]%
    making it into a symmetric monoidal strongly closed isomorphism of categories from $\left(\phantom{\mrp{\LUnitor'}}\Sets_{*}\right.$, $\otimes_{\Sets_{*}}$, $[-_{1},-_{2}]_{\Sets_{*}}$, $\Unit_{\Sets_{*}}$, $\LUnitor'$, $\RUnitor'$, $\left.\sigma'\right)$ to the closed symmetric monoidal category $\left(\phantom{\mrp{\LUnitor^{\Sets_{*}}}}\Sets_{*}\right.$, $\times$, $\Sets_{*}(-_{1},-_{2})$, $\Unit_{\Sets_{*}}$, $\LUnitor^{\Sets_{*}}$, $\RUnitor^{\Sets_{*}}$, $\left.\sigma^{\Sets_{*}}\right)$ of \cref{the-monoidal-structure-on-pointed-sets-associated-to-the-smash-product-of-pointed-sets}.

    \ProofBox{Constructing an Isomorphism $[-_{1},-_{2}]_{\Sets_{*}}\cong\Sets_{*}(-_{1},-_{2})$}%
    By \cref{TODO}, we have a natural isomorphism
    \[
        \Sets_{*}(S^{0},[-_{1},-_{2}]_{\Sets_{*}})%
        \cong%
        \Sets_{*}(-_{1},-_{2}).%
    \]%
    By \ChapterRef{\ChapterPointedSets, \cref{pointed-sets:elementary-properties-of-pointed-sets-morphisms-from-the-monoidal-unit} of \cref{pointed-sets:elementary-properties-of-pointed-sets}}{\cref{elementary-properties-of-pointed-sets-morphisms-from-the-monoidal-unit} of \cref{elementary-properties-of-pointed-sets}}, we also have a natural isomorphism
    \[
        \Sets_{*}(S^{0},[-_{1},-_{2}]_{\Sets_{*}})%
        \cong%
        [-_{1},-_{2}]_{\Sets_{*}}.%
    \]%
    Composing both natural isomorphisms, we obtain a natural isomorphism
    \[
        \Sets_{*}(-_{1},-_{2})%
        \cong%
        [-_{1},-_{2}]_{\Sets_{*}}.%
    \]%
    Given $X,Y\in\Obj(\Sets_{*})$, we will write
    \[
        \id^{\Hom}_{X,Y}%
        \colon%%
        \Sets_{*}(X,Y)%
        \isorightarrow%
        [X,Y]_{\Sets_{*}}%
    \]%
    for the component of this isomorphism at $(X,Y)$.

    \ProofBox{Constructing an Isomorphism $\mathord{\otimes_{\Sets_{*}}}\cong\mathord{\wedge}$}%
    Since $\otimes_{\Sets_{*}}$ is adjoint in each variable to $[-_{1},-_{2}]_{\Sets_{*}}$ by assumption and $\wedge$ is adjoint in each variable to $\Sets_{*}(-_{1},-_{2})$ by \ChapterRef{\ChapterConstructionsWithSets, \cref{constructions-with-sets:properties-of-sets-of-maps-adjointness} of \cref{constructions-with-sets:properties-of-sets-of-maps}}{\cref{properties-of-sets-of-maps-adjointness} of \cref{properties-of-sets-of-maps}}, uniqueness of adjoints (\cref{TODO}) gives us natural isomorphisms
    \begin{align*}
        X\otimes_{\Sets_{*}}- &\cong X\wedge-,\\
        -\otimes_{\Sets_{*}}Y &\cong Y\wedge-.
    \end{align*}
    By \cref{TODO}, we then have $\mathord{\otimes_{\Sets_{*}}}\cong\mathord{\wedge}$. We will write
    \[
        \id^{\otimes}_{\Sets_{*}|X,Y}%
        \colon%
        X\otimes_{\Sets_{*}}Y%
        \isorightarrow%
        X\wedge Y%
    \]%
    for the component of this isomorphism at $(X,Y)$.

    \ProofBox{Alternative Construction of an Isomorphism $\mathord{\otimes_{\Sets_{*}}}\cong\mathord{\wedge}$}%
    Alternatively, we may construct a natural isomorphism $\mathord{\otimes_{\Sets_{*}}}\cong\mathord{\wedge}$ as follows:
    \begin{enumerate}
        \item\label{proof-of-the-universal-property-of-sets-times-pt-constructing-an-isomorphism-otimes-times-1}Let $X\in\Obj(\Sets_{*})$.
        \item\label{proof-of-the-universal-property-of-sets-times-pt-constructing-an-isomorphism-otimes-times-2}Since $\otimes_{\Sets_{*}}$ is part of a closed monoidal structure, it preserves colimits in each variable by \cref{TODO}.
        \item\label{proof-of-the-universal-property-of-sets-times-pt-constructing-an-isomorphism-otimes-times-3}Since $X\cong\bigvee_{x\in X^{-}}S^{0}$ and $\otimes_{\Sets_{*}}$ preserves colimits in each variable, we have
            \begin{align*}
                X\otimes_{\Sets_{*}}Y &\cong (\bigvee_{x\in X^{-}}S^{0})\otimes_{\Sets_{*}}Y\\
                                      &\cong \bigvee_{x\in X^{-}}(S^{0}\otimes_{\Sets_{*}}Y)\\
                                      &\cong \bigvee_{x\in X^{-}}Y\\
                                      &\cong \bigvee_{x\in X^{-}}S^{0}\wedge Y\\
                                      &\cong (\bigvee_{x\in X^{-}}S^{0})\wedge Y\\
                                      &\cong X\wedge Y,
            \end{align*}
            naturally in $Y\in\Obj(\Sets_{*})$, where we have used that $S^{0}$ is the monoidal unit for $\otimes_{\Sets_{*}}$. Thus $X\otimes_{\Sets_{*}}-\cong X\wedge-$ for each $X\in\Obj(\Sets_{*})$.
        \item\label{proof-of-the-universal-property-of-sets-times-pt-constructing-an-isomorphism-otimes-times-4}Similarly, $-\otimes_{\Sets_{*}}Y\cong-\wedge Y$ for each $Y\in\Obj(\Sets_{*})$.
        \item\label{proof-of-the-universal-property-of-sets-times-pt-constructing-an-isomorphism-otimes-times-5}By \cref{TODO}, we then have $\mathord{\otimes_{\Sets_{*}}}\cong\mathord{\wedge}$.
    \end{enumerate}
    Below, we'll show that if a natural isomorphism $\mathord{\otimes_{\Sets_{*}}}\cong\mathord{\wedge}$ exists, then it must be unique. This will show that the isomorphism constructed above is equal to the isomorphism $\id^{\otimes}_{\Sets_{*}|X,Y}\colon X\otimes_{\Sets_{*}}Y\to X\wedge Y$ from before.

    \ProofBox{Constructing an Isomorphism $\id^{\otimes}_{\Unit}\colon\Unit_{\Sets_{*}}\to S^{0}$}%
    We define an isomorphism $\id^{\otimes}_{\Unit}\colon\Unit_{\Sets_{*}}\to S^{0}$ as the composition
    \[
        \Unit_{\Sets_{*}}%
        \xrightisoarrow{\RUnitor^{\Sets_{*},-1}_{\Unit_{\Sets_{*}}}}%
        \Unit_{\Sets_{*}}\wedge S^{0}%
        \xrightisoarrow{\id^{\otimes,-1}_{\Sets_{*}|\Unit_{\Sets_{*}}}}%
        \Unit_{\Sets_{*}}\otimes_{\Sets_{*}} S^{0}%
        \xrightisoarrow{\LUnitor'_{ S^{0}}}%
        S^{0}%
    \]%
    in $\Sets_{*}$.

    \ProofBox{Monoidal Left Unity of the Isomorphism $\mathord{\otimes_{\Sets_{*}}}\cong\mathord{\wedge}$}%
    We have to show that the diagram
    \[
        \begin{tikzcd}[row sep={0.0*\the\DL,between origins}, column sep={0.0*\the\DL,between origins}, background color=backgroundColor, ampersand replacement=\&]
            \&[0.5\ThreeCmPlusAQuarter]
            S^{0}\otimes_{\Sets_{*}}X
            \&[0.9\ThreeCmPlusAQuarter]
            S^{0}\wedge X
            \&[0.5\ThreeCmPlusAQuarter]
            \\[0.5\ThreeCmPlusAQuarter]
            \Unit_{\Sets_{*}}\otimes_{\Sets_{*}}X
            \&[0.5\ThreeCmPlusAQuarter]
            \&[0.9\ThreeCmPlusAQuarter]
            \&[0.5\ThreeCmPlusAQuarter]
            X
            % 1-Arrows
            \arrow[from=2-1,to=1-2,"\id^{\otimes}_{\Unit|\Sets_{*}}\otimes_{\Sets_{*}}\id_{X}",pos=0.325]%
            \arrow[from=1-2,to=1-3,"\id^{\otimes}_{\Sets_{*}|S^{0},X}"]%
            \arrow[from=1-3,to=2-4,"\LUnitor^{\Sets_{*}}_{X}",pos=0.525]%
            %
            \arrow[from=2-1,to=2-4,"\LUnitor'_{X}"',pos=0.445]%
        \end{tikzcd}
    \]%
    commutes. To this end, we will first show that the diagram
    \[
        \begin{tikzcd}[row sep={0.0*\the\DL,between origins}, column sep={0.0*\the\DL,between origins}, background color=backgroundColor, ampersand replacement=\&]
            \&[0.5\ThreeCmPlusAQuarter]
            S^{0}\otimes_{\Sets_{*}}S^{0}
            \&[0.45\ThreeCmPlusAQuarter]
            {}
            \&[0.45\ThreeCmPlusAQuarter]
            S^{0}\wedge S^{0}
            \&[0.5\ThreeCmPlusAQuarter]
            \\[0.5\ThreeCmPlusAQuarter]
            \Unit_{\Sets_{*}}\otimes_{\Sets_{*}}S^{0}
            \&[0.5\ThreeCmPlusAQuarter]
            \&[0.45\ThreeCmPlusAQuarter]
            {}
            \&[0.45\ThreeCmPlusAQuarter]
            \&[0.5\ThreeCmPlusAQuarter]
            S^{0}\mrp{,}
            % 1-Arrows
            \arrow[from=2-1,to=1-2,"\id^{\otimes}_{\Unit|\Sets_{*}}\otimes_{\Sets_{*}}\id_{S^{0}}",pos=0.425]%
            \arrow[from=1-2,to=1-4,"\id^{\otimes}_{\Sets_{*}|S^{0},S^{0}}"]%
            \arrow[from=1-4,to=2-5,"\LUnitor^{\Sets_{*}}_{S^{0}}",pos=0.575]%
            %
            \arrow[from=2-1,to=2-5,"\LUnitor'_{S^{0}}"',pos=0.425]%
            % Subdiagram Mark
            \arrow[from=1-3,to=2-3,phantom,"\scriptstyle(\dagger)",pos=0.425]%
        \end{tikzcd}
    \]%
    corresponding to the case $X=S^{0}$, commutes. Indeed, consider the diagram
    \begin{scalemath}
        \begin{tikzcd}[row sep={15.0*\the\DL,between origins}, column sep={15.0*\the\DL,between origins}, background color=backgroundColor, ampersand replacement=\&]
            \Unit_{\Sets_{*}}\otimes_{\Sets_{*}}S^{0}
            \&[1.0*\the\DL]
            (\Unit_{\Sets_{*}}\wedge S^{0})\otimes_{\Sets_{*}}S^{0}
            \&[4.5*\the\DL]
            (\Unit_{\Sets_{*}}\otimes_{\Sets_{*}}S^{0})\otimes_{\Sets_{*}}S^{0}
            \&
            S^{0}\otimes_{\Sets_{*}}S^{0}
            \\
            \Unit_{\Sets_{*}}\wedge S^{0}
            \&[1.0*\the\DL]
            (\Unit_{\Sets_{*}}\wedge S^{0})\wedge S^{0}
            \&[4.5*\the\DL]
            (\Unit_{\Sets_{*}}\otimes_{\Sets_{*}}S^{0})\wedge S^{0}
            \&
            S^{0}\wedge S^{0}
            \\
            \Unit_{\Sets_{*}}\otimes_{\Sets_{*}}S^{0}
            \&[1.0*\the\DL]
            \&[4.5*\the\DL]
            \Unit_{\Sets_{*}}\otimes_{\Sets_{*}}S^{0}
            \&
            S^{0}\mrp{,}
            % Arrows
            % First Row
            \arrow[from=1-1,to=1-2,"\RUnitor^{\Sets_{*},-1}_{S^{0}}\otimes_{\Sets_{*}}\id_{S^{0}}"]%
            \arrow[from=1-2,to=1-3,"\scalebox{0.9}{$\id^{\otimes,-1}_{\Sets_{*}|\Unit_{\Sets_{*},S^{0}}}\otimes_{\Sets_{*}}\id_{S^{0}}$}"]%
            \arrow[from=1-3,to=1-4,"\LUnitor'_{S^{0}}\otimes_{\Sets_{*}}\id_{S^{0}}"]%
            % Second Row
            \arrow[from=2-1,to=2-2,"\RUnitor^{\Sets_{*},-1}_{\Unit_{\Sets_{*}}}\wedge\id_{S^{0}}"{description},bend left=25]%
            \arrow[from=2-1,to=2-2,"\RUnitor^{\Sets_{*},-1}_{\Unit_{\Sets_{*}}\wedge S^{0}}"{description},bend right=25]%
            \arrow[from=2-2,to=2-3,"\id^{\otimes,-1}_{\Sets_{*}|\Unit_{\Sets_{*},S^{0}}}\wedge\id_{S^{0}}"description]%
            \arrow[from=2-3,to=2-4,"\LUnitor'_{S^{0}}\wedge\id_{S^{0}}"description]%
            % Third Row
            \arrow[from=3-1,to=3-3,Equals]%
            \arrow[from=3-3,to=3-4,"\LUnitor'_{S^{0}}"']%
            % First Column
            \arrow[from=1-1,to=2-1,"\id^{\otimes}_{\Sets_{*}|\Unit_{\Sets_{*}},S^{0}}"description]%
            \arrow[from=2-1,to=3-1,"\id^{\otimes,-1}_{\Sets_{*}|\Unit_{\Sets_{*}},S^{0}}"description]%
            \arrow[from=1-1,to=3-1,equals,lddr_to_path]
            % Second Column
            \arrow[from=1-2,to=2-2,"\id^{\otimes}_{\Sets_{*}|\Unit_{\Sets_{*}}\wedge S^{0},S^{0}}"description]%
            % Third Column
            \arrow[from=1-3,to=2-3,"\id^{\otimes}_{\Sets_{*}|\Unit_{\Sets_{*}}\otimes_{\Sets_{*}}S^{0},S^{0}}"description]%
            \arrow[from=2-3,to=3-3,"\RUnitor^{\Sets_{*}}_{\Unit_{\Sets_{*}}\otimes_{\Sets_{*}S^{0}}}"description]%
            % Fourth Column
            \arrow[from=1-4,to=2-4,"\id^{\otimes}_{\Sets_{*}|S^{0},S^{0}}"]%
            \arrow[from=2-4,to=3-4,"\RUnitor^{\Sets_{*}}_{S^{0}}=\LUnitor^{\Sets_{*}}_{S^{0}}"]%
            % Diagonal Arrow
            \arrow[from=3-1,to=2-3,"\RUnitor^{\Sets_{*},-1}_{\Unit_{\Sets_{*}}\otimes_{\Sets_{*}S^{0}}}"description,""'{name=R}]%
            % Subdiagrams
            \arrow[from=1-1,to=2-2,"(1)",phantom]%
            \arrow[from=1-2,to=2-3,"(2)",phantom]%
            \arrow[from=1-3,to=2-4,"(3)",phantom]%
            \arrow[from=2-1,to=2-2,"(4)"{pos=0.55},phantom]%
            %
            \arrow[from=2-1,to=R,"(5)",phantom,xshift=-1.0em,yshift=-2.25em]%
            \arrow[from=R,to=3-3,"(6)",phantom,xshift=+1.75em,yshift=+1.17em]%
            %
            \arrow[from=2-3,to=3-4,"(7)",phantom]%
        \end{tikzcd}
    \end{scalemath}
    whose boundary diagram corresponds to the diagram $(\dagger)$ above. In this diagram:
    \begin{itemize}
        \item Subdiagrams $(1)$, $(2)$, and $(3)$ commute by the naturality of $\id^{\otimes}_{\Sets_{*}}$.
        \item Subdiagram $(4)$ commutes by \cref{TODO}.
        \item Subdiagram $(5)$ commutes by the naturality of $\RUnitor^{\Sets_{*},-1}$.
        \item Subdiagram $(6)$ commutes trivially.
        \item Subdiagram $(7)$ commutes by the naturality of $\RUnitor^{\Sets_{*}}$, where the equality $\RUnitor^{\Sets_{*}}_{S^{0}}=\LUnitor^{\Sets_{*}}_{S^{0}}$ comes from \cref{TODO}.
    \end{itemize}
    Since all subdiagrams commute, so does the boundary diagram, i.e.\ the diagram $(\dagger)$ above. As a result, the diagram
    \[
        \begin{tikzcd}[row sep={0.0*\the\DL,between origins}, column sep={0.0*\the\DL,between origins}, background color=backgroundColor, ampersand replacement=\&]
            \&[0.5\ThreeCmPlusAQuarter]
            S^{0}\wedge S^{0}
            \&[0.45\ThreeCmPlusAQuarter]
            {}
            \&[0.45\ThreeCmPlusAQuarter]
            S^{0}\otimes_{\Sets_{*}}S^{0}
            \&[0.5\ThreeCmPlusAQuarter]
            \\[0.5\ThreeCmPlusAQuarter]
            S^{0}
            \&[0.5\ThreeCmPlusAQuarter]
            \&[0.45\ThreeCmPlusAQuarter]
            {}
            \&[0.45\ThreeCmPlusAQuarter]
            \&[0.5\ThreeCmPlusAQuarter]
            \Unit_{\Sets_{*}}\otimes_{\Sets_{*}}S^{0}
            % 1-Arrows
            \arrow[from=2-1,to=1-2,"\LUnitor^{\Sets_{*},-1}_{S^{0}}",pos=0.45]%
            \arrow[from=1-2,to=1-4,"\id^{\otimes,-1}_{\Sets_{*}|S^{0},S^{0}}"]%
            \arrow[from=1-4,to=2-5,"\id^{\otimes,-1}_{\Unit|\Sets_{*}}\otimes_{\Sets_{*}}\id_{S^{0}}",pos=0.55]%
            %
            \arrow[from=2-1,to=2-5,"\LUnitor^{\prime,-1}_{S^{0}}"',pos=0.5675]%
            % Subdiagram Mark
            \arrow[from=1-3,to=2-3,phantom,"\scriptstyle(\ddagger)",pos=0.425]%
        \end{tikzcd}
    \]%
    also commutes. Now, let $X\in\Obj(\Sets_{*})$, let $x\in X$, and consider the diagram
    \[
        \begin{tikzcd}[row sep={0.0*\the\DL,between origins}, column sep={0.0*\the\DL,between origins}, background color=backgroundColor, ampersand replacement=\&]
            \&[0.85\ThreeCm]
            S^{0}\wedge S^{0}
            \&[0.6\ThreeCm]
            {}
            \&[0.6\ThreeCm]
            S^{0}\otimes_{\Sets_{*}}S^{0}
            \&[0.85\ThreeCm]
            \\[0.5\ThreeCm]
            S^{0}
            \&[0.85\ThreeCm]
            \&[0.6\ThreeCm]
            {}
            \&[0.6\ThreeCm]
            \&[0.85\ThreeCm]
            \Unit_{\Sets_{*}}\otimes_{\Sets_{*}}S^{0}
            \\[1.0*\ThreeCm]
            \&[0.85\ThreeCm]
            S^{0}\wedge X
            \&[0.6\ThreeCm]
            {}
            \&[0.6\ThreeCm]
            S^{0}\otimes_{\Sets_{*}}X
            \&[0.85\ThreeCm]
            \\[0.5\ThreeCm]
            X
            \&[0.85\ThreeCm]
            \&[0.6\ThreeCm]
            {}
            \&[0.6\ThreeCm]
            \&[0.85\ThreeCm]
            \Unit_{\Sets_{*}}\otimes_{\Sets_{*}}X\mrp{.}
            % 1-Arrows
            % First Diagram
            \arrow[from=2-1,to=1-2,"\LUnitor^{\Sets_{*},-1}_{S^{0}}"{sloped}]%
            \arrow[from=1-2,to=1-4,"\id^{\otimes,-1}_{\Sets_{*}|S^{0},S^{0}}",pos=0.525]%
            \arrow[from=1-4,to=2-5,"\id^{\otimes,-1}_{\Unit|\Sets_{*}}\wedge\id_{S^{0}}"{sloped}]%
            % Second Diagram
            \arrow[from=4-1,to=3-2,"\LUnitor^{\Sets_{*},-1}_{X}"{description,sloped}]%
            \arrow[from=3-2,to=3-4,"\id^{\otimes,-1}_{\Sets_{*}|S^{0},X}"description,pos=0.55]%
            \arrow[from=3-4,to=4-5,"\id^{\otimes,-1}_{\Unit|\Sets_{*}}\wedge\id_{X}"{description,sloped}]%
            \arrow[from=4-1,to=4-5,"\LUnitor^{\prime,-1}_{X}"description,pos=0.55]%
            % Connecting Arrows
            \arrow[from=2-1,to=4-1,"{[x]}"']%
            \arrow[from=1-2,to=3-2,"{\id_{S^{0}}\wedge[x]}"description]%
            \arrow[from=1-4,to=3-4,"{\id_{S^{0}}\otimes_{\Sets_{*}}[x]}"description]%
            \arrow[from=2-5,to=4-5,"{\id_{\Unit_{\Sets_{*}}}\wedge[x]}"]%
            % Crossing Over
            \arrow[from=2-1,to=2-5,"\LUnitor^{\prime,-1}_{S^{0}}"'description,pos=0.55,crossing over]%
            % Subdiagrams
            \arrow[from=1-3,to=2-3,"\scriptstyle(\ddagger)"{xscale=2.0,yscale=0.7,pos=0.4},phantom]%
            \arrow[from=3-2,to=1-4,"\scriptstyle(1)",phantom]%
            \arrow[from=3-3,to=4-3,"\scriptstyle(2)"{xscale=2.0,yscale=0.7,pos=0.45},phantom]%
            \arrow[from=2-1,to=3-2,"(3)"{rotate=34.5+0.0, xslant=0.656, yslant=0},phantom]%
            \arrow[from=3-4,to=2-5,"(4)"{rotate=-34.5, xslant=-0.656, yslant=0},phantom]%
            \arrow[from=2-3,to=4-3,"(5)"{xscale=1.3,yscale=1.1},phantom]%
        \end{tikzcd}
    \]%
    Since:
    \begin{itemize}
        \item Subdiagram $(5)$ commutes by the naturality of $\LUnitor^{\prime,-1}$.
        \item Subdiagram $(\ddagger)$ commutes, as proved above.
        \item Subdiagram $(4)$ commutes by the naturality of $\id^{\otimes,-1}_{\Unit|\Sets_{*}}$.
        \item Subdiagram $(1)$ commutes by the naturality of $\id^{\otimes,-1}_{\Sets_{*}}$.
        \item Subdiagram $(3)$ commutes by the naturality of $\LUnitor^{\Sets_{*},-1}$.
    \end{itemize}
    it follows that the diagram
    \[
        \begin{tikzcd}[row sep={0.0*\the\DL,between origins}, column sep={0.0*\the\DL,between origins}, background color=backgroundColor, ampersand replacement=\&]
            \&[0.5\ThreeCmPlusAQuarter]
            \&[0.5\ThreeCmPlusAQuarter]
            S^{0}\wedge X
            \&[0.45\ThreeCmPlusAQuarter]
            {}
            \&[0.45\ThreeCmPlusAQuarter]
            S^{0}\otimes_{\Sets_{*}}X
            \&[0.5\ThreeCmPlusAQuarter]
            \\[0.5\ThreeCmPlusAQuarter]
            S^{0}%
            \&[0.5\ThreeCmPlusAQuarter]
            X
            \&[0.5\ThreeCmPlusAQuarter]
            \&[0.45\ThreeCmPlusAQuarter]
            {}
            \&[0.45\ThreeCmPlusAQuarter]
            \&[0.5\ThreeCmPlusAQuarter]
            \Unit_{\Sets_{*}}\otimes_{\Sets_{*}}X
            % 1-Arrows
            \arrow[from=2-1,to=2-2,"{[x]}"]%
            \arrow[from=2-2,to=1-3,"\LUnitor^{\Sets_{*},-1}_{X}",pos=0.45]%
            \arrow[from=1-3,to=1-5,"\id^{\otimes,-1}_{\Sets_{*}|S^{0},X}"]%
            \arrow[from=1-5,to=2-6,"\id^{\otimes,-1}_{\Unit|\Sets_{*}}\otimes_{\Sets_{*}}\id_{X}",pos=0.55]%
            \arrow[from=2-2,to=2-6,"\LUnitor^{\prime,-1}_{X}"',pos=0.5675]%
        \end{tikzcd}
    \]%
    \begin{envwebgif}
        Here's a step-by-step showcase of this argument: \webgif{monoidal-left-unity-of-id-otimes-sets-star.gif}
    \end{envwebgif}
    We then have
    \begin{align*}
        \LUnitor^{\prime,-1}_{X}(x) &= [\LUnitor^{\prime,-1}_{X}\circ[x]](1)\\
                                    &= [(\id^{\otimes,-1}_{\Unit|\Sets_{*}}\wedge\id_{X})\circ\id^{\otimes,-1}_{\Sets_{*}|S^{0},X}\circ\LUnitor^{\Sets_{*},-1}_{X}\circ[x]](1)\\
                                    &= [(\id^{\otimes,-1}_{\Unit|\Sets_{*}}\wedge\id_{X})\circ\id^{\otimes,-1}_{\Sets_{*}|S^{0},X}\circ\LUnitor^{\Sets_{*},-1}_{X}](x)
    \end{align*}
    for each $x\in X$, and thus we have
    \[
        \LUnitor^{\prime,-1}_{X}%
        =%
        (\id^{\otimes,-1}_{\Unit|\Sets_{*}}\wedge\id_{X})\circ\id^{\otimes,-1}_{\Sets_{*}|S^{0},X}\circ\LUnitor^{\Sets_{*},-1}_{X}.
    \]%
    Taking inverses then gives
    \[
        \LUnitor^{\prime}_{X}%
        =%
        \LUnitor^{\Sets_{*}}_{X}\circ\id^{\otimes}_{\Sets_{*}|S^{0},X}\circ(\id^{\otimes}_{\Unit|\Sets_{*}}\wedge\id_{X}),
    \]%
    showing that the diagram
    \[
        \begin{tikzcd}[row sep={0.0*\the\DL,between origins}, column sep={0.0*\the\DL,between origins}, background color=backgroundColor, ampersand replacement=\&]
            \&[0.5\ThreeCmPlusAQuarter]
            S^{0}\otimes_{\Sets_{*}}X
            \&[0.9\ThreeCmPlusAQuarter]
            S^{0}\wedge X
            \&[0.5\ThreeCmPlusAQuarter]
            \\[0.5\ThreeCmPlusAQuarter]
            \Unit_{\Sets_{*}}\otimes_{\Sets_{*}}X
            \&[0.5\ThreeCmPlusAQuarter]
            \&[0.9\ThreeCmPlusAQuarter]
            \&[0.5\ThreeCmPlusAQuarter]
            X
            % 1-Arrows
            \arrow[from=2-1,to=1-2,"\id^{\otimes}_{\Unit|\Sets_{*}}\otimes_{\Sets_{*}}\id_{X}",pos=0.325]%
            \arrow[from=1-2,to=1-3,"\id^{\otimes}_{\Sets_{*}|S^{0},X}"]%
            \arrow[from=1-3,to=2-4,"\LUnitor^{\Sets_{*}}_{X}",pos=0.525]%
            %
            \arrow[from=2-1,to=2-4,"\LUnitor'_{X}"',pos=0.445]%
        \end{tikzcd}
    \]%
    indeed commutes.

    \ProofBox{Braidedness of the Isomorphism $\mathord{\otimes_{\Sets_{*}}}\cong\mathord{\wedge}$}%
    We have to show that the diagram
    \[
        \begin{tikzcd}[row sep={5.0*\the\DL,between origins}, column sep={8.5*\the\DL,between origins}, background color=backgroundColor, ampersand replacement=\&]
            X\otimes_{\Sets_{*}}Y
            \arrow[r,"\id^{\otimes}_{\Sets_{*}|X,Y}"]
            \arrow[d,"\sigma'_{X,Y}"']
            \&
            X\wedge Y
            \arrow[d,"\sigma^{\Sets_{*}}_{X,Y}"]
            \\
            Y\otimes_{\Sets_{*}}X
            \arrow[r,"\id^{\otimes}_{\Sets_{*}|Y,X}"']
            \&
            Y\wedge X
        \end{tikzcd}
    \]%
    commutes. To this end, we will first show that the diagram
    \[
        \begin{tikzcd}[row sep={5.0*\the\DL,between origins}, column sep={9.75*\the\DL,between origins}, background color=backgroundColor, ampersand replacement=\&]
            S^{0}\otimes_{\Sets_{*}}S^{0}
            \arrow[r,"\id^{\otimes}_{\Sets_{*}|S^{0},S^{0}}"]
            \arrow[d,"\sigma'_{S^{0},S^{0}}"']
            \&
            S^{0}\wedge S^{0}
            \arrow[d,"\sigma^{\Sets_{*}}_{S^{0},S^{0}}"]
            \\
            S^{0}\otimes_{\Sets_{*}}S^{0}
            \arrow[r,"\id^{\otimes}_{\Sets_{*}|S^{0},S^{0}}"']
            \&
            S^{0}\wedge S^{0}
            % Subdiagrams
            \arrow[from=1-1,to=2-2,"(\dagger)",phantom]%
        \end{tikzcd}
    \]%
    commutes. To that end, we will first show that the diagram
    \[
        \begin{tikzcd}[row sep={5.0*\the\DL,between origins}, column sep={12.0*\the\DL,between origins}, background color=backgroundColor, ampersand replacement=\&]
            S^{0}\otimes_{\Sets_{*}}\Unit_{\Sets_{*}}
            \arrow[r,"\id^{\otimes}_{\Sets_{*}|S^{0},\Unit_{\Sets_{*}}}"]
            \arrow[d,"\sigma'_{S^{0},\Unit_{\Sets_{*}}}"']
            \&
            S^{0}\wedge \Unit_{\Sets_{*}}
            \arrow[d,"\sigma^{\Sets_{*}}_{S^{0},\Unit_{\Sets_{*}}}"]
            \\
            \Unit_{\Sets_{*}}\otimes_{\Sets_{*}}S^{0}
            \arrow[r,"\id^{\otimes}_{\Sets_{*}|\Unit_{\Sets_{*}},S^{0}}"']
            \&
            \Unit_{\Sets_{*}}\wedge S^{0}
            % Subdiagrams
            \arrow[from=1-1,to=2-2,"(\ddagger)",phantom]%
        \end{tikzcd}
    \]%
    commutes, and, to this end, we will first show that the diagram
    \[
        \begin{tikzcd}[row sep={6.0*\the\DL,between origins}, column sep={10.0*\the\DL,between origins}, background color=backgroundColor, ampersand replacement=\&]
            S^{0}\otimes_{\Sets_{*}}S^{0}
            \arrow[r,"\id^{\otimes}_{\Sets_{*}|S^{0},S^{0}}"]
            \&
            S^{0}\wedge S^{0}
            \arrow[d,"\LUnitor^{\Sets_{*}}_{S^{0}}"]
            \\
            S^{0}\otimes_{\Sets_{*}}\Unit_{\Sets_{*}}
            \arrow[u,"\id_{S^{0}}\otimes_{\Sets_{*}}\id^{\otimes}_{\Sets_{*}|\Unit}"]
            \arrow[r,"\RUnitor'_{S^{0}}"']
            \&
            S^{0}
            % Subdiagrams
            \arrow[from=1-1,to=1-2,"(\secS)",phantom,yshift=-3.0*\the\DL]%
        \end{tikzcd}
    \]%
    commutes. Indeed, consider the diagram
    \begin{scalemath}
        \begin{tikzcd}[row sep={9.0*\the\DL,between origins}, column sep={9.0*\the\DL,between origins}, background color=backgroundColor, ampersand replacement=\&]
            S^{0}\otimes_{\Sets_{*}}S^{0}
            \&
            \&
            \&
            \&
            S^{0}\wedge S^{0}
            \\
            \&
            \&
            \&
            \Unit_{\Sets_{*}}\wedge S^{0}
            \&
            \\
            \&
            \Unit_{\Sets_{*}}\otimes_{\Sets_{*}}\Unit_{\Sets_{*}}
            \&
            \&
            \Unit_{\Sets_{*}}\otimes_{\Sets_{*}}S^{0}
            \&
            \\
            \&
            \&
            \Unit_{\Sets_{*}}
            \&
            \&
            \\
            S^{0}\otimes_{\Sets_{*}}\Unit_{\Sets_{*}}
            \&
            \&
            \&
            \&
            S^{0}
            % Arrows
            \arrow[from=1-1,to=1-5,"\id^{\otimes}_{\Sets_{*}|S^{0},S^{0}}"]%
            \arrow[from=1-5,to=5-5,"\LUnitor^{\Sets_{*}}_{S^{0}}"]%
            %
            \arrow[from=1-1,to=5-1,"\id_{S^{0}}\otimes_{\Sets_{*}}\id^{\otimes}_{\Unit|\Sets_{*}}"']%
            \arrow[from=5-1,to=5-5,"\RUnitor'_{S^{0}}"']%
            %
            \arrow[from=3-2,to=1-1,"\id^{\otimes}_{\Unit|\Sets_{*}}\otimes_{\Sets_{*}}\id^{\otimes}_{\Unit|\Sets_{*}}"description]%
            \arrow[from=5-1,to=3-2,"\id^{\otimes,-1}_{\Unit|\Sets_{*}}\otimes_{\Sets_{*}}\id_{S^{0}}"description]%
            %
            \arrow[from=3-2,to=3-4,"\id_{\Unit_{\Sets_{*}}}\otimes_{\Sets_{*}}\id^{\otimes}_{\Unit|\Sets_{*}}"description]%
            %
            \arrow[from=3-4,to=2-4,"\id^{\otimes}_{\Sets_{*}|\Unit_{\Sets_{*}},S^{0}}"description]%
            \arrow[from=2-4,to=1-5,"\id^{\otimes}_{\Unit|\Sets_{*}}\wedge\id_{S^{0}}"description]%
            %
            \arrow[from=1-1,to=3-4,"\id^{\otimes,-1}_{\Unit|\Sets_{*}}\otimes_{\Sets_{*}}\id_{S^{0}}"description]%
            %
            \arrow[from=3-4,to=5-5,"\LUnitor'_{S^{0}}"description]%
            \arrow[from=3-2,to=4-3,"\RUnitor'_{\Unit_{\Sets_{*}}}=\LUnitor'_{\Unit_{\Sets_{*}}}"description]%
            \arrow[from=4-3,to=5-5,"\id^{\otimes}_{\Unit|\Sets_{*}}"description]%
            % Subdiagrams
            \arrow[from=1-1,to=1-5,"(1)",phantom,xshift=2.0*\the\DL,yshift=-5.0*\the\DL]%
            \arrow[from=1-1,to=5-1,"(2)",phantom,xshift=3.25*\the\DL]%
            \arrow[from=3-2,to=3-4,"(3)",phantom,xshift=-6.0*\the\DL,yshift=4.5*\the\DL]%
            \arrow[from=1-5,to=5-5,"(4)"{yshift=3.5*\the\DL},phantom,xshift=-4.0*\the\DL]%
            \arrow[from=4-3,to=3-4,"(5)",phantom,yshift=-0.5*\the\DL]%
            \arrow[from=5-1,to=4-3,"(6)",phantom,xshift=2.0*\the\DL,yshift=1.5*\the\DL]%
        \end{tikzcd}
    \end{scalemath}
    whose boundary diagram corresponds to diagram $(\secS)$ above. Since:
    \begin{itemize}
        \item Subdiagram  $(1)$ commutes by the naturality of $\id^{\otimes}_{\Sets_{*}}$;
        \item Subdiagrams $(2)$ and $(3)$ commute by the functoriality of $\otimes$;
        \item Subdiagram  $(4)$ commutes by the left monoidal unity of $(\id^{\otimes},\id^{\otimes}_{\Unit})$, which we proved above;
        \item Subdiagram  $(5)$ commutes by the naturality of $\LUnitor'$;
        \item Subdiagram  $(6)$ commutes by the naturality of $\RUnitor'$, where the equality $\RUnitor'_{\Unit_{\Sets_{*}}}=\LUnitor'_{\Unit_{\Sets_{*}}}$ comes from \cref{TODO};
    \end{itemize}
    it follows that the boundary diagram, i.e.\ diagram $(\secS)$, also commutes. Next, consider the diagram
    \begin{scalemath}
        \begin{tikzcd}[row sep={11.0*\the\DL,between origins}, column sep={11.0*\the\DL,between origins}, background color=backgroundColor, ampersand replacement=\&]
            S^{0}\otimes_{\Sets_{*}}\Unit_{\Sets_{*}}
            \&
            \&
            \&
            S^{0}\wedge\Unit_{\Sets_{*}}
            \\
            \&
            S^{0}\otimes_{\Sets_{*}}S^{0}
            \&
            S^{0}\wedge S^{0}
            \&
            \\
            \&
            \&
            S^{0}
            \&
            \\
            S^{0}
            \&
            \&
            \&
            \Unit_{\Sets_{*}}
            \\
            \Unit_{\Sets_{*}}\otimes_{\Sets_{*}}S^{0}
            \&
            \&
            \&
            \Unit_{\Sets_{*}}\wedge S^{0}
            % Arrows
            % First Row
            \arrow[from=1-1,to=1-4,"\id^{\otimes}_{\Sets_{*}|S^{0},\Unit_{\Sets_{*}}}"]%
            % Second Row
            \arrow[from=2-2,to=2-3,"\id^{\otimes}_{\Sets_{*}|S^{0},S^{0}}"description]%
            % Fourth Row
            \arrow[from=4-1,to=4-4,"\id^{\otimes,-1}_{\Sets_{*}|\Unit}"description]%
            \arrow[from=5-1,to=5-4,"\id^{\otimes}_{\Sets_{*}|\Unit_{\Sets_{*}},S^{0}}"']%
            % First Column
            \arrow[from=1-1,to=5-1,"\sigma'_{S^{0},\Unit_{\Sets_{*}}}",ldddr_to_path]%
            \arrow[from=1-1,to=4-1,"\RUnitor'_{S^{0}}"description]%
            \arrow[from=4-1,to=5-1,"\LUnitor^{\prime,-1}_{S^{0}}"description]%
            % Third Column
            \arrow[from=2-3,to=3-3,"\LUnitor^{\Sets_{*}}_{S^{0}}"description]%
            % Fourth Column
            \arrow[from=1-4,to=4-4,"\LUnitor^{\Sets_{*}}_{\Unit_{\Sets_{*}}}"description]%
            \arrow[from=4-4,to=5-4,"\RUnitor^{\Sets_{*},-1}_{\Unit_{\Sets_{*}}}"description]%
            \arrow[from=1-4,to=5-4,"\sigma^{\Sets_{*}}_{S^{0},\Unit_{\Sets_{*}}}"description,rdddl_to_path]%
            % Diagonal Arrows
            \arrow[from=1-1,to=2-2,"\id_{S^{0}}\otimes_{\Sets_{*}}\id^{\otimes}_{\Sets_{*}|\Unit}"description]%
            \arrow[from=2-3,to=1-4,"\id_{S^{0}}\otimes_{\Sets_{*}}\id^{\otimes,-1}_{\Sets_{*}|\Unit}"description]%
            \arrow[from=4-1,to=3-3,Equals]%
            \arrow[from=3-3,to=4-4,"\id^{\otimes,-1}_{\Sets_{*}|\Unit}"description]%
            % Subdiagrams
            \arrow[from=1-1,to=5-1,"(1)",phantom,xshift=-3.25*\the\DL]%
            \arrow[from=2-2,to=2-3,"(2)",phantom,yshift=5.5*\the\DL]%
            \arrow[from=1-1,to=4-1,"(\secS)",phantom,xshift=9.0*\the\DL,yshift=-2.0*\the\DL]%
            \arrow[from=1-4,to=4-4,"(3)",phantom,xshift=-5.5*\the\DL]%
            \arrow[from=4-1,to=4-4,"(4)",phantom,yshift=4.5*\the\DL,pos=0.65]%
            \arrow[from=4-1,to=4-4,"(5)",phantom,yshift=-5.5*\the\DL]%
            \arrow[from=1-4,to=5-4,"(6)",phantom,xshift=2.875*\the\DL]%
        \end{tikzcd}
    \end{scalemath}
    whose boundary diagram corresponds to the diagram $(\ddagger)$ above. Since:
    \begin{itemize}
        \item Subdiagrams $(1)$ and $(6)$ commute by \cref{TODO};
        \item Subdiagram  $(2)$ commutes by the naturality of $\id^{\otimes}_{\Sets_{*}}$;
        \item Subdiagram  $(\secS)$ commutes, as was shown above;
        \item Subdiagram  $(3)$ commutes by the naturality of $\LUnitor^{\Sets_{*}}$;
        \item Subdiagram  $(4)$ commutes trivially;
        \item Subdiagram  $(5)$ commutes by \ChapterRef{\ChapterConstructionsWithMonoidalCategories, \cref{constructions-with-monoidal-categories:properties-of-the-moduli-category-of-monoidal-structures-on-a-category-extra-monoidal-unity-constraints-3} of \cref{constructions-with-monoidal-categories:properties-of-the-moduli-category-of-monoidal-structures-on-a-category-extra-monoidal-unity-constraints} of \cref{constructions-with-monoidal-categories:properties-of-the-moduli-category-of-monoidal-structures-on-a-category}}{\cref{properties-of-the-moduli-category-of-monoidal-structures-on-a-category-extra-monoidal-unity-constraints-3} of \cref{properties-of-the-moduli-category-of-monoidal-structures-on-a-category-extra-monoidal-unity-constraints} of \cref{properties-of-the-moduli-category-of-monoidal-structures-on-a-category}}, whose proof uses only the left monoidal unity of $(\id^{\otimes},\id^{\otimes}_{\Unit})$, which has been proven above;
    \end{itemize}
    it follows that the boundary diagram, i.e.\ diagram $(\ddagger)$, also commutes. Next, consider the diagram
    \begin{scalemath}
        \begin{tikzcd}[row sep={10.0*\the\DL,between origins}, column sep={10.0*\the\DL,between origins}, background color=backgroundColor, ampersand replacement=\&]
            S^{0}\otimes_{\Sets_{*}}S^{0}
            \&
            \&[3.0*\the\DL]
            \&
            S^{0}\wedge S^{0}
            \\
            \&
            S^{0}\otimes_{\Sets_{*}}\Unit_{\Sets_{*}}
            \&[3.0*\the\DL]
            S^{0}\wedge\Unit_{\Sets_{*}}
            \&
            \\
            \&
            \Unit_{\Sets_{*}}\otimes_{\Sets_{*}}S^{0}
            \&[3.0*\the\DL]
            \Unit_{\Sets_{*}}\wedge S^{0}
            \&
            \\
            S^{0}\otimes_{\Sets_{*}}S^{0}
            \&
            \&[3.0*\the\DL]
            \&
            S^{0}\wedge S^{0}\mrp{,}
            % Arrows
            % Outer Square
            \arrow[from=1-1,to=1-4,"\id^{\otimes}_{S^{0},S^{0}}"]%
            \arrow[from=1-4,to=4-4,"\sigma^{\Sets_{*}}_{S^{0},S^{0}}"]%
            %
            \arrow[from=1-1,to=4-1,"\sigma'_{S^{0},S^{0}}"']%
            \arrow[from=4-1,to=4-4,"\id^{\otimes}_{S^{0},S^{0}}"']%
            % Inner Square
            \arrow[from=2-2,to=2-3,"\id^{\otimes}_{S^{0},\Unit_{\Sets_{*}}}"description]%
            \arrow[from=2-3,to=3-3,"\sigma^{\Sets_{*}}_{S^{0},\Unit_{\Sets_{*}}}"description]%
            %
            \arrow[from=2-2,to=3-2,"\sigma'_{S^{0},\Unit_{\Sets_{*}}}"'description]%
            \arrow[from=3-2,to=3-3,"\id^{\otimes}_{\Unit_{\Sets_{*}},S^{0}}"'description]%
            % Connecting Arrows
            \arrow[from=1-1,to=2-2,"\id_{S^{0}}\otimes_{\Sets_{*}}\id^{\otimes,-1}_{\Unit}"description]%
            \arrow[from=1-4,to=2-3,"\id_{S^{0}}\wedge\id^{\otimes,-1}_{\Unit}"description]%
            \arrow[from=3-2,to=4-1,"\id^{\otimes}_{\Unit}\otimes_{\Sets_{*}}\id_{S^{0}}"description]%
            \arrow[from=3-3,to=4-4,"\id^{\otimes}_{\Unit}\wedge\id_{S^{0}}"description]%
            % Subdiagrams
            \arrow[from=2-2,to=3-3,"(1)",phantom,yshift=10.0*\the\DL]%
            \arrow[from=2-2,to=3-2,"(2)",phantom,xshift=-5.0*\the\DL]%
            \arrow[from=2-2,to=3-3,"(\ddagger)",phantom]%
            \arrow[from=2-3,to=3-3,"(3)",phantom,xshift=5.0*\the\DL]%
            \arrow[from=2-2,to=3-3,"(4)",phantom,yshift=-10.0*\the\DL]%
        \end{tikzcd}
    \end{scalemath}
    whose boundary diagram corresponds to the diagram $(\dagger)$. Since:
    \begin{itemize}
        \item Subdiagram $(1)$ commutes by the naturality of $\id^{\otimes}_{\Sets_{*}}$;
        \item Subdiagram $(2)$ commutes by the naturality of $\sigma'$ and the fact that $\id^{\otimes}_{\Unit}$ is invertible;
        \item Subdiagram $(\ddagger)$ commutes as proved above;
        \item Subdiagram $(3)$ commutes by the naturality of $\sigma^{\Sets_{*}}$ and the fact that $\id^{\otimes}_{\Unit}$ is invertible;
        \item Subdiagram $(4)$ commutes by the naturality of $\id^{\otimes}_{\Sets_{*}}$;
    \end{itemize}
    it follows that the boundary diagram, i.e.\ diagram $(\dagger)$ also commutes. Taking inverses for the diagram $(\dagger)$, we see that the diagram
    \[
        \begin{tikzcd}[row sep={5.0*\the\DL,between origins}, column sep={9.5*\the\DL,between origins}, background color=backgroundColor, ampersand replacement=\&]
            S^{0}\wedge S^{0}
            \arrow[r,"\id^{\otimes,-1}_{\Sets_{*}|S^{0},S^{0}}"]
            \arrow[d,"\sigma^{\Sets_{*},-1}_{S^{0},S^{0}}"']
            \&
            S^{0}\otimes_{\Sets_{*}}S^{0}
            \arrow[d,"\sigma^{\prime,-1}_{S^{0},S^{0}}"]
            \\
            S^{0}\wedge S^{0}
            \arrow[r,"\id^{\otimes,-1}_{\Sets_{*}|S^{0},S^{0}}"']
            \&
            S^{0}\otimes_{\Sets_{*}}S^{0}
            % Diagram Marking
            \arrow[from=1-1,to=2-2,"\scriptstyle(\parP)",phantom]%
        \end{tikzcd}
    \]%
    commutes as well. Now, let $X,Y\in\Obj(\Sets_{*})$, let $x\in X$, let $y\in Y$, and consider the diagram
    \[
        \begin{tikzcd}[row sep={6.5*\the\DL,between origins}, column sep={6.5*\the\DL,between origins}, background color=backgroundColor, ampersand replacement=\&]
            S^{0}\wedge S^{0}
            \&
            \&
            S^{0}\otimes_{\Sets_{*}}S^{0}
            \&
            \\
            \&
            Y\wedge X
            \&
            \&
            Y\otimes_{\Sets_{*}}X
            \\
            S^{0}\wedge S^{0}
            \&
            \&
            S^{0}\otimes_{\Sets_{*}}S^{0}
            \&
            \\
            \&
            X\wedge Y
            \&
            \&
            X\otimes_{\Sets_{*}}Y
            % 1-Arrows
            % First Square
            \arrow[from=1-1,to=3-1,"\sigma^{\Sets_{*},-1}_{S^{0},S^{0}}"']%
            \arrow[from=3-1,to=3-3,"\id^{\otimes,-1}_{S^{0},S^{0}}"{description,pos=0.25}]%
            \arrow[from=1-1,to=1-3,"\id^{\otimes,-1}_{S^{0},S^{0}}"]%
            \arrow[from=1-3,to=3-3,"\sigma^{\prime,-1}_{S^{0},S^{0}}"{description,pos=0.35}]%
            % Second Square
            \arrow[from=2-2,to=4-2,"\sigma^{\Sets_{*},-1}_{A,Y}"{description,pos=0.35},crossing over]%
            \arrow[from=4-2,to=4-4,"\id^{\otimes,-1}_{\Sets_{*}|A,Y}"']%
            \arrow[from=2-2,to=2-4,"\id^{\otimes,-1}_{\Sets_{*}|Y,A}"{description,pos=0.35},crossing over]%
            \arrow[from=2-4,to=4-4,"\sigma^{\prime,-1}_{A,Y}"]%
            % Connecting Arrows
            \arrow[from=1-1,to=2-2,"{[y]\wedge[x]}"description]%
            \arrow[from=1-3,to=2-4,"{[y]\otimes_{\Sets_{*}}[x]}"]%
            \arrow[from=3-1,to=4-2,"{[x]\wedge[y]}"']%
            \arrow[from=3-3,to=4-4,"{[x]\otimes_{\Sets_{*}}[y]}"description]%
        \end{tikzcd}
    \]%
    which we partition into subdiagrams as follows:
    \begin{scalemath}
        \begin{tikzcd}[row sep={5.0*\the\DL,between origins}, column sep={5.0*\the\DL,between origins}, background color=backgroundColor, ampersand replacement=\&]
            S^{0}\wedge S^{0}
            \&
            \&
            S^{0}\otimes_{\Sets_{*}}S^{0}
            \&
            \\
            \&
            Y\wedge X
            \&
            \&
            Y\otimes_{\Sets_{*}}X
            \\
            S^{0}\wedge S^{0}
            \&
            \&
            \&
            \\
            \&
            X\wedge Y
            \&
            \&
            X\otimes_{\Sets_{*}}Y
            % 1-Arrows
            % First Square
            \arrow[from=1-1,to=3-1,"\sigma^{\Sets_{*},-1}_{S^{0},S^{0}}"']%
            \arrow[from=1-1,to=1-3,"\id^{\otimes,-1}_{S^{0},S^{0}}"]%
            % Second Square
            \arrow[from=2-2,to=4-2,"\sigma^{\Sets_{*},-1}_{X,Y}"{description},crossing over]%
            \arrow[from=4-2,to=4-4,"\id^{\otimes,-1}_{\Sets_{*}|X,Y}"']%
            \arrow[from=2-2,to=2-4,"\id^{\otimes,-1}_{\Sets_{*}|Y,X}"{description},crossing over]%
            \arrow[from=2-4,to=4-4,"\sigma^{\prime,-1}_{X,Y}"]%
            % Connecting Arrows
            \arrow[from=1-1,to=2-2,"{[y]\wedge[x]}"description]%
            \arrow[from=1-3,to=2-4,"{[y]\otimes_{\Sets_{*}}[x]}"]%
            \arrow[from=3-1,to=4-2,"{[x]\wedge[y]}"']%
            % Subdiagrams
            \arrow[from=2-2,to=1-3,"\scriptstyle(1)"{rotate=-0.3,xslant=-0.903569337,yslant=0,xscale=7.0341,yscale=4.4454,xscale=0.225,yscale=0.225},phantom]%
            \arrow[from=3-1,to=2-2,"\scriptstyle(2)"{rotate=-44.6,xslant=-0.965688775,yslant=0,xscale=8.6931,yscale=8.2852,xscale=0.15,yscale=0.15},phantom]%
            \arrow[from=4-2,to=2-4,"\scriptstyle(3)"{rotate=0,xslant=0,yslant=0,xscale=1.5,yscale=1.5},phantom]%
        \end{tikzcd}
        \quad
        \begin{tikzcd}[row sep={5.0*\the\DL,between origins}, column sep={5.0*\the\DL,between origins}, background color=backgroundColor, ampersand replacement=\&]
            S^{0}\wedge S^{0}
            \&
            \&
            S^{0}\otimes_{\Sets_{*}}S^{0}
            \&
            \\
            \&
            \&
            \&
            Y\otimes_{\Sets_{*}}X
            \\
            S^{0}\wedge S^{0}
            \&
            \&
            S^{0}\otimes_{\Sets_{*}}S^{0}
            \&
            \\
            \&
            X\wedge Y
            \&
            \&
            X\otimes_{\Sets_{*}}Y
            % 1-Arrows
            % First Square
            \arrow[from=1-1,to=3-1,"\sigma^{\Sets_{*},-1}_{S^{0},S^{0}}"']%
            \arrow[from=3-1,to=3-3,"\id^{\otimes,-1}_{S^{0},S^{0}}"{description}]%
            \arrow[from=1-1,to=1-3,"\id^{\otimes,-1}_{S^{0},S^{0}}"]%
            \arrow[from=1-3,to=3-3,"\sigma^{\prime,-1}_{S^{0},S^{0}}"{description}]%
            % Second Square
            \arrow[from=4-2,to=4-4,"\id^{\otimes,-1}_{\Sets_{*}|X,Y}"']%
            \arrow[from=2-4,to=4-4,"\sigma^{\prime,-1}_{X,Y}"]%
            % Connecting Arrows
            \arrow[from=1-3,to=2-4,"{[y]\otimes_{\Sets_{*}}[x]}"]%
            \arrow[from=3-1,to=4-2,"{[x]\wedge[y]}"']%
            \arrow[from=3-3,to=4-4,"{[x]\otimes_{\Sets_{*}}[y]}"description]%
            % Subdiagrams
            \arrow[from=1-1,to=3-3,"\scriptstyle(\parP)"{rotate=0,xslant=0,yslant=0,xscale=1.5,yscale=1.5},phantom]%
            \arrow[from=3-3,to=2-4,"\scriptstyle(4)"{rotate=-44.6,xslant=-0.965688775,yslant=0,xscale=8.6931,yscale=8.2852,xscale=0.15,yscale=0.15},phantom]%
            \arrow[from=4-2,to=3-3,"\scriptstyle(5)"{rotate=-0.3,xslant=-0.903569337,yslant=0,xscale=7.0341,yscale=4.4454,xscale=0.225,yscale=0.225},phantom]%
        \end{tikzcd}
    \end{scalemath}
    Since:
    \begin{itemize}
        \item Subdiagram $(2)$ commutes by the naturality of $\sigma^{\Sets_{*},-1}$.
        \item Subdiagram $(5)$ commutes by the naturality of $\id^{\otimes,-1}$.
        \item Subdiagram $(\parP)$ commutes, as proved above.
        \item Subdiagram $(4)$ commutes by the naturality of $\sigma^{\prime,-1}$.
        \item Subdiagram $(1)$ commutes by the naturality of $\id^{\otimes,-1}$.
    \end{itemize}
    it follows that the diagram
    \[
        \begin{tikzcd}[row sep={5.0*\the\DL,between origins}, column sep={8.5*\the\DL,between origins}, background color=backgroundColor, ampersand replacement=\&]
            S^{0}\wedge S^{0}
            \arrow[rd,"{[y]\wedge[x]}"sloped]
            \&[-1.0*\the\DL]
            \&
            \\[-1.0*\the\DL]
            \&[-1.0*\the\DL]
            Y\wedge X
            \arrow[r,"\id^{\otimes}_{\Sets_{*}|Y,X}"]
            \arrow[d,"\sigma^{\Sets_{*}}_{X,Y}"']
            \&
            Y\otimes_{\Sets_{*}}X
            \arrow[d,"\sigma'_{X,Y}"]
            \\
            \&[-1.0*\the\DL]
            X\wedge Y
            \arrow[r,"\id^{\otimes}_{\Sets_{*}|X,Y}"']
            \&
            X\otimes_{\Sets_{*}}Y
        \end{tikzcd}
    \]%
    commutes. We then have
    \begin{align*}
        [\id^{\otimes,-1}_{\Sets_{*}|X,Y}\circ\sigma^{\Sets_{*},-1}_{X,Y}](y,x) &= [\id^{\otimes,-1}_{\Sets_{*}|X,Y}\circ\sigma^{\Sets_{*},-1}_{X,Y}\circ([y]\wedge[x])](1,1)\\
                                                                        &= [\sigma^{\prime,-1}_{X,Y}\circ\id^{\otimes,-1}_{\Sets_{*}|Y,X}\circ([y]\wedge[x])](1,1)\\
                                                                        &= [\sigma^{\prime,-1}_{X,Y}\circ\id^{\otimes,-1}_{\Sets_{*}|Y,X}](y,x)
    \end{align*}
    for each $(y,x)\in Y\wedge X$, and thus we have
    \[
        \id^{\otimes,-1}_{\Sets_{*}|X,Y}\circ\sigma^{\Sets_{*},-1}_{X,Y}%
        =%
        \sigma^{\prime,-1}_{X,Y}\circ\id^{\otimes,-1}_{\Sets_{*}|Y,X}.%
    \]%
    Taking inverses then gives
    \[
        \sigma^{\Sets_{*}}_{X,Y}\circ\id^{\otimes}_{\Sets_{*}|X,Y}%
        =%
        \id^{\otimes}_{\Sets_{*}|Y,X}\circ\sigma^{\prime}_{X,Y},%
    \]%
    showing that the diagram
    \[
        \begin{tikzcd}[row sep={5.0*\the\DL,between origins}, column sep={8.5*\the\DL,between origins}, background color=backgroundColor, ampersand replacement=\&]
            A\otimes_{\Sets_{*}}B
            \arrow[r,"\id^{\otimes}_{\Sets_{*}|A,B}"]
            \arrow[d,"\sigma'_{A,B}"']
            \&
            A\wedge B
            \arrow[d,"\sigma^{\Sets_{*}}_{A,B}"]
            \\
            B\otimes_{\Sets_{*}}A
            \arrow[r,"\id^{\otimes}_{\Sets_{*}|B,A}"']
            \&
            B\wedge A
        \end{tikzcd}
    \]%
    indeed commutes.

    \ProofBox{Monoidal Right Unity of the Isomorphism $\mathord{\otimes_{\Sets_{*}}}\cong\mathord{\wedge}$}%
    We have to show that the diagram
    \[
        \begin{tikzcd}[row sep={0.0*\the\DL,between origins}, column sep={0.0*\the\DL,between origins}, background color=backgroundColor, ampersand replacement=\&]
            \&[0.5\ThreeCmPlusAQuarter]
            X\otimes_{\Sets_{*}}S^{0}
            \&[0.9\ThreeCmPlusAQuarter]
            X\wedge S^{0}
            \&[0.5\ThreeCmPlusAQuarter]
            \\[0.5\ThreeCmPlusAQuarter]
            X\otimes_{\Sets_{*}}\Unit_{\Sets_{*}}
            \&[0.5\ThreeCmPlusAQuarter]
            \&[0.9\ThreeCmPlusAQuarter]
            \&[0.5\ThreeCmPlusAQuarter]
            X
            % 1-Arrows
            \arrow[from=2-1,to=1-2,"\id_{X}\otimes_{\Sets_{*}}\id^{\otimes}_{\Unit|\Sets_{*}}",pos=0.325]%
            \arrow[from=1-2,to=1-3,"\id^{\otimes}_{\Sets_{*}|X,S^{0}}"]%
            \arrow[from=1-3,to=2-4,"\RUnitor^{\Sets_{*}}_{X}",pos=0.525]%
            %
            \arrow[from=2-1,to=2-4,"\RUnitor'_{X}"',pos=0.445]%
        \end{tikzcd}
    \]%
    commutes. To this end, we will first show that the diagram
    \[
        \begin{tikzcd}[row sep={0.0*\the\DL,between origins}, column sep={0.0*\the\DL,between origins}, background color=backgroundColor, ampersand replacement=\&]
            \&[0.5\ThreeCmPlusAQuarter]
            S^{0}\otimes_{\Sets_{*}}S^{0}
            \&[0.45\ThreeCmPlusAQuarter]
            {}
            \&[0.45\ThreeCmPlusAQuarter]
            S^{0}\wedge S^{0}
            \&[0.5\ThreeCmPlusAQuarter]
            \\[0.5\ThreeCmPlusAQuarter]
            S^{0}\otimes_{\Sets_{*}}\Unit_{\Sets_{*}}
            \&[0.5\ThreeCmPlusAQuarter]
            \&[0.45\ThreeCmPlusAQuarter]
            {}
            \&[0.45\ThreeCmPlusAQuarter]
            \&[0.5\ThreeCmPlusAQuarter]
            S^{0}\mrp{,}
            % 1-Arrows
            \arrow[from=2-1,to=1-2,"\id^{\otimes}_{\Unit|\Sets_{*}}\otimes_{\Sets_{*}}\id_{S^{0}}",pos=0.425]%
            \arrow[from=1-2,to=1-4,"\id^{\otimes}_{\Sets_{*}|S^{0},S^{0}}"]%
            \arrow[from=1-4,to=2-5,"\RUnitor^{\Sets_{*}}_{S^{0}}",pos=0.575]%
            %
            \arrow[from=2-1,to=2-5,"\RUnitor'_{S^{0}}"',pos=0.425]%
            % Subdiagram Mark
            \arrow[from=1-3,to=2-3,phantom,"\scriptstyle(\dagger)",pos=0.425]%
        \end{tikzcd}
    \]%
    corresponding to the case $X=S^{0}$, commutes. First, notice that we may write
    \[
        \sigma'_{S^{0},\Unit_{\Sets_{*}}}%
        \colon%
        S^{0}\otimes_{\Sets_{*}}\Unit_{\Sets_{*}}%
        \to%
        \Unit_{\Sets_{*}}\otimes_{\Sets_{*}}S^{0}%
    \]%
    as the composition
    \[
        \begin{tikzcd}[row sep={2.5*\the\DL,between origins}, column sep={12.0*\the\DL,between origins}, background color=backgroundColor, ampersand replacement=\&]
            S^{0}\otimes_{\Sets_{*}}\Unit_{\Sets_{*}}
            \arrow[r,"\id^{\otimes}_{S^{0},\Unit_{\Sets_{*}}}"]
            \&
            \mrp{S^{0}\wedge\Unit_{\Sets_{*}}}\phantom{\Unit_{\Sets_{*}}\otimes_{\Sets_{*}}S^{0}}
            \\
            \phantom{S^{0}\otimes_{\Sets_{*}}\Unit_{\Sets_{*}}}
            \arrow[r,"\LUnitor^{\Sets_{*}}_{\Unit_{\Sets_{*}}}"]
            \&
            \mrp{\Unit_{\Sets_{*}}}\phantom{\Unit_{\Sets_{*}}\otimes_{\Sets_{*}}S^{0}}
            \\
            \phantom{S^{0}\otimes_{\Sets_{*}}\Unit_{\Sets_{*}}}
            \arrow[r,"\RUnitor^{\Sets_{*},-1}_{\Unit_{\Sets_{*}}}"]
            \&
            \mrp{\Unit_{\Sets_{*}}\wedge S^{0}}\phantom{\Unit_{\Sets_{*}}\otimes_{\Sets_{*}}S^{0}}
            \\
            \phantom{S^{0}\otimes_{\Sets_{*}}\Unit_{\Sets_{*}}}
            \arrow[r,"\id^{\otimes,-1}_{\Unit_{\Sets_{*}},S^{0}}"]
            \&
            \Unit_{\Sets_{*}}\otimes_{\Sets_{*}}S^{0}\mrp{.}
        \end{tikzcd}
    \]%
    Indeed, we may write this composition as part of the diagram
    \[
        \begin{tikzcd}[row sep={8.0*\the\DL,between origins}, column sep={11.0*\the\DL,between origins}, background color=backgroundColor, ampersand replacement=\&]
            S^{0}\otimes_{\Sets_{*}}\Unit_{\Sets_{*}}
            \arrow[r,"\id^{\otimes}_{S^{0},\Unit_{\Sets_{*}}}"]
            \arrow[d,"\sigma'_{S^{0},\Unit_{\Sets_{*}}}"']
            \&
            S^{0}\wedge\Unit_{\Sets_{*}}
            \arrow[r,"\LUnitor^{\Sets_{*}}_{\Unit_{\Sets_{*}}}"]
            \arrow[d,"\sigma^{\Sets_{*}}_{S^{0},\Unit_{\Sets_{*}}}"description]
            \&
            \Unit_{\Sets_{*}}
            \arrow[ld,"\RUnitor^{\Sets_{*},-1}_{\Unit_{\Sets_{*}}}"description]
            \\
            \Unit_{\Sets_{*}}\otimes_{\Sets_{*}}S^{0}
            \arrow[r,"\id^{\otimes}_{\Unit_{\Sets_{*}},S^{0}}"']
            \&
            \Unit_{\Sets_{*}}\wedge S^{0}
            \arrow[r,"\id^{\otimes,-1}_{\Unit_{\Sets_{*}},S^{0}}"']
            \&
            \Unit_{\Sets_{*}}\otimes_{\Sets_{*}}S^{0}\mrp{,}
            % Subdiagrams
            \arrow[from=1-1,to=2-2,"\scriptstyle(1)",phantom]%
            \arrow[from=1-2,to=2-3,"\scriptstyle(2)",phantom,pos=0.225]%
        \end{tikzcd}
    \]%
    which commutes since:
    \begin{itemize}
        \item Subdiagram $(1)$ commutes by the braidedness of $\id^{\otimes}$, as proved above.
        \item Subdiagram $(2)$ commutes by \cref{TODO}.
    \end{itemize}
    Next, consider the diagram
    \begin{scalemath}
        \begin{tikzcd}[row sep={15.0*\the\DL,between origins}, column sep={15.0*\the\DL,between origins}, background color=backgroundColor, ampersand replacement=\&]
            \textcolor{OIvermillion}{S^{0}\otimes_{\Sets_{*}}\Unit_{\Sets_{*}}}
            \&[1.0*\the\DL]
            S^{0}\otimes_{\Sets_{*}}(\Unit_{\Sets_{*}}\wedge S^{0})
            \&[4.5*\the\DL]
            S^{0}\otimes_{\Sets_{*}}(\Unit_{\Sets_{*}}\otimes_{\Sets_{*}}S^{0})
            \&
            S^{0}\otimes_{\Sets_{*}}S^{0}
            \\
            \textcolor{OIvermillion}{S^{0}\wedge\Unit_{\Sets_{*}}}
            \&[1.0*\the\DL]
            S^{0}\wedge(\Unit_{\Sets_{*}}\wedge S^{0})
            \&[4.5*\the\DL]
            S^{0}\wedge(\Unit_{\Sets_{*}}\otimes_{\Sets_{*}}S^{0})
            \&
            S^{0}\wedge S^{0}
            \\
            \textcolor{OIvermillion}{\Unit_{\Sets_{*}}}
            \&[1.0*\the\DL]
            \textcolor{OIvermillion}{\Unit_{\Sets_{*}}\wedge S^{0}}
            \&[4.5*\the\DL]
            \textcolor{OIvermillion}{\Unit_{\Sets_{*}}\otimes_{\Sets_{*}}S^{0}}
            \&
            S^{0}\mrp{,}
            % Arrows
            % First Row
            \arrow[from=1-1,to=1-2,"\id_{S^{0}}\otimes_{\Sets_{*}}\RUnitor^{\Sets_{*},-1}_{S^{0}}"]%
            \arrow[from=1-2,to=1-3,"\scalebox{0.9}{$\id_{S^{0}}\otimes_{\Sets_{*}}\id^{\otimes,-1}_{\Sets_{*}|\Unit_{\Sets_{*},S^{0}}}$}"]%
            \arrow[from=1-3,to=1-4,"\id_{S^{0}}\otimes_{\Sets_{*}}\LUnitor'_{S^{0}}"]%
            % Second Row
            \arrow[from=2-1,to=2-2,"\id_{S^{0}}\wedge\RUnitor^{\Sets_{*},-1}_{\Unit_{\Sets_{*}}}"description]%
            \arrow[from=2-2,to=2-3,"\id_{S^{0}}\wedge\id^{\otimes,-1}_{\Sets_{*}|\Unit_{\Sets_{*},S^{0}}}"description]%
            \arrow[from=2-3,to=2-4,"\id_{S^{0}}\wedge\LUnitor'_{S^{0}}"description]%
            % Third Row
            \arrow[from=3-1,to=3-2,"\RUnitor^{\Sets_{*},-1}_{\Unit_{\Sets_{*}}}"description,OIvermillion]%
            \arrow[from=3-2,to=3-3,"\id^{\otimes,-1}_{\Sets_{*}|\Unit_{\Sets_{*},S^{0}}}"description,OIvermillion]%
            \arrow[from=3-3,to=3-4,"\LUnitor'_{S^{0}}"description]%
            % First Column
            \arrow[from=1-1,to=2-1,"\id^{\otimes}_{\Sets_{*}|S^{0},\Unit_{\Sets_{*}}}"description,OIvermillion]%
            \arrow[from=2-1,to=3-1,"\LUnitor^{\Sets_{*}}_{\Unit_{\Sets_{*}}}"description,OIvermillion]%
            % Second Column
            \arrow[from=1-2,to=2-2,"\id^{\otimes}_{\Sets_{*}|S^{0},\Unit_{\Sets_{*}}\wedge S^{0}}"description]%
            \arrow[from=2-2,to=3-2,"\LUnitor^{\Sets_{*}}_{\Unit_{\Sets_{*}}\wedge S^{0}}"description]%
            % Third Column
            \arrow[from=1-3,to=2-3,"\id^{\otimes}_{\Sets_{*}|S^{0},\Unit_{\Sets_{*}}\otimes_{\Sets_{*}}S^{0}}"description]%
            \arrow[from=2-3,to=3-3,"\LUnitor^{\Sets_{*}}_{\Unit_{\Sets_{*}}\otimes_{\Sets_{*}}S^{0}}"description]%
            % Fourth Column
            \arrow[from=1-4,to=2-4,"\id^{\otimes}_{\Sets_{*}|S^{0},S^{0}}"]%
            \arrow[from=2-4,to=3-4,"\LUnitor^{\Sets_{*}}_{S^{0}}=\RUnitor^{\Sets_{*}}_{S^{0}}"]%
            % Subdiagrams
            \arrow[from=1-1,to=2-2,"(1)",phantom]%
            \arrow[from=1-2,to=2-3,"(2)",phantom]%
            \arrow[from=1-3,to=2-4,"(3)",phantom]%
            \arrow[from=2-1,to=3-2,"(4)",phantom]%
            \arrow[from=2-2,to=3-3,"(5)",phantom]%
            \arrow[from=2-3,to=3-4,"(6)",phantom]%
        \end{tikzcd}
    \end{scalemath}
    whose boundary diagram corresponds to the diagram $(\dagger)$ above, since the composition in \textcolor{OIvermillion}{red} is equal to $\sigma'_{S^{0},\Unit_{\Sets_{*}}}$ as proved above, and then the composition in \textcolor{OIvermillion}{red} composed with $\LUnitor'_{S^{0}}$ is equal to $\RUnitor'_{S^{0}}$ by \cref{TODO}. In this diagram:
    \begin{itemize}
        \item Subdiagrams $(1)$, $(2)$, and $(3)$ commute by the naturality of $\id^{\otimes}_{\Sets_{*}}$.
        \item Subdiagrams $(4)$, $(5)$, and $(6)$ commute by the naturality of $\LUnitor^{\Sets_{*}}$, where the equality $\LUnitor^{\Sets_{*}}_{S^{0}}=\RUnitor^{\Sets_{*}}_{S^{0}}$ comes from \cref{TODO}.
    \end{itemize}
    Since all subdiagrams commute, so does the boundary diagram, i.e.\ the diagram $(\dagger)$ above. As a result, the diagram
    \[
        \begin{tikzcd}[row sep={0.0*\the\DL,between origins}, column sep={0.0*\the\DL,between origins}, background color=backgroundColor, ampersand replacement=\&]
            \&[0.5\ThreeCmPlusAQuarter]
            S^{0}\wedge S^{0}
            \&[0.45\ThreeCmPlusAQuarter]
            {}
            \&[0.45\ThreeCmPlusAQuarter]
            S^{0}\otimes_{\Sets_{*}}S^{0}
            \&[0.5\ThreeCmPlusAQuarter]
            \\[0.5\ThreeCmPlusAQuarter]
            S^{0}
            \&[0.5\ThreeCmPlusAQuarter]
            \&[0.45\ThreeCmPlusAQuarter]
            {}
            \&[0.45\ThreeCmPlusAQuarter]
            \&[0.5\ThreeCmPlusAQuarter]
            S^{0}\otimes_{\Sets_{*}}\Unit_{\Sets_{*}}\mrp{.}
            % 1-Arrows
            \arrow[from=2-1,to=1-2,"\RUnitor^{\Sets_{*},-1}_{S^{0}}",pos=0.45]%
            \arrow[from=1-2,to=1-4,"\id^{\otimes,-1}_{\Sets_{*}|S^{0},S^{0}}"]%
            \arrow[from=1-4,to=2-5,"\id_{S^{0}}\otimes_{\Sets_{*}}\id^{\otimes,-1}_{\Unit|\Sets_{*}}",pos=0.55]%
            %
            \arrow[from=2-1,to=2-5,"\RUnitor^{\prime,-1}_{S^{0}}"',pos=0.5675]%
            % Subdiagram Mark
            \arrow[from=1-3,to=2-3,phantom,"\scriptstyle(\ddagger)",pos=0.425]%
        \end{tikzcd}
    \]%
    also commutes. Now, let $X\in\Obj(\Sets_{*})$, let $x\in X$, and consider the diagram
    \[
        \begin{tikzcd}[row sep={0.0*\the\DL,between origins}, column sep={0.0*\the\DL,between origins}, background color=backgroundColor, ampersand replacement=\&]
            \&[0.85\ThreeCm]
            S^{0}\wedge S^{0}
            \&[0.6\ThreeCm]
            {}
            \&[0.6\ThreeCm]
            S^{0}\otimes_{\Sets_{*}}S^{0}
            \&[0.85\ThreeCm]
            \\[0.5\ThreeCm]
            S^{0}
            \&[0.85\ThreeCm]
            \&[0.6\ThreeCm]
            {}
            \&[0.6\ThreeCm]
            \&[0.85\ThreeCm]
            S^{0}\otimes_{\Sets_{*}}\Unit_{\Sets_{*}}
            \\[1.0*\ThreeCm]
            \&[0.85\ThreeCm]
            X\wedge S^{0}
            \&[0.6\ThreeCm]
            {}
            \&[0.6\ThreeCm]
            X\otimes_{\Sets_{*}}S^{0}
            \&[0.85\ThreeCm]
            \\[0.5\ThreeCm]
            X
            \&[0.85\ThreeCm]
            \&[0.6\ThreeCm]
            {}
            \&[0.6\ThreeCm]
            \&[0.85\ThreeCm]
            X\otimes_{\Sets_{*}}\Unit_{\Sets_{*}}\mrp{.}
            % 1-Arrows
            % First Diagram
            \arrow[from=2-1,to=1-2,"\RUnitor^{\Sets_{*},-1}_{S^{0}}"{sloped}]%
            \arrow[from=1-2,to=1-4,"\id^{\otimes,-1}_{\Sets_{*}|S^{0},S^{0}}",pos=0.525]%
            \arrow[from=1-4,to=2-5,"\id_{S^{0}}\wedge\id^{\otimes,-1}_{\Unit|\Sets_{*}}"{sloped}]%
            % Second Diagram
            \arrow[from=4-1,to=3-2,"\RUnitor^{\Sets_{*},-1}_{X}"{description,sloped}]%
            \arrow[from=3-2,to=3-4,"\id^{\otimes,-1}_{\Sets_{*}|X,S^{0}}"description,pos=0.55]%
            \arrow[from=3-4,to=4-5,"\id_{X}\wedge\id^{\otimes,-1}_{\Unit|\Sets_{*}}"{description,sloped}]%
            \arrow[from=4-1,to=4-5,"\RUnitor^{\prime,-1}_{X}"description,pos=0.55]%
            % Connecting Arrows
            \arrow[from=2-1,to=4-1,"{[x]}"']%
            \arrow[from=1-2,to=3-2,"{\id_{S^{0}}\wedge[x]}"description]%
            \arrow[from=1-4,to=3-4,"{\id_{S^{0}}\otimes_{\Sets_{*}}[x]}"description]%
            \arrow[from=2-5,to=4-5,"{\id_{\Unit_{\Sets_{*}}}\wedge[x]}"]%
            % Crossing Over
            \arrow[from=2-1,to=2-5,"\RUnitor^{\prime,-1}_{S^{0}}"'description,pos=0.55,crossing over]%
            % Subdiagrams
            \arrow[from=1-3,to=2-3,"\scriptstyle(\ddagger)"{xscale=2.0,yscale=0.7,pos=0.4},phantom]%
            \arrow[from=3-2,to=1-4,"\scriptstyle(1)",phantom]%
            \arrow[from=3-3,to=4-3,"\scriptstyle(2)"{xscale=2.0,yscale=0.7,pos=0.45},phantom]%
            \arrow[from=2-1,to=3-2,"(3)"{rotate=34.5, xslant=0.656, yslant=0},phantom]%
            \arrow[from=3-4,to=2-5,"(4)"{rotate=-34.5, xslant=-0.656, yslant=0},phantom]%
            \arrow[from=2-3,to=4-3,"(5)"{xscale=1.3,yscale=1.1},phantom]%
        \end{tikzcd}
    \]%
    Since:
    \begin{itemize}
        \item Subdiagram $(5)$ commutes by the naturality of $\RUnitor^{\prime,-1}$.
        \item Subdiagram $(\dagger)$ commutes, as proved above.
        \item Subdiagram $(4)$ commutes by the naturality of $\id^{\otimes,-1}_{\Unit|\Sets_{*}}$.
        \item Subdiagram $(1)$ commutes by the naturality of $\id^{\otimes,-1}_{\Sets_{*}}$.
        \item Subdiagram $(3)$ commutes by the naturality of $\RUnitor^{\Sets_{*},-1}$.
    \end{itemize}
    it follows that the diagram
    \[
        \begin{tikzcd}[row sep={0.0*\the\DL,between origins}, column sep={0.0*\the\DL,between origins}, background color=backgroundColor, ampersand replacement=\&]
            \&[0.5\ThreeCmPlusAQuarter]
            \&[0.5\ThreeCmPlusAQuarter]
            X\wedge S^{0}
            \&[0.45\ThreeCmPlusAQuarter]
            {}
            \&[0.45\ThreeCmPlusAQuarter]
            X\otimes_{\Sets_{*}}S^{0}
            \&[0.5\ThreeCmPlusAQuarter]
            \\[0.5\ThreeCmPlusAQuarter]
            S^{0}%
            \&[0.5\ThreeCmPlusAQuarter]
            X
            \&[0.5\ThreeCmPlusAQuarter]
            \&[0.45\ThreeCmPlusAQuarter]
            {}
            \&[0.45\ThreeCmPlusAQuarter]
            \&[0.5\ThreeCmPlusAQuarter]
            X\otimes_{\Sets_{*}}\Unit_{\Sets_{*}}
            % 1-Arrows
            \arrow[from=2-1,to=2-2,"{[x]}"]%
            \arrow[from=2-2,to=1-3,"\RUnitor^{\Sets_{*},-1}_{X}",pos=0.45]%
            \arrow[from=1-3,to=1-5,"\id^{\otimes,-1}_{\Sets_{*}|X,S^{0}}"]%
            \arrow[from=1-5,to=2-6,"\id_{X}\otimes_{\Sets_{*}}\id^{\otimes,-1}_{\Unit|\Sets_{*}}",pos=0.55]%
            \arrow[from=2-2,to=2-6,"\RUnitor^{\prime,-1}_{X}"',pos=0.5675]%
        \end{tikzcd}
    \]%
    \begin{envwebgif}
        Here's a step-by-step showcase of this argument: \webgif{monoidal-right-unity-of-id-otimes-sets-star.gif}
    \end{envwebgif}
    We then have
    \begin{align*}
        \RUnitor^{\prime,-1}_{X}(a) &= [\RUnitor^{\prime,-1}_{X}\circ[x]](1)\\
                                    &= [(\id_{X}\wedge\id^{\otimes,-1}_{\Unit|\Sets_{*}})\circ\id^{\otimes,-1}_{\Sets_{*}|S^{0},X}\circ\RUnitor^{\Sets_{*},-1}_{X}\circ[x]](1)\\
                                    &= [(\id_{X}\wedge\id^{\otimes,-1}_{\Unit|\Sets_{*}})\circ\id^{\otimes,-1}_{\Sets_{*}|S^{0},X}\circ\RUnitor^{\Sets_{*},-1}_{X}](a)
    \end{align*}
    for each $a\in X$, and thus we have
    \[
        \RUnitor^{\prime,-1}_{X}%
        =%
        (\id_{X}\wedge\id^{\otimes,-1}_{\Unit|\Sets_{*}})\circ\id^{\otimes,-1}_{\Sets_{*}|S^{0},X}\circ\RUnitor^{\Sets_{*},-1}_{X}.
    \]%
    Taking inverses then gives
    \[
        \RUnitor^{\prime}_{X}%
        =%
        \RUnitor^{\Sets_{*}}_{X}\circ\id^{\otimes}_{\Sets_{*}|S^{0},X}\circ(\id_{X}\wedge\id^{\otimes}_{\Unit|\Sets_{*}}),
    \]%
    showing that the diagram
    \[
        \begin{tikzcd}[row sep={0.0*\the\DL,between origins}, column sep={0.0*\the\DL,between origins}, background color=backgroundColor, ampersand replacement=\&]
            \&[0.5\ThreeCmPlusAQuarter]
            X\otimes_{\Sets_{*}}S^{0}
            \&[0.9\ThreeCmPlusAQuarter]
            X\wedge S^{0}
            \&[0.5\ThreeCmPlusAQuarter]
            \\[0.5\ThreeCmPlusAQuarter]
            X\otimes_{\Sets_{*}}\Unit_{\Sets_{*}}
            \&[0.5\ThreeCmPlusAQuarter]
            \&[0.9\ThreeCmPlusAQuarter]
            \&[0.5\ThreeCmPlusAQuarter]
            X
            % 1-Arrows
            \arrow[from=2-1,to=1-2,"\id_{X}\otimes_{\Sets_{*}}\id^{\otimes}_{\Unit|\Sets_{*}}",pos=0.325]%
            \arrow[from=1-2,to=1-3,"\id^{\otimes}_{\Sets_{*}|X,S^{0}}"]%
            \arrow[from=1-3,to=2-4,"\RUnitor^{\Sets_{*}}_{X}",pos=0.525]%
            %
            \arrow[from=2-1,to=2-4,"\RUnitor'_{X}"',pos=0.445]%
        \end{tikzcd}
    \]%
    indeed commutes.

    \ProofBox{Monoidality of the Isomorphism $\mathord{\otimes_{\Sets_{*}}}\cong\mathord{\wedge}$}%
    We have to show that the diagram
    \[
        \begin{tikzcd}[row sep={0.0*\the\DL,between origins}, column sep={0.0*\the\DL,between origins}, background color=backgroundColor, ampersand replacement=\&]
            \&[0.86602540378\TwoCmPlusHalf]
            (X\otimes_{\Sets_{*}}Y)\otimes_{\Sets_{*}}Z
            \arrow[ld,"\id^{\otimes}_{\Sets_{*}|X,Y}\otimes_{\Sets_{*}}\id_{Z}"',pos=0.65]
            \arrow[rd,"\alpha'_{X,Y,Z}",pos=0.65]
            \&[0.86602540378\TwoCmPlusHalf]
            \\[0.5\TwoCmPlusHalf]
            (X\wedge Y)\otimes_{\Sets_{*}}Z
            \arrow[d,"\id^{\otimes}_{\Sets_{*}|X\wedge Y,Z}"']
            \&[0.86602540378\TwoCmPlusHalf]
            \&[0.86602540378\TwoCmPlusHalf]
            X\otimes_{\Sets_{*}}(Y\otimes_{\Sets_{*}}Z)
            \arrow[d,"\id_{X}\otimes_{\Sets_{*}}\id^{\otimes}_{\Sets_{*}|Y,Z}"]
            \\[\TwoCmPlusHalf]
            (X\wedge Y)\wedge Z
            \arrow[rd,"\alpha^{\Sets_{*}}_{X,Y,Z}"',pos=0.45]
            \&[0.86602540378\TwoCmPlusHalf]
            \&[0.86602540378\TwoCmPlusHalf]
            X\otimes_{\Sets_{*}}(Y\wedge Z)
            \arrow[ld,"\id^{\otimes}_{\Sets_{*}|X,Y\wedge Z}",pos=0.45]
            \\[0.5\TwoCmPlusHalf]
            \&[0.86602540378\TwoCmPlusHalf]
            X\wedge(Y\wedge Z)
            \&[0.86602540378\TwoCmPlusHalf]
        \end{tikzcd}
    \]%
    commutes. To this end, we will first prove that the diagram
    \[
        \begin{tikzcd}[row sep={0.0*\the\DL,between origins}, column sep={0.0*\the\DL,between origins}, background color=backgroundColor, ampersand replacement=\&]
            \&[0.86602540378\TwoCmPlusHalf]
            (S^{0}\otimes_{\Sets_{*}}S^{0})\otimes_{\Sets_{*}}S^{0}
            \arrow[ld,"\id^{\otimes}_{\Sets_{*}|S^{0},S^{0}}\otimes_{\Sets_{*}}\id_{S^{0}}"',pos=0.65]
            \arrow[rd,"\alpha'_{S^{0},S^{0},S^{0}}",pos=0.65]
            \&[0.86602540378\TwoCmPlusHalf]
            \\[0.5\TwoCmPlusHalf]
            (S^{0}\wedge S^{0})\otimes_{\Sets_{*}}S^{0}
            \arrow[d,"\id^{\otimes}_{\Sets_{*}|S^{0}\wedge S^{0},S^{0}}"']
            \&[0.86602540378\TwoCmPlusHalf]
            \&[0.86602540378\TwoCmPlusHalf]
            S^{0}\otimes_{\Sets_{*}}(S^{0}\otimes_{\Sets_{*}}S^{0})
            \arrow[d,"\id_{S^{0}}\otimes_{\Sets_{*}}\id^{\otimes}_{\Sets_{*}|S^{0},S^{0}}"]
            \\[\TwoCmPlusHalf]
            (S^{0}\wedge S^{0})\wedge S^{0}
            \arrow[rd,"\alpha^{\Sets_{*}}_{S^{0},S^{0},S^{0}}"',pos=0.45]
            \&[0.86602540378\TwoCmPlusHalf]
            \&[0.86602540378\TwoCmPlusHalf]
            S^{0}\otimes_{\Sets_{*}}(S^{0}\wedge S^{0})
            \arrow[ld,"\id^{\otimes}_{\Sets_{*}|S^{0},S^{0}\wedge S^{0}}",pos=0.45]
            \\[0.5\TwoCmPlusHalf]
            \&[0.86602540378\TwoCmPlusHalf]
            S^{0}\wedge(S^{0}\wedge S^{0})
            \&[0.86602540378\TwoCmPlusHalf]
            % Subdiagrams
            \arrow[from=2-1,to=3-3,"\scriptstyle(\dagger)",phantom]%
        \end{tikzcd}
    \]%
    commutes, and, to that end, we will first show that the diagram
    \[
        \begin{tikzcd}[row sep={0.0*\the\DL,between origins}, column sep={0.0*\the\DL,between origins}, background color=backgroundColor, ampersand replacement=\&]
            \&[0.86602540378\TwoCmPlusHalf]
            (S^{0}\otimes_{\Sets_{*}}\Unit_{\Sets_{*}})\otimes_{\Sets_{*}}S^{0}
            \arrow[ld,"\id^{\otimes}_{\Sets_{*}|S^{0},\Unit_{\Sets_{*}}}\otimes_{\Sets_{*}}\id_{S^{0}}"',pos=0.65]
            \arrow[rd,"\alpha'_{S^{0},\Unit_{\Sets_{*}},S^{0}}",pos=0.65]
            \&[0.86602540378\TwoCmPlusHalf]
            \\[0.5\TwoCmPlusHalf]
            (S^{0}\wedge\Unit_{\Sets_{*}})\otimes_{\Sets_{*}}S^{0}
            \arrow[d,"\id^{\otimes}_{\Sets_{*}|S^{0}\wedge\Unit_{\Sets_{*}},S^{0}}"']
            \&[0.86602540378\TwoCmPlusHalf]
            \&[0.86602540378\TwoCmPlusHalf]
            S^{0}\otimes_{\Sets_{*}}(\Unit_{\Sets_{*}}\otimes_{\Sets_{*}}S^{0})
            \arrow[d,"\id_{S^{0}}\otimes_{\Sets_{*}}\id^{\otimes}_{\Sets_{*}|\Unit_{\Sets_{*}},S^{0}}"]
            \\[\TwoCmPlusHalf]
            (S^{0}\wedge\Unit_{\Sets_{*}})\wedge S^{0}
            \arrow[rd,"\alpha^{\Sets_{*}}_{S^{0},\Unit_{\Sets_{*}},S^{0}}"',pos=0.45]
            \&[0.86602540378\TwoCmPlusHalf]
            \&[0.86602540378\TwoCmPlusHalf]
            S^{0}\otimes_{\Sets_{*}}(\Unit_{\Sets_{*}}\wedge S^{0})
            \arrow[ld,"\id^{\otimes}_{\Sets_{*}|S^{0},\Unit_{\Sets_{*}}\wedge S^{0}}",pos=0.45]
            \\[0.5\TwoCmPlusHalf]
            \&[0.86602540378\TwoCmPlusHalf]
            S^{0}\wedge(\Unit_{\Sets_{*}}\wedge S^{0})
            \&[0.86602540378\TwoCmPlusHalf]
            % Subdiagrams
            \arrow[from=2-1,to=3-3,"\scriptstyle(\ddagger)",phantom]%
        \end{tikzcd}
    \]%
    commutes. Indeed, consider the diagram
    \begin{scalemath}
        \begin{tikzcd}[row sep={9.0*\the\DL,between origins}, column sep={9.0*\the\DL,between origins}, background color=backgroundColor, ampersand replacement=\&]
            \&
            \&
            (S^{0}\otimes_{\Sets_{*}}\Unit_{\Sets_{*}})\otimes_{\Sets_{*}}S^{0}
            \&
            \&
            \&
            \&
            \\
            \&
            \&
            \&
            \&
            \&
            \&
            \\
            (S^{0}\wedge\Unit_{\Sets_{*}})\otimes_{\Sets_{*}}S^{0}
            \&
            \&
            (S^{0}\otimes_{\Sets_{*}}S^{0})\otimes_{\Sets_{*}}S^{0}
            \&
            \&
            \&
            \&
            \\
            \&
            (S^{0}\wedge S^{0})\otimes_{\Sets_{*}}S^{0}
            \&
            \&
            \&
            S^{0}\otimes_{\Sets_{*}}S^{0}
            \&
            \&
            S^{0}\otimes_{\Sets_{*}}(\Unit_{\Sets_{*}}\otimes_{\Sets_{*}}S^{0})
            \\
            \&
            (S^{0}\wedge S^{0})\wedge S^{0}
            \&
            \&
            S^{0}\wedge S^{0}
            \&
            S^{0}\otimes_{\Sets_{*}}\Unit_{\Sets_{*}}
            \&
            \&
            S^{0}\otimes_{\Sets_{*}}(\Unit_{\Sets_{*}}\wedge S^{0})
            \\
            (S^{0}\wedge\Unit_{\Sets_{*}})\wedge S^{0}
            \&
            \&
            \&
            S^{0}\wedge\Unit_{\Sets_{*}}
            \&
            \&
            \&
            \\
            \&
            \&
            \&
            \&
            \&
            \&
            \\
            \&
            \&
            \&
            S^{0}\wedge(\Unit_{\Sets_{*}}\wedge S^{0})
            \&
            \&
            \&
            % Arrows
            \arrow[from=1-3,to=3-1,"\id^{\otimes}_{\Sets_{*}|S^{0},\Unit_{\Sets_{*}}}\otimes_{\Sets_{*}}\id_{S^{0}}"']%
            \arrow[from=3-1,to=6-1,"\id^{\otimes}_{\Sets_{*}|S^{0}\wedge\Unit_{\Sets_{*}},S^{0}}"']%
            \arrow[from=6-1,to=8-4,"\alpha^{\Sets_{*}}_{S^{0},\Unit_{\Sets_{*}},S^{0}}"']%
            %
            \arrow[from=1-3,to=4-7,"\alpha'_{S^{0},\Unit_{\Sets_{*}},S^{0}}"]%
            \arrow[from=4-7,to=5-7,"\id_{S^{0}}\otimes_{\Sets_{*}}\id^{\otimes}_{\Sets_{*}|\Unit_{\Sets_{*}},S^{0}}"]%
            \arrow[from=5-7,to=8-4,"\id^{\otimes}_{\Sets_{*}|S^{0},\Unit_{\Sets_{*}}\wedge S^{0}}"]%
            %
            \arrow[from=1-3,to=3-3,"(\id_{S^{0}}\otimes_{\Sets_{*}}\id^{\otimes}_{\Unit|\Sets_{*}})\otimes_{\Sets_{*}}\id_{S^{0}}"description]%
            \arrow[from=3-1,to=4-2,"(\id_{S^{0}}\wedge\id^{\otimes}_{\Unit|\Sets_{*}})\otimes_{\Sets_{*}}\id_{S^{0}}"description]%
            \arrow[from=3-3,to=4-2,"\id^{\otimes}_{\Sets_{*}|S^{0},S^{0}}\otimes_{\Sets_{*}}\id_{S^{0}}"description]%
            \arrow[from=4-2,to=4-5,"\RUnitor^{\Sets_{*}}_{S^{0}}\otimes_{\Sets_{*}}\id_{S^{0}}"description]%
            \arrow[from=4-2,to=5-2,"\id^{\otimes}_{\Sets_{*}|S^{0}\wedge S^{0},S^{0}}"description]%
            \arrow[from=5-2,to=6-1,"(\id_{S^{0}}\wedge\id^{\otimes}_{\Unit|\Sets_{*}})\wedge\id_{S^{0}}"description]%
            \arrow[from=5-2,to=5-4,"\RUnitor^{\Sets_{*}}_{S^{0}}\wedge\id_{S^{0}}"description,bend left=25]%
            \arrow[from=5-2,to=5-4,"\RUnitor^{\Sets_{*}}_{S^{0}\wedge S^{0}}"description,bend right=25]%
            %
            \arrow[from=4-5,to=5-4,"\id^{\otimes}_{\Sets_{*}|S^{0},S^{0}}"{description,pos=0.65}]%
            \arrow[from=5-4,to=6-4,"\id_{S^{0}}\wedge\id^{\otimes,-1}_{\Unit|\Sets_{*}}"{description,pos=0.525}]%
            %
            \arrow[from=4-5,to=5-5,"\id_{S^{0}}\otimes_{\Sets_{*}}\id^{\otimes,-1}_{\Unit|\Sets_{*}}"{description,pos=0.425}]%
            \arrow[from=5-5,to=6-4,"\id^{\otimes}_{\Sets_{*}|S^{0},\Unit_{\Sets_{*}}}"{description,pos=0.4}]%
            %
            \arrow[from=4-5,to=4-7,"\id_{S^{0}}\otimes_{\Sets_{*}}\LUnitor^{\prime,-1}_{S^{0}}"description]%
            \arrow[from=5-5,to=5-7,"\id_{S^{0}}\otimes_{\Sets_{*}}\RUnitor^{\Sets_{*},-1}_{S^{0}}"description]%
            %
            \arrow[from=6-1,to=6-4,"\RUnitor^{\Sets_{*}}_{S^{0}\wedge\Unit_{\Sets_{*}}}"description]%
            \arrow[from=6-4,to=8-4,"\id_{S^{0}}\wedge\RUnitor^{\Sets_{*}}_{S^{0}}"description]%
            \arrow[from=1-3,to=4-5,"\RUnitor'_{S^{0}}\otimes_{\Sets_{*}}\id_{S^{0}}"description]%
            % Subdiagrams
            \arrow[from=3-1,to=3-3,"(1)",phantom,xshift=2.75*\the\DL,yshift=2.75*\the\DL]%
            \arrow[from=3-3,to=4-7,"(2)",phantom,xshift=-12.25*\the\DL,yshift=2.0*\the\DL]%
            \arrow[from=3-3,to=4-7,"(3)",phantom,xshift=2.0*\the\DL,yshift=4.0*\the\DL]%
            \arrow[from=4-2,to=5-2,"(4)",phantom,xshift=-5.5*\the\DL]%
            \arrow[from=4-2,to=5-4,"(5)",phantom,xshift=2.0*\the\DL,yshift=1.0*\the\DL]%
            \arrow[from=5-2,to=5-4,"(6)",phantom,pos=0.45]%
            \arrow[from=5-2,to=6-4,"(7)",phantom,xshift=-3.0*\the\DL,yshift=-1.25*\the\DL]%
            \arrow[from=5-4,to=5-5,"(8)",phantom,pos=0.625]%
            \arrow[from=5-5,to=4-7,"(9)",phantom]%
            \arrow[from=6-4,to=8-4,"(10)",phantom,xshift=-8.0*\the\DL,yshift=4.0*\the\DL]%
            \arrow[from=6-4,to=5-5,"(11)",phantom,xshift=5.5*\the\DL,yshift=-4.5*\the\DL]%
        \end{tikzcd}
    \end{scalemath}
    whose boundary diagram corresponds to diagram $(\ddagger)$ above. Since:
    \begin{itemize}
        \item Subdiagrams $(1)$, $(4)$, $(5)$, $(8)$, and $(11)$ commute by the naturality of $\id^{\otimes}_{\Sets_{*}}$;
        \item Subdiagram  $(2)$ commutes by the right monoidal unity of $(\id^{\otimes}_{\Sets_{*}},\id^{\otimes}_{\Unit|\Sets_{*}})$;
        \item Subdiagram  $(3)$ commutes by the triangle identity for $(\alpha',\LUnitor',\RUnitor')$;
        \item Subdiagram  $(6)$ commutes by \cref{TODO};
        \item Subdiagram  $(7)$ commutes by the naturality of $\RUnitor^{\Sets_{*}}$;
        \item Subdiagram  $(9)$ commutes by \cref{TODO};
        \item Subdiagram  $(10)$ commutes by \cref{TODO};
    \end{itemize}
    it follows that the boundary diagram, i.e.\ diagram $(\ddagger)$, also commutes. Consider now the diagram
    \begin{scalemath}
        \begin{tikzcd}[row sep={0.0*\the\DL,between origins}, column sep={0.0*\the\DL,between origins}, background color=backgroundColor, ampersand replacement=\&]
            \&[1.73205081*\TwoCmPlusAQuarter]
            \&[1.73205081*\TwoCmPlusAQuarter]
            (S^{0}\otimes_{\Sets_{*}}S^{0})\otimes_{\Sets_{*}}S^{0}
            \&[1.73205081*\TwoCmPlusAQuarter]
            \&[1.73205081*\TwoCmPlusAQuarter]
            \\[2.0*\TwoCmPlusAQuarter]
            (S^{0}\wedge S^{0})\otimes_{\Sets_{*}}S^{0}
            \&[1.73205081*\TwoCmPlusAQuarter]
            \&[1.73205081*\TwoCmPlusAQuarter]
            (S^{0}\otimes_{\Sets_{*}}\Unit_{\Sets_{*}})\otimes_{\Sets_{*}}S^{0}
            \&[1.73205081*\TwoCmPlusAQuarter]
            \&[1.73205081*\TwoCmPlusAQuarter]
            S^{0}\otimes_{\Sets_{*}}(S^{0}\otimes_{\Sets_{*}}S^{0})
            \\[1.0*\TwoCmPlusAQuarter]
            \&[1.73205081*\TwoCmPlusAQuarter]
            (S^{0}\wedge\Unit_{\Sets_{*}})\otimes_{\Sets_{*}}S^{0}
            \&[1.73205081*\TwoCmPlusAQuarter]
            \&[1.73205081*\TwoCmPlusAQuarter]
            S^{0}\otimes_{\Sets_{*}}(\Unit_{\Sets_{*}}\otimes_{\Sets_{*}}S^{0})
            \&[1.73205081*\TwoCmPlusAQuarter]
            \\[2.0*\TwoCmPlusAQuarter]
            \&[1.73205081*\TwoCmPlusAQuarter]
            (S^{0}\wedge\Unit_{\Sets_{*}})\wedge S^{0}
            \&[1.73205081*\TwoCmPlusAQuarter]
            \&[1.73205081*\TwoCmPlusAQuarter]
            {S^{0}\otimes_{\Sets_{*}}(\Unit_{\Sets_{*}}\wedge S^{0})}
            \&[1.73205081*\TwoCmPlusAQuarter]
            \\[1.0*\TwoCmPlusAQuarter]
            (S^{0}\wedge S^{0})\wedge S^{0}
            \&[1.73205081*\TwoCmPlusAQuarter]
            \&[1.73205081*\TwoCmPlusAQuarter]
            S^{0}\wedge(\Unit_{\Sets_{*}}\wedge S^{0})
            \&[1.73205081*\TwoCmPlusAQuarter]
            \&[1.73205081*\TwoCmPlusAQuarter]
            S^{0}\otimes_{\Sets_{*}}(S^{0}\wedge S^{0})
            \\[2.0*\TwoCmPlusAQuarter]
            \&[1.73205081*\TwoCmPlusAQuarter]
            \&[1.73205081*\TwoCmPlusAQuarter]
            S^{0}\wedge(S^{0}\wedge S^{0})
            \&[1.73205081*\TwoCmPlusAQuarter]
            \&[1.73205081*\TwoCmPlusAQuarter]
            % Arrows
            \arrow[from=1-3,to=2-1,"\id^{\otimes}_{\Sets_{*}|S^{0},S^{0}}\otimes_{\Sets_{*}}\id_{S^{0}}"']%
            \arrow[from=2-1,to=5-1,"\id^{\otimes}_{\Sets_{*}|S^{0}\wedge S^{0},S^{0}}"']%
            \arrow[from=5-1,to=6-3,"\alpha^{\Sets_{*}}_{S^{0},S^{0},S^{0}}"']%
            %
            \arrow[from=1-3,to=2-5,"\alpha'_{S^{0},S^{0},S^{0}}"]%
            \arrow[from=2-5,to=5-5,"\id_{S^{0}}\otimes_{\Sets_{*}}\id^{\otimes}_{\Sets_{*}|S^{0},S^{0}}"]%
            \arrow[from=5-5,to=6-3,"\id^{\otimes}_{\Sets_{*}|S^{0},S^{0}\wedge S^{0}}"]%
            %
            \arrow[from=2-3,to=3-2,"\id^{\otimes}_{\Sets_{*}|S^{0},\Unit_{\Sets_{*}}}\otimes_{\Sets_{*}}\id_{S^{0}}"description]%
            \arrow[from=3-2,to=4-2,"\id^{\otimes}_{\Sets_{*}|S^{0}\wedge\Unit_{\Sets_{*}},S^{0}}"description]%
            \arrow[from=4-2,to=5-3,"\alpha^{\Sets_{*}}_{S^{0},\Unit_{\Sets_{*}},S^{0}}"description]%
            %
            \arrow[from=2-3,to=3-4,"\alpha'_{S^{0},\Unit_{\Sets_{*}},S^{0}}"description]%
            \arrow[from=3-4,to=4-4,"\id_{S^{0}}\otimes_{\Sets_{*}}\id^{\otimes}_{\Sets_{*}|\Unit_{\Sets_{*}},S^{0}}"description]%
            \arrow[from=4-4,to=5-3,"\id^{\otimes}_{\Sets_{*}|S^{0},\Unit_{\Sets_{*}}\wedge S^{0}}"description]%
            %
            \arrow[from=1-3,to=2-3,"{(\id_{S^{0}}\otimes_{\Sets_{*}}\id^{\otimes,-1}_{\Unit|\Sets_{*}})\otimes_{\Sets_{*}}\id_{S^{0}}}"description]%
            \arrow[from=2-1,to=3-2,"{(\id_{S^{0}}\wedge\id^{\otimes,-1}_{\Unit|\Sets_{*}})\otimes_{\Sets_{*}}\id_{S^{0}}}"description]%
            \arrow[from=2-5,to=3-4,"{\id_{S^{0}}\otimes_{\Sets_{*}}(\id^{\otimes,-1}_{\Unit|\Sets_{*}}\otimes_{\Sets_{*}}\id_{S^{0}})}"description]%
            \arrow[from=5-1,to=4-2,"{(\id_{S^{0}}\wedge\id^{\otimes,-1}_{\Unit|\Sets_{*}})\wedge\id_{S^{0}}}"description]%
            \arrow[from=5-5,to=4-4,"{\id_{S^{0}}\otimes_{\Sets_{*}}(\id^{\otimes,-1}_{\Unit|\Sets_{*}}\wedge\id_{S^{0}})}"description]%
            \arrow[from=5-3,to=6-3,"{\id_{S^{0}}\wedge(\id^{\otimes}_{\Unit|\Sets_{*}}\wedge\id_{S^{0}})}"description]%
            % Subdiagrams
            \arrow[from=2-3,to=5-3,"(\ddagger)",phantom]%
            \arrow[from=2-1,to=2-3,"(1)",phantom,xshift=0.4125*\TwoCmPlusAQuarter,yshift=0.25*\TwoCmPlusAQuarter]%
            \arrow[from=2-3,to=2-5,"(2)",phantom,xshift=-0.4125*\TwoCmPlusAQuarter,yshift=0.25*\TwoCmPlusAQuarter]%
            \arrow[from=3-2,to=4-2,"(3)",phantom,xshift=-0.5*1.73205081*\TwoCmPlusAQuarter]%
            \arrow[from=3-4,to=4-4,"(4)",phantom,xshift=0.5*1.73205081*\TwoCmPlusAQuarter]%
            \arrow[from=5-1,to=5-3,"(5)",phantom,xshift=0.4125*\TwoCmPlusAQuarter,yshift=-0.25*\TwoCmPlusAQuarter]%
            \arrow[from=5-3,to=5-5,"(6)",phantom,xshift=-0.4125*\TwoCmPlusAQuarter,yshift=-0.25*\TwoCmPlusAQuarter]%
        \end{tikzcd}
    \end{scalemath}
    whose boundary corresponds to diagram $(\dagger)$ above. Since:
    \begin{itemize}
        \item Subdiagrams $(1)$, $(3)$, $(4)$, and $(6)$ commute by the naturality of $\id^{\otimes}_{Sets_{*}}$;
        \item Subdiagram $(\ddagger)$ commutes, as proved above;
        \item Subdiagram $(2)$ commutes by the naturality of $\alpha'$;
        \item Subdiagram $(5)$ commutes by the naturality of $\alpha^{\Sets_{*}}$;
    \end{itemize}
    it follows that the boundary diagram, i.e.\ diagram $(\dagger)$, also commutes. Taking inverses on the diagram $(\dagger)$, we see that the diagram
    \[
        \begin{tikzcd}[row sep={0.0*\the\DL,between origins}, column sep={0.0*\the\DL,between origins}, background color=backgroundColor, ampersand replacement=\&]
            \&[0.86602540378\TwoCmPlusHalf]
            S^{0}\wedge(S^{0}\wedge S^{0})
            \arrow[ld,"\alpha^{\Sets_{*},-1}_{S^{0},S^{0},S^{0}}"',pos=0.5]
            \arrow[rd,"\id^{\otimes,-1}_{\Sets_{*}|S^{0},S^{0}\wedge S^{0}}",pos=0.5]
            \&[0.86602540378\TwoCmPlusHalf]
            \\[0.5\TwoCmPlusHalf]
            (S^{0}\wedge S^{0})\wedge S^{0}
            \arrow[d,"\id^{\otimes,-1}_{\Sets_{*}|S^{0}\wedge S^{0},S^{0}}"']
            \&[0.86602540378\TwoCmPlusHalf]
            \&[0.86602540378\TwoCmPlusHalf]
            S^{0}\otimes_{\Sets_{*}}(S^{0}\wedge S^{0})
            \arrow[d,"\id_{S^{0}}\otimes_{\Sets_{*}}\id^{\otimes,-1}_{\Sets_{*}|S^{0},S^{0}}"]
            \\[\TwoCmPlusHalf]
            (S^{0}\wedge S^{0})\otimes_{\Sets_{*}}S^{0}
            \arrow[rd,"\id^{\otimes,-1}_{\Sets_{*}|S^{0},S^{0}}\otimes_{\Sets_{*}}\id_{S^{0}}"',pos=0.35]
            \&[0.86602540378\TwoCmPlusHalf]
            \&[0.86602540378\TwoCmPlusHalf]
            S^{0}\otimes_{\Sets_{*}}(S^{0}\otimes_{\Sets_{*}}S^{0})
            \arrow[ld,"\alpha^{\prime,-1}_{S^{0},S^{0},S^{0}}",pos=0.35]
            \\[0.5\TwoCmPlusHalf]
            \&[0.86602540378\TwoCmPlusHalf]
            (S^{0}\otimes_{\Sets_{*}}S^{0})\otimes_{\Sets_{*}}S^{0}
            \&[0.86602540378\TwoCmPlusHalf]
            % Diagram Marking
            \arrow[from=1-2,to=4-2,"\scriptstyle(\dagger)",phantom]%
        \end{tikzcd}
    \]%
    commutes as well. Now, let $X,Y,Z\in\Obj(\Sets_{*})$, let $x\in X$, let $y\in Y$, let $z\in Z$, and consider the diagram
    \begin{scalemath}
        \begin{tikzcd}[row sep={0.0*\the\DL,between origins}, column sep={0.0*\the\DL,between origins}, background color=backgroundColor, ampersand replacement=\&]
            \&[0.86602540378\ThreeCmPlusHalf]
            S^{0}\wedge(S^{0}\wedge S^{0})
            \&[0.86602540378\ThreeCmPlusHalf]
            \&[0.785\ThreeCmPlusHalf]
            \&[0.86602540378\ThreeCmPlusHalf]
            \&[0.86602540378\ThreeCmPlusHalf]
            \\[0.45\ThreeCmPlusHalf]
            \&[0.86602540378\ThreeCmPlusHalf]
            \&[0.86602540378\ThreeCmPlusHalf]
            \&[0.785\ThreeCmPlusHalf]
            \&[0.86602540378\ThreeCmPlusHalf]
            X\wedge(Y\wedge Z)
            \&[0.86602540378\ThreeCmPlusHalf]
            \\[0.05\ThreeCmPlusHalf]
            (S^{0}\wedge S^{0})\wedge S^{0}
            \&[0.86602540378\ThreeCmPlusHalf]
            \&[0.86602540378\ThreeCmPlusHalf]
            S^{0}\otimes_{\Sets_{*}}(S^{0}\wedge S^{0})
            \&[0.785\ThreeCmPlusHalf]
            \&[0.86602540378\ThreeCmPlusHalf]
            \&[0.86602540378\ThreeCmPlusHalf]
            \\[0.45\ThreeCmPlusHalf]
            \&[0.86602540378\ThreeCmPlusHalf]
            \&[0.86602540378\ThreeCmPlusHalf]
            \&[0.785\ThreeCmPlusHalf]
            (X\wedge Y)\wedge Z
            \&[0.86602540378\ThreeCmPlusHalf]
            \&[0.86602540378\ThreeCmPlusHalf]
            X\otimes_{\Sets_{*}}(Y\wedge Z)
            \\[0.55\ThreeCmPlusHalf]
            (S^{0}\wedge S^{0})\otimes_{\Sets_{*}}S^{0}
            \&[0.86602540378\ThreeCmPlusHalf]
            \&[0.86602540378\ThreeCmPlusHalf]
            S^{0}\otimes_{\Sets_{*}}(S^{0}\otimes_{\Sets_{*}}S^{0})
            \&[0.785\ThreeCmPlusHalf]
            \&[0.86602540378\ThreeCmPlusHalf]
            \&[0.86602540378\ThreeCmPlusHalf]
            \\[0.45\ThreeCmPlusHalf]
            \&[0.86602540378\ThreeCmPlusHalf]
            \&[0.86602540378\ThreeCmPlusHalf]
            \&[0.785\ThreeCmPlusHalf]
            (X\wedge Y)\otimes_{\Sets_{*}}Z
            \&[0.86602540378\ThreeCmPlusHalf]
            \&[0.86602540378\ThreeCmPlusHalf]
            X\otimes_{\Sets_{*}}(Y\otimes_{\Sets_{*}}Z)
            \\[0.05\ThreeCmPlusHalf]
            \&[0.86602540378\ThreeCmPlusHalf]
            (S^{0}\otimes_{\Sets_{*}}S^{0})\otimes_{\Sets_{*}}S^{0}
            \&[0.86602540378\ThreeCmPlusHalf]
            \&[0.785\ThreeCmPlusHalf]
            \&[0.86602540378\ThreeCmPlusHalf]
            \&[0.86602540378\ThreeCmPlusHalf]
            \\[0.45\ThreeCmPlusHalf]
            \&[0.86602540378\ThreeCmPlusHalf]
            \&[0.86602540378\ThreeCmPlusHalf]
            \&[0.785\ThreeCmPlusHalf]
            \&[0.86602540378\ThreeCmPlusHalf]
            (X\otimes_{\Sets_{*}}Y)\otimes_{\Sets_{*}}Z
            \&[0.86602540378\ThreeCmPlusHalf]
            % 1-Arrows
            % First Hexagon Left
            \arrow[from=1-2,to=3-1,"\alpha^{\Sets_{*},-1}_{S^{0},S^{0},S^{0}}"sloped]%
            \arrow[from=3-1,to=5-1,"\id^{\otimes,-1}_{S^{0}\wedge S^{0},S^{0}}"']%
            \arrow[from=5-1,to=7-2,"\id^{\otimes,-1}_{S^{0},S^{0}}\otimes_{\Sets_{*}}\id_{S^{0}}"'sloped,pos=0.4]%
            % First Hexagon Right
            \arrow[from=1-2,to=3-3,"\id^{\otimes,-1}_{S^{0},S^{0}\wedge S^{0}}"{sloped,description}]%
            \arrow[from=3-3,to=5-3,"\id_{S^{0}}\otimes_{\Sets_{*}}\id^{\otimes,-1}_{S^{0},S^{0}}"description]%
            \arrow[from=5-3,to=7-2,"\alpha^{\prime,-1}_{S^{0},S^{0},S^{0}}"{sloped,pos=0.9}]%
            % Second Hexagon Left
            \arrow[from=6-4,to=8-5,"\id^{\otimes,-1}_{\Sets_{*}|X,Y}\otimes_{\Sets_{*}}\id_{Z}"'{sloped,description}]%
            % Second Hexagon Right
            \arrow[from=2-5,to=4-6,"\id^{\otimes,-1}_{\Sets_{*}|X,Y\wedge Z}"{sloped}]%
            \arrow[from=4-6,to=6-6,"\id_{X}\otimes_{\Sets_{*}}\id^{\otimes,-1}_{\Sets_{*}|Y,Z}"]%
            \arrow[from=6-6,to=8-5,"\alpha^{\prime,-1}_{X,Y,Z}"'{sloped}]%
            % Connecting Arrows
            \arrow[from=1-2,to=2-5,"{[x]\wedge([y]\wedge[z])}"sloped]%
            \arrow[from=3-1,to=4-4,"{([x]\wedge[y])\wedge[z]}"{description,sloped},crossing over]%
            \arrow[from=3-3,to=4-6,"{[x]\otimes_{\Sets_{*}}([y]\wedge[z])}"{description,sloped,pos=0.75}]%
            \arrow[from=5-1,to=6-4,"{([x]\wedge[y])\otimes_{\Sets_{*}}[z]}"description,sloped,crossing over]%
            \arrow[from=5-3,to=6-6,"{[x]\otimes_{\Sets_{*}}([y]\otimes_{\Sets_{*}}[z])}"{description,sloped,pos=0.5}]%
            \arrow[from=7-2,to=8-5,"{([x]\otimes_{\Sets_{*}}[y])\otimes_{\Sets_{*}}[z]}"',sloped]%
            % Crossing Over
            \arrow[from=2-5,to=4-4,"\alpha^{\Sets_{*},-1}_{X,Y,Z}"'{sloped,description},crossing over]%
            \arrow[from=4-4,to=6-4,"\id^{\otimes,-1}_{\Sets_{*}|X\wedge Y,Z}"'description,crossing over]%
        \end{tikzcd}
    \end{scalemath}
    which we partition into subdiagrams as follows:
    \begin{scalemath}
        \begin{tikzcd}[row sep={0.0*\the\DL,between origins}, column sep={0.0*\the\DL,between origins}, background color=backgroundColor, ampersand replacement=\&]
            \&[0.86602540378\ThreeCmPlusHalf]
            S^{0}\wedge(S^{0}\wedge S^{0})
            \&[0.86602540378\ThreeCmPlusHalf]
            \&[0.785\ThreeCmPlusHalf]
            \&[0.86602540378\ThreeCmPlusHalf]
            \&[0.86602540378\ThreeCmPlusHalf]
            \\[0.45\ThreeCmPlusHalf]
            \&[0.86602540378\ThreeCmPlusHalf]
            \&[0.86602540378\ThreeCmPlusHalf]
            \&[0.785\ThreeCmPlusHalf]
            \&[0.86602540378\ThreeCmPlusHalf]
            \&[0.86602540378\ThreeCmPlusHalf]
            \\[0.05\ThreeCmPlusHalf]
            (S^{0}\wedge S^{0})\wedge S^{0}
            \&[0.86602540378\ThreeCmPlusHalf]
            \&[0.86602540378\ThreeCmPlusHalf]
            S^{0}\otimes_{\Sets_{*}}(S^{0}\wedge S^{0})
            \&[0.785\ThreeCmPlusHalf]
            \&[0.86602540378\ThreeCmPlusHalf]
            \&[0.86602540378\ThreeCmPlusHalf]
            \\[0.45\ThreeCmPlusHalf]
            \&[0.86602540378\ThreeCmPlusHalf]
            \&[0.86602540378\ThreeCmPlusHalf]
            \&[0.785\ThreeCmPlusHalf]
            \&[0.86602540378\ThreeCmPlusHalf]
            \&[0.86602540378\ThreeCmPlusHalf]
            \\[0.55\ThreeCmPlusHalf]
            (S^{0}\wedge S^{0})\otimes_{\Sets_{*}}S^{0}
            \&[0.86602540378\ThreeCmPlusHalf]
            \&[0.86602540378\ThreeCmPlusHalf]
            S^{0}\otimes_{\Sets_{*}}(S^{0}\otimes_{\Sets_{*}}S^{0})
            \&[0.785\ThreeCmPlusHalf]
            \&[0.86602540378\ThreeCmPlusHalf]
            \&[0.86602540378\ThreeCmPlusHalf]
            \\[0.45\ThreeCmPlusHalf]
            \&[0.86602540378\ThreeCmPlusHalf]
            \&[0.86602540378\ThreeCmPlusHalf]
            \&[0.785\ThreeCmPlusHalf]
            \&[0.86602540378\ThreeCmPlusHalf]
            \&[0.86602540378\ThreeCmPlusHalf]
            \\[0.05\ThreeCmPlusHalf]
            \&[0.86602540378\ThreeCmPlusHalf]
            (S^{0}\otimes_{\Sets_{*}}S^{0})\otimes_{\Sets_{*}}S^{0}
            \&[0.86602540378\ThreeCmPlusHalf]
            \&[0.785\ThreeCmPlusHalf]
            \&[0.86602540378\ThreeCmPlusHalf]
            \&[0.86602540378\ThreeCmPlusHalf]
            \\[0.45\ThreeCmPlusHalf]
            \&[0.86602540378\ThreeCmPlusHalf]
            \&[0.86602540378\ThreeCmPlusHalf]
            \&[0.785\ThreeCmPlusHalf]
            \&[0.86602540378\ThreeCmPlusHalf]
            \&[0.86602540378\ThreeCmPlusHalf]
            % 1-Arrows
            % First Hexagon Left
            \arrow[from=1-2,to=3-1,"\alpha^{\Sets_{*},-1}_{S^{0},S^{0},S^{0}}"sloped]%
            \arrow[from=3-1,to=5-1,"\id^{\otimes,-1}_{S^{0}\wedge S^{0},S^{0}}"']%
            \arrow[from=5-1,to=7-2,"\id^{\otimes,-1}_{S^{0},S^{0}}\otimes_{\Sets_{*}}\id_{S^{0}}"'sloped,pos=0.4]%
            % First Hexagon Right
            \arrow[from=1-2,to=3-3,"\id^{\otimes,-1}_{S^{0},S^{0}\wedge S^{0}}"{sloped,description}]%
            \arrow[from=3-3,to=5-3,"\id_{S^{0}}\otimes_{\Sets_{*}}\id^{\otimes,-1}_{S^{0},S^{0}}"description]%
            \arrow[from=5-3,to=7-2,"\alpha^{\prime,-1}_{S^{0},S^{0},S^{0}}"{sloped,pos=0.9}]%
            % Subdiagram
            \arrow[from=1-2,to=7-2,"(\dagger)"{xscale=1.25,yscale=1.25},phantom]%
        \end{tikzcd}
    \end{scalemath}
    \begin{scalemath}
        \begin{tikzcd}[row sep={0.0*\the\DL,between origins}, column sep={0.0*\the\DL,between origins}, background color=backgroundColor, ampersand replacement=\&]
            \&[0.86602540378\ThreeCmPlusHalf]
            S^{0}\wedge(S^{0}\wedge S^{0})
            \&[0.86602540378\ThreeCmPlusHalf]
            \&[0.785\ThreeCmPlusHalf]
            \&[0.86602540378\ThreeCmPlusHalf]
            \&[0.86602540378\ThreeCmPlusHalf]
            \\[0.45\ThreeCmPlusHalf]
            \&[0.86602540378\ThreeCmPlusHalf]
            \&[0.86602540378\ThreeCmPlusHalf]
            \&[0.785\ThreeCmPlusHalf]
            \&[0.86602540378\ThreeCmPlusHalf]
            X\wedge(Y\wedge Z)
            \&[0.86602540378\ThreeCmPlusHalf]
            \\[0.05\ThreeCmPlusHalf]
            (S^{0}\wedge S^{0})\wedge S^{0}
            \&[0.86602540378\ThreeCmPlusHalf]
            \&[0.86602540378\ThreeCmPlusHalf]
            \&[0.785\ThreeCmPlusHalf]
            \&[0.86602540378\ThreeCmPlusHalf]
            \&[0.86602540378\ThreeCmPlusHalf]
            \\[0.45\ThreeCmPlusHalf]
            \&[0.86602540378\ThreeCmPlusHalf]
            \&[0.86602540378\ThreeCmPlusHalf]
            \&[0.785\ThreeCmPlusHalf]
            (X\wedge Y)\wedge Z
            \&[0.86602540378\ThreeCmPlusHalf]
            \&[0.86602540378\ThreeCmPlusHalf]
            \\[0.55\ThreeCmPlusHalf]
            (S^{0}\wedge S^{0})\otimes_{\Sets_{*}}S^{0}
            \&[0.86602540378\ThreeCmPlusHalf]
            \&[0.86602540378\ThreeCmPlusHalf]
            {}
            \&[0.785\ThreeCmPlusHalf]
            \&[0.86602540378\ThreeCmPlusHalf]
            \&[0.86602540378\ThreeCmPlusHalf]
            \\[0.45\ThreeCmPlusHalf]
            \&[0.86602540378\ThreeCmPlusHalf]
            \&[0.86602540378\ThreeCmPlusHalf]
            \&[0.785\ThreeCmPlusHalf]
            (X\wedge Y)\otimes_{\Sets_{*}}Z
            \&[0.86602540378\ThreeCmPlusHalf]
            \&[0.86602540378\ThreeCmPlusHalf]
            \\[0.05\ThreeCmPlusHalf]
            \&[0.86602540378\ThreeCmPlusHalf]
            (S^{0}\otimes_{\Sets_{*}}S^{0})\otimes_{\Sets_{*}}S^{0}
            \&[0.86602540378\ThreeCmPlusHalf]
            \&[0.785\ThreeCmPlusHalf]
            \&[0.86602540378\ThreeCmPlusHalf]
            \&[0.86602540378\ThreeCmPlusHalf]
            \\[0.45\ThreeCmPlusHalf]
            \&[0.86602540378\ThreeCmPlusHalf]
            \&[0.86602540378\ThreeCmPlusHalf]
            \&[0.785\ThreeCmPlusHalf]
            \&[0.86602540378\ThreeCmPlusHalf]
            (X\otimes_{\Sets_{*}}Y)\otimes_{\Sets_{*}}Z
            \&[0.86602540378\ThreeCmPlusHalf]
            % 1-Arrows
            % First Hexagon Left
            \arrow[from=1-2,to=3-1,"\alpha^{\Sets_{*},-1}_{S^{0},S^{0},S^{0}}"sloped]%
            \arrow[from=3-1,to=5-1,"\id^{\otimes,-1}_{S^{0}\wedge S^{0},S^{0}}"']%
            \arrow[from=5-1,to=7-2,"\id^{\otimes,-1}_{S^{0},S^{0}}\otimes_{\Sets_{*}}\id_{S^{0}}"'sloped,pos=0.4]%
            % Second Hexagon Left
            \arrow[from=6-4,to=8-5,"\id^{\otimes,-1}_{\Sets_{*}|X,Y}\otimes_{\Sets_{*}}\id_{Z}"'{sloped,description}]%
            % Connecting Arrows
            \arrow[from=1-2,to=2-5,"{[x]\wedge([y]\wedge[z])}"sloped]%
            \arrow[from=3-1,to=4-4,"{([x]\wedge[y])\wedge[z]}"{description,sloped},crossing over]%
            \arrow[from=5-1,to=6-4,"{([x]\wedge[y])\otimes_{\Sets_{*}}[z]}"description,sloped,crossing over]%
            \arrow[from=7-2,to=8-5,"{([x]\otimes_{\Sets_{*}}[y])\otimes_{\Sets_{*}}[z]}"',sloped]%
            % Crossing Over
            \arrow[from=2-5,to=4-4,"\alpha^{\Sets_{*},-1}_{X,Y,Z}"'{sloped,description},crossing over]%
            \arrow[from=4-4,to=6-4,"\id^{\otimes,-1}_{\Sets_{*}|X\wedge Y,Z}"'description,crossing over]%
            % Subdiagrams
            \arrow[from=1-2,to=4-4,"\scriptstyle(1)"{rotate=8.7, xslant=-1.28455656, yslant=0,xscale=10.1931, yscale=3.1857,rotate=-60,xscale=0.25,yscale=0.25},phantom]%
            \arrow[from=5-1,to=4-4,"\scriptstyle(2)"{rotate=-10,xslant=-0.183534345,yslant=0,xscale=16.9083,yscale=8.0921,rotate=1.0,xscale=0.15,yscale=0.15},phantom]%
            \arrow[from=7-2,to=6-4,"\scriptstyle(3)"{rotate=-11.1,xslant=-3.23047804,yslant=0,xscale=14.3481,yscale=1.9031,rotate=0.0,xscale=0.2,yscale=0.2},phantom]%
            % It's a (3)
            \arrow[from=4-4,to=5-3,dashed,bend left=20,xshift=-7.0em,yshift=-10.0em,end anchor={[xshift=3.0em,yshift=2.0em]}]%
            \arrow[from=4-4,to=5-3,dashed,bend left=20,xshift=-7.0em,yshift=-10.0em,end anchor={[xshift=3.0em,yshift=2.0em]},phantom,"\scriptstyle(3)",xshift=-2.75em,yshift=-0.9em]%
        \end{tikzcd}
    \end{scalemath}
    \begin{scalemath}
        \begin{tikzcd}[row sep={0.0*\the\DL,between origins}, column sep={0.0*\the\DL,between origins}, background color=backgroundColor, ampersand replacement=\&]
            \&[0.86602540378\ThreeCmPlusHalf]
            S^{0}\wedge(S^{0}\wedge S^{0})
            \&[0.86602540378\ThreeCmPlusHalf]
            \&[0.785\ThreeCmPlusHalf]
            \&[0.86602540378\ThreeCmPlusHalf]
            \&[0.86602540378\ThreeCmPlusHalf]
            \\[0.45\ThreeCmPlusHalf]
            \&[0.86602540378\ThreeCmPlusHalf]
            \&[0.86602540378\ThreeCmPlusHalf]
            \&[0.785\ThreeCmPlusHalf]
            \&[0.86602540378\ThreeCmPlusHalf]
            X\wedge(Y\wedge Z)
            \&[0.86602540378\ThreeCmPlusHalf]
            \\[0.05\ThreeCmPlusHalf]
            \&[0.86602540378\ThreeCmPlusHalf]
            \&[0.86602540378\ThreeCmPlusHalf]
            S^{0}\otimes_{\Sets_{*}}(S^{0}\wedge S^{0})
            \&[0.785\ThreeCmPlusHalf]
            \&[0.86602540378\ThreeCmPlusHalf]
            \&[0.86602540378\ThreeCmPlusHalf]
            \\[0.45\ThreeCmPlusHalf]
            \&[0.86602540378\ThreeCmPlusHalf]
            \&[0.86602540378\ThreeCmPlusHalf]
            \&[0.785\ThreeCmPlusHalf]
            \&[0.86602540378\ThreeCmPlusHalf]
            \&[0.86602540378\ThreeCmPlusHalf]
            X\otimes_{\Sets_{*}}(Y\wedge Z)
            \\[0.55\ThreeCmPlusHalf]
            \&[0.86602540378\ThreeCmPlusHalf]
            \&[0.86602540378\ThreeCmPlusHalf]
            S^{0}\otimes_{\Sets_{*}}(S^{0}\otimes_{\Sets_{*}}S^{0})
            \&[0.785\ThreeCmPlusHalf]
            \&[0.86602540378\ThreeCmPlusHalf]
            \&[0.86602540378\ThreeCmPlusHalf]
            \\[0.45\ThreeCmPlusHalf]
            \&[0.86602540378\ThreeCmPlusHalf]
            \&[0.86602540378\ThreeCmPlusHalf]
            \&[0.785\ThreeCmPlusHalf]
            \&[0.86602540378\ThreeCmPlusHalf]
            \&[0.86602540378\ThreeCmPlusHalf]
            X\otimes_{\Sets_{*}}(Y\otimes_{\Sets_{*}}Z)\mrp{.}
            \\[0.05\ThreeCmPlusHalf]
            \&[0.86602540378\ThreeCmPlusHalf]
            (S^{0}\otimes_{\Sets_{*}}S^{0})\otimes_{\Sets_{*}}S^{0}
            \&[0.86602540378\ThreeCmPlusHalf]
            \&[0.785\ThreeCmPlusHalf]
            \&[0.86602540378\ThreeCmPlusHalf]
            \&[0.86602540378\ThreeCmPlusHalf]
            \\[0.45\ThreeCmPlusHalf]
            \&[0.86602540378\ThreeCmPlusHalf]
            \&[0.86602540378\ThreeCmPlusHalf]
            \&[0.785\ThreeCmPlusHalf]
            \&[0.86602540378\ThreeCmPlusHalf]
            (X\otimes_{\Sets_{*}}Y)\otimes_{\Sets_{*}}Z
            \&[0.86602540378\ThreeCmPlusHalf]
            % 1-Arrows
            % First Hexagon Right
            \arrow[from=1-2,to=3-3,"\id^{\otimes,-1}_{S^{0},S^{0}\wedge S^{0}}"{sloped,description}]%
            \arrow[from=3-3,to=5-3,"\id_{S^{0}}\otimes_{\Sets_{*}}\id^{\otimes,-1}_{S^{0},S^{0}}"description]%
            \arrow[from=5-3,to=7-2,"\alpha^{\prime,-1}_{S^{0},S^{0},S^{0}}"{sloped,pos=0.4}]%
            % Second Hexagon Right
            \arrow[from=2-5,to=4-6,"\id^{\otimes,-1}_{\Sets_{*}|X,Y\wedge Z}"{sloped}]%
            \arrow[from=4-6,to=6-6,"\id_{X}\otimes_{\Sets_{*}}\id^{\otimes,-1}_{\Sets_{*}|Y,Z}"]%
            \arrow[from=6-6,to=8-5,"\alpha^{\prime,-1}_{X,Y,Z}"'{sloped,pos=0.4}]%
            % Connecting Arrows
            \arrow[from=1-2,to=2-5,"{[x]\wedge([y]\wedge[z])}"sloped]%
            \arrow[from=3-3,to=4-6,"{[x]\otimes_{\Sets_{*}}([y]\wedge[z])}"{description,sloped,pos=0.75}]%
            \arrow[from=5-3,to=6-6,"{[x]\otimes_{\Sets_{*}}([y]\otimes_{\Sets_{*}}[z])}"{description,sloped,pos=0.5}]%
            \arrow[from=7-2,to=8-5,"{([x]\otimes_{\Sets_{*}}[y])\otimes_{\Sets_{*}}[z]}"',sloped]%
            % Subdiagrams
            \arrow[from=3-3,to=2-5,"\scriptstyle(4)"{rotate=-11.1,xslant=-3.23047804,yslant=0,xscale=14.3481,yscale=1.9031,rotate=0.0,xscale=0.2,yscale=0.2},phantom]%
            \arrow[from=5-3,to=4-6,"\scriptstyle(5)"{rotate=-10,xslant=-0.183534345,yslant=0,xscale=16.9083,yscale=8.0921,rotate=1.0,xscale=0.15,yscale=0.15},phantom]%
            \arrow[from=7-2,to=6-6,"\scriptstyle(6)"{rotate=8.7, xslant=-1.28455656, yslant=0,xscale=10.1931, yscale=3.1857,rotate=-60,xscale=0.25,yscale=0.25},phantom]%
            % It's a (4)
            \arrow[from=3-3,to=2-5,dashed,bend left=20,crossing over,start anchor={[xshift=6.9em,yshift=0.1em]},end anchor={[xshift=-2.0em,yshift=1.5em]}]%
            \arrow[from=3-3,to=2-5,dashed,bend left=20,start anchor={[xshift=6.0em,yshift=0.5em]},end anchor={[xshift=-2.0em,yshift=1.5em]},phantom,"\scriptstyle(4)",xshift=2.6em,yshift=0.4em]%
        \end{tikzcd}
    \end{scalemath}
    Since:
    \begin{itemize}
        \item Subdiagram $(1)$ commutes by the naturality of $\alpha^{\Sets_{*},-1}$.
        \item Subdiagram $(2)$ commutes by the naturality of $\id^{\otimes,-1}_{\Sets_{*}}$.
        \item Subdiagram $(3)$ commutes by the naturality of $\id^{\otimes,-1}_{\Sets_{*}}$.
        \item Subdiagram $(\dagger)$ commutes, as proved above.
        \item Subdiagram $(4)$ commutes by the naturality of $\id^{\otimes,-1}_{\Sets_{*}}$.
        \item Subdiagram $(5)$ commutes by the naturality of $\id^{\otimes,-1}_{\Sets_{*}}$.
        \item Subdiagram $(6)$ commutes by the naturality of $\alpha^{\prime,-1}$.
    \end{itemize}
    it follows that the diagram
    \[
        \begin{tikzcd}[row sep={0.0*\the\DL,between origins}, column sep={0.0*\the\DL,between origins}, background color=backgroundColor, ampersand replacement=\&]
            \&[0.86602540378\TwoCmPlusHalf]
            S^{0}\wedge(S^{0}\wedge S^{0})
            \arrow[d,"{[x]\wedge([y]\wedge[z])}"'description]
            \&[0.86602540378\TwoCmPlusHalf]
            \\[0.75\TwoCmPlusHalf]
            \&[0.86602540378\TwoCmPlusHalf]
            X\wedge(Y\wedge Z)
            \arrow[ld,"\alpha^{\Sets_{*},-1}_{X,Y,Z}"',pos=0.5]
            \arrow[rd,"\id^{\otimes,-1}_{\Sets_{*}|X,Y\wedge Z}",pos=0.5]
            \&[0.86602540378\TwoCmPlusHalf]
            \\[0.5\TwoCmPlusHalf]
            (X\wedge Y)\wedge Z
            \arrow[d,"\id^{\otimes,-1}_{\Sets_{*}|X\wedge Y,Z}"']
            \&[0.86602540378\TwoCmPlusHalf]
            \&[0.86602540378\TwoCmPlusHalf]
            X\otimes_{\Sets_{*}}(Y\wedge Z)
            \arrow[d,"\id_{X}\wedge\id^{\otimes,-1}_{\Sets_{*}|Y,Z}"]
            \\[\TwoCmPlusHalf]
            (X\wedge Y)\otimes_{\Sets_{*}}Z
            \arrow[rd,"\id^{\otimes,-1}_{\Sets_{*}|X,Y}\otimes_{\Sets_{*}}\id_{Z}"',pos=0.35]
            \&[0.86602540378\TwoCmPlusHalf]
            \&[0.86602540378\TwoCmPlusHalf]
            X\otimes_{\Sets_{*}}(Y\otimes_{\Sets_{*}}Z)
            \arrow[ld,"\alpha^{\prime,-1}_{X,Y,Z}",pos=0.35]
            \\[0.5\TwoCmPlusHalf]
            \&[0.86602540378\TwoCmPlusHalf]
            (X\otimes_{\Sets_{*}}Y)\otimes_{\Sets_{*}}Z
            \&[0.86602540378\TwoCmPlusHalf]
        \end{tikzcd}
    \]%
    also commutes. We then have
    \begin{envscriptsize}
        \begin{align*}
            \left[(\id^{\otimes,-1}_{\Sets_{*}|X,Y}\otimes_{\Sets_{*}}\id_{Z})\circ\id^{\otimes,-1}_{\Sets_{*}|X\wedge Y,Z}\right.\\
            \left.{}\circ\alpha^{\Sets_{*},-1}_{X,Y,Z}\right](x,(y,z)) &= \left[(\id^{\otimes,-1}_{\Sets_{*}|X,Y}\otimes_{\Sets_{*}}\id_{Z})\circ\id^{\otimes,-1}_{\Sets_{*}|X\wedge Y,Z}\right.\\
                                                                                                     &\phantom{={}} \mkern4mu\left.{}{}\circ\alpha^{\Sets_{*},-1}_{X,Y,Z}\circ([x]\wedge([y]\wedge[z]))\right](1,(1,1))\\
                                                                                                     &= \left[\alpha^{\prime,-1}_{X,Y,Z}\circ(\id_{X}\wedge\id^{\otimes,-1}_{\Sets_{*}|Y,Z})\right.\\
                                                                                                     &\phantom{={}} \mkern4mu\left.{}\circ\id^{\otimes,-1}_{\Sets_{*}|X,Y\wedge Z}\circ([x]\wedge([y]\wedge[z]))\right](1,(1,1))\\
                                                                                                     &= [\alpha^{\prime,-1}_{X,Y,Z}\circ(\id_{X}\wedge\id^{\otimes,-1}_{\Sets_{*}|Y,Z})\circ\id^{\otimes,-1}_{\Sets_{*}|X,Y\wedge Z}](x,(y,z))
        \end{align*}
    \end{envscriptsize}
    for each $(x,(y,z))\in X\wedge(Y\wedge Z)$, and thus we have
    \begin{envscriptsize}
        \[
            (\id^{\otimes,-1}_{\Sets_{*}|X,Y}\otimes_{\Sets_{*}}\id_{Z})\circ\id^{\otimes,-1}_{\Sets_{*}|X\wedge Y,Z}\circ\alpha^{\Sets_{*},-1}_{X,Y,Z}%
            =%
            \alpha^{\prime,-1}_{X,Y,Z}\circ(\id_{X}\wedge\id^{\otimes,-1}_{\Sets_{*}|Y,Z})\circ\id^{\otimes,-1}_{\Sets_{*}|X,Y\wedge Z}.%
        \]%
    \end{envscriptsize}
    Taking inverses then gives
    \begin{envscriptsize}
        \[
            \alpha^{\Sets_{*}}_{X,Y,Z}\circ\id^{\otimes}_{\Sets_{*}|X\wedge Y,Z}\circ(\id^{\otimes}_{\Sets_{*}|X,Y}\otimes_{\Sets_{*}}\id_{Z})%
            =%
            \id^{\otimes}_{\Sets_{*}|X,Y\wedge Z}\circ(\id_{X}\wedge\id^{\otimes}_{\Sets_{*}|Y,Z})\circ\alpha^{\prime}_{X,Y,Z},%
        \]%
    \end{envscriptsize}
    showing that the diagram
    \[
        \begin{tikzcd}[row sep={0.0*\the\DL,between origins}, column sep={0.0*\the\DL,between origins}, background color=backgroundColor, ampersand replacement=\&]
            \&[0.86602540378\TwoCmPlusHalf]
            (X\otimes_{\Sets_{*}}Y)\otimes_{\Sets_{*}}Z
            \arrow[ld,"\id^{\otimes}_{\Sets_{*}|X,Y}\otimes_{\Sets_{*}}\id_{Z}"',pos=0.6]
            \arrow[rd,"\alpha'_{X,Y,Z}",pos=0.6]
            \&[0.86602540378\TwoCmPlusHalf]
            \\[0.5\TwoCmPlusHalf]
            (X\wedge Y)\otimes_{\Sets_{*}}Z
            \arrow[d,"\id^{\otimes}_{\Sets_{*}|X\wedge Y,Z}"']
            \&[0.86602540378\TwoCmPlusHalf]
            \&[0.86602540378\TwoCmPlusHalf]
            X\otimes_{\Sets_{*}}(Y\otimes_{\Sets_{*}}Z)
            \arrow[d,"\id_{X}\otimes_{\Sets_{*}}\id^{\otimes}_{\Sets_{*}|Y,Z}"]
            \\[\TwoCmPlusHalf]
            (X\wedge Y)\wedge Z
            \arrow[rd,"\alpha^{\Sets_{*}}_{X,Y,Z}"',pos=0.45]
            \&[0.86602540378\TwoCmPlusHalf]
            \&[0.86602540378\TwoCmPlusHalf]
            X\otimes_{\Sets_{*}}(Y\wedge Z)
            \arrow[ld,"\id^{\otimes}_{\Sets_{*}|X,Y\wedge Z}",pos=0.45]
            \\[0.5\TwoCmPlusHalf]
            \&[0.86602540378\TwoCmPlusHalf]
            X\wedge(Y\wedge Z)
            \&[0.86602540378\TwoCmPlusHalf]
        \end{tikzcd}
    \]%
    indeed commutes.

    \ProofBox{Uniqueness of the Isomorphism $\mathord{\otimes_{\Sets_{*}}}\cong\mathord{\wedge}$}%
    Let $\phi,\psi\colon-_{1}\otimes_{\Sets_{*}}-_{2}\Rightarrow-_{1}\wedge-_{2}$ be natural isomorphisms. Since these isomorphisms are compatible with the unitors of $\Sets_{*}$ with respect to $\wedge$ and $\otimes$ (as shown above), we have
    \begin{align*}
        \LUnitor'_{Y} &= \LUnitor^{\Sets_{*}}_{Y}\circ\phi_{S^{0},Y}\circ(\id^{\otimes}_{\Unit|\Sets}\otimes_{\Sets}\id_{Y}),\\
        \LUnitor'_{Y} &= \LUnitor^{\Sets_{*}}_{Y}\circ\psi_{S^{0},Y}\circ(\id^{\otimes}_{\Unit|\Sets}\otimes_{\Sets}\id_{Y}).
    \end{align*}
    Postcomposing both sides with $\LUnitor^{\Sets_{*},-1}_{Y}$ and then precomposing both sides with $\id^{\otimes,-1}_{\Unit|\Sets}\otimes_{\Sets}\id_{Y}$ gives
    \begin{align*}
        \LUnitor^{\Sets_{*},-1}_{Y}\circ\LUnitor'_{Y}\circ(\id^{\otimes,-1}_{\Unit|\Sets}\otimes_{\Sets}\id_{Y}) &= \phi_{S^{0},Y},\\
        \LUnitor^{\Sets_{*},-1}_{Y}\circ\LUnitor'_{Y}\circ(\id^{\otimes,-1}_{\Unit|\Sets}\otimes_{\Sets}\id_{Y}) &= \psi_{S^{0},Y},
    \end{align*}
    and thus we have
    \[
        \phi_{S^{0},Y}%
        =%
        \psi_{S^{0},Y}%
    \]%
    for each $Y\in\Obj(\Sets_{*})$. Now, let $x\in X$ and consider the naturality diagrams
    \begin{webcompile}
        \begin{tikzcd}[row sep={5.0*\the\DL,between origins}, column sep={10.5*\the\DL,between origins}, background color=backgroundColor, ampersand replacement=\&]
            S^{0}\wedge Y
            \arrow[r,"{[x]\wedge\id_{Y}}"]
            \arrow[d,"\phi_{S^{0},Y}"']
            \&
            X\wedge Y
            \arrow[d,"\phi_{X,Y}"]
            \\
            S^{0}\otimes_{\Sets_{*}}Y
            \arrow[r,"{[x]\otimes_{\Sets_{*}}\id_{Y}}"']
            \&
            X\otimes_{\Sets_{*}}Y
        \end{tikzcd}
        \quad
        \begin{tikzcd}[row sep={5.0*\the\DL,between origins}, column sep={10.5*\the\DL,between origins}, background color=backgroundColor, ampersand replacement=\&]
            S^{0}\wedge Y
            \arrow[r,"{[x]\wedge\id_{Y}}"]
            \arrow[d,"\psi_{S^{0},Y}"']
            \&
            X\wedge Y
            \arrow[d,"\psi_{X,Y}"]
            \\
            S^{0}\otimes_{\Sets_{*}}Y
            \arrow[r,"{[x]\otimes_{\Sets_{*}}\id_{Y}}"']
            \&
            X\otimes_{\Sets_{*}}Y
        \end{tikzcd}
    \end{webcompile}
    for $\phi$ and $\psi$ with respect to the morphisms $[x]$ and $\id_{Y}$. Having shown that $\phi_{S^{0},Y}=\psi_{S^{0},Y}$, we have
    \begin{align*}
        \phi_{X,Y}(x,y) &= [\phi_{X,Y}\circ([x]\wedge\id_{Y})](1,y)\\
                        &= [([x]\otimes_{\Sets_{*}}\id_{Y})\circ\phi_{S^{0},Y}](1,y)\\
                        &= [([x]\otimes_{\Sets_{*}}\id_{Y})\circ\psi_{S^{0},Y}](1,y)\\
                        &= [\psi_{X,Y}\circ([x]\wedge\id_{Y})](1,y)\\%
                        &= \psi_{X,Y}(x,y)
    \end{align*}
    for each $(x,y)\in X\wedge Y$. Therefore we have
    \[
        \phi_{X,Y}%
        =%
        \psi_{X,Y}%
    \]%
    for each $X,Y\in\Obj(\Sets_{*})$ and thus $\phi=\psi$, showing the isomorphism $\mathord{\otimes_{\Sets_{*}}}\cong\mathord{\times}$ to be unique.
\end{Proof}
\begin{corollary}{A Second Universal Property for $(\Sets_{*},\wedge,S^{0})$}{a-second-universal-property-for-sets-star-smash-s-zero}%
    The symmetric monoidal structure on the category $\Sets_{*}$ of \cref{the-monoidal-structure-on-pointed-sets-associated-to-the-smash-product-of-pointed-sets} is uniquely determined by the following requirements:
    \begin{enumerate}
        \item\label{a-second-universal-property-for-sets-star-smash-s-zero-sided-preservation-of-colimits}\SloganFont{Two-Sided Preservation of Colimits. }The tensor product
            \[
                \otimes_{\Sets_{*}}%
                \colon
                \Sets_{*}\times\Sets_{*}
                \to
                \Sets_{*}
            \]%
            of $\Sets_{*}$ preserves colimits separately in each variable.
        \item\label{a-second-universal-property-for-sets-star-smash-s-zero-the-unit-object-is-s-zero}\SloganFont{The Unit Object Is $S^{0}$. }We have $\Unit_{\Sets_{*}}\cong S^{0}$.
    \end{enumerate}
    More precisely, the full subcategory of the category $\ModuliCategory_{\E_{\infty}}(\Sets_{*})$ of \cref{TODO} spanned by the symmetric monoidal categories $\left(\phantom{\mrp{\LUnitor^{\Sets_{*}}}}\Sets_{*}\right.$, $\otimes_{\Sets_{*}}$, $\Unit_{\Sets_{*}}$, $\LUnitor^{\Sets_{*}}$, $\RUnitor^{\Sets_{*}}$, $\left.\sigma^{\Sets_{*}}\right)$ satisfying \cref{a-second-universal-property-for-sets-star-smash-s-zero-sided-preservation-of-colimits,a-second-universal-property-for-sets-star-smash-s-zero-the-unit-object-is-s-zero} is contractible.
\end{corollary}
\begin{Proof}{Proof of \cref{a-second-universal-property-for-sets-star-smash-s-zero}}%
    Since $\Sets_{*}$ is locally presentable (\cref{TODO}), it follows from \cref{TODO} that \cref{a-second-universal-property-for-sets-star-smash-s-zero} is equivalent to the existence of an internal Hom as in \cref{the-universal-property-of-sets-star-smash-s-zero-existence-of-an-internal-hom} of \cref{the-universal-property-of-sets-star-smash-s-zero}. The result then follows from \cref{the-universal-property-of-sets-star-smash-s-zero}.
\end{Proof}
\begin{corollary}{A Third Universal Property of the Smash Product of Pointed Sets}{a-third-universal-property-of-the-smash-product-of-pointed-sets}%
    The symmetric monoidal structure on the category $\Sets_{*}$ is the unique symmetric monoidal structure on $\Sets_{*}$ such that the free pointed set functor
    \[
        (-)^{+}
        \colon
        \Sets
        \to
        \Sets_{*}
    \]%
    admits a symmetric monoidal structure, i.e.\ the full subcategory of the category $\ModuliCategory_{\E_{\infty}}(\Sets_{*})$ of \cref{TODO} spanned by the symmetric monoidal categories $\left(\phantom{\mrp{\LUnitor^{\Sets_{*}}}}\Sets_{*}\right.$, $\otimes_{\Sets_{*}}$, $\Unit_{\Sets_{*}}$, $\LUnitor^{\Sets_{*}}$, $\RUnitor^{\Sets_{*}}$, $\left.\sigma^{\Sets_{*}}\right)$ with respect to which $(-)^{+}$ admits a symmetric monoidal structure is contractible.
\end{corollary}
\begin{Proof}{Proof of \cref{a-third-universal-property-of-the-smash-product-of-pointed-sets}}%
    Let $(\otimes_{\Sets_{*}},\Unit_{\Sets_{*}},\LUnitor^{\Sets_{*}},\RUnitor^{\Sets_{*}},\sigma^{\Sets_{*}})$ be a symmetric monoidal structure on $\Sets_{*}$ such that $(-)^{+}$ admits a symmetric monoidal structure with respect to $\otimes_{\Sets_{*}}$ and $\wedge$. We have isomorphisms
    \begin{align*}
        X\otimes_{\Sets_{*}}Y &\cong (X^{-})^{+}\otimes_{\Sets_{*}}(Y^{-})^{+}\\
                              &\cong (X^{-}\times Y^{-})^{+}\\
                              &\cong (X^{-})^{+}\wedge(Y^{-})^{+}\\
                              &\cong X\wedge Y,
    \end{align*}
    all natural in $X$ and $Y$. Now, since $\wedge$ preserves colimits in both variables and $\mathord{\otimes_{\Sets_{*}}}\cong\mathord{\wedge}$, it follows that $\otimes_{\Sets_{*}}$ also preserves colimits in both variables, so the result then follows from \cref{a-second-universal-property-for-sets-star-smash-s-zero}.
\end{Proof}
\subsection{Monoids With Respect to the Smash Product of Pointed Sets}\label{subsection-monoids-with-respect-to-the-smash-product-of-pointed-sets}
\begin{proposition}{Monoids With Respect to $\wedge$}{monoids-with-respect-to-the-smash-product-of-pointed-sets}%
    The category of monoids on $\smash{(\Sets_{*},\wedge,S^{0})}$ is isomorphic to the category of monoids with zero and morphisms between them.
\end{proposition}
\begin{Proof}{Proof of \cref{monoids-with-respect-to-the-smash-product-of-pointed-sets}}%
    See \ChapterMonoidsWithZero, in particular \ChapterRef{\ChapterMonoidsWithZero, \cref{monoids-with-zero:unwinding-monoids-with-zero}, \cref{monoids-with-zero:unwinding-morphisms-of-monoids-with-zero}, and \cref{monoids-with-zero:unwinding-the-category-of-monoids-with-zero}}{\cref{unwinding-monoids-with-zero}, \cref{unwinding-morphisms-of-monoids-with-zero}, and \cref{unwinding-the-category-of-monoids-with-zero}}.
\end{Proof}
\subsection{Comonoids With Respect to the Smash Product of Pointed Sets}\label{subsection-comonoids-with-respect-to-the-smash-product-of-pointed-sets}
\begin{proposition}{Comonoids With Respect to $\wedge$}{comonoids-with-respect-to-the-smash-product-of-pointed-sets}%
    The symmetric monoidal functor
    \[
        ((-)^{+},(-)^{+,\times},(-)^{+,\times}_{\Unit})
        \colon
        (\Sets,\times,\pt)
        \to
        (\Sets_{*},\wedge,S^{0}),
    \]%
    of \ChapterRef{\ChapterPointedSets, \cref{pointed-sets:properties-of-free-pointed-sets-symmetric-strong-monoidality-with-respect-to-smash-products} of \cref{pointed-sets:properties-of-free-pointed-sets}}{\cref{properties-of-free-pointed-sets-symmetric-strong-monoidality-with-respect-to-smash-products} of \cref{properties-of-free-pointed-sets}} lifts to an equivalence of categories
    \begin{align*}
        \CoMon(\Sets_{*},\wedge,S^{0}) &\eqcong \CoMon(\Sets,\times,\pt)\\
                                       &\cong   \Sets.
    \end{align*}
\end{proposition}
\begin{Proof}{Proof of \cref{comonoids-with-respect-to-the-smash-product-of-pointed-sets}}%
    We follow \cite[Lemma 2.4]{coalgebras-in-symmetric-monoidal-categories-of-spectra}.

    \ProofBox{Faithfulness}%
    Given morphisms $f,g\colon X\to Y$, if $f^{+}=g^{+}$, then we have 
    \begin{align*}
        f(x) &\defeq f^{+}(x)\\
             &=      g^{+}(x)\\
             &\defeq g(x)
    \end{align*}
    for each $x\in X^{+}$, and thus $f=g$, showing $(-)^{+}$ to be faithful.

    \ProofBox{Fullness}%
    Let $f\colon X^{+}\to Y^{+}$ be a morphism of comonoids in $\Sets_{*}$. By counitality, the diagram
    \[
        \begin{tikzcd}[row sep={4.0*\the\DL,between origins}, column sep={3.0*\the\DL,between origins}, background color=backgroundColor, ampersand replacement=\&]
            X^{+}
            \arrow[rr,"f"]
            \arrow[rd,"\epsilon^{+}_{X}"'{pos=0.425}]
            \&
            \&
            Y^{+}
            \arrow[ld,"\epsilon^{+}_{Y}"{pos=0.425}]
            \\
            \&
            S^{0}
            \&
        \end{tikzcd}
    \]%
    commutes. If $f(x)=\point_{Y}$ for $x\neq\point_{X}$, the commutativity of this diagram then gives
    \begin{align*}
        1 &= \epsilon^{+}_{X}(x)\\
          &= \epsilon^{+}_{Y}(f(x))\\
          &= \epsilon^{+}_{Y}(\point_{Y})\\
          &= 0,
    \end{align*}
    which is a contradiction. Thus $f$ is an active morphism of pointed sets, so there exists a map $f^{-}$ such that $(f^{-})^{+}=f$ (\ChapterRef{\ChapterPointedSets, \cref{pointed-sets:properties-of-sets-with-deleted-basepoints-functoriality} of \cref{pointed-sets:properties-of-sets-with-deleted-basepoints}}{\cref{properties-of-sets-with-deleted-basepoints-functoriality} of \cref{properties-of-sets-with-deleted-basepoints}}).

    \ProofBox{Essential Surjectivity}%
    Let $(X,\Delta_{X},\epsilon_{X})$ be a comonoid in $\Sets_{*}$. We claim that
    \[
        \Delta_{X}(x)%
        =%
        x\wedge x%
    \]%
    for each $x\in X$ with $x\neq\point_{X}$. Indeed:
    \begin{itemize}
        \item Suppose that $x\neq\point_{X}$ and write $\Delta_{X}(x)=x_{1}\wedge x_{2}$.
        \item Since $\id_{X}\wedge\epsilon_{X}$ is pointed, we have
            \[
                [\id_{X}\wedge\epsilon_{X}](x_{1}\wedge x_{2})%
                =%
                \point_{X\wedge S^{0}}.%
            \]%
        \item The counitality condition for $\Delta_{X}$, corresponding to the commutativity of the diagram
            \[
                \begin{tikzcd}[row sep={5.0*\the\DL,between origins}, column sep={6.0*\the\DL,between origins}, background color=backgroundColor, ampersand replacement=\&]
                    X
                    \arrow[r,"\Delta_{X}"]
                    \arrow[rd,"\RUnitor^{\Sets_{*},-1}_{X}"'{pos=0.525}]
                    \&
                    X\wedge X
                    \arrow[d,"\id_{X}\wedge\epsilon_{X}"]
                    \\
                    \&
                    X\wedge S^{0}
                \end{tikzcd}
            \]%
            gives
            \begin{align*}
                x\wedge 1 &= \RUnitor^{\Sets_{*},-1}_{X}(x)\\
                          &= [\id_{X}\wedge\epsilon_{X}\circ\Delta_{X}](x)\\
                          &= [\id_{X}\wedge\epsilon_{X}](\Delta_{X}(x))\\
                          &= [\id_{X}\wedge\epsilon_{X}](x_{1}\wedge x_{2})\\
                          &= \point_{X\wedge S^{0}},%
            \end{align*}
            which is a contradiction. Thus $x_{1}\neq\point_{X}$.
        \item Similarly, if $x\neq\point_{X}$, then $x_{2}\neq\point_{X}$.
        \item Next, we claim that $\epsilon_{X}(x_{2})=1$, as otherwise we would have
            \begin{align*}
                \point_{X\wedge S^{0}} &= x_{1}\wedge 0\\
                                       &= [\id_{X}\wedge\epsilon_{X}](x_{1}\wedge x_{2})\\
                                       &= [\id_{X}\wedge\epsilon_{X}](\Delta_{X}(x))\\
                                       &= [\id_{X}\wedge\epsilon_{X}\circ\Delta_{X}](x)\\
                                       &= \RUnitor^{\Sets_{*},-1}_{X}(x)\\
                                       &= x\wedge 1,
            \end{align*}
            a contradiction. Thus $\epsilon_{X}(x_{2})=1$.
        \item Similarly, if $x\neq\point_{X}$, then $\epsilon_{X}(x_{1})=1$.
        \item Now, since $\Delta_{X}$ is counital, we have
            \begin{align*}
                x\wedge 1 &= \RUnitor^{\Sets_{*},-1}_{X}(x)\\
                          &= [\id_{X}\wedge\epsilon_{X}\circ\Delta_{X}](x)\\
                          &= [\id_{X}\wedge\epsilon_{X}](\Delta_{X}(x))\\
                          &= [\id_{X}\wedge\epsilon_{X}](x_{1}\wedge x_{2})\\
                          &= x_{1}\wedge 1,
            \end{align*}
            so $x=x_{1}$.
        \item Similarly, $x=x_{2}$, and we are done.
    \end{itemize}
    Next, notice that $X\cong\epsilon^{-1}_{X}(0)\icoprod\epsilon^{-1}_{X}(1)$, and let $x\in\epsilon^{-1}_{X}(0)$. We then have
    \begin{align*}
        [(\id_{X}\wedge\epsilon_{X})\circ\Delta_{X}](x) &= [\id_{X}\wedge\epsilon_{X}](x\wedge x)\\
                                                        &= x\wedge 0\\
                                                        &= \point_{X\wedge S^{0}}.
    \end{align*}
    The counitality condition for $\Delta_{X}$ then gives $x=\point_{X}$, so $\epsilon^{-1}_{X}(0)=\{\point_{X}\}$. Thus we have $(\epsilon^{-1}_{X}(1))^{+}\cong X$, and this isomorphism is compatible with the comonoid structures when equipping $\epsilon^{-1}_{X}(1)$ with its unique comonoid structure. This shows that $(-)^{+}$ is essentially surjective.

    \ProofBox{Equivalence}%
    Since $(-)^{+}$ is fully faithful and essentially surjective, it is an equivalence by \ChapterRef{\ChapterCategories, \cref{categories:properties-of-equivalences-of-categories-characterisations-b} of \cref{categories:properties-of-equivalences-of-categories-characterisations} of \cref{categories:properties-of-equivalences-of-categories}}{\cref{properties-of-equivalences-of-categories-characterisations-b} of \cref{properties-of-equivalences-of-categories-characterisations} of \cref{properties-of-equivalences-of-categories}}.
\end{Proof}
\section{Miscellany}\label{section-tensor-products-of-pointed-sets-miscellany}
\subsection{The Smash Product of a Family of Pointed Sets}\label{subsection-smash-products-of-pointed-sets-the-smash-product-of-a-family-of-pointed-sets}
Let $\{(X_{i},x^{i}_{0})\}_{i\in I}$ be a family of pointed sets.%
\begin{definition}{The Smash Product of a Family of Pointed Sets}{the-smash-product-of-a-family-of-pointed-sets}%
    The \index[set-theory]{smash product!of a family of pointed sets}\textbf{smash product of the family $\smash{\{(X_{i},x^{i}_{0})\}_{i\in I}}$} is the pointed set \index[notation]{wedgeXi@$\bigwedge_{i\in I}X_{i}$}$\bigwedge_{i\in I}X_{i}$ consisting of:
    \begin{itemize}
        \item\SloganFont{The Underlying Set. }The set $\bigwedge_{i\in I}X_{i}$ defined by%
            \[
                \bigwedge_{i\in I}X_{i}%
                \defeq%
                (\prod_{i\in I}X_{i})/\unsim,%
            \]%
            where $\unsim$ is the equivalence relation on $\prod_{i\in I}X_{i}$ obtained by declaring
            \[
                (x_{i})_{i\in I}
                \sim%
                (y_{i})_{i\in I}%
            \]%
            if there exist $i_{0}\in I$ such that $x_{i_{0}}=x_{0}$ and $y_{i_{0}}=y_{0}$, for each $(x_{i})_{i\in I},(y_{i})_{i\in I}\in\prod_{i\in I}X_{i}$.
        \item\SloganFont{The Basepoint. }The element $[(x_{0})_{i\in I}]$ of $\bigwedge_{i\in I}X_{i}$.
    \end{itemize}
\end{definition}
\begin{appendices}
\begin{multicols}{2}[\section{Other Chapters}]
\noindent
\textbf{Preliminaries}
\begin{enumerate}
\item \hyperref[introduction:section-phantom]{Introduction}
\end{enumerate}
\textbf{Sets}
\begin{enumerate}
\setcounter{enumi}{2}
\item \hyperref[sets:section-phantom]{Sets}
\item \hyperref[constructions-with-sets:section-phantom]{Constructions With Sets}
\item \hyperref[monoidal-structures-on-the-category-of-sets:section-phantom]{Monoidal Structures on the Category of Sets}
\item \hyperref[pointed-sets:section-phantom]{Pointed Sets}
\item \hyperref[tensor-products-of-pointed-sets:section-phantom]{Tensor Products of Pointed Sets}
\end{enumerate}
\textbf{Relations}
\begin{enumerate}
\setcounter{enumi}{6}
\item \hyperref[relations:section-phantom]{Relations}
\item \hyperref[constructions-with-relations:section-phantom]{Constructions With Relations}
\item \hyperref[conditions-on-relations:section-phantom]{Conditions on Relations}
\end{enumerate}
\textbf{Category Theory}
\begin{enumerate}
\setcounter{enumi}{9}
\item \hyperref[categories:section-phantom]{Categories}
\end{enumerate}
\textbf{Monoidal Categories}
\begin{enumerate}
\setcounter{enumi}{10}
\item \hyperref[constructions-with-monoidal-categories:section-phantom]{Constructions With Monoidal Categories}
\end{enumerate}
\textbf{Bicategories}
\begin{enumerate}
\setcounter{enumi}{11}
\item \hyperref[types-of-morphisms-in-bicategories:section-phantom]{Types of Morphisms in Bicategories}
\end{enumerate}
\textbf{Extra Part}
\begin{enumerate}
\setcounter{enumi}{12}
\item \hyperref[notes:section-phantom]{Notes}
\end{enumerate}
\end{multicols}

\end{appendices}
\end{document}
