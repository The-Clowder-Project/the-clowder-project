\input{preamble}

% OK, start here.
%
\usepackage{fontspec}
\let\hyperwhite\relax
\let\hyperred\relax
\newcommand{\hyperwhite}{\hypersetup{citecolor=white,filecolor=white,linkcolor=white,urlcolor=white}}
\newcommand{\hyperred}{%
\hypersetup{%
    citecolor=TitlingRed,%
    filecolor=TitlingRed,%
    linkcolor=TitlingRed,%
     urlcolor=TitlingRed%
}}
\let\ChapterRef\relax
\newcommand{\ChapterRef}[2]{#1}
\setcounter{tocdepth}{2}
%▓▓▓▓▓▓▓▓▓▓▓▓▓▓▓▓▓▓▓▓▓▓▓▓▓▓▓▓▓▓▓▓▓
%▓▓ ╔╦╗╦╔╦╗╦  ╔═╗  ╔═╗╔═╗╔╗╔╔╦╗ ▓▓
%▓▓  ║ ║ ║ ║  ║╣   ╠╣ ║ ║║║║ ║  ▓▓
%▓▓  ╩ ╩ ╩ ╩═╝╚═╝  ╚  ╚═╝╝╚╝ ╩  ▓▓
%▓▓▓▓▓▓▓▓▓▓▓▓▓▓▓▓▓▓▓▓▓▓▓▓▓▓▓▓▓▓▓▓▓
%\usepackage{titlesec}
%▓▓▓▓▓▓▓▓▓▓▓▓▓▓▓▓▓▓▓▓▓▓▓▓▓▓▓▓▓▓▓▓▓▓▓▓▓▓▓▓▓▓▓▓▓▓▓▓▓▓▓▓▓▓▓
%▓▓ ╔╦╗╔═╗╔╗ ╦  ╔═╗  ╔═╗╔═╗  ╔═╗╔═╗╔╗╔╔╦╗╔═╗╔╗╔╔╦╗╔═╗ ▓▓
%▓▓  ║ ╠═╣╠╩╗║  ║╣   ║ ║╠╣   ║  ║ ║║║║ ║ ║╣ ║║║ ║ ╚═╗ ▓▓
%▓▓  ╩ ╩ ╩╚═╝╩═╝╚═╝  ╚═╝╚    ╚═╝╚═╝╝╚╝ ╩ ╚═╝╝╚╝ ╩ ╚═╝ ▓▓
%▓▓▓▓▓▓▓▓▓▓▓▓▓▓▓▓▓▓▓▓▓▓▓▓▓▓▓▓▓▓▓▓▓▓▓▓▓▓▓▓▓▓▓▓▓▓▓▓▓▓▓▓▓▓▓
\newcommand{\ChapterTableOfContents}{%
    \begingroup
    \addfontfeature{Numbers={Lining,Monospaced}}
    \hypersetup{hidelinks}\tableofcontents%
    \endgroup
}%

\makeatletter
\newcommand \DotFill {\leavevmode \cleaders \hb@xt@ .33em{\hss .\hss }\hfill \kern \z@}
\makeatother

\definecolor{ToCGrey}{rgb}{0.4,0.4,0.4}
\definecolor{mainColor}{rgb}{0.82745098,0.18431373,0.18431373}
\usepackage{titletoc}
\titlecontents{part}
[0.0em]
{\addvspace{1pc}\color{TitlingRed}\large\bfseries\text{Part }}
{\bfseries\textcolor{TitlingRed}{\contentslabel{0.0em}}\hspace*{1.35em}}
{}
{\textcolor{TitlingRed}{{\hfill\bfseries\contentspage\nobreak}}}
[]
\titlecontents{section}
[0.0em]
{\addvspace{1pc}}
{\color{black}\bfseries\textcolor{TitlingRed}{\contentslabel{0.0em}}\hspace*{1.35em}}
{}
{\textcolor{black}{\textbf{\DotFill}{\bfseries\contentspage\nobreak}}}
[]
\titlecontents{subsection}
[0.0em]
{}
{\hspace*{1.35em}\color{ToCGrey}{\contentslabel{0.0em}}\hspace*{2.1em}}
{}
{{\textcolor{ToCGrey}\DotFill}\textcolor{ToCGrey}{\contentspage}\nobreak}
[]
\usepackage{marginnote}
\renewcommand*{\marginfont}{\normalfont}
\usepackage{inconsolata}
\setmonofont{inconsolata}%
\let\ChapterRef\relax
\newcommand{\ChapterRef}[2]{#1}
\AtBeginEnvironment{subappendices}{%%
    \section*{\huge Appendices}%
}%

\begin{document}

\title{Pointed Sets}

\maketitle

\phantomsection
\label{section-phantom}

This chapter contains some foundational material on pointed sets.

\ChapterTableOfContents

\section{Pointed Sets}\label{section-pointed-sets}
\subsection{Foundations}\label{subsection-pointed-sets-foundations}
\begin{definition}{Pointed Sets}{pointed-sets}%
    A \index[set-theory]{pointed set}\textbf{pointed set}%
    %--- Begin Footnote ---%
    \footnote{%
        \SloganFont{Further Terminology: }In the context of monoids with zero as models for $\F_{1}$-algebras, pointed sets are viewed as \index[set-theory]{field with one element!module over}\textbf{$\F_{1}$-modules}.
        \par\vspace*{\TCBBoxCorrection}
    } %
    %---  End Footnote  ---%
    is equivalently:
    \begin{itemize}
        \item An $\E_{0}$-monoid in $(\NerveB(\Sets),\pt)$.
        \item A pointed object in $(\Sets,\pt)$.%
    \end{itemize}
\end{definition}
\begin{remark}{Unwinding \cref{pointed-sets}}{unwinding-pointed-sets}%
    In detail, a \textbf{pointed set} is a pair $(X,x_{0})$ consisting of:
    \begin{itemize}
        \item\SloganFont{The Underlying Set. }A set $X$, called the \textbf{underlying set of $(X,x_{0})$}.
        \item\SloganFont{The Basepoint. }A morphism
            \[
                [x_{0}]%
                \colon%
                \pt%
                \to%
                X%
            \]%
            in $\Sets$, determining an element $x_{0}\in X$, called the \textbf{basepoint of $X$}.
    \end{itemize}
\end{remark}
\begin{example}{The Zero Sphere}{the-zero-sphere}%
    The \index[set-theory]{pointed set!zero sphere@$0$-sphere}\textbf{$0$-sphere}%
    %--- Begin Footnote ---%
    \footnote{%
        \SloganFont{Further Terminology: }In the context of monoids with zero as models for $\F_{1}$-algebras, the $0$-sphere is viewed as the \index[set-theory]{pointed set!Fone@$\F_{1}$}\textbf{underlying pointed set of the field with one element}.
    } %
    %---  End Footnote  ---%
    is the pointed set \index[notation]{Szero@$S^{0}$}$\smash{(S^{0},0)}$%
    %--- Begin Footnote ---%
    \footnote{%
        \SloganFont{Further Notation: }In the context of monoids with zero as models for $\F_{1}$-algebras, $S^{0}$ is also denoted \index[notation]{Fone@$\F_{1}$}$(\F_{1},0)$.
        \par\vspace*{\TCBBoxCorrection}
    } %
    %---  End Footnote  ---%
    consisting of:
    \begin{itemize}
        \item\SloganFont{The Underlying Set. }The set $S^{0}$ defined by
            \[
                S^{0}
                \defeq
                \{0,1\}.
            \]%
        \item\SloganFont{The Basepoint. }The element $0$ of $S^{0}$.
    \end{itemize}
\end{example}
\begin{example}{The Trivial Pointed Set}{the-trivial-pointed-set}%
    The \index[set-theory]{pointed set!trivial}\textbf{trivial pointed set} is the pointed set \index[notation]{pt@$\pt$}$(\pt,\point)$ consisting of:
    \begin{itemize}
        \item\SloganFont{The Underlying Set. }The punctual set $\pt\defeq\{\point\}$.
        \item\SloganFont{The Basepoint. }The element $\point$ of $\pt$.
    \end{itemize}
\end{example}
\begin{example}{The Standard Pointed Set With $n+1$ Elements}{the-pointed-set-with-n-plus-one-elements}%
    The \index[set-theory]{pointed set!standard pointed set with $n+1$ elements}\textbf{standard pointed set with $n+1$ elements} is the pointed set \index[notation]{n@$\langle n\rangle$}$\langle n\rangle$ consisting of
    \begin{itemize}
        \item\SloganFont{The Underlying Set. }The set $\langle n\rangle$ defined by
            \[
                \langle n\rangle%
                \defeq%
                \{*\}\cup\{1,\ldots,n\}.%
            \]%
        \item\SloganFont{The Basepoint. }The element $*$ of $\langle n\rangle$.
    \end{itemize}
\end{example}
\subsection{Morphisms of Pointed Sets}\label{subsection-morphisms-of-pointed-sets}
\begin{definition}{Morphisms of Pointed Sets}{morphisms-of-pointed-sets}%
    A \index[set-theory]{morphism of pointed sets}\index[set-theory]{pointed set!morphism of}\textbf{morphism of pointed sets}%
    %--- Begin Footnote ---%
    \footnote{%
        \SloganFont{Further Terminology: }Also called a \index[set-theory]{pointed function|see {pointed set, morphism of}}\textbf{pointed function}.
    }%
    %---  End Footnote  ---%
    %--- Begin Footnote ---%
    \footnote{%
        \SloganFont{Further Terminology: }In the context of monoids with zero as models for $\F_{1}$-algebras, morphisms of pointed sets are also called \index[set-theory]{field with one element!module over, morphism of}\textbf{morphism of $\F_{1}$-modules}.
        \par\vspace*{\TCBBoxCorrection}
    } %
    %---  End Footnote  ---%
    is equivalently:
    \begin{itemize}
        \item A morphism of $\E_{0}$-monoids in $(\NerveB(\Sets),\pt)$.%
        \item A morphism of pointed objects in $(\Sets,\pt)$.%
    \end{itemize}
\end{definition}
\begin{remark}{Unwinding \cref{morphisms-of-pointed-sets}}{unwinding-morphisms-of-pointed-sets}%
    In detail, a \textbf{morphism of pointed sets} $f\colon(X,x_{0})\to(Y,y_{0})$ is a morphism of sets $f\colon X\to Y$ such that the diagram
    \[
        \begin{tikzcd}[row sep={4.0*\the\DL,between origins}, column sep={2.5*\the\DL,between origins}, background color=backgroundColor, ampersand replacement=\&]
            \&
            \pt
            \arrow[ld,"{[x_{0}]}"']
            \arrow[rd,"{[y_{0}]}"]
            \&
            \\
            X
            \arrow[rr,"f"']
            \&
            \&
            Y
        \end{tikzcd}
    \]%
    commutes, i.e.\ such that
    \[
        f(x_{0})
        =
        y_{0}.
    \]%
\end{remark}
\subsection{The Category of Pointed Sets}\label{subsection-the-category-of-pointed-sets}
\begin{definition}{The Category of Pointed Sets}{the-category-of-pointed-sets}%
    The \index[set-theory]{pointed set!category of}\textbf{category of pointed sets} is the category \index[notation]{Setsstar@$\Sets_{*}$}$\Sets_{*}$ defined equivalently as:
    \begin{itemize}
        \item The homotopy category of the $\infty$-category $\Mon_{\E_{0}}(\NerveB(\Sets),\pt)$ of \ChapterRef{\ChapterMonoidsInMonoidalInftyCategories, \cref{monoids-in-symmetric-monoidal-infty-categories:the-infty-category-of-e-k-monoids-over-an-ek-monoid-in-a-symmetric-monoidal-infty-category}}{\cref{the-infty-category-of-e-k-monoids-over-an-ek-monoid-in-a-symmetric-monoidal-infty-category}}.
        \item The category $\Sets_{*}$ of \ChapterRef{\ChapterConstructionsWithCategories, \cref{constructions-with-categories:the-category-of-pointed-objects-in-a-category-with-a-terminal-object}}{\cref{the-category-of-pointed-objects-in-a-category-with-a-terminal-object}}.
    \end{itemize}
\end{definition}
\begin{remark}{Unwinding \cref{the-category-of-pointed-sets}}{unwinding-the-category-of-pointed-sets}%
    In detail, the \textbf{category of pointed sets} is the category \index[notation]{Setsstar@$\Sets_{*}$}$\Sets_{*}$ where:
    \begin{itemize}
        \item\SloganFont{Objects. }The objects of $\Sets_{*}$ are pointed sets.
        \item\SloganFont{Morphisms. }The morphisms of $\Sets_{*}$ are morphisms of pointed sets.
        \item\SloganFont{Identities. }For each $(X,x_{0})\in\Obj(\Sets_{*})$, the unit map
            \[
                \Unit^{\Sets_{*}}_{(X,x_{0})}
                \colon
                \pt
                \to
                \Sets_{*}((X,x_{0}),(X,x_{0}))
            \]%
            of $\Sets_{*}$ at $(X,x_{0})$ is defined by%
            %--- Begin Footnote ---%
            \footnote{%
                Note that $\id_{X}$ is indeed a morphism of pointed sets, as we have $\id_{X}(x_{0})=x_{0}$.
            }%
            %---  End Footnote  ---%
            \[
                \id^{\Sets_{*}}_{(X,x_{0})}
                \defeq
                \id_{X}.
            \]%
        \item\SloganFont{Composition. }For each $(X,x_{0}),(Y,y_{0}),(Z,z_{0})\in\Obj(\Sets_{*})$, the composition map
            \begin{envscriptsize}
                \[
                    \circ^{\Sets_{*}}_{(X,x_{0}),(Y,y_{0}),(Z,z_{0})}
                    \colon
                    \Sets_{*}((Y,y_{0}),(Z,z_{0}))
                    \times
                    \Sets_{*}((X,x_{0}),(Y,y_{0}))
                    \to
                    \Sets_{*}((X,x_{0}),(Z,z_{0}))
                \]%
            \end{envscriptsize}
            of $\Sets_{*}$ at $((X,x_{0}),(Y,y_{0}),(Z,z_{0}))$ is defined by%
            %--- Begin Footnote ---%
            \footnote{%
                Note that the composition of two morphisms of pointed sets is indeed a morphism of pointed sets, as we have
                \begin{webcompile}
                    \begin{aligned}
                        g(f(x_{0})) &= g(y_{0})\\
                                    &= z_{0},
                    \end{aligned}
                    \qquad
                    \begin{tikzcd}[row sep={5.0*\the\DL,between origins}, column sep={4.5*\the\DL,between origins}, background color=backgroundColor, ampersand replacement=\&]
                        \&
                        \pt
                        \arrow[ld,"{[x_{0}]}"']
                        \arrow[d,"{[y_{0}]}"description]
                        \arrow[rd,"{[z_{0}]}"]
                        \&
                        \\
                        X
                        \arrow[r,"f"']
                        \&
                        Y
                        \arrow[r,"g"']
                        \&
                        Z
                    \end{tikzcd}
                \end{webcompile}
                \par\vspace*{\TCBBoxCorrection}
            }%
            %---  End Footnote  ---%
            \[
                g\mathbin{{\circ}^{\Sets_{*}}_{(X,x_{0}),(Y,y_{0}),(Z,z_{0})}}f
                \defeq
                g\circ f.
            \]%
    \end{itemize}
\end{remark}
\subsection{Elementary Properties of Pointed Sets}\label{subsection-elementary-properties-of-pointed-sets}
\begin{proposition}{Elementary Properties of Pointed Sets}{elementary-properties-of-pointed-sets}%
    Let $(X,x_{0})$ be a pointed set.
    \begin{enumerate}
        \item\label{elementary-properties-of-pointed-sets-completeness}\SloganFont{Completeness. }The category $\Sets_{*}$ of pointed sets and morphisms between them is complete, having in particular:
            \begin{enumerate}
                \item\label{elementary-properties-of-pointed-sets-completeness-products}Products, described as in \cref{products-of-pointed-sets}.
                \item\label{elementary-properties-of-pointed-sets-completeness-pullbacks}Pullbacks, described as in \cref{pullbacks-of-pointed-sets}.
                \item\label{elementary-properties-of-pointed-sets-completeness-equalisers}Equalisers, described as in \cref{equalisers-of-pointed-sets}.
            \end{enumerate}
        \item\label{elementary-properties-of-pointed-sets-cocompleteness}\SloganFont{Cocompleteness. }The category $\Sets_{*}$ of pointed sets and morphisms between them is cocomplete, having in particular:
            \begin{enumerate}
                \item\label{elementary-properties-of-pointed-sets-completeness-coproduct}Coproducts, described as in \cref{coproducts-of-pointed-sets}.
                \item\label{elementary-properties-of-pointed-sets-completeness-pushouts}Pushouts, described as in \cref{pushouts-of-pointed-sets};
                \item\label{elementary-properties-of-pointed-sets-completeness-coequalisers}Coequalisers, described as in \cref{coequalisers-of-pointed-sets}.
            \end{enumerate}
        \item\label{elementary-properties-of-pointed-sets-failure-to-be-cartesian-closed}\SloganFont{Failure To Be Cartesian Closed. }The category $\Sets_{*}$ is not Cartesian closed.%
            %--- Begin Footnote ---%
            \footnote{%
                The category $\Sets_{*}$ does admit a natural monoidal closed structure, however; see \ChapterTensorProductsOfPointedSets.
            }%
            %---  End Footnote  ---%
        \item\label{elementary-properties-of-pointed-sets-morphisms-from-the-monoidal-unit}\SloganFont{Morphisms From the Monoidal Unit. }We have a bijection of sets%
            %--- Begin Footnote ---%
            \footnote{%
                In other words, the forgetful functor
                \[%
                    \Wasureru%
                    \colon%
                    \Sets_{*}%
                    \to%
                    \Sets%
                \]%
                defined on objects by sending a pointed set to its underlying set is corepresentable by $S^{0}$.
                \par\vspace*{\TCBBoxCorrection}
            }%
            %---  End Footnote  ---%
            \[
                \Sets_{*}(S^{0},X)
                \cong
                X,
            \]%
            natural in $(X,x_{0})\in\Obj(\Sets_{*})$, internalising also to an isomorphism of pointed sets
            \[
                \eSets_{*}(S^{0},X)
                \cong
                (X,x_{0}),
            \]%
            again natural in $(X,x_{0})\in\Obj(\Sets_{*})$.
        \item\label{elementary-properties-of-pointed-sets-relation-to-partial-functions}\SloganFont{Relation to Partial Functions. }We have an equivalence of categories%
            %--- Begin Footnote ---%
            \footnote{%
                \textdbend\SloganFont{Warning: }This is not an isomorphism of categories, only an equivalence.
                \par\vspace*{\TCBBoxCorrection}
            }%
            %---  End Footnote  ---%
            \[
                \Sets_{*}%
                \eqcong%
                \Sets^{\mathrm{part.}}
            \]%
            between the category of pointed sets and pointed functions between them and the category of sets and partial functions between them, where:
            \begin{enumerate}
                \item\label{elementary-properties-of-pointed-sets-relation-to-partial-functions-a}\SloganFont{From Pointed Sets to Sets With Partial Functions. }The equivalence
                    \[
                        \xi%
                        \colon%
                        \Sets_{*}%
                        \righteqarrow%
                        \Sets^{\mathrm{part.}}%
                    \]%
                    sends:
                    \begin{enumerate}
                        \item\label{elementary-properties-of-pointed-sets-relation-to-partial-functions-a-i}A pointed set $(X,x_{0})$ to $X$.
                        \item\label{elementary-properties-of-pointed-sets-relation-to-partial-functions-a-ii}A pointed function
                            \[
                                f%
                                \colon%
                                (X,x_{0})%
                                \to%
                                (Y,y_{0})%
                            \]%
                             to the partial function
                            \[
                                \xi_{f}%
                                \colon%
                                X%
                                \to%
                                Y%
                            \]%
                            defined on $f^{-1}(Y\setminus y_{0})$ and given by
                            \[
                                \xi_{f}(x)%
                                \defeq%
                                f(x)%
                            \]%
                            for each $x\in f^{-1}(Y\setminus y_{0})$.
                    \end{enumerate}
                \item\label{elementary-properties-of-pointed-sets-relation-to-partial-functions-b}\SloganFont{From Sets With Partial Functions to Pointed Sets. }The equivalence
                    \[
                        \xi^{-1}%
                        \colon%
                        \Sets^{\mathrm{part.}}%
                        \righteqarrow%
                        \Sets_{*}%
                    \]%
                    sends:
                    \begin{enumerate}
                        \item\label{elementary-properties-of-pointed-sets-relation-to-partial-functions-b-i}A set $X$ is to the pointed set $(X,\star)$ with $\star$ an element that is not in $X$.
                        \item\label{elementary-properties-of-pointed-sets-relation-to-partial-functions-b-ii}A partial function
                            \[
                                f%
                                \colon%
                                X%
                                \to%
                                Y%
                            \]%
                            defined on $U\subset X$ to the pointed function
                            \[
                                \xi^{-1}_{f}%
                                \colon%
                                (X,x_{0})%
                                \to%
                                (Y,y_{0})%
                            \]%
                            defined by
                            \[
                                \xi_{f}(x)%
                                \defeq%
                                \begin{cases}
                                    f(x)  &\text{if $x\in U$,}\\
                                    y_{0} &\text{otherwise.}
                                \end{cases}
                            \]%
                            for each $x\in X$.
                    \end{enumerate}
            \end{enumerate}
        %\item\label{elementary-properties-of-pointed-sets-}\SloganFont{. }
    \end{enumerate}
\end{proposition}
\begin{Proof}{Proof of \cref{elementary-properties-of-pointed-sets}}%
    \FirstProofBox{\cref{elementary-properties-of-pointed-sets-completeness}: Completeness}%
    This follows from (the proofs) of \cref{products-of-pointed-sets,pullbacks-of-pointed-sets,equalisers-of-pointed-sets} and \ChapterRef{\ChapterLimitsAndColimits, \cref{limits-and-colimits:TODO}}{\cref{TODO}}.

    \ProofBox{\cref{elementary-properties-of-pointed-sets-cocompleteness}: Cocompleteness}%
    This follows from (the proofs) of \cref{coproducts-of-pointed-sets,pushouts-of-pointed-sets,coequalisers-of-pointed-sets} and \ChapterRef{\ChapterLimitsAndColimits, \cref{limits-and-colimits:TODO}}{\cref{TODO}}.

    \ProofBox{\cref{elementary-properties-of-pointed-sets-failure-to-be-cartesian-closed}: Failure To Be Cartesian Closed}%
    See \cite{MSE2855868}.

    \ProofBox{\cref{elementary-properties-of-pointed-sets-morphisms-from-the-monoidal-unit}: Morphisms From the Monoidal Unit}%
    Since a morphism from $S^{0}$ to a pointed set $(X,x_{0})$ sends $0\in S^{0}$ to $x_{0}$ and then can send $1\in S^{0}$ to any element of $X$, we obtain a bijection between pointed maps $S^{0}\to X$ and the elements of $X$.

    The isomorphism then
    \[
        \eSets_{*}(S^{0},X)
        \cong
        (X,x_{0})
    \]%
    follows by noting that $\Delta_{x_{0}}\colon S^{0}\to X$, the basepoint of $\eSets_{*}(S^{0},X)$, corresponds to the pointed map $S^{0}\to X$ picking the element $x_{0}$ of $X$, and thus we see that the bijection between pointed maps $S^{0}\to X$ and elements of $X$ is compatible with basepoints, lifting to an isomorphism of pointed sets.

    \ProofBox{\cref{elementary-properties-of-pointed-sets-relation-to-partial-functions}: Relation to Partial Functions}%
    See \cite{MSE884460}.
\end{Proof}
\subsection{Active and Inert Morphisms of Pointed Sets}\label{subsection-active-and-inert-morphisms-of-pointed-sets}
\begin{definition}{Active and Inert Morphisms of Pointed Sets}{active-and-inert-morphisms-of-pointed-sets}%
    Let $f\colon(X,x_{0})\to(Y,y_{0})$ be a morphism of pointed sets.
    \begin{enumerate}
        \item\label{active-and-inert-morphisms-of-pointed-sets-active-morphisms-of-pointed-sets}The morphism $f$ is \index[set-theory]{morphism of pointed sets!active}\textbf{active} if $f^{-1}(y_{0})=x_{0}$.
        \item\label{active-and-inert-morphisms-of-pointed-sets-inert-morphisms-of-pointed-sets}The morphism $f$ is \index[set-theory]{morphism of pointed sets!inert}\textbf{inert} if, for each $y\in Y$, the set $f^{-1}(y)$ has exactly one element.
    \end{enumerate}
\end{definition}
\begin{notation}{The Category of Pointed Sets and Active Morphisms}{the-category-of-pointed-sets-and-active-morphisms}%
    We write \index[notation]{Setsactivestar@$\Sets^{\actv}_{*}$}$\Sets^{\actv}_{*}$ for the wide subcategory of $\Sets_{*}$ spanned by pointed sets and the active maps between them.
\end{notation}
\begin{example}{Examples of Active and Inert Maps of Pointed Sets}{examples-of-active-and-inert-morphisms-of-pointed-sets}%
    Here are some examples of active and inert maps of pointed sets.
    \begin{enumerate}
        \item\label{examples-of-active-and-inert-morphisms-of-pointed-sets-multiplication}The map $\mu\colon\langle 2\rangle\to\langle 1\rangle$ given by
            \[
                \begin{tikzcd}[row sep={2.0*\the\DL,between origins}, column sep={5.0*\the\DL,between origins}, background color=backgroundColor, ampersand replacement=\&]
                    1
                    \&
                    1
                    \\
                    2
                    \&
                    \\
                    *
                    \&
                    *
                    % 1-Arrows
                    \arrow[from=1-1,to=1-2,mapsto]%
                    \arrow[from=2-1,to=1-2,mapsto]%
                    \arrow[from=3-1,to=3-2,mapsto]%
                \end{tikzcd}
            \]%
            is active but not inert.
        \item\label{examples-of-active-and-inert-morphisms-of-pointed-sets-2-to-2-send-2-to-basepoint}The map $f\colon\langle 2\rangle\to\langle 2\rangle$ given by
            \[
                \begin{tikzcd}[row sep={2.0*\the\DL,between origins}, column sep={5.0*\the\DL,between origins}, background color=backgroundColor, ampersand replacement=\&]
                    1
                    \&
                    1
                    \\
                    2
                    \&
                    2
                    \\
                    *
                    \&
                    *
                    % 1-Arrows
                    \arrow[from=1-1,to=1-2,mapsto]%
                    \arrow[from=2-1,to=3-2,mapsto]%
                    \arrow[from=3-1,to=3-2,mapsto]%
                \end{tikzcd}
            \]%
            is inert but not active.
        \item\label{examples-of-active-and-inert-morphisms-of-pointed-sets-factoring-maps}The map $f\colon\langle 3\rangle\to\langle 1\rangle$ given by
            \[
                \begin{tikzcd}[row sep={2.0*\the\DL,between origins}, column sep={5.0*\the\DL,between origins}, background color=backgroundColor, ampersand replacement=\&]
                    1
                    \&
                    1
                    \\
                    2
                    \&
                    \\
                    3
                    \&
                    \\
                    *
                    \&
                    *
                    % 1-Arrows
                    \arrow[from=1-1,to=1-2,mapsto]%
                    \arrow[from=2-1,to=1-2,mapsto]%
                    \arrow[from=3-1,to=4-2,mapsto]%
                    \arrow[from=4-1,to=4-2,mapsto]%
                \end{tikzcd}
            \]%
            is neither inert nor active. However, it factors as $f=a\circ i$, where
            \begin{align*}
                i &\colon \langle 3\rangle\to\langle 2\rangle,\\
                a &\colon \langle 2\rangle\to\langle 1\rangle
            \end{align*}
            are the morphisms of pointed sets given by
            \begin{webcompile}
                \begin{tikzcd}[row sep={2.0*\the\DL,between origins}, column sep={5.0*\the\DL,between origins}, background color=backgroundColor, ampersand replacement=\&]
                    1
                    \&
                    1
                    \\
                    2
                    \&
                    2
                    \\
                    3
                    \&
                    \\
                    *
                    \&
                    *
                    % 1-Arrows
                    \arrow[from=1-1,to=1-2,mapsto]%
                    \arrow[from=2-1,to=2-2,mapsto]%
                    \arrow[from=3-1,to=4-2,mapsto]%
                    \arrow[from=4-1,to=4-2,mapsto]%
                \end{tikzcd}
                \quad
                \begin{tikzcd}[row sep={2.0*\the\DL,between origins}, column sep={5.0*\the\DL,between origins}, background color=backgroundColor, ampersand replacement=\&]
                    1
                    \&
                    1
                    \\
                    2
                    \&
                    \\
                    *
                    \&
                    *\mrp{,}
                    \\
                    \&
                    % 1-Arrows
                    \arrow[from=1-1,to=1-2,mapsto]%
                    \arrow[from=2-1,to=1-2,mapsto]%
                    \arrow[from=3-1,to=3-2,mapsto]%
                \end{tikzcd}
            \end{webcompile}
            with $i$ being inert and $a$ being active.
    \end{enumerate}
\end{example}
\begin{proposition}{Properties of Active and Inert Maps of Pointed Sets}{properties-of-active-and-inert-morphisms-of-pointed-sets}%
    Let $(X,x_{0})$ and $(Y,y_{0})$ be pointed sets.
    \begin{enumerate}
        \item\label{properties-of-active-and-inert-morphisms-of-pointed-sets-active-inert-factorisation}\SloganFont{Active-Inert Factorisation. }Every morphism of pointed sets $f\colon(X,x_{0})\to(Y,y_{0})$ factors uniquely as
            \[
                f%
                =%
                a\circ i,%
            \]%
            where:
            \begin{enumerate}
                \item\label{properties-of-active-and-inert-morphisms-of-pointed-sets-active-inert-factorisation-1}The map $i\colon(X,x_{0})\to(K,k_{0})$ is an inert morphism of pointed sets
                \item\label{properties-of-active-and-inert-morphisms-of-pointed-sets-active-inert-factorisation-2}The map $a\colon(K,k_{0})\to(Y,y_{0})$ is an active morphism of pointed sets.
            \end{enumerate}
            Moreover, this determines an orthogonal factorisation system in $\Sets_{*}$.
        %\item\label{properties-of-active-and-inert-morphisms-of-pointed-sets-}\SloganFont{. }
    \end{enumerate}
\end{proposition}
\begin{Proof}{Proof of \cref{properties-of-active-and-inert-morphisms-of-pointed-sets}}%
    \FirstProofBox{\cref{properties-of-active-and-inert-morphisms-of-pointed-sets-active-inert-factorisation}: Active-Inert Factorisation}%
    Let $f\colon X\to Y$ be a morphism of pointed sets. We can factor $f$ as
    \[
        \begin{tikzcd}[row sep={3.0*\the\DL,between origins}, column sep={3.0*\the\DL,between origins}, background color=backgroundColor, ampersand replacement=\&]
            X
            \arrow[r,"i"]
            \&
            K
            \arrow[r,"a"]
            \&
            Y\mrp{,}
        \end{tikzcd}
    \]%
    where:
    \begin{itemize}
        \item $K$ is the pointed set given by
            \begin{align*}
                K &= \{x\in X\ \middle|\ f(x)\neq y_{0}\}\cup\{x_{0}\}\\
                  &= (X\setminus f^{-1}(y_{0}))\cup\{x_{0}\};
            \end{align*}
        \item $i\colon X\to K$ is the inert morphism of pointed sets given by
            \[
                i(x)%
                \defeq%
                \begin{cases}
                    x     &\text{if $x\in K$},\\%
                    x_{0} &\text{otherwise}
                \end{cases}
            \]%
            for each $x\in X$;
        \item $a\colon K\to Y$ is the active morphism of pointed sets given by
            \[
                a(x)%
                \defeq%
                f(x)%
            \]%
            for each $x\in K$.
    \end{itemize}
    Next, let
    \[
        \begin{tikzcd}[row sep={5.0*\the\DL,between origins}, column sep={5.0*\the\DL,between origins}, background color=backgroundColor, ampersand replacement=\&]
            X
            \arrow[r,"i"]
            \arrow[d,"f"']
            \&
            Y
            \arrow[d,"g"]
            \\
            A
            \arrow[r,"a"']
            \&
            B
        \end{tikzcd}
    \]%
    be a commutative diagram in $\Sets_{*}$. Consider the morphism $\phi\colon Y\to A$ given by
    \[
        \phi(y)%
        =%
        f(i^{-1}(y))
    \]%
    for each $y\in Y$ (which is well-defined since, as $i$ is inert, $i^{-1}(y)$ is a singleton for all $y\in Y$). We claim that $\phi$ is the unique diagonal filler in the diagram
    \[
        \begin{tikzcd}[row sep={5.0*\the\DL,between origins}, column sep={5.0*\the\DL,between origins}, background color=backgroundColor, ampersand replacement=\&]
            X
            \arrow[r,"i"]
            \arrow[d,"f"']
            \&
            Y
            \arrow[d,"g"]
            \arrow[ld,"\exists!"',dashed]
            \arrow[ld,"\phi",dashed]
            \\
            A
            \arrow[r,"a"']
            \&
            B\mrp{.}
        \end{tikzcd}
    \]%
    Indeed, this diagram commutes, as we have
    \begin{align*}
        [\phi\circ i](x) &\defeq \phi(i(x))\\
                         &\defeq f(i^{-1}(i(x)))\\
                         &=      f(x)
    \end{align*}
    for each $x\in X$ and
    \begin{align*}
        [a\circ\phi](y) &\defeq a(\phi(y))\\
                        &\defeq a(f(i^{-1}(y)))\\
                        &\defeq [a\circ f](i^{-1}(y))\\
                        &=      [g\circ i](i^{-1}(y))\\
                        &\defeq g(i(i^{-1}(y)))\\
                        &\defeq g(y)
    \end{align*}
    for each $y\in Y$. Moreover, given another morphism $\psi$ such that the diagram
    \[
        \begin{tikzcd}[row sep={5.0*\the\DL,between origins}, column sep={5.0*\the\DL,between origins}, background color=backgroundColor, ampersand replacement=\&]
            X
            \arrow[r,"i"]
            \arrow[d,"f"']
            \&
            Y
            \arrow[d,"g"]
            \arrow[ld,"\psi"description,dashed]
            \\
            A
            \arrow[r,"a"']
            \&
            B
        \end{tikzcd}
    \]%
    commutes, it follows that we must have $\psi=\phi$, since, given $y\in Y$, there exists a unique $x\in X$ such that $i(x)=y$, so we have
    \begin{align*}
        \psi(y) &=      \psi(i(x))\\
                &=      f(x)\\
                &=      f(i^{-1}(y))\\
                &\defeq \phi(y).
    \end{align*}
    This finishes the proof.
\end{Proof}
\section{Limits of Pointed Sets}\label{section-limits-of-pointed-sets}
\subsection{The Terminal Pointed Set}\label{subsection-the-terminal-pointed-set}
\begin{definition}{The Terminal Pointed Set}{the-terminal-pointed-set}%
    The \index[set-theory]{terminal pointed set}\textbf{terminal pointed set} is the terminal object of $\Sets_{*}$ as in \ChapterRef{\ChapterLimitsAndColimits, \cref{limits-and-colimits:terminal-objects}}{\cref{terminal-objects}}.
\end{definition}
\begin{construction}{Construction of the Terminal Pointed Set}{construction-of-the-terminal-pointed-set}%
    Concretely, the \index[set-theory]{terminal pointed set}\textbf{terminal pointed set} is the pair \index[notation]{pt@$\pt$}$\smash{((\pt,\point),\{!_{X}\}_{(X,x_{0})\in\Obj(\Sets_{*})})}$ consisting of:
    \begin{itemize}
        \item\SloganFont{The Limit. }The pointed set $(\pt,\point)$.
        \item\SloganFont{The Cone. }The collection of morphisms of pointed sets
            \[
                \{%
                    !_{X}%
                    \colon%
                    (X,x_{0})%
                    \to%
                    (\pt,\point)%
                \}_{(X,x_{0})\in\Obj(\Sets)}%
            \]%
            defined by
            \[
                !_{X}(x)%
                \defeq%
                \point%
            \]%
            for each $x\in X$ and each $(X,x_{0})\in\Obj(\Sets)$.
    \end{itemize}
\end{construction}
\begin{Proof}{Proof of \cref{construction-of-the-terminal-pointed-set}}%
    We claim that $(\pt,\point)$ is the terminal object of $\Sets_{*}$. Indeed, suppose we have a diagram of the form
    \[
        \begin{tikzcd}[row sep={5.0*\the\DL,between origins}, column sep={6.0*\the\DL,between origins}, background color=backgroundColor, ampersand replacement=\&]
            {(X,x_{0})}
            \&
            {(\pt,\point)}
        \end{tikzcd}
    \]%
    in $\Sets_{*}$. Then there exists a unique morphism of pointed sets
    \[
        \phi%
        \colon%
        (X,x_{0})%
        \to%
        (\pt,\point)%
    \]%
    making the diagram
    \[
        \begin{tikzcd}[row sep={5.0*\the\DL,between origins}, column sep={6.0*\the\DL,between origins}, background color=backgroundColor, ampersand replacement=\&]
            {(X,x_{0})}
            \arrow[r,"\phi"{pos=0.425},"\exists!"'{pos=0.425}, dashed]
            \&
            {(\pt,\point)}
        \end{tikzcd}
    \]%
    commute, namely $!_{X}$.
\end{Proof}
\subsection{Products of Families of Pointed Sets}\label{subsection-products-of-families-of-pointed-sets}
Let $\{(X_{i},x^{i}_{0})\}_{i\in I}$ be a family of pointed sets.%
\begin{definition}{The Product of a Family of Pointed Sets}{the-product-of-a-family-of-pointed-sets}%
    The \index[set-theory]{product of a family of pointed sets}\textbf{product of $\smash{\{(X_{i},x^{i}_{0})\}_{i\in I}}$} is the product of $\{(X_{i},x^{i}_{0})\}_{i\in I}$ in $\Sets_{*}$ as in \ChapterRef{\ChapterLimitsAndColimits, \cref{limits-and-colimits:the-product-of-a-family-of-objects}}{\cref{the-product-of-a-family-of-objects}}.
\end{definition}
\begin{construction}{Construction of the Product of a Family of Pointed Sets}{construction-of-the-product-of-a-family-of-pointed-sets}%
    Concretely, the \index[set-theory]{product of a family of pointed sets}\textbf{product of $\smash{\{(X_{i},x^{i}_{0})\}_{i\in I}}$} is the pair \index[notation]{prodiiniai@$\prod_{i\in I}A_{i}$}$\smash{((\prod_{i\in I}X_{i},(x^{i}_{0})_{i\in I}),\{\pr_{i}\}_{i\in I})}$ consisting of:
    \begin{itemize}
        \item\SloganFont{The Limit. }The pointed set $(\prod_{i\in I}X_{i},(x^{i}_{0})_{i\in I})$.%
        \item\SloganFont{The Cone. }The collection
            \[
                \{%
                    \pr_{i}
                    \colon%
                    (\prod_{i\in I}X_{i},(x^{i}_{0})_{i\in I})%
                    \to%
                    (X_{i},x^{i}_{0})%
                \}_{i\in I}%
            \]%
            of maps given by
            \[
                \pr_{i}((x_{j})_{j\in I})%
                \defeq%
                x_{i}%
            \]%
            for each $(x_{j})_{j\in I}\in\prod_{i\in I}X_{i}$ and each $i\in I$.
    \end{itemize}
\end{construction}
\begin{Proof}{Proof of \cref{construction-of-the-product-of-a-family-of-pointed-sets}}%
    We claim that $\smash{(\prod_{i\in I}X_{i},(x^{i}_{0})_{i\in I})}$ is the categorical product of $\smash{\{(X_{i},x^{i}_{0})\}_{i\in I}}$ in $\Sets_{*}$. Indeed, suppose we have, for each $i\in I$, a diagram of the form
    \[
        \begin{tikzcd}[row sep={5.0*\the\DL,between origins}, column sep={8.0*\the\DL,between origins}, background color=backgroundColor, ampersand replacement=\&]
            {(P,*)}
            \arrow[rd,"p_{i}"]
            \&
            \\
            {(\prod_{i\in I}X_{i},(x^{i}_{0})_{i\in I})}
            \arrow[r,"\pr_{i}"']
            \&
            {(X_{i},x^{i}_{0})}
        \end{tikzcd}
    \]%
    in $\Sets_{*}$. Then there exists a unique morphism of pointed sets
    \[
        \phi%
        \colon%
        (P,*)%
        \to%
        (\prod_{i\in I}X_{i},(x^{i}_{0})_{i\in I})%
    \]%
    making the diagram
    \[
        \begin{tikzcd}[row sep={5.0*\the\DL,between origins}, column sep={8.0*\the\DL,between origins}, background color=backgroundColor, ampersand replacement=\&]
            {(P,*)}
            \arrow[rd,"p_{i}"]
            \arrow[d,"\phi"',"\exists!",dashed]
            \&
            \\
            {(\prod_{i\in I}X_{i},(x^{i}_{0})_{i\in I})}
            \arrow[r,"\pr_{i}"']
            \&
            {(X_{i},x^{i}_{0})}
        \end{tikzcd}
    \]%
    commute, being uniquely determined by the condition $\pr_{i}\circ\phi=p_{i}$ for each $i\in I$ via
    \[
        \phi(x)%
        =%
        (p_{i}(x))_{i\in I}
    \]%
    for each $x\in P$. Note that this is indeed a morphism of pointed sets, as we have
    \begin{align*}
        \phi(*) &= (p_{i}(*))_{i\in I}\\%
                &= (x^{i}_{0})_{i\in I},%
    \end{align*}
    where we have used that $p_{i}$ is a morphism of pointed sets for each $i\in I$.
\end{Proof}
\begin{proposition}{Properties of Products of Families of Pointed Sets}{properties-of-products-of-families-of-pointed-sets}%
    Let $\{(X_{i},x^{i}_{0})\}_{i\in I}$ be a family of pointed sets.%
    \begin{enumerate}
        \item\label{properties-of-products-of-families-of-pointed-sets-functoriality}\SloganFont{Functoriality. }The assignment $\{(X_{i},x^{i}_{0})\}_{i\in I}\mapsto(\prod_{i\in I}X_{i},(x^{i}_{0})_{i\in I})$ defines a functor
            \[
                \prod_{i\in I}%
                \colon%
                \Fun(I_{\disc},\Sets_{*})%
                \to%
                \Sets_{*}.%
            \]%
        %\item\label{properties-of-products-of-families-of-pointed-sets-}\SloganFont{. }
    \end{enumerate}
\end{proposition}
\begin{Proof}{Proof of \cref{properties-of-products-of-families-of-pointed-sets}}%
    \FirstProofBox{\cref{properties-of-products-of-families-of-pointed-sets-functoriality}: Functoriality}%
    This follows from \ChapterRef{\ChapterLimitsAndColimits, \cref{limits-and-colimits:properties-of-co-limits-functoriality} of \cref{limits-and-colimits:properties-of-co-limits}}{\cref{properties-of-co-limits-functoriality} of \cref{properties-of-co-limits}}.
\end{Proof}
\subsection{Products}\label{subsection-products-of-pointed-sets}
Let $(X,x_{0})$ and $(Y,y_{0})$ be pointed sets.
\begin{definition}{Products of Pointed Sets}{products-of-pointed-sets}%
    The \index[set-theory]{pointed set!product of}\textbf{product of $(X,x_{0})$ and $(Y,y_{0})$} is the product of $(X,x_{0})$ and $(Y,y_{0})$ in $\Sets_{*}$ as in \ChapterRef{\ChapterLimitsAndColimits, \cref{limits-and-colimits:binary-products}}{\cref{binary-products}}.
\end{definition}
\begin{construction}{Construction of Products of Pointed Sets}{construction-of-products-of-pointed-sets}%
    Concretely, the \index[set-theory]{pointed set!product of}\textbf{product of $(X,x_{0})$ and $(Y,y_{0})$} is the pair consisting of:
    \begin{itemize}
        \item\SloganFont{The Limit. }The pointed set $(X\times Y,(x_{0},y_{0}))$.%
        \item\SloganFont{The Cone. }The morphisms of pointed sets
            \begin{align*}
                \pr_{1} &\colon (X\times Y,(x_{0},y_{0}))\to(X,x_{0}),\\
                \pr_{2} &\colon (X\times Y,(x_{0},y_{0}))\to(Y,y_{0})
            \end{align*}
            defined by
            \begin{align*}
                \pr_{1}(x,y) &\defeq x,\\
                \pr_{2}(x,y) &\defeq y
            \end{align*}
            for each $(x,y)\in X\times Y$.
    \end{itemize}
\end{construction}
\begin{Proof}{Proof of \cref{construction-of-products-of-pointed-sets}}%
    We claim that $(X\times Y,(x_{0},y_{0}))$ is the categorical product of $(X,x_{0})$ and $(Y,y_{0})$ in $\Sets_{*}$. Indeed, suppose we have a diagram of the form
    \[
        \begin{tikzcd}[row sep={8.0*\the\DL,between origins}, column sep={8.0*\the\DL,between origins}, background color=backgroundColor, ampersand replacement=\&,productArrows={8.0*\the\DL}{p_{1}}{p_{2}}]
            {}%
            \&
            {(P,*)}
            \&
            {}%
            \\
            {(X,x_{0})}
            \&
            {(X\times Y,(x_{0},y_{0}))}
            \arrow[l,"\pr_{1}"{pos=0.4},two heads]
            \arrow[r,"\pr_{2}"'{pos=0.4},two heads]
            \&
            {(Y,y_{0})}
        \end{tikzcd}
    \]%
    in $\Sets_{*}$. Then there exists a unique morphism of pointed sets
    \[
        \phi%
        \colon%
        (P,*)%
        \to%
        (X\times Y,(x_{0},y_{0}))%
    \]%
    making the diagram
    \[
        \begin{tikzcd}[row sep={8.0*\the\DL,between origins}, column sep={8.0*\the\DL,between origins}, background color=backgroundColor, ampersand replacement=\&,productArrows={8.0*\the\DL}{p_{1}}{p_{2}}]
            {}%
            \&
            {(P,*)}
            \arrow[d,"\phi"'{pos=0.5},"\exists!"{pos=0.5}, densely dashed]
            \&
            {}%
            \\
            {(X,x_{0})}
            \&
            {(X\times Y,(x_{0},y_{0}))}
            \arrow[l,"\pr_{1}"{pos=0.4},two heads]
            \arrow[r,"\pr_{2}"'{pos=0.4},two heads]
            \&
            {(Y,y_{0})}
        \end{tikzcd}
    \]%
    commute, being uniquely determined by the conditions
    \begin{align*}
        \pr_{1}\circ\phi &= p_{1},\\
        \pr_{2}\circ\phi &= p_{2}
    \end{align*}
    via
    \[
        \phi(x)%
        =%
        (p_{1}(x),p_{2}(x))%
    \]%
    for each $x\in P$. Note that this is indeed a morphism of pointed sets, as we have
    \begin{align*}
        \phi(*) &= (p_{1}(*),p_{2}(*))\\%
                &= (x_{0},y_{0}),%
    \end{align*}
    where we have used that $p_{1}$ and $p_{2}$ are morphisms of pointed sets.
\end{Proof}
\begin{proposition}{Properties of Products of Pointed Sets}{properties-of-products-of-pointed-sets}%
    Let $(X,x_{0})$, $(Y,y_{0})$, and $(Z,z_{0})$ be pointed sets.
    \begin{enumerate}
        \item\label{properties-of-products-of-pointed-sets-functoriality}\SloganFont{Functoriality. }The assignments
            \[
                (X,x_{0}),(Y,y_{0}),((X,x_{0}),(Y,y_{0}))\mapsto(X\times Y,(x_{0},y_{0}))%
            \]%
            define functors
            \[
                \Bifunctoriality{A\times-}{-\times B}{-_{1}\times-_{2}}{\Sets_{*}}{\Sets_{*}}{\Sets_{*}\times\Sets_{*}}{\Sets_{*}}%
            \]%
            defined in the same way as the functors of \ChapterRef{\ChapterConstructionsWithSets, \cref{constructions-with-sets:properties-of-products-of-sets-functoriality} of \cref{constructions-with-sets:properties-of-products-of-sets}}{\cref{properties-of-products-of-sets-functoriality} of \cref{properties-of-products-of-sets}}.
        \item\label{properties-of-products-of-pointed-sets-lack-of-adjointness}\SloganFont{Lack of Adjointness. }The functors $X\times-$ and $-\times Y$ do not admit right adjoints.
        \item\label{properties-of-products-of-pointed-sets-associativity}\SloganFont{Associativity. }We have an isomorphism of pointed sets%
            \[
                ((X\times Y)\times Z,((x_{0},y_{0}),z_{0}))
                \cong
                (X\times(Y\times Z),(x_{0},(y_{0},z_{0})))
            \]%
            natural in $(X,x_{0}),(Y,y_{0}),(Z,z_{0})\in\Obj(\Sets_{*})$.
        \item\label{properties-of-products-of-pointed-sets-unitality}\SloganFont{Unitality. }We have isomorphisms of pointed sets
            \begin{align*}
                (\pt,\star)\times(X,x_{0}) &\cong (X,x_{0}),\\
                (X,x_{0})\times(\pt,\star) &\cong (X,x_{0}),
            \end{align*}
            natural in $(X,x_{0})\in\Obj(\Sets_{*})$.
        \item\label{properties-of-products-of-pointed-sets-commutativity}\SloganFont{Commutativity. }We have an isomorphism of pointed sets
            \[
                (X\times Y,(x_{0},y_{0}))
                \cong
                (Y\times X,(y_{0},x_{0})),
            \]%
            natural in $(X,x_{0}),(Y,y_{0})\in\Obj(\Sets_{*})$.
        \item\label{properties-of-products-of-pointed-sets-symmetric-monoidality}\SloganFont{Symmetric Monoidality. }The triple $(\Sets_{*},\times,(\pt,\star))$ is a symmetric monoidal category.
        %\item\label{properties-of-products-of-pointed-sets-}\SloganFont{. }
    \end{enumerate}
\end{proposition}
\begin{Proof}{Proof of \cref{properties-of-products-of-pointed-sets}}%
    \FirstProofBox{\cref{properties-of-products-of-pointed-sets-functoriality}: Functoriality}%
    This is a special case of functoriality of limits, \ChapterRef{\ChapterLimitsAndColimits, \cref{limits-and-colimits:properties-of-co-limits-functoriality} of \cref{limits-and-colimits:properties-of-co-limits}}{\cref{properties-of-co-limits-functoriality} of \cref{properties-of-co-limits}}.

    \ProofBox{\cref{properties-of-products-of-pointed-sets-lack-of-adjointness}: Lack of Adjointness}%
    See \cite{MSE2855868}.

    \ProofBox{\cref{properties-of-products-of-pointed-sets-associativity}: Associativity}%
    This follows from \ChapterRef{\ChapterConstructionsWithSets, \cref{constructions-with-sets:properties-of-products-of-sets-associativity} of \cref{constructions-with-sets:properties-of-products-of-sets}}{\cref{properties-of-products-of-sets-associativity} of \cref{properties-of-products-of-sets}}.

    \ProofBox{\cref{properties-of-products-of-pointed-sets-unitality}: Unitality}%
    This follows from \ChapterRef{\ChapterConstructionsWithSets, \cref{constructions-with-sets:properties-of-products-of-sets-unitality} of \cref{constructions-with-sets:properties-of-products-of-sets}}{\cref{properties-of-products-of-sets-unitality} of \cref{properties-of-products-of-sets}}.

    \ProofBox{\cref{properties-of-products-of-pointed-sets-commutativity}: Commutativity}%
    This follows from \ChapterRef{\ChapterConstructionsWithSets, \cref{constructions-with-sets:properties-of-products-of-sets-commutativity} of \cref{constructions-with-sets:properties-of-products-of-sets}}{\cref{properties-of-products-of-sets-commutativity} of \cref{properties-of-products-of-sets}}.

    \ProofBox{\cref{properties-of-products-of-pointed-sets-symmetric-monoidality}: Symmetric Monoidality}%
    This follows from \ChapterRef{\ChapterConstructionsWithSets, \cref{constructions-with-sets:properties-of-products-of-sets-symmetric-monoidality} of \cref{constructions-with-sets:properties-of-products-of-sets}}{\cref{properties-of-products-of-sets-symmetric-monoidality} of \cref{properties-of-products-of-sets}}.
\end{Proof}
\subsection{Pullbacks}\label{subsection-pullbacks-of-pointed-sets}
Let $(X,x_{0})$, $(Y,y_{0})$, and $(Z,z_{0})$ be pointed sets and let $f\colon(X,x_{0})\to(Z,z_{0})$ and $g\colon(Y,y_{0})\to(Z,z_{0})$ be morphisms of pointed sets.
\begin{definition}{Pullbacks of Pointed Sets}{pullbacks-of-pointed-sets}%
    The \index[set-theory]{pointed set!pullback of}\textbf{pullback of $(X,x_{0})$ and $(Y,y_{0})$ over $(Z,z_{0})$ along $(f,g)$} is the pullback of $(X,x_{0})$ and $(Y,y_{0})$ over $(Z,z_{0})$ along $(f,g)$ in $\Sets_{*}$ as in \ChapterRef{\ChapterLimitsAndColimits, \cref{limits-and-colimits:pullbacks}}{\cref{pullbacks}}.
\end{definition}
\begin{construction}{Construction of Pullbacks of Pointed Sets}{construction-of-pullbacks-of-pointed-sets}%
    Concretely, the \index[set-theory]{pointed set!pullback of}\textbf{pullback of $(X,x_{0})$ and $(Y,y_{0})$ over $(Z,z_{0})$ along $(f,g)$} is the pair consisting of:
    \begin{itemize}
        \item\SloganFont{The Limit. }The pointed set $(X\times_{Z}Y,(x_{0},y_{0}))$.
        \item\SloganFont{The Cone. }The morphisms of pointed sets
            \begin{align*}
                \pr_{1} &\colon (X\times_{Z}Y,(x_{0},y_{0}))\to (X,x_{0}),\\
                \pr_{2} &\colon (X\times_{Z}Y,(x_{0},y_{0}))\to (Y,y_{0})
            \end{align*}
            defined by
            \begin{align*}
                \pr_{1}(x,y) &\defeq x,\\
                \pr_{2}(x,y) &\defeq y
            \end{align*}
            for each $(x,y)\in X\times_{Z}Y$.
    \end{itemize}
\end{construction}
\begin{Proof}{Proof of \cref{construction-of-pullbacks-of-pointed-sets}}%
    We claim that $X\times_{Z}Y$ is the categorical pullback of $(X,x_{0})$ and $(Y,y_{0})$ over $(Z,z_{0})$ with respect to $(f,g)$ in $\Sets_{*}$. First we need to check that the relevant pullback diagram commutes, i.e.\ that we have
    \begin{webcompile}
        f\circ\pr_{1}%
        =%
        g\circ\pr_{2},%
        \quad%
        \begin{tikzcd}[row sep={5.0*\the\DL,between origins}, column sep={8.5*\the\DL,between origins}, background color=backgroundColor, ampersand replacement=\&]
            {(X\times_{Z}Y,(x_{0},y_{0}))}
            \arrow[r,"\pr_{2}"]
            \arrow[d,"\pr_{1}"']
            \&
            {(Y,y_{0})}
            \arrow[d,"g"]
            \\
            {(X,x_{0})}
            \arrow[r,"f"']
            \&
            {(Z,z_{0})}\mrp{.}
        \end{tikzcd}
    \end{webcompile}
    Indeed, given $(x,y)\in X\times_{Z}Y$, we have
    \begin{align*}
        [f\circ\pr_{1}](x,y) &= f(\pr_{1}(x,y))\\%
                             &= f(x)\\%
                             &= g(y)\\%
                             &= g(\pr_{2}(x,y))\\%
                             &= [g\circ\pr_{2}](x,y),%
    \end{align*}
    where $f(x)=g(y)$ since $(x,y)\in X\times_{Z}Y$. Next, we prove that $X\times_{Z}Y$ satisfies the universal property of the pullback. Suppose we have a diagram of the form
    \[
        \begin{tikzcd}[row sep={5.0*\the\DL,between origins}, column sep={9.5*\the\DL,between origins}, background color=backgroundColor, ampersand replacement=\&]
            {(P,*)}
            \arrow[rrd, "p_{2}",  bend left =22.5]
            \arrow[rdd, "p_{1}"', bend right=32.5]
            \&[-2.0*\the\DL]
            \&
            \\[-2.0*\the\DL]
            \&[-2.0*\the\DL]
            {(X\times_{Z}Y,(x_{0},y_{0}))}
            \arrow[rd, phantom, "\lrcorner", very near start]
            \arrow[r, "\pr_{2}"description]
            \arrow[d, "\pr_{1}"description]
            \&
            {(Y,y_{0})}
            \arrow[d, "g"]
            \\
            \&[-2.0*\the\DL]
            {(X,x_{0})}
            \arrow[r, "f"']
            \&
            {(Z,z_{0})}
        \end{tikzcd}
    \]%
    in $\Sets_{*}$. Then there exists a unique morphism of pointed sets
    \[
        \phi%
        \colon%
        (P,*)%
        \to%
        (X\times_{Z}Y,(x_{0},y_{0}))%
    \]%
    making the diagram
    \[
        \begin{tikzcd}[row sep={5.0*\the\DL,between origins}, column sep={9.5*\the\DL,between origins}, background color=backgroundColor, ampersand replacement=\&]
            {(P,*)}
            \arrow[rrd, "p_{2}",  bend left =22.5]
            \arrow[rdd, "p_{1}"', bend right=32.5]
            \arrow[rd,  "\phi","\exists!"', dashed]
            \&[-2.0*\the\DL]
            \&
            \\[-2.0*\the\DL]
            \&[-2.0*\the\DL]
            {(X\times_{Z}Y,(x_{0},y_{0}))}
            \arrow[rd, phantom, "\lrcorner", very near start]
            \arrow[r, "\pr_{2}"description]
            \arrow[d, "\pr_{1}"description]
            \&
            {(Y,y_{0})}
            \arrow[d, "g"]
            \\
            \&[-2.0*\the\DL]
            {(X,x_{0})}
            \arrow[r, "f"']
            \&
            {(Z,z_{0})}
        \end{tikzcd}
    \]%
    commute, being uniquely determined by the conditions%
    \begin{align*}
        \pr_{1}\circ\phi &= p_{1},\\%
        \pr_{2}\circ\phi &= p_{2}%
    \end{align*}
    via
    \[
        \phi(x)%
        =%
        (p_{1}(x),p_{2}(x))%
    \]%
    for each $x\in P$, where we note that $(p_{1}(x),p_{2}(x))\in X\times Y$ indeed lies in $X\times_{Z}Y$ by the condition
    \[
        f\circ p_{1}%
        =%
        g\circ p_{2},%
    \]%
    which gives
    \[
        f(p_{1}(x))%
        =%
        g(p_{2}(x))%
    \]%
    for each $x\in P$, so that $(p_{1}(x),p_{2}(x))\in X\times_{Z}Y$. Lastly, we note that $\phi$ is indeed a morphism of pointed sets, as we have
    \begin{align*}
        \phi(*) &= (p_{1}(*),p_{2}(*))\\%
                &= (x_{0},y_{0}),%
    \end{align*}
    where we have used that $p_{1}$ and $p_{2}$ are morphisms of pointed sets.
\end{Proof}
\begin{proposition}{Properties of Pullbacks of Pointed Sets}{properties-of-pullbacks-of-pointed-sets}%
    Let $(X,x_{0})$, $(Y,y_{0})$, $(Z,z_{0})$, and $(A,a_{0})$ be pointed sets.
    \begin{enumerate}
        \item\label{properties-of-pullbacks-of-pointed-sets-functoriality}\SloganFont{Functoriality. }The assignment $(X,Y,Z,f,g)\mapsto X\times_{f,Z,g}Y$ defines a functor
            \[
                -_{1}\times_{-_{3}}-_{1}%
                \colon%
                \Fun(\CatFont{P},\Sets_{*})%
                \to%
                \Sets_{*},%
            \]%
            where $\CatFont{P}$ is the category that looks like this:
            \[
                \begin{tikzcd}[row sep={3.0*\the\DL,between origins}, column sep={3.0*\the\DL,between origins}, background color=backgroundColor, ampersand replacement=\&]
                    \&
                    \bullet
                    \arrow[d]
                    \\
                    \bullet
                    \arrow[r]
                    \&
                    \bullet\mrp{.}
                \end{tikzcd}
            \]%
            In particular, the action on morphisms of $-_{1}\times_{-_{3}}-_{1}$ is given by sending a morphism
            \[
                \begin{tikzcd}[row sep={4.0*\the\DL,between origins}, column sep={4.0*\the\DL,between origins}, background color=backgroundColor, ampersand replacement=\&]
                    X\times_{Z}Y
                    \arrow[rr]
                    \arrow[dd]
                    %\arrow[rd, dashed]
                    \arrow[rd,very near start,phantom,"\lrcorner"{xshift=0.675em}]
                    \&
                    \&
                    Y
                    \arrow[dd, "g"{description,pos=0.25}]
                    \arrow[rd, "\psi"]
                    \&
                    \\
                    \&
                    X'\times_{Z'}Y'
                    \arrow[rr,crossing over]
                    \arrow[rd,very near start,phantom,"\lrcorner"]
                    \&
                    \&
                    Y'
                    \arrow[dd, "g'"]
                    \\
                    X
                    \arrow[rr, "f"{pos=0.25}]
                    \arrow[rd, "\phi"']
                    \&
                    \&
                    Z
                    \arrow[rd,"\chi"description]
                    \&
                    \\
                    \&
                    X'
                    \arrow[rr, "f'"']
                    \arrow[from=uu, "", crossing over]
                    \&\&
                    Z'
                \end{tikzcd}
            \]%
            in $\Fun(\CatFont{P},\Sets_{*})$ to the morphism of pointed sets
            \[
                \xi%
                \colon%
                (X\times_{Z}Y,(x_{0},y_{0}))%
                \uearrow%
                (X'\times_{Z'}Y',(x'_{0},y'_{0}))%
            \]%
            given by
            \[
                \xi(x,y)%
                \defeq%
                (\phi(x),\psi(y))%
            \]%
            for each $(x,y)\in X\times_{Z}Y$, which is the unique morphism of pointed sets making the diagram
            \[
                \begin{tikzcd}[row sep={4.0*\the\DL,between origins}, column sep={4.0*\the\DL,between origins}, background color=backgroundColor, ampersand replacement=\&]
                    X\times_{Z}Y
                    \arrow[rr]
                    \arrow[dd]
                    \arrow[rd, dashed]
                    \arrow[rd,very near start,phantom,"\lrcorner"{xshift=0.675em}]
                    \&
                    \&
                    Y
                    \arrow[dd, "g"{description,pos=0.25}]
                    \arrow[rd, "\psi"]
                    \&
                    \\
                    \&
                    X'\times_{Z'}Y'
                    \arrow[rr,crossing over]
                    \arrow[rd,very near start,phantom,"\lrcorner"]
                    \&
                    \&
                    Y'
                    \arrow[dd, "g'"]
                    \\
                    X
                    \arrow[rr, "f"{pos=0.25}]
                    \arrow[rd, "\phi"']
                    \&
                    \&
                    Z
                    \arrow[rd,"\chi"description]
                    \&
                    \\
                    \&
                    X'
                    \arrow[rr, "f'"']
                    \arrow[from=uu, "", crossing over]
                    \&\&
                    Z'
                \end{tikzcd}
            \]%
            commute.
        \item\label{properties-of-pullbacks-of-pointed-sets-associativity}\SloganFont{Associativity. }Given a diagram
            \[
                \begin{tikzcd}[row sep={3.5*\the\DL,between origins}, column sep={3.5*\the\DL,between origins}, background color=backgroundColor, ampersand replacement=\&]
                    X
                    \&
                    \&
                    Y
                    \&
                    \&
                    Z
                    \\
                    \&
                    W
                    \&
                    \&
                    V
                    \&
                    % 1-Arrows
                    \arrow[from=1-1,to=2-2,"f"']%
                    \arrow[from=1-3,to=2-2,"g"]%
                    %
                    \arrow[from=1-3,to=2-4,"h"']%
                    \arrow[from=1-5,to=2-4,"k"]%
                \end{tikzcd}
            \]%
            in $\Sets_{*}$, we have isomorphisms of pointed sets
            \[
                (X\times_{W}Y)\times_{V}Z%
                \cong
                (X\times_{W}Y)\times_{Y}(Y\times_{V}Z)
                \cong
                X\times_{W}(Y\times_{V}Z),%
            \]%
            where these pullbacks are built as in the diagrams
            \begin{webcompile}
                \resizebox{\textwidth}{!}{$%
                    \begin{tikzcd}[row sep={3.5*\the\DL,between origins}, column sep={3.5*\the\DL,between origins}, background color=backgroundColor, ampersand replacement=\&]
                        \&
                        \&
                        (X\times_{W}Y)\times_{Y}Z
                        \&
                        \&
                        \\
                        \&
                        X\times_{W}Y
                        \&
                        \&
                        \&
                        \\
                        X
                        \&
                        \&
                        Y
                        \&
                        \&
                        Z\mrp{,}
                        \\
                        \&
                        W
                        \&
                        \&
                        V
                        \&
                        % 1-Arrows
                        \arrow[from=1-3,to=2-2]%
                        \arrow[from=1-3,to=3-5]%
                        %
                        \arrow[from=2-2,to=3-1]%
                        \arrow[from=2-2,to=3-3]%
                        %
                        \arrow[from=3-1,to=4-2,"f"']%
                        \arrow[from=3-3,to=4-2,"g"]%
                        \arrow[from=3-3,to=4-4,"h"']%
                        \arrow[from=3-5,to=4-4,"k"]%
                        %
                        \arrow[from=1-3,to=3-3,very near start,phantom,"\lrcorner"{rotate=-45}]
                        \arrow[from=2-2,to=4-2,very near start,phantom,"\lrcorner"{rotate=-45}]
                    \end{tikzcd}
                    \qquad
                    \begin{tikzcd}[row sep={3.5*\the\DL,between origins}, column sep={3.5*\the\DL,between origins}, background color=backgroundZolor, ampersand replacement=\&]
                        \&
                        \&
                        (X\times_{W}Y)\times_{Y}(Y\times_{V}Z)
                        \&
                        \&
                        \\
                        \&
                        X\times_{W}Y
                        \&
                        \&
                        Y\times_{V}Z
                        \&
                        \\
                        X
                        \&
                        \&
                        Y
                        \&
                        \&
                        Z\mrp{,}
                        \\
                        \&
                        W
                        \&
                        \&
                        V
                        \&
                        % 1-Arrows
                        \arrow[from=1-3,to=2-2]%
                        \arrow[from=1-3,to=2-4]%
                        %
                        \arrow[from=2-2,to=3-1]%
                        \arrow[from=2-2,to=3-3]%
                        \arrow[from=2-4,to=3-3]%
                        \arrow[from=2-4,to=3-5]%
                        %
                        \arrow[from=3-1,to=4-2,"f"']%
                        \arrow[from=3-3,to=4-2,"g"]%
                        \arrow[from=3-3,to=4-4,"h"']%
                        \arrow[from=3-5,to=4-4,"k"]%
                        %
                        \arrow[from=1-3,to=3-3,very near start,phantom,"\lrcorner"{rotate=-45}]
                        \arrow[from=2-2,to=4-2,very near start,phantom,"\lrcorner"{rotate=-45}]
                        \arrow[from=2-4,to=4-4,very near start,phantom,"\lrcorner"{rotate=-45}]
                    \end{tikzcd}
                    \qquad
                    \begin{tikzcd}[row sep={3.5*\the\DL,between origins}, column sep={3.5*\the\DL,between origins}, background color=backgroundZolor, ampersand replacement=\&]
                        \&
                        \&
                        X\times_{W}(Y\times_{V}Z)
                        \&
                        \&
                        \\
                        \&
                        \&
                        \&
                        Y\times_{V}Z
                        \&
                        \\
                        X
                        \&
                        \&
                        Y
                        \&
                        \&
                        Z\mrp{.}
                        \\
                        \&
                        W
                        \&
                        \&
                        V
                        \&
                        % 1-Arrows
                        \arrow[from=1-3,to=3-1]%
                        \arrow[from=1-3,to=2-4]%
                        %
                        \arrow[from=2-4,to=3-3]%
                        \arrow[from=2-4,to=3-5]%
                        %
                        \arrow[from=3-1,to=4-2,"f"']%
                        \arrow[from=3-3,to=4-2,"g"]%
                        \arrow[from=3-3,to=4-4,"h"']%
                        \arrow[from=3-5,to=4-4,"k"]%
                        %
                        \arrow[from=1-3,to=3-3,very near start,phantom,"\lrcorner"{rotate=-45}]
                        \arrow[from=2-4,to=4-4,very near start,phantom,"\lrcorner"{rotate=-45}]
                    \end{tikzcd}
                $}%
            \end{webcompile}
        \item\label{properties-of-pullbacks-of-pointed-sets-unitality}\SloganFont{Unitality. }We have isomorphisms of pointed sets
            \begin{webcompile}
                \begin{tikzcd}[row sep={5.0*\the\DL,between origins}, column sep={5.0*\the\DL,between origins}, background color=backgroundColor, ampersand replacement=\&]
                    A
                    \arrow[r,Equals]
                    \arrow[d,"f"']
                    \arrow[rd,very near start,phantom,"\lrcorner"]
                    \&
                    A
                    \arrow[d,"f"]
                    \\
                    X
                    \arrow[r,Equals]
                    \&
                    X
                \end{tikzcd}
                \qquad
                \begin{aligned}
                    X\times_{X}A &\cong A,\\
                    A\times_{X}X &\cong A,
                \end{aligned}
                \qquad
                \begin{tikzcd}[row sep={5.0*\the\DL,between origins}, column sep={5.0*\the\DL,between origins}, background color=backgroundColor, ampersand replacement=\&]
                    A
                    \arrow[r,"f"]
                    \arrow[d,Equals]
                    \arrow[rd,very near start,phantom,"\lrcorner"]
                    \&
                    X
                    \arrow[d,Equals]
                    \\
                    X
                    \arrow[r,"f"']
                    \&
                    X\mrp{.}
                \end{tikzcd}
            \end{webcompile}
        \item\label{properties-of-pullbacks-of-pointed-sets-commutativity}\SloganFont{Commutativity. }We have an isomorphism of pointed sets
            \begin{webcompile}
                \begin{tikzcd}[row sep={5.0*\the\DL,between origins}, column sep={5.0*\the\DL,between origins}, background color=backgroundColor, ampersand replacement=\&]
                    A\times_{X}B
                    \arrow[r]
                    \arrow[d]
                    \arrow[rd,very near start,phantom,"\lrcorner"]
                    \&
                    B
                    \arrow[d,"g"]
                    \\
                    A
                    \arrow[r,"f"']
                    \&
                    X\mrp{,}
                \end{tikzcd}
                \qquad
                A\times_{X}B
                \cong
                B\times_{X}A
                \qquad
                \begin{tikzcd}[row sep={5.0*\the\DL,between origins}, column sep={5.0*\the\DL,between origins}, background color=backgroundColor, ampersand replacement=\&]
                    B\times_{X}A
                    \arrow[r]
                    \arrow[d]
                    \arrow[rd,very near start,phantom,"\lrcorner"]
                    \&
                    A
                    \arrow[d,"f"]
                    \\
                    B
                    \arrow[r,"g"']
                    \&
                    X\mrp{.}
                \end{tikzcd}
            \end{webcompile}
        \item\label{properties-of-pullbacks-of-pointed-sets-interaction-with-products}\SloganFont{Interaction With Products. }We have an isomorphism of pointed sets
            \begin{webcompile}
                X\times_{\pt}Y%
                \cong%
                X\times Y,%
                \quad
                \begin{tikzcd}[row sep={5.0*\the\DL,between origins}, column sep={5.0*\the\DL,between origins}, background color=backgroundColor, ampersand replacement=\&]
                    X\times Y
                    \arrow[r]
                    \arrow[d]
                    \arrow[rd,very near start,phantom,"\lrcorner"]
                    \&
                    Y
                    \arrow[d,"!_{Y}"]
                    \\
                    X
                    \arrow[r,"!_{X}"']
                    \&
                    \pt\mrp{.}
                \end{tikzcd}
            \end{webcompile}
        \item\label{properties-of-pullbacks-of-pointed-sets-symmetric-monoidality}\SloganFont{Symmetric Monoidality. }The triple $(\Sets_{*},\times_{X},X)$ is a symmetric monoidal category.
        %\item\label{properties-of-pullbacks-of-pointed-sets-}\SloganFont{. }
    \end{enumerate}
\end{proposition}
\begin{Proof}{Proof of \cref{properties-of-pullbacks-of-pointed-sets}}%
    \FirstProofBox{\cref{properties-of-pullbacks-of-pointed-sets-functoriality}: Functoriality}%
    This is a special case of functoriality of co/limits, \ChapterRef{\ChapterLimitsAndColimits, \cref{limits-and-colimits:properties-of-co-limits-functoriality} of \cref{limits-and-colimits:properties-of-co-limits}}{\cref{properties-of-co-limits-functoriality} of \cref{properties-of-co-limits}}, with the explicit expression for $\xi$ following from the commutativity of the cube pullback diagram.

    \ProofBox{\cref{properties-of-pullbacks-of-pointed-sets-associativity}: Associativity}%
    This follows from \ChapterRef{\ChapterConstructionsWithSets, \cref{constructions-with-sets:properties-of-pullbacks-of-sets-associativity} of \cref{constructions-with-sets:properties-of-pullbacks-of-sets}}{\cref{properties-of-pullbacks-of-sets-associativity} of \cref{properties-of-pullbacks-of-pointed-sets}}.

    \ProofBox{\cref{properties-of-pullbacks-of-pointed-sets-unitality}: Unitality}%
    This follows from \ChapterRef{\ChapterConstructionsWithSets, \cref{constructions-with-sets:properties-of-pullbacks-of-sets-unitality} of \cref{constructions-with-sets:properties-of-pullbacks-of-sets}}{\cref{properties-of-pullbacks-of-sets-unitality} of \cref{properties-of-pullbacks-of-sets}}.

    \ProofBox{\cref{properties-of-pullbacks-of-pointed-sets-commutativity}: Commutativity}%
    This follows from \ChapterRef{\ChapterConstructionsWithSets, \cref{constructions-with-sets:properties-of-pullbacks-of-sets-commutativity} of \cref{constructions-with-sets:properties-of-pullbacks-of-sets}}{\cref{properties-of-pullbacks-of-sets-commutativity} of \cref{properties-of-pullbacks-of-sets}}.

    \ProofBox{\cref{properties-of-pullbacks-of-pointed-sets-interaction-with-products}: Interaction With Products}%
    This follows from \ChapterRef{\ChapterConstructionsWithSets, \cref{constructions-with-sets:properties-of-pullbacks-of-sets-interaction-with-products} of \cref{constructions-with-sets:properties-of-pullbacks-of-sets}}{\cref{properties-of-pullbacks-of-sets-interaction-with-products} of \cref{properties-of-pullbacks-of-sets}}.

    \ProofBox{\cref{properties-of-pullbacks-of-pointed-sets-symmetric-monoidality}: Symmetric Monoidality}%
    This follows from \ChapterRef{\ChapterConstructionsWithSets, \cref{constructions-with-sets:properties-of-pullbacks-of-sets-symmetric-monoidality} of \cref{constructions-with-sets:properties-of-pullbacks-of-sets}}{\cref{properties-of-pullbacks-of-sets-symmetric-monoidality} of \cref{properties-of-pullbacks-of-sets}}.
\end{Proof}
\subsection{Equalisers}\label{subsection-equalisers-of-pointed-sets}
Let $f,g\colon(X,x_{0})\rightrightarrows(Y,y_{0})$ be morphisms of pointed sets.
\begin{definition}{Equalisers of Pointed Sets}{equalisers-of-pointed-sets}%
    The \index[set-theory]{pointed set!equaliser of}\textbf{equaliser of $(f,g)$} is the equaliser of $f$ and $g$ in $\Sets_{*}$ as in \ChapterRef{\ChapterLimitsAndColimits, \cref{limits-and-colimits:equalisers}}{\cref{equalisers}}.
\end{definition}
\begin{construction}{Construction of Equalisers of Pointed Sets}{construction-of-equalisers-of-pointed-sets}%
    Concretely, the \index[set-theory]{pointed set!equaliser of}\textbf{equaliser of $(f,g)$} is the pair consisting of:
    \begin{itemize}
        \item\SloganFont{The Limit. }The pointed set $(\Eq(f,g),x_{0})$.
        \item\SloganFont{The Cone. }The morphism of pointed sets
            \[
                \eq(f,g)%
                \colon%
                (\Eq(f,g),x_{0})%
                \hookrightarrow%
                (X,x_{0})%
            \]%
            given by the canonical inclusion $\eq(f,g)\hookrightarrow\Eq(f,g)\hookrightarrow X$.
    \end{itemize}
\end{construction}
\begin{Proof}{Proof of \cref{construction-of-equalisers-of-pointed-sets}}%
    We claim that $(\Eq(f,g),x_{0})$ is the categorical equaliser of $f$ and $g$ in $\Sets_{*}$. First we need to check that the relevant equaliser diagram commutes, i.e.\ that we have
    \[
        f\circ\eq(f,g)%
        =%
        g\circ\eq(f,g),%
    \]%
    which indeed holds by the definition of the set $\Eq(f,g)$. Next, we prove that $\Eq(f,g)$ satisfies the universal property of the equaliser. Suppose we have a diagram of the form
    \[
        \begin{tikzcd}[row sep={5.0*\the\DL,between origins}, column sep={6.0*\the\DL,between origins}, background color=backgroundColor, ampersand replacement=\&]
            {(\Eq(f,g),x_{0})}
            \arrow[r,"{\eq(f,g)}",hook]
            \&[2.5*\the\DL]
            {(X,x_{0})}
            \arrow[r,"f", shift left =0.8]
            \arrow[r,"g"',shift right=0.8]
            \&
            {(Y,y_{0})}
            \\
            {(E,*)}
            \arrow[ru,"e"']
            \&[2.5*\the\DL]
            \&
            \&
        \end{tikzcd}
    \]%
    in $\Sets_{*}$. Then there exists a unique morphism of pointed sets
    \[
        \phi%
        \colon%
        (E,*)%
        \to%
        (\Eq(f,g),x_{0})
    \]%
    making the diagram
    \[
        \begin{tikzcd}[row sep={5.0*\the\DL,between origins}, column sep={6.0*\the\DL,between origins}, background color=backgroundColor, ampersand replacement=\&]
            {(\Eq(f,g),x_{0})}
            \arrow[r,"{\eq(f,g)}",hook]
            \&[2.5*\the\DL]
            {(X,x_{0})}
            \arrow[r,"f", shift left =0.8]
            \arrow[r,"g"',shift right=0.8]
            \&
            {(Y,y_{0})}
            \\
            {(E,*)}
            \arrow[ru,"e"']
            \arrow[u,"\phi","\exists!"',dashed]
            \&[2.5*\the\DL]
            \&
            \&
        \end{tikzcd}
    \]%
    commute, being uniquely determined by the condition%
    \[
        \eq(f,g)\circ\phi%
        =%
        e%
    \]%
    via
    \[
        \phi(x)%
        =%
        e(x)
    \]%
    for each $x\in E$, where we note that $e(x)\in A$ indeed lies in $\Eq(f,g)$ by the condition
    \[
        f\circ e%
        =%
        g\circ e,%
    \]%
    which gives
    \[
        f(e(x))%
        =%
        g(e(x))%
    \]%
    for each $x\in E$, so that $e(x)\in\Eq(f,g)$. Lastly, we note that $\phi$ is indeed a morphism of pointed sets, as we have
    \begin{align*}
        \phi(*) &= e(*)\\%
                &= x_{0},%
    \end{align*}
    where we have used that $e$ is a morphism of pointed sets.
\end{Proof}
\begin{proposition}{Properties of Equalisers of Pointed Sets}{properties-of-equalisers-of-pointed-sets}%
    Let $(X,x_{0})$ and $(Y,y_{0})$ be pointed sets and let $f,g,h\colon(X,x_{0})\to(Y,y_{0})$ be morphisms of pointed sets.
    \begin{enumerate}
        \item\label{properties-of-equalisers-of-pointed-sets-associativity}\SloganFont{Associativity. }We have isomorphisms of pointed sets%
            \begin{envsmallsize}
                \[
                    \underbrace{\Eq(f\circ\eq(g,h),g\circ\eq(g,h))}_{{}=\Eq(f\circ\eq(g,h),h\circ\eq(g,h))}%
                    \cong
                    \Eq(f,g,h)
                    \cong
                    \underbrace{\Eq(f\circ\eq(f,g),h\circ\eq(f,g))}_{{}=\Eq(g\circ\eq(f,g),h\circ\eq(f,g))},%
                \]%
            \end{envsmallsize}
            where $\Eq(f,g,h)$ is the limit of the diagram
            \[
                \begin{tikzcd}[row sep={5.0*\the\DL,between origins}, column sep={6.0*\the\DL,between origins}, background color=backgroundColor, ampersand replacement=\&]
                    {(X,x_{0})}%
                    \arrow[r,"f",shift left=2.2]%
                    \arrow[r,"g"description]%
                    \arrow[r,"h"',shift right=2.2]%
                    \&
                    {(Y,y_{0})}%
                \end{tikzcd}
            \]%
            in $\Sets_{*}$, being explicitly given by
            \[
                \Eq(f,g,h)%
                \cong%
                \{a\in A\ \middle|\ f(a)=g(a)=h(a)\}.%
            \]%
        \item\label{properties-of-equalisers-of-pointed-sets-unitality}\SloganFont{Unitality. }We have an isomorphism of pointed sets
            \[
                \Eq(f,f)%
                \cong%
                X.%
            \]%
        \item\label{properties-of-equalisers-of-pointed-sets-commutativity}\SloganFont{Commutativity. }We have an isomorphism of pointed sets
            \[
                \Eq(f,g)
                \cong
                \Eq(g,f).
            \]%
        %\item\label{properties-of-equalisers-of-pointed-sets-}\SloganFont{. }
    \end{enumerate}
\end{proposition}
\begin{Proof}{Proof of \cref{properties-of-equalisers-of-pointed-sets}}%
    \FirstProofBox{\cref{properties-of-equalisers-of-pointed-sets-associativity}: Associativity}%
    This follows from \ChapterRef{\ChapterConstructionsWithSets, \cref{constructions-with-sets:properties-of-equalisers-of-sets-associativity} of \cref{constructions-with-sets:properties-of-equalisers-of-sets}}{\cref{properties-of-equalisers-of-sets-associativity} of \cref{properties-of-equalisers-of-sets}}.

    \ProofBox{\cref{properties-of-equalisers-of-pointed-sets-unitality}: Unitality}%
    This follows from \ChapterRef{\ChapterConstructionsWithSets, \cref{constructions-with-sets:properties-of-equalisers-of-sets-unitality} of \cref{constructions-with-sets:properties-of-equalisers-of-sets}}{\cref{properties-of-equalisers-of-sets-unitality} of \cref{properties-of-equalisers-of-sets}}.

    \ProofBox{\cref{properties-of-equalisers-of-pointed-sets-commutativity}: Commutativity}%
    This follows from \ChapterRef{\ChapterConstructionsWithSets, \cref{constructions-with-sets:properties-of-equalisers-of-sets-commutativity} of \cref{constructions-with-sets:properties-of-equalisers-of-sets}}{\cref{properties-of-equalisers-of-sets-commutativity} of \cref{properties-of-equalisers-of-sets}}.
\end{Proof}
\section{Colimits of Pointed Sets}\label{section-colimits-of-pointed-sets}
\subsection{The Initial Pointed Set}\label{subsection-the-initial-pointed-set}
\begin{definition}{The Initial Pointed Set}{the-initial-pointed-set}%
    The \index[set-theory]{initial pointed set}\textbf{initial pointed set} is the initial object of $\Sets_{*}$ as in \ChapterRef{\ChapterLimitsAndColimits, \cref{limits-and-colimits:initial-objects}}{\cref{initial-objects}}.
\end{definition}
\begin{construction}{Construction of the Initial Pointed Set}{construction-of-the-initial-pointed-set}%
    Concretely, the \index[set-theory]{initial pointed set}\textbf{initial pointed set} is the pair \index[notation]{pt@$\pt$}$\smash{((\pt,\point),\{\iota_{X}\}_{(X,x_{0})\in\Obj(\Sets_{*})})}$ consisting of:
    \begin{itemize}
        \item\SloganFont{The Limit. }The pointed set $(\pt,\point)$.
        \item\SloganFont{The Cone. }The collection of morphisms of pointed sets
            \[
                \{%
                    \iota_{X}%
                    \colon%
                    (\pt,\point)%
                    \to%
                    (X,x_{0})%
                \}_{(X,x_{0})\in\Obj(\Sets)}%
            \]%
            defined by
            \[
                \iota_{X}(\point)%
                \defeq%
                x_{0}.%
            \]%
    \end{itemize}
\end{construction}
\begin{Proof}{Proof of \cref{construction-of-the-initial-pointed-set}}%
    We claim that $(\pt,\point)$ is the initial object of $\Sets_{*}$. Indeed, suppose we have a diagram of the form
    \[
        \begin{tikzcd}[row sep={5.0*\the\DL,between origins}, column sep={6.0*\the\DL,between origins}, background color=backgroundColor, ampersand replacement=\&]
            {(\pt,\point)}
            \&
            {(X,x_{0})}
        \end{tikzcd}
    \]%
    in $\Sets_{*}$. Then there exists a unique morphism of pointed sets
    \[
        \phi%
        \colon%
        (\pt,\point)%
        \to%
        (X,x_{0})%
    \]%
    making the diagram
    \[
        \begin{tikzcd}[row sep={5.0*\the\DL,between origins}, column sep={6.0*\the\DL,between origins}, background color=backgroundColor, ampersand replacement=\&]
            {(\pt,\point)}
            \arrow[r,"\phi"{pos=0.425},"\exists!"'{pos=0.425}, dashed]
            \&
            {(X,x_{0})}
        \end{tikzcd}
    \]%
    commute, namely $\iota_{X}$.
\end{Proof}
\subsection{Coproducts of Families of Pointed Sets}\label{subsection-coproducts-of-families-of-pointed-sets}
Let $\{(X_{i},x^{i}_{0})\}_{i\in I}$ be a family of pointed sets.%
\begin{definition}{Coproducts of Families of Pointed Sets}{coproducts-of-families-of-pointed-sets}%
    The \index[set-theory]{coproduct!of a family of pointed sets}\textbf{coproduct of the family $\smash{\{(X_{i},x^{i}_{0})\}_{i\in I}}$}%
    %--- Begin Footnote ---%
    \footnote{%
        \SloganFont{Further Terminology: }Also called the \index[set-theory]{pointed set!wedge sum of a family of pointed sets}\textbf{wedge sum of the family $\smash{\{(X_{i},x^{i}_{0})\}_{i\in I}}$}.
        \par\vspace*{\TCBBoxCorrection}
    } %
    %---  End Footnote  ---%
    is the coproduct of $\smash{\{(X_{i},x^{i}_{0})\}_{i\in I}}$ in $\Sets_{*}$ as in \ChapterRef{\ChapterLimitsAndColimits, \cref{limits-and-colimits:the-coproduct-of-a-family-of-objects}}{\cref{the-coproduct-of-a-family-of-objects}}.
\end{definition}
\begin{construction}{Construction of Coproducts of Families of Pointed Sets}{construction-of-coproducts-of-families-of-pointed-sets}%
    Concretely, the \index[set-theory]{coproduct!of a family of pointed sets}\textbf{coproduct of the family $\smash{\{(X_{i},x^{i}_{0})\}_{i\in I}}$} is the pair \index[notation]{veeiiniXi@$\bigvee_{i\in I}X_{i}$}$\smash{((\bigvee_{i\in I}X_{i},p_{0}),\{\inj_{i}\}_{i\in I})}$ consisting of:
    \begin{itemize}
        \item\SloganFont{The Colimit. }The pointed set $(\bigvee_{i\in I}X_{i},p_{0})$ consisting of:
            \begin{itemize}
                \item\SloganFont{The Underlying Set. }The set $\bigvee_{i\in I}X_{i}$ defined by%
                    \[
                        \bigvee_{i\in I}X_{i}%
                        \defeq%
                        (\coprod_{i\in I}X_{i})/\unsim,%
                    \]%
                    where $\unsim$ is the equivalence relation on $\coprod_{i\in I}X_{i}$ given by declaring
                    \[
                        (i,x^{i}_{0})%
                        \sim%
                        (j,x^{j}_{0})%
                    \]%
                    for each $i,j\in I$.
                \item\SloganFont{The Basepoint. }The element $p_{0}$ of $\bigvee_{i\in I}X_{i}$ defined by
                    \begin{align*}
                        p_{0} &\defeq [(i,x^{i}_{0})]\\
                              &=      [(j,x^{j}_{0})]
                    \end{align*}
                    for any $i,j\in I$.
            \end{itemize}
        \item\SloganFont{The Cocone. }The collection
            \[
                \{%
                    \inj_{i}
                    \colon%
                    (X_{i},x^{i}_{0})%
                    \to%
                    (\bigvee_{i\in I}X_{i},p_{0})%
                \}_{i\in I}%
            \]%
            of morphism of pointed sets given by
            \[
                \inj_{i}(x)%
                \defeq%
                (i,x)%
            \]%
            for each $x\in X_{i}$ and each $i\in I$.
    \end{itemize}
\end{construction}
\begin{Proof}{Proof of \cref{construction-of-coproducts-of-families-of-pointed-sets}}%
    We claim that $\smash{(\bigvee_{i\in I}X_{i},p_{0})}$ is the categorical coproduct of $\smash{\{(X_{i},x^{i}_{0})\}_{i\in I}}$ in $\Sets_{*}$. Indeed, suppose we have, for each $i\in I$, a diagram of the form
    \[
        \begin{tikzcd}[row sep={5.0*\the\DL,between origins}, column sep={7.5*\the\DL,between origins}, background color=backgroundColor, ampersand replacement=\&]
            \&
            {(C,*)}
            \\
            {(X_{i},x^{i}_{0})}
            \arrow[ru,"\iota_{i}"]
            \arrow[r,"\inj_{i}"']
            \&
            {(\displaystyle\bigvee_{i\in I}X_{i},p_{0})}
        \end{tikzcd}
    \]%
    in $\Sets_{*}$. Then there exists a unique morphism of pointed sets
    \[
        \phi%
        \colon%
        (\bigvee_{i\in I}X_{i},p_{0})%
        \to%
        (C,*)%
    \]%
    making the diagram
    \[
        \begin{tikzcd}[row sep={5.0*\the\DL,between origins}, column sep={7.5*\the\DL,between origins}, background color=backgroundColor, ampersand replacement=\&]
            \&
            {(C,*)}
            \\
            {(X_{i},x^{i}_{0})}
            \arrow[ru,"\iota_{i}"]
            \arrow[r,"\inj_{i}"']
            \&
            {(\displaystyle\bigvee_{i\in I}X_{i},p_{0})}
            \arrow[u,"\phi"{pos=0.45},"\exists!"'{pos=0.45},dashed]
        \end{tikzcd}
    \]%
    commute, being uniquely determined by the condition $\phi\circ\inj_{i}=\iota_{i}$ for each $i\in I$ via
    \[
        \phi([(i,x)])%
        =%
        \iota_{i}(x)
    \]%
    for each $[(i,x)]\in\bigvee_{i\in I}X_{i}$, where we note that $\phi$ is indeed a morphism of pointed sets, as we have
    \begin{align*}
        \phi(p_{0}) &= \iota_{i}([(i,x^{i}_{0})])\\
                    &= *,
    \end{align*}
    as $\iota_{i}$ is a morphism of pointed sets.
\end{Proof}
\begin{proposition}{Properties of Coproducts of Families of Pointed Sets}{properties-of-coproducts-of-families-of-pointed-sets}%
    Let $\{(X_{i},x^{i}_{0})\}_{i\in I}$ be a family of pointed sets.%
    \begin{enumerate}
        \item\label{properties-of-coproducts-of-families-of-pointed-sets-functoriality}\SloganFont{Functoriality. }The assignment $\{(X_{i},x^{i}_{0})\}_{i\in I}\mapsto(\bigvee_{i\in I}X_{i},p_{0})$ defines a functor
            \[
                \bigvee_{i\in I}%
                \colon%
                \Fun(I_{\disc},\Sets_{*})%
                \to%
                \Sets_{*}.%
            \]%
        %\item\label{properties-of-coproducts-of-families-of-sets-}\SloganFont{. }
    \end{enumerate}
\end{proposition}
\begin{Proof}{Proof of \cref{properties-of-coproducts-of-families-of-pointed-sets}}%
    \FirstProofBox{\cref{properties-of-coproducts-of-families-of-pointed-sets-functoriality}: Functoriality}%
    This follows from \ChapterRef{\ChapterLimitsAndColimits, \cref{limits-and-colimits:properties-of-co-limits-functoriality} of \cref{limits-and-colimits:properties-of-co-limits}}{\cref{properties-of-co-limits-functoriality} of \cref{properties-of-co-limits}}.
\end{Proof}
\subsection{Coproducts}\label{subsection-coproducts-of-pointed-sets}
Let $(X,x_{0})$ and $(Y,y_{0})$ be pointed sets.
\begin{definition}{Coproducts of Pointed Sets}{coproducts-of-pointed-sets}%
    The \index[set-theory]{pointed set!coproduct of}\textbf{coproduct of $(X,x_{0})$ and $(Y,y_{0})$}%
    %--- Begin Footnote ---%
    \footnote{%
        \SloganFont{Further Terminology: }Also called the \index[set-theory]{pointed set!wedge sum}\textbf{wedge sum of $(X,x_{0})$ and $(Y,y_{0})$}.
        \par\vspace*{\TCBBoxCorrection}
    } %
    %---  End Footnote  ---%
    is the coproduct of $(X,x_{0})$ and $(Y,y_{0})$ in $\Sets_{*}$ as in \ChapterRef{\ChapterLimitsAndColimits, \cref{limits-and-colimits:binary-products}}{\cref{binary-products}}.
\end{definition}
\begin{construction}{Construction of Coproducts of Pointed Sets}{construction-of-coproducts-of-pointed-sets}%
    Concretely, the \index[set-theory]{pointed set!coproduct of}\textbf{coproduct of $(X,x_{0})$ and $(Y,y_{0})$}, also called their \index[set-theory]{pointed set!wedge sum}\textbf{wedge sum}, is the pair consisting of:
    \begin{itemize}
        \item\SloganFont{The Colimit. }The pointed set $(X\vee Y,p_{0})$ consisting of:
            \begin{itemize}
                \item\SloganFont{The Underlying Set. }The set $X\vee Y$ defined by%
                    \begin{webcompile}
                        \begin{aligned}
                            (X\vee Y,p_{0}) &\defeq (X,x_{0})\icoprod(Y,y_{0})\\
                                            &\cong  (X\ipushout{\pt}Y,p_{0})\\
                                            &\cong  (X\icoprod Y/\unsim,p_{0}),
                        \end{aligned}
                        \qquad
                        \begin{tikzcd}[row sep={5.0*\the\DL,between origins}, column sep={5.0*\the\DL,between origins}, background color=backgroundColor, ampersand replacement=\&]
                            X\vee Y
                            \arrow[rd, phantom, "\ulcorner", very near start]
                            \&
                            Y
                            \arrow[l]
                            \\
                            X
                            \arrow[u]
                            \&
                            \pt\mrp{,}
                            \arrow[u,"{[y_{0}]}"',hook]
                            \arrow[l,"{[x_{0}]}",hook']
                        \end{tikzcd}
                    \end{webcompile}%
                    where $\unsim$ is the equivalence relation on $X\icoprod Y$ obtained by declaring $(0,x_{0})\sim(1,y_{0})$.
                \item\SloganFont{The Basepoint. }The element $p_{0}$ of $X\vee Y$ defined by
                    \begin{align*}
                        p_{0} &\defeq [(0,x_{0})]\\
                              &=      [(1,y_{0})].
                    \end{align*}
            \end{itemize}
        \item\SloganFont{The Cocone. }The morphisms of pointed sets
            \begin{align*}
                \inj_{1} &\colon (X,x_{0}) \to (X\vee Y,p_{0}),\\
                \inj_{2} &\colon (Y,y_{0}) \to (X\vee Y,p_{0}),
            \end{align*}
            given by
            \begin{align*}
                \inj_{1}(x) &\defeq [(0,x)],\\
                \inj_{2}(y) &\defeq [(1,y)],
            \end{align*}
            for each $x\in X$ and each $y\in Y$.
    \end{itemize}
\end{construction}
\begin{Proof}{Proof of \cref{construction-of-coproducts-of-pointed-sets}}%
    We claim that $(X\vee Y,p_{0})$ is the categorical coproduct of $(X,x_{0})$ and $(Y,y_{0})$ in $\Sets_{*}$. Indeed, suppose we have a diagram of the form
    \[
        \begin{tikzcd}[row sep={7.0*\the\DL,between origins}, column sep={7.0*\the\DL,between origins}, background color=backgroundColor, ampersand replacement=\&,coproductArrows={7.0*\the\DL}{\iota_{1}}{\iota_{2}}]
            {}%
            \&
            {(C,*)}
            \&
            {}%
            \\
            {(X,x_{0})}
            \&
            {(X\vee Y,p_{0})}
            \arrow[from=l,"\inj_{1}"',hook]
            \arrow[from=r,"\inj_{2}",hook']
            \&
            {(Y,y_{0})}
        \end{tikzcd}
    \]%
    in $\Sets$. Then there exists a unique morphism of pointed sets
    \[
        \phi%
        \colon%
        (X\vee Y,p_{0})%
        \to%
        (C,*)%
    \]%
    making the diagram
    \[
        \begin{tikzcd}[row sep={7.0*\the\DL,between origins}, column sep={7.0*\the\DL,between origins}, background color=backgroundColor, ampersand replacement=\&,coproductArrows={7.0*\the\DL}{\iota_{1}}{\iota_{2}}]
            {}%
            \&
            {(C,*)}
            \arrow[from=d,"\phi","\exists!"', dashed]
            \&
            {}%
            \\
            {(X,x_{0})}
            \&
            {(X\vee Y,p_{0})}
            \arrow[from=l,"\inj_{1}"',hook]
            \arrow[from=r,"\inj_{2}",hook']
            \&
            {(Y,y_{0})}
        \end{tikzcd}
    \]%
    commute, being uniquely determined by the conditions
    \begin{align*}
        \phi\circ\inj_{X} &= \iota_{X},\\
        \phi\circ\inj_{Y} &= \iota_{Y}
    \end{align*}
    via
    \[
        \phi(z)%
        =%
        \begin{cases}
            \iota_{X}(x) &\text{if $z=[(0,x)]$ with $x\in X$,}\\
            \iota_{Y}(y) &\text{if $z=[(1,y)]$ with $y\in Y$}
        \end{cases}
    \]%
    for each $z\in X\vee Y$, where we note that $\phi$ is indeed a morphism of pointed sets, as we have
    \begin{align*}
         \phi(p_{0}) &= \iota_{X}([(0,x_{0})])\\
                     &= \iota_{Y}([(1,y_{0})])\\
                     &= *,
    \end{align*}
    as $\iota_{X}$ and $\iota_{Y}$ are morphisms of pointed sets.
\end{Proof}
\begin{proposition}{Properties of Wedge Sums of Pointed Sets}{properties-of-wedge-sums-of-pointed-sets}%
    Let $(X,x_{0})$ and $(Y,y_{0})$ be pointed sets.
    \begin{enumerate}
        \item\label{properties-of-wedge-sums-of-pointed-sets-functoriality}\SloganFont{Functoriality. }The assignments
            \[
                (X,x_{0}),(Y,y_{0}),((X,x_{0}),(Y,y_{0}))\mapsto(X\vee Y,p_{0})%
            \]%
            define functors
            \begin{align*}
                X\vee-         &\colon \Sets_{*}                \to \Sets_{*},\\
                -\vee Y        &\colon \Sets_{*}                \to \Sets_{*},\\
                -_{1}\vee-_{2} &\colon \Sets_{*}\times\Sets_{*} \to \Sets_{*}.
            \end{align*}
        \item\label{properties-of-wedge-sums-of-pointed-sets-associativity}\SloganFont{Associativity. }We have an isomorphism of pointed sets
            \[
                (X\vee Y)\vee Z%
                \cong%
                X\vee(Y\vee Z),%
            \]%
            natural in $(X,x_{0}),(Y,y_{0}),(Z,z_{0})\in\Sets_{*}$.
        \item\label{properties-of-wedge-sums-of-pointed-sets-unitality}\SloganFont{Unitality. }We have isomorphisms of pointed sets
            \begin{align*}
                (\pt,*)\vee (X,x_{0}) &\cong (X,x_{0}),\\%
                (X,x_{0})\vee(\pt,*)  &\cong (X,x_{0}),%
            \end{align*}
            natural in $(X,x_{0})\in\Sets_{*}$.
        \item\label{properties-of-wedge-sums-of-pointed-sets-commutativity}\SloganFont{Commutativity. }We have an isomorphism of pointed sets
            \[
                X\vee Y
                \cong
                Y\vee X,
            \]%
            natural in $(X,x_{0}),(Y,y_{0})\in\Sets_{*}$.
        \item\label{properties-of-wedge-sums-of-pointed-sets-symmetric-monoidality}\SloganFont{Symmetric Monoidality. }The triple $(\Sets_{*},\vee,\pt)$ is a symmetric monoidal category.
        \item\label{properties-of-wedge-sums-of-pointed-sets-the-fold-map}\SloganFont{The Fold Map. }We have a natural transformation
            \begin{webcompile}
                \nabla
                \colon
                {\vee}\circ\Delta^{\Cats}_{\Sets_{*}}
                \Longrightarrow
                \id_{\Sets_{*}},
                \qquad
                \begin{tikzcd}[row sep={4.5*\the\DL,between origins}, column sep={3.5*\the\DL,between origins}, background color=backgroundColor, ampersand replacement=\&]
                    \&
                    \Sets_{*}\times\Sets_{*}
                    \arrow[rd,"\vee"]
                    \&
                    \\
                    \Sets_{*}
                    \arrow[ru,"\Delta^{\Cats}_{\Sets_{*}}"{pos=0.35}]
                    \arrow[rr,"\id_{\Sets_{*}}"',""'{name=2,pos=0.485},bend right=20]
                    \&
                    \&
                    \Sets_{*}\mrp{,}
                    % 2-Arrows
                    \arrow[from=1-2,to=2,"\nabla"description,shorten=0.25*\the\DL,Rightarrow]%
                \end{tikzcd}
            \end{webcompile}%
            called the \textbf{fold map}, whose component
            \[
                \nabla_{X}
                \colon
                X\vee X
                \to
                X
            \]%
            at $X$ is given by
            \[
                \nabla_{X}(p)%
                \defeq%
                \begin{cases}
                    x &\text{if $p=[(0,x)]$,}\\
                    x &\text{if $p=[(1,x)]$}
                \end{cases}
            \]%
            for each $p\in X\vee X$.
        %\item\label{properties-of-wedge-sums-of-pointed-sets-}\SloganFont{. }
    \end{enumerate}
\end{proposition}
\begin{Proof}{Proof of \cref{properties-of-wedge-sums-of-pointed-sets}}%
    \FirstProofBox{\cref{properties-of-wedge-sums-of-pointed-sets-functoriality}: Functoriality}%
    This follows from \ChapterRef{\ChapterLimitsAndColimits, \cref{limits-and-colimits:properties-of-co-limits-functoriality} of \cref{limits-and-colimits:properties-of-co-limits}}{\cref{properties-of-co-limits-functoriality} of \cref{properties-of-co-limits}}.

    \ProofBox{\cref{properties-of-wedge-sums-of-pointed-sets-associativity}: Associativity}%
    Omitted.

    \ProofBox{\cref{properties-of-wedge-sums-of-pointed-sets-unitality}: Unitality}%
    Omitted.

    \ProofBox{\cref{properties-of-wedge-sums-of-pointed-sets-commutativity}: Commutativity}%
    Omitted.

    \ProofBox{\cref{properties-of-wedge-sums-of-pointed-sets-symmetric-monoidality}: Symmetric Monoidality}%
    Omitted.

    \ProofBox{\cref{properties-of-wedge-sums-of-pointed-sets-the-fold-map}: The Fold Map}%
    Naturality for the transformation $\nabla$ is the statement that, given a morphism of pointed sets $f\colon(X,x_{0})\to(Y,y_{0})$, we have
    \begin{webcompile}
        \nabla_{Y}\circ(f\vee f)%
        =%
        f\circ\nabla_{X},%
        \quad%
        \begin{tikzcd}[row sep={5.0*\the\DL,between origins}, column sep={5.0*\the\DL,between origins}, background color=backgroundColor, ampersand replacement=\&]
            X\vee X
            \arrow[r,"\nabla_{X}"]
            \arrow[d,"f\vee f"']
            \&
            X
            \arrow[d,"f"]
            \\
            Y\vee Y
            \arrow[r,"\nabla_{Y}"']
            \&
            Y\mrp{.}
        \end{tikzcd}
    \end{webcompile}
    Indeed, we have
    \begin{align*}
        [\nabla_{Y}\circ(f\vee f)]([(i,x)]) &= \nabla_{Y}([(i,f(x))])\\
                                            &= f(x)\\
                                            &= f(\nabla_{X}([(i,x)]))\\
                                            &= [f\circ\nabla_{X}]([(i,x)])
    \end{align*}
    for each $[(i,x)]\in X\vee X$, and thus $\nabla$ is indeed a natural transformation.
\end{Proof}
\subsection{Pushouts}\label{subsection-pushouts-of-pointed-sets}
Let $(X,x_{0})$, $(Y,y_{0})$, and $(Z,z_{0})$ be pointed sets and let $f\colon(Z,z_{0})\to(X,x_{0})$ and $g\colon(Z,z_{0})\to(Y,y_{0})$ be morphisms of pointed sets.
\begin{definition}{Pushouts of Pointed Sets}{pushouts-of-pointed-sets}%
    The \index[set-theory]{pointed set!pushout of}\textbf{pushout of $(X,x_{0})$ and $(Y,y_{0})$ over $(Z,z_{0})$ along $(f,g)$} is the pushout of $(X,x_{0})$ and $(Y,y_{0})$ over $(Z,z_{0})$ along $(f,g)$ in $\Sets_{*}$ as in \ChapterRef{\ChapterLimitsAndColimits, \cref{limits-and-colimits:pushouts}}{\cref{pushouts}}.
\end{definition}
\begin{construction}{Construction of Pushouts of Pointed Sets}{construction-of-pushouts-of-pointed-sets}%
    Concretely, the \textbf{pushout of $(X,x_{0})$ and $(Y,y_{0})$ over $(Z,z_{0})$ along $(f,g)$} is the pair consisting of:
    \begin{itemize}
        \item\SloganFont{The Colimit. }The pointed set $(X\coprod_{f,Z,g}Y,p_{0})$, where:
            \begin{itemize}
                \item The set $X\coprod_{f,Z,g}Y$ is the pushout (of unpointed sets) of $X$ and $Y$ over $Z$ with respect to $f$ and $g$;
                \item We have $p_{0}=[x_{0}]=[y_{0}]$.
            \end{itemize}
        \item\SloganFont{The Cocone. }The morphisms of pointed sets
            \begin{align*}
                \inj_{1} &\colon(X,x_{0})\to(X\ipushout{Z}Y,p_{0}),\\
                \inj_{2} &\colon(Y,y_{0})\to(X\ipushout{Z}Y,p_{0})
            \end{align*}
            given by
            \begin{align*}
                \inj_{1}(x) &\defeq [(0,x)]\\%
                \inj_{2}(y) &\defeq [(1,y)]%
            \end{align*}
            for each $x\in X$ and each $y\in Y$.
    \end{itemize}
\end{construction}
\begin{Proof}{Proof of \cref{constructions-of-pushouts-of-pointed-sets}}%
    Firstly, we note that indeed $[x_{0}]=[y_{0}]$, as we have
    \begin{align*}
        x_{0} &= f(z_{0}),\\
        y_{0} &= g(z_{0})
    \end{align*}
    since $f$ and $g$ are morphisms of pointed sets, with the relation $\unsim$ on $X\icoprod_{Z}Y$ then identifying $x_{0}=f(z_{0})\sim g(z_{0})=y_{0}$.

    We now claim that $(X\ipushout{Z}Y,p_{0})$ is the categorical pushout of $(X,x_{0})$ and $(Y,y_{0})$ over $(Z,z_{0})$ with respect to $(f,g)$ in $\Sets_{*}$. First we need to check that the relevant pushout diagram commutes, i.e.\ that we have
    \begin{webcompile}
        \inj_{1}\circ f%
        =%
        \inj_{2}\circ g,%
        \quad%
        \begin{tikzcd}[row sep={5.0*\the\DL,between origins}, column sep={7.0*\the\DL,between origins}, background color=backgroundColor, ampersand replacement=\&]
            {(X\ipushout{Z}Y,p_{0})}
            \arrow[from=r,"\inj_{2}"']
            \arrow[from=d,"\inj_{1}"]
            \&
            {(Y,y_{0})}
            \arrow[from=d,"g"']
            \\
            {(X,x_{0})}
            \arrow[from=r,"f"]
            \&
            {(Z,z_{0})}\mrp{.}
        \end{tikzcd}
    \end{webcompile}
    Indeed, given $z\in Z$, we have
    \begin{align*}
        [\inj_{1}\circ f](z) &= \inj_{1}(f(z))\\%
                             &= [(0,f(z))]\\%
                             &= [(1,g(z))]\\%
                             &= \inj_{2}(g(z))\\%
                             &= [\inj_{2}\circ g](z),%
    \end{align*}
    where $[(0,f(z))]=[(1,g(z))]$ by the definition of the relation $\unsim$ on $X\icoprod Y$ (the coproduct of unpointed sets of $X$ and $Y$). Next, we prove that $X\icoprod_{Z}Y$ satisfies the universal property of the pushout. Suppose we have a diagram of the form
    \[
        \begin{tikzcd}[row sep={6.0*\the\DL,between origins}, column sep={8.5*\the\DL,between origins}, background color=backgroundColor, ampersand replacement=\&]
            {(P,*)}
            \arrow[from=rrd, "\iota_{2}"',bend right=25]
            \arrow[from=rdd, "\iota_{1}", bend left =27.5]
            \&[-2.0*\the\DL]
            \&
            \\[-2.0*\the\DL]
            \&[-2.0*\the\DL]
            {(X\ipushout{Z}Y,p_{0})}
            \arrow[rd, phantom, "\ulcorner", very near start]
            \arrow[from=r, "\inj_{2}"description]
            \arrow[from=d, "\inj_{1}"'description]
            \&
            {(Y,y_{0})}
            \arrow[from=d, "g"']
            \\
            \&[-2.0*\the\DL]
            {(X,x_{0})}
            \arrow[from=r, "f"]
            \&
            {(Z,z_{0})}
        \end{tikzcd}
    \]%
    in $\Sets_{*}$. Then there exists a unique morphism of pointed sets
    \[
        \phi%
        \colon%
        (X\ipushout{Z}Y,p_{0})%
        \to%
        (P,*)%
    \]%
    making the diagram
    \[
        \begin{tikzcd}[row sep={6.0*\the\DL,between origins}, column sep={8.5*\the\DL,between origins}, background color=backgroundColor, ampersand replacement=\&]
            {(P,*)}
            \arrow[from=rrd, "\iota_{2}"',bend right=25]
            \arrow[from=rdd, "\iota_{1}", bend left =27.5]
            \arrow[from=rd,  "\phi","\exists!"',dashed]
            \&[-2.0*\the\DL]
            \&
            \\[-2.0*\the\DL]
            \&[-2.0*\the\DL]
            {(X\ipushout{Z}Y,p_{0})}
            \arrow[rd, phantom, "\ulcorner", very near start]
            \arrow[from=r, "\inj_{2}"description]
            \arrow[from=d, "\inj_{1}"'description]
            \&
            {(Y,y_{0})}
            \arrow[from=d, "g"']
            \\
            \&[-2.0*\the\DL]
            {(X,x_{0})}
            \arrow[from=r, "f"]
            \&
            {(Z,z_{0})}
        \end{tikzcd}
    \]%
    commute, being uniquely determined by the conditions%
    \begin{align*}
        \phi\circ\inj_{1} &= \iota_{1},\\%
        \phi\circ\inj_{2} &= \iota_{2}%
    \end{align*}
    via
    \[
        \phi(p)%
        =%
        \begin{cases}
            \iota_{1}(x) &\text{if $x=[(0,x)]$,}\\
            \iota_{2}(y) &\text{if $x=[(1,y)]$}
        \end{cases}
    \]%
    for each $p\in X\ipushout{Z}Y$, where the well-definedness of $\phi$ is proven in the same way as in the proof of \ChapterRef{\ChapterConstructionsWithSets, \cref{constructions-with-sets:pushouts-of-sets}}{\cref{pushouts-of-sets}}. Finally, we show that $\phi$ is indeed a morphism of pointed sets, as we have
    \begin{align*}
        \phi(p_{0}) &= \phi([(0,x_{0})])\\
                    &= \iota_{1}(x_{0})\\
                    &= *,
    \end{align*}
    or alternatively
    \begin{align*}
        \phi(p_{0}) &= \phi([(1,y_{0})])\\
                    &= \iota_{2}(y_{0})\\
                    &= *,
    \end{align*}
    where we use that $\iota_{1}$ (resp.\ $\iota_{2}$) is a morphism of pointed sets.
\end{Proof}
\begin{proposition}{Properties of Pushouts of Pointed Sets}{properties-of-pushouts-of-pointed-sets}%
    Let $(X,x_{0})$, $(Y,y_{0})$, $(Z,z_{0})$, and $(A,a_{0})$ be pointed sets.
    \begin{enumerate}
        \item\label{properties-of-pushouts-of-pointed-sets-functoriality}\SloganFont{Functoriality. }The assignment $(X,Y,Z,f,g)\mapsto X\ipushout{f,Z,g}Y$ defines a functor
            \[
                -_{1}\ipushout{-_{3}}-_{1}%
                \colon%
                \Fun(\CatFont{P},\Sets)%
                \to%
                \Sets_{*},%
            \]%
            where $\CatFont{P}$ is the category that looks like this:
            \[
                \begin{tikzcd}[row sep={3.0*\the\DL,between origins}, column sep={3.0*\the\DL,between origins}, background color=backgroundColor, ampersand replacement=\&]
                    \&
                    \bullet
                    \arrow[from=d]
                    \\
                    \bullet
                    \arrow[from=r]
                    \&
                    \bullet\mrp{.}
                \end{tikzcd}
            \]%
            In particular, the action on morphisms of $-_{1}\ipushout{-_{3}}-_{1}$ is given by sending a morphism
            \[
                \begin{tikzcd}[row sep={4.0*\the\DL,between origins}, column sep={4.0*\the\DL,between origins}, background color=backgroundColor, ampersand replacement=\&]
                    X\ipushout{Z}Y
                    \arrow[from=rr]
                    \arrow[from=dd]
                    %\arrow[rd, dashed]
                    \arrow[rd,very near start,phantom,"\ulcorner"{xshift=0.675em}]
                    \&
                    \&
                    Y
                    \arrow[from=dd, "g"{description,pos=0.25}]
                    \arrow[rd, "\psi"]
                    \&
                    \\
                    \&
                    X'\ipushout{Z'}Y'
                    \arrow[from=rr,crossing over]
                    \arrow[rd,very near start,phantom,"\ulcorner"]
                    \&
                    \&
                    Y'
                    \arrow[from=dd, "g'"']
                    \\
                    X
                    \arrow[from=rr, "f"'{pos=0.25}]
                    \arrow[rd, "\phi"']
                    \&
                    \&
                    Z
                    \arrow[rd,"\chi"description]
                    \&
                    \\
                    \&
                    X'
                    \arrow[from=rr, "f'"]
                    \arrow[uu, "", crossing over]
                    \&\&
                    Z'
                \end{tikzcd}
            \]%
            in $\Fun(\CatFont{P},\Sets_{*})$ to the morphism of pointed sets
            \[
                \xi%
                \colon%
                (X\ipushout{Z}Y,p_{0})%
                \uearrow%
                (X'\ipushout{Z'}Y',p'_{0})%
            \]%
            given by
            \[
                \xi(p)%
                \defeq%
                \begin{cases}
                    \phi(x) &\text{if $p=[(0,x)]$},\\
                    \psi(y) &\text{if $p=[(1,y)]$}
                \end{cases}
            \]%
            for each $p\in X\ipushout{Z}Y$, which is the unique morphism of pointed sets making the diagram
            \[
                \begin{tikzcd}[row sep={4.0*\the\DL,between origins}, column sep={4.0*\the\DL,between origins}, background color=backgroundColor, ampersand replacement=\&]
                    X\ipushout{Z}Y
                    \arrow[from=rr]
                    \arrow[from=dd]
                    \arrow[rd, dashed]
                    \arrow[rd,very near start,phantom,"\ulcorner"{xshift=0.675em}]
                    \&
                    \&
                    Y
                    \arrow[from=dd, "g"{description,pos=0.25}]
                    \arrow[rd, "\psi"]
                    \&
                    \\
                    \&
                    X'\ipushout{Z'}Y'
                    \arrow[from=rr,crossing over]
                    \arrow[rd,very near start,phantom,"\ulcorner"]
                    \&
                    \&
                    Y'
                    \arrow[from=dd, "g'"']
                    \\
                    X
                    \arrow[from=rr, "f"'{pos=0.25}]
                    \arrow[rd, "\phi"']
                    \&
                    \&
                    Z
                    \arrow[rd,"\chi"description]
                    \&
                    \\
                    \&
                    X'
                    \arrow[from=rr, "f'"]
                    \arrow[uu, "", crossing over]
                    \&\&
                    Z'
                \end{tikzcd}
            \]%
            commute.
        \item\label{properties-of-pushouts-of-pointed-sets-associativity}\SloganFont{Associativity. }Given a diagram
            \[
                \begin{tikzcd}[row sep={3.5*\the\DL,between origins}, column sep={3.5*\the\DL,between origins}, background color=backgroundColor, ampersand replacement=\&]
                    X
                    \&
                    \&
                    Y
                    \&
                    \&
                    Z
                    \\
                    \&
                    W
                    \&
                    \&
                    V
                    \&
                    % 1-Arrows
                    \arrow[from=2-2,to=1-1,"f"]%
                    \arrow[from=2-2,to=1-3,"g"']%
                    %
                    \arrow[from=2-4,to=1-3,"h"]%
                    \arrow[from=2-4,to=1-5,"k"']%
                \end{tikzcd}
            \]%
            in $\Sets$, we have isomorphisms of pointed sets
            \[
                (X\ipushout{W}Y)\ipushout{V}Z%
                \cong
                (X\ipushout{W}Y)\ipushout{Y}(Y\ipushout{V}Z)
                \cong
                X\ipushout{W}(Y\ipushout{V}Z),%
            \]%
            where these pullbacks are built as in the diagrams
            \begin{webcompile}
                \resizebox{\textwidth}{!}{$%
                    \begin{tikzcd}[row sep={3.5*\the\DL,between origins}, column sep={3.5*\the\DL,between origins}, background color=backgroundZolor, ampersand replacement=\&]
                        \&
                        \&
                        (X\ipushout{W}Y)\ipushout{V}Z
                        \&
                        \&
                        \\
                        \&
                        X\ipushout{W}Y
                        \&
                        \&
                        \&
                        \\
                        X
                        \&
                        \&
                        Y
                        \&
                        \&
                        Z\mrp{,}
                        \\
                        \&
                        W
                        \&
                        \&
                        V
                        \&
                        % 1-Arrows
                        \arrow[from=2-2,to=1-3]%
                        \arrow[from=3-5,to=1-3]%
                        %
                        \arrow[from=3-1,to=2-2]%
                        \arrow[from=3-3,to=2-2]%
                        %
                        \arrow[from=4-2,to=3-1,"f"]%
                        \arrow[from=4-2,to=3-3,"g"']%
                        \arrow[from=4-4,to=3-3,"h"]%
                        \arrow[from=4-4,to=3-5,"k"']%
                        %
                        \arrow[from=1-3,to=3-3,very near start,phantom,"\urcorner"{rotate=45}]
                        \arrow[from=2-2,to=4-2,very near start,phantom,"\urcorner"{rotate=45}]
                    \end{tikzcd}
                    \qquad
                    \begin{tikzcd}[row sep={3.5*\the\DL,between origins}, column sep={3.5*\the\DL,between origins}, background color=backgroundZolor, ampersand replacement=\&]
                        \&
                        \&
                        (X\ipushout{W}Y)\ipushout{Y}(Y\ipushout{V}Z)
                        \&
                        \&
                        \\
                        \&
                        X\ipushout{W}Y
                        \&
                        \&
                        Y\ipushout{V}Z
                        \&
                        \\
                        X
                        \&
                        \&
                        Y
                        \&
                        \&
                        Z\mrp{,}
                        \\
                        \&
                        W
                        \&
                        \&
                        V
                        \&
                        % 1-Arrows
                        \arrow[from=2-2,to=1-3]%
                        \arrow[from=2-4,to=1-3]%
                        %
                        \arrow[from=3-1,to=2-2]%
                        \arrow[from=3-3,to=2-2]%
                        \arrow[from=3-3,to=2-4]%
                        \arrow[from=3-5,to=2-4]%
                        %
                        \arrow[from=4-2,to=3-1,"f"]%
                        \arrow[from=4-2,to=3-3,"g"']%
                        \arrow[from=4-4,to=3-3,"h"]%
                        \arrow[from=4-4,to=3-5,"k"']%
                        %
                        \arrow[from=1-3,to=3-3,very near start,phantom,"\ulcorner"{rotate=-45}]
                        \arrow[from=2-2,to=4-2,very near start,phantom,"\ulcorner"{rotate=-45}]
                        \arrow[from=2-4,to=4-4,very near start,phantom,"\ulcorner"{rotate=-45}]
                    \end{tikzcd}
                    \qquad
                    \begin{tikzcd}[row sep={3.5*\the\DL,between origins}, column sep={3.5*\the\DL,between origins}, background color=backgroundZolor, ampersand replacement=\&]
                        \&
                        \&
                        X\ipushout{W}(Y\ipushout{V}Z)
                        \&
                        \&
                        \\
                        \&
                        \&
                        \&
                        Y\ipushout{V}Z
                        \&
                        \\
                        X
                        \&
                        \&
                        Y
                        \&
                        \&
                        Z\mrp{.}
                        \\
                        \&
                        W
                        \&
                        \&
                        V
                        \&
                        % 1-Arrows
                        \arrow[from=3-1,to=1-3]%
                        \arrow[from=2-4,to=1-3]%
                        %
                        \arrow[from=3-3,to=2-4]%
                        \arrow[from=3-5,to=2-4]%
                        %
                        \arrow[from=4-2,to=3-1,"f"]%
                        \arrow[from=4-2,to=3-3,"g"']%
                        \arrow[from=4-4,to=3-3,"h"]%
                        \arrow[from=4-4,to=3-5,"k"']%
                        %
                        \arrow[from=1-3,to=3-3,very near start,phantom,"\ulcorner"{rotate=-45}]
                        \arrow[from=2-4,to=4-4,very near start,phantom,"\ulcorner"{rotate=-45}]
                    \end{tikzcd}
                $}%
            \end{webcompile}
        \item\label{properties-of-pushouts-of-pointed-sets-unitality}\SloganFont{Unitality. }We have isomorphisms of sets
            \begin{webcompile}
                \begin{tikzcd}[row sep={5.0*\the\DL,between origins}, column sep={5.0*\the\DL,between origins}, background color=backgroundColor, ampersand replacement=\&]
                    A
                    \arrow[r,Equals]
                    \arrow[from=d,"f"]
                    \arrow[rd,very near start,phantom,"\ulcorner"]
                    \&
                    A
                    \arrow[from=d,"f"']
                    \\
                    X
                    \arrow[r,Equals]
                    \&
                    X
                \end{tikzcd}
                \qquad
                \begin{aligned}
                    X\ipushout{X}A &\cong A,\\
                    A\ipushout{X}X &\cong A,
                \end{aligned}
                \qquad
                \begin{tikzcd}[row sep={5.0*\the\DL,between origins}, column sep={5.0*\the\DL,between origins}, background color=backgroundColor, ampersand replacement=\&]
                    A
                    \arrow[from=r,"f"']
                    \arrow[d,Equals]
                    \arrow[rd,very near start,phantom,"\ulcorner"]
                    \&
                    X
                    \arrow[d,Equals]
                    \\
                    X
                    \arrow[from=r,"f"]
                    \&
                    X\mrp{.}
                \end{tikzcd}
            \end{webcompile}
        \item\label{properties-of-pushouts-of-pointed-sets-commutativity}\SloganFont{Commutativity. }We have an isomorphism of sets
            \begin{webcompile}
                \begin{tikzcd}[row sep={5.0*\the\DL,between origins}, column sep={5.0*\the\DL,between origins}, background color=backgroundColor, ampersand replacement=\&]
                    X\ipushout{Z}Y
                    \arrow[from=r]
                    \arrow[from=d]
                    \arrow[rd,very near start,phantom,"\ulcorner"]
                    \&
                    Y
                    \arrow[from=d,"g"']
                    \\
                    X
                    \arrow[from=r,"f"]
                    \&
                    Z\mrp{,}
                \end{tikzcd}
                \qquad
                X\ipushout{Z}Y
                \cong
                Y\ipushout{Z}X
                \qquad
                \begin{tikzcd}[row sep={5.0*\the\DL,between origins}, column sep={5.0*\the\DL,between origins}, background color=backgroundColor, ampersand replacement=\&]
                    Y\ipushout{Z}X
                    \arrow[from=r]
                    \arrow[from=d]
                    \arrow[rd,very near start,phantom,"\ulcorner"]
                    \&
                    X
                    \arrow[from=d,"f"']
                    \\
                    Y
                    \arrow[from=r,"g"]
                    \&
                    Z\mrp{.}
                \end{tikzcd}
            \end{webcompile}
        \item\label{properties-of-pushouts-of-pointed-sets-interaction-with-coproducts}\SloganFont{Interaction With Coproducts. }We have
            \begin{webcompile}
                X\ipushout{\pt}Y%
                \cong%
                X\vee Y,%
                \quad
                \begin{tikzcd}[row sep={5.0*\the\DL,between origins}, column sep={5.0*\the\DL,between origins}, background color=backgroundColor, ampersand replacement=\&]
                    X\vee Y
                    \arrow[from=r]
                    \arrow[from=d]
                    \arrow[rd,very near start,phantom,"\ulcorner"]
                    \&
                    Y
                    \arrow[from=d,"{[y_{0}]}"',hook]
                    \\
                    X
                    \arrow[from=r,"{[x_{0}]}",hook']
                    \&
                    \pt\mrp{.}
                \end{tikzcd}
            \end{webcompile}
        \item\label{properties-of-pushouts-of-pointed-sets-symmetric-monoidality}\SloganFont{Symmetric Monoidality. }The triple $(\Sets_{*},\ipushout{X},(X,x_{0}))$ is a symmetric monoidal category.
        %\item\label{properties-of-pushouts-of-pointed-sets-}\SloganFont{. }
    \end{enumerate}
\end{proposition}
\begin{Proof}{Proof of \cref{properties-of-pushouts-of-pointed-sets}}%
    \FirstProofBox{\cref{properties-of-pushouts-of-pointed-sets-functoriality}: Functoriality}%
    This is a special case of functoriality of co/limits, \ChapterRef{\ChapterLimitsAndColimits, \cref{limits-and-colimits:properties-of-co-limits-functoriality} of \cref{limits-and-colimits:properties-of-co-limits}}{\cref{properties-of-co-limits-functoriality} of \cref{properties-of-co-limits}}, with the explicit expression for $\xi$ following from the commutativity of the cube pushout diagram.

    \ProofBox{\cref{properties-of-pushouts-of-pointed-sets-associativity}: Associativity}%
    This follows from \ChapterRef{\ChapterConstructionsWithSets, \cref{constructions-with-sets:properties-of-pushouts-of-sets-associativity} of \cref{constructions-with-sets:properties-of-pushouts-of-sets}}{\cref{properties-of-pushouts-of-sets-associativity} of \cref{properties-of-pushouts-of-sets}}.

    \ProofBox{\cref{properties-of-pushouts-of-pointed-sets-unitality}: Unitality}%
    This follows from \ChapterRef{\ChapterConstructionsWithSets, \cref{constructions-with-sets:properties-of-pushouts-of-sets-unitality} of \cref{constructions-with-sets:properties-of-pushouts-of-sets}}{\cref{properties-of-pushouts-of-sets-unitality} of \cref{properties-of-pushouts-of-sets}}.

    \ProofBox{\cref{properties-of-pushouts-of-pointed-sets-commutativity}: Commutativity}%
    This follows from \ChapterRef{\ChapterConstructionsWithSets, \cref{constructions-with-sets:properties-of-pushouts-of-sets-commutativity} of \cref{constructions-with-sets:properties-of-pushouts-of-sets}}{\cref{properties-of-pushouts-of-sets-commutativity} of \cref{properties-of-pushouts-of-sets}}.

    \ProofBox{\cref{properties-of-pushouts-of-pointed-sets-interaction-with-coproducts}: Interaction With Coproducts}%
    Omitted.

    \ProofBox{\cref{properties-of-pushouts-of-pointed-sets-symmetric-monoidality}: Symmetric Monoidality}%
    Omitted.
\end{Proof}
\subsection{Coequalisers}\label{subsection-coequalisers-of-pointed-sets}
Let $f,g\colon(X,x_{0})\rightrightarrows(Y,y_{0})$ be morphisms of pointed sets.
\begin{definition}{Coequalisers of Pointed Sets}{coequalisers-of-pointed-sets}%
    The \index[set-theory]{pointed set!coequaliser of}\textbf{coequaliser of $(f,g)$} is the pointed set $(\CoEq(f,g),[y_{0}])$.
\end{definition}
\begin{construction}{Construction of Coequalisers of Pointed Sets}{construction-of-coequalisers-of-pointed-sets}%
    The \index[set-theory]{pointed set!coequaliser of}\textbf{coequaliser of $(f,g)$} is the pair $\smash{((\CoEq(f,g),[y_{0}]),\coeq(f,g))}$ consisting of:
    \begin{itemize}
        \item\SloganFont{The Colimit. }The pointed set $(\CoEq(f,g),[y_{0}])$, where $\CoEq(f,g)$ is the coequaliser of $f$ and $g$ as in \ChapterRef{\ChapterConstructionsWithSets, \cref{constructions-with-sets:coequalisers-of-sets}}{\cref{coequalisers-of-sets}}.
        \item\SloganFont{The Cocone. }The map
            \[
                \coeq(f,g)%
                \colon%
                Y%
                \twoheadsrightarrow%
                (\CoEq(f,g),[y_{0}])%
            \]%
            given by the quotient map, as in \ChapterRef{\ChapterConstructionsWithSets, \cref{constructions-with-sets:construction-of-coequalisers-of-sets-the-cocone} of \cref{constructions-with-sets:construction-of-coequalisers-of-sets}}{\cref{construction-of-coequalisers-of-sets-the-cocone} of \cref{construction-of-coequalisers-of-sets}}.
    \end{itemize}
\end{construction}
\begin{Proof}{Proof of \cref{construction-of-coequalisers-of-pointed-sets}}%
    We claim that $(\CoEq(f,g),[y_{0}])$ is the categorical coequaliser of $f$ and $g$ in $\Sets_{*}$. First we need to check that the relevant coequaliser diagram commutes, i.e.\ that we have
    \[
        \coeq(f,g)\circ f%
        =%
        \coeq(f,g)\circ g.%
    \]%
    Indeed, we have
    \begin{align*}
        [\coeq(f,g)\circ f](x) &\defeq [\coeq(f,g)](f(x))\\
                               &\defeq [f(x)]\\
                               &=      [g(x)]\\
                               &\defeq [\coeq(f,g)](g(x))\\
                               &\defeq [\coeq(f,g)\circ g](x)%
    \end{align*}
    for each $x\in X$. Next, we prove that $\CoEq(f,g)$ satisfies the universal property of the coequaliser. Suppose we have a diagram of the form
    \[
        \begin{tikzcd}[row sep={5.0*\the\DL,between origins}, column sep={6.0*\the\DL,between origins}, background color=backgroundColor, ampersand replacement=\&]
            {(X,x_{0})}
            \arrow[r,"f", shift left =0.8]
            \arrow[r,"g"',shift right=0.8]
            \&
            {(Y,y_{0})}
            \arrow[r,"{\coeq(f,g)}",hook]
            \arrow[rd,"c"']
            \&[4.0*\the\DL]
            {(\CoEq(f,g),[y_{0}])}
            \\
            \&
            \&[4.0*\the\DL]
            {(C,*)}
        \end{tikzcd}
    \]%
    in $\Sets$. Then, since $c(f(a))=c(g(a))$ for each $a\in A$, it follows from \ChapterRef{\ChapterConditionsOnRelations, \cref{conditions-on-relations:properties-of-quotient-sets-descending-functions-to-quotient-sets-1,conditions-on-relations:properties-of-quotient-sets-descending-functions-to-quotient-sets-2} of \cref{conditions-on-relations:properties-of-quotient-sets}}{\cref{properties-of-quotient-sets-descending-functions-to-quotient-sets-1,properties-of-quotient-sets-descending-functions-to-quotient-sets-2} of \cref{properties-of-quotient-sets}} that there exists a unique map $\phi\colon\CoEq(f,g)\uearrow C$ making the diagram
    \[
        \begin{tikzcd}[row sep={5.0*\the\DL,between origins}, column sep={6.0*\the\DL,between origins}, background color=backgroundColor, ampersand replacement=\&]
            {(X,x_{0})}
            \arrow[r,"f", shift left =0.8]
            \arrow[r,"g"',shift right=0.8]
            \&
            {(Y,y_{0})}
            \arrow[r,"{\coeq(f,g)}",hook]
            \arrow[rd,"c"']
            \&[4.0*\the\DL]
            {(\CoEq(f,g),[y_{0}])}
            \arrow[d,"\phi"'{pos=0.475},"\exists!"{pos=0.475},dashed]
            \\
            \&
            \&[4.0*\the\DL]
            {(C,*)}
        \end{tikzcd}
    \]%
    commute, where we note that $\phi$ is indeed a morphism of pointed sets since
    \begin{align*}
        \phi([y_{0}]) &= [\phi\circ\coeq(f,g)]([y_{0}])\\
                      &= c([y_{0}])\\
                      &= *,
    \end{align*}
    where we have used that $c$ is a morphism of pointed sets.
\end{Proof}
\begin{proposition}{Properties of Coequalisers of Pointed Sets}{properties-of-coequalisers-of-pointed-sets}%
    Let $(X,x_{0})$ and $(Y,y_{0})$ be pointed sets and let $f,g,h\colon(X,x_{0})\to(Y,y_{0})$ be morphisms of pointed sets.
    \begin{enumerate}
        \item\label{properties-of-coequalisers-of-pointed-sets-associativity}\SloganFont{Associativity. }We have isomorphisms of pointed sets%
            \begin{envscriptsize}
                \[
                    \underbrace{\CoEq(\coeq(f,g)\circ f,\coeq(f,g)\circ h)}_{{}=\CoEq(\coeq(f,g)\circ g,\coeq(f,g)\circ h)}%
                    \cong
                    \CoEq(f,g,h)
                    \cong
                    \underbrace{\CoEq(\coeq(g,h)\circ f,\coeq(g,h)\circ g)}_{{}=\CoEq(\coeq(g,h)\circ f,\coeq(g,h)\circ h)},%
                \]%
            \end{envscriptsize}
            where $\CoEq(f,g,h)$ is the colimit of the diagram
            \[
                \begin{tikzcd}[row sep={5.0*\the\DL,between origins}, column sep={6.0*\the\DL,between origins}, background color=backgroundColor, ampersand replacement=\&]
                    {(X,x_{0})}%
                    \arrow[r,"f",shift left=2.0]%
                    \arrow[r,"g"description]%
                    \arrow[r,"h"',shift right=2.0]%
                    \&
                    {(Y,y_{0})}%
                \end{tikzcd}
            \]%
            in $\Sets_{*}$.
        \item\label{properties-of-coequalisers-of-pointed-sets-unitality}\SloganFont{Unitality. }We have an isomorphism of pointed sets
            \[
                \CoEq(f,f)%
                \cong%
                B.%
            \]%
        \item\label{properties-of-coequalisers-of-pointed-sets-commutativity}\SloganFont{Commutativity. }We have an isomorphism of pointed sets
            \[
                \CoEq(f,g)
                \cong
                \CoEq(g,f).
            \]%
        %\item\label{properties-of-coequalisers-of-pointed-sets-}\SloganFont{. }
    \end{enumerate}
\end{proposition}
\begin{Proof}{Proof of \cref{properties-of-coequalisers-of-pointed-sets}}%
    \FirstProofBox{\cref{properties-of-coequalisers-of-pointed-sets-associativity}: Associativity}%
    This follows from \ChapterRef{\ChapterConstructionsWithSets, \cref{constructions-with-sets:properties-of-coequalisers-of-sets-associativity} of \cref{constructions-with-sets:properties-of-coequalisers-of-sets}}{\cref{properties-of-coequalisers-of-sets-associativity} of \cref{properties-of-coequalisers-of-sets}}.

    \ProofBox{\cref{properties-of-coequalisers-of-pointed-sets-unitality}: Unitality}%
    This follows from \ChapterRef{\ChapterConstructionsWithSets, \cref{constructions-with-sets:properties-of-coequalisers-of-sets-unitality} of \cref{constructions-with-sets:properties-of-coequalisers-of-sets}}{\cref{properties-of-coequalisers-of-sets-unitality} of \cref{properties-of-coequalisers-of-sets}}.

    \ProofBox{\cref{properties-of-coequalisers-of-pointed-sets-commutativity}: Commutativity}%
    This follows from \ChapterRef{\ChapterConstructionsWithSets, \cref{constructions-with-sets:properties-of-coequalisers-of-sets-commutativity} of \cref{constructions-with-sets:properties-of-coequalisers-of-sets}}{\cref{properties-of-coequalisers-of-sets-commutativity} of \cref{properties-of-coequalisers-of-sets}}.
\end{Proof}
\section{Constructions With Pointed Sets}\label{section-constructions-with-pointed-sets}
\subsection{Free Pointed Sets}\label{subsection-free-pointed-sets}
Let $X$ be a set.
\begin{definition}{Free Pointed Sets}{free-pointed-sets}%
    The \index[set-theory]{pointed set!free}\textbf{free pointed set on $X$} is the pointed set \index[notation]{Xplus@$X^{+}$}$\smash{X^{+}}$ consisting of:
    \begin{itemize}
        \item\SloganFont{The Underlying Set. }The set $X^{+}$ defined by%
            %--- Begin Footnote ---%
            \footnote{%
                \SloganFont{Further Notation: }We sometimes write \index[notation]{pointX@$\point_{X}$}$\point_{X}$ for the basepoint of $X^{+}$ for clarity, specially when there are multiple free pointed sets involved in the current discussion.
                \par\vspace*{\TCBBoxCorrection}
            }%
            %---  End Footnote  ---%
            \begin{align*}
                X^{+} &\defeq X\icoprod\pt\\
                      &\defeq X\icoprod\{\point\}.
            \end{align*}
        \item\SloganFont{The Basepoint. }The element $\point$ of $X^{+}$.
    \end{itemize}
\end{definition}
\begin{proposition}{Properties of Free Pointed Sets}{properties-of-free-pointed-sets}%
    Let $X$ be a set.
    \begin{enumerate}
        \item\label{properties-of-free-pointed-sets-functoriality}\SloganFont{Functoriality. }The assignment $X\mapsto X^{+}$ defines a functor
            \[
                (-)^{+}
                \colon
                \Sets
                \to
                \Sets_{*},
            \]%
            where:
            \begin{itemize}
                \item\SloganFont{Action on Objects. }For each $X\in\Obj(\Sets)$, we have
                    \[
                        [(-)^{+}](X)%
                        \defeq%
                        X^{+},
                    \]%
                    where $X^{+}$ is the pointed set of \cref{free-pointed-sets}.
                \item\SloganFont{Action on Morphisms. }For each morphism $f\colon X\to Y$ of $\Sets$, the image
                    \[
                        f^{+}%
                        \colon%
                        X^{+}%
                        \to%
                        Y^{+}%
                    \]%
                    of $f$ by $(-)^{+}$ is the map of pointed sets defined by
                    \[
                        f^{+}(x)
                        \defeq
                        \begin{cases}
                            f(x)       &\text{if $x\in X$,}\\
                            \point_{Y} &\text{if $x=\point_{X}$.}
                        \end{cases}
                    \]%
            \end{itemize}
        \item\label{properties-of-free-pointed-sets-adjointness}\SloganFont{Adjointness. }We have an adjunction
            \begin{webcompile}
                \Adjunction#{(-)^{+}}#\Wasureru#\Sets#\Sets_{*},#
            \end{webcompile}%
            witnessed by a bijection of sets
            \begin{align*}
                \Sets_{*}((X^{+},\point_{X}),(Y,y_{0}))%
                \cong%
                \Sets(X,Y),%
            \end{align*}
            natural in $X\in\Obj(\Sets)$ and $(Y,y_{0})\in\Obj(\Sets_{*})$.
        \item\label{properties-of-free-pointed-sets-symmetric-strong-monoidality-with-respect-to-wedge-sums}\SloganFont{Symmetric Strong Monoidality With Respect to Wedge Sums. }The free pointed set functor of \cref{properties-of-free-pointed-sets-functoriality} has a symmetric strong monoidal structure
            \[
                ((-)^{+},(-)^{+,\icoprod},(-)^{+,\icoprod}_{\Unit})
                \colon
                (\Sets,\icoprod,\emptyset)
                \to
                (\Sets_{*},\vee,\pt),
            \]%
            being equipped with isomorphisms of pointed sets%
            \[
                \begin{gathered}
                    (-)^{+,\icoprod}_{X,Y}   \colon X^{+}\vee Y^{+} \isorightarrow (X\icoprod Y)^{+},\\
                    (-)^{+,\icoprod}_{\Unit} \colon \pt             \isorightarrow \emptyset^{+},
                \end{gathered}
            \]%
            natural in $X,Y\in\Obj(\Sets)$.
        \item\label{properties-of-free-pointed-sets-symmetric-strong-monoidality-with-respect-to-smash-products}\SloganFont{Symmetric Strong Monoidality With Respect to Smash Products. }The free pointed set functor of \cref{properties-of-free-pointed-sets-functoriality} has a symmetric strong monoidal structure
            \[
                ((-)^{+},(-)^{+},(-)^{+}_{\Unit})
                \colon
                (\Sets,\times,\pt)
                \to
                (\Sets_{*},\wedge,S^{0}),
            \]%
            being equipped with isomorphisms of pointed sets%
            \[
                \begin{gathered}
                    (-)^{+}_{X,Y}   \colon X^{+}\wedge Y^{+} \isorightarrow (X\times Y)^{+},\\
                    (-)^{+}_{\Unit} \colon S^{0}             \isorightarrow \pt^{+},
                \end{gathered}
            \]%
            natural in $X,Y\in\Obj(\Sets)$.
        %\item\label{properties-of-free-pointed-sets-}\SloganFont{. }
    \end{enumerate}
\end{proposition}
\begin{Proof}{Proof of \cref{properties-of-free-pointed-sets}}%
    \FirstProofBox{\cref{properties-of-free-pointed-sets-functoriality}: Functoriality}%
    We claim that $(-)^{+}$ is indeed a functor:
    \begin{itemize}
        \item\SloganFont{Preservation of Identities. }Let $X\in\Obj(\Sets)$. We have
            \[
                \id^{+}_{X}(x)%
                \defeq%
                \begin{cases}
                    x         &\text{if $x\in X$,}\\
                    \star_{X} &\text{if $x=\star_{X}$,}
                \end{cases}
            \]%
            for each $x\in X^{+}$, so $\id^{+}_{X}=\id_{X^{+}}$.
        \item\SloganFont{Preservation of Composition. }Given morphisms of sets
            \begin{align*}
                f &\colon X \to Y,\\
                g &\colon Y \to Z,
            \end{align*}
            we have
            \begin{align*}
                [g^{+}\circ f^{+}](x) &\defeq g^{+}(f^{+}(x))\\
                                      &\defeq g^{+}(f(x))\\
                                      &\defeq g(f(x))\\
                                      &\defeq [g\circ f]^{+}(x)%
            \end{align*}
            for each $x\in X$ and
            \begin{align*}
                [g^{+}\circ f^{+}](\star_{X}) &\defeq g^{+}(f^{+}(\star_{X}))\\
                                              &\defeq g^{+}(\star_{Y})\\
                                              &\defeq \star_{Z}\\
                                              &\defeq [g\circ f]^{+}(\star_{X}),
            \end{align*}
            so $(g\circ f)^{+}=g^{+}\circ f^{+}$.
    \end{itemize}
    This finishes the proof.

    \ProofBox{\cref{properties-of-free-pointed-sets-adjointness}: Adjointness}%
    We proceed in a few steps:
    \begin{itemize}
        \item\label{proof-of-properties-of-free-pointed-sets-adjointness-1}\SloganFont{Map \rmI. }We define a map
            \[
                \Phi_{X,Y}%
                \colon%
                \Sets_{*}(X^{+},Y)%
                \to%
                \Sets(X,Y)%
            \]%
            by sending a morphism of pointed sets
            \[
                \xi%
                \colon%
                (X^{+},\point_{X})%
                \to%
                (Y,y_{0})%
            \]%
            to the function%
            \[
                \xi^{\dagger}%
                \colon%
                X%
                \to%
                Y%
            \]%
            given by
            \[
                \xi^{\dagger}(x)%
                \defeq
                \xi(x)%
            \]%
            for each $x\in X$.
        \item\label{proof-of-properties-of-free-pointed-sets-adjointness-2}\SloganFont{Map \rmII. }We define a map
            \[
                \Psi_{X,Y}%
                \colon%
                \Sets(X,Y)%
                \to%
                \Sets_{*}(X^{+},Y)%
            \]%
            given by sending a function $\xi\colon X\to Y$ to the morphism of pointed sets
            \[
                \xi^{\dagger}%
                \colon%
                (X^{+},\point_{X})%
                \to%
                (Y,y_{0})%
            \]%
            defined by
            \[
                \xi^{\dagger}(x)%
                \defeq%
                \begin{cases}
                    \xi(x) & \text{if $x\in X$,}\\
                    y_{0}  & \text{if $x=\point_{X}$}
                \end{cases}
            \]%
            for each $x\in X^{+}$.
        \item\label{proof-of-properties-of-free-pointed-sets-adjointness-3}\SloganFont{Invertibility \rmI. }Given a morphism of pointed sets
            \[
                \xi%
                \colon%
                (X^{+},\star_{X})%
                \to%
                (Y,y_{0}),%
            \]%
            we have
            \begin{align*}
                [\Psi_{X,Y}\circ\Phi_{X,Y}](\xi) &\defeq \Psi_{X,Y}(\Phi_{X,Y}(\xi))\\%
                                                 &=      \Psi_{X,Y}(\xi^{\dagger})\\%
                                                 &\defeq \leftllbracket x\mapsto{\begin{cases}\xi^{\dagger}(x)&\text{if $x\in X$}\\y_{0}&\text{if $x=\star_{X}$}\end{cases}}\rightrrbracket\\%
                                                 &=      \leftllbracket x\mapsto{\begin{cases}\xi(x)&\text{if $x\in X$}\\y_{0}&\text{if $x=\star_{X}$}\end{cases}}\rightrrbracket\\%
                                                 &=      \xi\\%
                                                 &\defeq [\id_{\Sets_{*}(X^{+},Y)}](\xi).%
            \end{align*}
            Therefore we have
            \[
                \Psi_{X,Y}\circ\Phi_{X,Y}%
                =%
                \id_{\Sets_{*}(X^{+},Y)}.%
            \]%
        \item\label{proof-of-properties-of-free-pointed-sets-adjointness-4}\SloganFont{Invertibility \rmII. }Given a map of sets $\xi\colon X\to Y$, we have
            \begin{align*}
                [\Phi_{X,Y}\circ\Psi_{X,Y}](\xi) &\defeq \Phi_{X,Y}(\Psi_{X,Y}(\xi))\\%
                                                 &=      \Phi_{X,Y}(\xi^{\dagger})\\%
                                                 &=      \Phi_{X,Y}(\leftllbracket x\mapsto{\begin{cases}\xi(x)&\text{if $x\in X$}\\y_{0}&\text{if $x=\star_{X}$}\end{cases}}\rightrrbracket)\\%
                                                 &=      \llbracket x\mapsto\xi(x)\rrbracket\\%
                                                 &=      \xi\\%
                                                 &\defeq [\id_{\Sets(X,Y)}](\xi).%
            \end{align*}
            Therefore we have
            \[
                \Phi_{X,Y}\circ\Psi_{X,Y}%
                =%
                \id_{\Sets(X,Y)}.%
            \]%
        \item\label{proof-of-properties-of-free-pointed-sets-adjointness-5}\SloganFont{Naturality for $\Phi$, Part \rmI. }We need to show that, given a morphism of pointed sets
            \[
                f%
                \colon%
                (X,x_{0})%
                \to%
                (X',x'_{0}),%
            \]%
            the diagram
            \[
                \begin{tikzcd}[row sep={5.0*\the\DL,between origins}, column sep={9.5*\the\DL,between origins}, background color=backgroundColor, ampersand replacement=\&]
                    \Sets_{*}(X^{\prime,+},Y)%
                    \arrow[r,"\Phi_{X',Y}"]
                    \arrow[d,"f^{*}"']
                    \&
                    \Sets(X',Y)
                    \arrow[d,"f^{*}"]
                    \\
                    \Sets_{*}(X^{+},Y)%
                    \arrow[r,"\Phi_{X,Y}"']
                    \&
                    \Sets(X,Y)
                \end{tikzcd}
            \]%
            commutes. Indeed, given a morphism of pointed sets $\xi\colon X^{\prime,+}\to Y$, we have
            \begin{align*}
                [\Phi_{X,Y}\circ f^{*}](\xi) &= \Phi_{X,Y}(f^{*}(\xi))\\
                                             &= \Phi_{X,Y}(\xi\circ f)\\
                                             &= \xi\circ f\\
                                             &= \Phi_{X',Y}(\xi)\circ f\\
                                             &= f^{*}(\Phi_{X',Y}(\xi))\\
                                             &= f^{*}(\Phi_{X',Y}(\xi))\\
                                             &= [f^{*}\circ\Phi_{X',Y}](\xi).
            \end{align*}
            Therefore we have
            \[
                \Phi_{X,Y}\circ f^{*}%
                =%
                f^{*}\circ\Phi_{X',Y}%
            \]%
            and the naturality diagram for $\Phi$ above indeed commutes.
        \item\label{proof-of-properties-of-free-pointed-sets-adjointness-6}\SloganFont{Naturality for $\Phi$, Part \rmII. }We need to show that, given a morphism of pointed sets
            \[
                g%
                \colon%
                (Y,y_{0})%
                \to%
                (Y',y'_{0}),%
            \]%
            the diagram
            \[
                \begin{tikzcd}[row sep={5.0*\the\DL,between origins}, column sep={9.5*\the\DL,between origins}, background color=backgroundColor, ampersand replacement=\&]
                    \Sets_{*}(X^{+},Y)%
                    \arrow[r,"\Phi_{X,Y}"]
                    \arrow[d,"{g_{*}}"']
                    \&
                    \Sets(X,Y)
                    \arrow[d,"{g_{*}}"]
                    \\
                    \Sets_{*}(X^{+},Y'),%
                    \arrow[r,"\Phi_{X,Y'}"']
                    \&
                    \Sets(X,Y')%
                \end{tikzcd}
            \]%
            commutes. Indeed, given a morphism of pointed sets
            \[
                \xi^{\dagger}%
                \colon%
                X^{+}
                \to%
                Y,%
            \]%
            we have
            \begin{align*}
                [\Phi_{X,Y'}\circ g_{*}](\xi) &= \Phi_{X,Y'}(g_{*}(\xi))\\
                                              &= \Phi_{X,Y'}(g\circ\xi)\\
                                              &= g\circ\xi\\
                                              &= g\circ\Phi_{X,Y'}(\xi)\\
                                              &= g_{*}(\Phi_{X,Y'}(\xi))\\
                                              &= [g_{*}\circ\Phi_{X,Y'}](\xi).
            \end{align*}
            Therefore we have
            \[
                \Phi_{X,Y'}\circ g_{*}%
                =%
                g_{*}\circ\Phi_{X,Y'}%
            \]%
            and the naturality diagram for $\Phi$ above indeed commutes.
        \item\label{proof-of-properties-of-free-pointed-sets-adjointness-7}\SloganFont{Naturality for $\Psi$. }Since $\Phi$ is natural in each argument and $\Phi$ is a componentwise inverse to $\Psi$ in each argument, it follows from \ChapterRef{\ChapterCategories, \cref{categories:properties-of-natural-isomorphisms-componentwise-inverses-of-natural-transformations-assemble-into-natural-transformations} of \cref{categories:properties-of-natural-isomorphisms}}{\cref{properties-of-natural-isomorphisms-componentwise-inverses-of-natural-transformations-assemble-into-natural-transformations} of \cref{properties-of-natural-isomorphisms}} that $\Psi$ is also natural in each argument.
    \end{itemize}
    This finishes the proof.

    \ProofBox{\cref{properties-of-free-pointed-sets-symmetric-strong-monoidality-with-respect-to-wedge-sums}: Symmetric Strong Monoidality With Respect to Wedge Sums}%
    We construct the strong monoidal structure on $(-)^{+}$ with respect to $\icoprod$ and $\vee$ as follows:
    \begin{itemize}
        \item\SloganFont{The Strong Monoidality Constraints. }The isomorphism
            \[
                (-)^{+,\icoprod}_{X,Y}%
                \colon%
                X^{+}\vee Y^{+}%
                \isorightarrow
                (X\icoprod Y)^{+}
            \]%
            is given by
            \[
                (-)^{+,\icoprod}_{X,Y}(z)%
                =%
                \begin{cases}%
                    x                   &\text{if $z=[(0,x)]$ with $x\in X$,}\\%
                    y                   &\text{if $z=[(1,y)]$ with $y\in Y$,}\\%
                    \star_{X\icoprod Y} &\text{if $z=[(0,\star_{X})]$,}\\%
                    \star_{X\icoprod Y} &\text{if $z=[(1,\star_{Y})]$}%
                \end{cases}%
            \]%
            for each $z\in X^{+}\vee Y^{+}$, with inverse
            \[
                (-)^{+,\icoprod,-1}_{X,Y}
                \colon%
                (X\icoprod Y)^{+}
                \isorightarrow
                X^{+}\vee Y^{+}%
            \]%
            given by
            \[
                (-)^{+,\icoprod,-1}_{X,Y}(z)%
                \defeq%
                \begin{cases}
                    [(0,x)] &\text{if $z=[(0,x)]$,}\\
                    [(1,y)] &\text{if $z=[(1,y)]$,}\\
                    p_{0} &\text{if $z=\star_{X\icoprod Y}$}
                \end{cases}
            \]%
            for each $z\in(X\icoprod Y)^{+}$.
        \item\SloganFont{The Strong Monoidal Unity Constraint. }The isomorphism
            \[
                (-)^{+,\icoprod,\Unit}_{X,Y}%
                \colon%
                \pt%
                \isorightarrow
                \emptyset^{+}%
            \]%
            is given by sending $\point_{X}$ to $\star_{\emptyset}$.
    \end{itemize}
    The verification that these isomorphisms satisfy the coherence conditions making the functor $(-)^{+}$ into a symmetric strong monoidal functor is omitted.

    \ProofBox{\cref{properties-of-free-pointed-sets-symmetric-strong-monoidality-with-respect-to-smash-products}: Symmetric Strong Monoidality With Respect to Smash Products}%
    We construct the strong monoidal structure on $(-)^{+}$ with respect to $\times$ and $\wedge$ as follows:
    \begin{itemize}
        \item\SloganFont{The Strong Monoidality Constraints. }The isomorphism
            \[
                (-)^{+}_{X,Y}%
                \colon%
                X^{+}\wedge Y^{+}%
                \isorightarrow
                (X\times Y)^{+}
            \]%
            is given by
            \[
                (-)^{+}_{X,Y}(x\wedge y)%
                =%
                \begin{cases}%
                    (x,y)             &\text{if $x\neq\star_{X}$ and $y\neq\star_{Y}$}\\%
                    \star_{X\times Y} &\text{otherwise}%
                \end{cases}%
            \]%
            for each $x\wedge y\in X^{+}\wedge Y^{+}$, with inverse
            \[
                (-)^{+,-1}_{X,Y}
                \colon%
                (X\times Y)^{+}
                \isorightarrow
                X^{+}\wedge Y^{+}%
            \]%
            given by
            \[
                (-)^{+,-1}_{X,Y}(z)%
                \defeq%
                \begin{cases}
                    x\wedge y                &\text{if $z=(x,y)$ with $(x,y)\in X\times Y$,}\\
                    \star_{X}\wedge\star_{Y} &\text{if $z=\star_{X\times Y}$,}
                \end{cases}
            \]%
            for each $z\in(X\times Y)^{+}$.
        \item\SloganFont{The Strong Monoidal Unity Constraint. }The isomorphism
            \[
                (-)^{+,\Unit}_{X,Y}%
                \colon%
                S^{0}%
                \isorightarrow
                \pt^{+}%
            \]%
            is given by sending $0$ to $\star_{\pt}$ and $1$ to $\point$, where $\pt^{+}=\{\point,\star_{\pt}\}$.
    \end{itemize}
    The verification that these isomorphisms satisfy the coherence conditions making the functor $(-)^{+}$ into a symmetric strong monoidal functor is omitted.
\end{Proof}
\subsection{Deleting Basepoints}\label{subsection-deleting-basepoints}
Let $(X,x_{0})$ be a pointed set.
\begin{definition}{Sets With Deleted Basepoints}{sets-with-deleted-basepoints}%
    The \index[set-theory]{pointed set!deleting the basepoint}\textbf{set with deleted basepoint associated to $X$} is the set \index[notation]{Xminus@$X^{-}$}$\smash{X^{-}}$ defined by
    \[
        X^{-}%
        \defeq%
        X\setminus\{x_{0}\}.%
    \]%
\end{definition}
\begin{proposition}{Properties of Sets With Deleted Basepoints}{properties-of-sets-with-deleted-basepoints}%
    Let $(X,x_{0})$ be a pointed set.
    \begin{enumerate}
        \item\label{properties-of-sets-with-deleted-basepoints-functoriality}\SloganFont{Functoriality. }The assignment $(X,x_{0})\mapsto X^{-}$ defines a functor
            \[
                X^{-}%
                \colon%
                \Sets^{\actv}_{*}%
                \to%
                \Sets,%
            \]%
            where:
            \begin{itemize}
                \item\SloganFont{Action on Objects. }For each $X\in\Obj(\Sets^{\actv}_{*})$, we have
                    \[
                        [(-)^{-}](X)%
                        \defeq%
                        X^{-},
                    \]%
                    where $X^{-}$ is the set of \cref{sets-with-deleted-basepoints}.
                \item\SloganFont{Action on Morphisms. }For each morphism $f\colon X\to Y$ of $\Sets^{\actv}_{*}$, the image
                    \[
                        f^{-}%
                        \colon%
                        X^{-}%
                        \to%
                        Y^{-}%
                    \]%
                    of $f$ by $(-)^{-}$ is the map defined by
                    \[
                        f^{-}(x)
                        \defeq
                        f(x)
                    \]%
                    for each $x\in X^{-}$.
            \end{itemize}
        \item\label{properties-of-sets-with-deleted-basepoints-adjoint-equivalence}\SloganFont{Adjoint Equivalence. }We have an adjoint equivalence of categories
            \begin{webcompile}
                \AdjointEquivalence#{(-)^{-}}#(-)^{+}#\Sets^{\actv}_{*}#\Sets,#
            \end{webcompile}%
            witnessed by a bijection of sets
            \begin{align*}
                \Sets(X^{-},Y)%
                \cong%
                \Sets_{*}(X,Y^{+}),%
            \end{align*}
            natural in $X\in\Obj(\Sets_{*})$ and $Y\in\Obj(\Sets)$, and by isomorphisms
            \begin{align*}
                (X^{-})^{+} &\cong X,\\
                (Y^{+})^{-} &\cong Y,
            \end{align*}
            once again natural in $X\in\Obj(\Sets_{*})$ and $Y\in\Obj(\Sets)$.
        \item\label{properties-of-sets-with-deleted-basepoints-symmetric-strong-monoidality-with-respect-to-wedge-sums}\SloganFont{Symmetric Strong Monoidality With Respect to Wedge Sums. }The functor of \cref{properties-of-sets-with-deleted-basepoints-functoriality} has a symmetric strong monoidal structure
            \[
                ((-)^{-},(-)^{-,\vee},(-)^{-,\vee}_{\Unit})
                \colon
                (\Sets^{\actv}_{*},\vee,\pt),
                \to
                (\Sets,\icoprod,\emptyset),
            \]%
            being equipped with isomorphisms of pointed sets%
            \[
                \begin{gathered}
                    (-)^{-,\vee}_{X,Y}   \colon X^{-}\icoprod Y^{-} \isorightarrow (X\vee Y)^{-},\\
                    (-)^{-,\vee}_{\Unit} \colon \emptyset           \isorightarrow \pt^{-},
                \end{gathered}
            \]%
            natural in $X,Y\in\Obj(\Sets)$.
        \item\label{properties-of-sets-with-deleted-basepoints-symmetric-strong-monoidality-with-respect-to-smash-products}\SloganFont{Symmetric Strong Monoidality With Respect to Smash Products. }The free pointed set functor of \cref{properties-of-sets-with-deleted-basepoints-functoriality} has a symmetric strong monoidal structure
            \[
                ((-)^{-},(-)^{-,\times},(-)^{-,\times}_{\Unit})
                \colon
                (\Sets^{\actv}_{*},\wedge,S^{0}),
                \to
                (\Sets,\times,\pt)
            \]%
            being equipped with isomorphisms of pointed sets%
            \[
                \begin{gathered}
                    (-)^{-}_{X,Y}   \colon X^{-}\times Y^{-} \isorightarrow (X\wedge Y)^{-},\\
                    (-)^{-}_{\Unit} \colon \pt               \isorightarrow (S^{0})^{-},
                \end{gathered}
            \]%
            natural in $X,Y\in\Obj(\Sets)$.
        %\item\label{properties-of-sets-with-deleted-basepoints-}\SloganFont{. }
    \end{enumerate}
\end{proposition}
\begin{Proof}{Proof of \cref{properties-of-sets-with-deleted-basepoints}}%
    \FirstProofBox{\cref{properties-of-sets-with-deleted-basepoints-functoriality}: Functoriality}%
    We claim that $(-)^{-}$ is indeed a functor:
    \begin{itemize}
        \item\SloganFont{Preservation of Identities. }Let $X\in\Obj(\Sets)$. We have
            \[
                \id^{-}_{X}(x)%
                \defeq%
                x
            \]%
            for each $x\in X^{-}$, so $\id^{-}_{X}=\id_{X^{-}}$.
        \item\SloganFont{Preservation of Composition. }Given morphisms of pointed sets
            \begin{align*}
                f &\colon (X,x_{0}) \to (Y,y_{0}),\\
                g &\colon (Y,y_{0}) \to (Z,z_{0}),
            \end{align*}
            we have
            \begin{align*}
                [g^{-}\circ f^{-}](x) &\defeq g^{-}(f^{-}(x))\\
                                      &\defeq g^{-}(f(x))\\
                                      &\defeq g(f(x))\\
                                      &\defeq [g\circ f]^{-}(x)%
            \end{align*}
            for each $x\in X$, so $(g\circ f)^{-}=g^{-}\circ f^{-}$.
    \end{itemize}
    This finishes the proof.

    \ProofBox{\cref{properties-of-sets-with-deleted-basepoints-adjoint-equivalence}: Adjoint Equivalence}%
    We proceed in a few steps:
    \begin{enumerate}
        \item\label{proof-of-properties-of-sets-with-deleted-basepoints-adjoint-equivalence-1}\SloganFont{Map \rmI. }We define a map
            \[
                \Phi_{X,Y}%
                \colon%
                \Sets(X^{-},Y)%
                \to%
                \Sets^{\actv}_{*}(X,Y^{+})%
            \]%
            by sending a map $\xi\colon X^{-}\to Y$ to the active morphism of pointed sets%
            \[
                \xi^{\dagger}%
                \colon%
                X%
                \to%
                Y^{+}%
            \]%
            given by
            \[
                \xi^{\dagger}(x)%
                \defeq
                \begin{cases}
                    \xi(x)    &\text{if $x\in X^{-}$,}\\
                    \star_{Y} &\text{if $x=x_{0}$,}
                \end{cases}
            \]%
            for each $x\in X$, where this morphism is indeed active since $\xi(x)\in Y=Y^{+}\setminus\{\star_{Y}\}$ for all $x\in X^{-}$.
        \item\label{proof-of-properties-of-sets-with-deleted-basepoints-adjoint-equivalence-2}\SloganFont{Map \rmII. }We define a map
            \[
                \Psi_{X,Y}%
                \colon%
                \Sets^{\actv}_{*}(X,Y^{+})%
                \to%
                \Sets(X^{-},Y)%
            \]%
            given by sending an active morphism of pointed sets $\xi\colon X\to Y^{+}$ to the map
            \[
                \xi^{\dagger}%
                \colon%
                X^{-}%
                \to%
                Y%
            \]%
            defined by
            \[
                \xi^{\dagger}(x)%
                \defeq%
                \xi(x)
            \]%
            for each $x\in X^{-}$, which is indeed well-defined (in that $\xi(x)\in Y$ for all $x\in X^{-}$) since $\xi$ is active.
        \item\label{proof-of-properties-of-sets-with-deleted-basepoints-adjoint-equivalence-3}\SloganFont{Invertibility \rmI. }Given a map of sets $\xi\colon X^{-}\to Y$, we have
            \begin{align*}
                [\Psi_{X,Y}\circ\Phi_{X,Y}](\xi) &\defeq \Psi_{X,Y}(\Phi_{X,Y}(\xi))\\%
                                                 &\defeq \Psi_{X,Y}(\leftllbracket x\mapsto{\begin{cases}\xi(x)&\text{if $x\in X^{-}$}\\\star_{Y}&\text{if $x=x_{0}$}\end{cases}}\rightrrbracket)\\%
                                                 &=      \llbracket x\mapsto\xi(x)\rrbracket\\
                                                 &=      \xi\\
                                                 &=      [\id_{\Sets(X^{-},Y)}](\xi).
            \end{align*}
            Therefore we have
            \[
                \Psi_{X,Y}\circ\Phi_{X,Y}%
                =%
                \id_{\Sets(X^{-},Y)}.%
            \]%
        \item\label{proof-of-properties-of-sets-with-deleted-basepoints-adjoint-equivalence-4}\SloganFont{Invertibility \rmII. }Given a morphism of pointed sets
            \[
                \xi%
                \colon%
                (X,x_{0})%
                \to%
                (Y^{+},\star_{Y}),%
            \]%
            we have
            \begin{align*}
                [\Phi_{X,Y}\circ\Psi_{X,Y}](\xi) &\defeq \Phi_{X,Y}(\Psi_{X,Y}(\xi))\\%
                                                 &=      \Phi_{X,Y}(\llbracket x\mapsto\xi(x)\rrbracket)\\%
                                                 &=      \leftllbracket x\mapsto{\begin{cases}\xi(x)&\text{if $x\in X^{-}$}\\\star_{Y}&\text{if $x=x_{0}$}\end{cases}}\rightrrbracket\\
                                                 &=      \xi\\
                                                 &=      [\id_{\Sets^{\actv}_{*}(X,Y^{+})}](\xi).
            \end{align*}
            Therefore we have
            \[
                \Phi_{X,Y}\circ\Psi_{X,Y}%
                =%
                \id_{\Sets^{\actv}_{*}(X,Y^{+})}.%
            \]%
        \item\label{proof-of-properties-of-sets-with-deleted-basepoints-adjoint-equivalence-5}\SloganFont{Naturality for $\Phi$, Part \rmI. }We need to show that, given a morphism of pointed sets
            \[
                f%
                \colon%
                (X,x_{0})%
                \to%
                (X',x'_{0}),%
            \]%
            the diagram
            \[
                \begin{tikzcd}[row sep={5.0*\the\DL,between origins}, column sep={10.0*\the\DL,between origins}, background color=backgroundColor, ampersand replacement=\&]
                    \Sets(X^{',-},Y)%
                    \arrow[r,"\Phi_{X',Y}"]
                    \arrow[d,"f^{*}"']
                    \&
                    \Sets^{\actv}_{*}(X',Y^{+})
                    \arrow[d,"f^{*}"]
                    \\
                    \Sets_{*}(X^{-},Y)%
                    \arrow[r,"\Phi_{X,Y}"']
                    \&
                    \Sets^{\actv}_{*}(X,Y^{+})
                \end{tikzcd}
            \]%
            commutes. Indeed, given a map of sets $\xi\colon X'\to Y$, we have
            \begin{align*}
                [\Phi_{X,Y}\circ f^{*}](\xi) &= \Phi_{X,Y}(f^{*}(\xi))\\
                                             &= \Phi_{X,Y}(\xi\circ f)\\
                                             &= \leftllbracket x\mapsto{\begin{cases}\xi(f(x))&\text{if $f(x)\in X^{\prime,-}$}\\\star_{Y}&\text{if $f(x)=x'_{0}$}\end{cases}}\rightrrbracket\\
                                             &= f^{*}(\leftllbracket x'\mapsto{\begin{cases}\xi(x')&\text{if $x'\in X^{\prime,-}$}\\\star_{Y}&\text{if $x'=x'_{0}$}\end{cases}}\rightrrbracket)\\
                                             &= f^{*}(\Phi_{X',Y}(\xi))\\
                                             &= [f^{*}\circ\Phi_{X',Y}](\xi).
            \end{align*}
            Therefore we have
            \[
                \Phi_{X,Y}\circ f^{*}%
                =%
                f^{*}\circ\Phi_{X',Y},%
            \]%
            and the naturality diagram for $\Phi$ above indeed commutes.
        \item\label{proof-of-properties-of-sets-with-deleted-basepoints-adjoint-equivalence-6}\SloganFont{Naturality for $\Phi$, Part \rmII. }We need to show that, given a morphism of pointed sets
            \[
                g%
                \colon%
                (Y,y_{0})%
                \to%
                (Y',y'_{0}),%
            \]%
            the diagram
            \[
                \begin{tikzcd}[row sep={5.0*\the\DL,between origins}, column sep={10.0*\the\DL,between origins}, background color=backgroundColor, ampersand replacement=\&]
                    \Sets(X^{-},Y)
                    \arrow[r,"\Phi_{X,Y}"]
                    \arrow[d,"{g_{*}}"']
                    \&
                    \Sets^{\actv}_{*}(X,Y^{+})%
                    \arrow[d,"{g_{*}}"]
                    \\
                    \Sets(X^{-},Y')
                    \arrow[r,"\Phi_{X,Y'}"']
                    \&
                    \Sets^{\actv}_{*}(X,Y^{\prime,+})%
                \end{tikzcd}
            \]%
            commutes. Indeed, given a map of sets $\xi\colon X^{-}\to Y$, we have
            \begin{align*}
                [\Phi_{X,Y'}\circ g_{*}](\xi) &= \Phi_{X,Y'}(g_{*}(\xi))\\
                                              &= \Phi_{X,Y'}(g\circ\xi)\\
                                              &= \leftllbracket x\mapsto{\begin{cases}g(\xi(x))&\text{if $x\in X^{-}$}\\\star_{Y'}&\text{if $x=x_{0}$}\end{cases}}\rightrrbracket\\
                                              &= g_{*}(\leftllbracket x\mapsto{\begin{cases}\xi(x)&\text{if $x\in X^{-}$}\\\star_{Y}&\text{if $x=x_{0}$}\end{cases}}\rightrrbracket)\\
                                              &= g_{*}(\Phi_{X,Y'}(\xi))\\
                                              &= [g_{*}\circ\Phi_{X,Y'}](\xi).
            \end{align*}
            Therefore we have
            \[
                \Phi_{X,Y'}\circ g_{*}%
                =%
                g_{*}\circ\Phi_{X,Y'},%
            \]%
            and the naturality diagram for $\Phi$ above indeed commutes.
        \item\label{proof-of-properties-of-sets-with-deleted-basepoints-adjoint-equivalence-7}\SloganFont{Naturality for $\Psi$. }Since $\Phi$ is natural in each argument and $\Phi$ is a componentwise inverse to $\Psi$ in each argument, it follows from \ChapterRef{\ChapterCategories, \cref{categories:properties-of-natural-isomorphisms-componentwise-inverses-of-natural-transformations-assemble-into-natural-transformations} of \cref{categories:properties-of-natural-isomorphisms}}{\cref{properties-of-natural-isomorphisms-componentwise-inverses-of-natural-transformations-assemble-into-natural-transformations} of \cref{properties-of-natural-isomorphisms}} that $\Psi$ is also natural in each argument.
        \item\label{proof-of-properties-of-sets-with-deleted-basepoints-adjoint-equivalence-8}\SloganFont{Fully Faithfulness of $(-)^{-}$. }We aim to show that the assignment $f\mapsto f^{-}$ sets up a bijection
            \[
                (-)^{-}_{X,Y}%
                \colon%
                \Sets^{\actv}_{*}(X,Y)%
                \isorightarrow%
                \Sets(X^{-},Y^{-}).%
            \]%
            Indeed, the inverse map
            \[
                (-)^{-,-1}_{X,Y}%
                \colon%
                \Sets(X^{-},Y^{-})%
                \isorightarrow%
                \Sets^{\actv}_{*}(X,Y)%
            \]%
            is given by sending a map of sets $f\colon X^{-}\to Y^{-}$ to the active morphism of pointed sets $f^{\dagger}\colon X\to Y$ defined by
            \[
                f^{\dagger}(x)%
                \defeq%
                \begin{cases}
                    f(x)  &\text{if $x\in X^{-}$,}\\
                    y_{0} &\text{if $x=x_{0}$}
                \end{cases}
            \]%
            for each $x\in X$.
        \item\label{proof-of-properties-of-sets-with-deleted-basepoints-adjoint-equivalence-9}\SloganFont{Essential Surjectivity of $(-)^{-}$. }We need to show that, given an object $X\in\Obj(\Sets)$, there exists some $X'\in\Obj(\Sets^{\actv}_{*})$ such that $(X')^{-}\cong X$. Indeed, taking $X'=X^{+}$, we have
            \begin{align*}
                (X^{+})^{-} &\defeq (X\cup\{\star_{X}\})^{-}\\
                            &\defeq (X\cup\{\star_{X}\})\setminus\{\star_{X}\}\\
                            &=      X,
            \end{align*}
            and thus we have in fact an \emph{equality} $(X^{+})^{-}=X$, showing $(-)^{-}$ to be essentially surjective.
        \item\label{proof-of-properties-of-sets-with-deleted-basepoints-adjoint-equivalence-10}\SloganFont{The Functor $(-)^{-}$ Is an Equivalence. }Since $(-)^{-}$ is fully faithful and essentially surjective, it is an equivalence by \ChapterRef{\ChapterCategories, \cref{categories:properties-of-equivalences-of-categories-characterisations} of \cref{categories:properties-of-equivalences-of-categories}}{\cref{properties-of-equivalences-of-categories-characterisations} of \cref{properties-of-equivalences-of-categories}}.
    \end{enumerate}
    This finishes the proof.

    \ProofBox{\cref{properties-of-sets-with-deleted-basepoints-symmetric-strong-monoidality-with-respect-to-wedge-sums}: Symmetric Strong Monoidality With Respect to Wedge Sums}%
    We construct the strong monoidal structure on $(-)^{-}$ with respect to $\vee$ and $\icoprod$ as follows:
    \begin{itemize}
        \item\SloganFont{The Strong Monoidality Constraints. }The isomorphism
            \[
                (-)^{-,\vee}_{X,Y}%
                \colon%
                X^{-}\icoprod Y^{-}%
                \isorightarrow
                (X\vee Y)^{-}
            \]%
            is given by
            \[
                (-)^{-,\vee}_{X,Y}(z)%
                =%
                \begin{cases}%
                    [(0,x)] &\text{if $z=(0,x)$ with $x\in X$,}\\%
                    [(1,y)] &\text{if $z=(1,y)$ with $y\in Y$}%
                \end{cases}%
            \]%
            for each $z\in X^{-}\icoprod Y^{-}$, with inverse
            \[
                (-)^{-,\vee,-1}_{X,Y}
                \colon%
                (X\vee Y)^{-}
                \isorightarrow
                X^{-}\icoprod Y^{-}%
            \]%
            given by
            \[
                (-)^{-,\vee,-1}_{X,Y}(z)%
                \defeq%
                \begin{cases}
                    (0,x) &\text{if $z=[(0,x)]$,}\\
                    (1,y) &\text{if $z=[(1,y)]$,}
                \end{cases}
            \]%
            for each $z\in(X\vee Y)^{-}$.
        \item\SloganFont{The Strong Monoidal Unity Constraint. }The isomorphism
            \[
                (-)^{+,\vee,\Unit}_{X,Y}%
                \colon%
                \emptyset%
                \isorightarrow
                \pt^{-}%
            \]%
            is an equality.
    \end{itemize}
    The verification that these isomorphisms satisfy the coherence conditions making the functor $(-)^{-}$ into a symmetric strong monoidal functor is omitted.

    \ProofBox{\cref{properties-of-sets-with-deleted-basepoints-symmetric-strong-monoidality-with-respect-to-smash-products}: Symmetric Strong Monoidality With Respect to Smash Products}%
    We construct the strong monoidal structure on $(-)^{+}$ with respect to $\wedge$ and $\times$ as follows:
    \begin{itemize}
        \item\SloganFont{The Strong Monoidality Constraints. }The isomorphism
            \[
                (-)^{-}_{X,Y}%
                \colon%
                X^{-}\times Y^{-}%
                \isorightarrow
                (X\wedge Y)^{-}
            \]%
            is given by
            \[
                (-)^{-}_{X,Y}(x,y)%
                =%
                x\wedge y
            \]%
            for each $(x,y)\in X^{-}\times Y^{-}$, with inverse
            \[
                (-)^{-,-1}_{X,Y}
                \colon%
                (X\wedge Y)^{-}
                \isorightarrow
                X^{-}\times Y^{-}%
            \]%
            given by
            \[
                (-)^{-,-1}_{X,Y}(x\wedge y)%
                \defeq%
                (x,y)%
            \]%
            for each $x\wedge y\in(X\wedge Y)^{-}$.
        \item\SloganFont{The Strong Monoidal Unity Constraint. }The isomorphism
            \[
                (-)^{-,\Unit}_{X,Y}%
                \colon%
                \pt%
                \isorightarrow
                (S^{0})^{-}%
            \]%
            is given by sending $\point$ to $1$.
    \end{itemize}
    The verification that these isomorphisms satisfy the coherence conditions making the functor $(-)^{+}$ into a symmetric strong monoidal functor is omitted.
\end{Proof}
\begin{appendices}
\begin{multicols}{2}[\section{Other Chapters}]
\noindent
\textbf{Preliminaries}
\begin{enumerate}
\item \hyperref[introduction:section-phantom]{Introduction}
\end{enumerate}
\textbf{Sets}
\begin{enumerate}
\setcounter{enumi}{2}
\item \hyperref[sets:section-phantom]{Sets}
\item \hyperref[constructions-with-sets:section-phantom]{Constructions With Sets}
\item \hyperref[monoidal-structures-on-the-category-of-sets:section-phantom]{Monoidal Structures on the Category of Sets}
\item \hyperref[pointed-sets:section-phantom]{Pointed Sets}
\item \hyperref[tensor-products-of-pointed-sets:section-phantom]{Tensor Products of Pointed Sets}
\end{enumerate}
\textbf{Relations}
\begin{enumerate}
\setcounter{enumi}{6}
\item \hyperref[relations:section-phantom]{Relations}
\item \hyperref[constructions-with-relations:section-phantom]{Constructions With Relations}
\item \hyperref[conditions-on-relations:section-phantom]{Conditions on Relations}
\end{enumerate}
\textbf{Category Theory}
\begin{enumerate}
\setcounter{enumi}{9}
\item \hyperref[categories:section-phantom]{Categories}
\end{enumerate}
\textbf{Monoidal Categories}
\begin{enumerate}
\setcounter{enumi}{10}
\item \hyperref[constructions-with-monoidal-categories:section-phantom]{Constructions With Monoidal Categories}
\end{enumerate}
\textbf{Bicategories}
\begin{enumerate}
\setcounter{enumi}{11}
\item \hyperref[types-of-morphisms-in-bicategories:section-phantom]{Types of Morphisms in Bicategories}
\end{enumerate}
\textbf{Extra Part}
\begin{enumerate}
\setcounter{enumi}{12}
\item \hyperref[notes:section-phantom]{Notes}
\end{enumerate}
\end{multicols}

\end{appendices}
\end{document}
