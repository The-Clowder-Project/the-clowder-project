\input{preamble}

% OK, start here.
%
\usepackage{fontspec}
\let\hyperwhite\relax
\let\hyperred\relax
\newcommand{\hyperwhite}{\hypersetup{citecolor=white,filecolor=white,linkcolor=white,urlcolor=white}}
\newcommand{\hyperred}{%
\hypersetup{%
    citecolor=TitlingRed,%
    filecolor=TitlingRed,%
    linkcolor=TitlingRed,%
     urlcolor=TitlingRed%
}}
\let\ChapterRef\relax
\newcommand{\ChapterRef}[2]{#1}
\setcounter{tocdepth}{2}
%▓▓▓▓▓▓▓▓▓▓▓▓▓▓▓▓▓▓▓▓▓▓▓▓▓▓▓▓▓▓▓▓▓
%▓▓ ╔╦╗╦╔╦╗╦  ╔═╗  ╔═╗╔═╗╔╗╔╔╦╗ ▓▓
%▓▓  ║ ║ ║ ║  ║╣   ╠╣ ║ ║║║║ ║  ▓▓
%▓▓  ╩ ╩ ╩ ╩═╝╚═╝  ╚  ╚═╝╝╚╝ ╩  ▓▓
%▓▓▓▓▓▓▓▓▓▓▓▓▓▓▓▓▓▓▓▓▓▓▓▓▓▓▓▓▓▓▓▓▓
%\usepackage{titlesec}
%▓▓▓▓▓▓▓▓▓▓▓▓▓▓▓▓▓▓▓▓▓▓▓▓▓▓▓▓▓▓▓▓▓▓▓▓▓▓▓▓▓▓▓▓▓▓▓▓▓▓▓▓▓▓▓
%▓▓ ╔╦╗╔═╗╔╗ ╦  ╔═╗  ╔═╗╔═╗  ╔═╗╔═╗╔╗╔╔╦╗╔═╗╔╗╔╔╦╗╔═╗ ▓▓
%▓▓  ║ ╠═╣╠╩╗║  ║╣   ║ ║╠╣   ║  ║ ║║║║ ║ ║╣ ║║║ ║ ╚═╗ ▓▓
%▓▓  ╩ ╩ ╩╚═╝╩═╝╚═╝  ╚═╝╚    ╚═╝╚═╝╝╚╝ ╩ ╚═╝╝╚╝ ╩ ╚═╝ ▓▓
%▓▓▓▓▓▓▓▓▓▓▓▓▓▓▓▓▓▓▓▓▓▓▓▓▓▓▓▓▓▓▓▓▓▓▓▓▓▓▓▓▓▓▓▓▓▓▓▓▓▓▓▓▓▓▓
\newcommand{\ChapterTableOfContents}{%
    \begingroup
    \addfontfeature{Numbers={Lining,Monospaced}}
    \hypersetup{hidelinks}\tableofcontents%
    \endgroup
}%

\makeatletter
\newcommand \DotFill {\leavevmode \cleaders \hb@xt@ .33em{\hss .\hss }\hfill \kern \z@}
\makeatother

\definecolor{ToCGrey}{rgb}{0.4,0.4,0.4}
\definecolor{mainColor}{rgb}{0.82745098,0.18431373,0.18431373}
\usepackage{titletoc}
\titlecontents{part}
[0.0em]
{\addvspace{1pc}\color{TitlingRed}\large\bfseries\text{Part }}
{\bfseries\textcolor{TitlingRed}{\contentslabel{0.0em}}\hspace*{1.35em}}
{}
{\textcolor{TitlingRed}{{\hfill\bfseries\contentspage\nobreak}}}
[]
\titlecontents{section}
[0.0em]
{\addvspace{1pc}}
{\color{black}\bfseries\textcolor{TitlingRed}{\contentslabel{0.0em}}\hspace*{1.35em}}
{}
{\textcolor{black}{\textbf{\DotFill}{\bfseries\contentspage\nobreak}}}
[]
\titlecontents{subsection}
[0.0em]
{}
{\hspace*{1.35em}\color{ToCGrey}{\contentslabel{0.0em}}\hspace*{2.1em}}
{}
{{\textcolor{ToCGrey}\DotFill}\textcolor{ToCGrey}{\contentspage}\nobreak}
[]
\usepackage{marginnote}
\renewcommand*{\marginfont}{\normalfont}
\usepackage{inconsolata}
\setmonofont{inconsolata}%
\let\ChapterRef\relax
\newcommand{\ChapterRef}[2]{#1}
\AtBeginEnvironment{subappendices}{%%
    \section*{\huge Appendices}%
}%

\begin{document}

\title{Categories}

\maketitle

\phantomsection
\label{section-phantom}

This chapter contains some elementary material about categories, functors, and natural transformations. Notably, we discuss and explore:
\begin{enumerate}
    \item\label{categories-introduction-item-1}Categories (\cref{section-categories}).
    \item\label{categories-introduction-item-2}Examples of categories (\cref{section-examples-of-categories}).
    \item\label{categories-introduction-item-3}The quadruple adjunction $\pi_{0}\dashv{(-)_{\disc}}\dashv\Obj\dashv{(-)_{\indisc}}$ between the category of categories and the category of sets (\cref{section-the-quadruple-adjunction-with-sets}).
    \item\label{categories-introduction-item-4}Groupoids, categories in which all morphisms admit inverses (\cref{section-groupoids}).
    \item\label{categories-introduction-item-5}Functors (\cref{section-functors}).
    \item\label{categories-introduction-item-6}The conditions one may impose on functors in decreasing order of importance:
        \begin{enumerate}
            \item\label{categories-introduction-item-6-a}\cref{section-conditions-on-functors} introduces the foundationally important conditions one may impose on functors, such as faithfulness, conservativity, essential surjectivity, etc.
            \item\label{categories-introduction-item-6-b}\cref{section-more-conditions-on-functors} introduces more conditions one may impose on functors that are still important but less omni-present than those of \cref{section-conditions-on-functors}, such as being dominant, being a monomorphism, being pseudomonic, etc.
            \item\label{categories-introduction-item-6-c}\cref{section-even-more-conditions-on-functors} introduces some rather rare or uncommon conditions one may impose on functors that are nevertheless still useful to explicit record in this chapter.
        \end{enumerate}
    \item\label{categories-introduction-item-7}Natural transformations (\cref{section-natural-transformations}).
    \item\label{categories-introduction-item-8}The various categorical and 2-categorical structures formed by categories, functors, and natural transformations (\cref{section-categories-of-categories}).
\end{enumerate}
This chapter is under active revision. TODO:
\begin{itemize}
    \item Fix categories having an underlying set of objects by having them have an underlying setoid of objects (not necessarily by definition, as that'll likely be bothersome; at least \cref{section-the-quadruple-adjunction-with-sets} should be fixed and several remarks should be added at several points). Related: \cref{the-quadruple-adjunction-between-sets-and-cats-cannot-be-enhanced-to-a-two-categorical-adjunction}
\end{itemize}

\ChapterTableOfContents

\section{Categories}\label{section-categories}
\subsection{Foundations}\label{subsection-categories-foundations}
\begin{definition}{Categories}{categories}%
    A \index[categories]{category}\textbf{category} $\smash{(\CatFont{C},\circ^{\CatFont{C}},\Unit^{\CatFont{C}})}$ consists of:%
    \begin{itemize}
        \item\SloganFont{Objects. }A class $\Obj(\CatFont{C})$ of \textbf{objects}.
        \item\SloganFont{Morphisms. }For each $A,B\in\Obj(\CatFont{C})$, a class $\Hom_{\CatFont{C}}(A,B)$, called the \textbf{class of morphisms of $\CatFont{C}$ from $A$ to $B$}.
        \item\SloganFont{Identities. }For each $A\in\Obj(\CatFont{C})$, a map of sets%
            \[%
                \Unit^{\CatFont{C}}_{A}%
                \colon%
                \pt%
                \to%
                \Hom_{\CatFont{C}}(A,A),%
            \]%
            called the \textbf{unit map of $\CatFont{C}$ at $A$}, determining a morphism%
            \[%
                \id_{A}
                \colon
                A
                \to
                A
            \]%
            of $\CatFont{C}$, called the \textbf{identity morphism of $A$}.
        \item\SloganFont{Composition. }For each $A,B,C\in\Obj(\CatFont{C})$, a map of sets
            \[
                \circ^{\CatFont{C}}_{A,B,C}
                \colon
                \Hom_{\CatFont{C}}(B,C)\times\Hom_{\CatFont{C}}(A,B)
                \to
                \Hom_{\CatFont{C}}(A,C),
            \]%
            called the \textbf{composition map of $\CatFont{C}$ at $(A,B,C)$}.
    \end{itemize}
    such that the following conditions are satisfied:%
    \begin{enumerate}
        \item\SloganFont{Associativity. }The diagram
            \begin{scalemath}
                \begin{tikzcd}[row sep={0*\the\DL,between origins}, column sep={0*\the\DL,between origins}, background color=backgroundColor, ampersand replacement=\&]
                    \&[0.30901699437\FourCmPlusOneEighth]
                    \&[0.5\FourCmPlusOneEighth]
                    \Hom_{\CatFont{C}}(C,D)\times(\Hom_{\CatFont{C}}(B,C)\times\Hom_{\CatFont{C}}(A,B))
                    \&[0.5\FourCmPlusOneEighth]
                    \&[0.30901699437\FourCmPlusOneEighth]
                    \\[0.58778525229\FourCmPlusOneEighth]
                    (\Hom_{\CatFont{C}}(C,D)\times\Hom_{\CatFont{C}}(B,C))\times\Hom_{\CatFont{C}}(A,B)
                    \&[0.30901699437\FourCmPlusOneEighth]
                    \&[0.5\FourCmPlusOneEighth]
                    \&[0.5\FourCmPlusOneEighth]
                    \&[0.30901699437\FourCmPlusOneEighth]
                    \Hom_{\CatFont{C}}(C,D)\times\Hom_{\CatFont{C}}(A,C)
                    \\[0.95105651629\FourCmPlusOneEighth]
                    \&[0.30901699437\FourCmPlusOneEighth]
                    \Hom_{\CatFont{C}}(B,D)\times\Hom_{\CatFont{C}}(A,B)
                    \&[0.5\FourCmPlusOneEighth]
                    \&[0.5\FourCmPlusOneEighth]
                    \Hom_{\CatFont{C}}(A,D)
                    \&[0.30901699437\FourCmPlusOneEighth]
                    % 1-Arrows
                    % Left Boundary
                    \arrow[from=2-1,to=1-3,"\alpha^{\Sets}_{\Hom_{\CatFont{C}}(C,D),\Hom_{\CatFont{C}}(B,C),\Hom_{\CatFont{C}}(A,B)}"{pos=0.4125},isoarrowprime]%
                    \arrow[from=1-3,to=2-5,"\id_{\Hom_{\CatFont{C}}(C,D)}\times\circ^{\CatFont{C}}_{A,B,C}"{pos=0.6}]%
                    \arrow[from=2-5,to=3-4,"\circ^{\CatFont{C}}_{A,C,D}"{pos=0.425}]%
                    % Right Boundary
                    \arrow[from=2-1,to=3-2,"\circ^{\CatFont{C}}_{B,C,D}\times\id_{\Hom_{\CatFont{C}}(A,B)}"'{pos=0.425}]%
                    \arrow[from=3-2,to=3-4,"\circ^{\CatFont{C}}_{A,B,D}"']%
                \end{tikzcd}
            \end{scalemath}%
            commutes, i.e.\ for each composable triple $(f,g,h)$ of morphisms of $\CatFont{C}$, we have
            \[
                (f\circ g)\circ h
                =
                f\circ(g\circ h).
            \]
        \item\SloganFont{Left Unitality. }The diagram
            \[
                \begin{tikzcd}[row sep={5.0*\the\DL,between origins}, column sep={14.0*\the\DL,between origins}, background color=backgroundColor, ampersand replacement=\&]
                    \pt\times\Hom_{\CatFont{C}}(A,B)
                    \arrow[rd, "\LUnitor^{\Sets}_{\Hom_{\CatFont{C}}(A,B)}"{pos=0.4},isoarrowprime]
                    \arrow[d, "\Unit^{\CatFont{C}}_{B}\times\id_{\Hom_{\CatFont{C}}(A,B)}"']
                    \&\\
                    \Hom_{\CatFont{C}}(B,B)\times\Hom_{\CatFont{C}}(A,B)
                    \arrow[r, "\circ^{\CatFont{C}}_{A,B,B}"']
                    \&
                    \Hom_{\CatFont{C}}(A,B)
                \end{tikzcd}
            \]%
            commutes, i.e.\ for each morphism $f\colon A\to B$ of $\CatFont{C}$, we have
            \[
                \id_{B}\circ f%
                =%
                f.%
            \]%
        \item\SloganFont{Right Unitality. }The diagram
            \[
                \begin{tikzcd}[row sep={5.0*\the\DL,between origins}, column sep={14.0*\the\DL,between origins}, background color=backgroundColor, ampersand replacement=\&]
                    \Hom_{\CatFont{C}}(A,B)\times\pt
                    \arrow[rd, "\RUnitor^{\Sets}_{\Hom_{\CatFont{C}}(A,B)}"{pos=0.4},isoarrowprime]
                    \arrow[d, "\id_{\Hom_{\CatFont{C}}(A,B)}\times\Unit^{\CatFont{C}}_{A}"']
                    \&\\
                    \Hom_{\CatFont{C}}(A,B)\times\Hom_{\CatFont{C}}(A,A)
                    \arrow[r, "\circ^{\CatFont{C}}_{A,A,B}"']
                    \&
                    \Hom_{\CatFont{C}}(A,B)
                \end{tikzcd}
            \]%
            commutes, i.e.\ for each morphism $f\colon A\to B$ of $\CatFont{C}$, we have
            \[
                f\circ\id_{A}%
                =%
                f.%
            \]%
    \end{enumerate}
\end{definition}
\begin{notation}{Further Notation for Morphisms in Categories}{further-notation-for-morphisms-in-categories}%
    Let $\CatFont{C}$ be a category.
    \begin{enumerate}
        \item\label{further-notation-for-morphisms-in-categories-a}We also write \index[notation]{CAB@$\CatFont{C}(A,B)$}$\CatFont{C}(A,B)$ for $\Hom_{\CatFont{C}}(A,B)$.
        \item\label{further-notation-for-morphisms-in-categories-b}We write \index[notation]{MorC@$\Mor(\CatFont{C})$}$\Mor(\CatFont{C})$ for the class of all morphisms of $\CatFont{C}$.
    \end{enumerate}
\end{notation}
\begin{definition}{Size Conditions on Categories}{size-conditions-on-categories}%
    Let $\kappa$ be a regular cardinal. A category $\CatFont{C}$ is
    \begin{enumerate}
        \item\label{size-conditions-on-categories-a}\index[categories]{category!locally small}\textbf{Locally small} if, for each $A,B\in\Obj(\CatFont{C})$, the class $\Hom_{\CatFont{C}}(A,B)$ is a set.
        \item\label{size-conditions-on-categories-b}\index[categories]{category!locally essentially small}\textbf{Locally essentially small} if, for each $A,B\in\Obj(\CatFont{C})$, the class
            \[%
                \Hom_{\CatFont{C}}(A,B)/\{\text{isomorphisms}\}%
            \]%
            is a set.
        \item\label{size-conditions-on-categories-c}\index[categories]{category!small}\textbf{Small} if $\CatFont{C}$ is locally small and $\Obj(\CatFont{C})$ is a set.
        \item\label{size-conditions-on-categories-d}\index[categories]{category!kappa-small@$\kappa$-small}\textbf{$\kappa$-Small} if $\CatFont{C}$ is locally small, $\Obj(\CatFont{C})$ is a set, and we have $\Card{\Obj(\CatFont{C})}<\kappa$.%
    \end{enumerate}
\end{definition}
\subsection{Subcategories}\label{subsection-subcategories}
Let $\CatFont{C}$ be a category.
\begin{definition}{Subcategories}{subcategories}%
    A \index[categories]{subcategory}\textbf{subcategory} of $\CatFont{C}$ is a category $\CatFont{A}$ satisfying the following conditions:
    \begin{enumerate}
        \item\SloganFont{Objects. }We have $\Obj(\CatFont{A})\subset\Obj(\CatFont{C})$.
        \item\SloganFont{Morphisms. }For each $A,B\in\Obj(\CatFont{A})$, we have
            \[
                \Hom_{\CatFont{A}}(A,B)
                \subset
                \Hom_{\CatFont{C}}(A,B).
            \]
        \item\SloganFont{Identities. }For each $A\in\Obj(\CatFont{A})$, we have
            \[
                \Unit^{\CatFont{A}}_{A}
                =
                \Unit^{\CatFont{C}}_{A}.
            \]%
        \item\SloganFont{Composition. }For each $A,B,C\in\Obj(\CatFont{A})$, we have
            \[
                \circ^{\CatFont{A}}_{A,B,C}
                =
                \circ^{\CatFont{C}}_{A,B,C}.
            \]%
    \end{enumerate}
\end{definition}
\begin{definition}{Full Subcategories}{full-subcategories}%
    A subcategory $\CatFont{A}$ of $\CatFont{C}$ is \index[categories]{subcategory!full}\textbf{full} if the canonical inclusion functor $\CatFont{A}\to\CatFont{C}$ is full, i.e.\ if, for each $A,B\in\Obj(\CatFont{A})$, the inclusion
    \[
        \iota_{A,B}%
        \colon%
        \Hom_{\CatFont{A}}(A,B)%
        \hookrightarrow%
        \Hom_{\CatFont{C}}(A,B)%
    \]%
    is surjective (and thus bijective).
\end{definition}
\begin{definition}{Strictly Full Subcategories}{strictly-full-subcategories}%
    A subcategory $\CatFont{A}$ of a category $\CatFont{C}$ is \index[categories]{subcategory!strictly full}\textbf{strictly full} if it satisfies the following conditions:
    \begin{enumerate}
        \item\SloganFont{Fullness. }The subcategory $\CatFont{A}$ is full.
        \item\SloganFont{Closedness Under Isomorphisms. }The class $\Obj(\CatFont{A})$ is closed under isomorphisms.%
            %--- Begin Footnote ---%
            \footnote{%
                That is, given $A\in\Obj(\CatFont{A})$ and $C\in\Obj(\CatFont{C})$, if $C\cong A$, then $C\in\Obj(\CatFont{A})$.%
                \par\vspace*{\TCBBoxCorrection}
            }%
            %---  End Footnote  ---%
    \end{enumerate}
\end{definition}
\begin{definition}{Wide Subcategories}{wide-subcategories}%
    A subcategory $\CatFont{A}$ of $\CatFont{C}$ is \index[categories]{subcategory!wide}\textbf{wide}%
    %--- Begin Footnote ---%
    \footnote{%
        \SloganFont{Further Terminology: }Also called \index[categories]{subcategory!lluf}\textbf{lluf}.
        \par\vspace*{\TCBBoxCorrection}
    } %
    %---  End Footnote  ---%
    if $\Obj(\CatFont{A})=\Obj(\CatFont{C})$.
\end{definition}
\subsection{Skeletons of Categories}\label{subsection-skeletons-of-categories}
\begin{definition}{Skeletons of Categories}{skeletons-of-categories}%
    A%
    %--- Begin Footnote ---%
    \footnote{%
        Due to \cref{properties-of-skeletons-of-categories-uniqueness-up-to-equivalence} of \cref{properties-of-skeletons-of-categories}, which states that any two skeletons of a category are equivalent, we often refer to any such full subcategory $\Sk(\CatFont{C})$ of $\CatFont{C}$ as \emph{the} skeleton of $\CatFont{C}$.
        \par\vspace*{\TCBBoxCorrection}
    } %
    %---  End Footnote  ---%
    \index[categories]{category!skeleton of}\textbf{skeleton} of a category $\CatFont{C}$ is a full subcategory $\Sk(\CatFont{C})$ with one object from each isomorphism class of objects of $\CatFont{C}$.
\end{definition}
\begin{definition}{Skeletal Categories}{skeletal-categories}%
    A category $\CatFont{C}$ is \index[categories]{category!skeletal}\textbf{skeletal} if $\CatFont{C}\cong\Sk(\CatFont{C})$.%
    %--- Begin Footnote ---%
    \footnote{%
        That is, $\CatFont{C}$ is \textbf{skeletal} if isomorphic objects of $\CatFont{C}$ are equal.
        \par\vspace*{\TCBBoxCorrection}
    }%
    %---  End Footnote  ---%
\end{definition}
\begin{proposition}{Properties of Skeletons of Categories}{properties-of-skeletons-of-categories}%
    Let $\CatFont{C}$ be a category.
    \begin{enumerate}
        \item\label{properties-of-skeletons-of-categories-existence}\SloganFont{Existence. }Assuming the axiom of choice, $\Sk(\CatFont{C})$ always exists.
        \item\label{properties-of-skeletons-of-categories-pseudofunctoriality}\SloganFont{Pseudofunctoriality. }The assignment $\CatFont{C}\mapsto\Sk(\CatFont{C})$ defines a pseudofunctor
            \[%
                \Sk%
                \colon%
                \TwoCategoryOfCategories%
                \to%
                \TwoCategoryOfCategories.%
            \]%
        \item\label{properties-of-skeletons-of-categories-uniqueness-up-to-equivalence}\SloganFont{Uniqueness Up to Equivalence. }Any two skeletons of $\CatFont{C}$ are equivalent.
        \item\label{properties-of-skeletons-of-categories-inclusions-of-skeletons-are-equivalences}\SloganFont{Inclusions of Skeletons Are Equivalences. }The inclusion
            \[
                \iota_{\CatFont{C}}%
                \colon%
                \Sk(\CatFont{C})%
                \longhookrightarrow%
                \CatFont{C}
            \]%
            of a skeleton of $\CatFont{C}$ into $\CatFont{C}$ is an equivalence of categories.
        %\item\label{properties-of-skeletons-of-categories-}\SloganFont{. }
    \end{enumerate}
\end{proposition}
\begin{Proof}{Proof of \cref{properties-of-skeletons-of-categories}}%
    \FirstProofBox{\cref{properties-of-skeletons-of-categories-existence}: Existence}%
    See \cite[Section \say{Existence of Skeletons of Categories}]{nlab:skeleton}.

    \ProofBox{\cref{properties-of-skeletons-of-categories-pseudofunctoriality}: Pseudofunctoriality}%
    See \cite[Section \say{Skeletons as an Endo-Pseudofunctor on $\mathfrak{Cat}$}]{nlab:skeleton}.

    \ProofBox{\cref{properties-of-skeletons-of-categories-uniqueness-up-to-equivalence}: Uniqueness Up to Equivalence}%
    Omitted.

    \ProofBox{\cref{properties-of-skeletons-of-categories-inclusions-of-skeletons-are-equivalences}: Inclusions of Skeletons Are Equivalences}%
    Omitted.
\end{Proof}
\subsection{Precomposition and Postcomposition}\label{subsection-precomposition-and-postcomposition}
Let $\CatFont{C}$ be a category and let $A,B,C,X\in\Obj(\CatFont{C})$.
\begin{definition}{Precomposition and Postcomposition Functions}{precomposition-and-postcomposition-functions}%
    Let $f\colon A\to B$ and $g\colon B\to C$ be morphisms of $\CatFont{C}$.
    \begin{enumerate}
        \item\label{precomposition-and-postcomposition-functions-precomposition}The \index[categories]{precomposition}\textbf{precomposition function associated to $f$} is the function\index[notation]{fstar@$f^{*}$}%
            \[
                f^{*}
                \colon
                \Hom_{\CatFont{C}}(B,X)
                \to
                \Hom_{\CatFont{C}}(A,X)
            \]%
            defined by
            \[
                f^{*}(\phi)
                \defeq
                \phi\circ f
            \]%
            for each $\phi\in\Hom_{\CatFont{C}}(B,X)$.
        \item\label{precomposition-and-postcomposition-functions-postcomposition}The \index[categories]{postcomposition}\textbf{postcomposition function associated to $g$} is the function\index[notation]{fstar@$f^{*}$}%
            \[
                g_{*}
                \colon
                \Hom_{\CatFont{C}}(X,B)
                \to
                \Hom_{\CatFont{C}}(X,C)
            \]%
            defined by
            \[
                g_{*}(\phi)
                \defeq
                g\circ\phi
            \]%
            for each $\phi\in\Hom_{\CatFont{C}}(X,B)$.
    \end{enumerate}
\end{definition}
\begin{proposition}{Properties of Pre/Postcomposition}{properties-of-pre-postcomposition}%
    Let $A,B,C,D,X\in\Obj(\CatFont{C})$.
    \begin{enumerate}
        \item\label{properties-of-pre-postcomposition-interaction-between-precomposition-and-postcomposition}\SloganFont{Interaction Between Precomposition and Postcomposition. }Let $f\colon A\to B$ and $g\colon X\to Y$ be morphisms of $\CatFont{C}$. We have
            \begin{webcompile}
                g_{*}\circ f^{*}%
                =%
                f^{*}\circ g_{*},%
                \quad%
                \begin{tikzcd}[row sep={5.0*\the\DL,between origins}, column sep={8.5*\the\DL,between origins}, background color=backgroundColor, ampersand replacement=\&]
                    \Hom_{\CatFont{C}}(B,X)
                    \arrow[r,"g_{*}"]
                    \arrow[d,"f^{*}"']
                    \&
                    \Hom_{\CatFont{C}}(B,Y)
                    \arrow[d,"f^{*}"]
                    \\
                    \Hom_{\CatFont{C}}(A,X)
                    \arrow[r,"g_{*}"']
                    \&
                    \Hom_{\CatFont{C}}(A,Y)\mrp{.}
                \end{tikzcd}
            \end{webcompile}%
        \item\label{properties-of-pre-postcomposition-interaction-with-composition-1}\SloganFont{Interaction With Composition \rmI. }Let $f\colon A\to B$ and $g\colon B\to C$ be morphisms of $\CatFont{C}$. We have
            \begin{webcompile}
              \begin{aligned}
                    (g\circ f)_{*} &= g_{*}\circ f_{*},
                    \quad
                    \begin{tikzcd}[row sep={5.0*\the\DL,between origins}, column sep={8.5*\the\DL,between origins}, background color=backgroundColor, ampersand replacement=\&]
                        \Hom_{\CatFont{C}}(X,A)
                        \arrow[r,"f_{*}"]
                        \arrow[rd,"(g\circ f)_{*}"']
                        \&
                        \Hom_{\CatFont{C}}(X,B)
                        \arrow[d,"g_{*}"]
                        \\
                        \&
                        \Hom_{\CatFont{C}}(X,C)\mrp{,}
                    \end{tikzcd}
                    \\
                    (g\circ f)^{*} &= f^{*}\circ g^{*},%
                    \quad
                    \begin{tikzcd}[row sep={5.0*\the\DL,between origins}, column sep={8.5*\the\DL,between origins}, background color=backgroundColor, ampersand replacement=\&]
                        \Hom_{\CatFont{C}}(C,X)
                        \arrow[r,"g^{*}"]
                        \arrow[rd,"(g\circ f)^{*}"']
                        \&
                        \Hom_{\CatFont{C}}(B,X)
                        \arrow[d,"f^{*}"]
                        \\
                        \&
                        \Hom_{\CatFont{C}}(A,X)\mrp{.}
                    \end{tikzcd}
                \end{aligned}
            \end{webcompile}%
        \item\label{properties-of-pre-postcomposition-interaction-with-composition-2}\SloganFont{Interaction With Composition \rmII. }Let $f\colon A\to B$ and $g\colon B\to C$ be morphisms of $\CatFont{C}$. We have
            \begin{webcompile}
                \begin{tikzcd}[row sep={5.0*\the\DL,between origins}, column sep={6.5*\the\DL,between origins}, background color=backgroundColor, ampersand replacement=\&]
                    \pt
                    \arrow[r,"{[f]}"]
                    \arrow[rd,"{[g\circ f]}"']
                    \&
                    \Hom_{\CatFont{C}}(A,B)
                    \arrow[d,"g_{*}"]
                    \\
                    \&
                    \Hom_{\CatFont{C}}(A,C)
                \end{tikzcd}
                \quad
                \begin{aligned}
                    [g\circ f] &= g_{*}\circ[f],\\
                    [g\circ f] &= f^{*}\circ[g],
                \end{aligned}
                \quad
                \begin{tikzcd}[row sep={5.0*\the\DL,between origins}, column sep={6.5*\the\DL,between origins}, background color=backgroundColor, ampersand replacement=\&]
                    \pt
                    \arrow[r,"{[g]}"]
                    \arrow[rd,"{[g\circ f]}"']
                    \&
                    \Hom_{\CatFont{C}}(B,C)
                    \arrow[d,"f^{*}"]
                    \\
                    \&
                    \Hom_{\CatFont{C}}(A,C)\mrp{.}
                \end{tikzcd}
            \end{webcompile}%
        \item\label{properties-of-pre-postcomposition-interaction-with-composition-3}\SloganFont{Interaction With Composition \rmIII. }Let $f\colon X\to A$ and $g\colon C\to D$ be morphisms of $\CatFont{C}$. We have
            \begin{scalemath}
                \begin{aligned}
                    f^{*}\circ\mathord{\circ^{\CatFont{C}}_{A,B,C}}%
                    &=
                    \mathord{\circ^{\CatFont{C}}_{X,B,C}}\circ(\sfid\times f^{*}),%
                    \quad%
                    \begin{tikzcd}[row sep={5.0*\the\DL,between origins}, column sep={13.0*\the\DL,between origins}, background color=backgroundColor, ampersand replacement=\&]
                        \Hom_{\CatFont{C}}(B,C)\times\Hom_{\CatFont{C}}(A,B)
                        \arrow[r,"\circ^{\CatFont{C}}_{A,B,C}"]
                        \arrow[d,"\sfid\times f^{*}"']
                        \&
                        \Hom_{\CatFont{C}}(A,C)
                        \arrow[d,"f^{*}"]
                        \\
                        \Hom_{\CatFont{C}}(B,C)\times\Hom_{\CatFont{C}}(X,B)
                        \arrow[r,"\circ^{\CatFont{C}}_{X,B,C}"']
                        \&
                        \Hom_{\CatFont{C}}(X,C)\mrp{,}
                    \end{tikzcd}
                    \\
                    g_{*}\circ\mathord{\circ^{\CatFont{C}}_{A,B,C}}
                    &=
                    \mathord{\circ^{\CatFont{C}}_{A,B,D}}\circ(g_{*}\times\sfid),%
                    \quad%
                    \begin{tikzcd}[row sep={5.0*\the\DL,between origins}, column sep={13.0*\the\DL,between origins}, background color=backgroundColor, ampersand replacement=\&]
                        \Hom_{\CatFont{C}}(B,C)\times\Hom_{\CatFont{C}}(A,B)
                        \arrow[r,"\circ^{\CatFont{C}}_{A,B,C}"]
                        \arrow[d,"g_{*}\times\sfid"']
                        \&
                        \Hom_{\CatFont{C}}(A,C)
                        \arrow[d,"g^{*}"]
                        \\
                        \Hom_{\CatFont{C}}(B,D)\times\Hom_{\CatFont{C}}(A,B)
                        \arrow[r,"\circ^{\CatFont{C}}_{A,B,D}"']
                        \&
                        \Hom_{\CatFont{C}}(A,D)\mrp{.}
                    \end{tikzcd}
                \end{aligned}
            \end{scalemath}%
        \item\label{properties-of-pre-postcomposition-interaction-with-identities}\SloganFont{Interaction With Identities. }We have
            \begin{align*}
                \id^{*}_{A}   &= \id_{\Hom_{\CatFont{C}}(A,B)},\\
                (\id_{B})_{*} &= \id_{\Hom_{\CatFont{C}}(A,B)}.
            \end{align*}
        %\item\label{properties-of-pre-postcomposition-}\SloganFont{. }
    \end{enumerate}
\end{proposition}
\begin{Proof}{Proof of \cref{properties-of-pre-postcomposition}}%
    \FirstProofBox{\cref{properties-of-pre-postcomposition-interaction-between-precomposition-and-postcomposition}: Interaction Between Precomposition and Postcomposition}%
    For each $\phi\in\Hom_{\CatFont{C}}(B,X)$, we have
    \begin{align*}
        [g_{*}\circ f^{*}](\phi) &= g_{*}(\phi\circ f)\\
                                 &= g\circ(\phi\circ f)\\
                                 &= (g\circ\phi)\circ f\\
                                 &= f^{*}(g\circ\phi)\\
                                 &= [f^{*}\circ g_{*}](\phi).
    \end{align*}
    Thus $g_{*}\circ f^{*}=f^{*}\circ g_{*}$.

    \ProofBox{\cref{properties-of-pre-postcomposition-interaction-with-composition-1}: Interaction With Composition \rmI}%
    \SubProofBox{$(g\circ f)_{*}=g_{*}\circ f_{*}$}%
    For each $\phi\in\Hom_{\CatFont{C}}(X,A)$, we have
    \begin{align*}
        (g\circ f)_{*}(\phi) &= (g\circ f)\circ \phi\\
                             &= g\circ(f\circ \phi)\\
                             &= g\circ f_{*}(\phi)\\
                             &= g_{*}(f_{*}(\phi))\\
                             &= [g_{*}\circ f_{*}](\phi).
    \end{align*}
    Thus $(g\circ f)_{*}=g_{*}\circ f_{*}$.

    \SubProofBox{$(g\circ f)^{*}=g^{*}\circ f^{*}$}%
    For each $\phi\in\Hom_{\CatFont{C}}(C,X)$, we have
    \begin{align*}
        (g\circ f)^{*}(\phi) &= \phi\circ(g\circ f)\\
                             &= (\phi\circ g)\circ f\\
                             &= (g^{*}(\phi))\circ f\\
                             &= f^{*}(g^{*}(\phi))\\
                             &= [f^{*}\circ g^{*}](\phi).
    \end{align*}
    Thus $(g\circ f)^{*}=g^{*}\circ f^{*}$.

    \ProofBox{\cref{properties-of-pre-postcomposition-interaction-with-composition-2}: Interaction With Composition \rmII}%
    It suffices to show the equalities of the maps on $\point\in\pt$. We have
    \begin{align*}
        [g\circ f](\point) &= g\circ f\\
                           &= g_{*}(f)\\
                           &= g_{*}([f](\point))\\
                           &= (g_{*}\circ[f])(\point),
    \end{align*}
    and
    \begin{align*}
        [g\circ f](\point) &= g\circ f\\
                           &= f^{*}(g)\\
                           &= f^{*}([g](\point))\\
                           &= (f^{*}\circ[g])(\point).%
    \end{align*}
    Thus $[g\circ f]=g_{*}\circ[f]$ and $[g\circ f]=f^{*}\circ[g]$.

    \ProofBox{\cref{properties-of-pre-postcomposition-interaction-with-composition-3}: Interaction With Composition \rmIII}%
    \SubProofBox{$f^{*}\circ\mathord{\circ^{\CatFont{C}}_{A,B,C}}=\mathord{\circ^{\CatFont{C}}_{X,B,C}}\circ(\sfid\times f^{*})$}%
    For each $(\psi,\phi)\in\Hom_{\CatFont{C}}(B,C)\times\Hom_{\CatFont{C}}(A,B)$, we have
    \begin{align*}
        [f^{*}\circ\mathord{\circ^{\CatFont{C}}_{A,B,C}}](\psi,\phi) &= f^{*}(\psi\circ\phi)\\
                                                                     &= (\psi\circ\phi)\circ f\\
                                                                     &= \psi\circ(\phi\circ f)\\
                                                                     &= \mathord{\circ^{\CatFont{C}}_{X,B,C}}(\psi,\phi\circ f)\\
                                                                     &= \mathord{\circ^{\CatFont{C}}_{X,B,C}}(\psi,f^{*}(\phi))\\
                                                                     &= [\mathord{\circ^{\CatFont{C}}_{X,B,C}}\circ(\sfid\times f^{*})](\psi,\phi).
    \end{align*}
    Thus $f^{*}\circ\mathord{\circ^{\CatFont{C}}_{A,B,C}}=\mathord{\circ^{\CatFont{C}}_{X,B,C}}\circ(\sfid\times f^{*})$.

    \SubProofBox{$g_{*}\circ\mathord{\circ^{\CatFont{C}}_{A,B,C}}=\mathord{\circ^{\CatFont{C}}_{A,B,D}}\circ(g_{*}\times\sfid)$}%
    For each $(\psi,\phi)\in\Hom_{\CatFont{C}}(B,C)\times\Hom_{\CatFont{C}}(A,B)$, we have
    \begin{align*}
        [g_{*}\circ\mathord{\circ^{\CatFont{C}}_{A,B,C}}](\psi,\phi) &= g_{*}(\psi\circ\phi)\\
                                                                     &= g\circ(\psi\circ\phi)\\
                                                                     &= (g\circ\psi)\circ\phi\\
                                                                     &= \mathord{\circ^{\CatFont{C}}_{A,B,D}}(g\circ\psi,\phi)\\
                                                                     &= \mathord{\circ^{\CatFont{C}}_{A,B,D}}(g_{*}(\psi),\phi)\\
                                                                     &= [\mathord{\circ^{\CatFont{C}}_{A,B,D}}\circ(g_{*}\times\sfid)](\psi,\phi).
    \end{align*}
    Thus $g_{*}\circ\mathord{\circ^{\CatFont{C}}_{A,B,C}}=\mathord{\circ^{\CatFont{C}}_{A,B,D}}\circ(g_{*}\times\sfid)$.

    \ProofBox{\cref{properties-of-pre-postcomposition-interaction-with-identities}: Interaction With Identities}%
    We have
    \begin{align*}
        \id^{*}_{A}(\phi) &= \phi\circ\id_{A}\\
                          &= \phi\\
                          &= \id_{\Hom_{\CatFont{C}}(A,B)}(\phi)%
    \end{align*}
    and
    \begin{align*}
        (\id_{B})_{*}(\phi) &= \id_{B}\circ\phi\\
                            &= \phi\\
                            &= \id_{\Hom_{\CatFont{C}}(A,B)}(\phi)%
    \end{align*}
    for each $\phi\in\Hom_{\CatFont{C}}(A,B)$.
\end{Proof}
\section{Examples of Categories}\label{section-examples-of-categories}
\subsection{The Empty Category}\label{subsection-the-empty-category}
\begin{example}{The Empty Category}{the-empty-category}%
    The \index[categories]{category!empty}\textbf{empty category} is the category \index[notation]{emptycat@$\EmptyCategory$}$\EmptyCategory$ where%
    \begin{itemize}
        \item\SloganFont{Objects. }We have
            \[
                \Obj(\EmptyCategory)
                \defeq
                \emptyset.
            \]%
        \item\SloganFont{Morphisms. }We have
            \[
                \Mor(\EmptyCategory)
                \defeq
                \emptyset.
            \]%
        \item\SloganFont{Identities and Composition. }Having no objects, $\EmptyCategory$ has no unit nor composition maps.
    \end{itemize}
\end{example}
\subsection{The Punctual Category}\label{subsection-the-punctual-category}
\begin{example}{The Punctual Category}{punctual-category}%
    The \index[categories]{punctual category}\textbf{punctual category}%
    %--- Begin Footnote ---%
    \footnote{%
        \SloganFont{Further Terminology: }Also called the \index[categories]{singleton category|see {punctual category}}\textbf{singleton category}.
        \par\vspace*{\TCBBoxCorrection}
    } %
    %---  End Footnote  ---%
    is the category \index[notation]{ptcat@$\PunctualCategory$}$\PunctualCategory$ where
    \begin{itemize}
        \item\SloganFont{Objects. }We have
            \[
                \Obj(\PunctualCategory)
                \defeq
                \{\point\}.
            \]
        \item\SloganFont{Morphisms. }The unique $\Hom$-set of $\PunctualCategory$ is defined by
            \[
                \Hom_{\PunctualCategory}(\point,\point)
                \defeq
                \{\id_{\point}\}.
            \]%
        \item\SloganFont{Identities. }The unit map
            \[
                \Unit^{\PunctualCategory}_{\point}
                \colon
                \pt
                \to
                \Hom_{\PunctualCategory}(\point,\point)
            \]%
            of $\PunctualCategory$ at $\point$ is defined by
            \[
                \id^{\PunctualCategory}_{\point}
                \defeq
                \id_{\point}.
            \]%
        \item\SloganFont{Composition. }The composition map
            \[
                \circ^{\PunctualCategory}_{\point,\point,\point}
                \colon
                \Hom_{\PunctualCategory}(\point,\point)
                \times
                \Hom_{\PunctualCategory}(\point,\point)
                \to
                \Hom_{\PunctualCategory}(\point,\point)
            \]%
            of $\PunctualCategory$ at $(\point,\point,\point)$ is given by the bijection $\pt\times\pt\cong\pt$.
    \end{itemize}
\end{example}
\subsection{Monoids as One-Object Categories}\label{subsection-monoids-as-one-object-categories}
\begin{example}{Monoids as One-Object Categories}{monoids-as-one-object-categories}%
    We have an isomorphism of categories%
    %--- Begin Footnote ---%
    \footnote{%
        This can be enhanced to an isomorphism of $2$-categories
        \begin{webcompile}
            \Mon_{\twodisc}%
            \cong
            \PunctualBicategory\ttimes_{\Sets_{\twodisc}}\Cats_{\sftwo,*},%
            \quad%
            \begin{tikzcd}[row sep={5.0*\the\DL,between origins}, column sep={6.0*\the\DL,between origins}, background color=backgroundColor, ampersand replacement=\&]
                \Mon_{\twodisc}
                \arrow[r]
                \arrow[d]
                \arrow[rd,very near start,phantom,"\lrcorner"]
                \&
                \Cats_{\sftwo,*}
                \arrow[d,"\Obj"]
                \\
                \PunctualBicategory
                \arrow[r,"{[\pt]}"']
                \&
                \Sets_{\twodisc}
            \end{tikzcd}
        \end{webcompile}%
        between the discrete $2$-category $\Mon_{\twodisc}$ on $\Mon$ and the $2$-category of pointed categories with one object.
        \par\vspace*{\TCBBoxCorrection}
    }%
    %---  End Footnote  ---%
    \begin{webcompile}
        \Mon%
        \cong%
        \PunctualCategory\ttimes_{\Sets}\Cats,
        \quad
        \begin{tikzcd}[row sep={5.0*\the\DL,between origins}, column sep={5.0*\the\DL,between origins}, background color=backgroundColor, ampersand replacement=\&]
            \Mon
            \arrow[r]
            \arrow[d]
            \arrow[rd,very near start,phantom,"\lrcorner"]
            \&
            \Cats
            \arrow[d,"\Obj"]
            \\
            \PunctualCategory
            \arrow[r,"{[\pt]}"']
            \&
            \Sets
        \end{tikzcd}
    \end{webcompile}%
    via the delooping functor $\B\colon\Mon\to\Cats$ of \cref{categories-properties-of-deloopings-of-monoids-functoriality} of \cref{categories-properties-of-deloopings-of-monoids}, exhibiting monoids as exactly those categories having a single object.%TODO
\end{example}
\begin{Proof}{Proof of \cref{monoids-as-one-object-categories}}%
    Omitted.%TODO
\end{Proof}
\subsection{Ordinal Categories}\label{subsection-ordinal-categories}
\begin{example}{Ordinal Categories}{ordinal-categories}%
    The \index[categories]{ordinal category}\textbf{$n$th ordinal category} is the category \index[notation]{n@$\OrdinalCategory$}$\OrdinalCategory$ where%
    %--- Begin Footnote ---%
    \footnote{%
        In other words, $\OrdinalCategory$ is the category associated to the poset
        \[%
            [0]\to[1]\to\cdots\to[n-1]\to[n].%
        \]%
        The category $\OrdinalCategory$ for $n\geq2$ may also be defined in terms of $\OrdinalCategoryN{0}$ and joins (\ChapterRef{\ChapterConstructionsWithCategories, \cref{constructions-with-categories:the-join-of-two-categories}}{\cref{the-join-of-two-categories}}): we have isomorphisms of categories
        \begin{align*}
            \OrdinalCategoryN{1} &\cong \OrdinalCategoryN{0}\star\OrdinalCategoryN{0},\\
            \OrdinalCategoryN{2} &\cong \OrdinalCategoryN{1}\star\OrdinalCategoryN{0}\\
                                 &\cong (\OrdinalCategoryN{0}\star\OrdinalCategoryN{0})\star\OrdinalCategoryN{0},\\
            \OrdinalCategoryN{3} &\cong \OrdinalCategoryN{2}\star\OrdinalCategoryN{0}\\
                                 &\cong (\OrdinalCategoryN{1}\star\OrdinalCategoryN{0})\star\OrdinalCategoryN{0}\\
                                 &\cong ((\OrdinalCategoryN{0}\star\OrdinalCategoryN{0})\star\OrdinalCategoryN{0})\star\OrdinalCategoryN{0},\\
            \OrdinalCategoryN{4} &\cong \OrdinalCategoryN{3}\star\OrdinalCategoryN{0}\\
                                 &\cong (\OrdinalCategoryN{2}\star\OrdinalCategoryN{0})\star\OrdinalCategoryN{0}\\
                                 &\cong ((\OrdinalCategoryN{1}\star\OrdinalCategoryN{0})\star\OrdinalCategoryN{0})\star\OrdinalCategoryN{0}\\
                                 &\cong (((\OrdinalCategoryN{0}\star\OrdinalCategoryN{0})\star\OrdinalCategoryN{0})\star\OrdinalCategoryN{0})\star\OrdinalCategoryN{0},
        \end{align*}
        and so on.
        \par\vspace*{\TCBBoxCorrection}
    }%
    %---  End Footnote  ---%
    \begin{itemize}
        \item\SloganFont{Objects. }We have
            \[
                \Obj(\OrdinalCategory)
                \defeq
                \{[0],\ldots,[n]\}.
            \]%
        \item\SloganFont{Morphisms. }For each $[i],[j]\in\Obj(\OrdinalCategory)$, we have
            \[
                \Hom_{\OrdinalCategory}([i],[j])
                \defeq
                \begin{cases}
                    \{\id_{[i]}\}                &\text{if $[i]=[j]$,}\\
                    \{[i]\to[j]\}    &\text{if $[j]<[i]$,}\\
                    \emptyset                    &\text{if $[j]>[i]$.}
                \end{cases}
            \]%
        \item\SloganFont{Identities. }For each $[i]\in\Obj(\OrdinalCategory)$, the unit map
            \[
                \Unit^{\OrdinalCategory}_{[i]}
                \colon
                \pt
                \to
                \Hom_{\OrdinalCategory}([i],[i])
            \]%
            of $\OrdinalCategory$ at $[i]$ is defined by
            \[
                \id^{\OrdinalCategory}_{[i]}
                \defeq
                \id_{[i]}.
            \]%
        \item\SloganFont{Composition. }For each $[i],[j],[k]\in\Obj(\OrdinalCategory)$, the composition map
            \[
                \circ^{\OrdinalCategory}_{[i],[j],[k]}
                \colon
                \Hom_{\OrdinalCategory}([j],[k])
                \times
                \Hom_{\OrdinalCategory}([i],[j])
                \to
                \Hom_{\OrdinalCategory}([i],[k])
            \]%
            of $\OrdinalCategory$ at $([i],[j],[k])$ is defined by%
            \[
                \begin{gathered}
                    \id_{[i]}\circ\id_{[i]}                             = \id_{[i]},\\%
                    ([j]\to[k])\circ([i]\to[j]) = ([i]\to[k]).%
                \end{gathered}
            \]%
    \end{itemize}
\end{example}
\subsection{The Walking Arrow}\label{subsection-the-walking-arrow}
\begin{definition}{The Walking Arrow}{the-walking-arrow}%
    The \index[categories]{walking arrow}\textbf{walking arrow} is the category $\WalkingArrow$ defined as the first ordinal category.%
\end{definition}
\begin{remark}{Unwinding \cref{the-walking-arrow}}{unwinding-the-walking-arrow}%
    In detail, the walking arrow is the category $\WalkingArrow$ where:
    \begin{itemize}
        \item\SloganFont{Objects. }We have $\Obj(\WalkingArrow)=\{0,1\}$.
        \item\SloganFont{Morphisms. }We have
            \begin{align*}
                \Hom_{\WalkingArrow}(0,0) &= \{\id_{0}\},\\
                \Hom_{\WalkingArrow}(1,1) &= \{\id_{1}\},\\
                \Hom_{\WalkingArrow}(0,1) &= \{f_{01}\},\\
                \Hom_{\WalkingArrow}(1,0) &= \emptyset.
            \end{align*}
        \item\SloganFont{Identities and Composition. }The identities and composition of $\WalkingArrow$ are completely determined by the unitality and associativity axioms for $\WalkingArrow$.
    \end{itemize}
\end{remark}
\subsection{More Examples of Categories}\label{subsection-more-examples-of-categories}
\begin{example}{More Examples of Categories}{more-examples-of-categories}%
    Here we list some of the other categories appearing throughout this work.
    \begin{enumerate}
        \item\label{more-examples-of-categories-pointed-sets}The category $\Sets_{*}$ of pointed sets of \ChapterRef{\ChapterPointedSets, \cref{pointed-sets:the-category-of-pointed-sets}}{\cref{the-category-of-pointed-sets}}.
        \item\label{more-examples-of-categories-relations}The category $\sfRel$ of sets and relations of \ChapterRef{\ChapterRelations, \cref{relations:the-category-of-relations}}{\cref{the-category-of-relations}}.
        \item\label{more-examples-of-categories-spans}The category $\Span(A,B)$ of spans from a set $A$ to a set $B$ of \ChapterRef{\ChapterSpans, \cref{spans:the-category-of-spans-from-a-to-b}}{\cref{the-category-of-spans-from-a-to-b}}.
        \item\label{more-examples-of-categories-indexed-sets-k}The category $\ISets(K)$ of $K$-indexed sets of \ChapterRef{\ChapterIndexedSets, \cref{indexed-sets:the-category-of-k-indexed-sets}}{\cref{the-category-of-k-indexed-sets}}.
        \item\label{more-examples-of-categories-indexed-sets}The category $\ISets$ of indexed sets of \ChapterRef{\ChapterIndexedSets, \cref{indexed-sets:the-category-of-indexed-sets}}{\cref{the-category-of-indexed-sets}}.
        \item\label{more-examples-of-categories-fibred-sets-k}The category $\FibSets(K)$ of $K$-fibred sets of \ChapterRef{\ChapterFibredSets, \cref{fibred-sets:the-category-of-k-fibred-sets}}{\cref{the-category-of-k-fibred-sets}}.
        \item\label{more-examples-of-categories-fibred-sets}The category $\FibSets$ of fibred sets of \ChapterRef{\ChapterFibredSets, \cref{fibred-sets:the-category-of-fibred-sets}}{\cref{the-category-of-fibred-sets}}.
        \item\label{more-examples-of-categories-functor-categories}Categories of functors $\Fun(\CatFont{C},\CatFont{D})$ as in \cref{functor-categories}.
        \item\label{more-examples-of-categories-categories}The category of categories $\Cats$ of \cref{the-category-of-categories-and-functors}.
        \item\label{more-examples-of-categories-groupoids}The category of groupoids $\Grpd$ of \cref{the-category-of-small-groupoids}.
    \end{enumerate}
\end{example}
\subsection{Posetal Categories}\label{subsection-posetal-categories}
\begin{definition}{Posetal Categories}{posetal-categories}%
    Let $(X,\preceq_{X})$ be a poset.
    \begin{enumerate}
        \item\label{posetal-categories-posetal-category-associated-to-poset}The \index[categories]{posetal category!associated to a poset}\textbf{posetal category associated to $(X,\preceq_{X})$} is the category \index[notation]{Xpos@$X_{\pos}$}$X_{\pos}$ where
            \begin{itemize}
                \item\SloganFont{Objects. }We have
                    \[
                        \Obj(X_{\pos})%
                        \defeq%
                        X.%
                    \]%
                \item\SloganFont{Morphisms. }For each $a,b\in\Obj(X_{\pos})$, we have
                    \[
                        \Hom_{X_{\pos}}(a,b)%
                        \defeq%
                        \begin{cases}
                            \pt       &\text{if $a\preceq_{X}b$},\\%
                            \emptyset &\text{otherwise.}%
                        \end{cases}
                    \]%
                \item\SloganFont{Identities. }For each $a\in\Obj(X_{\pos})$, the unit map
                    \[%
                        \Unit^{X_{\pos}}_{a}%
                        \colon%
                        \pt%
                        \to%
                        \Hom_{X_{\pos}}(a,a)%
                    \]%
                    of $X_{\pos}$ at $a$ is given by the identity map.
                \item\SloganFont{Composition. }For each $a,b,c\in\Obj(X_{\pos})$, the composition map
                    \[
                        \circ^{X_{\pos}}_{a,b,c}%
                        \colon%
                        \Hom_{X_{\pos}}(b,c)%
                        \times%
                        \Hom_{X_{\pos}}(a,b)%
                        \to%
                        \Hom_{X_{\pos}}(a,c)%
                    \]%
                    of $X_{\pos}$ at $(a,b,c)$ is defined as either the inclusion $\emptyset\hookrightarrow\pt$ or the identity map of $\pt$, depending on whether we have $a\preceq_{X}b$, $b\preceq_{X}c$, and $a\preceq_{X}c$.
            \end{itemize}
        \item\label{posetal-categories-posetal-categories}A category $\CatFont{C}$ is \index[categories]{posetal category}\textbf{posetal}%
            %--- Begin Footnote ---%
            \footnote{%
                \SloganFont{Further Terminology: }Also called a \index[categories]{category!thin}\textbf{thin} category or a \index[categories]{zeroonecategory@$(0,1)$-category}\textbf{$(0,1)$-category}.
                \par\vspace*{\TCBBoxCorrection}
            } %
            %---  End Footnote  ---%
            if $\CatFont{C}$ is equivalent to $X_{\pos}$ for some poset $(X,\preceq_{X})$.
    \end{enumerate}
\end{definition}
\begin{proposition}{Properties of Posetal Categories}{properties-of-posetal-categories}%
    Let $(X,\preceq_{X})$ be a poset and let $\CatFont{C}$ be a category.
    \begin{enumerate}
        \item\label{properties-of-posetal-categories-functoriality}\SloganFont{Functoriality. }The assignment $(X,\preceq_{X})\mapsto X_{\pos}$ defines a functor
            \[
                (-)_{\pos}%
                \colon%
                \Pos%
                \to%
                \Cats.
            \]%
            where:
            \begin{itemize}
                \item\SloganFont{Action on Objects. }For each $X\in\Obj(\Pos)$, we have
                    \[
                        [(-)_{\pos}](X)%
                        \defeq%
                        X_{\pos},
                    \]%
                    where $X_{pos}$ is the category of of \cref{posetal-categories-posetal-category-associated-to-poset} of \cref{posetal-categories}.
                \item\SloganFont{Action on Morphisms. }For each morphism of posets $f\colon X\to Y$ in $\Pos$, the image
                    \[
                        f_{\pos}%
                        \colon%
                        X_{\pos}%
                        \to%
                        Y_{\pos}%
                    \]%
                    of $f$ by $(-)_{\pos}$ is the functor defined as follows:
                    \begin{itemize}
                        \item\SloganFont{The Action of $f_{\pos}$ on Objects. }For each $x\in\Obj(X_{\pos})$, we have
                            \[
                                f_{\pos}(x)%
                                \defeq%
                                f(x).%
                            \]%
                        \item\SloganFont{The Action of $f_{\pos}$ on Morphisms. }For each $x,y\in\Obj(X_{\pos})$, the action
                            \[
                                f_{\pos|x,y}%
                                \colon%
                                \Hom_{X_{\pos}}(x,y)%
                                \to%
                                \Hom_{Y_{\pos}}(f(x),f(y))%
                            \]%
                            of $f$ at $(x,y)$ is given by
                            \[
                                f_{\pos|x,y}(\pt_{\Hom_{X_{\pos}}(x,y)})%
                                \defeq%
                                \pt_{\Hom_{Y_{\pos}}(f(x),f(y))}%
                            \]%
                            if $x\preceq_{X}y$ or, otherwise, by the inclusion of the empty set into $\Hom_{Y_{\pos}}(f(x),f(y))$.
                    \end{itemize}
            \end{itemize}
        \item\label{properties-of-posetal-categories-fully-faithfulness}\SloganFont{Fully Faithfulness. }The functor $(-)_{\pos}$ of \cref{properties-of-posetal-categories-functoriality} is fully faithful.
        \item\label{properties-of-posetal-categories-characterisations}\SloganFont{Characterisations. }The following conditions are equivalent:
            \begin{enumerate}
                \item\label{properties-of-posetal-categories-characterisations-a}The category $\CatFont{C}$ is posetal.
                \item\label{properties-of-posetal-categories-characterisations-b}For each $A,B\in\Obj(\CatFont{C})$ and each $f,g\in\Hom_{\CatFont{C}}(A,B)$, we have $f=g$.
            \end{enumerate}
        \item\label{properties-of-posetal-categories-automatic-commutativity-of-diagrams}\SloganFont{Automatic Commutativity of Diagrams. }Every diagram in a posetal category commutes.
        %\item\label{properties-of-posetal-categories-}\SloganFont{. }
    \end{enumerate}
\end{proposition}
\begin{Proof}{Proof of \cref{properties-of-posetal-categories}}%
    \FirstProofBox{\cref{properties-of-posetal-categories-functoriality}: Functoriality}%
    First, note that given a morphism of posets $f\colon X\to Y$, the corresponding functor $f_{\pos}\colon X_{\pos}\to Y_{\pos}$ is indeed a functor: since all morphisms in the Hom-sets of $Y_{\pos}$ are equal, it preserves identities and compositions trivially.

    \indent Next, we claim that $(-)_{\pos}$ is indeed a functor:
    \begin{itemize}
        \item\SloganFont{Preservation of Identities. }Let $X\in\Obj(\Pos)$. Given $x,y\in X$ with $x\preceq_{X}y$, we have
            \begin{align*}
                (\id_{X})_{\pos}(x) &= \id_{X}(x)\\
                                    &= \id_{X_{\pos}}(x),
            \end{align*}
            so $(\id_{X})_{\pos}$ acts like the identity functor of $X_{\pos}$ on objects, and
            \begin{align*}
                (\id_{X})_{\pos}(\pt_{\Hom_{X_{\pos}}(x,y)}) &= \pt_{\Hom_{X_{\pos}}((\id_{X})_{\pos}(x),(\id_{X})_{\pos}(y))}\\
                                                             &= \pt_{\Hom_{X_{\pos}}(a,b)},
            \end{align*}
            so the same holds for morphisms. Thus $(\id_{X})_{\pos}=\id_{X_{\pos}}$.
        \item\SloganFont{Preservation of Composition. }Let $X,Y,Z\in\Obj(\Pos)$. Given morphisms of posets $f\colon X\to Y$ and $g\colon Y\to Z$, we need to show
            \[
                (g\circ f)_{\pos}%
                =%
                g_{\pos}\circ f_{\pos}.%
            \]%
            Indeed, given $x\in X$, we have
            \begin{align*}
                (g\circ f)_{\pos}(x) &= (g\circ f)(x)\\
                                     &= g(f(x))\\
                                     &= g_{\pos}(f_{\pos}(x))\\
                                     &= [g_{\pos}\circ f_{\pos}](x),
            \end{align*}
            so the identity holds on objects. Since $Z_{\pos}$ is a posetal category, the identity automatically holds on morphisms since
            \begin{align*}
                (g\circ f)_{\pos}(\pt_{\Hom_{X_{\pos}}(x,y)}) &= \pt_{\Hom_{Z_{\pos}}(g_{\pos}(f_{\pos}(x)),g_{\pos}(f_{\pos}(y)))}\\
                                                              &= [g_{\pos}\circ f_{\pos}](\pt_{\Hom_{X_{\pos}}(x,y)})
            \end{align*}
            for each $x,y\in X$ with $x\preceq_{X}y$.
    \end{itemize}
    Thus $(-)_{\pos}$ is indeed a functor.

    \ProofBox{\cref{properties-of-posetal-categories-fully-faithfulness}: Fully Faithfulness}%
    Omitted.

    \ProofBox{\cref{properties-of-posetal-categories-characterisations}: Characterisations}%
    Omitted.

    \ProofBox{\cref{properties-of-posetal-categories-automatic-commutativity-of-diagrams}: Automatic Commutativity of Diagrams}%
    This follows from the fact that if $\CatFont{C}$ is posetal, then there is at most one morphism between any two objects, namely $\pt$.
\end{Proof}
\section{The Quadruple Adjunction With Sets}\label{section-the-quadruple-adjunction-with-sets}
\subsection{Statement}\label{subsection-the-quadruple-adjunction-with-sets-statement}
Let $\CatFont{C}$ be a category.
\begin{proposition}{The Quadruple Adjunction Between $\Sets$ and $\Cats$}{the-quadruple-adjunction-between-sets-and-cats}%
    We have a quadruple adjunction%
    \begin{webcompile}
        \QuadrupleAdjunction#\pi_{0}#{(-)_{\disc}}#\Obj#{(-)_{\indisc}}#\Sets#\Cats,#
    \end{webcompile}%
    witnessed by bijections of sets
    \begin{align*}
        \Hom_{\Sets}(\pi_{0}(\CatFont{C}),X) &\cong \Hom_{\Cats}(\CatFont{C},X_{\disc}),\\
        \Hom_{\Cats}(X_{\disc},\CatFont{C})  &\cong \Hom_{\Sets}(X,\Obj(\CatFont{C})),\\
        \Hom_{\Sets}(\Obj(\CatFont{C}),X)    &\cong \Hom_{\Cats}(\CatFont{C},X_{\indisc}),
    \end{align*}
    natural in $\CatFont{C}\in\Obj(\Cats)$ and $X\in\Obj(\Sets)$, where
    \begin{itemize}
        \item The functor
            \[
                \pi_{0}%
                \colon%
                \Cats%
                \to%
                \Sets,%
            \]%
            the \textbf{connected components functor}, is the functor sending a category to its set of connected components of \cref{sets-of-connected-components-of-categories}.
        \item The functor
            \[
                (-)_{\disc}%
                \colon%
                \Sets%
                \to%
                \Cats,%
            \]%
            the \textbf{discrete category functor}, is the functor sending a set to its associated discrete category of \cref{discrete-categories-the-discrete-category-on-a-set}.
        \item The functor
            \[
                \Obj%
                \colon%
                \Cats%
                \to%
                \Sets,%
            \]%
            the \textbf{object functor}, is the functor sending a category to its set of objects.
        \item The functor
            \[
                (-)_{\indisc}%
                \colon%
                \Sets%
                \to%
                \Cats,%
            \]%
            the \textbf{indiscrete category functor}, is the functor sending a set to its associated indiscrete category of \cref{indiscrete-categories-the-indiscrete-category-on-a-set}.
    \end{itemize}
\end{proposition}
\begin{Proof}{Proof of \cref{the-quadruple-adjunction-between-sets-and-cats}}%
    Omitted.
\end{Proof}
\begin{warning}{\cref{the-quadruple-adjunction-between-sets-and-cats} Cannot Be Enhanced to a 2-Categorical Adjunction}{the-quadruple-adjunction-between-sets-and-cats-cannot-be-enhanced-to-a-two-categorical-adjunction}%
    (This is a stub, to be revised and expanded upon later.)

    The discrete category functor of \cref{the-quadruple-adjunction-between-sets-and-cats} lifts to a $2$-functor, but it fails to preserve 2-categorical colimits, and hence lacks a right 2-adjoint. For instance, the 2-pushout of $\pt\leftarrow S^{0}\to\pt$ in $\Sets_{\ldisc}$ is $\pt$, but in $\TwoCategoryOfCategories$ it is given by $\B{\Z}$.
\end{warning}
\subsection{Connected Components and Connected Categories}\label{subsection-connected-components-and-connected-categories}
\subsubsection{Connected Components of Categories}\label{subsubsection-the-quadruple-adjunction-with-sets-connected-components-of-categories}
Let $\CatFont{C}$ be a category.
\begin{definition}{Connected Components of Categories}{connected-components-of-categories}%
    A \index[categories]{category!connected component}\textbf{connected component} of $\CatFont{C}$ is a full subcategory $\CatFont{I}$ of $\CatFont{C}$ satisfying the following conditions:%
    %--- Begin Footnote ---%
    \footnote{%
        In other words, a \textbf{connected component} of $\CatFont{C}$ is an element of the set $\Obj(\CatFont{C})/\unsim$ with $\unsim$ the equivalence relation generated by the relation $\unsim'$ obtained by declaring $A\sim' B$ \textiff there exists a morphism of $\CatFont{C}$ from $A$ to $B$.
        \par\vspace*{\TCBBoxCorrection}
    }%
    %---  End Footnote  ---%
    \begin{enumerate}
        \item\SloganFont{Non-Emptiness. }We have $\Obj(\CatFont{I})\neq\emptyset$.
        \item\SloganFont{Connectedness. }There exists a zigzag of arrows between any two objects of $\CatFont{I}$.
    \end{enumerate}
\end{definition}
\subsubsection{Sets of Connected Components of Categories}\label{subsubsection-the-quadruple-adjunction-with-sets-sets-of-connected-components-of-categories}
Let $\CatFont{C}$ be a category.
\begin{definition}{Sets of Connected Components of Categories}{sets-of-connected-components-of-categories}%
    The \index[categories]{category!connected component, set of}\textbf{set of connected components of $\CatFont{C}$} is the set \index[notation]{pizeroC@$\pi_{0}(\CatFont{C})$}$\pi_{0}(\CatFont{C})$ whose elements are the connected components of $\CatFont{C}$.
\end{definition}
\begin{proposition}{Properties of Sets of Connected Components}{properties-of-sets-of-connected-components-of-categories}%
    Let $\CatFont{C}$ be a category.
    \begin{enumerate}
        \item\label{properties-of-sets-of-connected-components-of-categories-functoriality}\SloganFont{Functoriality. }The assignment $\CatFont{C}\mapsto\pi_{0}(\CatFont{C})$ defines a functor
            \[
                \pi_{0}
                \colon
                \Cats
                \to
                \Sets.
            \]%
        \item\label{properties-of-sets-of-connected-components-of-categories-adjointness}\SloganFont{Adjointness. }We have a quadruple adjunction
            \begin{webcompile}
                \QuadrupleAdjunction#\pi_{0}#{(-)_{\disc}}#\Obj#{(-)_{\indisc}}#\Sets#\Cats.#
            \end{webcompile}%
        \item\label{properties-of-sets-of-connected-components-of-categories-interaction-with-groupoids}\SloganFont{Interaction With Groupoids. }If $\CatFont{C}$ is a groupoid, then we have an isomorphism of categories
            \[
                \pi_{0}(\CatFont{C})%
                \cong%
                \K(\CatFont{C}),%
            \]%
            where $\K(\CatFont{C})$ is the set of isomorphism classes of $\CatFont{C}$ of \cref{the-set-of-isomorphism-classes-of-a-category}.%TODO
        \item\label{properties-of-sets-of-connected-components-of-categories-preservation-of-colimits}\SloganFont{Preservation of Colimits. }The functor $\pi_{0}$ of \cref{properties-of-sets-of-connected-components-of-categories-functoriality} preserves colimits. In particular, we have bijections of sets
            \[
                \begin{gathered}
                    \begin{aligned}
                        \pi_{0}(\CatFont{C}\icoprod\CatFont{D})                       &\cong \pi_{0}(\CatFont{C})\icoprod\pi_{0}(\CatFont{D}),\\
                        \pi_{0}(\CatFont{C}\ipushout{\CatFont{E}}\CatFont{D})         &\cong \pi_{0}(\CatFont{C})\ipushout{\pi_{0}(\CatFont{E})}\pi_{0}(\CatFont{D}),
                    \end{aligned}
                    \\
                    \pi_{0}(\CoEq(\CatFont{C}\xlongrightrightarrows{F}{G}\CatFont{D})) \cong \CoEq(\pi_{0}(\CatFont{C})\xlongrightrightarrows{\pi_{0}(F)}{\pi_{0}(G)}\pi_{0}(\CatFont{D})),
                \end{gathered}
            \]%
            natural in $\CatFont{C},\CatFont{D},\CatFont{E}\in\Obj(\Cats)$.
        \item\label{properties-of-sets-of-connected-components-of-categories-symmetric-strong-monoidality-with-respect-to-coproducts}\SloganFont{Symmetric Strong Monoidality With Respect to Coproducts. }The connected components functor of \cref{properties-of-sets-of-connected-components-of-categories-functoriality} has a symmetric strong monoidal structure
            \[
                (\pi_{0},\pi^{\icoprod}_{0},\pi^{\icoprod}_{0|\Unit})
                \colon
                (\Cats,\icoprod,\EmptyCategory)
                \to
                (\Sets,\icoprod,\emptyset),
            \]%
            being equipped with isomorphisms%
            \[
                \begin{gathered}
                    \pi^{\icoprod}_{0|\CatFont{C},\CatFont{D}} \colon \pi_{0}(\CatFont{C})\icoprod\pi_{0}(\CatFont{D}) \isorightarrow \pi_{0}(\CatFont{C}\icoprod\CatFont{D}),\\
                    \pi^{\icoprod}_{0|\Unit}                   \colon \emptyset                                        \isorightarrow \pi_{0}(\EmptyCategory),
                \end{gathered}
            \]%
            natural in $\CatFont{C},\CatFont{D}\in\Obj(\Cats)$.%
        \item\label{properties-of-sets-of-connected-components-of-categories-symmetric-strong-monoidality-with-respect-to-products}\SloganFont{Symmetric Strong Monoidality With Respect to Products. }The connected components functor of \cref{properties-of-sets-of-connected-components-of-categories-functoriality} has a symmetric strong monoidal structure
            \[
                (\pi_{0},\pi^{\times}_{0},\pi^{\times}_{0|\Unit})
                \colon
                (\Cats,\times,\PunctualCategory)
                \to
                (\Sets,\times,\pt),
            \]%
            being equipped with isomorphisms%
            \[
                \begin{gathered}
                    \pi^{\times}_{0|\CatFont{C},\CatFont{D}} \colon \pi_{0}(\CatFont{C})\times\pi_{0}(\CatFont{D}) \isorightarrow \pi_{0}(\CatFont{C}\times\CatFont{D}),\\
                    \pi^{\times}_{0|\Unit}                   \colon \pt                                            \isorightarrow \pi_{0}(\PunctualCategory),
                \end{gathered}
            \]%
            natural in $\CatFont{C},\CatFont{D}\in\Obj(\Cats)$.%
        %\item\label{properties-of-sets-of-connected-components-of-categories-}\SloganFont{. }
    \end{enumerate}
\end{proposition}
\begin{Proof}{Proof of \cref{properties-of-sets-of-connected-components-of-categories}}%
    \FirstProofBox{\cref{properties-of-sets-of-connected-components-of-categories-functoriality}: Functoriality}%
    Omitted.

    \ProofBox{\cref{properties-of-sets-of-connected-components-of-categories-adjointness}: Adjointness}%
    This is proved in \cref{the-quadruple-adjunction-between-sets-and-cats}.

    \ProofBox{\cref{properties-of-sets-of-connected-components-of-categories-interaction-with-groupoids}: Interaction With Groupoids}%
    Omitted.

    \ProofBox{\cref{properties-of-sets-of-connected-components-of-categories-preservation-of-colimits}: Preservation of Colimits}%
    This follows from \cref{properties-of-sets-of-connected-components-of-categories-adjointness} and \cref{properties-of-adjunctions-interaction-with-co-limits} of \cref{properties-of-adjunctions}.

    \ProofBox{\cref{properties-of-sets-of-connected-components-of-categories-symmetric-strong-monoidality-with-respect-to-coproducts}: Symmetric Strong Monoidality With Respect to Coproducts}%
    Omitted.

    \ProofBox{\cref{properties-of-sets-of-connected-components-of-categories-symmetric-strong-monoidality-with-respect-to-products}: Symmetric Strong Monoidality With Respect to Products}%
    Omitted.
\end{Proof}
\subsubsection{Connected Categories}\label{subsubsection-the-quadruple-adjunction-with-sets-connected-categories}
\begin{definition}{Connected Categories}{connected-categories}%
    A category $\CatFont{C}$ is \index[categories]{category!connected}\textbf{connected} if $\pi_{0}(\CatFont{C})\cong\pt$.%
    %--- Begin Footnote ---%
    \footnote{%
        \SloganFont{Further Terminology: }A category is \index[categories]{category!disconnected}\textbf{disconnected} if it is not connected.
    }%
    %---  End Footnote  ---%
    %--- Begin Footnote ---%
    \footnote{%
        \SloganFont{Example: }A groupoid is connected \textiff any two of its objects are isomorphic.
        \par\vspace*{\TCBBoxCorrection}
    }%
    %---  End Footnote  ---%
\end{definition}
\subsection{Discrete Categories}\label{subsection-the-quadruple-adjunction-with-sets-discrete-categories}
\begin{definition}{Discrete Categories}{discrete-categories}%
    Let $X$ be a set.
    \begin{enumerate}
        \item\label{discrete-categories-the-discrete-category-on-a-set}The \index[categories]{discrete category!on a set}\textbf{discrete category on $X$} is the category \index[notation]{Xdisc@$X_{\disc}$}$X_{\disc}$ where
            \begin{itemize}
                \item\SloganFont{Objects. }We have
                    \[
                        \Obj(X_{\disc})
                        \defeq
                        X.
                    \]%
                \item\SloganFont{Morphisms. }For each $A,B\in\Obj(X_{\disc})$, we have
                    \[
                        \Hom_{X_{\disc}}(A,B)
                        \defeq
                        \begin{cases}
                            \id_{A}   &\text{if $A=B$,}\\
                            \emptyset &\text{if $A\neq B$.}
                        \end{cases}
                    \]
                \item\SloganFont{Identities. }For each $A\in\Obj(X_{\disc})$, the unit map
                    \[
                        \Unit^{X_{\disc}}_{A}
                        \colon
                        \pt
                        \to
                        \Hom_{X_{\disc}}(A,A)
                    \]%
                    of $X_{\disc}$ at $A$ is defined by
                    \[
                        \id^{X_{\disc}}_{A}
                        \defeq
                        \id_{A}.
                    \]%
                \item\SloganFont{Composition. }For each $A,B,C\in\Obj(X_{\disc})$, the composition map
                    \[
                        \circ^{X_{\disc}}_{A,B,C}
                        \colon
                        \Hom_{X_{\disc}}(B,C)
                        \times
                        \Hom_{X_{\disc}}(A,B)
                        \to
                        \Hom_{X_{\disc}}(A,C)
                    \]%
                    of $X_{\disc}$ at $(A,B,C)$ is defined by
                    \[
                        \id_{A}\circ\id_{A}
                        \defeq
                        \id_{A}.
                    \]%
            \end{itemize}
        \item\label{discrete-categories-discrete-categories}A category $\CatFont{C}$ is \index[categories]{category!discrete}\textbf{discrete} if it is equivalent to $X_{\disc}$ for some set $X$.
    \end{enumerate}
\end{definition}
\begin{proposition}{Properties of Discrete Categories on Sets}{properties-of-discrete-categories-on-sets}%
    Let $X$ be a set.
    \begin{enumerate}
        \item\label{properties-of-discrete-categories-on-sets-functoriality}\SloganFont{Functoriality. }The assignment $X\mapsto X_{\disc}$ defines a functor
            \[
                (-)_{\disc}
                \colon
                \Sets
                \to
                \Cats.
            \]%
        \item\label{properties-of-discrete-categories-on-sets-adjointness}\SloganFont{Adjointness. }We have a quadruple adjunction
            \begin{webcompile}
                \QuadrupleAdjunction#\pi_{0}#{(-)_{\disc}}#\Obj#{(-)_{\indisc}}#\Sets#\Cats.#
            \end{webcompile}%
        \item\label{properties-of-discrete-categories-on-sets-symmetric-strong-monoidality-with-respect-to-coproducts}\SloganFont{Symmetric Strong Monoidality With Respect to Coproducts. }The functor of \cref{properties-of-discrete-categories-on-sets-functoriality} has a symmetric strong monoidal structure
            \[
                ((-)_{\disc},(-)^{\icoprod}_{\disc},(-)^{\icoprod}_{\disc|\Unit})
                \colon
                (\Sets,\icoprod,\emptyset)
                \to
                (\Cats,\icoprod,\EmptyCategory),
            \]%
            being equipped with isomorphisms%
            \[
                \begin{gathered}
                    (-)^{\icoprod}_{\disc|X,Y}   \colon X_{\disc}\icoprod Y_{\disc} \isorightarrow (X\icoprod Y)_{\disc},\\
                    (-)^{\icoprod}_{\disc|\Unit} \colon \EmptyCategory              \isorightarrow \emptyset_{\disc},
                \end{gathered}
            \]%
            natural in $X,Y\in\Obj(\Sets)$.%
        \item\label{properties-of-discrete-categories-on-sets-symmetric-strong-monoidality-with-respect-to-products}\SloganFont{Symmetric Strong Monoidality With Respect to Products. }The functor of \cref{properties-of-discrete-categories-on-sets-functoriality} has a symmetric strong monoidal structure
            \[
                ((-)_{\disc},(-)^{\times}_{\disc},(-)^{\times}_{\disc|\Unit})
                \colon
                (\Sets,\times,\pt)
                \to
                (\Cats,\times,\PunctualCategory),
            \]%
            being equipped with isomorphisms%
            \[
                \begin{gathered}
                    (-)^{\times}_{\disc|X,Y}   \colon X_{\disc}\times Y_{\disc} \isorightarrow (X\times Y)_{\disc},\\
                    (-)^{\times}_{\disc|\Unit} \colon \PunctualCategory         \isorightarrow \pt_{\disc},
                \end{gathered}
            \]%
            natural in $X,Y\in\Obj(\Sets)$.%
        %\item\label{properties-of-discrete-categories-on-sets-}\SloganFont{. }
    \end{enumerate}
\end{proposition}
\begin{Proof}{Proof of \cref{properties-of-discrete-categories-on-sets}}%
    \FirstProofBox{\cref{properties-of-discrete-categories-on-sets-functoriality}: Functoriality}%
    \begin{enumerate}
      \item We define the action of $(-)_\disc$ on morphisms $f \colon X \to Y$ between sets $X,Y$, e.g. a functor
            \[f_\disc \colon X_\disc \to Y_\disc.\]
            Let $A$ be an object of $X_\disc$, so $A \in X$.
            Then we set
            \begin{align*}
              f_\disc(A) &\defeq f(A) \in Y = \Obj(Y_\disc),
              f_\disc(\id_A) &\defeq \id_{f(A)}.
            \end{align*}
            Since the identities are the only morphisms in $X_\disc$, this map is well-defined.
            Then, $f_\disc$ is a functor, since the Hom-sets of $Y_\disc$ all have at most one element such that the functor equations must automatically hold.
      \item We need to show that the assignment $(-)_\disc \colon \Sets \to \Cats$ with the previously defined action on morphisms is functorial.
            Let $X,Y,Z$ be sets, $f \colon X \to Y, g \colon Y \to Z$, $A \in X$. Then
            \begin{align*}
            (\id_X)_\disc(A) &= \id_X(A) = A,\\
            (g \circ f)_\disc(A) &= (g\circ f)(A) \\&= g(f(A)) \\&= g_\disc(f_\disc(A)) = (g_\disc \circ f_\disc)(A).
            \end{align*}
            Again, the equations also hold on morphisms in $X_\disc$ due to there being at most one element in the Hom-sets of $Z_\disc$.
            Hence $(\id_X)_\disc = \id_{X_\disc}$, $(g \circ f)_\disc = g_\disc \circ f_\disc$, so $(-)_\disc$ is a functor.
      \end{enumerate}

    \ProofBox{\cref{properties-of-discrete-categories-on-sets-adjointness}: Adjointness}%
    This is proved in \cref{the-quadruple-adjunction-between-sets-and-cats}.

    \ProofBox{\cref{properties-of-discrete-categories-on-sets-symmetric-strong-monoidality-with-respect-to-coproducts}: Symmetric Strong Monoidality With Respect to Coproducts}%
    Omitted.

    \ProofBox{\cref{properties-of-discrete-categories-on-sets-symmetric-strong-monoidality-with-respect-to-products}: Symmetric Strong Monoidality With Respect to Products}%
    Omitted.
\end{Proof}
\subsection{Indiscrete Categories}\label{subsection-the-quadruple-adjunction-with-sets-indiscrete-categories}
\begin{definition}{Indiscrete Categories}{indiscrete-categories}%
    Let $X$ be a set.
    \begin{enumerate}
        \item\label{indiscrete-categories-the-indiscrete-category-on-a-set}The \index[categories]{indiscrete category!on a set}\textbf{indiscrete category on $X$}%
            %--- Begin Footnote ---%
            \footnote{%
                \SloganFont{Further Terminology: }Sometimes called the \textbf{chaotic category on $X$}.
                \par\vspace*{\TCBBoxCorrection}
            } %
            %---  End Footnote  ---%
            is the category \index[notation]{Xindisc@$X_{\indisc}$}$X_{\indisc}$ where
            \begin{itemize}
                \item\SloganFont{Objects. }We have
                    \[
                        \Obj(X_{\indisc})
                        \defeq
                        X.
                    \]%
                \item\SloganFont{Morphisms. }For each $A,B\in\Obj(X_{\indisc})$, we have
                    \begin{align*}
                        \Hom_{X_{\disc}}(A,B) &\defeq \{[A]\to[B]\}\\
                                              &\cong  \pt.
                    \end{align*}
                \item\SloganFont{Identities. }For each $A\in\Obj(X_{\indisc})$, the unit map
                    \[
                        \Unit^{X_{\indisc}}_{A}
                        \colon
                        \pt
                        \to
                        \Hom_{X_{\indisc}}(A,A)
                    \]%
                    of $X_{\indisc}$ at $A$ is defined by
                    \[
                        \id^{X_{\indisc}}_{A}
                        \defeq
                        \{[A]\to[A]\}.
                    \]%
                \item\SloganFont{Composition. }For each $A,B,C\in\Obj(X_{\indisc})$, the composition map
                    \[
                        \circ^{X_{\indisc}}_{A,B,C}
                        \colon
                        \Hom_{X_{\indisc}}(B,C)
                        \times
                        \Hom_{X_{\indisc}}(A,B)
                        \to
                        \Hom_{X_{\indisc}}(A,C)
                    \]%
                    of $X_{\disc}$ at $(A,B,C)$ is defined by
                    \[
                        ([B]\to[C])\circ([A]\to[B])
                        \defeq
                        ([A]\to[C]).
                    \]%
            \end{itemize}
        \item\label{indiscrete-categories-indiscrete-categories}A category $\CatFont{C}$ is \index[categories]{category!indiscrete}\textbf{indiscrete} if it is equivalent to $X_{\indisc}$ for some set $X$.
    \end{enumerate}
\end{definition}
\begin{proposition}{Properties of Indiscrete Categories on Sets}{properties-of-indiscrete-categories-on-sets}%
    Let $X$ be a set.
    \begin{enumerate}
        \item\label{properties-of-indiscrete-categories-on-sets-functoriality}\SloganFont{Functoriality. }The assignment $X\mapsto X_{\indisc}$ defines a functor
            \[
                (-)_{\indisc}
                \colon
                \Sets
                \to
                \Cats.
            \]%
        \item\label{properties-of-indiscrete-categories-on-sets-adjointness}\SloganFont{Adjointness. }We have a quadruple adjunction
            \begin{webcompile}
                \QuadrupleAdjunction#\pi_{0}#{(-)_{\disc}}#\Obj#{(-)_{\indisc}}#\Sets#\Cats.#
            \end{webcompile}%
        \item\label{properties-of-indiscrete-categories-on-sets-symmetric-strong-monoidality-with-respect-to-products}\SloganFont{Symmetric Strong Monoidality With Respect to Products. }The functor of \cref{properties-of-indiscrete-categories-on-sets-functoriality} has a symmetric strong monoidal structure
            \[
                ((-)_{\indisc},(-)^{\times}_{\indisc},(-)^{\times}_{\indisc|\Unit})
                \colon
                (\Sets,\times,\pt)
                \to
                (\Cats,\times,\PunctualCategory),
            \]%
            being equipped with isomorphisms%
            \[
                \begin{gathered}
                    (-)^{\times}_{\indisc|X,Y}   \colon X_{\indisc}\times Y_{\indisc} \isorightarrow (X\times Y)_{\indisc},\\
                    (-)^{\times}_{\indisc|\Unit} \colon \PunctualCategory             \isorightarrow \pt_{\indisc},
                \end{gathered}
            \]%
            natural in $X,Y\in\Obj(\Sets)$.%
        %\item\label{properties-of-indiscrete-categories-on-sets-}\SloganFont{. }
    \end{enumerate}
\end{proposition}
\begin{Proof}{Proof of \cref{properties-of-indiscrete-categories-on-sets}}%
    \FirstProofBox{\cref{properties-of-indiscrete-categories-on-sets-functoriality}: Functoriality}%
    Omitted.

    \ProofBox{\cref{properties-of-indiscrete-categories-on-sets-adjointness}: Adjointness}%
    This is proved in \cref{the-quadruple-adjunction-between-sets-and-cats}.

    \ProofBox{\cref{properties-of-indiscrete-categories-on-sets-symmetric-strong-monoidality-with-respect-to-products}: Symmetric Strong Monoidality With Respect to Products}%
    Omitted.
\end{Proof}
\section{Groupoids}\label{section-groupoids}
\subsection{Isomorphisms}\label{subsection-isomorphisms}
Let $\CatFont{C}$ be a category.
\begin{definition}{Isomorphisms}{isomorphisms}%
    A morphism $f\colon A\to B$ of $\CatFont{C}$ is an \index[categories]{isomorphism}\textbf{isomorphism} if there exists a morphism $\smash{f^{-1}\colon B\to A}$ of $\CatFont{C}$ such that
    \begin{align*}
        f\circ f^{-1} &= \id_{B},\\
        f^{-1}\circ f &= \id_{A}.
    \end{align*}
\end{definition}
\begin{notation}{The Set of Isomorphisms Between Two Objects in a Category}{the-set-of-isomorphisms-between-two-objects-in-a-category}%
    We write \index[notation]{IsoCAB@$\Iso_{\CatFont{C}}(A,B)$}$\Iso_{\CatFont{C}}(A,B)$ for the set of all isomorphisms in $\CatFont{C}$ from $A$ to $B$.
\end{notation}
\subsection{Groupoids}\label{subsection-groupoids}
\begin{definition}{Groupoids}{groupoids}%
    A \index[categories]{groupoid}\textbf{groupoid} is a category in which every morphism is an isomorphism.
\end{definition}
\begin{example}{Groups as One-Object Groupoids}{one-object-groupoids}%
    The isomorphism of categories of \cref{monoids-as-one-object-categories} restricts to an isomorphism
    \begin{webcompile}
        \Grp%
        \cong
        \PunctualCategory\ttimes_{\Sets}\Grpd,%
        \quad
        \begin{tikzcd}[row sep={5.0*\the\DL,between origins}, column sep={5.0*\the\DL,between origins}, background color=backgroundColor, ampersand replacement=\&]
            \Grp
            \arrow[r]
            \arrow[d]
            \arrow[rd,very near start,phantom,"\lrcorner"]
            \&
            \Grpd
            \arrow[d,"\Obj"]
            \\
            \PunctualCategory
            \arrow[r,"{[\pt]}"']
            \&
            \Sets
        \end{tikzcd}
    \end{webcompile}%
    where $\Grpd$ is the full subcategory of $\Cats$ spanned by the groupoids.

    In other words, we have an identification
    \[
        \{\text{Groups}\}
        \cong%
        \{\text{One-object groupoids}\}.
    \]%
    % TODO
\end{example}
\subsection{The Groupoid Completion of a Category}\label{subsection-the-groupoid-completion-of-a-category}
Let $\CatFont{C}$ be a category.
\begin{definition}{The Groupoid Completion of a Category}{the-groupoid-completion-of-a-category}%
    The \index[categories]{groupoid completion}\index[categories]{category!groupoid completion of}\textbf{groupoid completion of $\CatFont{C}$}%
    %--- Begin Footnote ---%
    \footnote{%
        \SloganFont{Further Terminology: }Also called the \index[categories]{Grothendieck groupoid!of a category}\textbf{Grothendieck groupoid of $\CatFont{C}$} or the \textbf{Grothendieck groupoid completion of $\CatFont{C}$}.
        \par\vspace*{\TCBBoxCorrection}
    } %
    %---  End Footnote  ---%
    is the pair $(\K_{0}(\CatFont{C}),\iota_{\CatFont{C}})$ consisting of
    \begin{itemize}
        \item A groupoid \index[notation]{KzeroC@$\K_{0}(\CatFont{C})$}$\K_{0}(\CatFont{C})$;
        \item A functor $\iota_{\CatFont{C}}\colon\CatFont{C}\to\K_{0}(\CatFont{C})$;
    \end{itemize}
    satisfying the following universal property:%
    %--- Begin Footnote ---%
    \footnote{%
        See \cref{properties-of-groupoid-completion-interaction-with-classifying-spaces} of \cref{properties-of-groupoid-completion} for an explicit construction.
        \par\vspace*{\TCBBoxCorrection}
    }%
    %---  End Footnote  ---%

    \begin{itemize}
        \item[\UP]Given another such pair $(\CatFont{G},i)$, there exists a unique functor $\K_{0}(\CatFont{C})\uearrow\CatFont{G}$ making the diagram
            \[
                \begin{tikzcd}[row sep={5.0*\the\DL,between origins}, column sep={5.0*\the\DL,between origins}, background color=backgroundColor, ampersand replacement=\&]
                    \&
                    \K_{0}(\CatFont{C})
                    \arrow[d,"\exists!",densely dashed]
                    \\
                    \CatFont{C}
                    \arrow[r,"i"']
                    \arrow[ru,"\iota_{\CatFont{C}}"]
                    \&
                    \CatFont{G}
                \end{tikzcd}
            \]%
            commute.%
    \end{itemize}
\end{definition}
\begin{construction}{Construction of the Groupoid Completion of a Category}{construction-of-the-groupoid-completion-of-a-category}%
    Concretely, the groupoid completion of $\CatFont{C}$ is the Gabriel--Zisman localisation $\Mor(\CatFont{C})^{-1}\CatFont{C}$ of $\CatFont{C}$ at the set $\Mor(\CatFont{C})$ of all morphisms of $\CatFont{C}$; see \ChapterRef{\ChapterConstructionsWithCategories, \cref{constructions-with-categories:section-gabriel-zisman-localisations}}{\cref{section-gabriel-zisman-localisations}}.

    (To be expanded upon later on.)
\end{construction}
\begin{Proof}{Proof of \cref{construction-of-the-groupoid-completion-of-a-category}}%
    Omitted.
\end{Proof}
\begin{proposition}{Properties of Groupoid Completion}{properties-of-groupoid-completion}%
    Let $\CatFont{C}$ be a category.
    \begin{enumerate}
        \item\label{properties-of-groupoid-completion-functoriality}\SloganFont{Functoriality. }The assignment $\CatFont{C}\mapsto\K_{0}(\CatFont{C})$ defines a functor
            \[
                \K_{0}
                \colon
                \Cats
                \to
                \Grpd.
            \]%
        \item\label{properties-of-groupoid-completion-2-functoriality}\SloganFont{2-Functoriality. }The assignment $\CatFont{C}\mapsto\K_{0}(\CatFont{C})$ defines a 2-functor
            \[
                \K_{0}
                \colon
                \TwoCategoryOfCategories
                \to
                \TwoCategoryOfGroupoids.
            \]%
        \item\label{properties-of-groupoid-completion-adjointness}\SloganFont{Adjointness. }We have an adjunction
            \begin{webcompile}
                \StraightHookAdjunction#\K_{0}#\iota#\Cats#\Grpd,#%
            \end{webcompile}%
            witnessed by a bijection of sets
            \[
                \Hom_{\Grpd}(\K_{0}(\CatFont{C}),\CatFont{G})%
                \cong
                \Hom_{\Cats}(\CatFont{C},\CatFont{G}),%
            \]%
            natural in $\CatFont{C}\in\Obj(\Cats)$ and $\CatFont{G}\in\Obj(\Grpd)$, forming, together with the functor $\Core$ of \cref{properties-of-the-core-of-a-category-functoriality} of \cref{properties-of-the-core-of-a-category}, a triple adjunction
            \begin{webcompile}
                \HookTripleAdjunction#\K_{0}#\iota#\Core#\Cats#\Grpd,#
            \end{webcompile}%
            witnessed by bijections of sets
            \begin{align*}
                \Hom_{\Grpd}(\K_{0}(\CatFont{C}),\CatFont{G}) &\cong \Hom_{\Cats}(\CatFont{C},\CatFont{G}),\\%
                \Hom_{\Cats}(\CatFont{G},\CatFont{D})         &\cong \Hom_{\Grpd}(\CatFont{G},\Core(\CatFont{D})),%
            \end{align*}
            natural in $\CatFont{C},\CatFont{D}\in\Obj(\Cats)$ and $\CatFont{G}\in\Obj(\Grpd)$.
        \item\label{properties-of-groupoid-completion-2-adjointness}\SloganFont{2-Adjointness. }We have a 2-adjunction
            \begin{webcompile}
                \StraightHookTwoAdjunction#\K_{0}#\iota#\Cats#\Grpd,#%
            \end{webcompile}%
            witnessed by an isomorphism of categories
            \[
                \Fun(\K_{0}(\CatFont{C}),\CatFont{G})%
                \cong
                \Fun(\CatFont{C},\CatFont{G}),%
            \]%
            natural in $\CatFont{C}\in\Obj(\Cats)$ and $\CatFont{G}\in\Obj(\Grpd)$, forming, together with the 2-functor $\Core$ of \cref{properties-of-the-core-of-a-category-2-functoriality} of \cref{properties-of-the-core-of-a-category}, a triple 2-adjunction
            \begin{webcompile}
                \HookTripleTwoAdjunction#\K_{0}#\iota#\Core#\Cats#\Grpd,#
            \end{webcompile}%
            witnessed by isomorphisms of categories
            \begin{align*}
                \Fun(\K_{0}(\CatFont{C}),\CatFont{G}) &\cong \Fun(\CatFont{C},\CatFont{G}),\\%
                \Fun(\CatFont{G},\CatFont{D})         &\cong \Fun(\CatFont{G},\Core(\CatFont{D})),%
            \end{align*}
            natural in $\CatFont{C},\CatFont{D}\in\Obj(\Cats)$ and $\CatFont{G}\in\Obj(\Grpd)$.
        \item\label{properties-of-groupoid-completion-interaction-with-classifying-spaces}\SloganFont{Interaction With Classifying Spaces. }We have an isomorphism of groupoids
            \[
                \K_{0}(\CatFont{C})
                \cong
                \Pi_{\leq1}(\abs{\NerveB(\CatFont{C})}),
            \]%
            natural in $\CatFont{C}\in\Obj(\Cats)$; i.e.\ the diagram
            \[
                \begin{tikzcd}[row sep={5.0*\the\DL,between origins}, column sep={2.5*\the\DL,between origins}, background color=backgroundColor, ampersand replacement=\&]
                    \Cats
                    \arrow[rr,"{\K_{0}}"]
                    \arrow[d,"\Nerve_{\bullet}"']
                    \&
                    {}
                    \&
                    \Grp
                    \\
                    \sSets
                    \arrow[rr,"{\abs{-}}"']
                    \&
                    {}
                    \&
                    \Top
                    \arrow[u,"\Pi_{\leq1}"']
                    % 2-Arrows
                    \arrow[from=2-2,to=1-2,densely dashed,shorten <= 0.375em,shorten >= 0.0em,Leftrightarrow,"\scalebox{2.0}{$\unsim$}"{sloped,inner sep=0.0em,pos=0.55}]%
                \end{tikzcd}
            \]%
            commutes up to natural isomorphism.
        \item\label{properties-of-groupoid-completion-symmetric-strong-monoidality-with-respect-to-coproducts}\SloganFont{Symmetric Strong Monoidality With Respect to Coproducts. }The groupoid completion functor of \cref{properties-of-groupoid-completion-functoriality} has a symmetric strong monoidal structure
            \[
                (\K_{0},\K^{\icoprod}_{0},\K^{\icoprod}_{0|\Unit})
                \colon
                (\Cats,\icoprod,\EmptyCategory)
                \to
                (\Grpd,\icoprod,\EmptyCategory)
            \]%
            being equipped with isomorphisms%
            \[
                \begin{gathered}
                    \K^{\icoprod}_{0|\CatFont{C},\CatFont{D}} \colon \K_{0}(\CatFont{C})\icoprod\K_{0}(\CatFont{D}) \isorightarrow \K_{0}(\CatFont{C}\icoprod\CatFont{D}),\\
                    \K^{\icoprod}_{0|\Unit}                   \colon \EmptyCategory                                 \isorightarrow \K_{0}(\EmptyCategory),
                \end{gathered}
            \]%
            natural in $\CatFont{C},\CatFont{D}\in\Obj(\Cats)$.
        \item\label{properties-of-groupoid-completion-symmetric-strong-monoidality-with-respect-to-products}\SloganFont{Symmetric Strong Monoidality With Respect to Products. }The groupoid completion functor of \cref{properties-of-groupoid-completion-functoriality} has a symmetric strong monoidal structure
            \[
                (\K_{0},\K^{\times}_{0},\K^{\times}_{0|\Unit})
                \colon
                (\Cats,\times,\PunctualCategory)
                \to
                (\Grpd,\times,\PunctualCategory)
            \]%
            being equipped with isomorphisms%
            \[
                \begin{gathered}
                    \K^{\times}_{0|\CatFont{C},\CatFont{D}} \colon \K_{0}(\CatFont{C})\times\K_{0}(\CatFont{D}) \isorightarrow \K_{0}(\CatFont{C}\times\CatFont{D}),\\
                    \K^{\times}_{0|\Unit}                   \colon \PunctualCategory                              \isorightarrow \K_{0}(\PunctualCategory),
                \end{gathered}
            \]%
            natural in $\CatFont{C},\CatFont{D}\in\Obj(\Cats)$.
        %\item\label{properties-of-groupoid-completion-}\SloganFont{. }
    \end{enumerate}
\end{proposition}
\begin{Proof}{Proof of \cref{properties-of-groupoid-completion}}%
    \FirstProofBox{\cref{properties-of-groupoid-completion-functoriality}: Functoriality}%
    Omitted.

    \ProofBox{\cref{properties-of-groupoid-completion-2-functoriality}: 2-Functoriality}%
    Omitted.

    \ProofBox{\cref{properties-of-groupoid-completion-adjointness}: Adjointness}%
    Omitted.

    \ProofBox{\cref{properties-of-groupoid-completion-2-adjointness}: 2-Adjointness}%
    Omitted.

    \ProofBox{\cref{properties-of-groupoid-completion-interaction-with-classifying-spaces}: Interaction With Classifying Spaces}%
    See Corollary 18.33 of \url{https://web.ma.utexas.edu/users/dafr/M392C-2012/Notes/lecture18.pdf}.

    \ProofBox{\cref{properties-of-groupoid-completion-symmetric-strong-monoidality-with-respect-to-coproducts}: Symmetric Strong Monoidality With Respect to Coproducts}%
    Omitted.

    \ProofBox{\cref{properties-of-groupoid-completion-symmetric-strong-monoidality-with-respect-to-products}: Symmetric Strong Monoidality With Respect to Products}%
    Omitted.
\end{Proof}
\subsection{The Core of a Category}\label{subsection-the-core-of-a-category}
Let $\CatFont{C}$ be a category.
\begin{definition}{The Core of a Category}{the-core-of-a-category}%
    The \index[categories]{category!core of}\textbf{core} of $\CatFont{C}$ is the pair $(\Core(\CatFont{C}),\iota_{\CatFont{C}})$ consisting of
    \begin{itemize}
        \item A groupoid \index[notation]{CoreC@$\Core(\CatFont{C})$}$\Core(\CatFont{C})$;
        \item A functor $\iota_{\CatFont{C}}\colon\Core(\CatFont{C})\longhookrightarrow\CatFont{C}$;
    \end{itemize}
    satisfying the following universal property:

    \begin{itemize}
        \item[\UP]Given another such pair $(\CatFont{G},i)$, there exists a unique functor $\CatFont{G}\uearrow\Core(\CatFont{C})$ making the diagram
            \[
                \begin{tikzcd}[row sep={5.0*\the\DL,between origins}, column sep={5.0*\the\DL,between origins}, background color=backgroundColor, ampersand replacement=\&]
                    \&
                    \Core(\CatFont{C})
                    \arrow[d, "\iota_{\CatFont{C}}",hook']
                    \\
                    \CatFont{G}
                    \arrow[ru, "\exists!", dashed]
                    \arrow[r, "i"']
                    \&
                    \CatFont{C}
                \end{tikzcd}
            \]%
            commute.
    \end{itemize}
\end{definition}
\begin{notation}{Alternative Notation for the Core of a Category}{alternative-notation-for-the-core-of-a-category}%
    We also write \index[notation]{Csimeq@$\CatFont{C}^{\simeq}$}$\CatFont{C}^{\simeq}$ for $\Core(\CatFont{C})$.
\end{notation}
\begin{construction}{Construction of the Core of a Category}{construction-of-the-core-of-a-category}%
    The core of $\CatFont{C}$ is the wide subcategory of $\CatFont{C}$ spanned by the isomorphisms of $\CatFont{C}$, i.e.\ the category $\Core(\CatFont{C})$ where%
    %--- Begin Footnote ---%
    \footnote{%
        \SloganFont{Slogan: }The groupoid $\Core(\CatFont{C})$ is the maximal subgroupoid of $\CatFont{C}$.
        \par\vspace*{\TCBBoxCorrection}
    }%
    %---  End Footnote  ---%
    \begin{enumerate}
        \item\SloganFont{Objects. }We have
            \[
                \Obj(\Core(\CatFont{C}))
                \defeq
                \Obj(\CatFont{C}).
            \]%
        \item\SloganFont{Morphisms. }The morphisms of $\Core(\CatFont{C})$ are the isomorphisms of $\CatFont{C}$.
    \end{enumerate}
\end{construction}
\begin{Proof}{Proof of \cref{construction-of-the-core-of-a-category}}%
    This follows from the fact that functors preserve isomorphisms (\cref{elementary-properties-of-functors-preservation-of-isomorphisms} of \cref{elementary-properties-of-functors}).
\end{Proof}
\begin{proposition}{Properties of the Core of a Category}{properties-of-the-core-of-a-category}%
    Let $\CatFont{C}$ be a category.
    \begin{enumerate}
        \item\label{properties-of-the-core-of-a-category-functoriality}\SloganFont{Functoriality. }The assignment $\CatFont{C}\mapsto\Core(\CatFont{C})$ defines a functor
            \[
                \Core
                \colon
                \Cats
                \to
                \Grpd.
            \]%
        \item\label{properties-of-the-core-of-a-category-2-functoriality}\SloganFont{2-Functoriality. }The assignment $\CatFont{C}\mapsto\Core(\CatFont{C})$ defines a 2-functor
            \[
                \Core
                \colon
                \TwoCategoryOfCategories%
                \to
                \TwoCategoryOfGroupoids.%
            \]%
        \item\label{properties-of-the-core-of-a-category-adjointness}\SloganFont{Adjointness. }We have an adjunction
            \begin{webcompile}
                \varHookAdjunction#\iota#\Core#\Grpd#\Cats,#
            \end{webcompile}%
            witnessed by a bijection of sets
            \[
                \Hom_{\Cats}(\CatFont{G},\CatFont{D})%
                \cong
                \Hom_{\Grpd}(\CatFont{G},\Core(\CatFont{D})),%
            \]%
            natural in $\CatFont{G}\in\Obj(\Grpd)$ and $\CatFont{D}\in\Obj(\Cats)$, forming, together with the functor $\K_{0}$ of \cref{properties-of-groupoid-completion-functoriality} of \cref{properties-of-groupoid-completion}, a triple adjunction
            \begin{webcompile}
                \HookTripleAdjunction#\K_{0}#\iota#\Core#\Cats#\Grpd,#
            \end{webcompile}%
            witnessed by bijections of sets
            \begin{align*}
                \Hom_{\Grpd}(\K_{0}(\CatFont{C}),\CatFont{G}) &\cong \Hom_{\Cats}(\CatFont{C},\CatFont{G}),\\%
                \Hom_{\Cats}(\CatFont{G},\CatFont{D})         &\cong \Hom_{\Grpd}(\CatFont{G},\Core(\CatFont{D})),%
            \end{align*}
            natural in $\CatFont{C},\CatFont{D}\in\Obj(\Cats)$ and $\CatFont{G}\in\Obj(\Grpd)$.
        \item\label{properties-of-the-core-of-a-category-2-adjointness}\SloganFont{2-Adjointness. }We have an adjunction
            \begin{webcompile}
                \varHookTwoAdjunction#\iota#\Core#\Grpd#\Cats,#
            \end{webcompile}%
            witnessed by an isomorphism of categories
            \[
                \Fun(\CatFont{G},\CatFont{D})%
                \cong
                \Fun(\CatFont{G},\Core(\CatFont{D})),%
            \]%
            natural in $\CatFont{G}\in\Obj(\Grpd)$ and $\CatFont{D}\in\Obj(\Cats)$, forming, together with the 2-functor $\K_{0}$ of \cref{properties-of-groupoid-completion-2-functoriality} of \cref{properties-of-groupoid-completion}, a triple 2-adjunction
            \begin{webcompile}
                \HookTripleTwoAdjunction#\K_{0}#\iota#\Core#\Cats#\Grpd,#
            \end{webcompile}%
            witnessed by isomorphisms of categories
            \begin{align*}
                \Fun(\K_{0}(\CatFont{C}),\CatFont{G}) &\cong \Fun(\CatFont{C},\CatFont{G}),\\%
                \Fun(\CatFont{G},\CatFont{D})         &\cong \Fun(\CatFont{G},\Core(\CatFont{D})),%
            \end{align*}
            natural in $\CatFont{C},\CatFont{D}\in\Obj(\Cats)$ and $\CatFont{G}\in\Obj(\Grpd)$.
        \item\label{properties-of-the-core-of-a-category-symmetric-strong-monoidality-with-respect-to-products}\SloganFont{Symmetric Strong Monoidality With Respect to Products. }The core functor of \cref{properties-of-the-core-of-a-category-functoriality} has a symmetric strong monoidal structure
            \[
                (\Core,\Core^{\times},\Core^{\times}_{\Unit})
                \colon
                (\Cats,\times,\PunctualCategory)
                \to
                (\Grpd,\times,\PunctualCategory)
            \]%
            being equipped with isomorphisms%
            \[
                \begin{gathered}
                    \Core^{\times}_{\CatFont{C},\CatFont{D}} \colon \Core(\CatFont{C})\times\Core(\CatFont{D}) \isorightarrow \Core(\CatFont{C}\times\CatFont{D}),\\
                    \Core^{\times}_{\Unit}                     \colon \PunctualCategory                          \isorightarrow \Core(\PunctualCategory),
                \end{gathered}
            \]%
            natural in $\CatFont{C},\CatFont{D}\in\Obj(\Cats)$.
        \item\label{properties-of-the-core-of-a-category-symmetric-strong-monoidality-with-respect-to-coproducts}\SloganFont{Symmetric Strong Monoidality With Respect to Coproducts. }The core functor of \cref{properties-of-the-core-of-a-category-functoriality} has a symmetric strong monoidal structure
            \[
                (\Core,\Core^{\icoprod},\Core^{\icoprod}_{\Unit})
                \colon
                (\Cats,\icoprod,\EmptyCategory)
                \to
                (\Grpd,\icoprod,\EmptyCategory)
            \]%
            being equipped with isomorphisms%
            \[
                \begin{gathered}
                    \Core^{\icoprod}_{\CatFont{C},\CatFont{D}} \colon \Core(\CatFont{C})\icoprod\Core(\CatFont{D}) \isorightarrow \Core(\CatFont{C}\icoprod\CatFont{D}),\\
                    \Core^{\icoprod}_{\Unit}                   \colon \EmptyCategory                               \isorightarrow \Core(\EmptyCategory),
                \end{gathered}
            \]%
            natural in $\CatFont{C},\CatFont{D}\in\Obj(\Cats)$.
        %\item\label{properties-of-the-core-of-a-category-}\SloganFont{. }
    \end{enumerate}
\end{proposition}
\begin{Proof}{Proof of \cref{properties-of-the-core-of-a-category}}%
    \FirstProofBox{\cref{properties-of-the-core-of-a-category-functoriality}: Functoriality}%
    Omitted.

    \ProofBox{\cref{properties-of-the-core-of-a-category-2-functoriality}: 2-Functoriality}%
    Omitted.

    \ProofBox{\cref{properties-of-the-core-of-a-category-adjointness}: Adjointness}%
    Omitted.%
    %The adjunction $(\K_{0}\dashv\iota)$ follows from the universal property of the Gabriel--Zisman localisation of a category with respect to a class of morphisms %(\ChapterRef{\ChapterModelCategories, \cref{model-categories:gabriel-zisman-localisations}}{\cref{gabriel-zisman-localisations}})
    %(\cref{gabriel-zisman-localisations}), while the adjunction $(\iota\dashv\Core)$ is a reformulation of the universal property of the core of a category (\cref{the-core-of-a-category}).%
    %%--- Begin Footnote ---%
    %\footnote{%
    %    \SloganFont{Reference: }\cite[Example 4.1.15]{category-theory-in-context}.%
    \par\vspace*{\TCBBoxCorrection}
    %}%
    %%---  End Footnote  ---%

    \ProofBox{\cref{properties-of-the-core-of-a-category-2-adjointness}: 2-Adjointness}%
    Omitted.

    \ProofBox{\cref{properties-of-the-core-of-a-category-symmetric-strong-monoidality-with-respect-to-products}: Symmetric Strong Monoidality With Respect to Products}%
    Omitted.

    \ProofBox{\cref{properties-of-the-core-of-a-category-symmetric-strong-monoidality-with-respect-to-coproducts}: Symmetric Strong Monoidality With Respect to Coproducts}%
    Omitted.
\end{Proof}
\section{Functors}\label{section-functors}
\subsection{Foundations}\label{subsection-functors-foundations}
Let $\CatFont{C}$ and $\CatFont{D}$ be categories.
\begin{definition}{Functors}{functors}%
    A \index[categories]{functor}\textbf{functor $F\colon\CatFont{C}\to\CatFont{D}$ from $\CatFont{C}$ to $\CatFont{D}$}%
    %--- Begin Footnote ---%
    \footnote{%
        \SloganFont{Further Terminology: }Also called a \textbf{covariant functor}.
    } %
    %---  End Footnote  ---%
    consists of:
    \begin{enumerate}
        \item\SloganFont{Action on Objects. }A map of sets
            \[
                F
                \colon
                \Obj(\CatFont{C})
                \to
                \Obj(\CatFont{D}),
            \]%
            called the \textbf{action on objects of $F$}.
        \item\SloganFont{Action on Morphisms. }For each $A,B\in\Obj(\CatFont{C})$, a map
            \[
                F_{A,B}
                \colon
                \Hom_{\CatFont{C}}(A,B)
                \to
                \Hom_{\CatFont{D}}(F(A),F(B)),
            \]
            called the \textbf{action on morphisms of $F$ at $(A,B)$}%
            %--- Begin Footnote ---%
            \footnote{%
                \SloganFont{Further Terminology: }Also called \textbf{action on $\Hom$-sets of $F$ at $(A,B)$}.
                \par\vspace*{\TCBBoxCorrection}
            }.%
            %---  End Footnote  ---%
    \end{enumerate}
    satisfying the following conditions:%
    \begin{enumerate}
        \item\SloganFont{Preservation of Identities. }For each $A\in\Obj(\CatFont{C})$, the diagram
            \[
                \begin{tikzcd}[row sep={5.0*\the\DL,between origins}, column sep={10.5*\the\DL,between origins}, background color=backgroundColor, ampersand replacement=\&]
                    \pt
                    \arrow[d, "\Unit^{\CatFont{C}}_{A}"']
                    \arrow[rd,"\Unit^{\CatFont{D}}_{F(A)}"]
                    \&
                    \\
                    \Hom_{\CatFont{C}}(A,A)
                    \arrow[r,"F_{A,A}"']
                    \&
                    \Hom_{\CatFont{D}}(F(A),F(A))
                \end{tikzcd}
            \]%
            commutes, i.e.\ we have
            \[
                F(\id_{A})
                =
                \id_{F(A)}.
            \]
        \item\SloganFont{Preservation of Composition. }For each $A,B,C\in\Obj(\CatFont{C})$, the diagram
            \[
                \begin{tikzcd}[row sep={5.0*\the\DL,between origins}, column sep={20.5*\the\DL,between origins}, background color=backgroundColor, ampersand replacement=\&]
                    {\Hom_{\CatFont{C}}(B,C)\times\Hom_{\CatFont{C}}(A,B)}
                    \arrow[d, "{F_{B,C}\times F_{A,B}}"']
                    \arrow[r, "\circ^{\CatFont{C}}_{A,B,C}"]
                    \&
                    {\Hom_{\CatFont{C}}(A,C)}
                    \arrow[d, "F_{A,C}"]
                    \\
                    {\Hom_{\CatFont{D}}(F(B),F(C))\times\Hom_{\CatFont{D}}(F(A),F(B))}
                    \arrow[r, "\circ^{\CatFont{D}}_{F(A),F(B),F(C)}"']
                    \&
                    {\Hom_{\CatFont{D}}(F(A),F(C))}
                \end{tikzcd}
            \]%
            commutes, i.e.\ for each composable pair $(g,f)$ of morphisms of $\CatFont{C}$, we have
            \[
                F(g\circ f)
                =
                F(g)\circ F(f).
            \]
    \end{enumerate}
\end{definition}
\begin{notation}{Subscript and Superscript Notation for Functors}{subscript-and-superscript-notation-for-functors}%
    Let $\CatFont{C}$ and $\CatFont{D}$ be categories, and write $\CatFont{C}^{\op}$ for the opposite category of $\CatFont{C}$ of \ChapterRef{\ChapterConstructionsWithCategories, \cref{constructions-with-categories:opposite-categories}}{\cref{opposite-categories}}.
    \begin{enumerate}
        \item\label{subscript-and-superscript-notation-for-functors-1}Given a functor
            \[
                F%
                \colon%
                \CatFont{C}%
                \to%
                \CatFont{D},%
            \]%
            we also write $F_{A}$ for $F(A)$.
        \item\label{subscript-and-superscript-notation-for-functors-2}Given a functor
            \[
                F%
                \colon%
                \CatFont{C}^{\op}%
                \to%
                \CatFont{D},%
            \]%
            we also write $F^{A}$ for $F(A)$.
        \item\label{subscript-and-superscript-notation-for-functors-3}Given a functor
            \[
                F%
                \colon%
                \CatFont{C}\times\CatFont{C}%
                \to%
                \CatFont{D},%
            \]%
            we also write $F_{A,B}$ for $F(A,B)$.
        \item\label{subscript-and-superscript-notation-for-functors-4}Given a functor
            \[
                F%
                \colon%
                \CatFont{C}^{\op}\times\CatFont{C}%
                \to%
                \CatFont{D},%
            \]%
            we also write $F^{A}_{B}$ for $F(A,B)$.
    \end{enumerate}
    We employ a similar notation for morphisms, writing e.g.\ $F_{f}$ for $F(f)$ given a functor $F\colon\CatFont{C}\to\CatFont{D}$.
\end{notation}
\begin{notation}{Additional Notation for Functors}{additional-notation-for-functors}%
    Following the notation $\llbracket x\mapsto f(x)\rrbracket$ for a function $f\colon X\to Y$ introduced in \ChapterRef{\ChapterSets, \cref{sets:additional-notation-for-functions}}{\cref{additional-notation-for-functions}}, we will sometimes denote a functor $F\colon\CatFont{C}\to\CatFont{D}$ by
    \[
        F%
        \defeq%
        \llbracket A\mapsto F(A)\rrbracket,%
    \]%
    specially when the action on morphisms of $F$ is clear from its action on objects.
\end{notation}
\begin{example}{Identity Functors}{identity-functors}%
    The \index[categories]{functor!identity}\textbf{identity functor} of a category $\CatFont{C}$ is the functor \index[notation]{idC@$\id_{\CatFont{C}}$}$\id_{\CatFont{C}}\colon\CatFont{C}\to\CatFont{C}$ where
    \begin{enumerate}
        \item\SloganFont{Action on Objects. }For each $A\in\Obj(\CatFont{C})$, we have
            \[
                \id_{\CatFont{C}}(A)
                \defeq
                A.
            \]
        \item\SloganFont{Action on Morphisms. }For each $A,B\in\Obj(\CatFont{C})$, the action on morphisms
            \[
                (\id_{\CatFont{C}})_{A,B}
                \colon
                \Hom_{\CatFont{C}}(A,B)
                \to
                \underbrace{\Hom_{\CatFont{C}}(\id_{\CatFont{C}}(A),\id_{\CatFont{C}}(B))}_{\defeq\Hom_{\CatFont{C}}(A,B)}
            \]%
            of $\id_{\CatFont{C}}$ at $(A,B)$ is defined by
            \[
                (\id_{\CatFont{C}})_{A,B}
                \defeq
                \id_{\Hom_{\CatFont{C}}(A,B)}.
            \]
    \end{enumerate}
\end{example}
\begin{Proof}{Proof of \cref{identity-functors}}%
    \FirstProofBox{Preservation of Identities}%
    We have $\id_{\CatFont{C}}(\id_{A})\defeq\id_{A}$ for each $A\in\Obj(\CatFont{C})$ by definition.

    \ProofBox{Preservation of Compositions}%
    For each composable pair $A\xrightarrow{f}B\xrightarrow{g}B$ of morphisms of $\CatFont{C}$, we have
    \begin{align*}
        \id_{\CatFont{C}}(g\circ f) &\defeq g\circ f\\
                                    &\defeq \id_{\CatFont{C}}(g)\circ\id_{\CatFont{C}}(f).
    \end{align*}
    This finishes the proof.
\end{Proof}
\begin{definition}{Composition of Functors}{composition-of-functors}%
    The \index[categories]{functor!composition of}\textbf{composition} of two functors $F\colon\CatFont{C}\to\CatFont{D}$ and $G\colon\CatFont{D}\to\CatFont{E}$ is the functor $G\circ F$ where
    \begin{itemize}
        \item\SloganFont{Action on Objects. }For each $A\in\Obj(\CatFont{C})$, we have
            \[
                [G\circ F](A)
                \defeq
                G(F(A)).
            \]%
        \item\SloganFont{Action on Morphisms. }For each $A,B\in\Obj(\CatFont{C})$, the action on morphisms
            \[
                (G\circ F)_{A,B}
                \colon
                \Hom_{\CatFont{C}}(A,B)
                \to
                \Hom_{\CatFont{E}}(G_{F_{A}},G_{F_{B}})
            \]%
            of $G\circ F$ at $(A,B)$ is defined by
            \[
                [G\circ F](f)
                \defeq
                G(F(f)).
            \]%
    \end{itemize}
\end{definition}
\begin{Proof}{Proof of \cref{composition-of-functors}}%
    \FirstProofBox{Preservation of Identities}%
    For each $A\in\Obj(\CatFont{C})$, we have
    \begin{align*}
        G_{F_{\id_{A}}} &= G_{\id_{F_{A}}}\tag{functoriality of $F$}\\
                        &= \id_{G_{F_{A}}}.\tag{functoriality of $G$}
    \end{align*}

    \ProofBox{Preservation of Composition}%
    For each composable pair $(g,f)$ of morphisms of $\CatFont{C}$, we have
    \begin{align*}
        G_{F_{g\circ f}} &= G_{F_{g}\circ F_{f}}\tag{functoriality of $F$}\\
                         &= G_{F_{g}}\circ G_{F_{f}}.\tag{functoriality of $G$}
    \end{align*}
    This finishes the proof.
\end{Proof}
\begin{proposition}{Elementary Properties of Functors}{elementary-properties-of-functors}%
    Let $F\colon\CatFont{C}\to\CatFont{D}$ be a functor.
    \begin{enumerate}
        \item\label{elementary-properties-of-functors-preservation-of-isomorphisms}\SloganFont{Preservation of Isomorphisms. }If $f$ is an isomorphism in $\CatFont{C}$, then $F(f)$ is an isomorphism in $\CatFont{D}$.%
            %--- Begin Footnote ---%
            \footnote{%
                When the converse holds, we call $F$ \emph{conservative}, see \cref{conservative-functors}.
                \par\vspace*{\TCBBoxCorrection}
            }%
            %---  End Footnote  ---%
        %\item\label{elementary-properties-of-functors-}\SloganFont{. }
    \end{enumerate}
\end{proposition}
\begin{Proof}{Proof of \cref{elementary-properties-of-functors}}%
    \FirstProofBox{\cref{elementary-properties-of-functors-preservation-of-isomorphisms}: Preservation of Isomorphisms}%
    Indeed, we have
    \begin{align*}
        F(f)^{-1}\circ F(f) &= F(f^{-1}\circ f)\\
                            &= F(\id_{A})\\
                            &= \id_{F(A)}
    \end{align*}
    and
    \begin{align*}
        F(f)\circ F(f)^{-1} &= F(f\circ f^{-1})\\
                            &= F(\id_{B})\\
                            &= \id_{F(B)},
    \end{align*}
    showing $F(f)$ to be an isomorphism.
\end{Proof}
\subsection{Contravariant Functors}\label{subsection-contravariant-functors}
Let $\CatFont{C}$ and $\CatFont{D}$ be categories, and let $\CatFont{C}^{\op}$ denote the opposite category of $\CatFont{C}$ of \ChapterRef{\ChapterConstructionsWithCategories, \cref{constructions-with-categories:opposite-categories}}{\cref{opposite-categories}}.
\begin{definition}{Contravariant Functors}{contravariant-functors}%
    A \index[categories]{functor!contravariant}\index[categories]{contravariant functor}\textbf{contravariant functor} from $\CatFont{C}$ to $\CatFont{D}$ is a functor from $\CatFont{C}^{\op}$ to $\CatFont{D}$.
\end{definition}
\begin{remark}{Unwinding \cref{contravariant-functors}}{unwinding-contravariant-functors}%
    In detail, a \textbf{contravariant functor} from $\CatFont{C}$ to $\CatFont{D}$ consists of:%
    \begin{enumerate}
        \item\SloganFont{Action on Objects. }A map of sets
            \[
                F
                \colon
                \Obj(\CatFont{C})
                \to
                \Obj(\CatFont{D}),
            \]%
            called the \textbf{action on objects of $F$}.
        \item\SloganFont{Action on Morphisms. }For each $A,B\in\Obj(\CatFont{C})$, a map
            \[
                F_{A,B}
                \colon
                \Hom_{\CatFont{C}}(A,B)
                \to
                \Hom_{\CatFont{D}}(F(B),F(A)),
            \]
            called the \textbf{action on morphisms of $F$ at $(A,B)$}.%
    \end{enumerate}
    satisfying the following conditions:%
    \begin{enumerate}
        \item\SloganFont{Preservation of Identities. }For each $A\in\Obj(\CatFont{C})$, the diagram
            \[
                \begin{tikzcd}[row sep={5.0*\the\DL,between origins}, column sep={10.5*\the\DL,between origins}, background color=backgroundColor, ampersand replacement=\&]
                    \pt
                    \arrow[d, "\Unit^{\CatFont{C}}_{A}"']
                    \arrow[rd,"\Unit^{\CatFont{D}}_{F(A)}"]
                    \&
                    \\
                    \Hom_{\CatFont{C}}(A,A)
                    \arrow[r,"F_{A,A}"']
                    \&
                    \Hom_{\CatFont{D}}(F(A),F(A))
                \end{tikzcd}
            \]%
            commutes, i.e.\ we have
            \[
                F(\id_{A})
                =
                \id_{F(A)}.
            \]
        \item\SloganFont{Preservation of Composition. }For each $A,B,C\in\Obj(\CatFont{C})$, the diagram
            \begin{scalemath}
                \begin{tikzcd}[row sep={0*\the\DL,between origins}, column sep={0*\the\DL,between origins}, background color=backgroundColor, ampersand replacement=\&]
                    \&[0.30901699437\ThreeCmPlusHalf]
                    \&[0.5\ThreeCmPlusHalf]
                    {\Hom_{\CatFont{D}}(F(C),F(B))\times\Hom_{\CatFont{D}}(F(B),F(A))}
                    \&[0.5\ThreeCmPlusHalf]
                    \&[0.30901699437\ThreeCmPlusHalf]
                    \\[0.58778525229\ThreeCmPlusHalf]
                    {\Hom_{\CatFont{C}}(B,C)\times\Hom_{\CatFont{C}}(A,B)}
                    \&[0.30901699437\ThreeCmPlusHalf]
                    \&[0.5\ThreeCmPlusHalf]
                    \&[0.5\ThreeCmPlusHalf]
                    \&[0.30901699437\ThreeCmPlusHalf]
                    {\Hom_{\CatFont{D}}(F(B),F(A))\times\Hom_{\CatFont{D}}(F(C),F(B))}
                    \\[0.95105651629\ThreeCmPlusHalf]
                    \&[0.30901699437\ThreeCmPlusHalf]
                    {\Hom_{\CatFont{C}}(A,C)}
                    \&[0.5\ThreeCmPlusHalf]
                    \&[0.5\ThreeCmPlusHalf]
                    {\Hom_{\CatFont{D}}(F(C),F(A))}
                    \&[0.30901699437\ThreeCmPlusHalf]
                    % 1-Arrows
                    % Left Boundary
                    \arrow[from=2-1,to=1-3,"{F_{B,C}\times F_{A,B}}"{pos=0.4125}]%
                    \arrow[from=1-3,to=2-5,isoarrowprime,"\sigma^{\Sets}_{\Hom_{\CatFont{D}}(F(C),F(B)),\Hom_{\CatFont{D}}(F(B),F(A))}"{pos=0.575}]%
                    \arrow[from=2-5,to=3-4,"\circ^{\CatFont{D}}_{F(C),F(B),F(A)}"{pos=0.425}]%
                    % Right Boundary
                    \arrow[from=2-1,to=3-2,"\circ^{\CatFont{C}}_{A,B,C}"'{pos=0.425}]%
                    \arrow[from=3-2,to=3-4,"F_{A,C}"']%
                \end{tikzcd}
            \end{scalemath}%
            commutes, i.e.\ for each composable pair $(g,f)$ of morphisms of $\CatFont{C}$, we have
            \[
                F(g\circ f)
                =
                F(f)\circ F(g).
            \]
    \end{enumerate}
\end{remark}
\begin{remark}{On the Term Contravariant Functor}{on-the-term-contravariant-functor}%
    Throughout this work we will not use the term \say{contravariant} functor, speaking instead simply of functors $F\colon\CatFont{C}^{\op}\to\CatFont{D}$. We will usually, however, write
    \[
        F_{A,B}%
        \colon%
        \Hom_{\CatFont{C}}(A,B)%
        \to%
        \Hom_{\CatFont{D}}(F(B),F(A))%
    \]%
    for the action on morphisms
    \[
        F_{A,B}%
        \colon%
        \Hom_{\CatFont{C}^{\op}}(A,B)%
        \to%
        \Hom_{\CatFont{D}}(F(A),F(B))%
    \]%
    of $F$, as well as write $F(g\circ f)=F(f)\circ F(g)$.
\end{remark}
\subsection{Forgetful Functors}\label{subsection-forgetful-functors}
\begin{definition}{Forgetful Functors}{forgetful-functors}%
    There isn't a precise definition of a \index[categories]{forgetful functor}\index[categories]{functor!forgetful}\textbf{forgetful functor}.
\end{definition}
\begin{remark}{Unwinding \cref{forgetful-functors}}{unwinding-forgetful-functors}%
    Despite there not being a formal or precise definition of a forgetful functor, the term is often very useful in practice, similarly to the word \say{canonical}. The idea is that a \say{forgetful functor} is a functor that forgets structure or properties, and is best explained through examples, such as the ones below (see \cref{forgetful-functors-that-forget-structure,forgetful-functors-that-forget-properties}).
\end{remark}
\begin{example}{Forgetful Functors That Forget Structure}{forgetful-functors-that-forget-structure}%
    Examples of forgetful functors that forget structure include:
    \begin{enumerate}
        \item\label{forgetful-functors-that-forget-structure-forgetting-group-structures}\SloganFont{Forgetting Group Structures. }The functor $\Grp\to\Sets$ sending a group $(G,\mu_{G},\eta_{G})$ to its underlying set $G$, forgetting the multiplication and unit maps $\mu_{G}$ and $\eta_{G}$ of $G$.
        \item\label{forgetful-functors-that-forget-structure-forgetting-topologies}\SloganFont{Forgetting Topologies. }The functor $\Top\to\Sets$ sending a topological space $(X,\mathcal{T}_{X})$ to its underlying set $X$, forgetting the topology $\mathcal{T}_{X}$.
        \item\label{forgetful-functors-that-forget-structure-forgetting-fibrations}\SloganFont{Forgetting Fibrations. }The functor $\FibSets(K)\to\Sets$ sending a $K$-fibred set $\phi_{X}\colon X\to K$ to the set $X$, forgetting the map $\phi_{X}$ and the base set $K$.
    \end{enumerate}
\end{example}
\begin{example}{Forgetful Functors That Forget Properties}{forgetful-functors-that-forget-properties}%
    Examples of forgetful functors that forget properties include:
    \begin{enumerate}
        \item\label{forgetful-functors-that-forget-properties-forgetting-commutativity}\SloganFont{Forgetting Commutativity. }The inclusion functor $\iota\colon\CMon\hookrightarrow\Mon$ which forgets the property of being commutative.
        \item\label{forgetful-functors-that-forget-properties-forgetting-inverses}\SloganFont{Forgetting Inverses. }The inclusion functor $\iota\colon\Grp\hookrightarrow\Mon$ which forgets the property of having inverses.
    \end{enumerate}
\end{example}
\begin{notation}{Notation For Forgetful Functors That Forget Structure}{notation-for-forgetful-functors-that-forget-structure}%
    Throughout this work, we will denote forgetful functors that forget structure by \index[notation]{Wasureru@$\Wasureru$}$\Wasureru$, e.g.\ as in
    \[
        \Wasureru%
        \colon%
        \Grp%
        \to%
        \Sets.%
    \]%
    The symbol $\Wasureru$, pronounced \emph{wasureru} (see \cref{pronunciation-of-the-words-in-notation-for-forgetful-functors-that-forget-structure-wasureru} of \cref{pronunciation-of-the-words-in-notation-for-forgetful-functors-that-forget-structure} below), means \emph{to forget}, and is a kanji found in the following words in Japanese and Chinese:
    \begin{enumerate}
        \item\label{notation-for-forgetful-functors-wasureru}\Japanese{忘れる}, transcribed as \emph{wasureru}, meaning \emph{to forget}.
        \item\label{notation-for-forgetful-functors-boukyaku-kanshu}\Japanese{忘却関手}, transcribed as \emph{boukyaku kanshu}, meaning \emph{forgetful functor}.
        \item\label{notation-for-forgetful-functors-wang-ji}\SimplifiedChinese{忘记} or \TraditionalChinese{忘記}, transcribed as \emph{wàngjì}, meaning \emph{to forget}.
        \item\label{notation-for-forgetful-functors-yiwang-hanzi}\SimplifiedChinese{遗忘函子} or \TraditionalChinese{遺忘函子}, transcribed as \emph{yíwàng hánzǐ}, meaning \emph{forgetful functor}.
    \end{enumerate}
\end{notation}
\begin{remark}{Pronunciation of the Words in \cref{notation-for-forgetful-functors-that-forget-structure}}{pronunciation-of-the-words-in-notation-for-forgetful-functors-that-forget-structure}%
    Here we collect the pronunciation of the words in \cref{notation-for-forgetful-functors-that-forget-structure} for accuracy and completeness.
    \begin{enumerate}
        \item\label{pronunciation-of-the-words-in-notation-for-forgetful-functors-that-forget-structure-wasureru}Pronunciation of \Japanese{忘れる}:
            \begin{itemize}
                \item \audio{wasureru-01.mp3}
                \item IPA broad transcription: [\IPA{wäsɯɾe̞ɾɯ}].
                \item IPA narrow transcription: [\IPA{ɰᵝäsɨᵝɾe̞ɾɯ̟ᵝ}].
            \end{itemize}
        \item\label{pronunciation-of-the-words-in-notation-for-forgetful-functors-that-forget-structure-boukyaku-kanshu}Pronunciation of \Japanese{忘却関手}:
            Pronunciation:
            \begin{itemize}
                \item \audio{wasureru-02.mp3}
                \item IPA broad transcription: [\IPA{bo̞ːkʲäkɯ kä̃ɰ̃ɕɯ}].
                \item IPA narrow transcription: [\IPA{bo̞ːkʲäkɯ̟ᵝ kä̃ɰ̃ɕɯᵝ}].
            \end{itemize}
        \item\label{pronunciation-of-the-words-in-notation-for-forgetful-functors-that-forget-structure-wang-ji}Pronunciation of \SimplifiedChinese{忘记}:
            \begin{itemize}
                \item \audio{wasureru-03.mp3}
                \item Broad IPA transcription: [\IPA{wɑŋtɕi}].
                \item Sinological IPA transcription: [\IPA{wɑŋ⁵¹⁻⁵³t͡ɕi⁵¹}].
            \end{itemize}
        \item\label{pronunciation-of-the-words-in-notation-for-forgetful-functors-that-forget-structure-yiwang-hanzi}Pronunciation of \SimplifiedChinese{遗忘函子}:
            \begin{itemize}
                \item \audio{wasureru-04.mp3}
                \item Broad IPA transcription: [\IPA{iwɑŋ xänt͡sz̩ɨ}].
                \item Sinological IPA transcription: [\IPA{i³⁵wɑŋ⁵¹ xän³⁵t͡sz̩²¹⁴⁻²¹⁽⁴⁾}].
            \end{itemize}
    \end{enumerate}
\end{remark}
\subsection{The Natural Transformation Associated to a Functor}\label{subsection-categories-the-natural-transformation-associated-to-a-functor}
\begin{definition}{The Natural Transformation Associated to a Functor}{the-natural-transformation-associated-to-a-functor}%
    Every functor $F\colon\CatFont{C}\to\CatFont{D}$ defines a natural transformation%
    %--- Begin Footnote ---%
    \footnote{%
        This is the $1$-categorical version of \ChapterRef{\ChapterConstructionsWithSets, \cref{constructions-with-sets:properties-of-characteristic-functions-the-inclusion-of-characteristic-relations-associated-to-a-function} of \cref{constructions-with-sets:properties-of-characteristic-functions}}{\cref{properties-of-characteristic-functions-the-inclusion-of-characteristic-relations-associated-to-a-function} of \cref{properties-of-characteristic-functions}}.
    }%
    %---  End Footnote  ---%
    \begin{webcompile}
        F^{\dagger}
        \colon
        \Hom_{\CatFont{C}}
        \Longrightarrow
        {\Hom_{\CatFont{D}}}\circ{(F^{\op}\times F)},
        \quad
        \begin{tikzcd}[row sep={5.0*\the\DL,between origins}, column sep={4.0*\the\DL,between origins}, background color=backgroundColor, ampersand replacement=\&]
            \CatFont{C}^{\op}\times\CatFont{C}
            \arrow[rr,"F^{\op}\times F"]
            \arrow[rd,"\Hom_{\CatFont{C}}"'{pos=0.475},""'{name=1}]
            \&
            \&
            \CatFont{D}^{\op}\times\CatFont{D}
            \arrow[ld,"\Hom_{\CatFont{D}}"{pos=0.475}]
            \\
            \&
            \Sets\mrp{,}
            \&
            % 2-Arrows
            \arrow[from=1,to=1-3,"F^{\dagger}"{description,pos=0.525},shorten <= 0.5em,shorten >= 0.0em,Rightarrow]%
        \end{tikzcd}
    \end{webcompile}%
    called the \index[categories]{natural transformation!associated to a functor}\textbf{natural transformation associated to $F$}, consisting of the collection
    \[
        \{
            F^{\dagger}_{A,B}
            \colon
            \Hom_{\CatFont{C}}(A,B)
            \to
            \Hom_{\CatFont{D}}(F_{A},F_{B})
        \}_{(A,B)\in\Obj(\CatFont{C}^{\op}\times\CatFont{C})}
    \]%
    with
    \[
        F^{\dagger}_{A,B}
        \defeq
        F_{A,B}.
    \]%
\end{definition}
\begin{Proof}{Proof of \cref{the-natural-transformation-associated-to-a-functor}}%
    The naturality condition for $F^{\dagger}$ is the requirement that for each morphism
    \[
        (\phi,\psi)
        \colon
        (X,Y)
        \to
        (A,B)
    \]%
    of $\CatFont{C}^{\op}\times\CatFont{C}$, the diagram
    \[
        \begin{tikzcd}[row sep={5.0*\the\DL,between origins}, column sep={18.5*\the\DL,between origins}, background color=backgroundColor, ampersand replacement=\&]
            \Hom_{\CatFont{C}}(X,Y)
            \arrow[r,"\phi^{*}\circ\psi_{*}=\psi_{*}\circ\phi^{*}"]
            \arrow[d,"F_{X,Y}"']
            \&
            \Hom_{\CatFont{C}}(A,B)
            \arrow[d,"F_{A,B}"]
            \\
            \Hom_{\CatFont{D}}(F_{X},F_{Y})
            \arrow[r,"{F(\phi)^{*}\circ F(\psi)_{*}=F(\psi)_{*}\circ F(\phi)^{*}}"']
            \&
            \Hom_{\CatFont{D}}(F_{A},F_{B})\mrp{,}
        \end{tikzcd}
    \]%
    acting on elements as
    \[
        \begin{tikzcd}[row sep={5.0*\the\DL,between origins}, column sep={12.0*\the\DL,between origins}, background color=backgroundColor, ampersand replacement=\&]
            f
            \arrow[r,mapsto]
            \arrow[d,mapsto]
            \&
            \psi\circ f\circ\phi
            \arrow[d,mapsto]
            \\
            {F(f)}
            \arrow[r,mapsto]
            \&
            {F(\psi)\circ F(f)\circ F(\psi)=F(\psi\circ f\circ\phi)}
        \end{tikzcd}
    \]%
    commutes, which follows from the functoriality of $F$.
\end{Proof}
\begin{proposition}{Properties of Natural Transformations Associated to Functors}{properties-of-natural-transformations-associated-to-functors}%
    Let $F\colon\CatFont{C}\to\CatFont{D}$ and $G\colon\CatFont{D}\to\CatFont{E}$ be functors.
    \begin{enumerate}
        \item\label{properties-of-natural-transformations-associated-to-functors-interaction-with-natural-isomorphisms}\SloganFont{Interaction With Natural Isomorphisms. }The following conditions are equivalent:
            \begin{enumerate}
                \item\label{properties-of-natural-transformations-associated-to-functors-interaction-with-natural-isomorphisms-a}The natural transformation $F^{\dagger}\colon\Hom_{\CatFont{C}}\Longrightarrow{\Hom_{\CatFont{D}}}\circ{(F^{\op}\times F)}$ associated to $F$ is a natural isomorphism.
                \item\label{properties-of-natural-transformations-associated-to-functors-interaction-with-natural-isomorphisms-b}The functor $F$ is fully faithful.
            \end{enumerate}
        \item\label{properties-of-natural-transformations-associated-to-functors-interaction-with-composition}\SloganFont{Interaction With Composition. }We have an equality of pasting diagrams
            \begin{scalemath}
                \begin{tikzcd}[row sep={6.0*\the\DL,between origins}, column sep={4.25*\the\DL,between origins}, background color=backgroundColor, ampersand replacement=\&]
                    \CatFont{C}^{\op}\times\CatFont{C}
                    \arrow[rr,"F^{\op}\times F"]
                    \arrow[rrd,"\Hom_{\CatFont{C}}"',""'{name=1}]
                    \&
                    \&
                    \CatFont{D}^{\op}\times\CatFont{D}
                    \arrow[rr,"G^{\op}\times G"]
                    \arrow[d,"\Hom_{\CatFont{D}}"description,""'{name=2}]
                    \&
                    \&
                    \CatFont{E}^{\op}\times\CatFont{E}
                    \arrow[lld,"\Hom_{\CatFont{E}}"]
                    \\
                    \&
                    \&
                    \Sets
                    \&
                    \&
                    % 2-Arrows
                    \arrow[from=1,to=1-3,"F^{\dagger}"{description,pos=0.6},shorten <= 0.5em, shorten >= -0.25em,Rightarrow]%
                    \arrow[from=2,to=1-5,"G^{\dagger}"{description,pos=0.6},shorten <= 1.5em,shorten >= +0.25em,Rightarrow]%
                \end{tikzcd}
                \bigequalssign
                \begin{tikzcd}[row sep={6.0*\the\DL,between origins}, column sep={5.5*\the\DL,between origins}, background color=backgroundColor, ampersand replacement=\&]
                    \CatFont{C}^{\op}\times\CatFont{C}
                    \arrow[rr,"{(G\circ F)^{\op}\times(G\circ F)}"]
                    \arrow[rd,"\Hom_{\CatFont{C}}"'{pos=0.475},""'{name=1}]
                    \&
                    \&
                    \CatFont{E}^{\op}\times\CatFont{E}\mrp{,}
                    \arrow[ld,"\Hom_{\CatFont{E}}"{pos=0.475}]
                    \\
                    \&
                    \Sets
                    \&
                    % 2-Arrows
                    \arrow[from=1,to=1-3,"{(G\circ F)^{\dagger}}"{description,pos=0.525},shorten <= 0.5em,shorten >= 0.0em,Rightarrow]%
                \end{tikzcd}
            \end{scalemath}%
            in $\TwoCategoryOfCategories$, i.e.\ we have
            \[
                (G\circ F)^{\dagger}%
                =%
                (G^{\dagger}\twocirc\id_{F^{\op}\times F})\circ F^{\dagger}.%
            \]%
        \item\label{properties-of-natural-transformations-associated-to-functors-interaction-with-identities}\SloganFont{Interaction With Identities. }We have
            \[
                \id^{\dagger}_{\CatFont{C}}%
                =%
                \id_{\Hom_{\CatFont{C}}(-_{1},-_{2})},%
            \]%
            i.e.\ the natural transformation associated to $\id_{\CatFont{C}}$ is the identity natural transformation of the functor $\Hom_{\CatFont{C}}(-_{1},-_{2})$.
        %\item\label{properties-of-natural-transformations-associated-to-functors-}\SloganFont{. }
    \end{enumerate}
\end{proposition}
\begin{Proof}{Proof of \cref{properties-of-natural-transformations-associated-to-functors}}%
    \FirstProofBox{\cref{properties-of-natural-transformations-associated-to-functors-interaction-with-natural-isomorphisms}: Interaction With Natural Isomorphisms}%
    Omitted.

    \ProofBox{\cref{properties-of-natural-transformations-associated-to-functors-interaction-with-composition}: Interaction With Composition}%
    Omitted.

    \ProofBox{\cref{properties-of-natural-transformations-associated-to-functors-interaction-with-identities}: Interaction With Identities}%
    Omitted.
\end{Proof}
\section{Conditions on Functors}\label{section-conditions-on-functors}
\subsection{Faithful Functors}\label{subsection-faithful-functors}
Let $\CatFont{C}$ and $\CatFont{D}$ be categories.
\begin{definition}{Faithful Functors}{faithful-functors}%
    A functor $F\colon\CatFont{C}\to\CatFont{D}$ is \index[categories]{functor!faithful}\textbf{faithful} if, for each $A,B\in\Obj(\CatFont{C})$, the action on morphisms
    \[
        F_{A,B}
        \colon
        \Hom_{\CatFont{C}}(A,B)
        \to
        \Hom_{\CatFont{D}}(F_{A},F_{B})
    \]%
    of $F$ at $(A,B)$ is injective.
\end{definition}
\begin{proposition}{Properties of Faithful Functors}{properties-of-faithful-functors}%
    Let $F\colon\CatFont{C}\to\CatFont{D}$ and $G\colon\CatFont{D}\to\CatFont{E}$ be functors.
    \begin{enumerate}
        \item\label{properties-of-faithful-functors-interaction-with-composition}\SloganFont{Interaction With Composition. }If $F$ and $G$ are faithful, then so is $G\circ F$.
        \item\label{properties-of-faithful-functors-interaction-with-postcomposition}\SloganFont{Interaction With Postcomposition. }The following conditions are equivalent:
            \begin{enumerate}
                \item\label{properties-of-faithful-functors-interaction-with-postcomposition-a}The functor $F\colon\CatFont{C}\to\CatFont{D}$ is faithful.
                \item\label{properties-of-faithful-functors-interaction-with-postcomposition-b}For each $\CatFont{X}\in\Obj(\Cats)$, the postcomposition functor
                    \[
                        F_{*}
                        \colon
                        \Fun(\CatFont{X},\CatFont{C})
                        \to
                        \Fun(\CatFont{X},\CatFont{D})
                    \]%
                    is faithful.
                \item\label{properties-of-faithful-functors-interaction-with-postcomposition-c}The functor $F\colon\CatFont{C}\to\CatFont{D}$ is a representably faithful morphism in $\TwoCategoryOfCategories$ in the sense of \ChapterRef{\ChapterTypesOfMorphismsInBicategories, \cref{types-of-morphisms-in-bicategories:representably-faithful-morphisms}}{\cref{representably-faithful-morphisms}}.
            \end{enumerate}
        \item\label{properties-of-faithful-functors-interaction-with-precomposition-1}\SloganFont{Interaction With Precomposition \rmI. }Let $F\colon\CatFont{C}\to\CatFont{D}$ be a functor.
            \begin{enumerate}
                \item\label{properties-of-faithful-functors-interaction-with-precomposition-1-a}If $F$ is faithful, then the precomposition functor
                    \[
                        F^{*}
                        \colon
                        \Fun(\CatFont{D},\CatFont{X})
                        \to
                        \Fun(\CatFont{C},\CatFont{X})
                    \]%
                    \demph{can fail} to be faithful.
                \item\label{properties-of-faithful-functors-interaction-with-precomposition-1-b}Conversely, if the precomposition functor
                    \[
                        F^{*}
                        \colon
                        \Fun(\CatFont{D},\CatFont{X})
                        \to
                        \Fun(\CatFont{C},\CatFont{X})
                    \]%
                    is faithful, then $F$ \demph{can fail} to be faithful.
            \end{enumerate}
        \item\label{properties-of-faithful-functors-interaction-with-precomposition-2}\SloganFont{Interaction With Precomposition \rmII. }If $F$ is essentially surjective, then the precomposition functor
            \[
                F^{*}
                \colon
                \Fun(\CatFont{D},\CatFont{X})
                \to
                \Fun(\CatFont{C},\CatFont{X})
            \]%
            is faithful.
        \item\label{properties-of-faithful-functors-interaction-with-precomposition-3}\SloganFont{Interaction With Precomposition \rmIII. }The following conditions are equivalent:
            \begin{enumerate}
                \item\label{properties-of-faithful-functors-interaction-with-precomposition-3-a}For each $\CatFont{X}\in\Obj(\Cats)$, the precomposition functor
                    \[
                        F^{*}
                        \colon
                        \Fun(\CatFont{D},\CatFont{X})
                        \to
                        \Fun(\CatFont{C},\CatFont{X})
                    \]%
                    is faithful.
                \item\label{properties-of-faithful-functors-interaction-with-precomposition-3-b}For each $\CatFont{X}\in\Obj(\Cats)$, the precomposition functor
                    \[
                        F^{*}
                        \colon
                        \Fun(\CatFont{D},\CatFont{X})
                        \to
                        \Fun(\CatFont{C},\CatFont{X})
                    \]%
                    is conservative.
                \item\label{properties-of-faithful-functors-interaction-with-precomposition-3-c}For each $\CatFont{X}\in\Obj(\Cats)$, the precomposition functor
                    \[
                        F^{*}
                        \colon
                        \Fun(\CatFont{D},\CatFont{X})
                        \to
                        \Fun(\CatFont{C},\CatFont{X})
                    \]%
                    is monadic.
                \item\label{properties-of-faithful-functors-interaction-with-precomposition-3-d}The functor $F\colon\CatFont{C}\to\CatFont{D}$ is a corepresentably faithful morphism in $\TwoCategoryOfCategories$ in the sense of \ChapterRef{\ChapterTypesOfMorphismsInBicategories, \cref{types-of-morphisms-in-bicategories:corepresentably-faithful-morphisms}}{\cref{corepresentably-faithful-morphisms}}.
                \item\label{properties-of-faithful-functors-interaction-with-precomposition-3-e}The components
                    \[
                        \eta_{G}%
                        \colon%
                        G%
                        \Longrightarrow%
                        \Ran_{F}(G\circ F)%
                    \]%
                    of the unit
                    \[
                        \eta%
                        \colon%
                        \id_{\Fun(\CatFont{D},\CatFont{X})}%
                        \Longrightarrow%
                        \Ran_{F}\circ F^{*}%
                    \]%
                    of the adjunction $F^{*}\dashv\Ran_{F}$ are all monomorphisms.
                \item\label{properties-of-faithful-functors-interaction-with-precomposition-3-f}The components
                    \[
                        \epsilon_{G}%
                        \colon%
                        \Lan_{F}(G\circ F)%
                        \Longrightarrow%
                        G%
                    \]%
                    of the counit
                    \[
                        \epsilon%
                        \colon%
                        \Lan_{F}\circ F^{*}%
                        \Longrightarrow%
                        \id_{\Fun(\CatFont{D},\CatFont{X})}%
                    \]%
                    of the adjunction $\Lan_{F}\dashv F^{*}$ are all epimorphisms.
                \item\label{properties-of-faithful-functors-interaction-with-precomposition-3-g}The functor $F$ is dominant (\cref{dominant-functors}), i.e.\ every object of $\CatFont{D}$ is a retract of some object in $\Im(F)$:
                    \begin{itemize}
                        \itemstar For each $B\in\Obj(\CatFont{D})$, there exist:
                            \begin{itemize}
                                \item An object $A$ of $\CatFont{C}$;
                                \item A morphism $s\colon B\to F(A)$ of $\CatFont{D}$;
                                \item A morphism $r\colon F(A)\to B$ of $\CatFont{D}$;
                            \end{itemize}
                            such that $r\circ s=\id_{B}$.
                    \end{itemize}
            \end{enumerate}
    \end{enumerate}
\end{proposition}
\begin{Proof}{Proof of \cref{properties-of-faithful-functors}}%
    \FirstProofBox{\cref{properties-of-faithful-functors-interaction-with-composition}: Interaction With Composition}%
    Since the map
    \[
        (G\circ F)_{A,B}
        \colon
        \Hom_{\CatFont{C}}(A,B)
        \to
        \Hom_{\CatFont{D}}(G_{F_{A}},G_{F_{B}}),%
    \]%
    defined as the composition
    \[
        \Hom_{\CatFont{C}}(A,B)%
        \xrightarrow{F_{A,B}}%
        \Hom_{\CatFont{D}}(F_{A},F_{B})%
        \xrightarrow{G_{F(A),F(B)}}%
        \Hom_{\CatFont{D}}(G_{F_{A}},G_{F_{B}}),%
    \]%
    is a composition of injective functions, it follows from \cref{TODO} that it is also injective. Therefore $G\circ F$ is faithful.

    \ProofBox{\cref{properties-of-faithful-functors-interaction-with-postcomposition}: Interaction With Postcomposition}%
    Omitted.

    \ProofBox{\cref{properties-of-faithful-functors-interaction-with-precomposition-1}: Interaction With Precomposition \rmI}%
    See \cite{MSE733163} for \cref{properties-of-faithful-functors-interaction-with-precomposition-1-a}. \cref{properties-of-faithful-functors-interaction-with-precomposition-1-b} follows from \cref{properties-of-faithful-functors-interaction-with-precomposition-2} and the fact that there are essentially surjective functors that are not faithful.

    \ProofBox{\cref{properties-of-faithful-functors-interaction-with-precomposition-2}: Interaction With Precomposition \rmII}%
    Omitted, but see \url{https://unimath.github.io/doc/UniMath/d4de26f//UniMath.CategoryTheory.precomp\_fully\_faithful.html} for a formalised proof.

    \ProofBox{\cref{properties-of-faithful-functors-interaction-with-precomposition-3}: Interaction With Precomposition \rmIII}%
    We claim \cref{properties-of-faithful-functors-interaction-with-precomposition-3-a,properties-of-faithful-functors-interaction-with-precomposition-3-b,properties-of-faithful-functors-interaction-with-precomposition-3-c,properties-of-faithful-functors-interaction-with-precomposition-3-d,properties-of-faithful-functors-interaction-with-precomposition-3-e,properties-of-faithful-functors-interaction-with-precomposition-3-f,properties-of-faithful-functors-interaction-with-precomposition-3-g} are equivalent:
    \begin{itemize}
        \item\SloganFont{\cref{properties-of-faithful-functors-interaction-with-precomposition-3-a,properties-of-faithful-functors-interaction-with-precomposition-3-d} Are Equivalent: }This is true by the definition of corepresentably faithful morphism; see \ChapterRef{\ChapterTypesOfMorphismsInBicategories, \cref{types-of-morphisms-in-bicategories:corepresentably-faithful-morphisms}}{\cref{corepresentably-faithful-morphisms}}.
        \item\SloganFont{\cref{properties-of-faithful-functors-interaction-with-precomposition-3-a,properties-of-faithful-functors-interaction-with-precomposition-3-b,properties-of-faithful-functors-interaction-with-precomposition-3-c,properties-of-faithful-functors-interaction-with-precomposition-3-g} Are Equivalent: }See \cite[Proposition 4.1]{on-functors-which-are-lax-epimorphisms} or alternatively \cite[Lemmas 3.1 and 3.2]{frey:on-the-2-categorical-duals-of-full-and-faithful-functors} for the equivalence between \cref{properties-of-faithful-functors-interaction-with-precomposition-3-a,properties-of-faithful-functors-interaction-with-precomposition-3-g}.
        \item\SloganFont{\cref{properties-of-faithful-functors-interaction-with-precomposition-3-a,properties-of-faithful-functors-interaction-with-precomposition-3-e,properties-of-faithful-functors-interaction-with-precomposition-3-f} Are Equivalent: }See \ChapterRef{\ChapterAdjunctionsAndTheYonedaLemma, \cref{adjunctions-and-the-yoneda-lemma:properties-of-adjunctions-interaction-with-faithfulness} of \cref{adjunctions-and-the-yoneda-lemma:properties-of-adjunctions}}{\cref{properties-of-adjunctions-interaction-with-faithfulness} of \cref{properties-of-adjunctions}}.
    \end{itemize}
    This finishes the proof.
\end{Proof}
\subsection{Full Functors}\label{subsection-full-functors}
Let $\CatFont{C}$ and $\CatFont{D}$ be categories.
\begin{definition}{Full Functors}{full-functors}%
    A functor $F\colon\CatFont{C}\to\CatFont{D}$ is \index[categories]{functor!full}\textbf{full} if, for each $A,B\in\Obj(\CatFont{C})$, the action on morphisms
    \[
        F_{A,B}
        \colon
        \Hom_{\CatFont{C}}(A,B)
        \to
        \Hom_{\CatFont{D}}(F_{A},F_{B})
    \]%
    of $F$ at $(A,B)$ is surjective.
\end{definition}
\begin{proposition}{Properties of Full Functors}{properties-of-full-functors}%
    Let $F\colon\CatFont{C}\to\CatFont{D}$ and $G\colon\CatFont{D}\to\CatFont{E}$ be functors.
    \begin{enumerate}
        \item\label{properties-of-full-functors-interaction-with-composition}\SloganFont{Interaction With Composition. }If $F$ and $G$ are full, then so is $G\circ F$.
        \item\label{properties-of-full-functors-interaction-with-postcomposition-1}\SloganFont{Interaction With Postcomposition \rmI. }If $F$ is full, then the postcomposition functor
            \[
                F_{*}
                \colon
                \Fun(\CatFont{X},\CatFont{C})
                \to
                \Fun(\CatFont{X},\CatFont{D})
            \]%
            \demph{can fail} to be full.
        \item\label{properties-of-full-functors-interaction-with-postcomposition-2}\SloganFont{Interaction With Postcomposition \rmII. }If, for each $\CatFont{X}\in\Obj(\Cats)$, the postcomposition functor
            \[
                F_{*}
                \colon
                \Fun(\CatFont{X},\CatFont{C})
                \to
                \Fun(\CatFont{X},\CatFont{D})
            \]%
            is full, then $F$ is also full.
        \item\label{properties-of-full-functors-interaction-with-precomposition-1}\SloganFont{Interaction With Precomposition \rmI. }If $F$ is full, then the precomposition functor
            \[
                F^{*}
                \colon
                \Fun(\CatFont{D},\CatFont{X})
                \to
                \Fun(\CatFont{C},\CatFont{X})
            \]%
            \demph{can fail} to be full.
        \item\label{properties-of-full-functors-interaction-with-precomposition-2}\SloganFont{Interaction With Precomposition \rmII. }If, for each $\CatFont{X}\in\Obj(\Cats)$, the precomposition functor
            \[
                F^{*}
                \colon
                \Fun(\CatFont{D},\CatFont{X})
                \to
                \Fun(\CatFont{C},\CatFont{X})
            \]%
            is full, then $F$ \demph{can fail} to be full.
        \item\label{properties-of-full-functors-interaction-with-precomposition-3}\SloganFont{Interaction With Precomposition \rmIII. }If $F$ is essentially surjective and full, then the precomposition functor
            \[
                F^{*}
                \colon
                \Fun(\CatFont{D},\CatFont{X})
                \to
                \Fun(\CatFont{C},\CatFont{X})
            \]%
            is full (and also faithful by \cref{properties-of-faithful-functors-interaction-with-precomposition-2} of \cref{properties-of-faithful-functors}).
        \item\label{properties-of-full-functors-interaction-with-precomposition-4}\SloganFont{Interaction With Precomposition \rmIV. }The following conditions are equivalent:
            \begin{enumerate}
                \item\label{properties-of-full-functors-interaction-with-precomposition-4-a}For each $\CatFont{X}\in\Obj(\Cats)$, the precomposition functor
                    \[
                        F^{*}
                        \colon
                        \Fun(\CatFont{D},\CatFont{X})
                        \to
                        \Fun(\CatFont{C},\CatFont{X})
                    \]%
                    is full.
                \item\label{properties-of-full-functors-interaction-with-precomposition-4-b}The functor $F\colon\CatFont{C}\to\CatFont{D}$ is a corepresentably full morphism in $\TwoCategoryOfCategories$ in the sense of \ChapterRef{\ChapterTypesOfMorphismsInBicategories, \cref{types-of-morphisms-in-bicategories:corepresentably-faithful-morphisms}}{\cref{corepresentably-faithful-morphisms}}.
                \item\label{properties-of-full-functors-interaction-with-precomposition-4-c}The components
                    \[
                        \eta_{G}%
                        \colon%
                        G%
                        \Longrightarrow%
                        \Ran_{F}(G\circ F)%
                    \]%
                    of the unit
                    \[
                        \eta%
                        \colon%
                        \id_{\Fun(\CatFont{D},\CatFont{X})}%
                        \Longrightarrow%
                        \Ran_{F}\circ F^{*}%
                    \]%
                    of the adjunction $F^{*}\dashv\Ran_{F}$ are all retractions/split epimorphisms.
                \item\label{properties-of-full-functors-interaction-with-precomposition-4-d}The components
                    \[
                        \epsilon_{G}%
                        \colon%
                        \Lan_{F}(G\circ F)%
                        \Longrightarrow%
                        G%
                    \]%
                    of the counit
                    \[
                        \epsilon%
                        \colon%
                        \Lan_{F}\circ F^{*}%
                        \Longrightarrow%
                        \id_{\Fun(\CatFont{D},\CatFont{X})}%
                    \]%
                    of the adjunction $\Lan_{F}\dashv F^{*}$ are all sections/split monomorphisms.
                \item\label{properties-of-full-functors-interaction-with-precomposition-4-e}For each $B\in\Obj(\CatFont{D})$, there exist:
                    \begin{itemize}
                        \item An object $A_{B}$ of $\CatFont{C}$;
                        \item A morphism $s_{B}\colon B\to F(A_{B})$ of $\CatFont{D}$;
                        \item A morphism $r_{B}\colon F(A_{B})\to B$ of $\CatFont{D}$;
                    \end{itemize}
                    satisfying the following condition:
                    \begin{itemize}
                        \itemstar For each $A\in\Obj(\CatFont{C})$ and each pair of morphisms
                            \begin{align*}
                                r &\colon F(A) \to B,\\
                                s &\colon B    \to F(A)
                            \end{align*}
                            of $\CatFont{D}$, we have
                            \[
                                [(A_{B},s_{B},r_{B})]%
                                =%
                                [(A,s,r\circ s_{B}\circ r_{B})]%
                            \]%
                            in $\int^{A\in\CatFont{C}}h^{B'}_{F_{A}}\times h^{F_{A}}_{B}$.
                    \end{itemize}
            \end{enumerate}
    \end{enumerate}
\end{proposition}
\begin{Proof}{Proof of \cref{properties-of-full-functors}}%
    \FirstProofBox{\cref{properties-of-full-functors-interaction-with-composition}: Interaction With Composition}%
    Since the map
    \[
        (G\circ F)_{A,B}
        \colon
        \Hom_{\CatFont{C}}(A,B)
        \to
        \Hom_{\CatFont{D}}(G_{F_{A}},G_{F_{B}}),%
    \]%
    defined as the composition
    \[
        \Hom_{\CatFont{C}}(A,B)%
        \xrightarrow{F_{A,B}}%
        \Hom_{\CatFont{D}}(F_{A},F_{B})%
        \xrightarrow{G_{F(A),F(B)}}%
        \Hom_{\CatFont{D}}(G_{F_{A}},G_{F_{B}}),%
    \]%
    is a composition of surjective functions, it follows from \cref{TODO} that it is also surjective. Therefore $G\circ F$ is full.

    \ProofBox{\cref{properties-of-full-functors-interaction-with-postcomposition-1}: Interaction With Postcomposition \rmI}%
    We follow the proof (completely formalised in cubical Agda!) given by Naïm Camille Favier in \cite{favier:postcompose-not-full}. Let $\CatFont{C}$ be the category where:
    \begin{itemize}
        \item\SloganFont{Objects. }We have $\Obj(\CatFont{C})=\{A,B\}$.
        \item\SloganFont{Morphisms. }We have
            \begin{align*}
                \Hom_{\CatFont{C}}(A,A) &= \{e_{A},\id_{A}\},\\
                \Hom_{\CatFont{C}}(B,B) &= \{e_{B},\id_{B}\},\\
                \Hom_{\CatFont{C}}(A,B) &= \{f,g\},\\
                \Hom_{\CatFont{C}}(B,A) &= \emptyset.
            \end{align*}
        \item\SloganFont{Composition. }The nontrivial compositions in $\CatFont{C}$ are the following:
            \[
                \begin{aligned}
                    e_{A}\circ e_{A} &= \id_{A},\\
                    e_{B}\circ e_{B} &= \id_{B},
                \end{aligned}
                \quad
                \begin{aligned}
                    f\circ e_{A}     &= g,\\
                    g\circ e_{A}     &= f,
                \end{aligned}
                \quad
                \begin{aligned}
                    e_{B}\circ f     &= f,\\
                    e_{B}\circ g     &= g.
                \end{aligned}
            \]%
    \end{itemize}
    We may picture $\CatFont{C}$ as follows:
    \[
        \begin{tikzcd}[row sep={4.0*\the\DL,between origins}, column sep={4.0*\the\DL,between origins}, background color=backgroundColor, ampersand replacement=\&]
            A
            \arrow[r,"f" ,shift left =0.8]
            \arrow[r,"g"',shift right=0.8]
            \arrow["e_{A}"',loop, distance=2em, in=215, out=145]
            \&
            B
            \arrow["e_{B}\mrp{.}", loop, distance=2em, in=325, out=35]
        \end{tikzcd}
    \]%
    Next, let $\CatFont{D}$ be the walking arrow category $\WalkingArrow$ of \cref{the-walking-arrow} and let $F\colon\CatFont{C}\to\WalkingArrow$ be the functor given on objects by
    \begin{align*}
        F(A) &= 0,\\
        F(B) &= 1
    \end{align*}
    and on non-identity morphisms by
    \[
        \begin{aligned}
            F(f)     &= f_{01},\\
            F(g)     &= f_{01},
        \end{aligned}
        \quad
        \begin{aligned}
            F(e_{A}) &= \id_{0},\\
            F(e_{B}) &= \id_{1}.
        \end{aligned}
    \]%
    Finally, let $\CatFont{X}=\B\Zn{2}$ be the walking involution and let $\iota_{A},\iota_{B}\colon\B\Zn{2}\rightrightarrows\CatFont{C}$ be the inclusion functors from $\B\Zn{2}$ to $\CatFont{C}$ with
    \begin{align*}
        \iota_{A}(\bullet) &= A,\\
        \iota_{B}(\bullet) &= B.
    \end{align*}
    Since every morphism in $\WalkingArrow$ has a preimage in $\CatFont{C}$ by $F$, the functor $F$ is full. Now, for $F_{*}$ to be full, the map
    \begin{webcompile}
        \phantom{F_{*|\iota_{A},\iota_{B}}\colon}
        \begin{tikzcd}[row sep=0.0*\the\DL, column sep=1.5*\the\DL, background color=backgroundColor, ampersand replacement=\&]
            \mathllap{F_{*|\iota_{A},\iota_{B}}\colon}\Nat(\iota_{A},\iota_{B})%
            \arrow[r]
            \&
            \Nat(F\circ\iota_{A},F\circ\iota_{B})%
            \\
            \alpha
            \arrow[r, mapsto]
            \&
            \id_{F}\twocirc\alpha
        \end{tikzcd}
        \mkern50mu
    \end{webcompile}%
    would need to be surjective. However, as we will show next, we have
    \begin{gather*}
        \Nat(\iota_{A},\iota_{B})             =     \emptyset,\\
        \Nat(F\circ\iota_{A},F\circ\iota_{B}) \cong \pt,
    \end{gather*}
    so this is impossible:
    \begin{itemize}
        \item\SloganFont{Proof of $\Nat(\iota_{A},\iota_{B})=\emptyset$: }A natural transformation $\alpha\colon\iota_{A}\Rightarrow\iota_{B}$ consists of a morphism
            \[
                \alpha%
                \colon%
                \underbrace{\iota_{A}(\bullet)}_{=A}%
                \to%
                \underbrace{\iota_{B}(\bullet)}_{=B}%
            \]%
            in $\CatFont{C}$ making the diagram
            \[
                \begin{tikzcd}[row sep={5.0*\the\DL,between origins}, column sep={6.0*\the\DL,between origins}, background color=backgroundColor, ampersand replacement=\&]
                    {\iota_{A}(\bullet)}
                    \arrow[r,"{\iota_{A}(e)}"]
                    \arrow[d,"\alpha"']
                    \&
                    {\iota_{A}(\bullet)}
                    \arrow[d,"\alpha"]
                    \\
                    {\iota_{B}(\bullet)}
                    \arrow[r,"{\iota_{B}(e)}"']
                    \&
                    {\iota_{B}(\bullet)}
                \end{tikzcd}
            \]%
            commute for each $e\in\Hom_{\B\Zn{2}}(\bullet,\bullet)\cong\Zn{2}$. We have two cases:
            \begin{enumerate}
                \item\label{proof-of-properties-of-full-functors-interaction-with-postcomposition-1-item-1}If $\alpha=f$, the naturality diagram for the unique nonidentity element of $\Zn{2}$ is given by
                    \[
                        \begin{tikzcd}[row sep={5.0*\the\DL,between origins}, column sep={5.0*\the\DL,between origins}, background color=backgroundColor, ampersand replacement=\&]
                            A
                            \arrow[r,"e_{A}"]
                            \arrow[d,"f"']
                            \&
                            A
                            \arrow[d,"f"]
                            \\
                            B
                            \arrow[r,"e_{B}"']
                            \&
                            B\mrp{.}
                        \end{tikzcd}
                    \]%
                    However, $e_{B}\circ f=f$ and $f\circ e_{A}=g$, so this diagram does not commute.
                \item\label{proof-of-properties-of-full-functors-interaction-with-postcomposition-1-item-2}If $\alpha=g$, the naturality diagram for the unique nonidentity element of $\Zn{2}$ is given by
                    \[
                        \begin{tikzcd}[row sep={5.0*\the\DL,between origins}, column sep={5.0*\the\DL,between origins}, background color=backgroundColor, ampersand replacement=\&]
                            A
                            \arrow[r,"e_{A}"]
                            \arrow[d,"g"']
                            \&
                            A
                            \arrow[d,"g"]
                            \\
                            B
                            \arrow[r,"e_{B}"']
                            \&
                            B\mrp{.}
                        \end{tikzcd}
                    \]%
                    However, $e_{B}\circ g=g$ and $g\circ e_{A}=f$, so this diagram does not commute.
            \end{enumerate}
            As a result, there are no natural transformations from $\iota_{A}$ to $\iota_{B}$.
        \item\SloganFont{Proof of $\Nat(F\circ\iota_{A},F\circ\iota_{B})\cong\pt$: }A natural transformation
            \[
                \beta%
                \colon%
                F\circ\iota_{A}%
                \Rightarrow%
                F\circ\iota_{B}%
            \]%
            consists of a morphism
            \[
                \beta%
                \colon%
                \underbrace{[F\circ\iota_{A}](\bullet)}_{=0}%
                \to%
                \underbrace{[F\circ\iota_{B}](\bullet)}_{=1}%
            \]%
            in $\WalkingArrow$ making the diagram
            \[
                \begin{tikzcd}[row sep={5.0*\the\DL,between origins}, column sep={10.0*\the\DL,between origins}, background color=backgroundColor, ampersand replacement=\&]
                    {[F\circ\iota_{A}](\bullet)}
                    \arrow[r,"{[F\circ\iota_{A}](e)}"]
                    \arrow[d,"\beta"']
                    \&
                    {[F\circ\iota_{A}](\bullet)}
                    \arrow[d,"\beta"]
                    \\
                    {[F\circ\iota_{B}](\bullet)}
                    \arrow[r,"{[F\circ\iota_{B}](e)}"']
                    \&
                    {[F\circ\iota_{B}](\bullet)}
                \end{tikzcd}
            \]%
            commute for each $e\in\Hom_{\B\Zn{2}}(\bullet,\bullet)\cong\Zn{2}$. Since the only morphism from $0$ to $1$ in $\WalkingArrow$ is $f_{01}$, we must have $\beta=f_{01}$ if such a transformation were to exist, and in fact it indeed does, as in this case the naturality diagram above becomes
            \[
                \begin{tikzcd}[row sep={5.0*\the\DL,between origins}, column sep={5.0*\the\DL,between origins}, background color=backgroundColor, ampersand replacement=\&]
                    0
                    \arrow[r,"\id_{0}"]
                    \arrow[d,"f_{01}"']
                    \&
                    0
                    \arrow[d,"f_{01}"]
                    \\
                    1
                    \arrow[r,"\id_{1}"']
                    \&
                    1
                \end{tikzcd}
            \]%
            for each $e\in\Zn{2}$, and this diagram indeed commutes, making $\beta$ into a natural transformation.
    \end{itemize}
    This finishes the proof.

    \ProofBox{\cref{properties-of-full-functors-interaction-with-postcomposition-2}: Interaction With Postcomposition \rmII}%
    Taking $\CatFont{X}=\PunctualCategory$, it follows by assumption that the functor
    \[
        F_{*}%
        \colon%
        \Fun(\PunctualCategory,\CatFont{C})%
        \to%
        \Fun(\PunctualCategory,\CatFont{D})%
    \]%
    is full. However, by \cref{properties-of-functor-categories-interaction-with-punctual-categories} of \cref{properties-of-functor-categories}, we have isomorphisms of categories
    \begin{align*}
        \Fun(\PunctualCategory,\CatFont{C}) &\cong \CatFont{C},\\
        \Fun(\PunctualCategory,\CatFont{D}) &\cong \CatFont{D}
    \end{align*}
    and the diagram
    \[
        \begin{tikzcd}[row sep={5.0*\the\DL,between origins}, column sep={8.0*\the\DL,between origins}, background color=backgroundColor, ampersand replacement=\&]
            {\Fun(\PunctualCategory,\CatFont{C})}%
            \arrow[r,"F_{*}"]
            \arrow[d,bigisoarrow]
            \&
            {\Fun(\PunctualCategory,\CatFont{D})}%
            \arrow[d,bigisoarrowprime]
            \\
            \CatFont{C}
            \arrow[r,"F"']
            \&
            \CatFont{D}
        \end{tikzcd}
    \]%
    commutes. It then follows from \cref{properties-of-full-functors-interaction-with-composition} that $F$ is full.

    \ProofBox{\cref{properties-of-full-functors-interaction-with-precomposition-1}: Interaction With Precomposition \rmI}%
    Omitted.

    \ProofBox{\cref{properties-of-full-functors-interaction-with-precomposition-2}: Interaction With Precomposition \rmII}%
    See \cite[p.~47]{lectures-on-n-categories-and-cohomology}.

    \ProofBox{\cref{properties-of-full-functors-interaction-with-precomposition-3}: Interaction With Precomposition \rmIII}%
    Omitted, but see \url{https://unimath.github.io/doc/UniMath/d4de26f//UniMath.CategoryTheory.precomp\_fully\_faithful.html} for a formalised proof.

    \ProofBox{\cref{properties-of-full-functors-interaction-with-precomposition-4}: Interaction With Precomposition \rmIV}%
    We claim \cref{properties-of-full-functors-interaction-with-precomposition-4-a,properties-of-full-functors-interaction-with-precomposition-4-b,properties-of-full-functors-interaction-with-precomposition-4-c,properties-of-full-functors-interaction-with-precomposition-4-d,properties-of-full-functors-interaction-with-precomposition-4-e} are equivalent:
    \begin{itemize}
        \item\SloganFont{\cref{properties-of-full-functors-interaction-with-precomposition-4-a,properties-of-full-functors-interaction-with-precomposition-4-b} Are Equivalent: }This is true by the definition of corepresentably full morphism; see \ChapterRef{\ChapterTypesOfMorphismsInBicategories, \cref{types-of-morphisms-in-bicategories:corepresentably-full-morphisms}}{\cref{corepresentably-full-morphisms}}.
        \item\SloganFont{\cref{properties-of-full-functors-interaction-with-precomposition-4-a,properties-of-full-functors-interaction-with-precomposition-4-c,properties-of-full-functors-interaction-with-precomposition-4-d} Are Equivalent: }See \ChapterRef{\ChapterAdjunctionsAndTheYonedaLemma, \cref{adjunctions-and-the-yoneda-lemma:properties-of-adjunctions-interaction-with-faithfulness} of \cref{adjunctions-and-the-yoneda-lemma:properties-of-adjunctions}}{\cref{properties-of-adjunctions-interaction-with-faithfulness} of \cref{properties-of-adjunctions}}.
        \item\SloganFont{\cref{properties-of-full-functors-interaction-with-precomposition-4-a,properties-of-full-functors-interaction-with-precomposition-4-e} Are Equivalent: }See \cite[Item (b) of Remark 4.3]{on-functors-which-are-lax-epimorphisms}.
    \end{itemize}
    This finishes the proof.
\end{Proof}
\begin{question}{Better Characterisations of Functors With Full Precomposition}{better-characterisations-of-functors-with-full-precomposition}%
    \cref{properties-of-full-functors-interaction-with-precomposition-4} of \cref{properties-of-full-functors} gives a characterisation of the functors $F$ for which $F^{*}$ is full, but the characterisations given there are really messy. Are there better ones?

    This question also appears as \cite{MO468121}.
\end{question}
\subsection{Fully Faithful Functors}\label{subsection-fully-faithful-functors}
Let $\CatFont{C}$ and $\CatFont{D}$ be categories.
\begin{definition}{Fully Faithful Functors}{fully-faithful-functors}%
    A functor $F\colon\CatFont{C}\to\CatFont{D}$ is \index[categories]{functor!fully faithful}\textbf{fully faithful} if $F$ is full and faithful, i.e.\ if, for each $A,B\in\Obj(\CatFont{C})$, the action on morphisms
    \[
        F_{A,B}
        \colon
        \Hom_{\CatFont{C}}(A,B)
        \to
        \Hom_{\CatFont{D}}(F_{A},F_{B})
    \]%
    of $F$ at $(A,B)$ is bijective.
\end{definition}
\begin{proposition}{Properties of Fully Faithful Functors}{properties-of-fully-faithful-functors}%
    Let $F\colon\CatFont{C}\to\CatFont{D}$ and $G\colon\CatFont{D}\to\CatFont{E}$ be functors.
    \begin{enumerate}
        \item\label{properties-of-fully-faithful-functors-characterisations}\SloganFont{Characterisations. }The following conditions are equivalent:
            \begin{enumerate}
                \item\label{properties-of-fully-faithful-functors-characterisations-a}The functor $F$ is fully faithful.
                \item\label{properties-of-fully-faithful-functors-characterisations-b}We have a pullback square
                    \begin{webcompile}
                        \Arr{\CatFont{C}}%
                        \cong%
                        (\CatFont{C}\times\CatFont{C})\times_{\CatFont{D}\times\CatFont{D}}\Arr{\CatFont{D}},%
                        \quad
                        \begin{tikzcd}[row sep={5.0*\the\DL,between origins}, column sep={7.25*\the\DL,between origins}, background color=backgroundColor, ampersand replacement=\&]
                            \Arr{\CatFont{C}}
                            \arrow[r,"\Arr{F}"]
                            \arrow[d,"\src\times\tgt"']
                            \arrow[rd,very near start,phantom,"\lrcorner"]
                            \&
                            \Arr{\CatFont{D}}
                            \arrow[d,"\src\times\tgt"]
                            \\
                            \CatFont{C}\times\CatFont{C}
                            \arrow[r,"F\times F"']
                            \&
                            \CatFont{D}\times\CatFont{D}
                        \end{tikzcd}
                    \end{webcompile}
                    in $\Cats$.
            \end{enumerate}
        \item\label{properties-of-fully-faithful-functors-interaction-with-composition}\SloganFont{Interaction With Composition. }If $F$ and $G$ are fully faithful, then so is $G\circ F$.
        \item\label{properties-of-fully-faithful-functors-conservativity}\SloganFont{Conservativity. }If $F$ is fully faithful, then $F$ is conservative.
        \item\label{properties-of-fully-faithful-functors-essential-injectivity}\SloganFont{Essential Injectivity. }If $F$ is fully faithful, then $F$ is essentially injective.
        \item\label{properties-of-fully-faithful-functors-interaction-with-co-limits}\SloganFont{Interaction With Co/Limits. }If $F$ is fully faithful, then $F$ reflects co/limits.
        \item\label{properties-of-fully-faithful-functors-interaction-with-postcomposition}\SloganFont{Interaction With Postcomposition. }The following conditions are equivalent:
            \begin{enumerate}
                \item\label{properties-of-fully-faithful-functors-interaction-with-postcomposition-a}The functor $F\colon\CatFont{C}\to\CatFont{D}$ is fully faithful.
                \item\label{properties-of-fully-faithful-functors-interaction-with-postcomposition-b}For each $\CatFont{X}\in\Obj(\Cats)$, the postcomposition functor
                    \[
                        F_{*}
                        \colon
                        \Fun(\CatFont{X},\CatFont{C})
                        \to
                        \Fun(\CatFont{X},\CatFont{D})
                    \]%
                    is fully faithful.
                \item\label{properties-of-fully-faithful-functors-interaction-with-postcomposition-c}The functor $F\colon\CatFont{C}\to\CatFont{D}$ is a representably fully faithful morphism in $\TwoCategoryOfCategories$ in the sense of \ChapterRef{\ChapterTypesOfMorphismsInBicategories, \cref{types-of-morphisms-in-bicategories:representably-fully-faithful-morphisms}}{\cref{representably-fully-faithful-morphisms}}.
            \end{enumerate}
        \item\label{properties-of-fully-faithful-functors-interaction-with-precomposition-1}\SloganFont{Interaction With Precomposition \rmI. }If $F$ is fully faithful, then the precomposition functor
            \[
                F^{*}
                \colon
                \Fun(\CatFont{D},\CatFont{X})
                \to
                \Fun(\CatFont{C},\CatFont{X})
            \]%
            \demph{can fail} to be fully faithful.
        \item\label{properties-of-fully-faithful-functors-interaction-with-precomposition-2}\SloganFont{Interaction With Precomposition \rmII. }If the precomposition functor
            \[
                F^{*}
                \colon
                \Fun(\CatFont{D},\CatFont{X})
                \to
                \Fun(\CatFont{C},\CatFont{X})
            \]%
            is fully faithful, then $F$ \demph{can fail} to be fully faithful (and in fact it can also fail to be either full or faithful).
        \item\label{properties-of-fully-faithful-functors-interaction-with-precomposition-3}\SloganFont{Interaction With Precomposition \rmIII. }If $F$ is essentially surjective and full, then the precomposition functor
            \[
                F^{*}
                \colon
                \Fun(\CatFont{D},\CatFont{X})
                \to
                \Fun(\CatFont{C},\CatFont{X})
            \]%
            is fully faithful.
        \item\label{properties-of-fully-faithful-functors-interaction-with-precomposition-4}\SloganFont{Interaction With Precomposition \rmIV. }The following conditions are equivalent:
            \begin{enumerate}
                \item\label{properties-of-fully-faithful-functors-interaction-with-precomposition-4-a}For each $\CatFont{X}\in\Obj(\Cats)$, the precomposition functor
                    \[
                        F^{*}
                        \colon
                        \Fun(\CatFont{D},\CatFont{X})
                        \to
                        \Fun(\CatFont{C},\CatFont{X})
                    \]%
                    is fully faithful.
                \item\label{properties-of-fully-faithful-functors-interaction-with-precomposition-4-b}The precomposition functor
                    \[
                        F^{*}
                        \colon
                        \Fun(\CatFont{D},\Sets)
                        \to
                        \Fun(\CatFont{C},\Sets)
                    \]%
                    is fully faithful.
                \item\label{properties-of-fully-faithful-functors-interaction-with-precomposition-4-c}The functor
                    \[
                        \Lan_{F}%
                        \colon
                        \Fun(\CatFont{C},\Sets)
                        \to
                        \Fun(\CatFont{D},\Sets)
                    \]%
                    is fully faithful.
                \item\label{properties-of-fully-faithful-functors-interaction-with-precomposition-4-d}The functor $F$ is a corepresentably fully faithful morphism in $\TwoCategoryOfCategories$ in the sense of \ChapterRef{\ChapterTypesOfMorphismsInBicategories, \cref{types-of-morphisms-in-bicategories:corepresentably-fully-faithful-morphisms}}{\cref{corepresentably-fully-faithful-morphisms}}.
                \item\label{properties-of-fully-faithful-functors-interaction-with-precomposition-4-e}The functor $F$ is absolutely dense.
                \item\label{properties-of-fully-faithful-functors-interaction-with-precomposition-4-f}The components
                    \[
                        \eta_{G}%
                        \colon%
                        G%
                        \Longrightarrow%
                        \Ran_{F}(G\circ F)%
                    \]%
                    of the unit
                    \[
                        \eta%
                        \colon%
                        \id_{\Fun(\CatFont{D},\CatFont{X})}%
                        \Longrightarrow%
                        \Ran_{F}\circ F^{*}%
                    \]%
                    of the adjunction $F^{*}\dashv\Ran_{F}$ are all isomorphisms.
                \item\label{properties-of-fully-faithful-functors-interaction-with-precomposition-4-g}The components
                    \[
                        \epsilon_{G}%
                        \colon%
                        \Lan_{F}(G\circ F)%
                        \Longrightarrow%
                        G%
                    \]%
                    of the counit
                    \[
                        \epsilon%
                        \colon%
                        \Lan_{F}\circ F^{*}%
                        \Longrightarrow%
                        \id_{\Fun(\CatFont{D},\CatFont{X})}%
                    \]%
                    of the adjunction $\Lan_{F}\dashv F^{*}$ are all isomorphisms.
                \item\label{properties-of-fully-faithful-functors-interaction-with-precomposition-4-h}The natural transformation
                    \[
                        \alpha%
                        \colon%
                        \Lan_{h_{F}}(h^{F})%
                        \Longrightarrow%
                        h%
                    \]%
                    with components
                    \[
                        \alpha_{B',B}%
                        \colon%
                        \int^{A\in\CatFont{C}}h^{B'}_{F_{A}}\times h^{F_{A}}_{B}%
                        \to%
                        h^{B'}_{B}
                    \]%
                    given by
                    \[
                        \alpha_{B',B}([(\phi,\psi)])%
                        =%
                        \psi\circ\phi%
                    \]%
                    is a natural isomorphism.
                \item\label{properties-of-fully-faithful-functors-interaction-with-precomposition-4-i}For each $B\in\Obj(\CatFont{D})$, there exist:
                    \begin{itemize}
                        \item An object $A_{B}$ of $\CatFont{C}$;
                        \item A morphism $s_{B}\colon B\to F(A_{B})$ of $\CatFont{D}$;
                        \item A morphism $r_{B}\colon F(A_{B})\to B$ of $\CatFont{D}$;
                    \end{itemize}
                    satisfying the following conditions:
                    \begin{enumerate}
                        \item\label{properties-of-fully-faithful-functors-interaction-with-precomposition-4-i-a}The triple $(F(A_{B}),r_{B},s_{B})$ is a retract of $B$, i.e.\ we have $r_{B}\circ s_{B}=\id_{B}$.
                        \item\label{properties-of-fully-faithful-functors-interaction-with-precomposition-4-i-b}For each morphism $f\colon B'\to B$ of $\CatFont{D}$, we have
                            \[
                                [(A_{B},s_{B'},f\circ r_{B'})]%
                                =%
                                [(A_{B},s_{B}\circ f,r_{B})]%
                            \]%
                            in $\int^{A\in\CatFont{C}}h^{B'}_{F_{A}}\times h^{F_{A}}_{B}$.
                    \end{enumerate}
            \end{enumerate}
    \end{enumerate}
\end{proposition}
\begin{Proof}{Proof of \cref{properties-of-fully-faithful-functors}}%
    \FirstProofBox{\cref{properties-of-fully-faithful-functors-characterisations}: Characterisations}%
    Omitted.

    \ProofBox{\cref{properties-of-fully-faithful-functors-interaction-with-composition}: Interaction With Composition}%
    Since the map
    \[
        (G\circ F)_{A,B}
        \colon
        \Hom_{\CatFont{C}}(A,B)
        \to
        \Hom_{\CatFont{D}}(G_{F_{A}},G_{F_{B}}),%
    \]%
    defined as the composition
    \[
        \Hom_{\CatFont{C}}(A,B)%
        \xrightarrow{F_{A,B}}%
        \Hom_{\CatFont{D}}(F_{A},F_{B})%
        \xrightarrow{G_{F(A),F(B)}}%
        \Hom_{\CatFont{D}}(G_{F_{A}},G_{F_{B}}),%
    \]%
    is a composition of bijective functions, it follows from \cref{TODO} that it is also bijective. Therefore $G\circ F$ is fully faithful.

    \ProofBox{\cref{properties-of-fully-faithful-functors-conservativity}: Conservativity}%
    This is a repetition of \cref{properties-of-conservative-functors-interaction-with-fully-faithfulness} of \cref{properties-of-conservative-functors}, and is proved there.

    \ProofBox{\cref{properties-of-fully-faithful-functors-essential-injectivity}: Essential Injectivity}%
    Omitted.

    \ProofBox{\cref{properties-of-fully-faithful-functors-interaction-with-co-limits}: Interaction With Co/Limits}%
    Omitted.

    \ProofBox{\cref{properties-of-fully-faithful-functors-interaction-with-postcomposition}: Interaction With Postcomposition}%
    This follows from \cref{properties-of-faithful-functors-interaction-with-postcomposition} of \cref{properties-of-faithful-functors} and \cref{properties-of-full-functors-interaction-with-postcomposition} of \cref{properties-of-full-functors}.

    \ProofBox{\cref{properties-of-fully-faithful-functors-interaction-with-precomposition-1}: Interaction With Precomposition \rmI}%
    See \cite{MSE733161} for an example of a fully faithful functor whose precomposition with which fails to be full.

    \ProofBox{\cref{properties-of-fully-faithful-functors-interaction-with-precomposition-2}: Interaction With Precomposition \rmII}%
    See \cite[Item 3]{MSE749304}.

    \ProofBox{\cref{properties-of-fully-faithful-functors-interaction-with-precomposition-3}: Interaction With Precomposition \rmIII}%
    Omitted, but see \url{https://unimath.github.io/doc/UniMath/d4de26f//UniMath.CategoryTheory.precomp\_fully\_faithful.html} for a formalised proof.

    \ProofBox{\cref{properties-of-fully-faithful-functors-interaction-with-precomposition-4}: Interaction With Precomposition \rmIV}%
    We claim \cref{properties-of-fully-faithful-functors-interaction-with-precomposition-4-a,properties-of-fully-faithful-functors-interaction-with-precomposition-4-b,properties-of-fully-faithful-functors-interaction-with-precomposition-4-c,properties-of-fully-faithful-functors-interaction-with-precomposition-4-d,properties-of-fully-faithful-functors-interaction-with-precomposition-4-e,properties-of-fully-faithful-functors-interaction-with-precomposition-4-f,properties-of-fully-faithful-functors-interaction-with-precomposition-4-g,properties-of-fully-faithful-functors-interaction-with-precomposition-4-h,properties-of-fully-faithful-functors-interaction-with-precomposition-4-i} are equivalent:
    \begin{itemize}
        \item\SloganFont{\cref{properties-of-fully-faithful-functors-interaction-with-precomposition-4-a,properties-of-fully-faithful-functors-interaction-with-precomposition-4-d} Are Equivalent: }This is true by the definition of corepresentably fully faithful morphism; see \ChapterRef{\ChapterTypesOfMorphismsInBicategories, \cref{types-of-morphisms-in-bicategories:corepresentably-fully-faithful-morphisms}}{\cref{corepresentably-fully-faithful-morphisms}}.
        \item\SloganFont{\cref{properties-of-fully-faithful-functors-interaction-with-precomposition-4-a,properties-of-fully-faithful-functors-interaction-with-precomposition-4-f,properties-of-fully-faithful-functors-interaction-with-precomposition-4-g} Are Equivalent: }See \ChapterRef{\ChapterAdjunctionsAndTheYonedaLemma, \cref{adjunctions-and-the-yoneda-lemma:properties-of-adjunctions-interaction-with-faithfulness} of \cref{adjunctions-and-the-yoneda-lemma:properties-of-adjunctions}}{\cref{properties-of-adjunctions-interaction-with-faithfulness} of \cref{properties-of-adjunctions}}.
        \item\SloganFont{\cref{properties-of-fully-faithful-functors-interaction-with-precomposition-4-a,properties-of-fully-faithful-functors-interaction-with-precomposition-4-b,properties-of-fully-faithful-functors-interaction-with-precomposition-4-c} Are Equivalent: }This follows from \cite[Proposition A.1.5]{notes-on-homotopical-algebra}.
        \item\SloganFont{\cref{properties-of-fully-faithful-functors-interaction-with-precomposition-4-a,properties-of-fully-faithful-functors-interaction-with-precomposition-4-e,properties-of-fully-faithful-functors-interaction-with-precomposition-4-h,properties-of-fully-faithful-functors-interaction-with-precomposition-4-i} Are Equivalent: }See \cite[Theorem 4.1]{frey:on-the-2-categorical-duals-of-full-and-faithful-functors} and \cite[Theorem 1.1]{on-functors-which-are-lax-epimorphisms}.
    \end{itemize}
    This finishes the proof.
\end{Proof}
\subsection{Conservative Functors}\label{subsection-conservative-functors}
Let $\CatFont{C}$ and $\CatFont{D}$ be categories.
\begin{definition}{Conservative Functors}{conservative-functors}%
    A functor $F\colon\CatFont{C}\to\CatFont{D}$ is \index[categories]{functor!conservative}\textbf{conservative} if it satisfies the following condition:%
    %--- Begin Footnote ---%
    \footnote{%
        \SloganFont{Slogan: }A functor $F$ is \textbf{conservative} if it reflects isomorphisms.
        \par\vspace*{\TCBBoxCorrection}
    }%
    %---  End Footnote  ---%
    \begin{itemize}
        \itemstar For each $f\in\Mor(\CatFont{C})$, if $F(f)$ is an isomorphism in $\CatFont{D}$, then $f$ is an isomorphism in $\CatFont{C}$.%
    \end{itemize}
\end{definition}
\begin{proposition}{Properties of Conservative Functors}{properties-of-conservative-functors}%
    Let $F\colon\CatFont{C}\to\CatFont{D}$ be a functor.
    \begin{enumerate}
        \item\label{properties-of-conservative-functors-characterisations}\SloganFont{Characterisations. }The following conditions are equivalent:
            \begin{enumerate}
                \item\label{properties-of-conservative-functors-characterisations-a}The functor $F$ is conservative.
                \item\label{properties-of-conservative-functors-characterisations-b}For each $f\in\Mor(\CatFont{C})$, the morphism $F(f)$ is an isomorphism in $\CatFont{D}$ \textiff $f$ is an isomorphism in $\CatFont{C}$.%
            \end{enumerate}
        \item\label{properties-of-conservative-functors-interaction-with-fully-faithfulness}\SloganFont{Interaction With Fully Faithfulness. }Every fully faithful functor is conservative.
        \item\label{properties-of-conservative-functors-interaction-with-precomposition}\SloganFont{Interaction With Precomposition. }The following conditions are equivalent:
            \begin{enumerate}
                \item\label{properties-of-conservative-functors-interaction-with-precomposition-3-a}For each $\CatFont{X}\in\Obj(\Cats)$, the precomposition functor
                    \[
                        F^{*}
                        \colon
                        \Fun(\CatFont{D},\CatFont{X})
                        \to
                        \Fun(\CatFont{C},\CatFont{X})
                    \]%
                    is conservative.
                \item\label{properties-of-conservative-functors-interaction-with-precomposition-3-b}The equivalent conditions of \cref{properties-of-faithful-functors-interaction-with-precomposition-3} of \cref{properties-of-faithful-functors} are satisfied.
            \end{enumerate}
        %\item\label{properties-of-conservative-functors-}\SloganFont{. }
    \end{enumerate}
\end{proposition}
\begin{Proof}{Proof of \cref{properties-of-conservative-functors}}%
    \FirstProofBox{\cref{properties-of-conservative-functors-characterisations}: Characterisations}%
    This follows from \cref{elementary-properties-of-functors-preservation-of-isomorphisms} of \cref{elementary-properties-of-functors}.

    \ProofBox{\cref{properties-of-conservative-functors-interaction-with-fully-faithfulness}: Interaction With Fully Faithfulness}%
    Let $F\colon\CatFont{C}\to\CatFont{D}$ be a fully faithful functor, let $f\colon A\to B$ be a morphism of $\CatFont{C}$, and suppose that $F_{f}$ is an isomorphism. We have
    \begin{align*}
        F(\id_{B}) &= \id_{F(B)}\\
                   &= F(f)\circ F(f)^{-1}\\
                   &= F(f\circ f^{-1}).
    \end{align*}
    Similarly, $F(\id_{A})=F(f^{-1}\circ f)$. But since $F$ is fully faithful, we must have
    \begin{align*}
        f\circ f^{-1} &= \id_{B},\\
        f^{-1}\circ f &= \id_{A},
    \end{align*}
    showing $f$ to be an isomorphism. Thus $F$ is conservative.
\end{Proof}
\begin{question}{Characterisations of Functors With Conservative Pre/Postcomposition}{characterisations-of-functors-with-conservative-pre-postcomposition}%
    Is there a characterisation of functors $F\colon\CatFont{C}\to\CatFont{D}$ satisfying the following condition:
    \begin{itemize}
        \itemstar For each $\CatFont{X}\in\Obj(\Cats)$, the postcomposition functor
            \[
                F_{*}%
                \colon%
                \Fun(\CatFont{X},\CatFont{C})%
                \to%
                \Fun(\CatFont{X},\CatFont{D})%
            \]%
            is conservative?
    \end{itemize}
    This question also appears as \cite{MO468125}.
\end{question}
\subsection{Essentially Injective Functors}\label{subsection-essentially-injective-functors}
Let $\CatFont{C}$ and $\CatFont{D}$ be categories.
\begin{definition}{Essentially Injective Functors}{essentially-injective-functors}%
    A functor $F\colon\CatFont{C}\to\CatFont{D}$ is \index[categories]{functor!essentially injective}\textbf{essentially injective} if it satisfies the following condition:
    \begin{itemize}
        \itemstar For each $A,B\in\Obj(\CatFont{C})$, if $F(A)\cong F(B)$, then $A\cong B$.
    \end{itemize}
\end{definition}
\begin{question}{Characterisations of Functors With Essentially Injective Pre/Postcomposition}{characterisations-of-functors-with-essentially-injective-pre-postcomposition}%
    Is there a characterisation of functors $F\colon\CatFont{C}\to\CatFont{D}$ such that:
    \begin{enumerate}
        \item\label{characterisations-of-functors-with-essentially-injective-pre-postcomposition-a}For each $\CatFont{X}\in\Obj(\Cats)$, the precomposition functor
            \[
                F^{*}%
                \colon%
                \Fun(\CatFont{D},\CatFont{X})%
                \to%
                \Fun(\CatFont{C},\CatFont{X})%
            \]%
            is essentially injective, i.e.\ if $\phi\circ F\cong\psi\circ F$, then $\phi\cong\psi$ for all functors $\phi$ and $\psi$?
        \item\label{characterisations-of-functors-with-essentially-injective-pre-postcomposition-b}For each $\CatFont{X}\in\Obj(\Cats)$, the postcomposition functor
            \[
                F_{*}%
                \colon%
                \Fun(\CatFont{X},\CatFont{C})%
                \to%
                \Fun(\CatFont{X},\CatFont{D})%
            \]%
            is essentially injective, i.e.\ if $F\circ\phi\cong F\circ\psi$, then $\phi\cong\psi$?
    \end{enumerate}
    This question also appears as \cite{MO468125}.
\end{question}
\subsection{Essentially Surjective Functors}\label{subsection-essentially-surjective-functors}
Let $\CatFont{C}$ and $\CatFont{D}$ be categories.
\begin{definition}{Essentially Surjective Functors}{essentially-surjective-functors}%
    A functor $F\colon\CatFont{C}\to\CatFont{D}$ is \index[categories]{functor!essentially surjective}\textbf{essentially surjective}%
    %--- Begin Footnote ---%
    \footnote{%
        \SloganFont{Further Terminology: }Also called an \index[categories]{functor!eso}\textbf{eso} functor, meaning \emph{essentially surjective on objects}.
        \par\vspace*{\TCBBoxCorrection}
    } %
    %---  End Footnote  ---%
    if it satisfies the following condition:
    \begin{itemize}
        \itemstar For each $D\in\Obj(\CatFont{D})$, there exists some object $A$ of $\CatFont{C}$ such that $F(A)\cong D$.
    \end{itemize}
\end{definition}
\begin{question}{Characterisations of Functors With Essentially Surjective Pre/Postcomposition}{characterisations-of-functors-with-essentially-surjective-pre-postcomposition}%
    Is there a characterisation of functors $F\colon\CatFont{C}\to\CatFont{D}$ such that:
    \begin{enumerate}
        \item\label{characterisations-of-functors-with-essentially-surjective-pre-postcomposition-a}For each $\CatFont{X}\in\Obj(\Cats)$, the precomposition functor
            \[
                F^{*}%
                \colon%
                \Fun(\CatFont{D},\CatFont{X})%
                \to%
                \Fun(\CatFont{C},\CatFont{X})%
            \]%
            is essentially surjective?
        \item\label{characterisations-of-functors-with-essentially-surjective-pre-postcomposition-b}For each $\CatFont{X}\in\Obj(\Cats)$, the postcomposition functor
            \[
                F_{*}%
                \colon%
                \Fun(\CatFont{X},\CatFont{C})%
                \to%
                \Fun(\CatFont{X},\CatFont{D})%
            \]%
            is essentially surjective?
    \end{enumerate}
    This question also appears as \cite{MO468125}.
\end{question}
\subsection{Equivalences of Categories}\label{subsection-equivalences-of-categories}
\begin{definition}{Equivalences of Categories}{equivalences-of-categories}%
    Let $\CatFont{C}$ and $\CatFont{D}$ be categories.
    \begin{enumerate}
        \item\label{equivalences-of-categories-equivalence-of-categories}An \index[categories]{category!equivalence of}\index[categories]{equivalence of categories}\textbf{equivalence of categories} between $\CatFont{C}$ and $\CatFont{D}$ consists of a pair of functors
            \begin{align*}
                F &\colon \CatFont{C}\to\CatFont{D},\\
                G &\colon \CatFont{D}\to\CatFont{C}
            \end{align*}
            together with natural isomorphisms%
            \begin{align*}
                \eta     &\colon \id_{\CatFont{C}} \Longrightisoarrow G\circ F,\\%
                \epsilon &\colon F\circ G          \Longrightisoarrow \id_{\CatFont{D}}.
            \end{align*}
        \item\label{equivalences-of-categories-adjoint-equivalence-of-categories}An \index[categories]{category!adjoint equivalence of}\index[categories]{adjoint equivalence of categories}\textbf{adjoint equivalence of categories} between $\CatFont{C}$ and $\CatFont{D}$ is an equivalence $(F,G,\eta,\epsilon)$ between $\CatFont{C}$ and $\CatFont{D}$ which is also an adjunction.
    \end{enumerate}
\end{definition}
\begin{proposition}{Properties of Equivalences of Categories}{properties-of-equivalences-of-categories}%
    Let $F\colon\CatFont{C}\to\CatFont{D}$ be a functor.
    \begin{enumerate}
        \item\label{properties-of-equivalences-of-categories-characterisations}\SloganFont{Characterisations. }If $\CatFont{C}$ and $\CatFont{D}$ are small%
            %--- Begin Footnote ---%
            \footnote{%
                Otherwise there will be size issues. One can also work with large categories and universes, or require $F$ to be \emph{constructively} essentially surjective; see \cite{MSE1465107}.
            }, %
            %---  End Footnote  ---%
            then the following conditions are equivalent:%
            %--- Begin Footnote ---%
            \footnote{%
                In ZFC, the equivalence between \cref{properties-of-equivalences-of-categories-characterisations-a} and \cref{properties-of-equivalences-of-categories-characterisations-b} is equivalent to the axiom of choice; see \cite{MO119454}.%
                In ZF, the equivalence between \cref{properties-of-equivalences-of-categories-characterisations-a} and \cref{properties-of-equivalences-of-categories-characterisations-b} is equivalent to the axiom of choice; see \cite{MO119454}.%

                In Univalent Foundations, this is true without requiring neither the axiom of choice nor the law of excluded middle.
            }%
            %---  End Footnote  ---%
            \begin{enumerate}
                \item\label{properties-of-equivalences-of-categories-characterisations-a}The functor $F$ is an equivalence of categories.
                \item\label{properties-of-equivalences-of-categories-characterisations-b}The functor $F$ is fully faithful and essentially surjective.
                \item\label{properties-of-equivalences-of-categories-characterisations-c}The induced functor
                    \[
                        \restriction{F}{\Sk(\CatFont{C})}%
                        \colon%
                        \Sk(\CatFont{C})%
                        \to%
                        \Sk(\CatFont{D})%
                    \]%
                    is an \emph{isomorphism} of categories.
                \item\label{properties-of-equivalences-of-categories-characterisations-d}For each $X\in\Obj(\Cats)$, the precomposition functor
                    \[
                        F^{*}%
                        \colon%
                        \Fun(\CatFont{D},\CatFont{X})%
                        \to%
                        \Fun(\CatFont{C},\CatFont{X})%
                    \]%
                    is an equivalence of categories.
                \item\label{properties-of-equivalences-of-categories-characterisations-e}For each $X\in\Obj(\Cats)$, the postcomposition functor
                    \[
                        F_{*}%
                        \colon%
                        \Fun(\CatFont{X},\CatFont{C})%
                        \to%
                        \Fun(\CatFont{X},\CatFont{D})%
                    \]%
                    is an equivalence of categories.
            \end{enumerate}
      \item\label{properties-of-equivalences-of-categories-inverses}\SloganFont{Inverses. }If $(F,G,\eta,\epsilon)$ is an equivalence with $F \colon \CatFont C \to CatFont D$, so is
            $(G,F,\epsilon^{-1},\eta^{-1})$.
      \item\label{properties-of-equivalences-of-categories-two-out-of-three}\SloganFont{Two-Out-of-Three. }Let
            \[
                \begin{tikzcd}[row sep={3.0*\the\DL,between origins}, column sep={2.0*\the\DL,between origins}, background color=backgroundColor, ampersand replacement=\&]
                    \CatFont{C}
                    \arrow[rd, "F"']
                    \arrow[rr, "G\circ F"]
                    \&
                    \&
                    \CatFont{E}
                    \\
                    \&
                    \CatFont{D}
                    \arrow[ru, "G"']
                    \&
                \end{tikzcd}
            \]%
            be a diagram in $\Cats$. If two out of the three functors among $F$, $G$, and $G\circ F$ are equivalences of categories, then so is the third.
        \item\label{properties-of-equivalences-of-categories-stability-under-composition}\SloganFont{Stability Under Composition. }Let
            \[
                \begin{tikzcd}[row sep={4.0*\the\DL,between origins}, column sep={4.0*\the\DL,between origins}, background color=backgroundColor, ampersand replacement=\&]
                    \CatFont{C}
                    \arrow[r, "F",  shift left=0.8]
                    \&
                    \arrow[l, "G",  shift left=0.8]
                    \CatFont{D}
                    \arrow[r, "F'", shift left=0.8]
                    \&
                    \arrow[l, "G'", shift left=0.8]
                    \CatFont{E}
                \end{tikzcd}
            \]%
            be a diagram in $\Cats$. If $(F,G)$ and $(F',G')$ are equivalences of categories, then so is their composite $(F'\circ F,G'\circ G)$.
        \item\label{properties-of-equivalences-of-categories-equivalences-vs-adjoint-equivalences}\SloganFont{Equivalences \vs Adjoint Equivalences. }Every equivalence of categories can be promoted to an adjoint equivalence.%
            %--- Begin Footnote ---%
            \footnote{%
                More precisely, we can promote an equivalence of categories $(F,G,\eta,\epsilon)$ to adjoint equivalences $(F,G,\eta',\epsilon)$ and $(F,G,\eta,\epsilon')$.
                \par\vspace*{\TCBBoxCorrection}
            }%
            %---  End Footnote  ---%
        \item\label{properties-of-equivalences-of-categories-interaction-with-groupoids}\SloganFont{Interaction With Groupoids. }If $\CatFont{C}$ and $\CatFont{D}$ are groupoids, then the following conditions are equivalent:
            \begin{enumerate}
                \item\label{properties-of-equivalences-of-categories-interaction-with-groupoids-a}The functor $F$ is an equivalence of groupoids.
                \item\label{properties-of-equivalences-of-categories-interaction-with-groupoids-b}The following conditions are satisfied:
                    \begin{enumerate}
                        \item\label{properties-of-equivalences-of-categories-interaction-with-groupoids-b-a}The functor $F$ induces a bijection
                            \[
                                \pi_{0}(F)%
                                \colon%
                                \pi_{0}(\CatFont{C})%
                                \to%
                                \pi_{0}(\CatFont{D})%
                            \]%
                            of sets.
                        \item\label{properties-of-equivalences-of-categories-interaction-with-groupoids-b-b}For each $A\in\Obj(\CatFont{C})$, the induced map
                            \[
                                F_{x,x}%
                                \colon%
                                \Aut_{\CatFont{C}}(A)%
                                \to%
                                \Aut_{\CatFont{D}}(F_{A})%
                            \]%
                            is an isomorphism of groups.
                    \end{enumerate}
            \end{enumerate}
    \end{enumerate}
\end{proposition}
\begin{Proof}{Proof of \cref{properties-of-equivalences-of-categories}}%
    \FirstProofBox{\cref{properties-of-equivalences-of-categories-characterisations}: Characterisations}%
    We claim that \cref{properties-of-equivalences-of-categories-characterisations-a,properties-of-equivalences-of-categories-characterisations-b,properties-of-equivalences-of-categories-characterisations-c,properties-of-equivalences-of-categories-characterisations-d,properties-of-equivalences-of-categories-characterisations-e} are indeed equivalent:
    \begin{enumerate}
      \item\SloganFont{\cref{properties-of-equivalences-of-categories-characterisations-a}$\implies$\cref{properties-of-equivalences-of-categories-characterisations-b}: }
            Let $F \colon \CatFont C \to \CatFont D$ be an equivalence. We show the three properties.
            \begin{enumerate}
              \item\SloganFont{Faithfulness: } Let $f,g \in \Hom_{\CatFont C}(A,B)$, $F(f) = F(g)$. Then $GF(f) = GF(g)$ and we have the following diagramm:
                \[
                    \begin{tikzcd}[row sep={5.0*\the\DL,between origins}, column sep={6.0*\the\DL,between origins}, background color=backgroundColor, ampersand replacement=\&]
                      GF(A)
                      \arrow[r, "{GF(f)}", shift left]
                      \arrow[r, "{GF(g)}"', shift right]
                      \arrow[d, "\eta^{-1}_{A}"']
                      \&
                      GF(B)
                      \arrow[d, "\eta^{-1}_{B}"]
                      \\
                      A
                      \arrow[r, "{f}", shift left]
                      \arrow[r, "{g}"', shift right]
                      \&
                      B
                    \end{tikzcd}
                    \]%
                    The squares where we remove $GF(g), g$ or $GF(f), f$ from the diagrams commute by naturality of $\eta^{-1}$ and the two upper morphisms are equal.
                    So we have
                    \[ f \circ \eta^{-1}_A = \eta^{-1}_B \circ GF(f) = \eta^{-1}_B \circ GF(g) = g \circ \eta^{-1}_A. \]
                    But $\eta^{-1}_A$ is an isomorphism, so $f = g$ and $F$ is faithful.
              \item\SloganFont{Fullness: } Let $f \in \Hom_{\CatFont D}(F(A), F(B))$.
                    Define \[h \in \Hom_{\CatFont C}(A,B), h = \eta^{-1}_B \circ G(f) \circ \eta_A,\]
                    yielding the commutative diagram
                    \[
                    \begin{tikzcd}[row sep={5.0*\the\DL,between origins}, column sep={6.0*\the\DL,between origins}, background color=backgroundColor, ampersand replacement=\&]
                      A
                      \arrow[r, "{h}"]
                      \arrow[d, "\eta_{A}"']
                      \&
                      B
                      \arrow[d, "\eta_{B}"]
                      \\
                      GF(A)
                      \arrow[r, "{G(f)}"']
                      \&
                      GF(B)
                    \end{tikzcd}
                    \]%
                    We must show $F(h) = f$.
                    Since $(G,F,\epsilon^{-1},\eta^{-1})$ is also an equivalence due to \cref{properties-of-equivalences-of-categories-inverses}, $G$ is faithful, so it suffices to show $GF(h) = G(f)$.

                    Note how from the naturality of $\eta$ applied to itself we get
                    \[ GF(\eta_A) \circ \eta_A = \eta_{GFA} \circ \eta_A, \]
                    for all $A \in \Obj(\CatFont C)$, and since $\eta$ is an isomorphism, this implies $\eta_{GFA} = GF(\eta_A)$.

                    By applying $GF$ to the above diagram and incorporating the equality, we can see that
                     \[
                    \begin{tikzcd}[row sep={5.0*\the\DL,between origins}, column sep={6.0*\the\DL,between origins}, background color=backgroundColor, ampersand replacement=\&]
                      GF(A)
                      \arrow[r, "{GF(h)}"]
                      \arrow[d, "\eta_{GF(A)} = GF(\eta_A)"']
                      \&
                      GF(B)
                      \arrow[d, "\eta_{GF(B)} = GF(\eta_B)"]
                      \\
                      GFGF(A)
                      \arrow[r, "{GFG(f)}"']
                      \&
                      GFGF(B)
                    \end{tikzcd}
                    \]%
                    must commute.
                    By naturality of $\eta$, so does
                      \[
                    \begin{tikzcd}[row sep={5.0*\the\DL,between origins}, column sep={6.0*\the\DL,between origins}, background color=backgroundColor, ampersand replacement=\&]
                      GF(A)
                      \arrow[r, "{G(f)}"]
                      \arrow[d, "\eta_{GF(A)}"']
                      \&
                      GF(B)
                      \arrow[d, "\eta_{GF(B)}"]
                      \\
                      GFGF(A)
                      \arrow[r, "{GFG(f)}"']
                      \&
                      GFGF(B)
                    \end{tikzcd}
                    \]%
                    and we have
                    \[ \eta_{GF(A)} \circ GF(h) = GFG(f) \circ \eta_{GF(B)} = \eta_{GF(A)} \circ G(f). \]
                    But $\eta$ is an isomorphism, so $GF(h) = G(f)$.
              \item\SloganFont{Essential surjectivity: }
                    Let $B \in \Obj(\CatFont D)$. We need to find some $A \in \Obj(\CatFont C)$ with $F(A) \cong B$. Choose $A \defeq G(B)$. Then $\epsilon_B \colon FG(B) \to B$ is an isomorphism by assumption. Hence $F$ is essentially surjective.
            \end{enumerate}
        \item\SloganFont{\cref{properties-of-equivalences-of-categories-characterisations-b}$\implies$\cref{properties-of-equivalences-of-categories-characterisations-a}: }Since $F$ is essentially surjective and $\CatFont{C}$ and $\CatFont{D}$ are small, we can choose, using the axiom of choice, for each $B\in\Obj(\CatFont{D})$, an object $j_{B}$ of $\CatFont{C}$ and an isomorphism $i_{B}\colon B\to F_{j_{B}}$ of $\CatFont{D}$.

            Since $F$ is fully faithful, we can extend the assignment $B\mapsto j_{B}$ to a \emph{unique} functor $j\colon\CatFont{D}\to\CatFont{C}$ such that the isomorphisms $i_{B}\colon B\to F_{j_{B}}$ assemble into a natural isomorphism $\eta\colon\id_{\CatFont{D}}\Longrightisoarrow F\circ j$, with a similar natural isomorphism $\epsilon\colon\id_{\CatFont{C}}\Longrightisoarrow j\circ F$. Hence $F$ is an equivalence.
        \item\SloganFont{\cref{properties-of-equivalences-of-categories-characterisations-a}$\implies$\cref{properties-of-equivalences-of-categories-characterisations-c}: }This follows from \cref{properties-of-skeletons-of-categories-inclusions-of-skeletons-are-equivalences} of \cref{properties-of-skeletons-of-categories}.
        \item\SloganFont{\cref{properties-of-equivalences-of-categories-characterisations-c}$\implies$\cref{properties-of-equivalences-of-categories-characterisations-a}: }Omitted.
        \item\SloganFont{\cref{properties-of-equivalences-of-categories-characterisations-a,properties-of-equivalences-of-categories-characterisations-d,properties-of-equivalences-of-categories-characterisations-e} Are Equivalent: }This follows from \cref{TODO}.%
    \end{enumerate}
    This finishes the proof of \cref{properties-of-equivalences-of-categories-characterisations}.
    \ProofBox{\cref{properties-of-equivalences-of-categories-inverses}: Inverses}%
    Follows since
    \begin{align*}
                \epsilon^{-1}     &\colon \id_{\CatFont{D}} \Longrightisoarrow F \circ G,\\%
                \eta^{-1} &\colon G\circ F \Longrightisoarrow \id_{\CatFont{C}}
    \end{align*}
    are natural isomorphisms if $\epsilon$ and $\eta$ are.
    \ProofBox{\cref{properties-of-equivalences-of-categories-two-out-of-three}: Two-Out-of-Three}%
    Omitted.

    \ProofBox{\cref{properties-of-equivalences-of-categories-stability-under-composition}: Stability Under Composition}%
    dfjhdsf

    \ProofBox{\cref{properties-of-equivalences-of-categories-equivalences-vs-adjoint-equivalences}: Equivalences \vs Adjoint Equivalences}%
    See \cite[Proposition 4.4.5]{category-theory-in-context}.

    \ProofBox{\cref{properties-of-equivalences-of-categories-interaction-with-groupoids}: Interaction With Groupoids}%
    See \cite[Proposition 4.4]{nlab:groupoid}.
\end{Proof}
\subsection{Isomorphisms of Categories}\label{subsection-isomorphisms-of-categories}
\begin{definition}{Isomorphisms of Categories}{isomorphisms-of-categories}%
    An \index[categories]{category!isomorphism of}\index[categories]{isomorphism of categories}\textbf{isomorphism of categories} is a pair of functors
    \begin{align*}
        F &\colon \CatFont{C}\to\CatFont{D},\\
        G &\colon \CatFont{D}\to\CatFont{C}
    \end{align*}
    such that we have
    \begin{align*}
        G\circ F &= \id_{\CatFont{C}},\\\\
        F\circ G &= \id_{\CatFont{D}}.
    \end{align*}
\end{definition}
\begin{example}{Equivalent But Non-Isomorphic Categories}{equivalent-but-non-isomorphic-categories}%
    Categories can be equivalent but non-isomorphic. For example, the category consisting of two isomorphic objects is equivalent to $\PunctualCategory$, but not isomorphic to it.
\end{example}
\begin{proposition}{Properties of Isomorphisms of Categories}{properties-of-isomorphisms-of-categories}%
    Let $F\colon\CatFont{C}\to\CatFont{D}$ be a functor.
    \begin{enumerate}
        \item\label{properties-of-isomorphisms-of-categories-characterisations}\SloganFont{Characterisations. }If $\CatFont{C}$ and $\CatFont{D}$ are small, then the following conditions are equivalent:
            \begin{enumerate}
                \item\label{properties-of-isomorphisms-of-categories-characterisations-a}The functor $F$ is an isomorphism of categories.
                \item\label{properties-of-isomorphisms-of-categories-characterisations-b}The functor $F$ is fully faithful and bijective on objects.
                \item\label{properties-of-isomorphisms-of-categories-characterisations-c}For each $X\in\Obj(\Cats)$, the precomposition functor
                    \[
                        F^{*}%
                        \colon%
                        \Fun(\CatFont{D},\CatFont{X})%
                        \to%
                        \Fun(\CatFont{C},\CatFont{X})%
                    \]%
                    is an isomorphism of categories.
                \item\label{properties-of-isomorphisms-of-categories-characterisations-d}For each $X\in\Obj(\Cats)$, the postcomposition functor
                    \[
                        F_{*}%
                        \colon%
                        \Fun(\CatFont{X},\CatFont{C})%
                        \to%
                        \Fun(\CatFont{X},\CatFont{D})%
                    \]%
                    is an isomorphism of categories.
            \end{enumerate}
        %\item\label{properties-of-isomorphisms-of-categories-}\SloganFont{. }
    \end{enumerate}
\end{proposition}
\begin{Proof}{Proof of \cref{properties-of-isomorphisms-of-categories}}%
    \FirstProofBox{\cref{properties-of-isomorphisms-of-categories-characterisations}: Characterisations}%
    We claim that \cref{properties-of-isomorphisms-of-categories-characterisations-a,properties-of-isomorphisms-of-categories-characterisations-b,properties-of-isomorphisms-of-categories-characterisations-c,properties-of-isomorphisms-of-categories-characterisations-d} are indeed equivalent:
    \begin{enumerate}
        \item\SloganFont{\cref{properties-of-isomorphisms-of-categories-characterisations-a,properties-of-isomorphisms-of-categories-characterisations-b} Are Equivalent: }Omitted, but similar to \cref{properties-of-equivalences-of-categories-characterisations} of \cref{properties-of-equivalences-of-categories}.
        \item\SloganFont{\cref{properties-of-isomorphisms-of-categories-characterisations-a,properties-of-isomorphisms-of-categories-characterisations-c,properties-of-isomorphisms-of-categories-characterisations-d} Are Equivalent: }This follows from \cref{TODO}.%
    \end{enumerate}
    This finishes the proof.
\end{Proof}
\section{More Conditions on Functors}\label{section-more-conditions-on-functors}
\subsection{Dominant Functors}\label{subsection-dominant-functors}
Let $\CatFont{C}$ and $\CatFont{D}$ be categories.
\begin{definition}{Dominant Functors}{dominant-functors}%
    A functor $F\colon\CatFont{C}\to\CatFont{D}$ is \index[categories]{functor!dominant}\textbf{dominant} if every object of $\CatFont{D}$ is a retract of some object in $\Im(F)$, i.e.:
    \begin{itemize}
        \itemstar For each $B\in\Obj(\CatFont{D})$, there exist:
            \begin{itemize}
                \item An object $A$ of $\CatFont{C}$;
                \item A morphism $r\colon F(A)\to B$ of $\CatFont{D}$;
                \item A morphism $s\colon B\to F(A)$ of $\CatFont{D}$;
            \end{itemize}
            such that we have
            \begin{webcompile}
                r\circ s%
                =%
                \id_{B},%
                \quad
                \begin{tikzcd}[row sep={5.0*\the\DL,between origins}, column sep={5.0*\the\DL,between origins}, background color=backgroundColor, ampersand replacement=\&]
                    B
                    \arrow[r,"s"]
                    \arrow[rd,"\id_{B}"']
                    \&
                    {F(A)}
                    \arrow[d,"r"]
                    \\
                    \&
                    B\mrp{.}
                \end{tikzcd}
            \end{webcompile}
    \end{itemize}
\end{definition}
\begin{proposition}{Properties of Dominant Functors}{properties-of-dominant-functors}%
    Let $F,G\colon\CatFont{C}\rightrightarrows\CatFont{D}$ be functors and let $I\colon\CatFont{X}\to\CatFont{C}$ be a functor.
    \begin{enumerate}
        \item\label{properties-of-dominant-functors-interaction-with-right-whiskering}\SloganFont{Interaction With Right Whiskering. }If $I$ is full and dominant, then the map
            \[
                -\twocirc\id_{I}
                \colon%
                \Nat(F,G)%
                \to%
                \Nat(F\circ I,G\circ I)%
            \]%
            is a bijection.
        \item\label{properties-of-dominant-functors-interaction-with-adjunctions}\SloganFont{Interaction With Adjunctions. }Let $(F,G)\colon\CatFont{C}\rightleftadjointarrows\CatFont{D}$ be an adjunction.
            \begin{enumerate}
                \item\label{properties-of-dominant-functors-interaction-with-adjunctions-a}If $F$ is dominant, then $G$ is faithful.
                \item\label{properties-of-dominant-functors-interaction-with-adjunctions-b}The following conditions are equivalent:
                    \begin{enumerate}
                        \item\label{properties-of-dominant-functors-interaction-with-adjunctions-b-a}The functor $G$ is full.
                        \item\label{properties-of-dominant-functors-interaction-with-adjunctions-b-b}The restriction
                            \[
                                \restriction{G}{\Im_{F}}%
                                \colon%
                                \Im(F)%
                                \to%
                                \CatFont{C}%
                            \]%
                            of $G$ to $\Im(F)$ is full.
                    \end{enumerate}
            \end{enumerate}
        %\item\label{properties-of-dominant-functors-}\SloganFont{. }
    \end{enumerate}
\end{proposition}
\begin{Proof}{Proof of \cref{properties-of-dominant-functors}}%
    \FirstProofBox{\cref{properties-of-dominant-functors-interaction-with-right-whiskering}: Interaction With Right Whiskering}%
    See \cite[Proposition 1.4]{idempotent-triples-and-completion}.

    \ProofBox{\cref{properties-of-dominant-functors-interaction-with-adjunctions}: Interaction With Adjunctions}%
    See \cite[Proposition 1.7]{idempotent-triples-and-completion}.
\end{Proof}
\begin{question}{Characterisations of Functors With Dominant Pre/Postcomposition}{characterisations-of-functors-with-dominant-pre-postcomposition}%
    Is there a characterisation of functors $F\colon\CatFont{C}\to\CatFont{D}$ such that:
    \begin{enumerate}
        \item\label{characterisations-of-functors-with-dominant-pre-postcomposition-a}For each $\CatFont{X}\in\Obj(\Cats)$, the precomposition functor
            \[
                F^{*}%
                \colon%
                \Fun(\CatFont{D},\CatFont{X})%
                \to%
                \Fun(\CatFont{C},\CatFont{X})%
            \]%
            is dominant?
        \item\label{characterisations-of-functors-with-dominant-pre-postcomposition-b}For each $\CatFont{X}\in\Obj(\Cats)$, the postcomposition functor
            \[
                F_{*}%
                \colon%
                \Fun(\CatFont{X},\CatFont{C})%
                \to%
                \Fun(\CatFont{X},\CatFont{D})%
            \]%
            is dominant?
    \end{enumerate}
    This question also appears as \cite{MO468125}.
\end{question}
\subsection{Monomorphisms of Categories}\label{subsection-monomorphisms-of-categories}
Let $\CatFont{C}$ and $\CatFont{D}$ be categories.
\begin{definition}{Monomorphisms of Categories}{monomorphisms-of-categories}%
    A functor $F\colon\CatFont{C}\to\CatFont{D}$ is a \index[categories]{functor!monomorphism}\textbf{monomorphism of categories} if it is a monomorphism in $\Cats$ (see \ChapterRef{\ChapterTypesOfMorphismsInCategories, \cref{types-of-morphisms-in-categories:monomorphisms}}{\cref{monomorphisms}}).%
\end{definition}
\begin{proposition}{Properties of Monomorphisms of Categories}{properties-of-monomorphisms-of-categories}%
    Let $F\colon\CatFont{C}\to\CatFont{D}$ be a functor.
    \begin{enumerate}
        \item\label{properties-of-monomorphisms-of-categories-characterisations}\SloganFont{Characterisations. }The following conditions are equivalent:
            \begin{enumerate}
                \item\label{properties-of-monomorphisms-of-categories-characterisations-a}The functor $F$ is a monomorphism of categories.
                \item\label{properties-of-monomorphisms-of-categories-characterisations-b}The functor $F$ is injective on objects and morphisms, i.e.\ $F$ is injective on objects and the map
                    \[
                        F%
                        \colon%
                        \Mor(\CatFont{C})%
                        \to%
                        \Mor(\CatFont{D})%
                    \]%
                    is injective.
            \end{enumerate}
        %\item\label{properties-of-monomorphisms-of-categories-}\SloganFont{. }
    \end{enumerate}
\end{proposition}
\begin{Proof}{Proof of \cref{properties-of-monomorphisms-of-categories}}%
    \FirstProofBox{\cref{properties-of-monomorphisms-of-categories-characterisations}: Characterisations}%
    Omitted.
\end{Proof}
\begin{question}{Characterisations of Functors With Monic Pre/Postcomposition}{characterisations-of-functors-with-monic-pre-postcomposition}%
    Is there a characterisation of functors $F\colon\CatFont{C}\to\CatFont{D}$ such that:
    \begin{enumerate}
        \item\label{characterisations-of-functors-with-monic-pre-postcomposition-a}For each $\CatFont{X}\in\Obj(\Cats)$, the precomposition functor
            \[
                F^{*}%
                \colon%
                \Fun(\CatFont{D},\CatFont{X})%
                \to%
                \Fun(\CatFont{C},\CatFont{X})%
            \]%
            is a monomorphism of categories?
        \item\label{characterisations-of-functors-with-monic-pre-postcomposition-b}For each $\CatFont{X}\in\Obj(\Cats)$, the postcomposition functor
            \[
                F_{*}%
                \colon%
                \Fun(\CatFont{X},\CatFont{C})%
                \to%
                \Fun(\CatFont{X},\CatFont{D})%
            \]%
            is a monomorphism of categories?
    \end{enumerate}
    This question also appears as \cite{MO468125}.
\end{question}
\subsection{Epimorphisms of Categories}\label{subsection-epimorphisms-of-categories}
Let $\CatFont{C}$ and $\CatFont{D}$ be categories.
\begin{definition}{Epimorphisms of Categories}{epimorphisms-of-categories}%
    A functor $F\colon\CatFont{C}\to\CatFont{D}$ is a \index[categories]{functor!epimorphism}\textbf{epimorphism of categories} if it is a epimorphism in $\Cats$ (see \ChapterRef{\ChapterTypesOfMorphismsInCategories, \cref{types-of-morphisms-in-categories:epimorphisms}}{\cref{epimorphisms}}).%
\end{definition}
\begin{proposition}{Properties of Epimorphisms of Categories}{properties-of-epimorphisms-of-categories}%
    Let $F\colon\CatFont{C}\to\CatFont{D}$ be a functor.
    \begin{enumerate}
        \item\label{properties-of-epimorphisms-of-categories-characterisations}\SloganFont{Characterisations. }The following conditions are equivalent:%
            %--- Begin Footnote ---%
            \footnote{%
                \SloganFont{Further Terminology: }This statement is known as \index[categories]{Isbell's zigzag theorem}\textbf{Isbell's zigzag theorem}.
                \par\vspace*{\TCBBoxCorrection}
            }%
            %---  End Footnote  ---%
            \begin{enumerate}
                \item\label{properties-of-epimorphisms-of-categories-characterisations-a}The functor $F$ is a epimorphism of categories.
                \item\label{properties-of-epimorphisms-of-categories-characterisations-b}For each morphism $f\colon A\to B$ of $\CatFont{D}$, we have a diagram
                    \begin{scalemath}
                        \begin{tikzcd}[row sep={5.0*\the\DL,between origins}, column sep={5.0*\the\DL,between origins}, background color=backgroundColor, ampersand replacement=\&]
                            \&
                            \&
                            \&
                            \&
                            A
                            \&
                            \&
                            \&
                            \&
                            \\[3.0*\the\DL]
                            X_{1}
                            \&
                            \&
                            X_{2}
                            \&
                            \&
                            X_{3}
                            \&
                            \cdots
                            \&
                            X_{m}
                            \&
                            \&
                            A
                            \\
                            \&
                            Y_{1}
                            \&
                            \&
                            Y_{2}
                            \&
                            \&
                            \cdots
                            \&
                            \&
                            Y_{m}
                            \&
                            \\[3.0*\the\DL]
                            \&
                            \&
                            \&
                            \&
                            B
                            \&
                            \&
                            \&
                            \&
                            % 1-Arrows
                            \arrow[from=1-5,to=2-1,"\phi_{1}"']%
                            \arrow[from=1-5,to=2-3,"\phi_{2}"description]%
                            \arrow[from=1-5,to=2-5,"\phi_{3}"description]%
                            \arrow[from=1-5,to=2-7,"\phi_{m}"description]%
                            \arrow[from=1-5,to=2-9,"\id_{A}",dashed]%
                            %
                            \arrow[from=2-1,to=3-2,"\alpha_{1}"description,dashed]%
                            \arrow[from=2-3,to=3-2,"\alpha_{2}"description,dashed]%
                            \arrow[from=2-3,to=3-4,"\alpha_{3}"description,dashed]%
                            \arrow[from=2-5,to=3-4,"\alpha_{4}"description,dashed]%
                            \arrow[from=2-5,to=3-6,"\alpha_{6}"description,dashed]%
                            \arrow[from=2-7,to=3-6,"\alpha_{2m-2}"description,dashed]%
                            \arrow[from=2-7,to=3-8,"\alpha_{2m-1}"description,dashed]%
                            \arrow[from=2-9,to=3-8,"\alpha_{2m}"description,dashed]%
                            %
                            \arrow[from=2-1,to=4-5,"\alpha_{0}"',bend right=30,dashed]%
                            \arrow[from=3-2,to=4-5,"\psi_{1}"'description]%
                            \arrow[from=3-4,to=4-5,"\psi_{2}"description]%
                            \arrow[from=3-6,to=4-5,"\psi_{3}"description]%
                            \arrow[from=3-8,to=4-5,"\psi_{m}"description]%
                        \end{tikzcd}
                    \end{scalemath}
                    in $\CatFont{D}$ satisfying the following conditions:
                    \begin{enumerate}
                        \item\label{properties-of-epimorphisms-of-categories-characterisations-b-a}We have $f=\alpha_{0}\circ\phi_{1}$.
                        \item\label{properties-of-epimorphisms-of-categories-characterisations-b-b}We have $f=\psi_{m}\circ\alpha_{2m}$.
                        \item\label{properties-of-epimorphisms-of-categories-characterisations-b-c}For each $0\leq i\leq 2m$, we have $\alpha_{i}\in\Mor(\Im(F))$.
                    \end{enumerate}
            \end{enumerate}
        \item\label{properties-of-epimorphisms-of-categories-surjectivity-on-objects}\SloganFont{Surjectivity on Objects. }If $F$ is an epimorphism of categories, then $F$ is surjective on objects.
        %\item\label{properties-of-epimorphisms-of-categories-}\SloganFont{. }
    \end{enumerate}
\end{proposition}
\begin{Proof}{Proof of \cref{properties-of-epimorphisms-of-categories}}%
    \FirstProofBox{\cref{properties-of-epimorphisms-of-categories-characterisations}: Characterisations}%
    See \cite{epimorphisms-and-dominions-3}.

    \ProofBox{\cref{properties-of-epimorphisms-of-categories-surjectivity-on-objects}: Surjectivity on Objects}%
    Omitted.
\end{Proof}
\begin{question}{Characterisations of Functors With Epic Pre/Postcomposition}{characterisations-of-functors-with-epic-pre-postcomposition}%
    Is there a characterisation of functors $F\colon\CatFont{C}\to\CatFont{D}$ such that:
    \begin{enumerate}
        \item\label{characterisations-of-functors-with-epic-pre-postcomposition-a}For each $\CatFont{X}\in\Obj(\Cats)$, the precomposition functor
            \[
                F^{*}%
                \colon%
                \Fun(\CatFont{D},\CatFont{X})%
                \to%
                \Fun(\CatFont{C},\CatFont{X})%
            \]%
            is an epimorphism of categories?
        \item\label{characterisations-of-functors-with-epic-pre-postcomposition-b}For each $\CatFont{X}\in\Obj(\Cats)$, the postcomposition functor
            \[
                F_{*}%
                \colon%
                \Fun(\CatFont{X},\CatFont{C})%
                \to%
                \Fun(\CatFont{X},\CatFont{D})%
            \]%
            is an epimorphism of categories?
    \end{enumerate}
    This question also appears as \cite{MO468125}.
\end{question}
\subsection{Pseudomonic Functors}\label{subsection-pseudomonic-functors}
Let $\CatFont{C}$ and $\CatFont{D}$ be categories.
\begin{definition}{Pseudomonic Functors}{pseudomonic-functors}%
    A functor $F\colon\CatFont{C}\to\CatFont{D}$ is \index[categories]{functor!pseudomonic}\textbf{pseudomonic} if it satisfies the following conditions:
    \begin{enumerate}
        \item\label{pseudomonic-functors-1}For all diagrams of the form
            \[
                \begin{tikzcd}[row sep={5.0*\the\DL,between origins}, column sep={5.0*\the\DL,between origins}, background color=backgroundColor, ampersand replacement=\&]
                    \CatFont{X}
                    \arrow[r,"\phi"{name=1}, bend left=30]
                    \arrow[r,"\psi"'{name=2},bend right=30]
                    \&
                    \CatFont{C}
                    \arrow[r,"F"]
                    \&[-1.5*\the\DL]
                    \CatFont{D}\mrp{,}
                    % 2-Arrows
                    \arrow[from=1,to=2,"\alpha"',xshift=-0.3em,shorten=0.25em,Rightarrow]%
                    \arrow[from=1,to=2,"\beta",  xshift=+0.3em,shorten=0.25em,Rightarrow]%
                \end{tikzcd}
            \]%
            if we have
            \[
                \id_{F}\twocirc\alpha%
                =%
                \id_{F}\twocirc\beta,%
            \]%
            then $\alpha=\beta$.
        \item\label{pseudomonic-functors-2}For each $\CatFont{X}\in\Obj(\Cats)$ and each natural isomorphism
            \begin{webcompile}
                \beta%
                \colon%
                F\circ\phi%
                \Longrightisoarrow%
                F\circ\psi,%
                \quad%
                \begin{tikzcd}[row sep={5.0*\the\DL,between origins}, column sep={5.0*\the\DL,between origins}, background color=backgroundColor, ampersand replacement=\&]
                    \CatFont{X}
                    \arrow[r,"F\circ\phi"{name=1}, bend left=30]
                    \arrow[r,"F\circ\psi"'{name=2},bend right=30]
                    \&
                    \CatFont{D}\mrp{,}
                    % 2-Arrows
                    \arrow[from=1,to=2,"\beta"',shorten=0.25em,Rightarrow]%
                \end{tikzcd}
            \end{webcompile}
            there exists a natural isomorphism
            \begin{webcompile}
                \alpha%
                \colon%
                \phi%
                \Longrightisoarrow%
                \psi,%
                \quad%
                \begin{tikzcd}[row sep={5.0*\the\DL,between origins}, column sep={5.0*\the\DL,between origins}, background color=backgroundColor, ampersand replacement=\&]
                    \CatFont{X}
                    \arrow[r,"\phi"{name=1}, bend left=30]
                    \arrow[r,"\psi"'{name=2},bend right=30]
                    \&
                    \CatFont{C}
                    % 2-Arrows
                    \arrow[from=1,to=2,"\alpha"',shorten=0.25em,Rightarrow]%
                \end{tikzcd}
            \end{webcompile}
            such that we have an equality
            \begin{webcompile}
                \begin{tikzcd}[row sep={5.0*\the\DL,between origins}, column sep={5.0*\the\DL,between origins}, background color=backgroundColor, ampersand replacement=\&]
                    \CatFont{X}
                    \arrow[r,"\phi"{name=1}, bend left=30]
                    \arrow[r,"\psi"'{name=2},bend right=30]
                    \&
                    \CatFont{C}
                    \arrow[r,"F"]
                    \&[-1.5*\the\DL]
                    \CatFont{D}
                    % 2-Arrows
                    \arrow[from=1,to=2,"\alpha"',shorten=0.25em,Rightarrow]%
                \end{tikzcd}
                \quad
                \bigequalssign
                \quad
                \begin{tikzcd}[row sep={5.0*\the\DL,between origins}, column sep={5.0*\the\DL,between origins}, background color=backgroundColor, ampersand replacement=\&]
                    \CatFont{X}
                    \arrow[r,"F\circ\phi"{name=1}, bend left=30]
                    \arrow[r,"F\circ\psi"'{name=2},bend right=30]
                    \&
                    \CatFont{D}
                    % 2-Arrows
                    \arrow[from=1,to=2,"\beta"',shorten=0.25em,Rightarrow]%
                \end{tikzcd}
            \end{webcompile}
            of pasting diagrams, i.e.\ such that we have
            \[
                \beta%
                =%
                \id_{F}\twocirc\alpha.%
            \]%
    \end{enumerate}
\end{definition}
\begin{proposition}{Properties of Pseudomonic Functors}{properties-of-pseudomonic-functors}%
    Let $F\colon\CatFont{C}\to\CatFont{D}$ be a functor.
    \begin{enumerate}
        \item\label{properties-of-pseudomonic-functors-characterisations}\SloganFont{Characterisations. }The following conditions are equivalent:
            \begin{enumerate}
                \item\label{properties-of-pseudomonic-functors-characterisations-a}The functor $F$ is pseudomonic.
                \item\label{properties-of-pseudomonic-functors-characterisations-b}The functor $F$ satisfies the following conditions:
                    \begin{enumerate}
                        \item\label{properties-of-pseudomonic-functors-characterisations-b-a}The functor $F$ is faithful, i.e.\ for each $A,B\in\Obj(\CatFont{C})$, the action on morphisms
                            \[
                                F_{A,B}
                                \colon
                                \Hom_{\CatFont{C}}(A,B)
                                \to
                                \Hom_{\CatFont{D}}(F_{A},F_{B})
                            \]%
                            of $F$ at $(A,B)$ is injective.
                        \item\label{properties-of-pseudomonic-functors-characterisations-b-b}For each $A,B\in\Obj(\CatFont{C})$, the restriction
                            \[
                                F^{\iso}_{A,B}
                                \colon
                                \Iso_{\CatFont{C}}(A,B)
                                \to
                                \Iso_{\CatFont{D}}(F_{A},F_{B})
                            \]%
                            of the action on morphisms of $F$ at $(A,B)$ to isomorphisms is surjective.
                    \end{enumerate}
                \item\label{properties-of-pseudomonic-functors-characterisations-c}We have an isocomma square of the form
                    \begin{webcompile}
                        \CatFont{C}%
                        \eqcong%
                        \CatFont{C}\isocomma_{\CatFont{D}}\CatFont{C},%
                        \quad
                        \begin{tikzcd}[row sep={5.0*\the\DL,between origins}, column sep={5.0*\the\DL,between origins}, background color=backgroundColor, ampersand replacement=\&]
                            \CatFont{C}
                            \arrow[r,"\id_{\CatFont{C}}"]
                            \arrow[d,"\id_{\CatFont{C}}"']
                            \&
                            \CatFont{C}
                            \arrow[d,"F"]
                            \\
                            \CatFont{C}
                            \arrow[r,"F"']
                            \&
                            \CatFont{D}
                            % 2-Arrows
                            \arrow[from=2-1,to=1-2,Leftrightarrow,dashed,shorten=0.5em]%
                        \end{tikzcd}
                    \end{webcompile}
                    in $\TwoCategoryOfCategories$ up to equivalence.
                \item\label{properties-of-pseudomonic-functors-characterisations-d}We have an isocomma square of the form
                    \begin{webcompile}
                        \CatFont{C}%
                        \eqcong%
                        \CatFont{C}\isocomma_{\Arr{\CatFont{D}}}\CatFont{D},%
                        \quad
                        \begin{tikzcd}[row sep={5.0*\the\DL,between origins}, column sep={5.0*\the\DL,between origins}, background color=backgroundColor, ampersand replacement=\&]
                            \CatFont{C}
                            \arrow[r,hook']
                            \arrow[d,"F"']
                            \&
                            {\Arr{\CatFont{C}}}
                            \arrow[d,"{\Arr{F}}"]
                            \\
                            \CatFont{D}
                            \arrow[r,hook]
                            \&
                            {\Arr{\CatFont{D}}}
                            % 2-Arrows
                            \arrow[from=2-1,to=1-2,Leftrightarrow,dashed,shorten=0.5em]%
                        \end{tikzcd}
                    \end{webcompile}
                    in $\TwoCategoryOfCategories$ up to equivalence.
                \item\label{properties-of-pseudomonic-functors-characterisations-e}For each $\CatFont{X}\in\Obj(\Cats)$, the postcomposition%
                    %--- Begin Footnote ---%
                    \footnote{%
                        Asking the precomposition functors
                        \[
                            F^{*}%
                            \colon%
                            \Fun(\CatFont{D},\CatFont{X})%
                            \to%
                            \Fun(\CatFont{C},\CatFont{X})%
                        \]%
                        to be pseudomonic leads to pseudoepic functors; see \cref{properties-of-pseudoepic-functors-characterisations-b} of \cref{properties-of-pseudoepic-functors-characterisations} of \cref{properties-of-pseudoepic-functors}.
                        \par\vspace*{\TCBBoxCorrection}
                    } %
                    %---  End Footnote  ---%
                    functor
                    \[
                        F_{*}%
                        \colon%
                        \Fun(\CatFont{X},\CatFont{C})%
                        \to%
                        \Fun(\CatFont{X},\CatFont{D})%
                    \]%
                    is pseudomonic.
            \end{enumerate}
        \item\label{properties-of-pseudomonic-functors-conservativity}\SloganFont{Conservativity. }If $F$ is pseudomonic, then $F$ is conservative.
        \item\label{properties-of-pseudomonic-functors-essential-injectivity}\SloganFont{Essential Injectivity. }If $F$ is pseudomonic, then $F$ is essentially injective.
    \end{enumerate}
\end{proposition}
\begin{Proof}{Proof of \cref{properties-of-pseudomonic-functors}}%
    \FirstProofBox{\cref{properties-of-pseudomonic-functors-characterisations}: Characterisations}%
    Omitted.
    %We claim that \cref{properties-of-pseudomonic-functors-characterisations-a,properties-of-pseudomonic-functors-characterisations-b,properties-of-pseudomonic-functors-characterisations-c} are indeed equivalent:
    %\begin{enumerate}
    %    \item\SloganFont{\cref{properties-of-pseudomonic-functors-characterisations-a}$\implies$\cref{properties-of-pseudomonic-functors-characterisations-b}: }
    %    \item\SloganFont{\cref{properties-of-pseudomonic-functors-characterisations-b}$\implies$\cref{properties-of-pseudomonic-functors-characterisations-a}: }
    %    \item\SloganFont{\cref{properties-of-pseudomonic-functors-characterisations-a}$\implies$\cref{properties-of-pseudomonic-functors-characterisations-c}: }
    %    \item\SloganFont{\cref{properties-of-pseudomonic-functors-characterisations-c}$\implies$\cref{properties-of-pseudomonic-functors-characterisations-a}: }
    %\end{enumerate}
    %This finishes the proof of \cref{properties-of-pseudomonic-functors-characterisations}.

    \ProofBox{\cref{properties-of-pseudomonic-functors-conservativity}: Conservativity}%
    Omitted.

    \ProofBox{\cref{properties-of-pseudomonic-functors-essential-injectivity}: Essential Injectivity}%
    Omitted.
\end{Proof}
\subsection{Pseudoepic Functors}\label{subsection-pseudoepic-functors}
Let $\CatFont{C}$ and $\CatFont{D}$ be categories.
\begin{definition}{Pseudoepic Functors}{pseudoepic-functors}%
    A functor $F\colon\CatFont{C}\to\CatFont{D}$ is \index[categories]{functor!pseudoepic}\textbf{pseudoepic} if it satisfies the following conditions:
    \begin{enumerate}
        \item\label{pseudoepic-functors-1}For all diagrams of the form
            \[
                \begin{tikzcd}[row sep={5.0*\the\DL,between origins}, column sep={5.0*\the\DL,between origins}, background color=backgroundColor, ampersand replacement=\&]
                    \CatFont{C}
                    \arrow[r,"F"]
                    \&[-1.5*\the\DL]
                    \CatFont{D}
                    \arrow[r,"\phi"{name=1}, bend left=30]
                    \arrow[r,"\psi"'{name=2},bend right=30]
                    \&
                    \CatFont{X}\mrp{,}
                    % 2-Arrows
                    \arrow[from=1,to=2,"\alpha"',xshift=-0.3em,shorten=0.25em,Rightarrow]%
                    \arrow[from=1,to=2,"\beta",  xshift=+0.3em,shorten=0.25em,Rightarrow]%
                \end{tikzcd}
            \]%
            if we have
            \[
                \alpha\twocirc\id_{F}%
                =%
                \beta\twocirc\id_{F},%
            \]%
            then $\alpha=\beta$.
        \item\label{pseudoepic-functors-2}For each $X\in\Obj(\CatFont{C})$ and each $2$-isomorphism
            \begin{webcompile}
                \beta%
                \colon%
                \phi\circ F%
                \Longrightisoarrow%
                \psi\circ F,%
                \quad%
                \begin{tikzcd}[row sep={5.0*\the\DL,between origins}, column sep={5.0*\the\DL,between origins}, background color=backgroundColor, ampersand replacement=\&]
                    \CatFont{C}
                    \arrow[r,"\phi\circ F"{name=1}, bend left=30]
                    \arrow[r,"\psi\circ F"'{name=2},bend right=30]
                    \&
                    \CatFont{X}
                    % 2-Arrows
                    \arrow[from=1,to=2,"\beta"',shorten=0.25em,Rightarrow]%
                \end{tikzcd}
            \end{webcompile}
            of $\CatFont{C}$, there exists a $2$-isomorphism
            \begin{webcompile}
                \alpha%
                \colon%
                \phi%
                \Longrightisoarrow%
                \psi,%
                \quad%
                \begin{tikzcd}[row sep={5.0*\the\DL,between origins}, column sep={5.0*\the\DL,between origins}, background color=backgroundColor, ampersand replacement=\&]
                    \CatFont{D}
                    \arrow[r,"\phi"{name=1}, bend left=30]
                    \arrow[r,"\psi"'{name=2},bend right=30]
                    \&
                    \CatFont{X}
                    % 2-Arrows
                    \arrow[from=1,to=2,"\alpha"',shorten=0.25em,Rightarrow]%
                \end{tikzcd}
            \end{webcompile}
            of $\CatFont{C}$ such that we have an equality
            \begin{webcompile}
                \begin{tikzcd}[row sep={5.0*\the\DL,between origins}, column sep={5.0*\the\DL,between origins}, background color=backgroundColor, ampersand replacement=\&]
                    \CatFont{C}
                    \arrow[r,"F"]
                    \&[-1.5*\the\DL]
                    \CatFont{D}
                    \arrow[r,"\phi"{name=1}, bend left=30]
                    \arrow[r,"\psi"'{name=2},bend right=30]
                    \&
                    \CatFont{X}
                    % 2-Arrows
                    \arrow[from=1,to=2,"\alpha"',shorten=0.25em,Rightarrow]%
                \end{tikzcd}
                \quad
                \bigequalssign
                \quad
                \begin{tikzcd}[row sep={5.0*\the\DL,between origins}, column sep={5.0*\the\DL,between origins}, background color=backgroundColor, ampersand replacement=\&]
                    \CatFont{C}
                    \arrow[r,"\phi\circ F"{name=1}, bend left=30]
                    \arrow[r,"\psi\circ F"'{name=2},bend right=30]
                    \&
                    \CatFont{X}
                    % 2-Arrows
                    \arrow[from=1,to=2,"\beta"',shorten=0.25em,Rightarrow]%
                \end{tikzcd}
            \end{webcompile}
            of pasting diagrams in $\CatFont{C}$, i.e.\ such that we have
            \[
                \beta%
                =%
                \alpha\twocirc\id_{F}.%
            \]%
    \end{enumerate}
\end{definition}
\begin{proposition}{Properties of Pseudoepic Functors}{properties-of-pseudoepic-functors}%
    Let $F\colon\CatFont{C}\to\CatFont{D}$ be a functor.
    \begin{enumerate}
        \item\label{properties-of-pseudoepic-functors-characterisations}\SloganFont{Characterisations. }The following conditions are equivalent:
            \begin{enumerate}
                \item\label{properties-of-pseudoepic-functors-characterisations-a}The functor $F$ is pseudoepic.
                \item\label{properties-of-pseudoepic-functors-characterisations-b}For each $\CatFont{X}\in\Obj(\Cats)$, the functor
                    \[
                        F^{*}%
                        \colon%
                        \Fun(\CatFont{D},\CatFont{X})%
                        \to%
                        \Fun(\CatFont{C},\CatFont{X})%
                    \]%
                    given by precomposition by $F$ is pseudomonic.
                \item\label{properties-of-pseudoepic-functors-characterisations-c}We have an isococomma square of the form
                    \begin{webcompile}
                        \CatFont{D}%
                        \eqcong%
                        \CatFont{D}\isococomma_{\CatFont{C}}\CatFont{D},%
                        \quad
                        \begin{tikzcd}[row sep={5.0*\the\DL,between origins}, column sep={5.0*\the\DL,between origins}, background color=backgroundColor, ampersand replacement=\&]
                            \CatFont{D}
                            \arrow[from=r,"\id_{\CatFont{D}}"']
                            \arrow[from=d,"\id_{\CatFont{D}}"]
                            \&
                            \CatFont{D}
                            \arrow[from=d,"F"']
                            \\
                            \CatFont{D}
                            \arrow[from=r,"F"]
                            \&
                            \CatFont{C}
                            % 2-Arrows
                            \arrow[from=2-1,to=1-2,Leftrightarrow,dashed,shorten=0.5em]%
                        \end{tikzcd}
                    \end{webcompile}
                    in $\TwoCategoryOfCategories$ up to equivalence.
            \end{enumerate}
        \item\label{properties-of-pseudoepic-functors-dominance}\SloganFont{Dominance. }If $F$ is pseudoepic, then $F$ is dominant (\cref{dominant-functors}).
        %\item\label{properties-of-pseudoepic-functors-}\SloganFont{. }
    \end{enumerate}
\end{proposition}
\begin{Proof}{Proof of \cref{properties-of-pseudoepic-functors}}%
    \FirstProofBox{\cref{properties-of-pseudoepic-functors-characterisations}: Characterisations}%
    Omitted.

    \ProofBox{\cref{properties-of-pseudoepic-functors-dominance}: Dominance}%
    If $F$ is pseudoepic, then
    \[
        F^{*}%
        \colon%
        \Fun(\CatFont{D},\CatFont{X})%
        \to%
        \Fun(\CatFont{C},\CatFont{X})%
    \]%
    is pseudomonic for all $\CatFont{X}\in\Obj(\Cats)$, and thus in particular faithful. By \cref{properties-of-faithful-functors-interaction-with-precomposition-3-g} of \cref{properties-of-faithful-functors-interaction-with-precomposition-3} of \cref{properties-of-faithful-functors}, this is equivalent to requiring $F$ to be dominant.
\end{Proof}
\begin{question}{Characterisations of Pseudoepic Functors}{characterisations-of-pseudoepic-functors}%
    Is there a nice characterisation of the pseudoepic functors, similarly to the characterisaiton of pseudomonic functors given in \cref{properties-of-pseudomonic-functors-characterisations-b} of \cref{properties-of-pseudomonic-functors-characterisations} of \cref{properties-of-pseudomonic-functors}?

    This question also appears as \cite{MO321971}.
\end{question}
\begin{question}{Must a Pseudomonic and Pseudoepic Functor Be an Equivalence of Categories}{must-a-pseudomonic-and-pseudoepic-functor-be-an-equivalence-of-categories}%
    A pseudomonic and pseudoepic functor is dominant, faithful, essentially injective, and full on isomorphisms. Is it necessarily an equivalence of categories? If not, how bad can this fail, i.e.\ how far can a pseudomonic and pseudoepic functor be from an equivalence of categories?

    This question also appears as \cite{MO468334}.
\end{question}
\begin{question}{Characterisations of Functors With Pseudoepic Pre/Postcomposition}{characterisations-of-functors-with-pseudoepic-pre-postcomposition}%
    Is there a characterisation of functors $F\colon\CatFont{C}\to\CatFont{D}$ such that:
    \begin{enumerate}
        \item\label{characterisations-of-functors-with-pseudoepic-pre-postcomposition-a}For each $\CatFont{X}\in\Obj(\Cats)$, the precomposition functor
            \[
                F^{*}%
                \colon%
                \Fun(\CatFont{D},\CatFont{X})%
                \to%
                \Fun(\CatFont{C},\CatFont{X})%
            \]%
            is pseudoepic?
        \item\label{characterisations-of-functors-with-pseudoepic-pre-postcomposition-b}For each $\CatFont{X}\in\Obj(\Cats)$, the postcomposition functor
            \[
                F_{*}%
                \colon%
                \Fun(\CatFont{X},\CatFont{C})%
                \to%
                \Fun(\CatFont{X},\CatFont{D})%
            \]%
            is pseudoepic?
    \end{enumerate}
    This question also appears as \cite{MO468125}.
\end{question}
\section{Even More Conditions on Functors}\label{section-even-more-conditions-on-functors}
\subsection{Injective on Objects Functors}\label{subsection-injective-on-objects-functors}
Let $\CatFont{C}$ and $\CatFont{D}$ be categories.
\begin{definition}{Injective on Objects Functors}{injective-on-objects-functors}%
    A functor $F\colon\CatFont{C}\to\CatFont{D}$ is \index[categories]{functor!injective on objects}\textbf{injective on objects} if the action on objects
    \[
        F%
        \colon%
        \Obj(\CatFont{C})%
        \to%
        \Obj(\CatFont{D})%
    \]%
    of $F$ is injective.
\end{definition}
\begin{proposition}{Properties of Injective on Objects Functors}{properties-of-injective-on-objects-functors}%
    Let $F\colon\CatFont{C}\to\CatFont{D}$ be a functor.
    \begin{enumerate}
        \item\label{properties-of-injective-on-objects-functors-characterisations}\SloganFont{Characterisations. }The following conditions are equivalent:
            \begin{enumerate}
                \item\label{properties-of-injective-on-objects-functors-characterisations-a}The functor $F$ is injective on objects.
                \item\label{properties-of-injective-on-objects-functors-characterisations-b}The functor $F$ is an isocofibration in $\TwoCategoryOfCategories$.
            \end{enumerate}
        %\item\label{properties-of-injective-on-objects-functors-}\SloganFont{. }
    \end{enumerate}
\end{proposition}
\begin{Proof}{Proof of \cref{properties-of-injective-on-objects-functors}}%
    \FirstProofBox{\cref{properties-of-injective-on-objects-functors-characterisations}: Characterisations}%
    Omitted.
\end{Proof}
\subsection{Surjective on Objects Functors}\label{subsection-surjective-on-objects-functors}
Let $\CatFont{C}$ and $\CatFont{D}$ be categories.
\begin{definition}{Surjective on Objects Functors}{surjective-on-objects-functors}%
    A functor $F\colon\CatFont{C}\to\CatFont{D}$ is \index[categories]{functor!surjective on objects}\textbf{surjective on objects} if the action on objects
    \[
        F%
        \colon%
        \Obj(\CatFont{C})%
        \to%
        \Obj(\CatFont{D})%
    \]%
    of $F$ is surjective.
\end{definition}
\subsection{Bijective on Objects Functors}\label{subsection-bijective-on-objects-functors}
Let $\CatFont{C}$ and $\CatFont{D}$ be categories.
\begin{definition}{Bijective on Objects Functors}{bijective-on-objects-functors}%
    A functor $F\colon\CatFont{C}\to\CatFont{D}$ is \index[categories]{functor!bijective on objects}\textbf{bijective on objects}%
    %--- Begin Footnote ---%
    \footnote{%
        \SloganFont{Further Terminology: }Also called a \index[categories]{functor!bo}\textbf{bo} functor.
        \par\vspace*{\TCBBoxCorrection}
    } %
    %---  End Footnote  ---%
    if the action on objects
    \[
        F%
        \colon%
        \Obj(\CatFont{C})%
        \to%
        \Obj(\CatFont{D})%
    \]%
    of $F$ is a bijection.
\end{definition}
\subsection{Functors Representably Faithful on Cores}\label{subsection-functors-representably-faithful-on-cores}
Let $\CatFont{C}$ and $\CatFont{D}$ be categories.
\begin{definition}{Functors Representably Faithful on Cores}{functors-representably-faithful-on-cores}%
    A functor $F\colon\CatFont{C}\to\CatFont{D}$ is \index[categories]{functor!representably faithful on cores}\textbf{representably faithful on cores} if, for each $X\in\Obj(\Cats)$, the postcomposition by $F$ functor
    \[
        F_{*}%
        \colon%
        \Core(\Fun(\CatFont{X},\CatFont{C}))%
        \to
        \Core(\Fun(\CatFont{X},\CatFont{D}))%
    \]%
    is faithful.
\end{definition}
\begin{remark}{Unwinding \cref{functors-representably-faithful-on-cores}}{unwinding-functors-representably-faithful-on-cores}%
    In detail, a functor $F\colon\CatFont{C}\to\CatFont{D}$ is \textbf{representably faithful on cores} if, given a diagram of the form
    \[
        \begin{tikzcd}[row sep={5.0*\the\DL,between origins}, column sep={5.0*\the\DL,between origins}, background color=backgroundColor, ampersand replacement=\&]
            \CatFont{X}
            \arrow[r,"\phi"{name=1}, bend left=30]
            \arrow[r,"\psi"'{name=2},bend right=30]
            \&
            \CatFont{C}
            \arrow[r,"F"]
            \&[-1.5*\the\DL]
            \CatFont{D}\mrp{,}
            % 2-Arrows
            \arrow[from=1,to=2,"\alpha"',xshift=-0.3em,shorten=0.25em,Rightarrow]%
            \arrow[from=1,to=2,"\beta",  xshift=+0.3em,shorten=0.25em,Rightarrow]%
        \end{tikzcd}
    \]%
    if $\alpha$ and $\beta$ are natural isomorphisms and we have
    \[
        \id_{F}\twocirc\alpha%
        =%
        \id_{F}\twocirc\beta,%
    \]%
    then $\alpha=\beta$.
\end{remark}
\begin{question}{Characterisation of Functors Representably Faithful on Cores}{characterisation-of-functors-representably-faithful-on-cores}%
    Is there a characterisation of functors representably faithful on cores?
\end{question}
\subsection{Functors Representably Full on Cores}\label{subsection-functors-representably-full-on-cores}
Let $\CatFont{C}$ and $\CatFont{D}$ be categories.
\begin{definition}{Functors Representably Full on Cores}{functors-representably-full-on-cores}%
    A functor $F\colon\CatFont{C}\to\CatFont{D}$ is \index[categories]{functor!representably full on cores}\textbf{representably full on cores} if, for each $X\in\Obj(\Cats)$, the postcomposition by $F$ functor
    \[
        F_{*}%
        \colon%
        \Core(\Fun(\CatFont{X},\CatFont{C}))%
        \to
        \Core(\Fun(\CatFont{X},\CatFont{D}))%
    \]%
    is full.
\end{definition}
\begin{remark}{Unwinding \cref{functors-representably-full-on-cores}}{unwinding-functors-representably-full-on-cores}%
    In detail, a functor $F\colon\CatFont{C}\to\CatFont{D}$ is \textbf{representably full on cores} if, for each $\CatFont{X}\in\Obj(\Cats)$ and each natural isomorphism
    \begin{webcompile}
        \beta%
        \colon%
        F\circ\phi%
        \Longrightisoarrow%
        F\circ\psi,%
        \quad%
        \begin{tikzcd}[row sep={5.0*\the\DL,between origins}, column sep={5.0*\the\DL,between origins}, background color=backgroundColor, ampersand replacement=\&]
            \CatFont{X}
            \arrow[r,"F\circ\phi"{name=1}, bend left=30]
            \arrow[r,"F\circ\psi"'{name=2},bend right=30]
            \&
            \CatFont{D}\mrp{,}
            % 2-Arrows
            \arrow[from=1,to=2,"\beta"',shorten=0.25em,Rightarrow]%
        \end{tikzcd}
    \end{webcompile}
    there exists a natural isomorphism
    \begin{webcompile}
        \alpha%
        \colon%
        \phi%
        \Longrightisoarrow%
        \psi,%
        \quad%
        \begin{tikzcd}[row sep={5.0*\the\DL,between origins}, column sep={5.0*\the\DL,between origins}, background color=backgroundColor, ampersand replacement=\&]
            \CatFont{X}
            \arrow[r,"\phi"{name=1}, bend left=30]
            \arrow[r,"\psi"'{name=2},bend right=30]
            \&
            \CatFont{C}
            % 2-Arrows
            \arrow[from=1,to=2,"\alpha"',shorten=0.25em,Rightarrow]%
        \end{tikzcd}
    \end{webcompile}
    such that we have an equality
    \begin{webcompile}
        \begin{tikzcd}[row sep={5.0*\the\DL,between origins}, column sep={5.0*\the\DL,between origins}, background color=backgroundColor, ampersand replacement=\&]
            \CatFont{X}
            \arrow[r,"\phi"{name=1}, bend left=30]
            \arrow[r,"\psi"'{name=2},bend right=30]
            \&
            \CatFont{C}
            \arrow[r,"F"]
            \&[-1.5*\the\DL]
            \CatFont{D}
            % 2-Arrows
            \arrow[from=1,to=2,"\alpha"',shorten=0.25em,Rightarrow]%
        \end{tikzcd}
        \quad
        \bigequalssign
        \quad
        \begin{tikzcd}[row sep={5.0*\the\DL,between origins}, column sep={5.0*\the\DL,between origins}, background color=backgroundColor, ampersand replacement=\&]
            \CatFont{X}
            \arrow[r,"F\circ\phi"{name=1}, bend left=30]
            \arrow[r,"F\circ\psi"'{name=2},bend right=30]
            \&
            \CatFont{D}
            % 2-Arrows
            \arrow[from=1,to=2,"\beta"',shorten=0.25em,Rightarrow]%
        \end{tikzcd}
    \end{webcompile}
    of pasting diagrams in $\TwoCategoryOfCategories$, i.e.\ such that we have
    \[
        \beta%
        =%
        \id_{F}\twocirc\alpha.%
    \]%
\end{remark}
\begin{question}{Characterisation of Functors Representably Full on Cores}{characterisation-of-functors-representably-full-on-cores}%
    Is there a characterisation of functors representably full on cores?

    This question also appears as \cite{MO468125}.
\end{question}
\subsection{Functors Representably Fully Faithful on Cores}\label{subsection-functors-representably-fully-faithful-on-cores}
Let $\CatFont{C}$ and $\CatFont{D}$ be categories.
\begin{definition}{Functors Representably Fully Faithful on Cores}{functors-representably-fully-faithful-on-cores}%
    A functor $F\colon\CatFont{C}\to\CatFont{D}$ is \index[categories]{functor!representably fully faithful on cores}\textbf{representably fully faithful on cores} if, for each $X\in\Obj(\Cats)$, the postcomposition by $F$ functor
    \[
        F_{*}%
        \colon%
        \Core(\Fun(\CatFont{X},\CatFont{C}))%
        \to
        \Core(\Fun(\CatFont{X},\CatFont{D}))%
    \]%
    is fully faithful.
\end{definition}
\begin{remark}{Unwinding \cref{functors-representably-fully-faithful-on-cores}}{unwinding-functors-representably-fully-faithful-on-cores}%
    In detail, a functor $F\colon\CatFont{C}\to\CatFont{D}$ is \textbf{representably fully faithful on cores} if it satisfies the conditions in \cref{unwinding-functors-representably-faithful-on-cores,unwinding-functors-representably-full-on-cores}, i.e.:
    \begin{enumerate}
        \item\label{unwinding-functors-representably-fully-faithful-on-cores-1}For all diagrams of the form
            \[
                \begin{tikzcd}[row sep={5.0*\the\DL,between origins}, column sep={5.0*\the\DL,between origins}, background color=backgroundColor, ampersand replacement=\&]
                    \CatFont{X}
                    \arrow[r,"\phi"{name=1}, bend left=30]
                    \arrow[r,"\psi"'{name=2},bend right=30]
                    \&
                    \CatFont{C}
                    \arrow[r,"F"]
                    \&[-1.5*\the\DL]
                    \CatFont{D}\mrp{,}
                    % 2-Arrows
                    \arrow[from=1,to=2,"\alpha"',xshift=-0.3em,shorten=0.25em,Rightarrow]%
                    \arrow[from=1,to=2,"\beta",  xshift=+0.3em,shorten=0.25em,Rightarrow]%
                \end{tikzcd}
            \]%
            with $\alpha$ and $\beta$ natural isomorphisms, if we have $\id_{F}\twocirc\alpha=\id_{F}\twocirc\beta$, then $\alpha=\beta$.
        \item\label{unwinding-functors-representably-fully-faithful-on-cores-2}For each $\CatFont{X}\in\Obj(\Cats)$ and each natural isomorphism
            \begin{webcompile}
                \beta%
                \colon%
                F\circ\phi%
                \Longrightisoarrow%
                F\circ\psi,%
                \quad%
                \begin{tikzcd}[row sep={5.0*\the\DL,between origins}, column sep={5.0*\the\DL,between origins}, background color=backgroundColor, ampersand replacement=\&]
                    \CatFont{X}
                    \arrow[r,"F\circ\phi"{name=1}, bend left=30]
                    \arrow[r,"F\circ\psi"'{name=2},bend right=30]
                    \&
                    \CatFont{D}
                    % 2-Arrows
                    \arrow[from=1,to=2,"\beta"',shorten=0.25em,Rightarrow]%
                \end{tikzcd}
            \end{webcompile}
            of $\CatFont{C}$, there exists a natural isomorphism
            \begin{webcompile}
                \alpha%
                \colon%
                \phi%
                \Longrightisoarrow%
                \psi,%
                \quad%
                \begin{tikzcd}[row sep={5.0*\the\DL,between origins}, column sep={5.0*\the\DL,between origins}, background color=backgroundColor, ampersand replacement=\&]
                    \CatFont{X}
                    \arrow[r,"\phi"{name=1}, bend left=30]
                    \arrow[r,"\psi"'{name=2},bend right=30]
                    \&
                    \CatFont{C}
                    % 2-Arrows
                    \arrow[from=1,to=2,"\alpha"',shorten=0.25em,Rightarrow]%
                \end{tikzcd}
            \end{webcompile}
            of $\CatFont{C}$ such that we have an equality
            \begin{webcompile}
                \begin{tikzcd}[row sep={5.0*\the\DL,between origins}, column sep={5.0*\the\DL,between origins}, background color=backgroundColor, ampersand replacement=\&]
                    \CatFont{X}
                    \arrow[r,"\phi"{name=1}, bend left=30]
                    \arrow[r,"\psi"'{name=2},bend right=30]
                    \&
                    \CatFont{C}
                    \arrow[r,"F"]
                    \&[-1.5*\the\DL]
                    \CatFont{D}
                    % 2-Arrows
                    \arrow[from=1,to=2,"\alpha"',shorten=0.25em,Rightarrow]%
                \end{tikzcd}
                \quad
                \bigequalssign
                \quad
                \begin{tikzcd}[row sep={5.0*\the\DL,between origins}, column sep={5.0*\the\DL,between origins}, background color=backgroundColor, ampersand replacement=\&]
                    \CatFont{X}
                    \arrow[r,"F\circ\phi"{name=1}, bend left=30]
                    \arrow[r,"F\circ\psi"'{name=2},bend right=30]
                    \&
                    \CatFont{D}
                    % 2-Arrows
                    \arrow[from=1,to=2,"\beta"',shorten=0.25em,Rightarrow]%
                \end{tikzcd}
            \end{webcompile}
            of pasting diagrams in $\TwoCategoryOfCategories$, i.e.\ such that we have
            \[
                \beta%
                =%
                \id_{F}\twocirc\alpha.%
            \]%
    \end{enumerate}
\end{remark}
\begin{question}{Characterisation of Functors Representably Fully Faithful on Cores}{characterisation-of-functors-representably-fully-faithful-on-cores}%
    Is there a characterisation of functors representably fully faithful on cores?
\end{question}
\subsection{Functors Corepresentably Faithful on Cores}\label{subsection-functors-corepresentably-faithful-on-cores}
Let $\CatFont{C}$ and $\CatFont{D}$ be categories.
\begin{definition}{Functors Corepresentably Faithful on Cores}{functors-corepresentably-faithful-on-cores}%
    A functor $F\colon\CatFont{C}\to\CatFont{D}$ is \index[categories]{functor!corepresentably faithful on cores}\textbf{corepresentably faithful on cores} if, for each $X\in\Obj(\Cats)$, the postcomposition by $F$ functor
    \[
        F_{*}%
        \colon%
        \Core(\Fun(\CatFont{X},\CatFont{C}))%
        \to
        \Core(\Fun(\CatFont{X},\CatFont{D}))%
    \]%
    is faithful.
\end{definition}
\begin{remark}{Unwinding \cref{functors-corepresentably-faithful-on-cores}}{unwinding-functors-corepresentably-faithful-on-cores}%
    In detail, a functor $F\colon\CatFont{C}\to\CatFont{D}$ is \textbf{corepresentably faithful on cores} if, given a diagram of the form
    \[
        \begin{tikzcd}[row sep={5.0*\the\DL,between origins}, column sep={5.0*\the\DL,between origins}, background color=backgroundColor, ampersand replacement=\&]
            \CatFont{C}
            \arrow[r,"F"]
            \&[-1.5*\the\DL]
            \CatFont{D}
            \arrow[r,"\phi"{name=1}, bend left=30]
            \arrow[r,"\psi"'{name=2},bend right=30]
            \&
            \CatFont{X}\mrp{,}
            % 2-Arrows
            \arrow[from=1,to=2,"\alpha"',xshift=-0.3em,shorten=0.25em,Rightarrow]%
            \arrow[from=1,to=2,"\beta",  xshift=+0.3em,shorten=0.25em,Rightarrow]%
        \end{tikzcd}
    \]%
    if $\alpha$ and $\beta$ are natural isomorphisms and we have
    \[
        \alpha\twocirc\id_{F}%
        =%
        \beta\twocirc\id_{F},%
    \]%
    then $\alpha=\beta$.
\end{remark}
\begin{question}{Characterisation of Functors Corepresentably Faithful on Cores}{characterisation-of-functors-corepresentably-faithful-on-cores}%
    Is there a characterisation of functors corepresentably faithful on cores?
\end{question}
\subsection{Functors Corepresentably Full on Cores}\label{subsection-functors-corepresentably-full-on-cores}
Let $\CatFont{C}$ and $\CatFont{D}$ be categories.
\begin{definition}{Functors Corepresentably Full on Cores}{functors-corepresentably-full-on-cores}%
    A functor $F\colon\CatFont{C}\to\CatFont{D}$ is \index[categories]{functor!corepresentably full on cores}\textbf{corepresentably full on cores} if, for each $X\in\Obj(\Cats)$, the postcomposition by $F$ functor
    \[
        F_{*}%
        \colon%
        \Core(\Fun(\CatFont{X},\CatFont{C}))%
        \to
        \Core(\Fun(\CatFont{X},\CatFont{D}))%
    \]%
    is full.
\end{definition}
\begin{remark}{Unwinding \cref{functors-corepresentably-full-on-cores}}{unwinding-functors-corepresentably-full-on-cores}%
    In detail, a functor $F\colon\CatFont{C}\to\CatFont{D}$ is \textbf{corepresentably full on cores} if, for each $\CatFont{X}\in\Obj(\Cats)$ and each natural isomorphism
    \begin{webcompile}
        \beta%
        \colon%
        \phi\circ F%
        \Longrightisoarrow%
        \psi\circ F,%
        \quad%
        \begin{tikzcd}[row sep={5.0*\the\DL,between origins}, column sep={5.0*\the\DL,between origins}, background color=backgroundColor, ampersand replacement=\&]
            \CatFont{C}
            \arrow[r,"\phi\circ F"{name=1}, bend left=30]
            \arrow[r,"\psi\circ F"'{name=2},bend right=30]
            \&
            \CatFont{X}\mrp{,}
            % 2-Arrows
            \arrow[from=1,to=2,"\beta"',shorten=0.25em,Rightarrow]%
        \end{tikzcd}
    \end{webcompile}
    there exists a natural isomorphism
    \begin{webcompile}
        \alpha%
        \colon%
        \phi%
        \Longrightisoarrow%
        \psi,%
        \quad%
        \begin{tikzcd}[row sep={5.0*\the\DL,between origins}, column sep={5.0*\the\DL,between origins}, background color=backgroundColor, ampersand replacement=\&]
            \CatFont{D}
            \arrow[r,"\phi"{name=1}, bend left=30]
            \arrow[r,"\psi"'{name=2},bend right=30]
            \&
            \CatFont{X}
            % 2-Arrows
            \arrow[from=1,to=2,"\alpha"',shorten=0.25em,Rightarrow]%
        \end{tikzcd}
    \end{webcompile}
    such that we have an equality
    \begin{webcompile}
        \begin{tikzcd}[row sep={5.0*\the\DL,between origins}, column sep={5.0*\the\DL,between origins}, background color=backgroundColor, ampersand replacement=\&]
            \CatFont{X}
            \arrow[r,"\phi"{name=1}, bend left=30]
            \arrow[r,"\psi"'{name=2},bend right=30]
            \&
            \CatFont{C}
            \arrow[r,"F"]
            \&[-1.5*\the\DL]
            \CatFont{D}
            % 2-Arrows
            \arrow[from=1,to=2,"\alpha"',shorten=0.25em,Rightarrow]%
        \end{tikzcd}
        \quad
        \bigequalssign
        \quad
        \begin{tikzcd}[row sep={5.0*\the\DL,between origins}, column sep={5.0*\the\DL,between origins}, background color=backgroundColor, ampersand replacement=\&]
            \CatFont{X}
            \arrow[r,"F\circ\phi"{name=1}, bend left=30]
            \arrow[r,"F\circ\psi"'{name=2},bend right=30]
            \&
            \CatFont{D}
            % 2-Arrows
            \arrow[from=1,to=2,"\beta"',shorten=0.25em,Rightarrow]%
        \end{tikzcd}
    \end{webcompile}
    of pasting diagrams in $\TwoCategoryOfCategories$, i.e.\ such that we have
    \[
        \beta%
        =%
        \alpha\twocirc\id_{F}.%
    \]%
\end{remark}
\begin{question}{Characterisation of Functors Corepresentably Full on Cores}{characterisation-of-functors-corepresentably-full-on-cores}%
    Is there a characterisation of functors corepresentably full on cores?

    This question also appears as \cite{MO468125}.
\end{question}
\subsection{Functors Corepresentably Fully Faithful on Cores}\label{subsection-functors-corepresentably-fully-faithful-on-cores}
Let $\CatFont{C}$ and $\CatFont{D}$ be categories.
\begin{definition}{Functors Corepresentably Fully Faithful on Cores}{functors-corepresentably-fully-faithful-on-cores}%
    A functor $F\colon\CatFont{C}\to\CatFont{D}$ is \index[categories]{functor!corepresentably fully faithful on cores}\textbf{corepresentably fully faithful on cores} if, for each $X\in\Obj(\Cats)$, the postcomposition by $F$ functor
    \[
        F_{*}%
        \colon%
        \Core(\Fun(\CatFont{X},\CatFont{C}))%
        \to
        \Core(\Fun(\CatFont{X},\CatFont{D}))%
    \]%
    is fully faithful.
\end{definition}
\begin{remark}{Unwinding \cref{functors-corepresentably-fully-faithful-on-cores}}{unwinding-functors-corepresentably-fully-faithful-on-cores}%
    In detail, a functor $F\colon\CatFont{C}\to\CatFont{D}$ is \textbf{corepresentably fully faithful on cores} if it satisfies the conditions in \cref{unwinding-functors-corepresentably-faithful-on-cores,unwinding-functors-corepresentably-full-on-cores}, i.e.:
    \begin{enumerate}
        \item\label{unwinding-functors-corepresentably-fully-faithful-on-cores-1}For all diagrams of the form
            \[
                \begin{tikzcd}[row sep={5.0*\the\DL,between origins}, column sep={5.0*\the\DL,between origins}, background color=backgroundColor, ampersand replacement=\&]
                    \CatFont{C}
                    \arrow[r,"F"]
                    \&[-1.5*\the\DL]
                    \CatFont{D}
                    \arrow[r,"\phi"{name=1}, bend left=30]
                    \arrow[r,"\psi"'{name=2},bend right=30]
                    \&
                    \CatFont{X}\mrp{,}
                    % 2-Arrows
                    \arrow[from=1,to=2,"\alpha"',xshift=-0.3em,shorten=0.25em,Rightarrow]%
                    \arrow[from=1,to=2,"\beta",  xshift=+0.3em,shorten=0.25em,Rightarrow]%
                \end{tikzcd}
            \]%
            if $\alpha$ and $\beta$ are natural isomorphisms and we have
            \[
                \alpha\twocirc\id_{F}%
                =%
                \beta\twocirc\id_{F},%
            \]%
            then $\alpha=\beta$.
        \item\label{unwinding-functors-corepresentably-fully-faithful-on-cores-2}For each $\CatFont{X}\in\Obj(\Cats)$ and each natural isomorphism
            \begin{webcompile}
                \beta%
                \colon%
                \phi\circ F%
                \Longrightisoarrow%
                \psi\circ F,%
                \quad%
                \begin{tikzcd}[row sep={5.0*\the\DL,between origins}, column sep={5.0*\the\DL,between origins}, background color=backgroundColor, ampersand replacement=\&]
                    \CatFont{C}
                    \arrow[r,"\phi\circ F"{name=1}, bend left=30]
                    \arrow[r,"\psi\circ F"'{name=2},bend right=30]
                    \&
                    \CatFont{X}\mrp{,}
                    % 2-Arrows
                    \arrow[from=1,to=2,"\beta"',shorten=0.25em,Rightarrow]%
                \end{tikzcd}
            \end{webcompile}
            there exists a natural isomorphism
            \begin{webcompile}
                \alpha%
                \colon%
                \phi%
                \Longrightisoarrow%
                \psi,%
                \quad%
                \begin{tikzcd}[row sep={5.0*\the\DL,between origins}, column sep={5.0*\the\DL,between origins}, background color=backgroundColor, ampersand replacement=\&]
                    \CatFont{D}
                    \arrow[r,"\phi"{name=1}, bend left=30]
                    \arrow[r,"\psi"'{name=2},bend right=30]
                    \&
                    \CatFont{X}
                    % 2-Arrows
                    \arrow[from=1,to=2,"\alpha"',shorten=0.25em,Rightarrow]%
                \end{tikzcd}
            \end{webcompile}
            such that we have an equality
            \begin{webcompile}
                \begin{tikzcd}[row sep={5.0*\the\DL,between origins}, column sep={5.0*\the\DL,between origins}, background color=backgroundColor, ampersand replacement=\&]
                    \CatFont{X}
                    \arrow[r,"\phi"{name=1}, bend left=30]
                    \arrow[r,"\psi"'{name=2},bend right=30]
                    \&
                    \CatFont{C}
                    \arrow[r,"F"]
                    \&[-1.5*\the\DL]
                    \CatFont{D}
                    % 2-Arrows
                    \arrow[from=1,to=2,"\alpha"',shorten=0.25em,Rightarrow]%
                \end{tikzcd}
                \quad
                \bigequalssign
                \quad
                \begin{tikzcd}[row sep={5.0*\the\DL,between origins}, column sep={5.0*\the\DL,between origins}, background color=backgroundColor, ampersand replacement=\&]
                    \CatFont{X}
                    \arrow[r,"F\circ\phi"{name=1}, bend left=30]
                    \arrow[r,"F\circ\psi"'{name=2},bend right=30]
                    \&
                    \CatFont{D}
                    % 2-Arrows
                    \arrow[from=1,to=2,"\beta"',shorten=0.25em,Rightarrow]%
                \end{tikzcd}
            \end{webcompile}
            of pasting diagrams in $\TwoCategoryOfCategories$, i.e.\ such that we have
            \[
                \beta%
                =%
                \alpha\twocirc\id_{F}.%
            \]%
    \end{enumerate}
\end{remark}
\begin{question}{Characterisation of Functors Corepresentably Fully Faithful on Cores}{characterisation-of-functors-corepresentably-fully-faithful-on-cores}%
    Is there a characterisation of functors corepresentably fully faithful on cores?
\end{question}
\section{Natural Transformations}\label{section-natural-transformations}
\subsection{Transformations}\label{subsection-natural-transformations-transformations}
Let $\CatFont{C}$ and $\CatFont{D}$ be categories and let $F,G\colon\CatFont{C}\rightrightarrows\CatFont{D}$ be functors.
\begin{definition}{Transformations}{transformations-between-functors}%
    A \index[categories]{transformation between functors}\index[categories]{functor!transformation of}\textbf{transformation}%
    %--- Begin Footnote ---%
    \footnote{%
        \SloganFont{Further Terminology: }Also called an \textbf{unnatural transformation} for emphasis.
        \par\vspace*{\TCBBoxCorrection}
    } %
    %---  End Footnote  ---%
    \textbf{$\smash{\alpha\colon F\Rightarrow G}$ from $F$ to $G$} is a collection%
    \[
        \{%
            \alpha_{A}%
            \colon%
            F(A)%
            \to%
            G(A)%
        \}_{A\in\Obj(\CatFont{C})}%
    \]%
    of morphisms of $\CatFont{D}$.
\end{definition}
\begin{notation}{The Set of Transformations Between Two Functors}{the-set-of-transformations-between-two-functors}%
    We write \index[notation]{TransFG@$\Trans(F,G)$}$\Trans(F,G)$ for the set of transformations from $F$ to $G$.
\end{notation}
\begin{remark}{The Set of Transformations as a Product}{the-set-of-transformations-as-a-product}%
    We have an isomorphism
    \[
        \Trans(F,G)%
        \cong%
        \prod_{A\in\CatFont{C}}\Hom_{\CatFont{D}}(F_{A},G_{A}).%
    \]%
\end{remark}
\begin{Proof}{Proof of \cref{the-set-of-transformations-as-a-product}}%
    Omitted.
\end{Proof}
\subsection{Natural Transformations}\label{subsection-natural-transformations-natural-transformations}
Let $\CatFont{C}$ and $\CatFont{D}$ be categories and $F,G\colon\CatFont{C}\rightrightarrows\CatFont{D}$ be functors.
\begin{definition}{Natural Transformations}{natural-transformations}%
    A \index[categories]{natural transformation}\textbf{natural transformation $\alpha\colon F\Rightarrow G$ from $F$ to $G$} is a transformation%
    \[
        \{%
            \alpha_{A}%
            \colon%
            F(A)%
            \to%
            G(A)%
        \}_{A\in\Obj(\CatFont{C})}%
    \]%
    from $F$ to $G$ such that, for each morphism $f\colon A\to B$ of $\CatFont{C}$, the diagram
    \[
        \begin{tikzcd}[row sep={5.0*\the\DL,between origins}, column sep={6.0*\the\DL,between origins}, background color=backgroundColor, ampersand replacement=\&]
            F(A)
            \arrow[r, "{F(f)}"]
            \arrow[d, "\alpha_{A}"']
            \&
            F(B)
            \arrow[d, "\alpha_{B}"]
            \\
            G(A)
            \arrow[r, "{G(f)}"']
            \&
            G(B)
        \end{tikzcd}
    \]%
    commutes.%
\end{definition}
\begin{remark}{Further Terminology and Notation for Natural Transformations}{further-terminology-and-notation-for-natural-transformations}%
    Let $\alpha\colon F\Rightarrow G$ be a natural transformation.
    \begin{enumerate}
        \item\label{further-terminology-and-notation-for-natural-transformations-1}For each $A\in\Obj(\CatFont{C})$, the morphism $\alpha_{A}\colon F_{A}\to G_{A}$ is called the \index[categories]{natural transformation!component of}\textbf{component of $\alpha$ at $A$}.
        \item\label{further-terminology-and-notation-for-natural-transformations-2}We denote natural transformations such as $\alpha$ in diagrams as
            \[
                \begin{tikzcd}[row sep={5.0*\the\DL,between origins}, column sep={5.0*\the\DL,between origins}, background color=backgroundColor, ampersand replacement=\&]
                    \CatFont{C}
                    \arrow[r, "F"{name=F}, bend left=30]
                    \arrow[r, "G"'{name=G}, bend right=30]
                    \&
                    \CatFont{D}\mrp{.}
                    %--- Adjunction Symbol
                    \arrow[from=F, to=G, Rightarrow, shorten=0.25em, "\alpha"'{pos=0.45}]
                \end{tikzcd}
            \]%
    \end{enumerate}
\end{remark}
\begin{notation}{The Set of Natural Transformations Between Two Functors}{the-set-of-natural-transformations-between-two-functors}%
    We write \index[notation]{NatFG@$\Nat(F,G)$}$\Nat(F,G)$ for the set of natural transformations from $F$ to $G$.
\end{notation}
\begin{definition}{Equality of Natural Transformations}{equality-of-natural-transformations}%
    Two natural transformations $\alpha,\beta\colon F\Rightarrow G$ are \index[categories]{natural transformation!equality of}\textbf{equal} if we have
    \[\alpha_{A}=\beta_{A}\]%
    for each $A\in\Obj(\CatFont{C})$.
\end{definition}
\subsection{Examples of Natural Transformations}\label{subsection-examples-of-natural-transformations}
\begin{example}{Identity Natural Transformations}{identity-natural-transformations}%
    The \index[categories]{natural transformation!identity natural transformation}\textbf{identity natural transformation} \index[notation]{idF@$\id_{F}$}$\id_{F}\colon F\Rightarrow F$ \textbf{of $F$} is the natural transformation consisting of the collection
    \[
        \{%
            (\id_{F})_{A}%
            \colon%
            F(A)%
            \to%
            F(A)%
        \}_{A\in\Obj(\CatFont{C})}
    \]
    defined by
    \[
        (\id_{F})_{A}%
        \defeq%
        \id_{F(A)}%
    \]%
    for each $A\in\Obj(\CatFont{C})$.
\end{example}
\begin{Proof}{Proof of \cref{identity-natural-transformations}}%
    The naturality condition for $\id_{F}$ is the requirement that, for each morphism $f\colon A\to B$ of $\CatFont{C}$, the diagram
    \[
        \begin{tikzcd}[row sep={5.0*\the\DL,between origins}, column sep={5.5*\the\DL,between origins}, background color=backgroundColor, ampersand replacement=\&]
            {F(A)}
            \arrow[r, "{F(f)}"]
            \arrow[d, "\id_{F(A)}"']
            \&
            {F(B)}
            \arrow[d, "\id_{F(B)}"]
            \\
            {F(A)}
            \arrow[r, "{F(f)}"']
            \&
            {F(B)}
        \end{tikzcd}
    \]%
    commutes. This follows from unitality of the composition of $\CatFont{D}$, as we have
    \begin{align*}
        F(f)\circ\id_{F(A)} &= F(f)\\
                            &= \id_{F(B)}\circ F(f),\\
    \end{align*}
    where we have applied unitality twice.
\end{Proof}
\begin{example}{Natural Transformations Between Morphisms of Monoids}{natural-transformations-between-morphisms-of-monoids}%
    Let $A$ and $B$ be monoids and let $f,g\colon A\rightrightarrows B$ be morphisms of monoids. Applying the delooping construction of \cref{TODO}, we obtain functors $\B{f},\B{g}\colon\B{A}\rightrightarrows\B{B}$. We then have
    \[
        \Nat(\B{f},\B{g})%
        \cong%
        \{%
            b\in B%
            \ \middle|\ %
            \begin{aligned}
                &\text{for each $a\in A$, we}\\%
                &\text{have $bf(a)=g(a)b$}%
            \end{aligned}
        \}.%
    \]%
\end{example}
\begin{Proof}{Proof of \cref{natural-transformations-between-morphisms-of-monoids}}%
    Unwinding the definitions in this case, we see that a transformation $\alpha$ from $\B{f}$ to $\B{g}$ consists of a collection
    \[
        \{%
            \alpha_{\bullet}%
            \colon%
            \bullet%
            \to%
            \bullet%
        \}_{\bullet\in\Obj(\B{A})}%
    \]%
    of morphisms of $\B{B}$ indexed by $\Obj(\B{A})$. Since $\Obj(\B{A})=\pt$ and the morphisms of $\B{B}$ are precisely the elements of $B$, it follows that $\alpha$ corresponds precisely to the data of an element $b\in B$. Now, a transformation $[b]\colon\B{f}\Rightarrow\B{g}$ is natural precisely if, for each $a\in\Hom_{\B{A}}(\bullet,\bullet)\defeq A$, the diagram
    \[
        \begin{tikzcd}[row sep={5.0*\the\DL,between origins}, column sep={6.5*\the\DL,between origins}, background color=backgroundColor, ampersand replacement=\&]
            \B{f}(\bullet)
            \arrow[r,"{\B{f}(a)}"]
            \arrow[d,"{[b]_{\bullet}}"']
            \&
            \B{f}(\bullet)
            \arrow[d,"{[b]_{\bullet}}"]
            \\
            \B{g}(\bullet)
            \arrow[r,"{\B{g}(a)}"']
            \&
            \B{g}(\bullet)
        \end{tikzcd}
    \]%
    commutes. Unwinding the definitions, we see that this diagram is given by
    \[
        \begin{tikzcd}[row sep={5.0*\the\DL,between origins}, column sep={5.0*\the\DL,between origins}, background color=backgroundColor, ampersand replacement=\&]
            \bullet
            \arrow[r,"{f(a)}"]
            \arrow[d,"b"']
            \&
            \bullet
            \arrow[d,"b"]
            \\
            \bullet
            \arrow[r,"{g(a)}"']
            \&
            \bullet\mrp{,}
        \end{tikzcd}
    \]%
    and hence corresponds precisely to the condition $g(a)b=bf(a)$.
\end{Proof}
\subsection{Vertical Composition of Natural Transformations}\label{subsection-vertical-composition-of-natural-transformations}
\begin{definition}{Vertical Composition of Natural Transformations}{vertical-composition-of-natural-transformations}%
    The \index[categories]{natural transformation!vertical composition}\textbf{vertical composition} of two natural transformations $\alpha\colon F\Longrightarrow G$ and $\beta\colon G\Longrightarrow H$ as in the diagram%
    \[
        \begin{tikzcd}[row sep={6.0*\the\DL,between origins}, column sep={6.0*\the\DL,between origins}, background color=backgroundColor, ampersand replacement=\&]
            \CatFont{C}
            \arrow[r, bend left=60,  "F"{pos=0.49},""{name=1,pos=0.49}]
            \arrow[r, bend right=0,  ""{name=2,description},"G"'description]
            \arrow[r, bend right=60, "H"'{pos=0.49},""'{name=3,pos=0.49}]
            \&
            \CatFont{D}
            %--- 2-Arrows
            \arrow[Rightarrow, shorten=0.25em, from=1, to=2, "\alpha"']
            \arrow[Rightarrow, shorten=0.25em, from=2, to=3, "\beta"'{pos=0.45}]
        \end{tikzcd}
    \]%
    is the natural transformation \index[notation]{betaafteralpha@$\beta\circ\alpha$}$\beta\circ\alpha\colon F\Longrightarrow H$ consisting of the collection
    \[
        \{
            (\beta\circ\alpha)_{A}
            \colon
            F(A)
            \to
            H(A)
        \}_{A\in\Obj(\CatFont{C})}
    \]%
    with
    \[
        (\beta\circ\alpha)_{A}
        \defeq
        \beta_{A}\circ\alpha_{A}
    \]
    for each $A\in\Obj(\CatFont{C})$.
\end{definition}
\begin{Proof}{Proof of \cref{vertical-composition-of-natural-transformations}}%
    The naturality condition for $\beta\circ\alpha$ is the requirement that the boundary of the diagram
    \[
        \begin{tikzcd}[row sep={5.0*\the\DL,between origins}, column sep={7.5*\the\DL,between origins}, background color=backgroundColor, ampersand replacement=\&]
            F(A)
            \arrow[r, "{F(f)}"]
            \arrow[d, "\alpha_{A}"']
            \&
            F(B)
            \arrow[d, "\alpha_{B}"]
            \\
            G(A)
            \arrow[r, "{G(f)}"'description]
            \arrow[d, "\beta_{A}"']
            \&
            G(B)
            \arrow[d, "\beta_{B}"]
            \\
            H(A)
            \arrow[r, "{H(f)}"']
            \&
            H(B)
            % Subdiagrams
            \arrow[from=1-1,to=2-2, phantom, "\scriptstyle(1)"]
            \arrow[from=2-1,to=3-2, phantom, "\scriptstyle(2)"]
        \end{tikzcd}
    \]%
    commutes. Since
    \begin{itemize}
        \item Subdiagram (1) commutes by the naturality of $\alpha$.
        \item Subdiagram (2) commutes by the naturality of $\beta$.
    \end{itemize}
    so does the boundary diagram. Hence $\beta\circ\alpha$ is a natural transformation.
\end{Proof}
\begin{proposition}{Properties of Vertical Composition of Natural Transformations}{properties-of-vertical-composition-of-natural-transformations}%
    Let $\CatFont{C}$, $\CatFont{D}$, and $\CatFont{E}$ be categories.
    \begin{enumerate}
        \item\label{properties-of-vertical-composition-of-natural-transformations-functionality}\SloganFont{Functionality. }The assignment $(\beta,\alpha)\mapsto\beta\circ\alpha$ defines a function
            \[
                \circ_{F,G,H}%
                \colon%
                \Nat(G,H)%
                \times
                \Nat(F,G)%
                \to%
                \Nat(F,H).%
            \]%
        \item\label{properties-of-vertical-composition-of-natural-transformations-associativity}\SloganFont{Associativity. }Let $F,G,H,K\colon\CatFont{C}\rightrightrightrightarrows\CatFont{D}$ be functors. The diagram
            \begin{scalemath}
                \begin{tikzcd}[row sep={0*\the\DL,between origins}, column sep={0*\the\DL,between origins}, background color=backgroundColor, ampersand replacement=\&]
                    \&[0.30901699437\ThreeCmPlusHalf]
                    \&[0.5\ThreeCmPlusHalf]
                    \Nat(H,K)\times(\Nat(G,H)\times\Nat(F,G))
                    \&[0.5\ThreeCmPlusHalf]
                    \&[0.30901699437\ThreeCmPlusHalf]
                    \\[0.58778525229\ThreeCmPlusHalf]
                    (\Nat(H,K)\times\Nat(G,H))\times\Nat(F,G)
                    \&[0.30901699437\ThreeCmPlusHalf]
                    \&[0.5\ThreeCmPlusHalf]
                    \&[0.5\ThreeCmPlusHalf]
                    \&[0.30901699437\ThreeCmPlusHalf]
                    \Nat(H,K)\times\Nat(F,H)
                    \\[0.95105651629\ThreeCmPlusHalf]
                    \&[0.30901699437\ThreeCmPlusHalf]
                    \Nat(G,K)\times\Nat(F,G)
                    \&[0.5\ThreeCmPlusHalf]
                    \&[0.5\ThreeCmPlusHalf]
                    \Nat(F,K)
                    \&[0.30901699437\ThreeCmPlusHalf]
                    % 1-Arrows
                    % Left Boundary
                    \arrow[from=2-1,to=1-3,"\alpha^{\Sets}_{\Nat(H,K),\Nat(G,H),\Nat(F,G)}"{pos=0.4125},isoarrowprime]%
                    \arrow[from=1-3,to=2-5,"\id_{\Nat(H,K)}\times\circ_{F,G,H}"{pos=0.6}]%
                    \arrow[from=2-5,to=3-4,"\circ_{F,H,K}"{pos=0.425}]%
                    % Right Boundary
                    \arrow[from=2-1,to=3-2,"\circ_{G,H,K}\times\id_{\Nat(F,G)}"'{pos=0.425}]%
                    \arrow[from=3-2,to=3-4,"\circ_{F,G,K}"']%
                \end{tikzcd}
            \end{scalemath}%
            commutes, i.e.\ given natural transformations
            \[
                F\xLongrightarrow{\alpha}G\xLongrightarrow{\beta}H\xLongrightarrow{\gamma}K,%
            \]%
            we have
            \[
                (\gamma\circ\beta)\circ\alpha%
                =%
                \gamma\circ(\beta\circ\alpha).%
            \]%
        \item\label{properties-of-vertical-composition-of-natural-transformations-unitality}\SloganFont{Unitality. }Let $F,G\colon\CatFont{C}\rightrightarrows\CatFont{D}$ be functors.
            \begin{enumerate}
                \item\SloganFont{Left Unitality. }The diagram
                    \[
                        \begin{tikzcd}[row sep={5.0*\the\DL,between origins}, column sep={11.5*\the\DL,between origins}, background color=backgroundColor, ampersand replacement=\&]
                            \pt\times\Nat(F,G)
                            \arrow[rd, "\LUnitor^{\Sets}_{\Nat(F,G)}"{pos=0.4},isoarrowprime]
                            \arrow[d, "{[\id_{G}]\times\id_{\Nat(F,G)}}"']
                            \&\\
                            \Nat(G,G)\times\Nat(F,G)
                            \arrow[r, "\circ_{F,G,G}"']
                            \&
                            \Nat(F,G)
                        \end{tikzcd}
                    \]%
                    commutes, i.e.\ given a natural transformation $\alpha\colon F\Longrightarrow G$, we have
                    \[
                        \id_{G}\circ\alpha%
                        =%
                        \alpha.%
                    \]%
                \item\SloganFont{Right Unitality. }The diagram
                    \[
                        \begin{tikzcd}[row sep={5.0*\the\DL,between origins}, column sep={11.5*\the\DL,between origins}, background color=backgroundColor, ampersand replacement=\&]
                            \Nat(F,G)\times\pt
                            \arrow[rd, "\RUnitor^{\Sets}_{\Nat(F,G)}"{pos=0.4},isoarrowprime]
                            \arrow[d, "{\id_{\Nat(F,G)}\times[\id_{F}]}"']
                            \&\\
                            \Nat(F,G)\times\Nat(F,F)
                            \arrow[r, "\circ^{\CatFont{C}}_{F,F,G}"']
                            \&
                            \Nat(F,G)
                        \end{tikzcd}
                    \]%
                    commutes, i.e.\ given a natural transformation $\alpha\colon F\Longrightarrow G$, we have
                    \[
                        \alpha\circ\id_{F}%
                        =%
                        \alpha.
                    \]%
            \end{enumerate}
        \item\label{properties-of-vertical-composition-of-natural-transformations-middle-four-exchange}\SloganFont{Middle Four Exchange. }\index[categories]{middle four exchange!in Cats@in $\Cats$}Let $F_{1},F_{2},F_{3}\colon\CatFont{C}\to\CatFont{D}$ and $G_{1},G_{2},G_{3}\colon\CatFont{D}\to\CatFont{E}$ be functors. The diagram
            \begin{scalemath}
                \begin{tikzcd}[row sep={10.0*\the\DL,between origins}, column sep={15.0*\the\DL,between origins}, background color=backgroundColor, ampersand replacement=\&]
                    (\Nat(G_{2},G_{3})\times\Nat(G_{1},G_{2}))\times(\Nat(F_{2},F_{3})\times\Nat(F_{1},F_{2}))
                    \arrow[rr,leftrightarrow,dashed,"\mu_{4}","\scalebox{1.5}{$\sim$}"'{sloped,outer sep=-0.0em}]
                    \arrow[d,"\circ_{G_{1},G_{2},G_{3}}\times\circ_{F_{1},F_{2},F_{3}}"']
                    \&
                    {}
                    \&
                    (\Nat(G_{2},G_{3})\times\Nat(F_{2},F_{3}))\times(\Nat(G_{1},G_{2})\times\Nat(F_{1},F_{2}))
                    \arrow[d,"\twocirc_{F_{2},F_{3},G_{2},G_{3}}\times\twocirc_{F_{1},F_{2},G_{1},G_{2}}"]
                    \\
                    \Nat(G_{1},G_{3})\times\Nat(F_{1},F_{3})
                    \arrow[rd,"\twocirc_{F_{1},F_{3},G_{1},G_{3}}"']
                    \&
                    {}
                    \&
                    \Nat(G_{2}\circ F_{2},G_{3}\circ F_{3})\times\Nat(G_{1}\circ F_{1},G_{2}\circ F_{2})
                    \arrow[ld,"\circ_{G_{1}\circ F_{1},G_{2}\circ F_{2},G_{3}\circ F_{3}}"]
                    \\[-2.0*\the\DL]
                    {}
                    \&
                    \Nat(G_{1}\circ F_{1},G_{3}\circ F_{3})
                    \&
                    {}
                \end{tikzcd}%
            \end{scalemath}
            commutes, i.e.\ given a diagram
            \[
                \begin{tikzcd}[row sep={7.0*\the\DL,between origins}, column sep={6.0*\the\DL,between origins}, background color=backgroundColor, ampersand replacement=\&]
                    \CatFont{C}
                    \arrow[r,"F_{1}",""{name=f1,pos=0.49},    bend left  = 60]
                    \arrow[r,"F_{2}"{name=f2,description}]
                    \arrow[r,"F_{3}"',""'{name=f3,pos=0.49},bend right = 60]
                    \&
                    \CatFont{D}
                    \arrow[r,"G_{1}",""{name=g1,pos=0.51},    bend left  = 60]
                    \arrow[r,"G_{2}"{name=g2,description}]
                    \arrow[r,"G_{3}"',""'{name=g3,pos=0.52},bend right = 60]
                    \&
                    \CatFont{E}
                    %--- 2-Arrows
                    \arrow[Rightarrow, shorten <= 0.25em, from=f1, to=f2, pos=0.475, "\alpha"'{pos=0.55}]
                    \arrow[Rightarrow, shorten >= 0.25em, from=f2, to=f3, pos=0.475, "\alpha'"'{pos=0.25}]
                    \arrow[Rightarrow, shorten <= 0.25em, from=g1, to=g2, pos=0.475, "\beta"'{pos=0.55}]
                    \arrow[Rightarrow, shorten >= 0.25em, from=g2, to=g3, pos=0.475, "\beta'"'{pos=0.35}]
                \end{tikzcd}
            \]%
            in $\TwoCategoryOfCategories$, we have
            \[
                (\beta'\twocirc\alpha')\circ(\beta\twocirc\alpha)%
                =%
                (\beta'\circ\beta)\twocirc(\alpha'\circ\alpha).%
            \]%
        %\item\label{properties-of-vertical-composition-of-natural-transformations-}\SloganFont{. }
    \end{enumerate}
\end{proposition}
\begin{Proof}{Proof of \cref{properties-of-vertical-composition-of-natural-transformations}}%
    \FirstProofBox{\cref{properties-of-vertical-composition-of-natural-transformations-functionality}: Functionality}%
    Omitted.

    \ProofBox{\cref{properties-of-vertical-composition-of-natural-transformations-associativity}: Associativity}%
    Indeed, we have
    \begin{align*}
        ((\gamma\circ\beta)\circ\alpha)_{A} &\defeq (\gamma\circ\beta)_{A}\circ\alpha_{A}\\
                                            &\defeq (\gamma_{A}\circ\beta_{A})\circ\alpha_{A}\\
                                            &=      \gamma_{A}\circ(\beta_{A}\circ\alpha_{A})\\
                                            &\defeq \gamma_{A}\circ(\beta\circ\alpha)_{A}\\
                                            &\defeq (\gamma\circ(\beta\circ\alpha))_{A}
    \end{align*}
    for each $A\in\Obj(\CatFont{C})$, showing the desired equality.

    \ProofBox{\cref{properties-of-vertical-composition-of-natural-transformations-unitality}: Unitality}%
    We have
    \begin{align*}
        (\id_{G}\circ\alpha)_{A} &= \id_{G}\circ\alpha_{A}\\
                                 &= \alpha_{A},\\
        (\alpha\circ\id_{F})_{A} &= \alpha_{A}\circ\id_{F}\\
                                 &= \alpha_{A}
    \end{align*}
    for each $A\in\Obj(\CatFont{C})$, showing the desired equality.

    \ProofBox{\cref{properties-of-vertical-composition-of-natural-transformations-middle-four-exchange}: Middle Four Exchange}%
    This is proved in \cref{properties-of-horizontal-composition-of-natural-transformations-middle-four-exchange} of \cref{properties-of-horizontal-composition-of-natural-transformations}.
\end{Proof}
\subsection{Horizontal Composition of Natural Transformations}\label{subsection-horizontal-composition-properties-of-natural-transformations}
\begin{definition}{Horizontal Composition of Natural Transformations}{horizontal-composition-of-two-natural-transformations}%
    The \index[categories]{natural transformation!horizontal composition}\textbf{horizontal composition}%
    %--- Begin Footnote ---%
    \footnote{%
        \SloganFont{Further Terminology: }Also called the \index[categories]{Godement product|see {natural transformation, horizontal composition}}\textbf{Godement product} of $\alpha$ and $\beta$.
    }%
    %---  End Footnote  ---%
    %--- Begin Footnote ---%
    \footnote{%
        Horizontal composition forms a map
        \[
            \twocirc_{(F,H),(G,K)}%
            \colon%
            \Nat(H,K)%
            \times
            \Nat(F,G)%
            \to%
            \Nat(H\circ F,K\circ G).%
        \]%
        \par\vspace*{\TCBBoxCorrection}
    } %
    %---  End Footnote  ---%
    of two natural transformations $\alpha\colon F\Longrightarrow G$ and $\beta\colon H\Longrightarrow K$ as in the diagram
    \[
        \begin{tikzcd}[row sep={5.0*\the\DL,between origins}, column sep={5.0*\the\DL,between origins}, background color=backgroundColor, ampersand replacement=\&]
            \CatFont{C}
            \arrow[r,"F"{name=F1} ,bend left  = 30]
            \arrow[r,"G"'{name=G1},bend right = 30]
            \&
            \CatFont{D}
            \arrow[r,"H"{name=F2} ,bend left  = 30]
            \arrow[r,"K"'{name=G2},bend right = 30]
            \&
            \CatFont{E}
            %--- 2-Arrows
            \arrow[Rightarrow,shorten=0.25em,from=F1,to=G1,"\alpha"'{pos=0.475}]
            \arrow[Rightarrow,shorten=0.25em,from=F2,to=G2,"\beta"']
        \end{tikzcd}
    \]%
    of $\alpha$ and $\beta$ is the natural transformation\index[notation]{betastaralpha@$\beta\twocirc\alpha$}%
    \[\beta\twocirc\alpha\colon(H\circ F)\Longrightarrow(K\circ G),\]
    as in the diagram
    \[
        \begin{tikzcd}[row sep={6.0*\the\DL,between origins}, column sep={6.0*\the\DL,between origins}, background color=backgroundColor, ampersand replacement=\&]
            \CatFont{C}
            \arrow[r,"H\circ F"{name=F1}, bend left  = 45]
            \arrow[r,"K\circ G"'{name=G1},bend right = 45]
            \&
            \CatFont{E}\mathrlap{,}
            %--- 2-Arrows
            \arrow[Rightarrow, shorten=0.25em, from=F1, to=G1, "\beta\twocirc\alpha"'description,pos=0.475]
        \end{tikzcd}
    \]%
    consisting of the collection
    \[
        \{
            (\beta\twocirc\alpha)_{A}
            \colon
            H(F(A))
            \to
            K(G(A))
        \}_{A\in\Obj(\CatFont{C})},
    \]%
    of morphisms of $\CatFont{E}$ with
    \begin{webcompile}
        \begin{aligned}
            (\beta\twocirc\alpha)_{A} &\defeq \beta_{G(A)}\circ H(\alpha_{A})\\
                                      &=      K(\alpha_{A})\circ\beta_{F(A)}\mrp{,}
        \end{aligned}
        \qquad
        \begin{tikzcd}[row sep={5.0*\the\DL,between origins}, column sep={8.0*\the\DL,between origins}, background color=backgroundColor, ampersand replacement=\&]
            H(F(A))
            \arrow[r, "{H(\alpha_{A})}"]
            \arrow[d, "\beta_{F(A)}"']
            \&
            H(G(A))
            \arrow[d, "\beta_{G(A)}"]
            \\
            K(F(A))
            \arrow[r, "{K(\alpha_{A})}"']
            \&
            K(G(A))\mrp{.}
        \end{tikzcd}
    \end{webcompile}%
\end{definition}
\begin{Proof}{Proof of \cref{horizontal-composition-of-two-natural-transformations}}%
    First, we claim that we indeed have
    \begin{webcompile}
        \beta_{G(A)}\circ H(\alpha_{A})%
        =%
        K(\alpha_{A})\circ\beta_{F(A)},%
        \quad%
        \begin{tikzcd}[row sep={5.0*\the\DL,between origins}, column sep={8.0*\the\DL,between origins}, background color=backgroundColor, ampersand replacement=\&]
            H(F(A))
            \arrow[r, "{H(\alpha_{A})}"]
            \arrow[d, "\beta_{F(A)}"']
            \&
            H(G(A))
            \arrow[d, "\beta_{G(A)}"]
            \\
            K(F(A))
            \arrow[r, "{K(\alpha_{A})}"']
            \&
            K(G(A))\mrp{.}
        \end{tikzcd}
    \end{webcompile}%
    This is, however, simply the naturality square for $\beta$ applied to the morphism $\alpha_{A}\colon F(A)\to G(A)$. Next, we check the naturality condition for $\beta\twocirc\alpha$, which is the requirement that the boundary of the diagram
    \[
        \begin{tikzcd}[row sep={6.5*\the\DL,between origins}, column sep={11.0*\the\DL,between origins}, background color=backgroundColor, ampersand replacement=\&]
            H(F(A))
            \arrow[r, "{H(F(f))}"]
            \arrow[d, "{H(\alpha_{A})}"']
            \&
            H(F(B))
            \arrow[d, "{H(\alpha_{B})}"]
            \\
            H(G(A))
            \arrow[r, "{H(G(f))}"'description]
            \arrow[d, "\beta_{G(A)}"']
            \&
            H(G(B))
            \arrow[d, "\beta_{G(B)}"]
            \\
            K(G(A))
            \arrow[r, "{K(G(f))}"']
            \&
            K(G(B))
            % Subdiagrams
            \arrow[from=1-1,to=2-2, phantom, "\scriptstyle(1)"]
            \arrow[from=2-1,to=3-2, phantom, "\scriptstyle(2)"]
        \end{tikzcd}
    \]%
    commutes. Since
    \begin{itemize}
        \item Subdiagram (1) commutes by the naturality of $\alpha$.
        \item Subdiagram (2) commutes by the naturality of $\beta$.
    \end{itemize}
    so does the boundary diagram. Hence $\beta\circ\alpha$ is a natural transformation.%
    %--- Begin Footnote ---%
    \footnote{%
        \SloganFont{Reference: }\cite[Proposition 1.3.4]{borceux1994handbook1}.
        \par\vspace*{\TCBBoxCorrection}
    }%
    %---  End Footnote  ---%
\end{Proof}
\begin{definition}{Whiskering of Functors With Natural Transformations}{whiskering-of-functors-with-natural-transformations}%
    Let
    \[
        \begin{tikzcd}[row sep={4.0*\the\DL,between origins}, column sep={4.0*\the\DL,between origins}, background color=backgroundColor, ampersand replacement=\&]
            \CatFont{X}
            \arrow[r,"F"]
            \&[-1.0*\the\DL]
            \CatFont{C}
            \arrow[r,"\phi"{name=phi}, bend left =30]
            \arrow[r,"\psi"'{name=psi},bend right=30]
            \&
            \mrp{\CatFont{D}}\phantom{\CatFont{C}}
            \arrow[r,"G"]
            \&[-1.0*\the\DL]
            \CatFont{Y}
            % 2-arrows
            \arrow[from=phi,to=psi,"\alpha"'{pos=0.475},shorten=0.25em,Rightarrow]%
        \end{tikzcd}
    \]%
    be a diagram in $\TwoCategoryOfCategories$.
    \begin{enumerate}
        \item\label{whiskering-of-functors-with-natural-transformations-left-whiskering}The \index[categories]{whiskering!left}\textbf{left whiskering of $\alpha$ with $G$} is the natural transformation\index[notation]{idFstaralpha@$\id_{F}\star\alpha$}%
            %--- Begin Footnote ---%
            \footnote{%
                \SloganFont{Further Notation: }Also written \index[notation]{Falpha@$F\alpha$}$G\alpha$ or \index[notation]{Fstaralpha@$F\twocirc\alpha$}$G\twocirc\alpha$, although we won't use either of these notations in this work.
            }%
            %---  End Footnote  ---%
            \[
                \id_{G}\star\alpha%
                \colon%
                G\circ\phi%
                \Longrightarrow%
                G\circ\psi.%
            \]%
        \item\label{whiskering-of-functors-with-natural-transformations-right-whiskering}The \index[categories]{whiskering!right}\textbf{right whiskering of $\alpha$ with $F$} is the natural transformation\index[notation]{alphastarF@$\alpha\star F$}%
            %--- Begin Footnote ---%
            \footnote{%
                \SloganFont{Further Notation: }Also written \index[notation]{alphaF@$\alpha F$}$\alpha F$ or \index[notation]{Fstaralpha@$\alpha\twocirc F$}$\alpha\twocirc F$, although we won't use either of these notations in this work.
                \par\vspace*{\TCBBoxCorrection}
            }%
            %---  End Footnote  ---%
            \[
                \alpha\star\id_{F}%
                \colon%
                \phi\circ F%
                \Longrightarrow%
                \psi\circ F.%
            \]%
    \end{enumerate}
\end{definition}
\begin{proposition}{Properties of Horizontal Composition of Natural Transformations}{properties-of-horizontal-composition-of-natural-transformations}%
    Let $\CatFont{C}$, $\CatFont{D}$, and $\CatFont{E}$ be categories.
    \begin{enumerate}
        \item\label{properties-of-horizontal-composition-of-natural-transformations-functionality}\SloganFont{Functionality. }The assignment $(\beta,\alpha)\mapsto\beta\twocirc\alpha$ defines a function
            \[
                \twocirc_{(F,G),(H,K)}%
                \colon%
                \Nat(H,K)%
                \times
                \Nat(F,G)%
                \to%
                \Nat(H\circ F,K\circ G).%
            \]%
        \item\label{properties-of-horizontal-composition-of-natural-transformations-associativity}\SloganFont{Associativity. }Let
            \[
                \begin{tikzcd}[row sep={3.0*\the\DL,between origins}, column sep={3.0*\the\DL,between origins}, background color=backgroundColor, ampersand replacement=\&]
                    \CatFont{C}
                    \arrow[r,"F_{1}"{name=F1} ,shift left  = 1.0]
                    \arrow[r,"G_{1}"'{name=G1},shift right = 1.0]
                    \&
                    \CatFont{D}
                    \arrow[r,"F_{2}"{name=F2} ,shift left  = 1.0]
                    \arrow[r,"G_{2}"'{name=G2},shift right = 1.0]
                    \&
                    \CatFont{E}
                    \arrow[r,"F_{3}"{name=F3} ,shift left  = 1.0]
                    \arrow[r,"G_{3}"'{name=G3},shift right = 1.0]
                    \&
                    \CatFont{F}
                \end{tikzcd}
            \]%
            be a diagram in $\TwoCategoryOfCategories$. The diagram
            \begin{scalemath}
                \begin{tikzcd}[row sep={7.0*\the\DL,between origins}, column sep={25.0*\the\DL,between origins}, background color=backgroundColor, ampersand replacement=\&]
                    \Nat(F_{3},G_{3})\times
                    \Nat(F_{2},G_{2})
                    \times
                    \Nat(F_{1},G_{1})
                    \arrow[r,"\twocirc_{(F_{2},G_{2}),(F_{3},G_{3})}\times\sfid"]
                    \arrow[d,"\sfid\times\twocirc_{(F_{1},G_{1}),(F_{2},G_{2})}"']
                    \&
                    \Nat(F_{3}\circ F_{2},G_{3}\circ G_{2})
                    \times
                    \Nat(F_{1},G_{1})
                    \arrow[d,"\twocirc_{(F_{3}\circ F_{2}),(G_{3}\circ G_{2},F_{1},G_{1})}"]
                    \\
                    \Nat(F_{3},G_{3})
                    \times
                    \Nat(F_{2}\circ F_{1},G_{2}\circ G_{1})
                    \arrow[r,"\twocirc_{(F_{2}\circ F_{1}),(G_{2}\circ G_{1},F_{3},G_{3})}"']
                    \&
                    \Nat(F_{3}\circ F_{2}\circ F_{1},G_{3}\circ G_{2}\circ G_{1})
                \end{tikzcd}%
            \end{scalemath}
            commutes, i.e.\ given natural transformations
            \[
                \begin{tikzcd}[row sep={5.0*\the\DL,between origins}, column sep={5.0*\the\DL,between origins}, background color=backgroundColor, ampersand replacement=\&]
                    \CatFont{C}
                    \arrow[r,"F_{1}"{name=F1} ,bend left  = 30]
                    \arrow[r,"G_{1}"'{name=G1},bend right = 30]
                    \&
                    \CatFont{D}
                    \arrow[r,"F_{2}"{name=F2} ,bend left  = 30]
                    \arrow[r,"G_{2}"'{name=G2},bend right = 30]
                    \&
                    \CatFont{E}
                    \arrow[r,"F_{3}"{name=F3} ,bend left  = 30]
                    \arrow[r,"G_{3}"'{name=G3},bend right = 30]
                    \&
                    \CatFont{F}\mrp{,}
                    %--- 2-Arrows
                    \arrow[Rightarrow,shorten=0.25em,from=F1,to=G1,"\alpha"'{pos=0.475}]
                    \arrow[Rightarrow,shorten=0.25em,from=F2,to=G2,"\beta"']
                    \arrow[Rightarrow,shorten=0.25em,from=F3,to=G3,"\gamma"']
                \end{tikzcd}
            \]%
            we have
            \[
                (\gamma\twocirc\beta)\twocirc\alpha%
                =%
                \gamma\twocirc(\beta\twocirc\alpha).%
            \]%
        \item\label{properties-of-horizontal-composition-of-natural-transformations-interaction-with-identities}\SloganFont{Interaction With Identities. }Let $F\colon\CatFont{C}\to\CatFont{D}$ and $G\colon\CatFont{D}\to\CatFont{E}$ be functors. The diagram
            \[
                \begin{tikzcd}[row sep={5.0*\the\DL,between origins}, column sep={13.0*\the\DL,between origins}, background color=backgroundColor, ampersand replacement=\&]
                    \pt\times\pt
                    \arrow[r,"{[\id_{G}]\times[\id_{F}]}"]
                    \&
                    \Nat(G,G)\times\Nat(F,F)
                    \arrow[d,"\twocirc_{(F,F),(G,G)}"]
                    \\
                    \pt
                    \arrow[r,"{[\id_{G\circ F}]}"']
                    \arrow[u,isoarrow]
                    \&
                    \Nat(G\circ F,G\circ F)
                \end{tikzcd}
            \]%
            commutes, i.e.\ we have
            \[
                \id_{G}\twocirc\id_{F}%
                =%
                \id_{G\circ F}.%
            \]
        \item\label{properties-of-horizontal-composition-of-natural-transformations-middle-four-exchange}\SloganFont{Middle Four Exchange. }\index[categories]{middle four exchange!in Cats@in $\Cats$}Let $F_{1},F_{2},F_{3}\colon\CatFont{C}\to\CatFont{D}$ and $G_{1},G_{2},G_{3}\colon\CatFont{D}\to\CatFont{E}$ be functors. The diagram
            \begin{scalemath}
                \begin{tikzcd}[row sep={10.0*\the\DL,between origins}, column sep={15.0*\the\DL,between origins}, background color=backgroundColor, ampersand replacement=\&]
                    (\Nat(G_{2},G_{3})\times\Nat(G_{1},G_{2}))\times(\Nat(F_{2},F_{3})\times\Nat(F_{1},F_{2}))
                    \arrow[rr,leftrightarrow,dashed,"\mu_{4}","\scalebox{1.5}{$\sim$}"'{sloped,outer sep=-0.0em}]
                    \arrow[d,"\circ_{G_{1},G_{2},G_{3}}\times\circ_{F_{1},F_{2},F_{3}}"']
                    \&
                    {}
                    \&
                    (\Nat(G_{2},G_{3})\times\Nat(F_{2},F_{3}))\times(\Nat(G_{1},G_{2})\times\Nat(F_{1},F_{2}))
                    \arrow[d,"\twocirc_{F_{2},F_{3},G_{2},G_{3}}\times\twocirc_{F_{1},F_{2},G_{1},G_{2}}"]
                    \\
                    \Nat(G_{1},G_{3})\times\Nat(F_{1},F_{3})
                    \arrow[rd,"\twocirc_{F_{1},F_{3},G_{1},G_{3}}"']
                    \&
                    {}
                    \&
                    \Nat(G_{2}\circ F_{2},G_{3}\circ F_{3})\times\Nat(G_{1}\circ F_{1},G_{2}\circ F_{2})
                    \arrow[ld,"\circ_{G_{1}\circ F_{1},G_{2}\circ F_{2},G_{3}\circ F_{3}}"]
                    \\[-2.0*\the\DL]
                    {}
                    \&
                    \Nat(G_{1}\circ F_{1},G_{3}\circ F_{3})
                    \&
                    {}
                \end{tikzcd}%
            \end{scalemath}
            commutes, i.e.\ given a diagram
            \[
                \begin{tikzcd}[row sep={7.0*\the\DL,between origins}, column sep={6.0*\the\DL,between origins}, background color=backgroundColor, ampersand replacement=\&]
                    \CatFont{C}
                    \arrow[r,"F_{1}",""{name=f1,pos=0.49},    bend left  = 60]
                    \arrow[r,"F_{2}"{name=f2,description}]
                    \arrow[r,"F_{3}"',""'{name=f3,pos=0.49},bend right = 60]
                    \&
                    \CatFont{D}
                    \arrow[r,"G_{1}",""{name=g1,pos=0.51},    bend left  = 60]
                    \arrow[r,"G_{2}"{name=g2,description}]
                    \arrow[r,"G_{3}"',""'{name=g3,pos=0.52},bend right = 60]
                    \&
                    \CatFont{E}
                    %--- 2-Arrows
                    \arrow[Rightarrow, shorten <= 0.25em, from=f1, to=f2, pos=0.475, "\alpha"'{pos=0.55}]
                    \arrow[Rightarrow, shorten >= 0.25em, from=f2, to=f3, pos=0.475, "\alpha'"'{pos=0.25}]
                    \arrow[Rightarrow, shorten <= 0.25em, from=g1, to=g2, pos=0.475, "\beta"'{pos=0.55}]
                    \arrow[Rightarrow, shorten >= 0.25em, from=g2, to=g3, pos=0.475, "\beta'"'{pos=0.35}]
                \end{tikzcd}
            \]%
            in $\TwoCategoryOfCategories$, we have
            \[
                (\beta'\twocirc\alpha')\circ(\beta\twocirc\alpha)%
                =%
                (\beta'\circ\beta)\twocirc(\alpha'\circ\alpha).%
            \]%
        %\item\label{properties-of-horizontal-composition-of-natural-transformations-}\SloganFont{. }
    \end{enumerate}
\end{proposition}
\begin{Proof}{Proof of \cref{properties-of-horizontal-composition-of-natural-transformations}}%
    \FirstProofBox{\cref{properties-of-horizontal-composition-of-natural-transformations-functionality}: Functionality}%
    Omitted.

    \ProofBox{\cref{properties-of-horizontal-composition-of-natural-transformations-associativity}: Associativity}%
    Omitted.

    \ProofBox{\cref{properties-of-horizontal-composition-of-natural-transformations-interaction-with-identities}: Interaction With Identities}%
    We have
    \begin{align*}
        (\id_{G}\twocirc\id_{F})_{A} &\defeq (\id_{G})_{F_{A}}\circ G_{(\id_{F})_{A}}\\
                                     &\defeq \id_{G_{F_{A}}}\circ   G_{\id_{F_{A}}}\\
                                     &=      \id_{G_{F_{A}}}\circ\id_{G_{F_{A}}}\\
                                     &=      \id_{G_{F_{A}}}\\
                                     &\defeq (\id_{G\circ F})_{A}
    \end{align*}
    for each $A\in\Obj(\CatFont{C})$, showing the desired equality.

    \ProofBox{\cref{properties-of-horizontal-composition-of-natural-transformations-middle-four-exchange}: Middle Four Exchange}%
    Let $A\in\Obj(\CatFont{C})$ and consider the diagram
    \begin{scalemath}
        \begin{tikzcd}[row sep={6.0*\the\DL,between origins}, column sep={6.0*\the\DL,between origins}, background color=backgroundColor, ampersand replacement=\&]
            \&[3.0*\the\DL]
            \&
            G_{1}(F_{3}(A))
            \arrow[rd, "\beta_{F_{3}(A)}"]
            \&
            \&[3.0*\the\DL]
            \\
            G_{1}(F_{1}(A))
            \arrow[r, "{G_{1}(\alpha_{A})}"]
            \&[3.0*\the\DL]
            G_{1}(F_{2}(A))
            \arrow[ru, "{G_{1}(\alpha'_{A})}"]
            \arrow[rd, "\beta_{F_{2}(A)}"']
            \&
            \&
            G_{2}(F_{3}(A))
            \arrow[r, "\beta'_{F_{3}(A)}"]
            \&[3.0*\the\DL]
            G_{3}(F_{3}(A))\mrp{.}
            \\
            \&[3.0*\the\DL]
            \&
            G_{2}(F_{2}(A))
            \arrow[ru, "{G_{2}(\alpha'_{A})}"']
            \&
            \&[3.0*\the\DL]
            % Subdiagrams
            \arrow[from=1-3,to=3-3, phantom, "(1)"]
        \end{tikzcd}
    \end{scalemath}
    The top composition
    \begin{scalemath}
        \begin{tikzcd}[row sep={6.0*\the\DL,between origins}, column sep={6.0*\the\DL,between origins}, background color=backgroundColor, ampersand replacement=\&]
            \&[3.0*\the\DL]
            \&
            G_{1}(F_{3}(A))
            \arrow[rd, "\beta_{F_{3}(A)}"]
            \&
            \&[3.0*\the\DL]
            \\
            G_{1}(F_{1}(A))
            \arrow[r, "{G_{1}(\alpha_{A})}"]
            \&[3.0*\the\DL]
            G_{1}(F_{2}(A))
            \arrow[ru, "{G_{1}(\alpha'_{A})}"]
            \arrow[rd, "\beta_{F_{2}(A)}"',gray!40]
            \&
            \&
            G_{2}(F_{3}(A))
            \arrow[r, "\beta'_{F_{3}(A)}"]
            \&[3.0*\the\DL]
            G_{3}(F_{3}(A))\mrp{.}
            \\
            \&[3.0*\the\DL]
            \&
            \textcolor{gray!40}{G_{2}(F_{2}(A))}
            \arrow[ru, "{G_{2}(\alpha'_{A})}"',gray!40]
            \&
            \&[3.0*\the\DL]
            % Subdiagrams
            \arrow[from=1-3,to=3-3, phantom, "(1)",gray!40]
        \end{tikzcd}
    \end{scalemath}
    is given by $((\beta'\circ\beta)\twocirc(\alpha'\circ\alpha))_{A}$, while the bottom composition
    \begin{scalemath}
        \begin{tikzcd}[row sep={6.0*\the\DL,between origins}, column sep={6.0*\the\DL,between origins}, background color=backgroundColor, ampersand replacement=\&]
            \&[3.0*\the\DL]
            \&
            \textcolor{gray!40}{G_{1}(F_{3}(A))}
            \arrow[rd, "\beta_{F_{3}(A)}",gray!40]
            \&
            \&[3.0*\the\DL]
            \\
            {G_{1}(F_{1}(A))}
            \arrow[r, "{G_{1}(\alpha_{A})}"]
            \&[3.0*\the\DL]
            {G_{1}(F_{2}(A))}
            \arrow[ru, "{G_{1}(\alpha'_{A})}",gray!40]
            \arrow[rd, "\beta_{F_{2}(A)}"']
            \&
            \&
            {G_{2}(F_{3}(A))}
            \arrow[r, "\beta'_{F_{3}(A)}"]
            \&[3.0*\the\DL]
            {G_{3}(F_{3}(A))\mrp{.}}
            \\
            \&[3.0*\the\DL]
            \&
            {G_{2}(F_{2}(A))}
            \arrow[ru, "{G_{2}(\alpha'_{A})}"']
            \&
            \&[3.0*\the\DL]
            % Subdiagrams
            \arrow[from=1-3,to=3-3, phantom, "(1)",gray!40]
        \end{tikzcd}
    \end{scalemath}
    is given by $((\beta'\twocirc\alpha')\circ(\beta\twocirc\alpha))_{A}$. Now, Subdiagram (1) corresponds to the naturality condition
    \begin{webcompile}
        G_{2}(\alpha'_{A})\circ\beta_{F_{2}(A)}%
        =%
        \beta_{F_{3}}(A)\circ G_{1}(\alpha'_{A}),%
        \quad
        \begin{tikzcd}[row sep={5.0*\the\DL,between origins}, column sep={8.75*\the\DL,between origins}, background color=backgroundColor, ampersand replacement=\&]
            G_{1}(F_{2}(A))
            \arrow[r,"{G_{1}(\alpha'_{A})}"]
            \arrow[d,"\beta_{F_{2}(A)}"']
            \&
            G_{1}(F_{3}(A))
            \arrow[d,"\beta_{F_{3}(A)}"]
            \\
            G_{2}(F_{2}(A))
            \arrow[r,"{G_{2}(\alpha'_{A})}"']
            \&
            G_{2}(F_{3}(A))
        \end{tikzcd}
    \end{webcompile}
    for $\beta\colon G_{1}\Longrightarrow G_{2}$ at $\alpha'_{A}\colon F_{2}(A)\to F_{3}(A)$, and thus commutes. Thus we have
    \[
        ((\beta'\circ\beta)\twocirc(\alpha'\circ\alpha))_{A}
        =
        ((\beta'\twocirc\alpha')\circ(\beta\twocirc\alpha))_{A}
    \]%
    for each $A\in\Obj(\CatFont{C})$ and therefore
    \[
        (\beta'\twocirc\alpha')\circ(\beta\twocirc\alpha)%
        =%
        (\beta'\circ\beta)\twocirc(\alpha'\circ\alpha).%
    \]%
    This finishes the proof.
\end{Proof}
\subsection{Properties of Natural Transformations}\label{subsection-properties-of-natural-transformations}
\begin{proposition}{Natural Transformations as Categorical Homotopies}{natural-transformations-as-categorical-homotopies}%
    Let $F,G\colon\CatFont{C}\rightrightarrows\CatFont{D}$ be functors. The following data are equivalent:%
    %--- Begin Footnote ---%
    \footnote{%
        Taken from \cite{MO64365}.
        \par\vspace*{\TCBBoxCorrection}
    }%
    %---  End Footnote  ---%
    \begin{enumerate}
        \item\label{natural-transformations-as-categorical-homotopies-1}A natural transformation $\alpha\colon F\Longrightarrow G$.
        \item\label{natural-transformations-as-categorical-homotopies-2}A functor $[\alpha]\colon\CatFont{C}\to\CatFont{D}^{\OrdinalCategoryN{1}}$ filling the diagram
            \[
                \begin{tikzcd}[row sep={5.0*\the\DL,between origins}, column sep={6.0*\the\DL,between origins}, background color=backgroundColor, ampersand replacement=\&]
                    \&
                    \CatFont{D}
                    \\
                    \CatFont{C}
                    \arrow[ru, "F"]
                    \arrow[rd, "G"']
                    \arrow[r, "{[\alpha]}"description]
                    \&
                    \CatFont{D}^{\OrdinalCategoryN{1}}\mrp{.}
                    \arrow[u, "\ev_{0}"',two heads]
                    \arrow[d, "\ev_{1}" ,two heads]
                    \\
                    \&
                    \CatFont{D}
                \end{tikzcd}
            \]%
        \item\label{natural-transformations-as-categorical-homotopies-3}A functor $[\alpha]\colon\CatFont{C}\times\OrdinalCategoryN{1}\to\CatFont{D}$ filling the diagram
            \[
                \begin{tikzcd}[row sep={5.0*\the\DL,between origins}, column sep={6.0*\the\DL,between origins}, background color=backgroundColor, ampersand replacement=\&]
                    \CatFont{C}
                    \arrow[rd, "F"]
                    \&
                    \\
                    \CatFont{C}\times\OrdinalCategoryN{1}
                    \arrow[u, "\ev_{0}" ,two heads]
                    \arrow[d, "\ev_{1}"',two heads]
                    \arrow[r, "{[\alpha]}"description]
                    \&
                    \CatFont{D}\mrp{.}
                    \\
                    \CatFont{C}
                    \arrow[ru, "G"']
                    \&
                \end{tikzcd}
            \]%
    \end{enumerate}
\end{proposition}
\begin{Proof}{Proof of \cref{natural-transformations-as-categorical-homotopies}}%
    \FirstProofBox{From \cref{natural-transformations-as-categorical-homotopies-1} to \cref{natural-transformations-as-categorical-homotopies-2} and Back}%
    We may identify $\CatFont{D}^{\OrdinalCategoryN{1}}$ with $\Arr{\CatFont{D}}$. Given a natural transformation $\alpha\colon F\Longrightarrow G$, we have a functor
    \begin{webcompile}
        \newsavebox{\BoxNaturalSquare}
        \savebox{\BoxNaturalSquare}{%
            {\footnotesize%
                \begin{tikzcd}[row sep={5.0*\the\DL,between origins}, column sep={5.0*\the\DL,between origins}, background color=backgroundColor, ampersand replacement=\&,cramped]
                    F_{A}
                    \arrow[r,"F_{f}"]
                    \arrow[d,"\alpha_{A}"']
                    \&
                    F_{B}
                    \arrow[d,"\alpha_{B}"]
                    \\
                    G_{A}
                    \arrow[r,"G_{f}"']
                    \&
                    G_{B}
                \end{tikzcd}%
            }%
        }%
        \begin{tikzcd}[row sep=0.0em, column sep=3.0*\the\DL, background color=backgroundColor, ampersand replacement=\&]
            \mathllap{[\alpha]\colon}\CatFont{C}
            \arrow[r]
            \&
            \CatFont{D}^{\OrdinalCategoryN{1}}
            \\
            A
            \arrow[r, mapsto]
            \&
            \alpha_{A}
            \\
            (f\colon A\to B)
            \arrow[r, mapsto]
            \&
            \left(\usebox{\BoxNaturalSquare}\right)%
        \end{tikzcd}%
    \end{webcompile}%%
    making the diagram in \cref{natural-transformations-as-categorical-homotopies-2} commute. Conversely, every such functor gives rise to a natural transformation from $F$ to $G$, and these constructions are inverse to each other.

    \ProofBox{From \cref{natural-transformations-as-categorical-homotopies-2} to \cref{natural-transformations-as-categorical-homotopies-3} and Back}%
    This follows from \cref{properties-of-functor-categories-adjointness} of \cref{properties-of-functor-categories}.
\end{Proof}
\subsection{Natural Isomorphisms}\label{subsection-natural-isomorphisms}
Let $\CatFont{C}$ and $\CatFont{D}$ be categories and let $F,G\colon\CatFont{C}\rightrightarrows\CatFont{D}$ be functors.
\begin{definition}{Natural Isomorphisms}{natural-isomorphisms}%
    A natural transformation $\alpha\colon F\Longrightarrow G$ is a \index[categories]{natural isomorphism}\textbf{natural isomorphism} if there exists a natural transformation $\alpha^{-1}\colon G\Longrightarrow F$ such that
    \begin{align*}
        \alpha^{-1}\circ\alpha=\id_{F},\\
        \alpha\circ\alpha^{-1}=\id_{G}.
    \end{align*}
\end{definition}
\begin{proposition}{Properties of Natural Isomorphisms}{properties-of-natural-isomorphisms}%
    Let $\alpha\colon F\Longrightarrow G$ be a natural transformation.
    \begin{enumerate}
        \item\label{properties-of-natural-isomorphisms-characterisations}\SloganFont{Characterisations. }The following conditions are equivalent:
            \begin{enumerate}
                \item\label{properties-of-natural-isomorphisms-characterisations-a}The natural transformation $\alpha$ is a natural isomorphism.
                \item\label{properties-of-natural-isomorphisms-characterisations-b}For each $A\in\Obj(\CatFont{C})$, the morphism $\alpha_{A}\colon F_{A}\to G_{A}$ is an isomorphism.
            \end{enumerate}
        \item\label{properties-of-natural-isomorphisms-componentwise-inverses-of-natural-transformations-assemble-into-natural-transformations}\SloganFont{Componentwise Inverses of Natural Transformations Assemble Into Natural Transformations. }Let $\alpha^{-1}\colon G\Longrightarrow F$ be a transformation such that, for each $A\in\Obj(\CatFont{C})$, we have
            \begin{align*}
                \alpha^{-1}_{A}\circ\alpha_{A} &= \id_{F(A)},\\
                \alpha_{A}\circ\alpha^{-1}_{A} &= \id_{G(A)}.
            \end{align*}
            Then $\alpha^{-1}$ is a natural transformation.
        %\item\label{properties-of-natural-isomorphisms-}\SloganFont{. }
    \end{enumerate}
\end{proposition}
\begin{Proof}{Proof of \cref{properties-of-natural-isomorphisms}}%
    \FirstProofBox{\cref{properties-of-natural-isomorphisms-characterisations}: Characterisations}%
    The implication \cref{properties-of-natural-isomorphisms-characterisations-a}$\implies$\cref{properties-of-natural-isomorphisms-characterisations-b} is clear, whereas the implication \cref{properties-of-natural-isomorphisms-characterisations-b}$\implies$\cref{properties-of-natural-isomorphisms-characterisations-a} follows from \cref{properties-of-natural-isomorphisms-componentwise-inverses-of-natural-transformations-assemble-into-natural-transformations}.

    \ProofBox{\cref{properties-of-natural-isomorphisms-componentwise-inverses-of-natural-transformations-assemble-into-natural-transformations}: Componentwise Inverses of Natural Transformations Assemble Into Natural Transformations}%
    The naturality condition for $\alpha^{-1}$ corresponds to the commutativity of the diagram
    \[
        \begin{tikzcd}[row sep={5.0*\the\DL,between origins}, column sep={6.0*\the\DL,between origins}, background color=backgroundColor, ampersand replacement=\&]
            G(A)
            \arrow[r, "{G(f)}"]
            \arrow[d, "\alpha^{-1}_{A}"']
            \&
            G(B)
            \arrow[d, "\alpha^{-1}_{B}"]
            \\
            F(A)
            \arrow[r, "{F(f)}"']
            \&
            F(B)
        \end{tikzcd}
    \]%
    for each $A,B\in\Obj(\CatFont{C})$ and each $f\in\Hom_{\CatFont{C}}(A,B)$. Considering the diagram
    \[
        \begin{tikzcd}[row sep={5.0*\the\DL,between origins}, column sep={7.5*\the\DL,between origins}, background color=backgroundColor, ampersand replacement=\&]
            G(A)
            \arrow[r, "{G(f)}"]
            \arrow[d, "\alpha^{-1}_{A}"']
            \&
            G(B)
            \arrow[d, "\alpha^{-1}_{B}"]
            \\
            F(A)
            \arrow[r, "{F(f)}"description]
            \arrow[d, "\alpha_{A}"']
            \&
            F(B)
            \arrow[d, "\alpha_{B}"]
            \\
            G(A)
            \arrow[r, "{G(f)}"']
            \&
            G(B)\mrp{,}
            % Subdiagrams
            \arrow[from=1-1,to=2-2, phantom, "\scriptstyle(1)"]
            \arrow[from=2-1,to=3-2, phantom, "\scriptstyle(2)"]
        \end{tikzcd}
    \]%
    where the boundary diagram as well as Subdiagram (2) commute, we have
    \begin{align*}
        G(f) &= G(f)\circ\id_{G(A)}\\
             &= G(f)\circ\alpha_{A}\circ\alpha^{-1}_{A}\\
             &= \alpha_{B}\circ F(f)\circ\alpha^{-1}_{A}.
    \end{align*}
    Postcomposing both sides with $\alpha^{-1}_{B}$, we get
    \begin{align*}
        \alpha^{-1}_{B}\circ G(f) &= \alpha^{-1}_{B}\circ\alpha_{B}\circ F(f)\circ\alpha^{-1}_{A}\\
                                  &= \id_{F(B)}\circ F(f)\circ\alpha^{-1}_{A}\\
                                  &= F(f)\circ\alpha^{-1}_{A},
    \end{align*}
    which is the naturality condition we wanted to show. Thus $\alpha^{-1}$ is a natural transformation.
\end{Proof}
\section{Categories of Categories}\label{section-categories-of-categories}
\subsection{Functor Categories}\label{subsection-functor-categories}
Let $\CatFont{C}$ be a category and $\CatFont{D}$ be a small category.
\begin{definition}{Functor Categories}{functor-categories}%
    The \index[categories]{functor category}\textbf{category of functors from $\CatFont{C}$ to $\CatFont{D}$}%
    %--- Begin Footnote ---%
    \footnote{%
        \SloganFont{Further Terminology: }Also called the \textbf{functor category} $\Fun(\CatFont{C},\CatFont{D})$.
    } %
    %---  End Footnote  ---%
    is the category \index[notation]{FunCD@$\Fun(\CatFont{C},\CatFont{D})$}$\Fun(\CatFont{C},\CatFont{D})$%
    %--- Begin Footnote ---%
    \footnote{%
        \SloganFont{Further Notation: }Also written \index[notation]{DC@$\CatFont{D}^{\CatFont{C}}$}$\CatFont{D}^{\CatFont{C}}$ and \index[notation]{CD@$[\CatFont{C},\CatFont{D}]$}$[\CatFont{C},\CatFont{D}]$.
        \par\vspace*{\TCBBoxCorrection}
    } %
    %---  End Footnote  ---%
    where
    \begin{itemize}
        \item\SloganFont{Objects. }The objects of $\Fun(\CatFont{C},\CatFont{D})$ are functors from $\CatFont{C}$ to $\CatFont{D}$.
        \item\SloganFont{Morphisms. }For each $F,G\in\Obj(\Fun(\CatFont{C},\CatFont{D}))$, we have
            \[
                \Hom_{\Fun(\CatFont{C},\CatFont{D})}(F,G)
                \defeq
                \Nat(F,G).
            \]%
        \item\SloganFont{Identities. }For each $F\in\Obj(\Fun(\CatFont{C},\CatFont{D}))$, the unit map
            \[
                \Unit^{\Fun(\CatFont{C},\CatFont{D})}_{F}
                \colon
                \pt
                \to
                \Nat(F,F)
            \]%
            of $\Fun(\CatFont{C},\CatFont{D})$ at $F$ is given by
            \[
                \id^{\Fun(\CatFont{C},\CatFont{D})}_{F}
                \defeq
                \id_{F},
            \]%
            where $\id_{F}\colon F\Longrightarrow F$ is the identity natural transformation of $F$ of \cref{identity-natural-transformations}.
        \item\SloganFont{Composition. }For each $F,G,H\in\Obj(\Fun(\CatFont{C},\CatFont{D}))$, the composition map
            \[
                \circ^{\Fun(\CatFont{C},\CatFont{D})}_{F,G,H}
                \colon
                \Nat(G,H)
                \times
                \Nat(F,G)
                \to
                \Nat(F,H)
            \]%
            of $\Fun(\CatFont{C},\CatFont{D})$ at $(F,G,H)$ is given by
            \[
                \beta\mathbin{{\circ}^{\Fun(\CatFont{C},\CatFont{D})}_{F,G,H}}\alpha
                \defeq
                \beta\circ\alpha,
            \]%
            where $\beta\circ\alpha$ is the vertical composition of $\alpha$ and $\beta$ of \cref{properties-of-vertical-composition-of-natural-transformations-functionality} of \cref{properties-of-vertical-composition-of-natural-transformations}.
    \end{itemize}
\end{definition}
\begin{proposition}{Properties of Functor Categories}{properties-of-functor-categories}%
    Let $\CatFont{C}$ and $\CatFont{D}$ be categories and let $F\colon\CatFont{C}\to\CatFont{D}$ be a functor.
    \begin{enumerate}
        \item\label{properties-of-functor-categories-functoriality}\SloganFont{Functoriality. }The assignments $\CatFont{C},\CatFont{D},(\CatFont{C},\CatFont{D})\mapsto\Fun(\CatFont{C},\CatFont{D})$ define functors
            \[
                \BifunctorialityPeriod{\Fun(\CatFont{C},-)}{\Fun(-,\CatFont{D})}{\Fun(-_{1},-_{2})}{\Cats}{\Cats\mrp{{}^{\op}}}{\Cats^{\op}\times\Cats}{\Cats}%
            \]%
        \item\label{properties-of-functor-categories-2-functoriality}\SloganFont{2-Functoriality. }The assignments $\CatFont{C},\CatFont{D},(\CatFont{C},\CatFont{D})\mapsto\Fun(\CatFont{C},\CatFont{D})$ define 2-functors
            \[
                \BifunctorialityPeriod{\Fun(\CatFont{C},-)}{\Fun(-,\CatFont{D})}{\Fun(-_{1},-_{2})}{\TwoCategoryOfCategories}{\TwoCategoryOfCategories^{\mrp{\op}}}{\TwoCategoryOfCategories^{\op}\times\TwoCategoryOfCategories}{\TwoCategoryOfCategories}%
            \]%
        \item\label{properties-of-functor-categories-adjointness}\SloganFont{Adjointness. }We have adjunctions
            \begin{webcompile}
                \begin{gathered}
                    \Adjunction#\CatFont{C}\times-#\Fun(\CatFont{C},-)#\Cats#\Cats,#\\
                    \Adjunction#-\times\CatFont{D}#\Fun(\CatFont{D},-)#\Cats#\Cats,#
                \end{gathered}
            \end{webcompile}
            witnessed by bijections of sets
            \begin{align*}
                \Hom_{\Cats}(\CatFont{C}\times\CatFont{D},\CatFont{E}) &\cong \Hom_{\Cats}(\CatFont{D},\Fun(\CatFont{C},\CatFont{E})),\\
                \Hom_{\Cats}(\CatFont{C}\times\CatFont{D},\CatFont{E}) &\cong \Hom_{\Cats}(\CatFont{C},\Fun(\CatFont{D},\CatFont{E})),
            \end{align*}
            natural in $\CatFont{C},\CatFont{D},\CatFont{E}\in\Obj(\Cats)$.
        \item\label{properties-of-functor-categories-2-adjointness}\SloganFont{2-Adjointness. }We have 2-adjunctions
            \begin{webcompile}
                \begin{gathered}
                    \TwoAdjunction#\CatFont{C}\times-#\Fun(\CatFont{C},-)#\TwoCategoryOfCategories#\TwoCategoryOfCategories,#\\
                    \TwoAdjunction#-\times\CatFont{D}#\Fun(\CatFont{D},-)#\TwoCategoryOfCategories#\TwoCategoryOfCategories,#
                \end{gathered}
            \end{webcompile}
            witnessed by isomorphisms of categories
            \begin{align*}
                \Fun(\CatFont{C}\times\CatFont{D},\CatFont{E}) &\cong \Fun(\CatFont{D},\Fun(\CatFont{C},\CatFont{E})),\\
                \Fun(\CatFont{C}\times\CatFont{D},\CatFont{E}) &\cong \Fun(\CatFont{C},\Fun(\CatFont{D},\CatFont{E})),
            \end{align*}
            natural in $\CatFont{C},\CatFont{D},\CatFont{E}\in\Obj(\TwoCategoryOfCategories)$.
        \item\label{properties-of-functor-categories-interaction-with-punctual-categories}\SloganFont{Interaction With Punctual Categories. }We have a canonical isomorphism of categories
            \[
                \Fun(\PunctualCategory,\CatFont{C})
                \cong
                \CatFont{C},
            \]%
            natural in $\CatFont{C}\in\Obj(\Cats)$.
        \item\label{properties-of-functor-categories-objectwise-computation-of-co-limits}\SloganFont{Objectwise Computation of Co/Limits. }Let
            \[
                D
                \colon
                \CatFont{I}
                \to
                \Fun(\CatFont{C},\CatFont{D})
            \]%
            be a diagram in $\Fun(\CatFont{C},\CatFont{D})$. We have isomorphisms
            \begin{align*}
                \lim(D)_{A}   &\cong \lim_{i\in\CatFont{I}}(D_{i}(A)),\\
                \colim(D)_{A} &\cong \colim_{i\in\CatFont{I}}(D_{i}(A)),
            \end{align*}
            naturally in $A\in\Obj(\CatFont{C})$.
        \item\label{properties-of-functor-categories-interaction-with-co-cocompleteness}\SloganFont{Interaction With Co/Completeness. }If $\CatFont{E}$ is co/complete, then so is $\Fun(\CatFont{C},\CatFont{E})$.
        \item\label{properties-of-functor-categories-monomorphisms-and-epimorphisms}\SloganFont{Monomorphisms and Epimorphisms. }Let $\alpha\colon F\Longrightarrow G$ be a morphism of $\Fun(\CatFont{C},\CatFont{D})$. The following conditions are equivalent:
            \begin{enumerate}
                \item\label{properties-of-functor-categories-monomorphisms-and-epimorphisms-a}The natural transformation
                    \[
                        \alpha
                        \colon
                        F
                        \Longrightarrow
                        G
                    \]%
                    is a monomorphism (resp.\ epimorphism) in $\Fun(\CatFont{C},\CatFont{D})$.
                \item\label{properties-of-functor-categories-monomorphisms-and-epimorphisms-b}For each $A\in\Obj(\CatFont{C})$, the morphism
                    \[
                        \alpha_{A}
                        \colon
                        F_{A}
                        \to
                        G_{A}
                    \]%
                    is a monomorphism (resp.\ epimorphism) in $\CatFont{D}$.
            \end{enumerate}
    \end{enumerate}
\end{proposition}
\begin{Proof}{Proof of \cref{properties-of-functor-categories}}%
    \FirstProofBox{\cref{properties-of-functor-categories-functoriality}: Functoriality}%
    Omitted.

    \ProofBox{\cref{properties-of-functor-categories-2-functoriality}: 2-Functoriality}%
    Omitted.

    \ProofBox{\cref{properties-of-functor-categories-adjointness}: Adjointness}%
    Omitted.

    \ProofBox{\cref{properties-of-functor-categories-2-adjointness}: 2-Adjointness}%
    Omitted.

    \ProofBox{\cref{properties-of-functor-categories-interaction-with-punctual-categories}: Interaction With Punctual Categories}%
    Omitted.

    \ProofBox{\cref{properties-of-functor-categories-objectwise-computation-of-co-limits}: Objectwise Computation of Co/Limits}%
    Omitted.

    \ProofBox{\cref{properties-of-functor-categories-interaction-with-co-cocompleteness}: Interaction With Co/Completeness}%
    This follows from \cref{properties-of-functor-categories-objectwise-co-limits}.

    \ProofBox{\cref{properties-of-functor-categories-monomorphisms-and-epimorphisms}: Monomorphisms and Epimorphisms}%
    Omitted.
\end{Proof}
\subsection{The Category of Categories and Functors}\label{subsection-the-category-of-categories-and-functors}
\begin{definition}{The Category of Categories and Functors}{the-category-of-categories-and-functors}%
    The \index[categories]{category of categories}\textbf{category of} (\textbf{small}) \textbf{categories and functors} is the category \index[notation]{Cats@$\Cats$}$\Cats$ where
    \begin{itemize}
        \item\SloganFont{Objects. }The objects of $\Cats$ are small categories.
        \item\SloganFont{Morphisms. }For each $\CatFont{C},\CatFont{D}\in\Obj(\Cats)$, we have
            \[
                \Hom_{\Cats}(\CatFont{C},\CatFont{D})
                \defeq
                \Obj(\Fun(\CatFont{C},\CatFont{D})).
            \]%
        \item\SloganFont{Identities. }For each $\CatFont{C}\in\Obj(\Cats)$, the unit map
            \[
                \Unit^{\Cats}_{\CatFont{C}}
                \colon
                \pt
                \to
                \Hom_{\Cats}(\CatFont{C},\CatFont{C})
            \]%
            of $\Cats$ at $\CatFont{C}$ is defined by
            \[
                \id^{\Cats}_{\CatFont{C}}
                \defeq
                \id_{\CatFont{C}},
            \]%
            where $\id_{\CatFont{C}}\colon\CatFont{C}\to\CatFont{C}$ is the identity functor of $\CatFont{C}$ of \cref{identity-functors}.
        \item\SloganFont{Composition. }For each $\CatFont{C},\CatFont{D},\CatFont{E}\in\Obj(\Cats)$, the composition map
            \[
                \circ^{\Cats}_{\CatFont{C},\CatFont{D},\CatFont{E}}
                \colon
                \Hom_{\Cats}(\CatFont{D},\CatFont{E})
                \times
                \Hom_{\Cats}(\CatFont{C},\CatFont{D})
                \to
                \Hom_{\Cats}(\CatFont{C},\CatFont{E})
            \]%
            of $\Cats$ at $(\CatFont{C},\CatFont{D},\CatFont{E})$ is given by
            \[
                G\mathbin{{\circ}^{\Cats}_{\CatFont{C},\CatFont{D},\CatFont{E}}}F
                \defeq
                G\circ F,
            \]%
            where $G\circ F\colon\CatFont{C}\to\CatFont{E}$ is the composition of $F$ and $G$ of \cref{composition-of-functors}.
    \end{itemize}
\end{definition}
\begin{proposition}{Properties of the Category $\Cats$}{properties-of-the-category-cats}%
    Let $\CatFont{C}$ be a category.
    \begin{enumerate}
        \item\label{properties-of-the-category-cats-co-completeness}\SloganFont{Co/Completeness. }The category $\Cats$ is complete and cocomplete.%
        \item\label{properties-of-the-category-cats-cartesian-monoidal-structure}\SloganFont{Cartesian Monoidal Structure. }The quadruple $(\Cats,\times,\PunctualCategory,\Fun)$ is a Cartesian closed monoidal category.
    \end{enumerate}
\end{proposition}
\begin{Proof}{Proof of \cref{properties-of-the-category-cats}}%
    \FirstProofBox{\cref{properties-of-the-category-cats-co-completeness}: Co/Completeness}%
    Omitted.

    \ProofBox{\cref{properties-of-the-category-cats-cartesian-monoidal-structure}: Cartesian Monoidal Structure}%
    Omitted.
\end{Proof}
\subsection{The $2$-Category of Categories, Functors, and Natural Transformations}\label{subsection-the-2-category-of-categories-functors-and-natural-transformations}
\begin{definition}{The $2$-Category of Categories}{the-2-category-of-categories}%
    The \index[categories]{two-category@$2$-category!of small categories}\textbf{$2$-category of} (\textbf{small}) \textbf{categories, functors, and natural transformations} is the $2$-category \index[notation]{Catstwo@$\TwoCategoryOfCategories$}$\TwoCategoryOfCategories$ where%
    \begin{itemize}
        \item\SloganFont{Objects. }The objects of $\TwoCategoryOfCategories$ are small categories.
        \item\SloganFont{$\cHom$-Categories. }For each $\CatFont{C},\CatFont{D}\in\Obj(\TwoCategoryOfCategories)$, we have
            \[
                \cHom_{\TwoCategoryOfCategories}(\CatFont{C},\CatFont{D})
                \defeq
                \Fun(\CatFont{C},\CatFont{D}).
            \]%
        \item\SloganFont{Identities. }For each $\CatFont{C}\in\Obj(\TwoCategoryOfCategories)$, the unit functor
            \[
                \Unit^{\TwoCategoryOfCategories}_{\CatFont{C}}
                \colon
                \PunctualCategory
                \to
                \Fun(\CatFont{C},\CatFont{C})
            \]%
            of $\TwoCategoryOfCategories$ at $\CatFont{C}$ is the functor picking the identity functor $\id_{\CatFont{C}}\colon\CatFont{C}\to\CatFont{C}$ of $\CatFont{C}$.
        \item\SloganFont{Composition. }For each $\CatFont{C},\CatFont{D},\CatFont{E}\in\Obj(\TwoCategoryOfCategories)$, the composition bifunctor
            \[
                \circ^{\TwoCategoryOfCategories}_{\CatFont{C},\CatFont{D},\CatFont{E}}
                \colon
                \cHom_{\TwoCategoryOfCategories}(\CatFont{D},\CatFont{E})
                \times
                \cHom_{\TwoCategoryOfCategories}(\CatFont{C},\CatFont{D})
                \to
                \cHom_{\TwoCategoryOfCategories}(\CatFont{C},\CatFont{E})
            \]%
            of $\TwoCategoryOfCategories$ at $(\CatFont{C},\CatFont{D},\CatFont{E})$ is the functor where
            \begin{itemize}
                \item\SloganFont{Action on Objects. }For each object $(G,F)\in\Obj(\cHom_{\TwoCategoryOfCategories}(\CatFont{D},\CatFont{E})\times\cHom_{\TwoCategoryOfCategories}(\CatFont{C},\CatFont{D}))$, we have
                    \[
                        \circ^{\TwoCategoryOfCategories}_{\CatFont{C},\CatFont{D},\CatFont{E}}(G,F)
                        \defeq
                        G\circ F.
                    \]%
                \item\SloganFont{Action on Morphisms. }For each morphism $(\beta,\alpha)\colon(K,H)\Longrightarrow(G,F)$ of $\cHom_{\TwoCategoryOfCategories}(\CatFont{D},\CatFont{E})\times\cHom_{\TwoCategoryOfCategories}(\CatFont{C},\CatFont{D})$, we have
                    \[
                        \circ^{\TwoCategoryOfCategories}_{\CatFont{C},\CatFont{D},\CatFont{E}}(\beta,\alpha)
                        \defeq
                        \beta\twocirc\alpha,
                    \]%
                    where $\beta\twocirc\alpha$ is the horizontal composition of $\alpha$ and $\beta$ of \cref{horizontal-composition-of-two-natural-transformations}.
            \end{itemize}
    \end{itemize}
\end{definition}
\begin{proposition}{Properties of the 2-Category $\TwoCategoryOfCategories$}{properties-of-the-2-category-cats}%
    Let $\CatFont{C}$ be a category.
    \begin{enumerate}
        \item\label{properties-of-the-2-category-cats-co-completeness}\SloganFont{2-Categorical Co/Completeness. }The 2-category $\TwoCategoryOfCategories$ is complete and cocomplete as a 2-category, having all 2-categorical and bicategorical co/limits.%
    \end{enumerate}
\end{proposition}
\begin{Proof}{Proof of \cref{properties-of-the-2-category-cats}}%
    \FirstProofBox{\cref{properties-of-the-2-category-cats-co-completeness}: Co/Completeness}%
    Omitted.
\end{Proof}
\subsection{The Category of Groupoids}\label{subsection-the-category-of-groupoids}
\begin{definition}{The Category of Small Groupoids}{the-category-of-small-groupoids}%
    The \index[categories]{category!of small groupoids}\textbf{category of} (\textbf{small}) \textbf{groupoids} is the full subcategory \index[notation]{Grpd@$\Grpd$}$\Grpd$ of $\Cats$ spanned by the groupoids.
\end{definition}
\subsection{The $2$-Category of Groupoids}\label{subsection-the-2-category-of-groupoids}
\begin{definition}{The $2$-Category of Small Groupoids}{the-2-category-of-small-groupoids}%
    The \index[categories]{two-category@$2$-category!of small groupoids}\textbf{$2$-category of} (\textbf{small}) \textbf{groupoids} is the full sub-$2$-category \index[notation]{Grpdtwo@$\TwoCategoryOfGroupoids$}$\TwoCategoryOfGroupoids$ of $\TwoCategoryOfCategories$ spanned by the groupoids.
\end{definition}
\begin{appendices}
\begin{multicols}{2}[\section{Other Chapters}]
\noindent
\textbf{Preliminaries}
\begin{enumerate}
\item \hyperref[introduction:section-phantom]{Introduction}
\end{enumerate}
\textbf{Sets}
\begin{enumerate}
\setcounter{enumi}{2}
\item \hyperref[sets:section-phantom]{Sets}
\item \hyperref[constructions-with-sets:section-phantom]{Constructions With Sets}
\item \hyperref[monoidal-structures-on-the-category-of-sets:section-phantom]{Monoidal Structures on the Category of Sets}
\item \hyperref[pointed-sets:section-phantom]{Pointed Sets}
\item \hyperref[tensor-products-of-pointed-sets:section-phantom]{Tensor Products of Pointed Sets}
\end{enumerate}
\textbf{Relations}
\begin{enumerate}
\setcounter{enumi}{6}
\item \hyperref[relations:section-phantom]{Relations}
\item \hyperref[constructions-with-relations:section-phantom]{Constructions With Relations}
\item \hyperref[conditions-on-relations:section-phantom]{Conditions on Relations}
\end{enumerate}
\textbf{Category Theory}
\begin{enumerate}
\setcounter{enumi}{9}
\item \hyperref[categories:section-phantom]{Categories}
\end{enumerate}
\textbf{Monoidal Categories}
\begin{enumerate}
\setcounter{enumi}{10}
\item \hyperref[constructions-with-monoidal-categories:section-phantom]{Constructions With Monoidal Categories}
\end{enumerate}
\textbf{Bicategories}
\begin{enumerate}
\setcounter{enumi}{11}
\item \hyperref[types-of-morphisms-in-bicategories:section-phantom]{Types of Morphisms in Bicategories}
\end{enumerate}
\textbf{Extra Part}
\begin{enumerate}
\setcounter{enumi}{12}
\item \hyperref[notes:section-phantom]{Notes}
\end{enumerate}
\end{multicols}

\end{appendices}
\end{document}
