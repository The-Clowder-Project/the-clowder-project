\input{preamble}

% OK, start here.
%
\usepackage{fontspec}
\let\hyperwhite\relax
\let\hyperred\relax
\newcommand{\hyperwhite}{\hypersetup{citecolor=white,filecolor=white,linkcolor=white,urlcolor=white}}
\newcommand{\hyperred}{%
\hypersetup{%
    citecolor=TitlingRed,%
    filecolor=TitlingRed,%
    linkcolor=TitlingRed,%
     urlcolor=TitlingRed%
}}
\let\ChapterRef\relax
\newcommand{\ChapterRef}[2]{#1}
\setcounter{tocdepth}{2}
%▓▓▓▓▓▓▓▓▓▓▓▓▓▓▓▓▓▓▓▓▓▓▓▓▓▓▓▓▓▓▓▓▓
%▓▓ ╔╦╗╦╔╦╗╦  ╔═╗  ╔═╗╔═╗╔╗╔╔╦╗ ▓▓
%▓▓  ║ ║ ║ ║  ║╣   ╠╣ ║ ║║║║ ║  ▓▓
%▓▓  ╩ ╩ ╩ ╩═╝╚═╝  ╚  ╚═╝╝╚╝ ╩  ▓▓
%▓▓▓▓▓▓▓▓▓▓▓▓▓▓▓▓▓▓▓▓▓▓▓▓▓▓▓▓▓▓▓▓▓
%\usepackage{titlesec}
%▓▓▓▓▓▓▓▓▓▓▓▓▓▓▓▓▓▓▓▓▓▓▓▓▓▓▓▓▓▓▓▓▓▓▓▓▓▓▓▓▓▓▓▓▓▓▓▓▓▓▓▓▓▓▓
%▓▓ ╔╦╗╔═╗╔╗ ╦  ╔═╗  ╔═╗╔═╗  ╔═╗╔═╗╔╗╔╔╦╗╔═╗╔╗╔╔╦╗╔═╗ ▓▓
%▓▓  ║ ╠═╣╠╩╗║  ║╣   ║ ║╠╣   ║  ║ ║║║║ ║ ║╣ ║║║ ║ ╚═╗ ▓▓
%▓▓  ╩ ╩ ╩╚═╝╩═╝╚═╝  ╚═╝╚    ╚═╝╚═╝╝╚╝ ╩ ╚═╝╝╚╝ ╩ ╚═╝ ▓▓
%▓▓▓▓▓▓▓▓▓▓▓▓▓▓▓▓▓▓▓▓▓▓▓▓▓▓▓▓▓▓▓▓▓▓▓▓▓▓▓▓▓▓▓▓▓▓▓▓▓▓▓▓▓▓▓
\newcommand{\ChapterTableOfContents}{%
    \begingroup
    \addfontfeature{Numbers={Lining,Monospaced}}
    \hypersetup{hidelinks}\tableofcontents%
    \endgroup
}%

\let\DotFill\relax
\makeatletter
\newcommand \DotFill {\leavevmode \cleaders \hb@xt@ .33em{\hss .\hss }\hfill \kern \z@}
\makeatother

\definecolor{ToCGrey}{rgb}{0.4,0.4,0.4}
\definecolor{mainColor}{rgb}{0.82745098,0.18431373,0.18431373}
\usepackage{titletoc}
\titlecontents{part}
[0.0em]
{\addvspace{1pc}\color{TitlingRed}\large\bfseries\text{Part }}
{\bfseries\textcolor{TitlingRed}{\contentslabel{0.0em}}\hspace*{1.35em}}
{}
{\textcolor{TitlingRed}{{\hfill\bfseries\contentspage\nobreak}}}
[]
\titlecontents{section}
[0.0em]
{\addvspace{1pc}}
{\color{black}\bfseries\textcolor{TitlingRed}{\contentslabel{0.0em}}\hspace*{1.65em}}
{}
{\textcolor{black}{\textbf{\DotFill}{\bfseries\contentspage\nobreak}}}
[]
\titlecontents{subsection}
[0.0em]
{}
{\hspace*{1.65em}\color{ToCGrey}{\contentslabel{0.0em}}\hspace*{2.5em}}
{}
{{\textcolor{ToCGrey}\DotFill}\textcolor{ToCGrey}{\contentspage}\nobreak}
[]
\usepackage{marginnote}
\renewcommand*{\marginfont}{\normalfont}
\usepackage{inconsolata}
\setmonofont{inconsolata}%
\let\ChapterRef\relax
\newcommand{\ChapterRef}[2]{#1}
\AtBeginEnvironment{subappendices}{%%
    \section*{\huge Appendices}%
}%

\begin{document}

\title{Introduction}

\maketitle

\phantomsection
\label{section-phantom}

This chapter contains some general information about the Clowder Project.

\ChapterTableOfContents

\section{Introduction}\label{section-introduction}
\subsection{Content and Scope}\label{subsection-scope}
In this section, we outline what content is expected to be covered in the Clowder Project.
\subsubsection{Elementary Category Theory}\label{subsubsection-elementary-category-theory}
First and foremost, the Clowder Project aims to cover the foundations of category theory. This comprises all the usual topics treated in basic textbooks in category theory, such as \cite{categories-for-the-working-mathematician} or \cite{category-theory-in-context}, like adjunctions, co/limits, Kan extensions, co/ends, monoidal categories, etc.
\subsection{Goals}\label{subsection-goals}
\subsubsection{Filling Gaps in the Literature}\label{subsubsection-filling-gaps-in-the-literature}
There are a number of very significant gaps in the category theory literature, some of which the Clowder Project hopes to fill. Here we list them.%
%--- Begin Footnote ---%
\footnote{%
    See also \cite{MO494959}.
}%
%---  End Footnote  ---%
\begin{gap}{The Tensor Product of Presentable Categories}{subsection-gap-the-tensor-product-of-presentable-categories}%
    Even though its analogue for $\infty$-categories has for years been a widely used tool%
    %--- Begin Footnote ---%
    \footnote{%
        See \cite{MO490557}.
        \par\vspace*{\TCBBoxCorrection}
    }, %
    %---  End Footnote  ---%
    a comprehensive treatment of the tensor product of presentable categories seems to be currently missing.
\end{gap}
\begin{gap}{Explicit Descriptions of Co/Limits of Categories}{subsection-gap-explicit-descriptions-of-co-limits-of-categories}%
    An exhaustive concrete description of the various limits and colimits of categories, including 2-dimensional ones such as coequifiers, isoinserters, etc.
\end{gap}
\begin{gap}{Dinatural Transformation Co/Classifiers}{subsection-gap-dinatural-transformation-co-classifiers}%
    Dinatural transformation co/classifiers, characterised by isomorphisms of the form $\Nat(F,G)\cong\DiNat(\Gamma(F),G)$.%
    %--- Begin Footnote ---%
    \footnote{%
        Although these appear already in Dubuc--Street's paper introducing dinatural transformations, \cite{dubuc-street-dinatural-transformations}, it seems that all foundational treatments that followed after ended up not covering this concept.
    }%
    %---  End Footnote  ---%
\end{gap}
\begin{gap}{The Tensor Product of Symmetric Monoidal Categories}{subsection-gap-the-tensor-product-of-symmetric-monoidal-categories}%
    A treatment of the tensor product of symmetric monoidal categories, replicating \cite{the-symmetric-monoidal-2-category-of-permutative-categories} for the non-strict case.
\end{gap}
\begin{gap}{A Comprehensive Treatment of the Theory of Promonoidal Categories}{subsubsection-gap-a-comprehensive-treatment-of-the-theory-of-promonoidal-categories}%
    A comprehensive and exhaustive treatment of the theory of promonoidal categories, including notions such as dualisability internal

    and how these may be viewed as categorifications of \say{hypermonoids}.
\end{gap}
\subsubsection{Homogenizing Conventions, Notation, and Terminology}\label{subsubsection-homogenizing-conventions-notation-and-terminology}
\subsection{Prerequisites/Assumed Background}\label{subsection-prerequisites-assumed-background}
\subsection{Community Engagement and Collaboration}\label{subsection-community-engagement-and-collaboration}
\section{Miscellany}\label{section-miscellany}
\subsection{Infrastructure and Technical Implementation}\label{subsection-infrastructure-and-technical-implementation}
\begin{appendices}
\begin{multicols}{2}[\section{Other Chapters}]
\noindent
\textbf{Sets}
\begin{enumerate}
\item \hyperref[sets:section-phantom]{Sets}
\item \hyperref[constructions-with-sets:section-phantom]{Constructions With Sets}
\item \hyperref[pointed-sets:section-phantom]{Pointed Sets}
\item \hyperref[tensor-products-of-pointed-sets:section-phantom]{Tensor Products of Pointed Sets}
\end{enumerate}
\textbf{Relations}
\begin{enumerate}
\setcounter{enumi}{4}
\item \hyperref[relations:section-phantom]{Relations}
\item \hyperref[constructions-with-relations:section-phantom]{Constructions With Relations}
\item \hyperref[equivalence-relations-and-apartness-relations:section-phantom]{Equivalence Relations and Apartness Relations}
\end{enumerate}
\textbf{Category Theory}
\begin{enumerate}
\setcounter{enumi}{7}
\item \hyperref[categories:section-phantom]{Categories}
\end{enumerate}
\textbf{Bicategories}
\begin{enumerate}
\setcounter{enumi}{8}
\item \hyperref[types-of-morphisms-in-bicategories:section-phantom]{Types of Morphisms in Bicategories}
\end{enumerate}
\end{multicols}

\end{appendices}
\end{document}
