\input{preamble}

% OK, start here.
%
\usepackage{fontspec}
\let\hyperwhite\relax
\let\hyperred\relax
\newcommand{\hyperwhite}{\hypersetup{citecolor=white,filecolor=white,linkcolor=white,urlcolor=white}}
\newcommand{\hyperred}{%
\hypersetup{%
    citecolor=TitlingRed,%
    filecolor=TitlingRed,%
    linkcolor=TitlingRed,%
     urlcolor=TitlingRed%
}}
\let\ChapterRef\relax
\newcommand{\ChapterRef}[2]{#1}
\setcounter{tocdepth}{2}
%▓▓▓▓▓▓▓▓▓▓▓▓▓▓▓▓▓▓▓▓▓▓▓▓▓▓▓▓▓▓▓▓▓
%▓▓ ╔╦╗╦╔╦╗╦  ╔═╗  ╔═╗╔═╗╔╗╔╔╦╗ ▓▓
%▓▓  ║ ║ ║ ║  ║╣   ╠╣ ║ ║║║║ ║  ▓▓
%▓▓  ╩ ╩ ╩ ╩═╝╚═╝  ╚  ╚═╝╝╚╝ ╩  ▓▓
%▓▓▓▓▓▓▓▓▓▓▓▓▓▓▓▓▓▓▓▓▓▓▓▓▓▓▓▓▓▓▓▓▓
%\usepackage{titlesec}
%▓▓▓▓▓▓▓▓▓▓▓▓▓▓▓▓▓▓▓▓▓▓▓▓▓▓▓▓▓▓▓▓▓▓▓▓▓▓▓▓▓▓▓▓▓▓▓▓▓▓▓▓▓▓▓
%▓▓ ╔╦╗╔═╗╔╗ ╦  ╔═╗  ╔═╗╔═╗  ╔═╗╔═╗╔╗╔╔╦╗╔═╗╔╗╔╔╦╗╔═╗ ▓▓
%▓▓  ║ ╠═╣╠╩╗║  ║╣   ║ ║╠╣   ║  ║ ║║║║ ║ ║╣ ║║║ ║ ╚═╗ ▓▓
%▓▓  ╩ ╩ ╩╚═╝╩═╝╚═╝  ╚═╝╚    ╚═╝╚═╝╝╚╝ ╩ ╚═╝╝╚╝ ╩ ╚═╝ ▓▓
%▓▓▓▓▓▓▓▓▓▓▓▓▓▓▓▓▓▓▓▓▓▓▓▓▓▓▓▓▓▓▓▓▓▓▓▓▓▓▓▓▓▓▓▓▓▓▓▓▓▓▓▓▓▓▓
\newcommand{\ChapterTableOfContents}{%
    \begingroup
    \addfontfeature{Numbers={Lining,Monospaced}}
    \hypersetup{hidelinks}\tableofcontents%
    \endgroup
}%

\makeatletter
\newcommand \DotFill {\leavevmode \cleaders \hb@xt@ .33em{\hss .\hss }\hfill \kern \z@}
\makeatother

\definecolor{ToCGrey}{rgb}{0.4,0.4,0.4}
\definecolor{mainColor}{rgb}{0.82745098,0.18431373,0.18431373}
\usepackage{titletoc}
\titlecontents{part}
[0.0em]
{\addvspace{1pc}\color{TitlingRed}\large\bfseries\text{Part }}
{\bfseries\textcolor{TitlingRed}{\contentslabel{0.0em}}\hspace*{1.35em}}
{}
{\textcolor{TitlingRed}{{\hfill\bfseries\contentspage\nobreak}}}
[]
\titlecontents{section}
[0.0em]
{\addvspace{1pc}}
{\color{black}\bfseries\textcolor{TitlingRed}{\contentslabel{0.0em}}\hspace*{1.35em}}
{}
{\textcolor{black}{\textbf{\DotFill}{\bfseries\contentspage\nobreak}}}
[]
\titlecontents{subsection}
[0.0em]
{}
{\hspace*{1.35em}\color{ToCGrey}{\contentslabel{0.0em}}\hspace*{2.1em}}
{}
{{\textcolor{ToCGrey}\DotFill}\textcolor{ToCGrey}{\contentspage}\nobreak}
[]
\usepackage{marginnote}
\renewcommand*{\marginfont}{\normalfont}
\usepackage{inconsolata}
\setmonofont{inconsolata}%
\let\ChapterRef\relax
\newcommand{\ChapterRef}[2]{#1}
\AtBeginEnvironment{subappendices}{%%
    \section*{\huge Appendices}%
}%

\begin{document}

\title{Introduction}

\maketitle

\phantomsection
\label{section-phantom}

This chapter contains some general information about the Clowder Project.

\ChapterTableOfContents

\section{Introduction}\label{section-introduction}
\subsection{Goals}\label{subsection-goals}
The Clowder Project has the following goals:
\begin{enumerate}
    \item\label{subsection-goals-item-1}\SloganFont{Homogenize Conventions, Notation, and Terminology. }
    \item\label{subsection-goals-item-2}\SloganFont{Fill Gaps in the Category Theory Literature. }There are quite a few significant gaps in the category theory literature, and we hope to fill some of them with the Clowder Project. For a list of some of them, see \cref{subsection-list-of-gaps-in-the-category-theory-literature}.
    \item\label{subsection-goals-item-3}\SloganFont{. }
\end{enumerate}
\subsection{Prerequisites/Assumed Background}\label{subsection-prerequisites-assumed-background}
\subsection{Content and Scope}\label{subsection-scope}
In this section, we outline what content is expected to be covered in the Clowder Project.
\subsubsection{Elementary Category Theory}\label{subsubsection-elementary-category-theory}
First and foremost, the Clowder Project aims to cover the foundations of category theory. This comprises all the usual topics treated in basic textbooks in category theory, such as \cite{categories-for-the-working-mathematician} or \cite{category-theory-in-context}, like adjunctions, co/limits, Kan extensions, co/ends, monoidal categories, etc.
\subsection{Community Engagement and Collaboration}\label{subsection-community-engagement-and-collaboration}
\section{Miscellany}\label{section-miscellany}
\subsection{Infrastructure and Technical Implementation}\label{subsection-infrastructure-and-technical-implementation}
\subsection{List of Gaps in the Category Theory Literature}\label{subsection-list-of-gaps-in-the-category-theory-literature}
The Clowder Project aims to address several significant gaps in the existing literature on category theory, as detailed below. See also \cite{MO494959}.
\begin{gap}{The Tensor Product of Presentable Categories}{gap-the-tensor-product-of-presentable-categories}%
    Even though its analogue for $\infty$-categories has for years been a widely used tool%
    %--- Begin Footnote ---%
    \footnote{%
        See \cite{MO490557}.
        \par\vspace*{\TCBBoxCorrection}
    }, %
    %---  End Footnote  ---%
    a comprehensive treatment of the tensor product of presentable categories seems to be currently missing.
\end{gap}
\begin{gap}{Explicit Descriptions of Co/Limits of Categories}{gap-explicit-descriptions-of-co-limits-of-categories}%
    An exhaustive concrete description of the various limits and colimits of categories, including 2-dimensional ones, is missing.
\end{gap}
\begin{gap}{Dinatural Transformation Co/Classifiers}{gap-dinatural-transformation-co-classifiers}%
    There seems to be no unified presentation of dinatural transformation co/classifiers in the literature. These are characterised by isomorphisms of the form
    \[
        \Nat(F,G)%
        \cong%
        \DiNat(\Gamma(F),G),%
    \]%
    and were originally studied in Dubuc--Street's paper introducing dinatural transformations, \cite{dubuc-street-dinatural-transformations}.

    \indent Even though these arguably form a fundamental piece of the framework of co/end calculus, it seems that all foundational treatments that followed after ended up not covering this concept.
\end{gap}
\begin{gap}{The Tensor Product of Symmetric Monoidal Categories}{gap-the-tensor-product-of-symmetric-monoidal-categories}%
    The tensor product of symmetric monoidal categories had been a missing concept from the literature for years. Recently, \cite{the-symmetric-monoidal-2-category-of-permutative-categories} covered the case of permutative categories. It would be nice, however, to also have a treatment of the non-strict case available.
\end{gap}
\begin{gap}{A Comprehensive Treatment of the Theory of Promonoidal Categories}{gap-a-comprehensive-treatment-of-the-theory-of-promonoidal-categories}%
    A comprehensive and exhaustive treatment of the theory of promonoidal categories is currently missing. There are several important notions undefined, like:
    \begin{itemize}
        \item Promonoidal profunctors.
        \item Dualisability internal to a promonoidal category.
        \item Invertibility internal to a promonoidal category.
    \end{itemize}
    Moreover, it would be nice to record how promonoidal categories may be viewed as categorifications of \say{hypermonoids} (i.e.\ monoids in $\Rel$).
\end{gap}
\begin{gap}{A Comprehensive Treatment of the Theory of Multicategories}{gap-a-comprehensive-treatment-of-the-theory-of-multicategories}%
    A comprehensive and exhaustive treatment of the theory of multicategories is currently missing. There are several important notions undefined, like:
    \begin{itemize}
        \item Co/limits internal to multicategories.
    \end{itemize}
    See \cite{MO484647}.
\end{gap}
\begin{gap}{A Compendium of Examples of 2-Categorical Notions}{gap-a-compendium-of-examples-of-2-categorical-notions}%
    It would be nice to have an extensive collection of examples of what a given 2-categorical notion looks like in a 2-category. For instance, it would be nice to explicitly list what internal adjunctions look like in $\sfbfRel$, $\Span$, $\Prof$, etc.

    \indent See \ChapterRef{\ChapterRelations, \cref{relations:section-properties-of-the-2-category-of-relations}}{\cref{section-properties-of-the-2-category-of-relations}} for a concrete example of what is meant by this gap.
\end{gap}
\begin{gap}{Centres and Traces of Categories}{gap-centres-and-traces-of-categories}%
    The literature on centres and traces of categories is really small. There are lots of results missing%
    %--- Begin Footnote ---%
    \footnote{%
        E.g.\ There's a certain interaction between traces of categories and Leinster's eventual image.
    } %
    %---  End Footnote  ---%
    and very few worked examples%
    %--- Begin Footnote ---%
    \footnote{%
        E.g.\ what is the trace of Connes's cycle category? Such a computation doesn't seem to be available.
        \par\vspace*{\TCBBoxCorrection}
    }.%
    %---  End Footnote  ---%
\end{gap}
\begin{gap}{Natural Cotransformations}{gap-natural-cotransformations}%
    Natural transformations satisfy an isomorphism of the form
    \[
        \Nat(F,G)%
        \cong%
        \int_{A\in\CatFont{C}}\Hom_{\CatFont{D}}(F_{A},G_{A}).%
    \]%
    It is then exceedingly natural to define \textit{natural cotransformations} via an isomorphism of the form
    \[
        \CoNat(F,G)%
        \cong%
        \int^{A\in\CatFont{C}}\Hom_{\CatFont{D}}(F_{A},G_{A})%
    \]%
    and study their properties. This generalises traces of categories, since we have
    \[
        \Tr(\CatFont{C})%
        =%
        \CoNat(\id_{\CatFont{C}},\id_{\CatFont{C}}),
    \]%
    much like $\rmZ(\CatFont{C})=\Nat(\id_{\CatFont{C}},\id_{\CatFont{C}})$.
\end{gap}
\begin{gap}{Isbell Duality}{gap-isbell-duality}%
    There are several results and examples in the theory of Isbell duality missing from the literature, with a comprehensive treatment still lacking.
\end{gap}
\begin{gap}{A Comprehensive Theory of 2-Dimensional Co/Ends}{gap-a-comprehensive-theory-of-2-dimensional-co-ends}%
    The currently available treatments of 2-dimensional co/ends are unsatisfactory.%
    %--- Begin Footnote ---%
    \footnote{%
        For instance, none of them define 2-dimensional co/ends via 2-dimensional dinatural transformations and then go on to develop a general theory from there.
        \par\vspace*{\TCBBoxCorrection}
    }%
    %---  End Footnote  ---%
\end{gap}
\begin{gap}{A Comprehensive treatment of Factorisation Systems}{gap-a-comprehensive-treatment-of-factorisation-systems}%
    A comprehensive treatment of factorisation systems is currently missing; see \cite{MO495003}.
\end{gap}
\begin{appendices}
\begin{multicols}{2}[\section{Other Chapters}]
\noindent
\textbf{Preliminaries}
\begin{enumerate}
\item \hyperref[introduction:section-phantom]{Introduction}
\end{enumerate}
\textbf{Sets}
\begin{enumerate}
\setcounter{enumi}{2}
\item \hyperref[sets:section-phantom]{Sets}
\item \hyperref[constructions-with-sets:section-phantom]{Constructions With Sets}
\item \hyperref[monoidal-structures-on-the-category-of-sets:section-phantom]{Monoidal Structures on the Category of Sets}
\item \hyperref[pointed-sets:section-phantom]{Pointed Sets}
\item \hyperref[tensor-products-of-pointed-sets:section-phantom]{Tensor Products of Pointed Sets}
\end{enumerate}
\textbf{Relations}
\begin{enumerate}
\setcounter{enumi}{6}
\item \hyperref[relations:section-phantom]{Relations}
\item \hyperref[constructions-with-relations:section-phantom]{Constructions With Relations}
\item \hyperref[conditions-on-relations:section-phantom]{Conditions on Relations}
\end{enumerate}
\textbf{Category Theory}
\begin{enumerate}
\setcounter{enumi}{9}
\item \hyperref[categories:section-phantom]{Categories}
\end{enumerate}
\textbf{Monoidal Categories}
\begin{enumerate}
\setcounter{enumi}{10}
\item \hyperref[constructions-with-monoidal-categories:section-phantom]{Constructions With Monoidal Categories}
\end{enumerate}
\textbf{Bicategories}
\begin{enumerate}
\setcounter{enumi}{11}
\item \hyperref[types-of-morphisms-in-bicategories:section-phantom]{Types of Morphisms in Bicategories}
\end{enumerate}
\textbf{Extra Part}
\begin{enumerate}
\setcounter{enumi}{12}
\item \hyperref[notes:section-phantom]{Notes}
\end{enumerate}
\end{multicols}

\end{appendices}
\end{document}
