\usepackage{etoolbox}
\makeatletter
\displaywidth=\textwidth
\displayindent=-\leftskip
\patchcmd\start@gather{$$}{%
  $$%
  \displaywidth=\textwidth
  \displayindent=-\leftskip
}{}{\errmessage{Cannot patch \string\start@gather}}
\patchcmd\start@align{$$}{%
  $$%
  \displaywidth=\textwidth
  \displayindent=-\leftskip
}{}{\errmessage{Cannot patch \string\start@align}}
\patchcmd\start@multline{$$}{%
  $$%
  \displaywidth=\textwidth
  \displayindent=-\leftskip
}{}{\errmessage{Cannot patch \string\start@multline}}
\patchcmd\mathdisplay{$$}{%
  $$%
  \displaywidth=\textwidth
  \displayindent=-\leftskip
}{}{\errmessage{Cannot patch \string\mathdisplay}}
\makeatother
\makeatletter
\newcommand{\displaybump}{\hbox to \@totalleftmargin{\hfil}}
\makeatother
\usepackage{pdfpages}
\usepackage[page]{appendix}
\AtBeginEnvironment{subappendices}{%%
    \section*{\huge Appendices}\vspace{-0.75\baselineskip}%
}%
\usepackage{makecell}
\usepackage{cellspace}
\let\mathrm\relax
\let\mathbf\relax
\newcommand{\mathrm}[1]{\text{#1}}
\newcommand{\mathbf}[1]{\textbf{#1}}
\newcommand{\GrC}[2]{
    \mathchoice%
    {\scalebox{1.0}{$\textstyle\int^{#1}#2$}}%
    {\scalebox{1.0}{$\textstyle\int^{#1}#2$}}%
    {\scalebox{0.75}{$\textstyle\int^{#1}#2$}}%
    {\scalebox{0.6}{$\textstyle\int^{#1}#2$}}%
}
\newcommand{\vGrC}[2]{
    \mathchoice%
    {\scalebox{1.0}{$\textstyle\int_{#1}#2$}}%
    {\scalebox{1.0}{$\textstyle\int_{#1}#2$}}%
    {\scalebox{0.75}{$\textstyle\int_{#1}#2$}}%
    {\scalebox{0.6}{$\textstyle\int_{#1}#2$}}%
}
\usepackage{stackengine}
\usepackage{scalerel}
\newcommand\pig[1]{\scalerel*[5pt]{\big#1}{\ensurestackMath{\addstackgap[1.5pt]{\big#1}}}}
\newcommand\pigl[1]{\mathopen{\pig{#1}}}
\newcommand\pigr[1]{\mathclose{\pig{#1}}}
\usepackage{epigraph,xpatch}

\epigraphnoindent

\makeatletter
\newcommand{\doubleepigraph}[4]{%
  \vspace{\beforeepigraphskip}
  \vbox{%
    \xpatchcmd{\@epitext}{{minipage}}{{minipage}[t]}{}{}%
    \epigraphsize
    \begin{\epigraphflush}
    \begin{minipage}[b]{\epigraphwidth}
      \@epitext{#1}\\
      \@episource{\begin{flushleft}#2\strut\end{flushleft}}%
    \end{minipage}\hfill
    \begin{minipage}[b]{\epigraphwidth}
      \@epitext{#3}\\
      \@episource{#4\strut}%
    \end{minipage}%
    \end{\epigraphflush}%
  }%
  %\nointerlineskip
  %\vspace*{\afterepigraphskip}%
  %\ifepigraphnoindent\@afterheading\fi
}
\makeatother

\setlength{\epigraphwidth}{0.45\textwidth}
\tikzcdset{%
    prompter/.code={%
        \tikzset{%
            nodes in empty cells=false,cells={nodes={draw,execute at end node={\makebox[0pt][l]{$~{}_{%
                \smash{%
                    \raisebox{-0.75em}{
                        \colorbox{OIblue}{%
                            \textcolor{white}{%
                                \textbf{\textsf{\the\pgfmatrixcurrentrow-\the\pgfmatrixcurrentcolumn}}%
                            }%
                        }%
                    }%
                }%
            }$}}}}%
        }%
    }%
}%
\tikzcdset{%
    prompter all/.code={%
        \tikzset{%
            nodes in empty cells=true,cells={nodes={draw,execute at end node={\makebox[0pt][l]{$~{}_{%
                \smash{%
                    \raisebox{-0.75em}{
                        \colorbox{OIblue}{%
                            \textcolor{white}{%
                                \textbf{\textsf{\the\pgfmatrixcurrentrow-\the\pgfmatrixcurrentcolumn}}%
                            }%
                        }%
                    }%
                }%
            }$}}}}%
        }%
    }%
}%
\usepackage{fp}
\newcommand{\fsize}[2]{{\fontsize{#1pt}{#1pt * \real{1.2}}\selectfont #2}}
\newcommand{\ffsize}[1]{\fontsize{#1pt}{#1pt * \real{1.2}}\selectfont}
\usepackage{stmaryrd}
\newcommand{\bfLUnitor}{\boldsymbol{\lambda}}
\newcommand{\bfRUnitor}{\boldsymbol{\rho}}
\newcommand{\bfsigma}{\boldsymbol{\sigma}}
\newcommand{\bfbeta}{\boldsymbol{\beta}}
\newcommand{\bfalpha}{\boldsymbol{\alpha}}
\newcommand{\textiff}{iff\xspace}
\newcommand{\OrdinalCategory}{\mathbbold{n}}
\newcommand{\OrdinalCategoryNPlus}[2]{\mathbbold{#1}+\mathbbold{#2}}
\newcommand{\OrdinalCategoryN}[1]{\mathbbold{#1}}
\newcommand{\Unit}{\mathbbold{1}}
\newfontfamily\IPAFont[Path=ABSOLUTEPATH/fonts/brill/,Scale=1.0]{Brill-Roman.ttf}
\newcommand{\IPA}[1]{{\text{\IPAFont{#1}}}}
