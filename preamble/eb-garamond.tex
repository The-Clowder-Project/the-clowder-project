\usepackage{fontspec}
\usepackage{inconsolata}
\setmonofont{inconsolata}%
\usepackage[protrusion=true,expansion,tracking=true,babel]{microtype}
\usepackage[cmintegrals,cmbraces]{newtxmath}
\usepackage{ebgaramond-maths}
\newcommand{\Russian}[1]{#1}
\usepackage[margin=4.5cm]{geometry}
% https://tex.stackexchange.com/questions/215270/can-someone-explain-this-weird-font-behavior-ebgaramond-maths
\makeatletter
  \DeclareSymbolFont{ntxletters}{OML}{ntxmi}{m}{it}
  \SetSymbolFont{ntxletters}{bold}{OML}{ntxmi}{b}{it}
  \re@DeclareMathSymbol{\leftharpoonup}{\mathrel}{ntxletters}{"28}
  \re@DeclareMathSymbol{\leftharpoondown}{\mathrel}{ntxletters}{"29}
  \re@DeclareMathSymbol{\rightharpoonup}{\mathrel}{ntxletters}{"2A}
  \re@DeclareMathSymbol{\rightharpoondown}{\mathrel}{ntxletters}{"2B}
  \re@DeclareMathSymbol{\triangleleft}{\mathbin}{ntxletters}{"2F}
  \re@DeclareMathSymbol{\triangleright}{\mathbin}{ntxletters}{"2E}
  \re@DeclareMathSymbol{\partial}{\mathord}{ntxletters}{"40}
  \re@DeclareMathSymbol{\flat}{\mathord}{ntxletters}{"5B}
  \re@DeclareMathSymbol{\natural}{\mathord}{ntxletters}{"5C}
  \re@DeclareMathSymbol{\star}{\mathbin}{ntxletters}{"3F}
  \re@DeclareMathSymbol{\smile}{\mathrel}{ntxletters}{"5E}
  \re@DeclareMathSymbol{\frown}{\mathrel}{ntxletters}{"5F}
  \re@DeclareMathSymbol{\sharp}{\mathord}{ntxletters}{"5D}
  \re@DeclareMathAccent{\vec}{\mathord}{ntxletters}{"7E}
\makeatother
\newfontfamily\EBGaramondBBFont[Path = ABSOLUTEPATH/fonts/]{Garamond-Math.otf}
\newcommand{\EBGaramondBB}[1]{{\text{\EBGaramondBBFont{#1}}}}
\let\mathbb\relax
\newcommand{\mathbb}[1]{\text{\EBGaramondBB{#1}}}
\defaultfontfeatures{RawFeature={+axis={wght=450}}}
\newfontfamily\YsabeauFont[Path = ABSOLUTEPATH/fonts/ysabeau/]{Ysabeau.ttf}
\newcommand{\Ysabeau}[1]{{\text{\YsabeauFont{#1}}}}
\let\mathsf\relax
\newcommand{\mathsf}[1]{\text{\YsabeauFont{#1}}}
%
\renewcommand{\Unit}{\mathbb{1}}
\renewcommand{\Zero}{\mathbb{0}}
\renewcommand{\OrdinalCategory}{\mathbb{n}}
\renewcommand{\OrdinalCategoryNPlus}[2]{\mathbb{#1}+\mathbb{#2}}
\renewcommand{\OrdinalCategoryN}[1]{\mathbb{#1}}
