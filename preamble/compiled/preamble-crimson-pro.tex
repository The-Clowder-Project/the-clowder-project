\IfFileExists{stacks-project.cls}{\documentclass{stacks-project}}{\documentclass{amsart}}
% https://plastex.github.io/plastex/plastex/sect0006.html
\newif\ifplastex
\plastexfalse
%
\ifplastex
\else
    \let\secS\S
    \let\S\relax
    \let\parP\P
    \let\P\relax
\fi
%
\usepackage{verbatim}
\newenvironment{reference}{\comment}{\endcomment}
\newenvironment{slogan}{\comment}{\endcomment}
\newenvironment{history}{\comment}{\endcomment}
\usepackage{multicol}

% Colours
\usepackage[table]{xcolor}
\newcommand{\tbw}[1]{\textcolor{white}{#1}}
\definecolor{tableColor}{rgb}{0.5,0.2,0.0}
\definecolor{TitlingRed}{RGB}{186,16,0}
\definecolor{paperColor}{RGB}{255,255,255}%\definecolor{paperColor}{RGB}{255,253,235}
\colorlet{backgroundColor}{paperColor}
%-------------------------------------------------%
% ╔═╗╦╔═╔═╗╔╗ ╔═╗  ╦╔╦╗╔═╗  ╔═╗╔═╗╦  ╔═╗╔╦╗╔╦╗╔═╗ %
% ║ ║╠╩╗╠═╣╠╩╗║╣───║ ║ ║ ║  ╠═╝╠═╣║  ║╣  ║  ║ ║╣  %
% ╚═╝╩ ╩╩ ╩╚═╝╚═╝  ╩ ╩ ╚═╝  ╩  ╩ ╩╩═╝╚═╝ ╩  ╩ ╚═╝ %
%-------------------------------------------------%
\definecolor{OIblue}{RGB}{0,114,178}
\definecolor{OIgreen}{RGB}{1,158,115}
\definecolor{OIorange}{RGB}{230,159,0}
\definecolor{OIreddishPurple}{RGB}{204,121,167}
\definecolor{OIskyBlue}{RGB}{86,180,233}
\definecolor{OIvermillion}{RGB}{213,94,0}
\definecolor{OIyellow}{RGB}{240,228,66}

% This is generally recommended for better output
\usepackage[english]{babel}
\usepackage[protrusion=true,expansion,tracking=true,babel]{microtype}

% For cross-file-references
\ifplastex
\else
    \usepackage{nameref,zref-xr}
\zxrsetup{toltxlabel}
% Sets
\zexternaldocument*[sets:]{sets}
\zexternaldocument*[constructions-with-sets:]{constructions-with-sets}
\zexternaldocument*[monoidal-structures-on-the-category-of-sets:]{monoidal-structures-on-the-category-of-sets}
\zexternaldocument*[pointed-sets:]{pointed-sets}
\zexternaldocument*[tensor-products-of-pointed-sets:]{tensor-products-of-pointed-sets}
% Relations
\zexternaldocument*[relations:]{relations}
\zexternaldocument*[constructions-with-relations:]{constructions-with-relations}
\zexternaldocument*[conditions-on-relations:]{conditions-on-relations}
% Preorders, Partial Orders, and Posets
\zexternaldocument*[preorders:]{preorders}
\zexternaldocument*[posets:]{posets}
% Graphs
\zexternaldocument*[graphs:]{graphs}
% Indexed and Fibred Sets
\zexternaldocument*[indexed-sets:]{indexed-sets}
\zexternaldocument*[fibred-sets:]{fibred-sets}
\zexternaldocument*[un-straightening-for-indexed-and-fibred-sets:]{un-straightening-for-indexed-and-fibred-sets}
% Categories
\zexternaldocument*[categories:]{categories}
\zexternaldocument*[constructions-with-categories:]{constructions-with-categories}
\zexternaldocument*[presheaves-and-the-yoneda-lemma:]{presheaves-and-the-yoneda-lemma}
\zexternaldocument*[adjunctions:]{adjunctions}
\zexternaldocument*[limits-and-colimits:]{limits-and-colimits}
\zexternaldocument*[ends-and-coends:]{ends-and-coends}
\zexternaldocument*[kan-extensions:]{kan-extensions}
\zexternaldocument*[centres-and-traces-of-categories:]{centres-and-traces-of-categories}
% Monoidal Categories
\zexternaldocument*[monoidal-categories:]{monoidal-categories}
\zexternaldocument*[constructions-with-monoidal-categories:]{constructions-with-monoidal-categories}
% Higher Category Theory
\zexternaldocument*[types-of-morphisms-in-bicategories:]{types-of-morphisms-in-bicategories}
\zexternaldocument*[internal-adjunctions:]{internal-adjunctions}
% Monoids
\zexternaldocument*[monoid-actions:]{monoid-actions}
% Groups
\zexternaldocument*[groups:]{groups}
% Preordered Algebraic Structures
\zexternaldocument*[preordered-monoids:]{preordered-monoids}
\zexternaldocument*[preordered-groups:]{preordered-groups}
\zexternaldocument*[preordered-rings:]{preordered-rings}
\zexternaldocument*[preordered-vector-spaces:]{preordered-vector-spaces}
% *-Algebras
\zexternaldocument*[star-rings:]{star-rings}
\zexternaldocument*[star-modules:]{star-modules}
\zexternaldocument*[star-algebras:]{star-algebras}
% Representation Theory
\zexternaldocument*[representation-theory-of-groups:]{representation-theory-of-groups}
% Topology
\zexternaldocument*[topological-spaces:]{topological-spaces}
\zexternaldocument*[constructions-with-topological-spaces:]{constructions-with-topological-spaces}
\zexternaldocument*[topologies-on-powersets:]{topologies-on-powersets}
\zexternaldocument*[conditions-on-topological-spaces:]{conditions-on-topological-spaces}
\zexternaldocument*[sheaves-on-topological-spaces:]{sheaves-on-topological-spaces}
\zexternaldocument*[topological-stacks:]{topological-stacks}
\zexternaldocument*[locales:]{locales}
\zexternaldocument*[uniform-spaces:]{uniform-spaces}
\zexternaldocument*[metric-spaces:]{metric-spaces}
% Topological Algebra
\zexternaldocument*[topological-monoids:]{topological-monoids}
\zexternaldocument*[topological-groups:]{topological-groups}
% Real Analysis
\zexternaldocument*[real-analysis:]{real-analysis}
% Measure Theory
\zexternaldocument*[measurable-spaces:]{measurable-spaces}
\zexternaldocument*[measures-and-integration:]{measures-and-integration}
% Functional Analysis
\zexternaldocument*[topological-vector-spaces:]{topological-vector-spaces}
\zexternaldocument*[c-star-algebras:]{c-star-algebras}
\zexternaldocument*[distributions:]{distributions}
% Differential Geometry
\zexternaldocument*[topological-manifolds:]{topological-manifolds}
\zexternaldocument*[smooth-manifolds:]{smooth-manifolds}
% Simplicial Stuff
\zexternaldocument*[the-simplex-category:]{the-simplex-category}
% Cyclic Stuff
\zexternaldocument*[the-cycle-category:]{the-cycle-category}
% Cubical Stuff
\zexternaldocument*[the-cube-category:]{the-cube-category}
% Cellular Stuff
\zexternaldocument*[the-cell-category:]{the-cell-category}
% Globular Stuff
\zexternaldocument*[the-globe-category:]{the-globe-category}
% Other
\zexternaldocument*[research-sandbox:]{research-sandbox}
%%%%%%%%%%%%%%%%%%%%%%%%%%%%%%%%%%%%%%%%%%%%%%%
%% ┌─┐┬ ┬┌─┐┌─┐┌┬┐┌─┐┬─┐  ┌┬┐┌─┐┌─┐┬─┐┌─┐┌─┐ %%
%% │  ├─┤├─┤├─┘ │ ├┤ ├┬┘  │││├─┤│  ├┬┘│ │└─┐ %%
%% └─┘┴ ┴┴ ┴┴   ┴ └─┘┴└─  ┴ ┴┴ ┴└─┘┴└─└─┘└─┘ %%
%%%%%%%%%%%%%%%%%%%%%%%%%%%%%%%%%%%%%%%%%%%%%%%
% Sets
\newcommand{\ChapterSets}{\hyperref[sets:section-phantom]{Sets}\xspace}
\newcommand{\ChapterConstructionsWithSets}{\hyperref[constructions-with-sets:section-phantom]{Constructions With Sets}\xspace}
\newcommand{\ChapterMonoidalStructuresOnTheCategoryOfSets}{\hyperref[monoidal-structures-on-the-category-of-sets:section-phantom]{Monoidal Structures on the Category of Sets}\xspace}
\newcommand{\ChapterPointedSets}{\hyperref[pointed-sets:section-phantom]{Pointed Sets}\xspace}
\newcommand{\ChapterTensorProductsOfPointedSets}{\hyperref[tensor-products-of-pointed-sets:section-phantom]{Tensor Products of Pointed Sets}\xspace}
% Relations
\newcommand{\ChapterRelations}{\hyperref[relations:section-phantom]{Relations}\xspace}
\newcommand{\ChapterConstructionsWithRelations}{\hyperref[constructions-with-relations:section-phantom]{Constructions With Relations}\xspace}
\newcommand{\ChapterConditionsOnRelations}{\hyperref[conditions-on-relations:section-phantom]{Conditions on Relations}\xspace}
% Preorders, Partial Orders, and Posets
\newcommand{\ChapterPreorders}{\hyperref[preorders:section-phantom]{Preorders}\xspace}
\newcommand{\ChapterPosets}{\hyperref[posets:section-phantom]{Posets}\xspace}
\newcommand{\ChapterPreordersPartialOrdersAndPosets}{\hyperref[preorders-partial-orders-and-posets:section-phantom]{Preorders, Partial Orders, and Posets}\xspace}
% Graphs
\newcommand{\ChapterGraphs}{\hyperref[graphs:section-phantom]{Graphs}\xspace}
% Indexed and Fibred Sets
\newcommand{\ChapterIndexedSets}{\hyperref[indexed-sets:section-phantom]{Indexed Sets}\xspace}
\newcommand{\ChapterFibredSets}{\hyperref[fibred-sets:section-phantom]{Fibred Sets}\xspace}
\newcommand{\ChapterUnStraighteningForIndexedAndFibredSets}{\hyperref[un-straightening-forindexed-and-fibred-sets:section-phantom]{Un/Straightening for Indexed and Fibred Sets}\xspace}
% Categories
\newcommand{\ChapterCategories}{\hyperref[categories:section-phantom]{Categories}\xspace}
\newcommand{\ChapterConstructionsWithCategories}{\hyperref[constructions-with-categories:section-phantom]{Constructions With Categories}\xspace}
\newcommand{\ChapterPresheavesAndTheYonedaLemma}{\hyperref[presheaves-and-the-yoneda-lemma:section-phantom]{Presheaves and the Yoneda Lemma}\xspace}
\newcommand{\ChapterAdjunctions}{\hyperref[adjunctions:section-phantom]{Adjunctions}\xspace}
\newcommand{\ChapterLimitsAndColimits}{\hyperref[limits-and-colimits:section-phantom]{Limits and Colimits}\xspace}
\newcommand{\ChapterEndsAndCoends}{\hyperref[ends-and-coends:section-phantom]{Ends and Coends}\xspace}
\newcommand{\ChapterKanExtensions}{\hyperref[kan-extensions:section-phantom]{Kan Extensions}\xspace}
\newcommand{\ChapterCentresAndTracesOfCategories}{\hyperref[centres-and-traces-of-categories:section-phantom]{Centres and Traces of Categories}\xspace}
% Monoidal Categories
\newcommand{\ChapterMonoidalCategories}{\hyperref[monoidal-categories:section-phantom]{Monoidal Categories}\xspace}
\newcommand{\ChapterConstructionsWithMonoidalCategories}{\hyperref[constructions-with-monoidal-categories:section-phantom]{Constructions With Monoidal Categories}\xspace}
% Higher Category Theory
\newcommand{\ChapterTypesOfMorphismsInBicategories}{\hyperref[types-of-morphisms-in-bicategories:section-phantom]{Types of Morphisms in Bicategories}\xspace}
\newcommand{\ChapterInternalAdjunctions}{\hyperref[internal-adjunctions:section-phantom]{Internal Adjunctions}\xspace}
% Monoids
\newcommand{\ChapterMonoidActions}{\hyperref[monoid-actions:section-phantom]{Monoid Actions}\xspace}
% Groups
\newcommand{\ChapterGroups}{\hyperref[groups:section-phantom]{Groups}\xspace}
% Preordered Algebraic Structures
\newcommand{\ChapterPreorderedMonoids}{\hyperref[preordered-monoids:section-phantom]{Preordered Monoids}\xspace}
\newcommand{\ChapterPreorderedGroups}{\hyperref[preordered-groups:section-phantom]{Preordered Groups}\xspace}
\newcommand{\ChapterPreorderedRings}{\hyperref[preordered-rings:section-phantom]{Preordered Rings}\xspace}
\newcommand{\ChapterPreorderedVectorSpaces}{\hyperref[preordered-vector-spaces:section-phantom]{Preordered Vector Spaces}\xspace}
% *-Algebras
\newcommand{\ChapterStarRings}{\hyperref[star-rings:section-phantom]{*-Rings}\xspace}
\newcommand{\ChapterStarModules}{\hyperref[star-modules:section-phantom]{*-Modules}\xspace}
\newcommand{\ChapterStarAlgebras}{\hyperref[star-algebras:section-phantom]{*-Algebras}\xspace}
% Representation Theory
\newcommand{\ChapterRepresentationTheoryOfGroups}{\hyperref[representation-theory-of-groups:section-phantom]{Representation Theory of Groups}\xspace}
% Topology
\newcommand{\ChapterTopologicalSpaces}{\hyperref[topological-spaces:section-phantom]{Topological Spaces}\xspace}
\newcommand{\ChapterConstructionsWithTopologicalSpaces}{\hyperref[constructions-with-topological-spaces:section-phantom]{Constructions With Topological Spaces}\xspace}
\newcommand{\ChapterTopologiesOnPowersets}{\hyperref[topologies-on-powersets:section-phantom]{Topologies on Powersets}\xspace}
\newcommand{\ChapterConditionsOnTopologicalSpaces}{\hyperref[conditions-on-topological-spaces:section-phantom]{Conditions on Topological Spaces}\xspace}
\newcommand{\ChapterSheavesOnTopologicalSpaces}{\hyperref[sheaves-on-topological-spaces:section-phantom]{Sheaves on Topological Spaces}\xspace}
\newcommand{\ChapterTopologicalStacks}{\hyperref[topological-stacks:section-phantom]{Topological Stacks}\xspace}
\newcommand{\ChapterLocales}{\hyperref[locales:section-phantom]{Locales}\xspace}
\newcommand{\ChapterUniformSpaces}{\hyperref[uniform-spaces:section-phantom]{Uniform Spaces}\xspace}
\newcommand{\ChapterMetricSpaces}{\hyperref[metric-spaces:section-phantom]{Metric Spaces}\xspace}
% Topological Algebra
\newcommand{\ChapterTopologicalMonoids}{\hyperref[topological-monoids:section-phantom]{Topological Monoids}\xspace}
\newcommand{\ChapterTopologicalGroups}{\hyperref[topological-groups:section-phantom]{Topological Groups}\xspace}
% Real Analysis
\newcommand{\ChapterRealAnalysis}{\hyperref[real-analysis:section-phantom]{Real Analysis}\xspace}
% Measure Theory
\newcommand{\ChapterMeasurableSpaces}{\hyperref[measurable-spaces:section-phantom]{Measurable Spaces}\xspace}
\newcommand{\ChapterMeasuresAndIntegration}{\hyperref[measures-and-integration:section-phantom]{Measures and Integration}\xspace}
% Functional Analysis
\newcommand{\ChapterTopologicalVectorSpaces}{\hyperref[topological-vector-spaces:section-phantom]{Topological Vector Spaces}\xspace}
\newcommand{\ChapterCStarAlgebras}{\hyperref[c-star-algebras:section-phantom]{$\rmC^{*}$-Algebras}\xspace}
\newcommand{\ChapterDistributions}{\hyperref[distributions:section-phantom]{Distributions}\xspace}
% Riesz Spaces
\newcommand{\ChapterRieszSpaces}{\hyperref[riesz-spaces:section-phantom]{Riesz Spaces}\xspace}
% Differential Geometry
\newcommand{\ChapterTopologicalManifolds}{\hyperref[topological-manifolds:section-phantom]{Topological Manifolds}\xspace}
\newcommand{\ChapterSmoothManifolds}{\hyperref[smooth-manifolds:section-phantom]{Smooth Manifolds}\xspace}
% Simplicial Stuff
\newcommand{\ChapterTheSimplexCategory}{\hyperref[the-simplex-category:section-phantom]{The Simplex Category}\xspace}
% Cyclic Stuff
\newcommand{\ChapterTheCycleCategory}{\hyperref[the-cycle-category:section-phantom]{The Cycle Category}\xspace}
% Cubical Stuff
\newcommand{\ChapterTheCubeCategory}{\hyperref[the-cube-category:section-phantom]{The Cube Category}\xspace}
% Cellular Stuff
\newcommand{\ChapterTheCellCategory}{\hyperref[the-cell-category:section-phantom]{The Cell Category}\xspace}
% Globular Stuff
\newcommand{\ChapterTheGlobeCategory}{\hyperref[the-globe-category:section-phantom]{The Globe Category}\xspace}
% Temporary Stuff
\newcommand{\ChapterMonoidsInMonoidalInftyCategories}{\cref{TODO}}
\newcommand{\ChapterAdjunctionsAndTheYonedaLemma}{\cref{TODO}}
\newcommand{\ChapterBicategories}{\cref{TODO}}
\newcommand{\ChapterDiagonalCategoryTheory}{\cref{TODO}}
\newcommand{\ChapterHypergroups}{\cref{TODO}}
\newcommand{\ChapterHypermonoids}{\cref{TODO}}
\newcommand{\ChapterMonads}{\cref{TODO}}
\newcommand{\ChapterModulesInMonoidalCategories}{\cref{TODO}}
\newcommand{\ChapterMonadsAndComonads}{\cref{TODO}}
\newcommand{\ChapterEnrichedPresheavesAndTheEnrichedYonedaLemma}{\cref{TODO}}
\newcommand{\ChapterModulesInMonoidalInftyCategories}{\cref{TODO}}
\newcommand{\ChapterMonoidsWithZero}{\cref{TODO}}
\newcommand{\ChapterPreordersAndPartialOrders}{\cref{TODO}}
\newcommand{\ChapterProfunctors}{\cref{TODO}}
\newcommand{\ChapterSpans}{\cref{TODO}}
\newcommand{\ChapterTensorProductsOfMonoids}{\cref{TODO}}
\newcommand{\ChapterTypesOfMorphismsInCategories}{\cref{TODO}}
\newcommand{\ChapterSimplicialObjects}{\cref{TODO}}
\newcommand{\ChapterSimplicialCategories}{\cref{TODO}}
\newcommand{\ChapterSimplicialHomotopyTheory}{\cref{TODO}}

\fi
\newlength{\HexagonLength}
\newlength{\OctagonLength}
\newlength{\PromonoidalLength}
\newlength{\edgelengthforpentagondiagramsmall}
\newlength{\edgelengthforpentagondiagram}
\newlength{\edgelengthinproofofyonedalemmaforsieves}
\newlength{\edgelength}
\setlength{\HexagonLength}{4cm}
\setlength{\OctagonLength}{3cm}
\setlength{\PromonoidalLength}{3.25cm}
\setlength{\edgelengthforpentagondiagramsmall}{4.35cm}
\setlength{\edgelengthforpentagondiagram}{5.0cm}
\setlength{\edgelengthinproofofyonedalemmaforsieves}{2.5cm}
\setlength{\edgelength}{3.5cm}
\newlength{\OneCm}
\newlength{\TwoCm}
\newlength{\ThreeCm}
\newlength{\FourCm}
\newlength{\FiveCm}
\newlength{\SixCm}
\newlength{\SevenCm}
\newlength{\TwelveCmPlusHalf}
\newlength{\SevenCmPlusHalf}
\newlength{\OneCmPlusHalf}
\newlength{\TwoCmPlusHalf}
\newlength{\ThreeCmPlusHalf}
\newlength{\FourCmPlusHalf}
\newlength{\FiveCmPlusHalf}
\newlength{\SixCmPlusHalf}
\newlength{\TwoCmPlusAQuarter}
\newlength{\TwoCmPlusThreeQuarters}
\newlength{\ThreeCmPlusAQuarter}
\newlength{\FourCmPlusAQuarter}
\newlength{\FourCmPlusOneEighth}
\newlength{\FourCmPlusThreeQuarters}
\setlength{\OneCm}{1.0cm}
\setlength{\TwoCm}{2.0cm}
\setlength{\ThreeCm}{3.0cm}
\setlength{\FourCm}{4.0cm}
\setlength{\FiveCm}{5.0cm}
\setlength{\SixCm}{6.0cm}
\setlength{\SixCmPlusHalf}{6.5cm}
\setlength{\SevenCm}{7.0cm}
\setlength{\SevenCmPlusHalf}{7.5cm}
\setlength{\TwelveCmPlusHalf}{12.5cm}
\setlength{\OneCmPlusHalf}{1.5cm}
\setlength{\TwoCmPlusHalf}{2.5cm}
\setlength{\ThreeCmPlusHalf}{3.5cm}
\setlength{\FourCmPlusHalf}{4.5cm}
\setlength{\FiveCmPlusHalf}{5.5cm}
\setlength{\TwoCmPlusAQuarter}{2.25cm}
\setlength{\ThreeCmPlusAQuarter}{3.25cm}
\setlength{\FourCmPlusAQuarter}{4.25cm}
\setlength{\FourCmPlusOneEighth}{4.125cm}
\setlength{\FourCmPlusThreeQuarters}{4.75cm}
\newlength{\TCBBoxCorrection}
\setlength{\TCBBoxCorrection}{-0.5\baselineskip}
\setlength{\TwoCmPlusThreeQuarters}{2.75cm}



% Package for hypertext links:
\usepackage{hyperref}
\hypersetup{%
    colorlinks,%
    citecolor=TitlingRed,%
    filecolor=TitlingRed,%
    linkcolor=TitlingRed,%
     urlcolor=TitlingRed%
}%
%%%% COMMANDS %%%%
\usepackage{amssymb}
\usepackage{minitoc}
\setcounter{minitocdepth}{2}
\newcommand{\Minitoc}{{\hypersetup{hidelinks}\minitoc}}
%%%% MATH DEFINITIONS %%%%
\usepackage{centernot}
\newcommand{\nin}{\centernot{\in}}
%
\let\originalleft\left
\let\originalright\right
\renewcommand{\left}{\mathopen{}\mathclose\bgroup\originalleft}
\renewcommand{\right}{\aftergroup\egroup\originalright}
%%%%%%
\newenvironment{webcompile}{\[}{\]}
\ifplastex
\else
    \usepackage{adjustbox}
    \newenvironment{scalemath}[1][\linewidth]{\[\begin{adjustbox}{width=#1,center}$}{$\end{adjustbox}\]}
\fi
\usepackage{mathtools}
\newcommand{\mcp}[1]{\mathclap{#1}}
\newcommand{\mrp}[1]{\mathrlap{#1}}
\newcommand{\mlp}[1]{\mathllap{#1}}
%\usepackage{graphicx}
\newcommand{\icoprod}{%
    \mathchoice%
    {\mathbin{\textstyle\coprod}}%
    {\mathbin{\textstyle\coprod}}%
    {\mathbin{\scriptstyle\textstyle\coprod}}%
    {\mathbin{\scriptscriptstyle\textstyle\coprod}}%
}%
%\newcommand{\ipushout}[1]{%
%    \mathchoice%
%    {\mathbin{\textstyle\coprod_{#1}}}%
%    {\mathbin{\textstyle\coprod_{#1}}}%
%    {\mathbin{\scalebox{0.75}{$\textstyle\coprod_{#1}$}}}%
%    {\mathbin{\scalebox{0.75}{$\textstyle\coprod_{#1}$}}}%
%}%
\newcommand{\xlonghookrightarrow}[1]{\overset{#1}{\hookrightarrow}}
\newcommand{\longhookrightarrow}{\hookrightarrow}
\newcommand{\xlongrightrightarrows}[2]{\underset{#2}{\overset{#1}{\rightrightarrows}}}
\newcommand{\xlongleftrightarrows}[2]{\underset{#2}{\overset{#1}{\leftrightarrows}}}
\newcommand{\xlongleftrightarrow}[1]{\overset{#1}{\leftrightarrow}}
\newcommand{\xlongtwoheadsrightarrow}[1]{\overset{#1}{\twoheadrightarrow}}
\newcommand{\rightleftrightarrows}{\overset{\rightleftarrows}{\to}}
\newcommand{\isorightarrow}{\overset{\scriptstyle\unsim}{\dashrightarrow}}
\newcommand{\rightisoarrow}{\isorightarrow}
\newcommand{\xrightisoarrow}[1]{\overset{#1}{\underset{\scriptstyle\unsim}{\dashrightarrow}}}
\newcommand{\rightequalsarrow}{\mathbin{\overset{=}{\rightarrow}}}
\newcommand{\righteqarrow}{\mathbin{\overset{\cong}{\rightarrow}}}
\newcommand{\uearrow}{\xlongrightarrow{\exists!}}
\newcommand{\Longrightisoarrow}{\mathbin{\overset{\mathord{\sim}}{\Longrightarrow}}}
\newcommand{\shortrightarrow}{\rightarrow}
%\newcommand{\icoprod}{\mathbin{\textstyle\coprod}}
\newcommand{\ipushout}[1]{\mathbin{\textstyle\coprod_{#1}}}
\DeclareMathOperator*{\iicoprod}{\textstyle\coprod}
\newcommand{\iipushout}[1]{\mathbin{\iicoprod_{#1}}}
%\usepackage{MnSymbol}
%\newcommand{\twocirc}{\mathbin{\star}}%in commands
\let\xLongrightarrow\relax
\newcommand{\xLongrightarrow}[1]{\mathbin{\overset{#1}{\Longrightarrow}}}
\newcommand{\xsubset}[1]{\mathbin{\overset{#1}{\subset}}}
\newcommand{\rightleftadjointarrows}{\mathbin{\rightleftarrows}}
\newcommand{\xrightrightarrows}[2]{\stackrel{\stackrel{#1}{\rightarrow}}{\stackrel{\rightarrow}{\scriptsize#2}}}
\newcommand{\unnatLongrightarrow}{\mathbin{\overset{\mathrm{unnat}}{\Longrightarrow}}}
\newcommand{\rightrightrightarrows}{\overset{\rightrightarrows}{\to}}
\newcommand{\rightrightrightrightarrows}{\overset{\rightrightarrows}{\rightrightarrows}}
\newcommand{\udashv}{\rotatebox[origin=c]{270}{$\dashv$}}
\newcommand{\longsqrightarrow}{\rightsquigarrow}
\newcommand{\longleftdotsrightarrows}{TODO}
\DeclareMathOperator*{\ccolim}{\mathrm{colim}^{\tiny\rotatebox[origin=c]{90}{$\circlearrowleft$}}}
\DeclareMathOperator*{\clim}{\mathrm{lim}^{\tiny\rotatebox[origin=c]{90}{$\circlearrowleft$}}}
\makeatletter
\DeclareRobustCommand{\evdots}{% vdots but with equal spacing
    \vbox{\baselineskip4\p@\lineskiplimit\z@\kern0\p@\hbox{.}\hbox{.}\hbox{.}}}
\makeatother
\renewcommand\restriction[2]{\left.#1\right\vert_{#2}}

\usepackage{amsthm}
\usepackage[capitalise,nameinlink,noabbrev]{cleveref}
\crefname{definition}{Definition}{Definitions}
\Crefname{definition}{Definition}{Definitions}
\usepackage{xspace}
% Theorem environments.
%
\theoremstyle{definition}
\newtheorem{definition}{Definition}[subsubsection]
\newtheorem{example}[definition]{Example}
%\theoremstyle{plain}
\newtheorem{theorem}[definition]{Theorem}
\newtheorem{proposition}[definition]{Proposition}
\newtheorem{question}[definition]{Question}
\newtheorem{proposition-definition}[definition]{Proposition-Definition}
\newtheorem{corollary}[definition]{Corollary}
\newtheorem{lemma}[definition]{Lemma}
%\theoremstyle{remark}
\newtheorem{remark}[definition]{Remark}
\newtheorem{warning}[definition]{Warning}
\newtheorem{notation}[definition]{Notation}
\newtheorem{construction}[definition]{Construction}
\newtheorem{oldtag}[definition]{Old Tag}

\renewcommand{\thedefinition}{\thesubsubsection.\arabic{definition}}
\renewcommand{\theexample}{\thesubsubsection.\arabic{definition}}
\renewcommand{\thetheorem}{\thesubsubsection.\arabic{definition}}
\renewcommand{\theproposition}{\thesubsubsection.\arabic{definition}}
\renewcommand{\thequestion}{\thesubsubsection.\arabic{definition}}
\renewcommand{\thelemma}{\thesubsubsection.\arabic{definition}}
\renewcommand{\theremark}{\thesubsubsection.\arabic{definition}}
\renewcommand{\theconstruction}{\thesubsubsection.\arabic{definition}}
\numberwithin{equation}{subsubsection}

\usepackage{tikz}
\usetikzlibrary{arrows.meta,calc,positioning,decorations.markings,fit,patterns,decorations.pathreplacing,3d,decorations.pathmorphing}
\pgfdeclarelayer{background}
\pgfsetlayers{background,main}
\usepackage{tikz-cd}
\newlength{\DL}
\setlength{\DL}{0.9em}% 0.9 / 1.1625
\tikzset{double line with arrow/.style args={#1,#2}{decorate,decoration={markings,mark=at position 0 with {\coordinate (ta-base-1) at (0,1pt);\coordinate (ta-base-2) at (0,-1pt);},mark=at position 1 with {\draw[#1] (ta-base-1) -- (0,1pt);\draw[#2] (ta-base-2) -- (0,-1pt);}}}}
\tikzset{Equals/.style={-,double line with arrow={-,-}}}
\tikzcdset{
    arrow style=tikz,
    %diagrams={>={Straight Barb[scale=0.75]}}
    diagrams={>={Stealth[round,length=4pt,width=4.95pt,inset=2.75pt]}}
}
\tikzcdset{%
    bigisoarrow/.style={%
        "\scalebox{1.75}{$\unsim$}"{sloped}, dash pattern=on 1.65pt off 1.65pt, outer sep=-1.5pt%
        %dash pattern=on 1.65pt off 1.65pt,%
    }%
}%
\tikzcdset{%
    bigisoarrowprime/.style={%
        "\scalebox{1.75}{$\unsim$}"'{sloped}, dash pattern=on 1.65pt off 1.65pt, outer sep=+0.1pt%
        %dash pattern=on 1.65pt off 1.65pt,%
    }%
}%
\tikzcdset{%
    isoarrow/.style={%
        "\scalebox{1.25}{$\unsim$}"{sloped,pos=0.45}, dash pattern=on 1.65pt off 1.65pt, outer sep=-2.5pt%
        %dash pattern=on 1.65pt off 1.65pt,%
    }%
}%
\tikzcdset{%
    isoarrowprime/.style={%
        "\scalebox{1.25}{$\unsim$}"'{sloped}, dash pattern=on 1.65pt off 1.65pt, outer sep=-0.75pt%
        %dash pattern=on 1.65pt off 1.65pt,%
    }%
}%
\tikzset{mid vert/.style={/utils/exec=\tikzset{every node/.append style={outer sep=0.8ex}},postaction=decorate,decoration={markings,mark=at position #1 with {\draw[-] (0,0.75ex) -- (0,-0.75ex);}}},mid vert/.default=0.5}
\ifplastex
\else
    \tikzset{lddr_to_path/.style={to path={-| ([xshift=-5ex]\tikztotarget.west) \ifx\relax#1\relax \else node[near end,left]{$\scriptstyle#1$} \fi |- (\tikztotarget)}}}
    \tikzset{rddl_to_path/.style={to path={-| ([xshift=5ex]\tikztotarget.east) \ifx\relax#1\relax \else node[near end,right]{$\scriptstyle#1$} \fi -- (\tikztotarget)}}}
    \tikzset{lddr_to_path_large/.style={to path={-| ([xshift=-10ex]\tikztotarget.west) \ifx\relax#1\relax \else node[near end,left]{$\scriptstyle#1$} \fi |- (\tikztotarget)}}}
    \tikzset{rddl_to_path_large/.style={to path={-| ([xshift=10ex]\tikztotarget.east) \ifx\relax#1\relax \else node[near end,right]{$\scriptstyle#1$} \fi -- (\tikztotarget)}}}
    \tikzset{ldddr_to_path/.style={to path={-| ([xshift=-5ex]\tikztotarget.west) \ifx\relax#1\relax \else node[near end,left]{$\scriptstyle#1$} \fi |- (\tikztotarget)}}}
    \tikzset{rdddl_to_path/.style={to path={-| ([xshift=5ex]\tikztotarget.east) \ifx\relax#1\relax \else node[near end,right]{$\scriptstyle#1$} \fi -- (\tikztotarget)}}}
    \tikzset{ldddr_to_path_large/.style={to path={-| ([xshift=-10ex]\tikztotarget.west) \ifx\relax#1\relax \else node[near end,left]{$\scriptstyle#1$} \fi |- (\tikztotarget)}}}
    \tikzset{rdddl_to_path_large/.style={to path={-| ([xshift=10ex]\tikztotarget.east) \ifx\relax#1\relax \else node[near end,right]{$\scriptstyle#1$} \fi -- (\tikztotarget)}}}
\fi

\usepackage[cal=cm]{mathalfa}
\usepackage{fontspec}
\SetTracking{encoding=*, shape=sc}{-5}
\newfontfamily\JapaneseFont[Path=ABSOLUTEPATH/fonts/japanese/,Scale=1.0]{NotoSansJP-Regular.ttf}
\newcommand{\Japanese}[1]{{\text{\JapaneseFont{#1}}}}
\newfontfamily\SimplifiedChineseFont[Path=ABSOLUTEPATH/fonts/chinese/,Scale=1.0]{NotoSansSC-Regular.ttf}
\newcommand{\SimplifiedChinese}[1]{{\text{\SimplifiedChineseFont{#1}}}}
\newfontfamily\TraditionalChineseFont[Path=ABSOLUTEPATH/fonts/chinese/,Scale=1.0]{NotoSansTC-Regular.ttf}
\newcommand{\TraditionalChinese}[1]{{\text{\SimplifiedChineseFont{#1}}}}
\newcommand{\FontForCategories}[1]{\mathsf{#1}}
\newcommand{\FontForEnrichedCategories}[1]{\textup{\textsf{\textbf{#1}}}}
\DeclareMathAlphabet{\dutchcal}{U}{dutchcal}{m}{n}
\SetMathAlphabet{\dutchcal}{bold}{U}{dutchcal}{b}{n}
\DeclareMathAlphabet{\CatFont}{U}{tx-cal}{m}{n}
\SetMathAlphabet{\CatFont}{bold}{U}{tx-cal}{b}{n}
\newcommand{\SheafFont}[1]{\dutchcal{#1}}
\newcommand{\mathsfup}[1]{\textsf{#1}}
\newcommand{\demph}[1]{\textbf{\emph{#1}}}
\usepackage{manfnt}
\newcommand{\textdbendEnd}{}
\newcommand{\negphantom}[1]{\settowidth{\dimen0}{#1}\hspace*{-\dimen0}}
\DeclareSymbolFont{bbold}{U}{bbold}{m}{n}
\DeclareSymbolFontAlphabet{\mathbbold}{bbold}

\ifplastex
\else
    
\fi
\usepackage{centernot}
\newcommand{\abs}[1]{\left\lvert#1\right\rvert}
\let\lim\relax
\DeclareMathOperator*{\colim}{\text{colim}}
\DeclareMathOperator*{\lim}{\text{lim}}
\let\min\relax
\let\max\relax
\DeclareMathOperator*{\min}{\text{min}}
\DeclareMathOperator*{\max}{\text{max}}
\let\widebar\relax
\newcommand{\widebar}[1]{\overline{#1}}
\newcommand{\widebarS}[1]{\smash{\overline{#1}}}
\newcommand{\wcolim}{\text{colim}}
\newcommand{\wlim}{\text{lim}}
\newcommand{\Lan}{\text{Lan}}
\newcommand{\Ran}{\text{Ran}}
\newcommand{\Wit}{\text{Wit}}
\newcommand{\pos}{\mathsf{pos}}
\newcommand{\dir}{\mathrm{dir}}
\newcommand{\init}{\mathrm{init}}
\newcommand{\fnl}{\mathrm{fnl}}
\newcommand{\sgn}{\mathrm{sgn}}
\newcommand{\Lift}{\text{Lift}}
\newcommand{\Rift}{\text{Rift}}
\newcommand{\St}{\text{St}}
\newcommand{\Un}{\text{Un}}
\newcommand{\U}{\mathrm{U}}
\newcommand{\Gr}{\text{Gr}}
\newcommand{\Sd}{\mathrm{Sd}}
\newcommand{\GL}{\mathrm{GL}}
\newcommand{\Tr}{\mathrm{Tr}}
\newcommand{\tr}{\mathrm{tr}}
\newcommand{\Cl}{\mathrm{Cl}}
\newcommand{\cl}{\mathrm{cl}}
\newcommand{\rmZ}{\mathrm{Z}}
\newcommand{\rmcl}{\mathrm{cl}}
\newcommand{\rmCl}{\mathrm{Cl}}
\newcommand{\rmC}{\mathrm{C}}
\newcommand{\rmint}{\mathrm{int}}
\newcommand{\rmInt}{\mathrm{Int}}
\newcommand{\rmExt}{\mathrm{Ext}}
\newcommand{\rmQ}{\mathrm{Q}}
\newcommand{\rmAlg}{\mathrm{Alg}}
\newcommand{\Alg}{\FontForCategories{Alg}}
\newcommand{\rmT}{\mathrm{T}}
\newcommand{\Ex}{\mathrm{Ex}}
\newcommand{\sd}{\mathrm{sd}}
\newcommand{\rmchar}{\mathrm{char}}
\let\div\relax
\newcommand{\div}{\mid}
\newcommand{\ndiv}{\centernot\div}
\newcommand{\Cats}{\FontForCategories{Cats}}
\newcommand{\sfOpen}{\FontForCategories{Open}}
\newcommand{\sCats}{\FontForCategories{sCats}}
\newcommand{\Prof}{\FontForCategories{Prof}}
\newcommand{\CoPSh}{\FontForCategories{CoPSh}}
\newcommand{\Shv}{\FontForCategories{Shv}}
\newcommand{\Ho}{\FontForCategories{Ho}}
\newcommand{\Path}{\FontForCategories{Path}}
\newcommand{\ContFun}{\FontForCategories{Fun}^{\mathcal{C}}}
\newcommand{\ContPos}{\FontForCategories{PosFun}^{\mathcal{C}}}
\newcommand{\Tors}{\mathrm{Tors}}
\newcommand{\sfEnd}{\FontForCategories{End}}
\newcommand{\sfAut}{\FontForCategories{Aut}}
\newcommand{\rmEnd}{\mathrm{End}}
\newcommand{\Mat}{\mathrm{Mat}}
\DeclareMathOperator{\invL}{\chi^{\rmL}}%
\DeclareMathOperator{\invR}{\chi^{\rmR}}%
\newcommand{\invLs}[1]{\chi^{\rmL}_{#1}}%
\newcommand{\sfInv}{\FontForCategories{Inv}}
\newcommand{\vmod}[1]{\mathrm{mod}\ #1}
\newcommand{\vvec}[1]{\smash{\vec{#1}}}
\newcommand{\bbB}{\mathbb{B}}
\newcommand{\sfIdem}{\FontForCategories{Idem}}
\newcommand{\invRs}[1]{\chi^{\rmR}_{#1}}%
\newcommand{\rmAut}{\mathrm{Aut}}
\newcommand{\End}{\mathrm{End}}
\newcommand{\Aut}{\mathrm{Aut}}
\newcommand{\neqcong}{\centernot{\eqcong}}
\DeclareMathOperator{\functorBun}{\mathrm{Bun}}
\newcommand{\RepresentableProfunctor}[1]{\smash{\widehat{F}^*}}
\newcommand{\num}{\mathrm{num}}
\newcommand{\CorepresentableProfunctor}[1]{\smash{\widehat{F}_*}}
\newcommand{\EG}{\mathrm{EG}}
\newcommand{\BG}{\mathrm{BG}}
\newcommand{\Mod}{\FontForCategories{Mod}}
\newcommand{\Vect}{\FontForCategories{Vect}}
\newcommand{\LMod}{\FontForCategories{LMod}}
\newcommand{\RMod}{\FontForCategories{RMod}}
\newcommand{\BiMod}{\FontForCategories{BiMod}}
\newcommand{\PseudoFun}{\FontForCategories{PseudoFun}}
\newcommand{\LaxFun}{\FontForCategories{LaxFun}}
\newcommand{\TwoCategoryOfCategories}{\FontForCategories{Cats}_{\FontForCategories{2}}}
\newcommand{\TwoCategoryOfBimodules}{\FontForCategories{BiMod}_{\FontForCategories{2}}}
\newcommand{\TwoCategoryOfGroupoids}{\FontForCategories{Grpd}_{\FontForCategories{2}}}
\newcommand{\ISets}{\FontForCategories{ISets}}
\newcommand{\FibSets}{\FontForCategories{FibSets}}
\newcommand{\eISets}{\FontForEnrichedCategories{ISets}}
\newcommand{\eFibSets}{\FontForEnrichedCategories{FibSets}}
\newcommand{\Mon}{\FontForCategories{Mon}}
\newcommand{\RInvMon}{\FontForCategories{RInvMon}}
\newcommand{\MonAct}{\FontForCategories{MonAct}}
\newcommand{\otimesDay}{\circledast}
\newcommand{\Day}{\mathrm{Day}}
\newcommand{\FreeAlg}{\FontForCategories{FreeAlg}}
\newcommand{\PunctualCategory}{\FontForCategories{pt}}
\newcommand{\PunctualBicategory}{\FontForCategories{pt}_{\FontForCategories{bi}}}
\newcommand{\sftwo}{\FontForCategories{2}}
\newcommand{\sfps}{\FontForCategories{ps}}
\newcommand{\CoMon}{\FontForCategories{CoMon}}
\newcommand{\B}{\FontForCategories{B}}
\newcommand{\Bil}{\mathrm{Bil}}
\newcommand{\tleft}{\mathbin{\triangleleft}}
\newcommand{\tright}{\mathbin{\triangleright}}
\newcommand{\EmptyCategory}{\emptyset_{\FontForCategories{cat}}}
\newcommand{\Sk}{\FontForCategories{Sk}}
\newcommand{\inj}{\mathrm{inj}}
\newcommand{\sfid}{\FontForCategories{id}}
\newcommand{\eSets}{\FontForEnrichedCategories{Sets}}
\newcommand{\Sing}{\mathrm{Sing}}
\newcommand{\SingB}{\mathrm{Sing}_{\bullet}}
\newcommand{\point}{\star}
\newcommand{\CoEq}{\text{CoEq}}
\newcommand{\DFib}{\FontForCategories{DFib}}
\newcommand{\Eq}{\text{Eq}}
\newcommand{\E}{\mathbb{E}}
\newcommand{\Fun}{\FontForCategories{Fun}}
\newcommand{\F}{\mathbb{F}}
\newcommand{\Grp}{\FontForCategories{Grp}}
\newcommand{\Hom}{\textup{Hom}}
\newcommand{\Iso}{\textup{Iso}}
\newcommand{\iso}{\textup{iso}}
\newcommand{\src}{\textup{src}}
\newcommand{\tgt}{\textup{tgt}}
\newcommand{\Mor}{\textup{Mor}}
\newcommand{\Card}[1]{\#{#1}}%\newcommand{\Card}[1]{{\##1}}
\newcommand{\Fix}{\mathrm{Fix}}
\newcommand{\GlobeCategory}{\mathbb{G}}
\newcommand{\CubeCategory}{\square}
\newcommand{\GammaCategory}{\Gamma}
\newcommand{\CycleCategory}{\Lambda}
\newcommand{\ParacycleCategory}{\Lambda_{\infty}}
\newcommand{\TreeCategory}{\Omega}
\newcommand{\ThetaCategory}{\Theta}
\newcommand{\Orb}{\FontForCategories{Orb}}
\newcommand{\LUnitor}{\lambda}
\newcommand{\MinusOneCats}{\FontForCategories{Cats}_{-1}}
\newcommand{\N}{\mathbb{N}}
\newcommand{\Trans}{\text{Trans}}
\newcommand{\Nat}{\text{Nat}}
\newcommand{\CoTrans}{\text{CoTrans}}
\newcommand{\CoNat}{\text{CoNat}}
\newcommand{\Obj}{\text{Obj}}
\newcommand{\PSh}{\FontForCategories{PSh}}
\newcommand{\Pos}{\FontForCategories{Pos}}
\newcommand{\ePos}{\FontForEnrichedCategories{Pos}}
\newcommand{\RUnitor}{\rho}
\newcommand{\Rel}{\mathrm{Rel}}
\newcommand{\eRel}{\mathbf{Rel}}
\newcommand{\sfRel}{\FontForCategories{Rel}}
\newcommand{\MRel}{\FontForCategories{MRel}}
\newcommand{\Adj}{\FontForCategories{Adj}}
\newcommand{\Span}{\FontForCategories{Span}}
\newcommand{\rmSpan}{\mathrm{Span}}
\newcommand{\rmD}{\mathrm{D}}
\newcommand{\MapSpan}{\FontForCategories{MapSpan}}
\newcommand{\sfbfRel}{\FontForEnrichedCategories{Rel}}
\newcommand{\R}{\mathbb{R}}
\newcommand{\FinSets}{\FontForCategories{FinSets}}
\newcommand{\FinAb}{\FontForCategories{FinAb}}
\newcommand{\Sets}{\FontForCategories{Sets}}
\newcommand{\SloganFont}[1]{\textit{#1}}
\newcommand{\TV}{\{\true,\false\}}
\newcommand{\TTV}{\{\ttrue,\tfalse\}}
\newcommand{\Q}{\mathbb{Q}}
\newcommand{\C}{\mathbb{C}}
\newcommand{\Z}{\mathbb{Z}}
\newcommand{\T}{\mathbb{T}}
\newcommand{\ncomm}{\textcolor{red}{\scalebox{2.0}{$\times$}}}
\newcommand{\Ker}{\mathrm{Ker}}
\newcommand{\CoIm}{\mathrm{CoIm}}
\newcommand{\coim}{\mathrm{coim}}
\newcommand{\refl}{\FontForCategories{refl}}
\newcommand{\symm}{\FontForCategories{symm}}
\newcommand{\trans}{\FontForCategories{trans}}
\newcommand{\rmrefl}{\mathrm{refl}}
\newcommand{\rmsymm}{\mathrm{symm}}
\newcommand{\rmtrans}{\mathrm{trans}}
\newcommand{\rmeq}{\mathrm{eq}}
\newcommand{\cocont}{\mathrm{cocont}}
\newcommand{\triv}{\mathrm{triv}}
\newcommand{\cotriv}{\mathrm{cotriv}}
\newcommand{\Wasureru}{\Japanese{忘}}
\newcommand{\ZeroCats}{\FontForCategories{Cats}_{0}}
\newcommand{\Zn}[1]{\mathbb{Z}_{/#1}}
\newcommand{\e}{e}
\newcommand{\rmA}{\mathrm{A}}
\newcommand{\Rep}{\FontForCategories{Rep}}
\newcommand{\rmRep}{\mathrm{Rep}}
\newcommand{\std}{\mathrm{std}}
\newcommand{\rmCld}{\mathrm{Cld}}
\newcommand{\rmOpen}{\mathrm{Open}}
\newcommand{\rmCpt}{\mathrm{Cpt}}
\newcommand{\Cld}{\FontForCategories{Cld}}
\newcommand{\Open}{\FontForCategories{Open}}
\newcommand{\Cpt}{\FontForCategories{Cpt}}
\newcommand{\coeq}{\text{coeq}}
\newcommand{\defeq}{\mathrel{\smash{\overset{\mathclap{\scriptscriptstyle\text{def}}}=}}}
\newcommand{\eqquestion}{\mathrel{\smash{\overset{\mathclap{\scriptscriptstyle\text{?}}}=}}}
\newcommand{\eqstar}{\mathrel{\smash{\overset{\mathclap{\scriptscriptstyle(\dagger)}}{=}}}}
%\newcommand{\eqstar}{\mathrel{\smash{\overset{\mathclap{\scriptscriptstyle(\dagger)}}{=}}}}
\newcommand{\eqclm}{\mathrel{\smash{\overset{\mathclap{\scriptscriptstyle\text{clm}}}{=}}}}
\newcommand{\indisc}{\FontForCategories{indisc}}
\newcommand{\K}{\mathrm{K}}
\newcommand{\iGrpd}{\Grpd_\infty}
\newcommand{\disc}{\FontForCategories{disc}}
\newcommand{\twodisc}{\FontForCategories{2disc}}
\newcommand{\Grpd}{\FontForCategories{Grpd}}
\newcommand{\Core}{\FontForCategories{Core}}
\newcommand{\sSets}{\FontForCategories{sSets}}
\let\Top\relax
\newcommand{\Top}{\FontForCategories{Top}}
\newcommand{\LocConnTop}{\FontForCategories{LocConnTop}}
\newcommand{\TAb}{\FontForCategories{TAb}}
\newcommand{\TVect}{\FontForCategories{TVect}}
\newcommand{\TVS}{\FontForCategories{TVS}}
\newcommand{\CHaus}{\FontForCategories{CHaus}}
\newcommand{\CMon}{\FontForCategories{CMon}}
\newcommand{\Ab}{\FontForCategories{Ab}}
\newcommand{\ab}{\mathrm{ab}}
\newcommand{\tf}{\mathrm{tf}}
\newcommand{\tfAb}{\FontForCategories{Ab}^{\FontForCategories{tf}}}
\DeclareMathOperator{\Nerve}{\mathrm{N}}
\newcommand{\NerveB}{\mathrm{N}_{\bullet}}
\DeclareMathOperator{\sNerve}{\mathrm{N}^{\mathrm{hc}}}
\newcommand{\sNerveB}{\mathrm{N}^{\mathrm{hc}}_{\bullet}}
\newcommand{\cat}{\FontForCategories{cat}}
\newcommand{\eHom}{\mathbf{Hom}}
\newcommand{\eqcong}{\mathrel{\smash{\overset{\scriptscriptstyle\text{eq.}}\cong}}}
\newcommand{\comma}{\mathbin{\overset{\to}{\times}}}% comma category
\newcommand{\isocomma}{\mathbin{\overset{\leftrightarrow}{\times}}}% comma category
\newcommand{\cocomma}{\mathbin{\overset{\to}{\icoprod}}}% comma category
\newcommand{\isococomma}{\mathbin{\overset{\leftrightarrow}{\icoprod}}}% comma category
\newcommand{\eq}{\text{eq}}
\newcommand{\false}{\FontForCategories{false}}
\newcommand{\tfalse}{\FontForCategories{f}}
\newcommand{\id}{\text{id}}
\newcommand{\opsup}{^{\op}}
\newcommand{\op}{\FontForCategories{op}}
\newcommand{\co}{\FontForCategories{co}}
\newcommand{\coop}{\FontForCategories{coop}}
\newcommand{\pr}{\text{pr}}
\newcommand{\pt}{\text{pt}}
\newcommand{\rmB}{\mathrm{B}}
\newcommand{\rmR}{\mathrm{R}}
\newcommand{\rmL}{\mathrm{L}}
\newcommand{\rmcp}{\text{cp}}
\newcommand{\rmii}{ii}
\newcommand{\rmim}{\text{im}}
\newcommand{\rmi}{i}
\newcommand{\sdiff}{\mathbin{\triangle}}
\newcommand{\sfc}{\mathsfup{c}}
\newcommand{\Arr}[1]{\FontForCategories{Arr}(#1)}
\newcommand{\true}{\FontForCategories{true}}
\newcommand{\ttrue}{\FontForCategories{t}}
\newcommand{\twoheadsrightarrow}{\twoheadrightarrow}
\newcommand{\unsim}{\mathord{\sim}}
\newcommand{\yo}{\Japanese{よ}}
\newcommand{\dom}{\mathrm{dom}}
\let\d\relax
\newcommand{\d}{\mathrm{d}}
\newcommand{\laxcolim}{\mathsfup{colim}^{\mathsfup{lax}}}
\newcommand{\range}{\mathrm{range}}
\newcommand{\rank}{\mathrm{rank}}
\newcommand{\bfX}{\mathbf{X}}
\newcommand{\bfY}{\mathbf{Y}}
\newcommand{\bfZ}{\mathbf{Z}}
\newcommand{\catEl}[2]{\textstyle\int_{#1}#2}
\newcommand{\vcatEl}[2]{\textstyle\int^{#1}#2}
\newcommand{\procirc}{\mathbin{\diamond}}
\newcommand{\dblRel}{\FontForCategories{Rel}^{\dbl}}
\newcommand{\dbl}{\FontForCategories{dbl}}
\newcommand{\doublecirc}{\mathbin{\odot}}
\DeclareMathOperator*{\ttimes}{\mathbin{\times}}
\newcommand{\eColl}{\mathbf{Coll}}
\newcommand{\Coll}{\FontForCategories{Coll}}
\newcommand{\Ring}{\FontForCategories{Ring}}
\newcommand{\CRing}{\FontForCategories{CRing}}
\newcommand{\StarAlg}{\FontForCategories{Alg}^{*}}
\newcommand{\StarCAlg}{\FontForCategories{CAlg}^{*}}
\newcommand{\StarRing}{\FontForCategories{Ring}^{*}}
\newcommand{\StarCRing}{\FontForCategories{CRing}^{*}}
\newcommand{\StarMod}{\FontForCategories{Mod}^{*}}
\newcommand{\InvMod}{\FontForCategories{InvMod}}
\newcommand{\ceil}[1]{\lceil#1\rceil}
\newcommand{\floor}[1]{\lfloor#1\rfloor}
\newcommand{\ph}{\mathsfup{h}}
\newcommand{\FreePsAlg}{\FontForCategories{FreePsAlg}}
\newcommand{\rmColl}{\mathrm{Coll}}
\newcommand{\fib}{\mathrm{fib}}
\newcommand{\nneq}{\mathrel{\smash{\overset{\scriptscriptstyle\mathrm{poss.}}\neq}}}% not necessarily equal to/possibly not equal to
\let\Im\relax
\newcommand{\Im}{\mathrm{Im}}
\newcommand{\WalkingSpan}{\Lambda}
\newcommand{\ev}{\mathrm{ev}}
\newcommand{\coev}{\mathrm{coev}}
\newcommand{\cHom}{\FontForCategories{Hom}}
\newcommand{\cIso}{\FontForCategories{Iso}}
\newcommand{\dblSpan}{\FontForCategories{Span}^{\dbl}}
\newcommand{\bidisc}{\FontForCategories{bidisc}}
% TODO
%\let\nsubset\relax
%\newcommand{\nsubset}{\not{\subset}}
\newcommand{\twocirc}{\star}
\newcommand{\IsbellMonad}{\mathsf{Isb}}
\newcommand{\varIsbellMonad}{\mathsf{CoIsb}}
\newcommand{\IsbellSpec}{\mathsf{Spec}}
\newcommand{\IsbellO}{\mathsf{O}}
\newcommand{\IsbellI}{\mathsf{I}}
\newcommand{\bigequalssign}{\scalebox{2.0}{$=$}}
\newcommand{\mate}[1]{#1^{\dagger}}
\newcommand{\vs}{vs.}
\newcommand{\rightproarrow}{\mathrel{\rightarrow\kern-9.5pt\mrp{|}\kern6pt}}
\newcommand{\rightproarrows}{\mathrel{\rightrightarrows\kern-9.5pt\mrp{|}\kern6pt}}
\newcommand{\xrightproarrow}[1]{\mathbin{\overset{#1}{\rightproarrow}}}
%
\newcommand*\cocolon{% \nobreak \mskip6mu plus1mu \mathpunct{}%
    \nobreak
    \mspace{6mu plus 1mu}
    {:}
    \nonscript\mkern-\thinmuskip
    \mathpunct{}
    \mspace{2mu}
}
\newcommand{\Bifunctoriality}[7]{%
    \begin{array}{ccc}
        #1\colon\mkern-15mu &#4 \mkern-17.5mu&{}\mathbin{\to}#7,\\
        #2\colon\mkern-15mu &#5 \mkern-17.5mu&{}\mathbin{\to}#7,\\
        #3\colon\mkern-15mu &#6 \mkern-17.5mu&{}\mathbin{\to}#7,
    \end{array}
}%
\newcommand{\BifunctorialityPeriod}[7]{%
    \begin{array}{ccc}
        #1\colon\mkern-15mu &#4 \mkern-17.5mu&{}\mathbin{\to}#7,\\
        #2\colon\mkern-15mu &#5 \mkern-17.5mu&{}\mathbin{\to}#7,\\
        #3\colon\mkern-15mu &#6 \mkern-17.5mu&{}\mathbin{\to}#7.
    \end{array}
}%
\newcommand{\wwidehat}[1]{\smash{\widehat{#1}}}
\DeclareMathOperator*{\invlim}{{\displaystyle\underset{\longleftarrow}{\lim}}}
\DeclareMathOperator*{\dircolim}{{\displaystyle\underset{\longrightarrow}{\colim}}}
\newcommand{\HilbertCube}{Q}
\newcommand{\bfx}{\mathbf{x}}
\newcommand{\bfy}{\mathbf{y}}
\newcommand{\cld}{\mathrm{cld}}
\newcommand{\open}{\mathrm{open}}
\newcommand{\notdownarrow}{\centernot{\downarrow}}
\newcommand{\Vie}{\mathrm{Vie}}
\newcommand{\undownarrow}{\mathord{\downarrow}}
\newcommand{\rmI}{I}
\newcommand{\mrmI}{\mathrm{I}}
\newcommand{\rmII}{II}
\newcommand{\rmIII}{III}
\newcommand{\rmIV}{IV}
\newcommand{\rmV}{V}
\newcommand{\rmVI}{VI}
\newcommand{\rmVII}{VII}
\newcommand{\rmVIII}{VIII}
\newcommand{\rmIX}{IX}
\newcommand{\rmX}{X}
\newcommand{\rmXI}{XI}
\newcommand{\rmXII}{XII}
\newcommand{\rmXIII}{XIII}
\newcommand{\rmXIV}{XIV}
\newcommand{\rmXV}{XV}
\newcommand{\rmXVI}{XVI}
\newcommand{\fin}{\mathrm{fin}}
\newcommand{\Tzero}{T₀}
\newcommand{\Tone}{T₁}
\newcommand{\Ttwo}{T₂}
\newcommand{\Tthree}{T₃}
\newcommand{\Tfour}{T₄}
\newcommand{\Tfive}{T₅}
\newcommand{\Tsix}{T₆}
\newcommand{\Ell}{L}
\newcommand{\PreOrd}{\FontForCategories{PreOrd}}
\newcommand{\PreOrdMon}{\FontForCategories{PreOrdMon}}
\newcommand{\PreOrdGrp}{\FontForCategories{PreOrdGrp}}
\newcommand{\rmPreOrd}{\mathrm{PreOrd}}
\newcommand{\ePreOrd}{\FontForEnrichedCategories{PreOrd}}
\newcommand{\rmpreord}{\mathrm{preord}}
\newcommand{\PartOrd}{\FontForCategories{PartOrd}}
\newcommand{\rmPartOrd}{\mathrm{PartOrd}}
\newcommand{\ePartOrd}{\FontForEnrichedCategories{PartOrd}}
\newcommand{\rmptord}{\mathrm{ptord}}
\newcommand{\unpreceq}{\mathord{\preceq}}
\newcommand{\unleq}{\mathord{\leq}}
\newcommand{\antisymm}{\FontForCategories{antisymm}}
\newcommand{\as}{\FontForCategories{as}}
\newcommand{\rmantisymm}{\mathrm{antisymm}}
\newcommand{\rmas}{\mathrm{as}}
\newcommand{\ReflRel}{\FontForCategories{ReflRel}}
\newcommand{\eReflRel}{\FontForEnrichedCategories{ReflRel}}
\newcommand{\SymmRel}{\FontForCategories{SymmRel}}
\newcommand{\eSymmRel}{\FontForEnrichedCategories{SymmRel}}
\newcommand{\TransRel}{\FontForCategories{TransRel}}
\newcommand{\eTransRel}{\FontForEnrichedCategories{TransRel}}
\newcommand{\EqRel}{\FontForCategories{EqRel}}
\newcommand{\eEqRel}{\FontForEnrichedCategories{EqRel}}
\newcommand{\AntiSymmRel}{\FontForCategories{AntiSymmRel}}
\newcommand{\eAntiSymmRel}{\FontForEnrichedCategories{AntiSymmRel}}
\newcommand{\DirGrph}{\FontForCategories{DirGrph}}
\newcommand{\Quiv}{\FontForCategories{Quiv}}
\newcommand{\ModuliCategory}{\CatFont{M}}
%
\newcommand{\Comp}{\FontForCategories{Comp}}
\newcommand{\CoComp}{\FontForCategories{CoComp}}
\newcommand{\CoCompPos}{\CoComp(\Pos)}
\newcommand{\CoCompCats}{\CoComp(\Cats)}
\newcommand{\SupLat}{\FontForCategories{SupLat}}
\newcommand{\InfLat}{\FontForCategories{InfLat}}
\newcommand{\CompCats}{\Comp(\Cats)}
\newcommand{\Gal}{\mathrm{Gal}}
\newcommand{\TopElement}{\infty}
\newcommand{\BottomElement}{\varnothing}
\newcommand{\Fld}{\FontForCategories{Fld}}
\newcommand{\OrdFld}{\FontForCategories{OrdFld}}
\newcommand{\TOrdFld}{\FontForCategories{TOrdFld}}
\newcommand{\PartOrdGrp}{\FontForCategories{PartOrdGrp}}
\newcommand{\actv}{\mathrm{actv}}
\newcommand{\MonCats}{\FontForCategories{MonCats}}
\newcommand{\AL}{\mathrm{AL}}
\newcommand{\QHom}{\mathrm{QHom}}
\newcommand{\lex}{\mathrm{lex}}
\newcommand{\WalkingArrow}{\OrdinalCategoryN{1}}

\usepackage{xurl}%
\usepackage[maxbibnames=20,backend=biber,backref=true,style=alphabetic,citestyle=alphabetic]{biblatex}%
\addbibresource{ABSOLUTEPATH/bibliography.bib}

\usepackage[splitindex]{imakeidx}
\usepackage{lettrine}
%
\makeindex[intoc=false, options= -s index_style.ist, name=notation, columns=2, title={Index of Notation}]
%
\makeindex[intoc=false, options= -s index_style.ist, name=representation-theory, columns=2, title={Index of Representation Theory}]
%\makeindex[intoc=false, options= -s index_style.ist, name=algebra, columns=2, title={Index of Algebra}]
%\makeindex[intoc=false, options= -s index_style.ist, name=algebraic-geometry, columns=2, title={Index of Algebraic Geometry}]
%\makeindex[intoc=false, options= -s index_style.ist, name=analysis, columns=2, title={Index of Analysis}]
\makeindex[intoc=false, options= -s index_style.ist, name=categories, columns=2, title={Index of Category Theory}]
%\makeindex[intoc=false, options= -s index_style.ist, name=cubical-stuff, columns=2, title={Index of Cubical Stuff}]
%\makeindex[intoc=false, options= -s index_style.ist, name=cellular-stuff, columns=2, title={Index of Cellular Stuff}]
%\makeindex[intoc=false, options= -s index_style.ist, name=globular-stuff, columns=2, title={Index of Globular Stuff}]
%\makeindex[intoc=false, options= -s index_style.ist, name=cyclic-stuff, columns=2, title={Index of Cyclic Stuff}]
%\makeindex[intoc=false, options= -s index_style.ist, name=differential-geometry, columns=2, title={Index of Differential Geometry}]
%\makeindex[intoc=false, options= -s index_style.ist, name=functional-analysis, columns=2, title={Index of Functional Analysis}]
\makeindex[intoc=false, options= -s index_style.ist, name=higher-categories, columns=2, title={Index of Higher Category Theory}]
%\makeindex[intoc=false, options= -s index_style.ist, name=homological-algebra, columns=2, title={Index of Homological Algebra}]
%\makeindex[intoc=false, options= -s index_style.ist, name=homotopical-algebra, columns=2, title={Index of Homotopical Algebra}]
%\makeindex[intoc=false, options= -s index_style.ist, name=homotopy-theory, columns=2, title={Index of Homotopy Theory}]
%\makeindex[intoc=false, options= -s index_style.ist, name=infty-categories, columns=2, title={Index of \texorpdfstring{$\infty$}{∞}-Categories}]
%\makeindex[intoc=false, options= -s index_style.ist, name=measure-theory, columns=2, title={Index of Measure Theory}]
%\makeindex[intoc=false, options= -s index_style.ist, name=monoids, columns=2, title={Index of Monoids}]
%\makeindex[intoc=false, options= -s index_style.ist, name=number-theory, columns=2, title={Index of Number Theory}]
%\makeindex[intoc=false, options= -s index_style.ist, name=probability-theory, columns=2, title={Index of Probability Theory}]
%\makeindex[intoc=false, options= -s index_style.ist, name=p-adic-geometry, columns=2, title={Index of $p$-Adic Geometry}]
%\makeindex[intoc=false, options= -s index_style.ist, name=physics, columns=2, title={Index of Physics}]
\makeindex[intoc=false, options= -s index_style.ist, name=set-theory, columns=2, title={Index of Set Theory}]
%\makeindex[intoc=false, options= -s index_style.ist, name=simplicial-stuff, columns=2, title={Index of Simplicial Stuff}]
%\makeindex[intoc=false, options= -s index_style.ist, name=stochastic-analysis, columns=2, title={Index of Stochastic Analysis}]
%\makeindex[intoc=false, options= -s index_style.ist, name=supersymmetry, columns=2, title={Index of Supersymmetry}]
%\makeindex[intoc=false, options= -s index_style.ist, name=topology, columns=2, title={Index of Topology}]
%\makeindex[intoc=false, options= -s index_style.ist, name=type-theory, columns=2, title={Index of Type Theory}]

\setlength{\parindent}{0pt}
\ifplastex
\else
    \usepackage{footnotehyper}
\let\oldFootnote\footnote
\newcommand\nextToken\relax
\renewcommand\footnote[1]{%
    \oldFootnote{#1}\futurelet\nextToken\isFootnote%
}%
\newcommand\isFootnote{%
    \ifx\footnote\nextToken\textsuperscript{,}\fi%
}%

    \usepackage{fancyhdr}
\pagestyle{fancy}% Change page style to fancy
\fancyhf{}% Clear header/footer
\fancyhead[L]{\nouppercase{\rightmark}\hfill\thepage}%
\fancyfoot[C]{}%
\setlength{\headheight}{15.0pt}

    \usepackage{etoolbox}
\makeatletter
\displaywidth=\textwidth
\displayindent=-\leftskip
\patchcmd\start@gather{$$}{%
  $$%
  \displaywidth=\textwidth
  \displayindent=-\leftskip
}{}{\errmessage{Cannot patch \string\start@gather}}
\patchcmd\start@align{$$}{%
  $$%
  \displaywidth=\textwidth
  \displayindent=-\leftskip
}{}{\errmessage{Cannot patch \string\start@align}}
\patchcmd\start@multline{$$}{%
  $$%
  \displaywidth=\textwidth
  \displayindent=-\leftskip
}{}{\errmessage{Cannot patch \string\start@multline}}
\patchcmd\mathdisplay{$$}{%
  $$%
  \displaywidth=\textwidth
  \displayindent=-\leftskip
}{}{\errmessage{Cannot patch \string\mathdisplay}}
\makeatother
\makeatletter
\newcommand{\displaybump}{\hbox to \@totalleftmargin{\hfil}}
\makeatother
\usepackage{pdfpages}
\usepackage[page]{appendix}
\AtBeginEnvironment{subappendices}{%%
    \section*{\huge Appendices}\vspace{-0.75\baselineskip}%
}%
\usepackage{makecell}
\usepackage{cellspace}
\let\mathrm\relax
\let\mathbf\relax
\newcommand{\mathrm}[1]{\text{#1}}
\newcommand{\mathbf}[1]{\textbf{#1}}
\newcommand{\GrC}[2]{
    \mathchoice%
    {\scalebox{1.0}{$\textstyle\int^{#1}#2$}}%
    {\scalebox{1.0}{$\textstyle\int^{#1}#2$}}%
    {\scalebox{0.75}{$\textstyle\int^{#1}#2$}}%
    {\scalebox{0.6}{$\textstyle\int^{#1}#2$}}%
}
\newcommand{\vGrC}[2]{
    \mathchoice%
    {\scalebox{1.0}{$\textstyle\int_{#1}#2$}}%
    {\scalebox{1.0}{$\textstyle\int_{#1}#2$}}%
    {\scalebox{0.75}{$\textstyle\int_{#1}#2$}}%
    {\scalebox{0.6}{$\textstyle\int_{#1}#2$}}%
}
\usepackage{stackengine}
\usepackage{scalerel}
\newcommand\pig[1]{\scalerel*[5pt]{\big#1}{\ensurestackMath{\addstackgap[1.5pt]{\big#1}}}}
\newcommand\pigl[1]{\mathopen{\pig{#1}}}
\newcommand\pigr[1]{\mathclose{\pig{#1}}}
\usepackage{epigraph,xpatch}

\epigraphnoindent

\makeatletter
\newcommand{\doubleepigraph}[4]{%
  \vspace{\beforeepigraphskip}
  \vbox{%
    \xpatchcmd{\@epitext}{{minipage}}{{minipage}[t]}{}{}%
    \epigraphsize
    \begin{\epigraphflush}
    \begin{minipage}[b]{\epigraphwidth}
      \@epitext{#1}\\
      \@episource{\begin{flushleft}#2\strut\end{flushleft}}%
    \end{minipage}\hfill
    \begin{minipage}[b]{\epigraphwidth}
      \@epitext{#3}\\
      \@episource{#4\strut}%
    \end{minipage}%
    \end{\epigraphflush}%
  }%
  %\nointerlineskip
  %\vspace*{\afterepigraphskip}%
  %\ifepigraphnoindent\@afterheading\fi
}
\makeatother

\setlength{\epigraphwidth}{0.45\textwidth}
\tikzcdset{%
    prompter/.code={%
        \tikzset{%
            nodes in empty cells=false,cells={nodes={draw,execute at end node={\makebox[0pt][l]{$~{}_{%
                \smash{%
                    \raisebox{-0.75em}{
                        \colorbox{OIblue}{%
                            \textcolor{white}{%
                                \textbf{\textsf{\the\pgfmatrixcurrentrow-\the\pgfmatrixcurrentcolumn}}%
                            }%
                        }%
                    }%
                }%
            }$}}}}%
        }%
    }%
}%
\tikzcdset{%
    prompter all/.code={%
        \tikzset{%
            nodes in empty cells=true,cells={nodes={draw,execute at end node={\makebox[0pt][l]{$~{}_{%
                \smash{%
                    \raisebox{-0.75em}{
                        \colorbox{OIblue}{%
                            \textcolor{white}{%
                                \textbf{\textsf{\the\pgfmatrixcurrentrow-\the\pgfmatrixcurrentcolumn}}%
                            }%
                        }%
                    }%
                }%
            }$}}}}%
        }%
    }%
}%
\usepackage{fp}
\newcommand{\fsize}[2]{{\fontsize{#1pt}{#1pt * \real{1.2}}\selectfont #2}}
\newcommand{\ffsize}[1]{\fontsize{#1pt}{#1pt * \real{1.2}}\selectfont}
\usepackage{stmaryrd}
\newcommand{\bfLUnitor}{\boldsymbol{\lambda}}
\newcommand{\bfRUnitor}{\boldsymbol{\rho}}
\newcommand{\bfalpha}{\boldsymbol{\alpha}}
\newcommand{\bfbeta}{\boldsymbol{\beta}}
\newcommand{\bfsigma}{\boldsymbol{\sigma}}
\newcommand{\bfepsilon}{\boldsymbol{\epsilon}}
\newcommand{\bfmu}{\boldsymbol{\mu}}
\newcommand{\textiff}{iff\xspace}
\newcommand{\OrdinalCategory}{\mathbbold{n}}
\newcommand{\OrdinalCategoryNPlus}[2]{\mathbbold{#1}+\mathbbold{#2}}
\newcommand{\OrdinalCategoryN}[1]{\mathbbold{#1}}
\newcommand{\Unit}{\mathbbold{1}}
\newcommand{\Zero}{\mathbbold{0}}
\newfontfamily\IPAFont[Path=ABSOLUTEPATH/fonts/brill/,Scale=1.0]{Brill-Roman.ttf}
\newcommand{\IPA}[1]{{\text{\IPAFont{#1}}}}
\newcommand{\coyo}{\mathchoice{\mathbin{\scalebox{1.0}{\rotatebox[origin=c]{180}{\Japanese{よ}}\!}}}{\mathbin{\scalebox{1.0}{\rotatebox[origin=c]{180}{\Japanese{よ}}\!}}}{\mathbin{\scalebox{0.75}{\rotatebox[origin=c]{180}{\Japanese{よ}}\!}}}{\mathbin{\scalebox{0.6}{\rotatebox[origin=c]{180}{\Japanese{よ}}\!}}}}
\newcommand{\AugmentedSimplexCategory}{\mathbbold{\Delta}_{+}}
\newcommand{\SimplexCategory}{\mathbbold{\Delta}}
\usepackage[mathscr]{eucal}
%\let\EuScript\relax
%\newcommand{\EuScript}[1]{\mathscr{#1}}
\newcommand{\CoContFun}{\FontForCategories{Fun}^{\rotatebox[origin=c]{180}{$\mathcal{C}$}}}
\newcommand{\CoContPos}{\FontForCategories{PosFun}^{\rotatebox[origin=c]{180}{$\mathcal{C}$}}}
\let\emptyset\relax
\newcommand{\emptyset}{\text{Ø}}
\let\mod\relax
\newcommand{\mod}[1]{\ \ (\mathrm{mod}\ #1)}
\newcommand{\mmod}[1]{\ (\mathrm{mod}\ #1)}
\newcommand{\less}{<}
\newcommand{\greater}{>}
\begingroup
\catcode`(\active \xdef({\left\string(}  % ( is defined as \left(
\catcode`)\active \xdef){\right\string)} % ) is defined as \right)
\catcode`[\active \xdef[{\left\string[}  % [ is defined as \left[
\catcode`]\active \xdef]{\right\string]} % ] is defined as \right]
\endgroup
\let\leftllbracket\relax
\let\rightrrbracket\relax
\newcommand{\leftllbracket}{\llbracket}
\newcommand{\rightrrbracket}{\rrbracket}

\fi
\raggedbottom
\usepackage{bigfoot}
\DeclareNewFootnote{default}
\newcommand{\ChapterRef}[2]{#2}
\newcommand{\say}[1]{``#1''}
\definecolor{darkGreen}{rgb}{0.2,0.5,0.2}
\definecolor{darkRed}{rgb}{0.65,0.05,0.05}%
\setlength\premulticols{10\baselineskip}
\newcommand{\UP}{({\footnotesize\textsc{\textbf{UP}}})}
\setcounter{secnumdepth}{3}
\setcounter{tocdepth}{3}
\makeatletter
\renewcommand\thesubsubsection{\thesubsection.\ifnum\value{subsubsection}=0 \relax 1\else \arabic{subsubsection}\fi}
\makeatother
\newcommand{\ptag}[1]{\tag{#1}}
\newcommand{\norg}{}
\author{The Clowder Project Authors}
\allowdisplaybreaks
\usepackage{listings}
\definecolor{codegreen}{rgb}{0,0.6,0}
\definecolor{codegray}{rgb}{0.5,0.5,0.5}
\definecolor{codepurple}{rgb}{0.58,0,0.82}
\definecolor{backcolour}{rgb}{0.95,0.95,0.92}
\usepackage{fontspec}
\usepackage{inconsolata}
\setmonofont{inconsolata}%
\usepackage[protrusion=true,expansion,tracking=true,babel]{microtype}
\usepackage{CrimsonPro}
\usepackage[libertine]{newtxmath}
%% The font package uses mweights.sty which has som issues with the
%% \normalfont command. The following two lines fixes this issue.
\let\oldnormalfont\normalfont
\def\normalfont{\oldnormalfont\mdseries}
\newcommand{\Russian}[1]{#1}
\usepackage[margin=4.5cm]{geometry}
