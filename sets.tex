\input{preamble}

% OK, start here.
%
\usepackage{fontspec}
\let\hyperwhite\relax
\let\hyperred\relax
\newcommand{\hyperwhite}{\hypersetup{citecolor=white,filecolor=white,linkcolor=white,urlcolor=white}}
\newcommand{\hyperred}{%
\hypersetup{%
    citecolor=TitlingRed,%
    filecolor=TitlingRed,%
    linkcolor=TitlingRed,%
     urlcolor=TitlingRed%
}}
\let\ChapterRef\relax
\newcommand{\ChapterRef}[2]{#1}
\setcounter{tocdepth}{2}
%▓▓▓▓▓▓▓▓▓▓▓▓▓▓▓▓▓▓▓▓▓▓▓▓▓▓▓▓▓▓▓▓▓
%▓▓ ╔╦╗╦╔╦╗╦  ╔═╗  ╔═╗╔═╗╔╗╔╔╦╗ ▓▓
%▓▓  ║ ║ ║ ║  ║╣   ╠╣ ║ ║║║║ ║  ▓▓
%▓▓  ╩ ╩ ╩ ╩═╝╚═╝  ╚  ╚═╝╝╚╝ ╩  ▓▓
%▓▓▓▓▓▓▓▓▓▓▓▓▓▓▓▓▓▓▓▓▓▓▓▓▓▓▓▓▓▓▓▓▓
%\usepackage{titlesec}
%▓▓▓▓▓▓▓▓▓▓▓▓▓▓▓▓▓▓▓▓▓▓▓▓▓▓▓▓▓▓▓▓▓▓▓▓▓▓▓▓▓▓▓▓▓▓▓▓▓▓▓▓▓▓▓
%▓▓ ╔╦╗╔═╗╔╗ ╦  ╔═╗  ╔═╗╔═╗  ╔═╗╔═╗╔╗╔╔╦╗╔═╗╔╗╔╔╦╗╔═╗ ▓▓
%▓▓  ║ ╠═╣╠╩╗║  ║╣   ║ ║╠╣   ║  ║ ║║║║ ║ ║╣ ║║║ ║ ╚═╗ ▓▓
%▓▓  ╩ ╩ ╩╚═╝╩═╝╚═╝  ╚═╝╚    ╚═╝╚═╝╝╚╝ ╩ ╚═╝╝╚╝ ╩ ╚═╝ ▓▓
%▓▓▓▓▓▓▓▓▓▓▓▓▓▓▓▓▓▓▓▓▓▓▓▓▓▓▓▓▓▓▓▓▓▓▓▓▓▓▓▓▓▓▓▓▓▓▓▓▓▓▓▓▓▓▓
\newcommand{\ChapterTableOfContents}{%
    \begingroup
    \addfontfeature{Numbers={Lining,Monospaced}}
    \hypersetup{hidelinks}\tableofcontents%
    \endgroup
}%

\makeatletter
\newcommand \DotFill {\leavevmode \cleaders \hb@xt@ .33em{\hss .\hss }\hfill \kern \z@}
\makeatother

\definecolor{ToCGrey}{rgb}{0.4,0.4,0.4}
\definecolor{mainColor}{rgb}{0.82745098,0.18431373,0.18431373}
\usepackage{titletoc}
\titlecontents{part}
[0.0em]
{\addvspace{1pc}\color{TitlingRed}\large\bfseries\text{Part }}
{\bfseries\textcolor{TitlingRed}{\contentslabel{0.0em}}\hspace*{1.35em}}
{}
{\textcolor{TitlingRed}{{\hfill\bfseries\contentspage\nobreak}}}
[]
\titlecontents{section}
[0.0em]
{\addvspace{1pc}}
{\color{black}\bfseries\textcolor{TitlingRed}{\contentslabel{0.0em}}\hspace*{1.35em}}
{}
{\textcolor{black}{\textbf{\DotFill}{\bfseries\contentspage\nobreak}}}
[]
\titlecontents{subsection}
[0.0em]
{}
{\hspace*{1.35em}\color{ToCGrey}{\contentslabel{0.0em}}\hspace*{2.1em}}
{}
{{\textcolor{ToCGrey}\DotFill}\textcolor{ToCGrey}{\contentspage}\nobreak}
[]
\usepackage{marginnote}
\renewcommand*{\marginfont}{\normalfont}
\usepackage{inconsolata}
\setmonofont{inconsolata}%
\let\ChapterRef\relax
\newcommand{\ChapterRef}[2]{#1}
\AtBeginEnvironment{subappendices}{%%
    \section*{\huge Appendices}%
}%

\begin{document}

\title{Sets}

\maketitle

\phantomsection
\label{section-phantom}

This chapter (will eventually) contain material on axiomatic set theory, as well as a couple other things.

\ChapterTableOfContents

\section{Sets and Functions}\label{section-sets-and-functions}
\subsection{Functions}\label{subsection-sets-and-functions-functions}
\begin{definition}{Functions}{functions}%
    A \index[set-theory]{function}\textbf{function} is a functional and total relation.
\end{definition}
\begin{notation}{Additional Notation for Functions}{additional-notation-for-functions}%
    Throughout this work, we will sometimes denote a function $f\colon X\to Y$ by
    \[
        f%
        \defeq%
        \llbracket x\mapsto f(x)\rrbracket.%
    \]%
    \begin{enumerate}
        \item For example, given a function%
            \[
                \Phi%
                \colon%
                \Hom_{\Sets}(X,Y)%
                \to%
                K%
            \]%
            taking values on a set of functions such as $\Hom_{\Sets}(X,Y)$, we will sometimes also write
            \[
                \Phi(f)%
                \defeq%
                \Phi(\llbracket x\mapsto f(x)\rrbracket).%
            \]%
        \item This notational choice is based on the lambda notation
            \[
                f%
                \defeq%
                (\lambda x.\ f(x)),%
            \]%
            but uses a \say{$\mathord{\mapsto}$} symbol for better spacing and double brackets instead of either:
            \begin{enumerate}
                \item Square brackets $[x\mapsto f(x)]$;
                \item Parentheses $(x\mapsto f(x))$;
            \end{enumerate}
            hoping to improve readability when dealing with e.g.:
            \begin{enumerate}
                \item Equivalence classes, cf.:
                    \begin{enumerate}
                        \item $\llbracket[x]\mapsto f([x])\rrbracket$%
                        \item $[[x]\mapsto f([x])]$%
                        \item $(\lambda[x].\ f([x]))$%
                    \end{enumerate}
                \item Function evaluations, cf.:
                    \begin{enumerate}
                        \item $\Phi(\llbracket x\mapsto f(x)\rrbracket)$%
                        \item $\Phi((x\mapsto f(x)))$%
                        \item $\Phi((\lambda x.\ f(x)))$%
                    \end{enumerate}
            \end{enumerate}
        \item We will also sometimes write $-_{1}$, $-_{2}$, etc.\ for the arguments of a function. Some examples include:
            \begin{enumerate}
                \item Writing $f(-_{1})$ for a function $f\colon A\to B$.
                \item Writing $f(-_{1},-_{2})$ for a function $f\colon A\times B\to C$.
                \item Given a function $f\colon A\times B\to C$, writing
                    \[
                        f(a,-)%
                        \colon%
                        B%
                        \to%
                        C%
                    \]%
                    for the function $\llbracket b\mapsto f(a,b)\rrbracket$.
                \item Denoting a composition of the form%
                    \[
                        A\times B%
                        \xlongrightarrow{\phi\times\id_{B}}%
                        A'\times B%
                        \xlongrightarrow{f}%
                        C%
                    \]%
                    by $f(\phi(-_{1}),-_{2})$.
            \end{enumerate}
        \item Finally, given a function $f\colon A\to B$, we write
            \[
                \ev_{a}(f)%
                \defeq%
                f(a)%
            \]%
            for the value of $f$ at some $a\in A$.
    \end{enumerate}
    For an example of the above notations being used in practice, see the proof of the adjunction
    \begin{webcompile}
        \AdjunctionShort#A\times -#{\Hom_{\Sets}(A,-)}#\Sets#\Sets,#
    \end{webcompile}
    stated in \ChapterRef{\ChapterConstructionsWithSets, \cref{constructions-with-sets:properties-of-products-of-sets-adjointness} of \cref{constructions-with-sets:properties-of-products-of-sets}}{\cref{properties-of-products-of-sets-adjointness} of \cref{properties-of-products-of-sets}}.
\end{notation}
\section{The Enrichment of Sets in Classical Truth Values}\label{section-the-enrichment-of-sets-in-classical-truth-values}
\subsection{\texorpdfstring{$(-2)$}{(-2)}-Categories}\label{subsection-minus-two-categories}
\begin{definition}{$(-2)$-Categories}{minus-two-categories}%
    A \index[set-theory]{minus two category@$(-2)$-category}\textbf{$(-2)$-category} is the \say{necessarily true} truth value.%
    %--- Begin Footnote ---%
    \footnote{%
        Thus, there is only one $(-2)$-category.
    }%
    %---  End Footnote  ---%
    %--- Begin Footnote ---%
    \footnote{%
        A $(-n)$-category for $n=3,4,\ldots$ is also the \say{necessarily true} truth value, coinciding with a $(-2)$-category.
    }%
    %---  End Footnote  ---%
    %--- Begin Footnote ---%
    \footnote{%
        For motivation, see \cite[p.~13]{lectures-on-n-categories-and-cohomology}.
        \par\vspace*{\TCBBoxCorrection}
    }%
    %---  End Footnote  ---%
\end{definition}
\subsection{\texorpdfstring{$(-1)$}{(-1)}-Categories}\label{subsection-minus-one-categories}
\begin{definition}{$(-1)$-Categories}{minus-one-categories}%
    A \index[set-theory]{minus one category@$(-1)$-category}\textbf{$(-1)$-category} is a classical truth value.%
\end{definition}
\begin{remark}{Motivation for $(-1)$-Categories}{motivation-for-minus-one-categories}%
    %--- Begin Footnote ---%
    \footnote{%
        For more motivation, see \cite[p.~13]{lectures-on-n-categories-and-cohomology}.
        \par\vspace*{\TCBBoxCorrection}
    }%
    %---  End Footnote  ---%
    $(-1)$-categories should be thought of as being \say{categories enriched in $(-2)$-categories}, having a collection of objects and, for each pair of objects, a $\Hom$-object $\Hom(x,y)$ that is a $(-2)$-category (i.e.\ trivial).
    % BEGIN RAW HTML %
    <hr style="margin-top: 0.0em; visibility: hidden;">
    % BEGIN LATEX HTML %

    \vspace{0.5\baselineskip}
    % END RAW HTML %
    Therefore, a $(-1)$-category $\CatFont{C}$ is either:%
    %--- Begin Footnote ---%
    \footnote{%
        See \cite[pp.~33--34]{lectures-on-n-categories-and-cohomology}.
    }%
    %---  End Footnote  ---%
    \begin{enumerate}
        \item\label{motivation-for-minus-one-categories-empty}\emph{Empty}, having no objects.
        \item\label{motivation-for-minus-one-categories-contractible}\emph{Contractible}, having a collection of objects $\{a,b,c,\ldots\}$, but with $\Hom_{\CatFont{C}}(a,b)$ being a $(-2)$-category (i.e.\ trivial) for all $a,b\in\Obj(\CatFont{C})$, forcing all objects of $\CatFont{C}$ to be uniquely isomorphic to each other.
    \end{enumerate}
    As such, there are only two $(-1)$-categories, up to equivalence:
    \begin{enumerate}
        \item\label{motivation-for-minus-one-categories-false}The $(-1)$-category $\false$ (the empty one);
        \item\label{motivation-for-minus-one-categories-true}The $(-1)$-category $\true$  (the contractible one).%
    \end{enumerate}
\end{remark}
\begin{definition}{The Poset of Truth Values}{the-poset-of-truth-values}%
    The \index[set-theory]{poset!of truth values}\textbf{poset of truth values}%
    %--- Begin Footnote ---%
    \footnote{%
        \SloganFont{Further Terminology: }Also called the \index[set-theory]{poset!of minus one categories@of $(-1)$-categories}\textbf{poset of $(-1)$-categories}.%
    } %
    %---  End Footnote  ---%
    is the poset \index[notation]{truefalse@$\TV$}$\smash{(\TV,\preceq)}$ consisting of%
    \begin{itemize}
        \item\SloganFont{The Underlying Set. }The set $\TV$ whose elements are the truth values $\true$ and $\false$.
        \item\SloganFont{The Partial Order. }The partial order
            \[
                \preceq%
                \colon%
                \TV\times\TV%
                \to
                \TV%
            \]%
            on $\TV$ defined by%
            %--- Begin Footnote ---%
            \footnote{%
                This partial order coincides with logical implication.
                \par\vspace*{\TCBBoxCorrection}
            }%
            %---  End Footnote  ---%
            \begin{align*}
                \false\preceq\false &\defeq \true,\\
                \true\preceq\false  &\defeq \false,\\
                \false\preceq\true  &\defeq \true,\\
                \true\preceq\true   &\defeq \true.
            \end{align*}
    \end{itemize}
\end{definition}
\begin{notation}{Further Notation for The Poset of Truth Values}{further-notation-for-the-poset-of-truth-values}%
    We also write \index[notation]{tf@$\TTV$}$\TTV$ for the poset $\TV$.
\end{notation}
\begin{proposition}{Cartesian Closedness of the Poset of Truth Values}{cartesian-closedness-of-the-poset-of-truth-values}%
    The poset of truth values $\TTV$ is Cartesian closed with product given by%
    %--- Begin Footnote ---%
    \footnote{%
        Note that $\times$ coincides with the \say{and} operator, while $\eHom_{\TTV}$ coincides with the logical implication operator.
        \par\vspace*{\TCBBoxCorrection}
    }%
    %---  End Footnote  ---%
    \begin{align*}
        \ttrue  \times \ttrue  &= \ttrue,\\
        \ttrue  \times \tfalse &= \tfalse,\\
        \tfalse \times \ttrue  &= \tfalse,\\
        \tfalse \times \tfalse &= \tfalse,
    \end{align*}
    and internal Hom $\eHom_{\TTV}$ given by the partial order of $\TTV$, i.e.\ by
    \begin{align*}
        \eHom_{\TTV}(\ttrue,\ttrue)   &= \ttrue,\\
        \eHom_{\TTV}(\ttrue,\tfalse)  &= \tfalse,\\
        \eHom_{\TTV}(\tfalse,\ttrue)  &= \ttrue,\\
        \eHom_{\TTV}(\tfalse,\tfalse) &= \ttrue.
    \end{align*}
\end{proposition}
\begin{Proof}{Proof of \cref{cartesian-closedness-of-the-poset-of-truth-values}}%
    \ProofBox{Existence of Products}%
    We claim that the products $\ttrue\times\ttrue$, $\ttrue\times\tfalse$, $\tfalse\times\ttrue$, and $\tfalse\times\tfalse$ satisfy the universal property of the product in $\TTV$. Indeed, consider the diagrams
    \begin{webcompile}
        \resizebox{\textwidth}{!}{$%
            \begin{tikzcd}[row sep={4.0*\the\DL,between origins}, column sep={4.0*\the\DL,between origins}, background color=backgroundColor, ampersand replacement=\&]
                \&
                P_{1}
                \arrow[rd, "p^{1}_{2}", bend left=15]
                \arrow[d, "\exists!" description,dashed]
                \arrow[ld, "p^{1}_{1}"', bend right=15]
                \&
                \\
                \ttrue
                \&
                \underbrace{\ttrue\times\ttrue}_{=\ttrue}%
                \arrow[l, "\pr_{1}", two heads]
                \arrow[r, "\pr_{2}"', two heads]
                \&
                \ttrue
            \end{tikzcd}
            \quad
            \begin{tikzcd}[row sep={4.0*\the\DL,between origins}, column sep={4.0*\the\DL,between origins}, background color=backgroundColor, ampersand replacement=\&]
                \&
                P_{2}
                \arrow[rd, "p^{2}_{2}", bend left=15]
                \arrow[d, "\exists!" description,dashed]
                \arrow[ld, "p^{2}_{1}"', bend right=15]
                \&
                \\
                \ttrue
                \&
                \underbrace{\ttrue\times\tfalse}_{=\tfalse}%
                \arrow[l, "\pr_{1}", two heads]
                \arrow[r, "\pr_{2}"', two heads]
                \&
                \tfalse
            \end{tikzcd}
            \quad
            \begin{tikzcd}[row sep={4.0*\the\DL,between origins}, column sep={4.0*\the\DL,between origins}, background color=backgroundColor, ampersand replacement=\&]
                \&
                P_{3}
                \arrow[rd, "p^{3}_{2}", bend left=15]
                \arrow[d, "\exists!" description,dashed]
                \arrow[ld, "p^{3}_{1}"', bend right=15]
                \&
                \\
                \tfalse
                \&
                \underbrace{\tfalse\times\ttrue}_{=\tfalse}%
                \arrow[l, "\pr_{1}", two heads]
                \arrow[r, "\pr_{2}"', two heads]
                \&
                \ttrue
            \end{tikzcd}
            \quad
            \begin{tikzcd}[row sep={4.0*\the\DL,between origins}, column sep={4.0*\the\DL,between origins}, background color=backgroundColor, ampersand replacement=\&]
                \&
                P_{4}
                \arrow[rd, "p^{4}_{2}", bend left=15]
                \arrow[d, "\exists!" description,dashed]
                \arrow[ld, "p^{4}_{1}"', bend right=15]
                \&
                \\
                \tfalse
                \&
                \underbrace{\tfalse\times\tfalse}_{=\tfalse}%
                \arrow[l, "\pr_{1}", two heads]
                \arrow[r, "\pr_{2}"', two heads]
                \&
                \tfalse\mrp{.}
            \end{tikzcd}
        $}%
    \end{webcompile}
    Here:
    \begin{enumerate}
        \item If $P_{1}=\ttrue$, then $p^{1}_{1}=p^{1}_{2}=\id_{\ttrue}$, and there's indeed a unique morphism from $P_{1}$ to $\ttrue$ making the diagram commute, namely $\id_{\ttrue}$;
        \item If $P_{1}=\tfalse$, then $p^{1}_{1}=p^{1}_{2}$ are given by the unique morphism from $\tfalse$ to $\ttrue$, and there's indeed a unique morphism from $P_{1}$ to $\ttrue$ making the diagram commute, namely the unique morphism from $\tfalse$ to $\ttrue$;
        \item If $P_{2}=\ttrue$, then there is no morphism $p^{2}_{2}$.
        \item If $P_{2}=\tfalse$, then $p^{2}_{1}$ is the unique morphism from $\tfalse$ to $\ttrue$ while $p^{2}_{2}=\id_{\tfalse}$, and there's indeed a unique morphism from $P_{2}$ to $\tfalse$ making the diagram commute, namely $\id_{\tfalse}$;
        \item The proof for $P_{3}$ is similar to the one for $P_{2}$;
        \item If $P_{4}=\ttrue$, then there is no morphism $p^{4}_{1}$ or $p^{4}_{2}$.
        \item If $P_{4}=\tfalse$, then $p^{4}_{1}=p^{4}_{2}=\id_{\tfalse}$, and there's indeed a unique morphism from $P_{4}$ to $\tfalse$ making the diagram commute, namely $\id_{\tfalse}$.
    \end{enumerate}

    \ProofBox{Cartesian Closedness}%
    We claim there's a bijection
    \[
        \Hom_{\TTV}(A\times B,C)%
        \cong%
        \Hom_{\TTV}(A,\eHom_{\TTV}(B,C))%
    \]%
    natural in $A,B,C\in\TTV$. Indeed:
    \begin{itemize}
        \item For $(A,B,C)=(\ttrue,\ttrue,\ttrue)$, we have
            \begin{align*}
                \Hom_{\TTV}(\ttrue\times\ttrue,\ttrue)    &\cong \Hom_{\TTV}(\ttrue,\ttrue)\\
                                                          &=     \{\id_{\true}\}\\
                                                          &\cong \Hom_{\TTV}(\ttrue,\ttrue)\\
                                                          &\cong \Hom_{\TTV}(\ttrue,\eHom_{\TTV}(\ttrue,\ttrue)).
            \end{align*}
        \item For $(A,B,C)=(\ttrue,\ttrue,\tfalse)$, we have
            \begin{align*}
                \Hom_{\TTV}(\ttrue\times\ttrue,\tfalse)   &\cong \Hom_{\TTV}(\ttrue,\tfalse)\\
                                                          &=     \emptyset\\
                                                          &\cong \Hom_{\TTV}(\ttrue,\tfalse)\\
                                                          &\cong \Hom_{\TTV}(\ttrue,\eHom_{\TTV}(\ttrue,\tfalse)).
            \end{align*}
        \item For $(A,B,C)=(\ttrue,\tfalse,\ttrue)$, we have
            \begin{align*}
                \Hom_{\TTV}(\ttrue\times\tfalse,\ttrue)   &\cong \Hom_{\TTV}(\tfalse,\ttrue)\\
                                                          &\cong \pt\\
                                                          &\cong \Hom_{\TTV}(\tfalse,\ttrue)\\
                                                          &\cong \Hom_{\TTV}(\tfalse,\eHom_{\TTV}(\tfalse,\ttrue)).
            \end{align*}
        \item For $(A,B,C)=(\ttrue,\tfalse,\tfalse)$, we have
            \begin{align*}
                \Hom_{\TTV}(\ttrue\times\tfalse,\tfalse)  &\cong \Hom_{\TTV}(\tfalse,\tfalse)\\
                                                          &\cong \{\id_{\false}\}\\
                                                          &\cong \Hom_{\TTV}(\tfalse,\tfalse)\\
                                                          &\cong \Hom_{\TTV}(\ttrue,\eHom_{\TTV}(\tfalse,\tfalse)).
            \end{align*}
        \item For $(A,B,C)=(\tfalse,\ttrue,\ttrue)$, we have
            \begin{align*}
                \Hom_{\TTV}(\tfalse\times\ttrue,\ttrue)   &\cong \Hom_{\TTV}(\tfalse,\ttrue)\\
                                                          &\cong \pt\\
                                                          &\cong \Hom_{\TTV}(\tfalse,\ttrue)\\
                                                          &\cong \Hom_{\TTV}(\tfalse,\eHom_{\TTV}(\ttrue,\ttrue)).
            \end{align*}
        \item For $(A,B,C)=(\tfalse,\ttrue,\tfalse)$, we have
            \begin{align*}
                \Hom_{\TTV}(\tfalse\times\ttrue,\tfalse)  &\cong \Hom_{\TTV}(\tfalse,\tfalse)\\
                                                          &\cong \{\id_{\false}\}\\
                                                          &\cong \Hom_{\TTV}(\tfalse,\tfalse)\\
                                                          &\cong \Hom_{\TTV}(\tfalse,\eHom_{\TTV}(\ttrue,\tfalse)).
            \end{align*}
        \item For $(A,B,C)=(\tfalse,\tfalse,\ttrue)$, we have
            \begin{align*}
                \Hom_{\TTV}(\tfalse\times\tfalse,\ttrue)  &\cong \Hom_{\TTV}(\tfalse,\ttrue)\\
                                                          &\cong \pt\\
                                                          &\cong \Hom_{\TTV}(\tfalse,\ttrue)\\
                                                          &\cong \Hom_{\TTV}(\tfalse,\eHom_{\TTV}(\tfalse,\ttrue)).
            \end{align*}
        \item For $(A,B,C)=(\tfalse,\tfalse,\tfalse)$, we have
            \begin{align*}
                \Hom_{\TTV}(\tfalse\times\tfalse,\tfalse) &\cong \Hom_{\TTV}(\tfalse,\tfalse)\\
                                                          &=     \{\id_{\false}\}\\
                                                          &\cong \Hom_{\TTV}(\tfalse,\tfalse)\\
                                                          &\cong \Hom_{\TTV}(\tfalse,\eHom_{\TTV}(\tfalse,\tfalse)).
            \end{align*}
    \end{itemize}
    The proof of naturality is omitted.
\end{Proof}
\subsection{$0$-Categories}\label{subsection-zero-categories}
\begin{definition}{$0$-Categories}{zero-categories}%
    A \index[set-theory]{zero category@$0$-category}\textbf{$0$-category} is a poset.%
    %--- Begin Footnote ---%
    \footnote{%
        \SloganFont{Motivation: }A $0$-category is precisely a category enriched in the poset of $(-1)$-categories.
        \par\vspace*{\TCBBoxCorrection}
    }%
    %---  End Footnote  ---%
\end{definition}
\begin{definition}{$0$-Groupoids}{zero-groupoids}%
    A \index[set-theory]{zero groupoid@$0$-groupoid}\textbf{$0$-groupoid} is a $0$-category in which every morphism is invertible.%
    %--- Begin Footnote ---%
    \footnote{%
        That is, a \emph{set}.
        \par\vspace*{\TCBBoxCorrection}
    }%
    %---  End Footnote  ---%
\end{definition}
\subsection{Tables of Analogies Between Set Theory and Category Theory}\label{subsection-table-of-analogies-between-set-theory-and-category-theory}
Here we record some analogies between notions in set theory and category theory. The analogies relating to presheaves relate equally well to copresheaves, as the opposite $X^{\mathrm{op}}$ of a set $X$ is just $X$ again.
\begin{remark}{Basic Analogies Between Set Theory and Category Theory}{basic-analogies-between-set-theory-and-category-theory}%
    The basic analogies between set theory and category theory are summarised in the following table:
    % BEGIN RAW HTML %
    <div class="site-table-container" style="margin-top:1rem;">
      <div class="site-table">
        <table>
          <tr>
            <th style="text-align: center; background-color: #A60D0D; color: white;"><b>Set Theory</b></th>
            <th style="text-align: center; background-color: #A60D0D; color: white;"><b>Category Theory</b></th>
          </tr>
          <tr>
            <td>Enrichment in $\TV$</td>
            <td>Enrichment in $\Sets$</td>
          </tr>
          <tr>
            <td>Set $X$</td>
            <td>Category $\mathcal{C}$</td>
          </tr>
          <tr>
            <td>Element $x\in X$</td>
            <td>Object $X\in\Obj(\mathcal{C})$</td>
          </tr>
          <tr>
            <td>Function</td>
            <td>Functor</td>
          </tr>
          <tr>
            <td>Function $X\to\TV$</td>
            <td>Copresheaf $\mathcal{C}\to\mathsf{Sets}$</td>
          </tr>
          <tr>
            <td>Function $X\to\TV$</td>
            <td>Presheaf $\mathcal{C}^{\op}\to\mathsf{Sets}$</td>
          </tr>
        </table>
      </div>
    </div>
    % BEGIN LATEX HTML %
    \begingroup%
    \renewcommand{\arraystretch}{1.2}
    \begin{center}
        \begin{tabular}{|c|c|}\hline\rowcolor{darkRed}
            \textcolor{white}{\textbf{\textsc{Set Theory}}} & \textcolor{white}{\textsc{Category Theory}} \\\hline\rowcolor{backgroundColor}
            Enrichment in $\TV$                    & Enrichment in $\Sets$                       \\\rowcolor{black!05!backgroundColor}
            Set $X$                                & Category $\CatFont{C}$                      \\\rowcolor{backgroundColor}
            Element $x\in X$                       & Object $X\in\Obj(\CatFont{C})$              \\\rowcolor{black!05!backgroundColor}
            Function                               & Functor                                     \\\rowcolor{backgroundColor}
            Function $X\to\TV$                     & Copresheaf $\CatFont{C}\to\Sets$              \\\rowcolor{black!05!backgroundColor}
            Function $X\to\TV$                     & Presheaf $\CatFont{C}^{\op}\to\Sets$        \\\hline
        \end{tabular}
    \end{center}
    \endgroup
    % END RAW HTML %
\end{remark}
\begin{remark}{Analogies Between Set Theory and Category Theory: Powersets and Categories of Presheaves}{analogies-between-set-theory-and-category-theory-powersets-and-categories-of-presheaves}%
    The category of presheaves $\PSh(\CatFont{C})$ and the category of copresheaves $\CoPSh(\CatFont{C})$ on a category $\CatFont{C}$ are the 1-categorical counterparts to the powerset $\mathcal{P}(X)$ of subsets of a set $X$. The further analogies built upon this are summarised in the following table:
    % BEGIN RAW HTML %
    <div class="site-table-container" style="margin-top:1rem;">
      <div class="site-table">
        <table>
          <tr>
            <th style="text-align: center; background-color: #A60D0D; color: white;"><b>Set Theory</b></th>
            <th style="text-align: center; background-color: #A60D0D; color: white;"><b>Category Theory</b></th>
          </tr>
          <tr>
            <td>Powerset $\mathcal{P}(X)$<br>(\ChapterRef{\ChapterConstructionsWithSets, \cref{constructions-with-sets:powersets}}{\cref{powersets}})</td>
            <td>Presheaf category $\mathsf{PSh}(\mathcal{C})$<br>(\ChapterRef{\ChapterPresheavesAndTheYonedaLemma, \cref{presheaves-and-the-yoneda-lemma:the-category-of-presheaves-on-a-category}}{\cref{the-category-of-presheaves-on-a-category}})</td>
          </tr>
          <tr>
            <td>Characteristic function<br>$\chi_{x}\colon X\to\TTV$<br>(\ChapterRef{\ChapterConstructionsWithSets, \cref{constructions-with-sets:characteristic-functions}}{\cref{characteristic-functions}})</td>
            <td>Representable Presheaf<br>$h_{X}\colon\mathcal{C}^{\op}\to\Sets$<br>(\ChapterRef{\ChapterPresheavesAndTheYonedaLemma, \cref{presheaves-and-the-yoneda-lemma:representable-presheaves-representable-presheaves-associated-to-an-object} of \cref{presheaves-and-the-yoneda-lemma:representable-presheaves}}{\cref{representable-presheaves-representable-presheaves-associated-to-an-object} of \cref{representable-presheaves}})</td>
          </tr>
          <tr>
            <td>Characteristic embedding<p style="margin-top:0.5em;margin-bottom:0.5em">$\chi_{(-)}\colon X\hookrightarrow\mathcal{P}(X)$</p>(\ChapterRef{\ChapterConstructionsWithSets, \cref{constructions-with-sets:characteristic-functions-characteristic-embedding} of \cref{constructions-with-sets:characteristic-functions}}{\cref{characteristic-functions-characteristic-relation} of \cref{characteristic-functions}})</td>
            <td>Yoneda embedding<p style="margin-top:0.5em;margin-bottom:0.5em">$\yo\colon\mathcal{C}\hookrightarrow\mathsf{PSh}(\mathcal{C})$</p>(\ChapterRef{\ChapterPresheavesAndTheYonedaLemma, \cref{presheaves-and-the-yoneda-lemma:the-yoneda-embedding}}{\cref{the-yoneda-embedding}})</td>
          </tr>
          <tr>
          <td>Characteristic relation $\chi_{X}(-_{1},-_{2})\colon X\times X\to\TTV$<br>(\ChapterRef{\ChapterConstructionsWithSets, \cref{constructions-with-sets:characteristic-functions-characteristic-relation} of \cref{constructions-with-sets:characteristic-functions}}{\cref{characteristic-functions-characteristic-relation} of \cref{characteristic-functions}})</td>
          <td>$\Hom$ profunctor $\Hom_{\mathcal{C}}(-_{1},-_{2})\colon\CatFont{C}^{\op}\times\CatFont{C}\to\Sets$<br>(\cref{TODO})</td>
          </tr>
          <tr>
          <td>The Yoneda lemma for sets<p style="margin-top:0.5em;margin-bottom:0.5em">$\Hom_{\mathcal{P}(X)}(\chi_{x},\chi_{U})\cong\chi_{U}(x)$</p>(\ChapterRef{\ChapterConstructionsWithSets, \cref{constructions-with-sets:the-yoneda-lemma-for-sets}}{\cref{presheaves-and-the-yoneda-lemma-for-sets:the-yoneda-lemma}})</td>
            <td>The Yoneda lemma for categories<p style="margin-top:0.5em;margin-bottom:0.5em">$\Nat(h_{X},\SheafFont{F})\cong\SheafFont{F}(X)$</p>(\ChapterRef{\ChapterPresheavesAndTheYonedaLemma, \cref{presheaves-and-the-yoneda-lemma:the-yoneda-lemma}}{\cref{presheaves-and-the-yoneda-lemma:the-yoneda-lemma}})</td>
          </tr>
          <tr>
            <td>The characteristic<br>embedding is fully faithful,<p style="margin-top:0.5em;margin-bottom:0.5em">$\Hom_{\mathcal{P}(X)}(\chi_{x},\chi_{y})\cong\chi_{X}(x,y)$</p>(\ChapterRef{\ChapterConstructionsWithSets, \cref{constructions-with-sets:the-characteristic-embedding-is-fully-faithful}}{\cref{presheaves-and-the-characteristic-embedding-is-fully-faithful:the-yoneda-lemma}})</td>
            <td>The Yoneda<br>embedding is fully faithful,<p style="margin-top:0.5em;margin-bottom:0.5em">$\Nat(h_{X},h_{Y})\cong\Hom_{\mathcal{C}}(X,Y)$</p>(\ChapterRef{\ChapterPresheavesAndTheYonedaLemma, \cref{presheaves-and-the-yoneda-lemma:properties-of-the-yoneda-embedding-fully-faithfulness} of \cref{presheaves-and-the-yoneda-lemma:properties-of-the-yoneda-embedding}}{\cref{properties-of-the-yoneda-embedding-fully-faithfulness} of \cref{properties-of-the-yoneda-embedding}})</td>
          </tr>
          <tr>
            <td>Subsets are unions<br>of their elements,<p style="margin-top:0.5em;margin-bottom:0.5em">$\displaystyle U=\bigcup_{x\in U}\{x\}$<br>or<br>$\displaystyle\chi_{U}=\mkern-18mu\colim_{\chi_{x}\in\Sets(U,\TTV)}(\chi_{x})$</p></td>
            <td>Presheaves are colimits<br>of representables,<p style="margin-top:0.5em;margin-bottom:0.5em">$\displaystyle\SheafFont{F}\cong\colim_{h_{X}\in\int_{\mathcal{C}}\SheafFont{F}}(h_{X})$</p>(\cref{TODO})</td>
          </tr>
        </table>
      </div>
    </div>
    % BEGIN LATEX HTML %
    \begingroup%
    \setlength\cellspacetoplimit{3pt}
    \setlength\cellspacebottomlimit{3pt}
    \renewcommand{\arraystretch}{1.2}
    \begin{center}
        \begin{tabular}{|Sc|Sc|}\hline\rowcolor{darkRed}
            \textcolor{white}{\textbf{\textsc{Set Theory}}}                                                                                                                                    & \textcolor{white}{\textbf{\textsc{Category Theory}}}                                                                                                   \\\hline\rowcolor{backgroundColor}
            Powerset $\mathcal{P}(X)$                                                                                                                                                          & Presheaf category $\PSh(\CatFont{C})$                                                                                                         \\\rowcolor{black!05!backgroundColor}
            \Gape[0pt][2pt]{\makecell{Characteristic function\\$\chi_{\{x\}}\colon X\to\TTV$}}                                                                                                                                             & \Gape[0pt][2pt]{\makecell{Representable presheaf\\$h_{X}\colon\CatFont{C}^{\op}\hookrightarrow\Sets$}}                                                                                                                \\\rowcolor{backgroundColor}
            \Gape[0pt][2pt]{\makecell{Characteristic embedding\\$\chi_{(-)}\colon X\hookrightarrow\mathcal{P}(X)$}}                                                                            & \Gape[0pt][2pt]{\makecell{Yoneda embedding\\$\yo\colon\CatFont{C}^{\op}\hookrightarrow\PSh(\CatFont{C})$}}                                                    \\\rowcolor{black!05!backgroundColor}
            \Gape[0pt][2pt]{\makecell{Characteristic relation\\$\chi_{X}(-_{1},-_{2})\colon X\times X\to\TTV$}}                                                                            & \Gape[0pt][2pt]{\makecell{Hom profunctor\\$\Hom_{\CatFont{C}}(-_{1},-_{2})\colon\CatFont{C}^{\op}\times\CatFont{C}\to\Sets$}}                                                    \\
            \Gape[0pt][2pt]{\makecell{The Yoneda lemma for sets\\$\Hom_{\mathcal{P}(X)}(\chi_{x},\chi_{U})=\chi_{U}(x)$}}                                                               & \Gape[0pt][2pt]{\makecell{The Yoneda lemma for categories\\$\Nat(h_{X},\SheafFont{F})\cong\SheafFont{F}(X)$}}                                                  \\\rowcolor{black!05!backgroundColor}
            \Gape[0pt][2pt]{\makecell{The characteristic\\ embedding is fully faithful,\\$\Hom_{\mathcal{P}(X)}(\chi_{x},\chi_{y})=\chi_{X}(x,y)$}}                                     & \Gape[0pt][2pt]{\makecell{The Yoneda embedding\\is fully faithful,\\$\Nat(h_{X},h_{Y})\cong\Hom_{\CatFont{C}}(X,Y)$}}                                        \\\rowcolor{backgroundColor}
            \Gape[0pt][2pt]{\makecell{Subsets are unions\\of their elements\\ $\displaystyle U=\bigcup_{x\in U}\{x\}$\\or\\$\displaystyle\chi_{U}=\colim_{\chi_{x}\in\mathcal{P}(U)}(\chi_{x})$}} & \Gape[0pt][2pt]{\makecell{Presheaves are colimits\\of representables,\\$\displaystyle\SheafFont{F}\cong\colim_{h_{X}\in\int_{\mathcal{C}}\SheafFont{F}}(h_{X})$}} \\\hline
        \end{tabular}
    \end{center}
    \endgroup
    % END RAW HTML %
\end{remark}
\begin{remark}{Analogies Between Set Theory and Category Theory: Categories of Elements}{analogies-between-set-theory-and-category-theory-categories-of-elements}%
    We summarise the analogies between un/straightening in set theory and category theory in the following table:
    % BEGIN RAW HTML %
    <div class="site-table-container" style="margin-top:1rem;">
      <div class="site-table" style="transform: translateX(4.875rem);">
        <table style="width: 76\%;">
          <tr>
            <th style="text-align: center; background-color: #A60D0D; color: white;"><b>Set Theory</b></th>
            <th style="text-align: center; background-color: #A60D0D; color: white;"><b>Category Theory</b></th>
          </tr>
          <tr>
            <td>Assignment $U\mapsto\chi_{U}$</td>
            <td>Assignment $\SheafFont{F}\mapsto\int_{\mathcal{C}}\SheafFont{F}$<br>(the category<br>of elements)</td>
          </tr>
          <tr>
            <td>Un/straightening isomorphism<br>$\mathcal{P}(X)\cong\Sets(X,\TTV)$</p>(\ChapterRef{\ChapterConstructionsWithSets, \cref{constructions-with-sets:properties-of-powersets-as-sets-of-functions-relations-powersets-as-sets-of-functions-1} of \cref{constructions-with-sets:properties-of-powersets-as-sets-of-functions-relations}}{\cref{properties-of-powersets-as-sets-of-functions-relations-powersets-as-sets-of-functions-1} of \cref{properties-of-powersets-as-sets-of-functions-relations}})</td>
            <td>Un/straightening equivalence<br>$\PSh(\mathcal{C})\eqcong\DFib(\mathcal{C})$</p>(\cref{TODO})</td>
          </tr>
        </table>
      </div>
    </div>
    % BEGIN LATEX HTML %
    Categories of elements:
    \begingroup%
    \setlength\cellspacetoplimit{3pt}
    \setlength\cellspacebottomlimit{3pt}
    \renewcommand{\arraystretch}{1.2}
    \begin{center}
        \begin{tabular}{|Sc|Sc|}\hline\rowcolor{darkRed}
            \textcolor{white}{\textbf{\textsc{Set Theory}}}                                                                      & \textcolor{white}{\textbf{\textsc{Category Theory}}}                                                                                                                        \\\hline\rowcolor{backgroundColor}
            \Gape[0pt][2pt]{\makecell{Assignment\\$U\mapsto\chi_{U}$}}                                                                                        & \Gape[0pt][2pt]{\makecell{Assignment\\$\SheafFont{F}\mapsto\catEl{\CatFont{C}}{\SheafFont{F}}$\\(the category of elements)}}                                          \\\rowcolor{black!05!backgroundColor}
            \Gape[0pt][2pt]{\makecell{Un/straightening isomorphism\\$\mathcal{P}(X)\cong\Sets(X,\TTV)$}} & \Gape[0pt][2pt]{\makecell{Un/straightening equivalence\\$\PSh(\CatFont{C})\eqcong\DFib(\CatFont{C})$}} \\\hline
        \end{tabular}
    \end{center}
    \endgroup
    % END RAW HTML %
\end{remark}
\begin{remark}{Analogies Between Set Theory and Category Theory: Functions Between Powersets and Functors Between Presheaf Categories}{analogies-between-set-theory-and-category-theory-functions-between-powersets-and-functors-between-presheaf-categories}%
    We summarise the analogies between functions $\mathcal{P}(A)\to\mathcal{P}(B)$ and functors $\PSh(\CatFont{C})\to\PSh(\CatFont{D})$ in the following table:
    % BEGIN RAW HTML %
    <div class="site-table-container" style="margin-top:1rem;">
      <div class="site-table" style="transform: translateX(6rem);">
        <table style="width: 71\%;">
          <tr>
            <th style="text-align: center; background-color: #A60D0D; color: white;"><b>Set Theory</b></th>
            <th style="text-align: center; background-color: #A60D0D; color: white;"><b>Category Theory</b></th>
          </tr>
          <tr>
            <td>Direct image function<p style="margin-top:0.5em;margin-bottom:0.5em">$f_{*}\colon\mathcal{P}(X)\to\mathcal{P}(Y)$</p>(\ChapterRef{\ChapterConstructionsWithSets, \cref{constructions-with-sets:the-direct-image-function-associated-to-a-function}}{\cref{the-direct-image-function-associated-to-a-function}})</td>
            <td>Direct image functor<p style="margin-top:0.5em;margin-bottom:0.5em">$F_{*}\colon\mathsf{PSh}(\mathcal{C})\to\mathsf{PSh}(\mathcal{D})$</p>(\cref{TODO})</td>
          </tr>
          <tr>
            <td>Inverse image function<p style="margin-top:0.5em;margin-bottom:0.5em">$f^{-1}\colon\mathcal{P}(Y)\to\mathcal{P}(X)$</p>(\ChapterRef{\ChapterConstructionsWithSets, \cref{constructions-with-sets:the-inverse-image-function-associated-to-a-function}}{\cref{the-inverse-image-function-associated-to-a-function}})</td>
            <td>Inverse image functor<p style="margin-top:0.5em;margin-bottom:0.5em">$F^{*}\colon\mathsf{PSh}(\mathcal{D})\to\mathsf{PSh}(\mathcal{C})$</p>(\cref{TODO})</td>
          </tr>
          <tr>
            <td>Direct image with<br>compact support function<p style="margin-top:0.5em;margin-bottom:0.5em">$f_{!}\colon\mathcal{P}(X)\to\mathcal{P}(Y)$</p>(\ChapterRef{\ChapterConstructionsWithSets, \cref{constructions-with-sets:the-direct-image-with-compact-support-function-associated-to-a-function}}{\cref{the-direct-image-with-compact-support-function-associated-to-a-function}})</td>
            <td>Direct image with<br>compact support functor<p style="margin-top:0.5em;margin-bottom:0.5em">$F_{!}\colon\mathsf{PSh}(\mathcal{C})\to\mathsf{PSh}(\mathcal{D})$</p>(\cref{TODO})</td>
          </tr>
        </table>
      </div>
    </div>
    % BEGIN LATEX HTML %
    \begingroup%
    \setlength\cellspacetoplimit{3pt}
    \setlength\cellspacebottomlimit{3pt}
    \renewcommand{\arraystretch}{1.2}
    \begin{center}
        \begin{tabular}{|Sc|Sc|}\hline\rowcolor{darkRed}
            \textcolor{white}{\textbf{\textsc{Set Theory}}}                                                                       & \textcolor{white}{\textbf{\textsc{Category Theory}}}                                                                                \\\hline\rowcolor{backgroundColor}
            \Gape[0pt][2pt]{\makecell{Direct image function\\$f_{*}\colon\mathcal{P}(X)\to\mathcal{P}(Y)$}}                       & \Gape[0pt][2pt]{\makecell{Direct image functor\\$F_{*}\colon\PSh(\CatFont{C})\to\PSh(\CatFont{D})$}}                     \\\rowcolor{black!05!backgroundColor}
            \Gape[0pt][2pt]{\makecell{Inverse image function\\$f^{-1}\colon\mathcal{P}(Y)\to\mathcal{P}(X)$}}                     & \Gape[0pt][2pt]{\makecell{Inverse image functor\\$F^{*}\colon\PSh(\CatFont{D})\to\PSh(\CatFont{C})$}}                       \\\rowcolor{backgroundColor}
            \Gape[0pt][2pt]{\makecell{Direct image with\\compact support function\\$f_{!}\colon\mathcal{P}(X)\to\mathcal{P}(Y)$}} & \Gape[0pt][2pt]{\makecell{Direct image with\\compact support functor\\$F_{!}\colon\PSh(\CatFont{C})\to\PSh(\CatFont{D})$}} \\\hline
        \end{tabular}
    \end{center}
    \endgroup
    % END RAW HTML %
\end{remark}
\begin{remark}{Analogies Between Set Theory and Category Theory: Relations and Profunctors}{analogies-between-set-theory-and-category-theory-relations-and-profunctors}%
    We summarise the analogies between functions relations and profunctors in the following table:
    % BEGIN RAW HTML %
    <div class="site-table-container" style="margin-top:1rem;">
      <div class="site-table">
        <table>
          <tr>
            <th style="text-align: center; background-color: #A60D0D; color: white;"><b>Set Theory</b></th>
            <th style="text-align: center; background-color: #A60D0D; color: white;"><b>Category Theory</b></th>
          </tr>
          <tr>
            <td>Relation<p style="margin-top:0.5em;margin-bottom:0.5em">$R\colon X\times Y\to\TTV$</p></td>
            <td>Profunctor<p style="margin-top:0.5em;margin-bottom:0.5em">$\mathfrak{p}\colon\mathcal{D}^{\op}\times\mathcal{C}\to\Sets$</p></td>
          </tr>
          <tr>
            <td>Relation<p style="margin-top:0.5em;margin-bottom:0.5em">$R\colon X\to\mathcal{P}(Y)$</p></td>
            <td>Profunctor<p style="margin-top:0.5em;margin-bottom:0.5em">$\mathfrak{p}\colon\mathcal{C}\to\PSh(\mathcal{D})$</p></td>
          </tr>
          <tr>
            <td>Relation as a cocontinuous<br>morphism of posets<p style="margin-top:0.5em;margin-bottom:0.5em">$R\colon(\mathcal{P}(X),\subset)\to(\mathcal{P}(Y),\subset)$</p></td>
            <td>Profunctor as a<br>colimit-preserving functor<p style="margin-top:0.5em;margin-bottom:0.5em">$\mathfrak{p}\colon\mathsf{PSh}(\mathcal{C})\to\mathsf{PSh}(\mathcal{D})$</p></td>
          </tr>
        </table>
      </div>
    </div>
    % BEGIN LATEX HTML %
    \begingroup%
    \setlength\cellspacetoplimit{3pt}
    \setlength\cellspacebottomlimit{3pt}
    \renewcommand{\arraystretch}{1.2}
    \begin{center}
        \begin{tabular}{|Sc|Sc|}\hline\rowcolor{darkRed}
            \textcolor{white}{\textbf{\textsc{Set Theory}}}                                                                                                                                       & \textcolor{white}{\textbf{\textsc{Category Theory}}}                                                                                                               \\\hline\rowcolor{backgroundColor}
            Relation $R\colon X\times Y\to\TTV$                                                                                                                                                   & Profunctor $\mathfrak{p}\colon\CatFont{D}^{\op}\times\CatFont{C}\to\Sets$                                                                                 \\\rowcolor{black!05!backgroundColor}
            Relation $R\colon X\to\mathcal{P}(Y)$                                                                                                                                      & Profunctor $\mathfrak{p}\colon\CatFont{C}\to\PSh(\CatFont{D})$                                                                                 \\\rowcolor{backgroundColor}
            \Gape[0pt][2pt]{\makecell{Relation as a cocontinuous\\morphism of posets\\$R\colon(\mathcal{P}(X),\subset)\to(\mathcal{P}(Y),\subset)$}} & \Gape[0pt][2pt]{\makecell{Profunctor as a \\colimit-preserving functor\\$\mathfrak{p}\colon\PSh(\CatFont{C})\to\PSh(\CatFont{D})$}} \\\hline
        \end{tabular}
    \end{center}
    \endgroup
    % END RAW HTML %
\end{remark}
\begin{appendices}
\begin{multicols}{2}[\section{Other Chapters}]
\noindent
\textbf{Preliminaries}
\begin{enumerate}
\item \hyperref[introduction:section-phantom]{Introduction}
\end{enumerate}
\textbf{Sets}
\begin{enumerate}
\setcounter{enumi}{2}
\item \hyperref[sets:section-phantom]{Sets}
\item \hyperref[constructions-with-sets:section-phantom]{Constructions With Sets}
\item \hyperref[monoidal-structures-on-the-category-of-sets:section-phantom]{Monoidal Structures on the Category of Sets}
\item \hyperref[pointed-sets:section-phantom]{Pointed Sets}
\item \hyperref[tensor-products-of-pointed-sets:section-phantom]{Tensor Products of Pointed Sets}
\end{enumerate}
\textbf{Relations}
\begin{enumerate}
\setcounter{enumi}{6}
\item \hyperref[relations:section-phantom]{Relations}
\item \hyperref[constructions-with-relations:section-phantom]{Constructions With Relations}
\item \hyperref[conditions-on-relations:section-phantom]{Conditions on Relations}
\end{enumerate}
\textbf{Category Theory}
\begin{enumerate}
\setcounter{enumi}{9}
\item \hyperref[categories:section-phantom]{Categories}
\end{enumerate}
\textbf{Monoidal Categories}
\begin{enumerate}
\setcounter{enumi}{10}
\item \hyperref[constructions-with-monoidal-categories:section-phantom]{Constructions With Monoidal Categories}
\end{enumerate}
\textbf{Bicategories}
\begin{enumerate}
\setcounter{enumi}{11}
\item \hyperref[types-of-morphisms-in-bicategories:section-phantom]{Types of Morphisms in Bicategories}
\end{enumerate}
\textbf{Extra Part}
\begin{enumerate}
\setcounter{enumi}{12}
\item \hyperref[notes:section-phantom]{Notes}
\end{enumerate}
\end{multicols}

\end{appendices}
\end{document}
