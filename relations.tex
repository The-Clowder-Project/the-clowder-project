\input{preamble}

% OK, start here.
%
\usepackage{fontspec}
\let\hyperwhite\relax
\let\hyperred\relax
\newcommand{\hyperwhite}{\hypersetup{citecolor=white,filecolor=white,linkcolor=white,urlcolor=white}}
\newcommand{\hyperred}{%
\hypersetup{%
    citecolor=TitlingRed,%
    filecolor=TitlingRed,%
    linkcolor=TitlingRed,%
     urlcolor=TitlingRed%
}}
\let\ChapterRef\relax
\newcommand{\ChapterRef}[2]{#1}
\setcounter{tocdepth}{2}
%▓▓▓▓▓▓▓▓▓▓▓▓▓▓▓▓▓▓▓▓▓▓▓▓▓▓▓▓▓▓▓▓▓
%▓▓ ╔╦╗╦╔╦╗╦  ╔═╗  ╔═╗╔═╗╔╗╔╔╦╗ ▓▓
%▓▓  ║ ║ ║ ║  ║╣   ╠╣ ║ ║║║║ ║  ▓▓
%▓▓  ╩ ╩ ╩ ╩═╝╚═╝  ╚  ╚═╝╝╚╝ ╩  ▓▓
%▓▓▓▓▓▓▓▓▓▓▓▓▓▓▓▓▓▓▓▓▓▓▓▓▓▓▓▓▓▓▓▓▓
%\usepackage{titlesec}
%▓▓▓▓▓▓▓▓▓▓▓▓▓▓▓▓▓▓▓▓▓▓▓▓▓▓▓▓▓▓▓▓▓▓▓▓▓▓▓▓▓▓▓▓▓▓▓▓▓▓▓▓▓▓▓
%▓▓ ╔╦╗╔═╗╔╗ ╦  ╔═╗  ╔═╗╔═╗  ╔═╗╔═╗╔╗╔╔╦╗╔═╗╔╗╔╔╦╗╔═╗ ▓▓
%▓▓  ║ ╠═╣╠╩╗║  ║╣   ║ ║╠╣   ║  ║ ║║║║ ║ ║╣ ║║║ ║ ╚═╗ ▓▓
%▓▓  ╩ ╩ ╩╚═╝╩═╝╚═╝  ╚═╝╚    ╚═╝╚═╝╝╚╝ ╩ ╚═╝╝╚╝ ╩ ╚═╝ ▓▓
%▓▓▓▓▓▓▓▓▓▓▓▓▓▓▓▓▓▓▓▓▓▓▓▓▓▓▓▓▓▓▓▓▓▓▓▓▓▓▓▓▓▓▓▓▓▓▓▓▓▓▓▓▓▓▓
\newcommand{\ChapterTableOfContents}{%
    \begingroup
    \addfontfeature{Numbers={Lining,Monospaced}}
    \hypersetup{hidelinks}\tableofcontents%
    \endgroup
}%

\makeatletter
\newcommand \DotFill {\leavevmode \cleaders \hb@xt@ .33em{\hss .\hss }\hfill \kern \z@}
\makeatother

\definecolor{ToCGrey}{rgb}{0.4,0.4,0.4}
\definecolor{mainColor}{rgb}{0.82745098,0.18431373,0.18431373}
\usepackage{titletoc}
\titlecontents{part}
[0.0em]
{\addvspace{1pc}\color{TitlingRed}\large\bfseries\text{Part }}
{\bfseries\textcolor{TitlingRed}{\contentslabel{0.0em}}\hspace*{1.35em}}
{}
{\textcolor{TitlingRed}{{\hfill\bfseries\contentspage\nobreak}}}
[]
\titlecontents{section}
[0.0em]
{\addvspace{1pc}}
{\color{black}\bfseries\textcolor{TitlingRed}{\contentslabel{0.0em}}\hspace*{1.35em}}
{}
{\textcolor{black}{\textbf{\DotFill}{\bfseries\contentspage\nobreak}}}
[]
\titlecontents{subsection}
[0.0em]
{}
{\hspace*{1.35em}\color{ToCGrey}{\contentslabel{0.0em}}\hspace*{2.1em}}
{}
{{\textcolor{ToCGrey}\DotFill}\textcolor{ToCGrey}{\contentspage}\nobreak}
[]
\usepackage{marginnote}
\renewcommand*{\marginfont}{\normalfont}
\usepackage{inconsolata}
\setmonofont{inconsolata}%
\let\ChapterRef\relax
\newcommand{\ChapterRef}[2]{#1}
\AtBeginEnvironment{subappendices}{%%
    \section*{\huge Appendices}%
}%

\begin{document}

\title{Relations}

\maketitle

\phantomsection
\label{section-phantom}

This chapter contains some material about relations. Notably, we discuss and explore:
\begin{enumerate}
    \item\label{relations-introduction-item-1}The definition of relations (\cref{subsection-relations-foundations}).
    \item\label{relations-introduction-item-2}How relations may be viewed as decategorification of profunctors (\cref{subsection-relations-as-decategorifications-of-profunctors}).
    \item\label{relations-introduction-item-3}The various kind of categories that relations form, namely:
        \begin{enumerate}
            \item\label{relations-introduction-item-3a}A category (\cref{subsection-the-category-of-relations}).
            \item\label{relations-introduction-item-3b}A monoidal category (\cref{subsection-the-closed-symmetric-monoidal-category-of-relations}).
            \item\label{relations-introduction-item-3c}A $2$-category (\cref{subsection-the-2-category-of-relations}).
            \item\label{relations-introduction-item-3d}A double category (\cref{subsection-the-double-category-of-relations}).
        \end{enumerate}
    \item\label{relations-introduction-item-4}The various categorical properties of the $2$-category of relations, including:
        \begin{enumerate}
            \item\label{relations-introduction-item-4a}The self-duality of $\sfRel$ and $\sfbfRel$ (\cref{self-duality-for-the-2-category-of-relations}).
            \item\label{relations-introduction-item-4b}Identifications of equivalences and isomorphisms in $\sfbfRel$ with bijections (\cref{isomorphisms-and-equivalences-in-rel}).
            \item\label{relations-introduction-item-4c}Identifications of adjunctions in $\sfbfRel$ with functions (\cref{adjunctions-in-rel}).
            \item\label{relations-introduction-item-4d}Identifications of monads in $\sfbfRel$ with preorders (\cref{monads-in-rel}).
            \item\label{relations-introduction-item-4e}Identifications of comonads in $\sfbfRel$ with subsets (\cref{comonads-in-rel}).
            \item\label{relations-introduction-item-4f}A description of the monoids and comonoids in $\sfbfRel$ with respect to the Cartesian product (\cref{co-monoids-in-rel}).
            \item\label{relations-introduction-item-4g}Characterisations of monomorphisms in $\sfRel$ (\cref{characterisations-of-monomorphisms-in-rel}).
            \item\label{relations-introduction-item-4h}Characterisations of $2$-categorical notions of monomorphisms in $\sfbfRel$ (\cref{2-categorical-monomorphisms-in-rel}).
            \item\label{relations-introduction-item-4i}Characterisations of epimorphisms in $\sfRel$ (\cref{characterisations-of-epimorphisms-in-rel}).
            \item\label{relations-introduction-item-4j}Characterisations of $2$-categorical notions of epimorphisms in $\sfbfRel$ (\cref{2-categorical-epimorphisms-in-rel}).
            \item\label{relations-introduction-item-4k}The partial co/completeness of $\sfRel$ (\cref{co-limits-in-rel}).
            \item\label{relations-introduction-item-4l}The existence or non-existence of Kan extensions and Kan lifts in $\sfRel$ (\cref{kan-extensions-and-kan-lifts-in-rel}).
            \item\label{relations-introduction-item-4m}The closedness of $\sfbfRel$ (\cref{closedness-of-rel}).
            \item\label{relations-introduction-item-4n}The identification of $\sfbfRel$ with the category of free algebras of the powerset monad on $\Sets$ (\cref{rel-as-a-category-of-free-algebras}).
        \end{enumerate}
    \item\label{relations-introduction-item-5}A description of two notions of \say{skew composition} on $\eRel(A,B)$, giving rise to left and right skew monoidal structures analogous to the left skew monoidal structure on $\Fun(\CatFont{C},\CatFont{D})$ appearing in the definition of a relative monad (\cref{section-the-left-skew-monoidal-structure-on-rel-a-b,section-the-right-skew-monoidal-structure-on-rel-a-b}).
\end{enumerate}

\ChapterTableOfContents

\section{Relations}\label{section-relations}
\subsection{Foundations}\label{subsection-relations-foundations}
Let $A$ and $B$ be sets.
\begin{definition}{Relations}{relations}%
    A \index[set-theory]{relation}\textbf{relation $R\colon A\rightproarrow B$ from $A$ to $B$}%
    %--- Begin Footnote ---%
    \footnote{%
        \SloganFont{Further Terminology: }Also called a \index[set-theory]{multivalued function!see {relation}}\textbf{multivalued function from $A$ to $B$}.
    }%
    %---  End Footnote  ---%
    %--- Begin Footnote ---%
    \footnote{%
        \SloganFont{Further Terminology: }When $A=B$, we also call $R\subset A\times A$ a \textbf{relation on $A$}.
        \par\vspace*{\TCBBoxCorrection}
    } %
    %---  End Footnote  ---%
    is equivalently:
    \begin{enumerate}
        \item\label{relations-1}A subset $R$ of $A\times B$.%
        \item\label{relations-2}A function from $A\times B$                                       to $\TV$.
        \item\label{relations-3}A function from $A$                                               to $\mathcal{P}(B)$.
        \item\label{relations-4}A function from $B$                                               to $\mathcal{P}(A)$.
        \item\label{relations-5}A cocontinuous morphism of posets from $(\mathcal{P}(A),\subset)$ to $(\mathcal{P}(B),\subset)$.
        \item\label{relations-6}A continuous morphism of posets from $(\mathcal{P}(B),\supset)$ to $(\mathcal{P}(A),\supset)$.
    \end{enumerate}
\end{definition}
\begin{Proof}{Proof of the Equivalences in \cref{relations}}%
    (We will prove that \cref{relations-1,relations-2,relations-3,relations-4,relations-5,relations-6} are indeed equivalent in a bit.)
\end{Proof}
\begin{remark}{Unwinding \cref{relations-1}, \rmI}{unwinding-relations-1}%
    We may think of a relation $R\colon A\rightproarrow B$ as a function from $A$ to $B$ that is \emph{multivalued}, assigning to each element $a$ in $A$ a set $R(a)$ of elements of $B$, thought of as the \emph{set of values of $R$ at $a$}.

    \indent Note that this includes also the possibility of $R$ having no value at all on a given $a\in A$ when $R(a)=\emptyset$.
\end{remark}
\begin{remark}{Unwinding \cref{relations-2}, \rmII}{unwinding-relations-2}%
    Another way of stating the equivalence between \cref{relations-1,relations-2,relations-3,relations-4,relations-5} of \cref{relations} is by saying that we have bijections of sets
    \begin{align*}
        \{\text{relations from $A$ to $B$}\} &\cong \mathcal{P}(A\times B)\\
                                             &\cong \Sets(A\times B,\TV)\\
                                             &\cong \Sets(A,\mathcal{P}(B))\\
                                             &\cong \Sets(B,\mathcal{P}(A))\\
                                             &\cong \CoContPos(\mathcal{P}(A),\mathcal{P}(B))\\
                                             &\cong \ContPos(\mathcal{P}(B),\mathcal{P}(A))
    \end{align*}
    natural in $A,B\in\Obj(\Sets)$, where $\mathcal{P}(A)$ and $\mathcal{P}(B)$ are endowed with the poset structure given by inclusion.
\end{remark}
\begin{Proof}{Proof of the Equivalences in \cref{relations}}%
    We claim that \cref{relations-1,relations-2,relations-3,relations-4,relations-5} are indeed equivalent:
    \begin{itemize}
        \item\SloganFont{\cref{relations-1}$\iff$\cref{relations-2}: }This is a special case of \ChapterRef{\ChapterConstructionsWithSets, \cref{constructions-with-sets:properties-of-characteristic-functions-of-subsets-bijectivity,constructions-with-sets:properties-of-characteristic-functions-of-subsets-naturality} of \cref{constructions-with-sets:properties-of-characteristic-functions-of-subsets}}{\cref{properties-of-characteristic-functions-of-subsets-bijectivity,properties-of-characteristic-functions-of-subsets-naturality} of \cref{properties-of-characteristic-functions-of-subsets}}.
        \item\SloganFont{\cref{relations-2}$\iff$\cref{relations-3}: }This follows from the bijections
            \begin{align*}
                \Sets(A\times B,\TV) &\cong \Sets(A,\Sets(B,\TV))\\
                                     &\cong \Sets(A,\mathcal{P}(B)),
            \end{align*}
            where the last bijection is from \ChapterRef{\ChapterConstructionsWithSets, \cref{constructions-with-sets:properties-of-characteristic-functions-of-subsets-bijectivity,constructions-with-sets:properties-of-characteristic-functions-of-subsets-naturality} of \cref{constructions-with-sets:properties-of-characteristic-functions-of-subsets}}{\cref{properties-of-characteristic-functions-of-subsets-bijectivity,properties-of-characteristic-functions-of-subsets-naturality} of \cref{properties-of-characteristic-functions-of-subsets}}.
        \item\SloganFont{\cref{relations-2}$\iff$\cref{relations-4}: }This follows from the bijections
            \begin{align*}
                \Sets(A\times B,\TV) &\cong \Sets(B,\Sets(B,\TV))\\
                                     &\cong \Sets(B,\mathcal{P}(A)),
            \end{align*}
            where again the last bijection is from \ChapterRef{\ChapterConstructionsWithSets, \cref{constructions-with-sets:properties-of-characteristic-functions-of-subsets-bijectivity,constructions-with-sets:properties-of-characteristic-functions-of-subsets-naturality} of \cref{constructions-with-sets:properties-of-characteristic-functions-of-subsets}}{\cref{properties-of-characteristic-functions-of-subsets-bijectivity,properties-of-characteristic-functions-of-subsets-naturality} of \cref{properties-of-characteristic-functions-of-subsets}}.
        \item\SloganFont{\cref{relations-2}$\iff$\cref{relations-5}: }This follows from the universal property of the powerset $\mathcal{P}(X)$ of a set $X$ as the free cocompletion of $X$ via the characteristic embedding
            \[
                \chi_{X}
                \colon
                X
                \hookrightarrow
                \mathcal{P}(X)
            \]%
            of $X$ into $\mathcal{P}(X)$, as in \ChapterRef{\ChapterConstructionsWithSets, \cref{constructions-with-sets:powersets-as-free-cocompletions-universal-property}}{\cref{powersets-as-free-cocompletions-universal-property}}. In particular, the bijection
            \[
                \Sets(A,\mathcal{P}(B))%
                \cong
                \CoContPos(\mathcal{P}(A),\mathcal{P}(B))
            \]%
            is given by extending each $f\colon A\to\mathcal{P}(B)$ in $\Sets(A,\mathcal{P}(B))$ from $A$ to all of $\mathcal{P}(A)$ by taking its left Kan extension along $\chi_{X}$, recovering the direct image function $f_{*}\colon\mathcal{P}(A)\to\mathcal{P}(B)$ of $f$ of \ChapterRef{\ChapterConstructionsWithSets, \cref{constructions-with-sets:the-direct-image-function-associated-to-a-function}}{\cref{the-direct-image-function-associated-to-a-function}}.
        \item\SloganFont{\cref{relations-5}$\iff$\cref{relations-6}: }Omitted.
    \end{itemize}
    This finishes the proof.
\end{Proof}
\begin{notation}{Further Notation for Relations}{further-notation-for-relations}%
    Let $A$ and $B$ be sets and let $R\colon\rightproarrow B$ be a relation from $A$ to $B$.
    \begin{enumerate}
        \item\label{further-notation-for-relations-the-set-of-relations-between-two-sets}We write \index[notation]{RelAB@$\Rel(A,B)$}$\Rel(A,B)$ for the set of relations from $A$ to $B$.
        \item\label{further-notation-for-relations-the-poset-of-relations-between-two-sets}We write \index[notation]{RelAB@$\eRel(A,B)$}$\eRel(A,B)$ for the sub-poset of $(\mathcal{P}(A\times B),\subset)$ spanned by the relations from $A$ to $B$.
        \item\label{further-notation-for-relations-a-simr-b}Given $a\in A$ and $b\in B$, we write $a\sim_{R}b$ to mean $(a,b)\in R$.
        \item\label{further-notation-for-relations-r-b-a}When viewing $R$ as a function
            \[
                R%
                \colon%
                A\times B%
                \to%
                \TTV,%
            \]%
            we write $R^{b}_{a}$ for the value of $R$ at $(a,b)$.%
            %--- Begin Footnote ---%
            \footnote{%
                The choice to write $R^{b}_{a}$ in place of $R^{a}_{b}$ is to keep the notation consistent with the notation we will later employ for profunctors in \ChapterProfunctors.
                \par\vspace*{\TCBBoxCorrection}
            }%
            %---  End Footnote  ---%
    \end{enumerate}
\end{notation}
\begin{proposition}{Properties of Relations}{properties-of-relations}%
    Let $A$ and $B$ be sets and let $R,S\colon A\rightproarrow B$ be relations.
    \begin{enumerate}
        \item\label{properties-of-relations-end-formula-for-the-set-of-inclusions-of-relations}\SloganFont{End Formula for the Set of Inclusions of Relations. }We have
            \[
                \Hom_{\eRel(A,B)}(R,S)%
                \cong%
                \int_{a\in A}\int_{b\in B}\Hom_{\TTV}(R^{b}_{a},S^{b}_{a}).%
            \]%
        %\item\label{properties-of-relations-}\SloganFont{. }
    \end{enumerate}
\end{proposition}
\begin{Proof}{Proof of \cref{properties-of-relations}}%
    \FirstProofBox{\cref{properties-of-relations-end-formula-for-the-set-of-inclusions-of-relations}: End Formula for the Set of Inclusions of Relations}%
    Unwinding the expression inside the end on the right hand side, we have
    \[
        \int_{a\in A}\int_{b\in B}\Hom_{\TTV}(R^{b}_{a},S^{b}_{a})%
        \cong%
        \begin{cases}%
            \pt       &\text{if, for each $a\in A$ and each $b\in B$,}\\%
                      &\text{we have $\Hom_{\TTV}(R^{b}_{a},S^{b}_{a})\cong\pt$}\\%
            \emptyset &\text{otherwise.}%
        \end{cases}%
    \]%
    Since we have $\Hom_{\TTV}(R^{b}_{a},S^{b}_{a})=\{\true\}\cong\pt$ exactly when $R^{b}_{a}=\false$ or $R^{b}_{a}=S^{b}_{a}=\true$, we get
    \[
        \int_{a\in A}\int_{b\in B}\Hom_{\TTV}(R^{b}_{a},S^{b}_{a})%
        \cong%
        \begin{cases}%
            \pt       &\text{if, for each $a\in A$ and each $b\in B$,}\\%
                      &\text{if $a\sim_{R}b$, then $a\sim_{S}b$,}\\%
            \emptyset &\text{otherwise.}%
        \end{cases}%
    \]%
    On the left hand-side, we have
    \[
        \Hom_{\eRel(A,B)}(R,S)%
        \cong%
        \begin{cases}%
            \pt       &\text{if $R\subset S$,}\\%
            \emptyset &\text{otherwise.}%
        \end{cases}%
    \]%
    Since $(a\sim_{R}b\implies a\sim_{S}b)$ \textiff $R\subset S$, the two sets above are isomorphic. This finishes the proof.
\end{Proof}
\subsection{Relations as Decategorifications of Profunctors}\label{subsection-relations-as-decategorifications-of-profunctors}
\begin{remark}{Relations as Decategorifications of Profunctors \rmI}{relations-as-decategorifications-of-profunctors-1}%
    The notion of a relation is a decategorification of that of a profunctor:
    \begin{enumerate}
        \item\label{relations-as-decategorifications-of-profunctors-1-item-1}A profunctor from a category $\CatFont{C}$ to a category $\CatFont{D}$ is a functor
            \[%
                \mathfrak{p}%
                \colon%
                \CatFont{D}^{\op}\times\CatFont{C}%
                \to%
                \Sets.%
            \]%
        \item\label{relations-as-decategorifications-of-profunctors-1-item-2}A relation on sets $A$ and $B$ is a function%
            \[%
                R%
                \colon%
                A\times B%
                \to%
                \TV.%
            \]%
    \end{enumerate}
    Here we notice that:
    \begin{itemize}
        \item The opposite $X^{\op}$ of a set $X$ is itself, as $(-)^{\op}\colon\Cats\to\Cats$ restricts to the identity endofunctor on $\Sets$.
        \item The values that profunctors and relations take are analogous:
            \begin{itemize}
                \item A category is enriched over the category
                    \[
                        \Sets%
                        \defeq%
                        \ZeroCats%
                    \]%
                    of sets, with profunctors taking values on it.
                \item A set is enriched over the set
                    \[
                        \TV%
                        \defeq%
                        \MinusOneCats%
                    \]%
                    of classical truth values, with relations taking values on it.
            \end{itemize}
    \end{itemize}
\end{remark}
\begin{remark}{Relations as Decategorifications of Profunctors \rmII}{relations-as-decategorifications-of-profunctors-2}%
    Extending \cref{relations-as-decategorifications-of-profunctors-1}, the equivalent definitions of relations in \cref{relations} are also related to the corresponding ones for profunctors (\cref{TODO}), which state that a profunctor $\mathfrak{p}\colon\CatFont{C}\rightproarrow\CatFont{D}$ is equivalently:
    \begin{enumerate}
        \item\label{profunctors-equivalent-characterisations-item-1}A                    functor $\mathfrak{p}\colon\CatFont{D}^{\op}\times\CatFont{C}\to\Sets$.
        \item\label{profunctors-equivalent-characterisations-item-2}A                    functor $\mathfrak{p}\colon\CatFont{C}\to\PSh(\CatFont{D})$.
        \item\label{profunctors-equivalent-characterisations-item-3}A                    functor $\mathfrak{p}\colon\CatFont{D}^{\op}\to\CoPSh(\CatFont{C})$.
        \item\label{profunctors-equivalent-characterisations-item-4}A colimit-preserving functor $\mathfrak{p}\colon\PSh(\CatFont{C})\to\PSh(\CatFont{D})$.
        \item\label{profunctors-equivalent-characterisations-item-5}A   limit-preserving functor $\mathfrak{p}\colon\CoPSh(\CatFont{D})^{\op}\to\CoPSh(\CatFont{C})^{\op}$.
    \end{enumerate}
    Indeed:
    \begin{itemize}
        \item The equivalence between \cref{profunctors-equivalent-characterisations-item-1,profunctors-equivalent-characterisations-item-2} (and also that between \cref{profunctors-equivalent-characterisations-item-1,profunctors-equivalent-characterisations-item-3}, which is proved analogously) is an instance of currying, both for profunctors as well as for relations, using the isomorphisms
            \begin{align*}
                \Sets(A\times B,\TV) &\cong \Sets(A,\Sets(B,\TV))\\
                                     &\cong \Sets(A,\mathcal{P}(B)),
            \end{align*}
            and
            \begin{align*}
                \Fun(\CatFont{D}^{\op}\times\CatFont{D},\Sets) &\cong \Fun(\CatFont{C},\Fun(\CatFont{D}^{\op},\Sets))\\
                                                               &\cong \Fun(\CatFont{C},\PSh(\CatFont{D})).
            \end{align*}
        \item The equivalence between \cref{profunctors-equivalent-characterisations-item-2,profunctors-equivalent-characterisations-item-4} follows from the universal properties of:
            \begin{itemize}
                \item The powerset $\mathcal{P}(X)$ of a set $X$ as the free cocompletion of $X$ via the characteristic embedding
                    \[
                        \chi_{(-)}
                        \colon
                        X
                        \hookrightarrow
                        \mathcal{P}(X)
                    \]%
                    of $X$ into $\mathcal{P}(X)$, as stated and proved in \ChapterRef{\ChapterConstructionsWithSets, \cref{constructions-with-sets:powersets-as-free-cocompletions-universal-property}}{\cref{powersets-as-free-cocompletions-universal-property}}.
                \item The category $\PSh(\CatFont{C})$ of presheaves on a category $\CatFont{C}$ as the free cocompletion of $\CatFont{C}$ via the Yoneda embedding
                    \[
                        \yo
                        \colon
                        \CatFont{C}
                        \hookrightarrow
                        \PSh(\CatFont{C})
                    \]%
                    of $\CatFont{C}$ into $\PSh(\CatFont{C})$, as stated and proved in \ChapterRef{\ChapterPresheavesAndTheYonedaLemma, \cref{presheaves-and-the-yoneda-lemma:properties-of-the-yoneda-embedding-as-the-free-cocompletion} of \cref{presheaves-and-the-yoneda-lemma:properties-of-the-yoneda-embedding}}{\cref{properties-of-the-yoneda-embedding-as-the-free-cocompletion} of \cref{properties-of-the-yoneda-embedding}}.
            \end{itemize}
        \item The equivalence between \cref{profunctors-equivalent-characterisations-item-3,profunctors-equivalent-characterisations-item-5} follows from the universal properties of:
            \begin{itemize}
                \item The powerset $\mathcal{P}(X)$ of a set $X$ as the free completion of $X$ via the characteristic embedding
                    \[
                        \chi_{(-)}
                        \colon
                        X
                        \hookrightarrow
                        \mathcal{P}(X)
                    \]%
                    of $X$ into $\mathcal{P}(X)$, as stated and proved in \ChapterRef{\ChapterConstructionsWithSets, \cref{constructions-with-sets:powersets-as-free-completions-universal-property}}{\cref{powersets-as-free-completions-universal-property}}.
                \item The category $\CoPSh(\CatFont{D})^{\op}$ of copresheaves on a category $\CatFont{D}$ as the free completion of $\CatFont{D}$ via the dual Yoneda embedding
                    \[
                        \coyo
                        \colon
                        \CatFont{D}
                        \hookrightarrow
                        \CoPSh(\CatFont{D})^{\op}
                    \]%
                    of $\CatFont{D}$ into $\CoPSh(\CatFont{D})^{\op}$, as stated and proved in \ChapterRef{\ChapterPresheavesAndTheYonedaLemma, \cref{presheaves-and-the-yoneda-lemma:properties-of-the-yoneda-embedding-as-the-free-completion} of \cref{presheaves-and-the-yoneda-lemma:properties-of-the-yoneda-embedding}}{\cref{properties-of-the-yoneda-embedding-as-the-free-completion} of \cref{properties-of-the-yoneda-embedding}}.
            \end{itemize}
    \end{itemize}
\end{remark}
\subsection{Examples of Relations}\label{subsection-examples-of-relations}
\begin{example}{The Trivial Relation}{the-trivial-relation}%
    The \index[set-theory]{relation!trivial}\textbf{trivial relation on $A$ and $B$} is the relation \index[notation]{simtriv@$\unsim_{\triv}$}$\unsim_{\triv}$ defined equivalently as follows:
    \begin{enumerate}
        \item\label{the-trivial-relation-1}As a subset of $A\times B$, we have
            \[
                \unsim_{\triv}%
                \defeq%
                A\times B.%
            \]%
        \item\label{the-trivial-relation-2}As a function from $A\times B$ to $\TV$, the relation $\unsim_{\triv}$ is the constant function
            \[%
                \Delta_{\true}%
                \colon%
                A\times B%
                \to%
                \TV%
            \]%
            from $A\times B$ to $\TV$ taking the value $\true$.
        \item\label{the-trivial-relation-3}As a function from $A$ to $\mathcal{P}(B)$, the relation $\unsim_{\triv}$ is the function
            \[%
                \Delta_{\true}%
                \colon%
                A%
                \to%
                \mathcal{P}(B)%
            \]%
            defined by
            \[
                \Delta_{\true}(a)%
                \defeq%
                B%
            \]%
            for each $a\in A$.
        \item\label{the-trivial-relation-4}Lastly, it is the unique relation $R$ on $A$ and $B$ such that we have $a\sim_{R}b$ for each $a\in A$ and each $b\in B$.
    \end{enumerate}
\end{example}
\begin{example}{The Cotrivial Relation}{the-cotrivial-relation}%
    The \index[set-theory]{relation!cotrivial}\textbf{cotrivial relation on $A$ and $B$} is the relation \index[notation]{simcotriv@$\unsim_{\cotriv}$}$\unsim_{\cotriv}$ defined equivalently as follows:%
    \begin{enumerate}
        \item\label{the-cotrivial-relation-1}As a subset of $A\times B$, we have
            \[
                \unsim_{\cotriv}%
                \defeq%
                \emptyset.%
            \]%
        \item\label{the-cotrivial-relation-2}As a function from $A\times B$ to $\TV$, the relation $\unsim_{\cotriv}$ is the constant function
            \[%
                \Delta_{\false}%
                \colon%
                A\times B%
                \to%
                \TV%
            \]%
            from $A\times B$ to $\TV$ taking the value $\false$.
        \item\label{the-cotrivial-relation-3}As a function from $A$ to $\mathcal{P}(B)$, the relation $\unsim_{\cotriv}$ is the function
            \[%
                \Delta_{\false}%
                \colon%
                A%
                \to%
                \mathcal{P}(B)%
            \]%
            defined by
            \[
                \Delta_{\false}(a)%
                \defeq%
                \emptyset%
            \]%
            for each $a\in A$.
        \item\label{the-cotrivial-relation-4}Lastly, it is the unique relation $R$ on $A$ and $B$ such that we have $a\nsim_{R}b$ for each $a\in A$ and each $b\in B$.
    \end{enumerate}
\end{example}
\begin{example}{The Characteristic Relation of a Set}{the-characteristic-relation}%
    The characteristic relation
    \[%
        \chi_{X}(-_{1},-_{2})%
        \colon%
        X\times X%
        \to%
        \TTV%
    \]%
    on $X$ of \ChapterRef{\ChapterConstructionsWithSets, \cref{constructions-with-sets:the-characteristic-relation-of-a-set}}{\cref{the-characteristic-relation-of-a-set}}, defined by%
    \[
        \chi_{X}(x,y)
        \defeq
        \begin{cases}
            \true  &\text{if $x=y$,}\\
            \false &\text{if $x\neq y$}
        \end{cases}
    \]%
    for each $x,y\in X$, is another example of a relation.
\end{example}
\begin{example}{Square Roots}{square-roots}%
    Square roots are examples of relations:
    \begin{enumerate}
        \item\label{square-roots-1}\SloganFont{Square Roots in $\R$. }The assignment $x\mapsto\sqrt{x}$ defines a relation
            \[
                \sqrt{-}%
                \colon%
                \R%
                \to%
                \mathcal{P}(\R)%
            \]%
            from $\R$ to itself, being explicitly given by
            \[
                \sqrt{x}%
                \defeq%
                \begin{cases}
                    0                                  &\text{if $x=0$,}\\
                    \{-\sqrt{\abs{x}},\sqrt{\abs{x}}\} &\text{if $x\neq0$.}
                \end{cases}
            \]%
        \item\label{square-roots-2}\SloganFont{Square Roots in $\Q$. }Square roots in $\Q$ are similar to square roots in $\R$, though now additionally it may also occur that $\sqrt{-}\colon\Q\to\mathcal{P}(\Q)$ sends a rational number $x$ (e.g.\ $2$) to the empty set (since $\sqrt{2}\nin\Q$).
    \end{enumerate}
\end{example}
\begin{example}{Complex Logarithms}{complex-logarithms}%
    The complex logarithm defines a relation
    \[
        \log%
        \colon%
        \C%
        \to%
        \mathcal{P}(\C)%
    \]%
    from $\C$ to itself, where we have
    \[
        \log(a+bi)%
        \defeq%
        \{%
            \log(\sqrt{a^{2}+b^{2}})+i\arg(a+bi)+(2\pi i)k%
            \ \middle|\ %
            k\in\Z%
        \}%
    \]%
    for each $a+bi\in\C$.
\end{example}
\begin{example}{More Examples of Relations}{more-examples-of-relations}%
    See \cite{wikipedia:multivalued-function} for more examples of relations, such as antiderivation, inverse trigonometric functions, and inverse hyperbolic functions.
\end{example}
\subsection{Functional Relations}\label{subsection-functional-relations}
Let $A$ and $B$ be sets.
\begin{definition}{Functional Relations}{functional-relations}%
    A relation $R\colon A\rightproarrow B$ is \index[set-theory]{relation!functional}\textbf{functional} if, for each $a\in A$, the set $R(a)$ is either empty or a singleton.
\end{definition}
\begin{proposition}{Properties of Functional Relations}{properties-of-functional-relations}%
    Let $R\colon A\rightproarrow B$ be a relation.
    \begin{enumerate}
        \item\label{properties-of-functional-relations-characterisations}\SloganFont{Characterisations. }The following conditions are equivalent:
            \begin{enumerate}
                \item\label{properties-of-functional-relations-characterisations-1}The relation $R$ is functional.
                \item\label{properties-of-functional-relations-characterisations-2}We have $R\procirc R^{\dagger}\subset\chi_{B}$.
            \end{enumerate}
        %\item\label{properties-of-functional-relations-}\SloganFont{. }
    \end{enumerate}
\end{proposition}
\begin{Proof}{Proof of \cref{properties-of-functional-relations}}%
    \FirstProofBox{\cref{properties-of-functional-relations-characterisations}: Characterisations}%
    We claim that \cref{properties-of-functional-relations-characterisations-1,properties-of-functional-relations-characterisations-2} are indeed equivalent:
    \begin{itemize}
        \item\SloganFont{\cref{properties-of-functional-relations-characterisations-1}$\implies$\cref{properties-of-functional-relations-characterisations-2}: }Let $(b,b')\in B\times B$. We need to show that
            \[
                [R\procirc R^{\dagger}](b,b')%
                \preceq_{\TTV}%
                \chi_{B}(b,b'),%
            \]%
            i.e.\ that if there exists some $a\in A$ such that $b\sim_{R^{\dagger}}a$ and $a\sim_{R}b'$, then $b=b'$. But since $b\sim_{R^{\dagger}}a$ is the same as $a\sim_{R}b$, we have both $a\sim_{R}b$ and $a\sim_{R}b'$ at the same time, which implies $b=b'$ since $R$ is functional.
        \item\SloganFont{\cref{properties-of-functional-relations-characterisations-2}$\implies$\cref{properties-of-functional-relations-characterisations-1}: }Suppose that we have $a\sim_{R}b$ and $a\sim_{R}b'$ for $b,b'\in B$. We claim that $b=b'$:
            \begin{itemize}
                \item Since $a\sim_{R}b$, we have $b\sim_{R^{\dagger}}a$.
                \item Since $R\procirc R^{\dagger}\subset\chi_{B}$, we have
                    \[
                        [R\procirc R^{\dagger}](b,b')%
                        \preceq_{\TTV}%
                        \chi_{B}(b,b'),%
                    \]%
                    and since $b\sim_{R^{\dagger}}a$ and $a\sim_{R}b'$, it follows that $[R\procirc R^{\dagger}](b,b')=\true$, and thus $\chi_{B}(b,b')=\true$ as well, i.e.\ $b=b'$.
            \end{itemize}
    \end{itemize}
    This finishes the proof.
\end{Proof}
\subsection{Total Relations}\label{subsection-total-relations}
Let $A$ and $B$ be sets.
\begin{definition}{Total Relations}{total-relations}%
    A relation $R\colon A\rightproarrow B$ is \index[set-theory]{relation!total}\textbf{total} if, for each $a\in A$, we have $R(a)\neq\emptyset$.
\end{definition}
\begin{proposition}{Properties of Total Relations}{properties-of-total-relations}%
    Let $R\colon A\rightproarrow B$ be a relation.
    \begin{enumerate}
        \item\label{properties-of-total-relations-characterisations}\SloganFont{Characterisations. }The following conditions are equivalent:
            \begin{enumerate}
                \item\label{properties-of-total-relations-characterisations-1}The relation $R$ is total.
                \item\label{properties-of-total-relations-characterisations-2}We have $\chi_{A}\subset R^{\dagger}\procirc R$.
            \end{enumerate}
        %\item\label{properties-of-total-relations-}\SloganFont{. }
    \end{enumerate}
\end{proposition}
\begin{Proof}{Proof of \cref{properties-of-total-relations}}%
    \FirstProofBox{\cref{properties-of-total-relations-characterisations}: Characterisations}%
    We claim that \cref{properties-of-total-relations-characterisations-1,properties-of-total-relations-characterisations-2} are indeed equivalent:
    \begin{itemize}
        \item\SloganFont{\cref{properties-of-total-relations-characterisations-1}$\implies$\cref{properties-of-total-relations-characterisations-2}: }We have to show that, for each $(a,a')\in A$, we have
            \[
                \chi_{A}(a,a')%
                \preceq_{\TTV}%
                [R^{\dagger}\procirc R](a,a'),%
            \]%
            i.e.\ that if $a=a'$, then there exists some $b\in B$ such that $a\sim_{R}b$ and $b\sim_{R^{\dagger}}a'$ (i.e.\ $a\sim_{R}b$ again), which follows from the totality of $R$.
        \item\SloganFont{\cref{properties-of-total-relations-characterisations-2}$\implies$\cref{properties-of-total-relations-characterisations-1}: }Given $a\in A$, since $\chi_{A}\subset R^{\dagger}\procirc R$, we must have
            \[
                \{a\}%
                \subset%
                [R^{\dagger}\procirc R](a),%
            \]%
            implying that there must exist some $b\in B$ such that $a\sim_{R}b$ and $b\sim_{R^{\dagger}}a$ (i.e.\ $a\sim_{R}b$) and thus $R(a)\neq\emptyset$, as $b\in R(a)$.
    \end{itemize}
    This finishes the proof.
\end{Proof}
\section{Categories of Relations}\label{subsection-categories-of-relations}
\subsection{The Category of Relations Between Two Sets}\label{subsection-the-category-of-relations-between-two-sets}
\begin{definition}{The Category of Relations Between Two Sets}{the-category-of-relations-between-two-sets}%
    The \index[set-theory]{relation!category of}\textbf{category of relations from $A$ to $B$} is the category \index[notation]{RelAB@$\eRel(A,B)$}$\eRel(A,B)$ defined by%
    %--- Begin Footnote ---%
    \footnote{%
        Here we choose to abuse notation by writing $\eRel(A,B)$ instead of $\eRel(A,B)_{\pos}$ for the posetal category of relations from $A$ to $B$, even though the same notation is used for the poset of relations from $A$ to $B$.
        \par\vspace*{\TCBBoxCorrection}
    } %
    %---  End Footnote  ---%
    \[
        \eRel(A,B)%
        \defeq
        \eRel(A,B)_{\pos},%
    \]%
    where $\eRel(A,B)_{\pos}$ is the posetal category associated to the poset $\eRel(A,B)$ of \cref{further-notation-for-relations-the-poset-of-relations-between-two-sets} of \cref{further-notation-for-relations} and \ChapterRef{\ChapterCategories, \cref{categories:posetal-categories}}{\cref{posetal-categories}}.%
\end{definition}
\subsection{The Category of Relations}\label{subsection-the-category-of-relations}
\begin{definition}{The Category of Relations}{the-category-of-relations}%
    The \index[set-theory]{category of relations}\index[set-theory]{relation!category of@category of}\textbf{category of relations} is the category \index[notation]{Rel@$\sfRel$}$\sfRel$ where
    \begin{itemize}
        \item\SloganFont{Objects. }The objects of $\sfRel$ are sets.
        \item\SloganFont{Morphisms. }For each $A,B\in\Obj(\Sets)$, we have
            \[
                \sfRel(A,B)
                \defeq
                \Rel(A,B).
            \]%
        \item\SloganFont{Identities. }For each $A\in\Obj(\sfRel)$, the unit map
            \[
                \Unit^{\sfRel}_{A}
                \colon
                \pt
                \to
                \Rel(A,A)
            \]%
            of $\sfRel$ at $A$ is defined by
            \[
                \id^{\sfRel}_{A}
                \defeq
                \chi_{A}(-_{1},-_{2}),
            \]%
            where $\chi_{A}(-_{1},-_{2})$ is the characteristic relation of $A$ of \ChapterRef{\ChapterConstructionsWithSets, \cref{constructions-with-sets:characteristic-functions-characteristic-relation} of \cref{constructions-with-sets:characteristic-functions}}{\cref{characteristic-functions-characteristic-relation} of \cref{characteristic-functions}}.
        \item\SloganFont{Composition. }For each $A,B,C\in\Obj(\sfRel)$, the composition map
            \[
                \circ^{\sfRel}_{A,B,C}%
                \colon%
                \Rel(B,C)%
                \times%
                \Rel(A,B)%
                \to%
                \Rel(A,C)%
            \]%
            of $\sfRel$ at $(A,B,C)$ is defined by%
            \[
                S\mathbin{{\circ}^{\sfRel}_{A,B,C}}R
                \defeq
                S\procirc R
            \]%
            for each $(S,R)\in\Rel(B,C)\times\Rel(A,B)$, where $S\procirc R$ is the composition of $S$ and $R$ of \ChapterRef{\ChapterConstructionsWithRelations, \cref{constructions-with-relations:composition-of-relations}}{\cref{composition-of-relations}}.
    \end{itemize}
\end{definition}
\subsection{The Closed Symmetric Monoidal Category of Relations}\label{subsection-the-closed-symmetric-monoidal-category-of-relations}
\subsubsection{The Monoidal Product}\label{subsubsection-the-closed-symmetric-monoidal-category-of-relations-the-monoidal-product}
\begin{definition}{The Monoidal Product of $\sfRel$}{the-monoidal-product-of-rel}%
    The \index[set-theory]{category of relations!monoidal product of}\textbf{monoidal product of $\sfRel$} is the functor
    \[
        \times%
        \colon%
        \sfRel\times\sfRel%
        \to%
        \sfRel%
    \]%
    where
    \begin{itemize}
        \item\SloganFont{Action on Objects. }For each $A,B\in\Obj(\sfRel)$, we have%
            \[%
                \mathord{\times}(A,B)%
                \defeq%
                A\times B,%
            \]%
            where $A\times B$ is the Cartesian product of sets of \ChapterRef{\ChapterConstructionsWithSets, \cref{constructions-with-sets:binary-products-of-sets}}{\cref{binary-products-of-sets}}.
        \item\SloganFont{Action on Morphisms. }For each $(A,C),(B,D)\in\Obj(\sfRel\times\sfRel)$, the action on morphisms
            \[
                \times_{(A,C),(B,D)}%
                \colon%
                \Rel(A,B)\times\Rel(C,D)%
                \to%
                \Rel(A\times C,B\times D)%
            \]%
            of $\times$ is given by sending a pair of morphisms $(R,S)$ of the form
            \begin{align*}
                R &\colon A\rightproarrow B,\\
                S &\colon C\rightproarrow D
            \end{align*}
            to the relation
            \[
                R\times S%
                \colon%
                A\times C%
                \rightproarrow%
                B\times D%
            \]%
            of \ChapterRef{\ChapterConstructionsWithRelations, \cref{constructions-with-relations:binary-products-of-relations}}{\cref{binary-products-of-relations}}.
    \end{itemize}
\end{definition}
\subsubsection{The Monoidal Unit}\label{subsubsection-the-closed-symmetric-monoidal-category-of-relations-the-monoidal-unit}
\begin{definition}{The Monoidal Unit of $\sfRel$}{the-monoidal-unit-of-rel}%
    The \index[set-theory]{category of relations!monoidal unit of}\textbf{monoidal unit of $\sfRel$} is the functor
    \[
        \Unit^{\sfRel}%
        \colon%
        \pt%
        \to%
        \sfRel%
    \]%
    picking the set
    \[
        \Unit_{\sfRel}%
        \defeq%
        \pt%
    \]%
    of $\sfRel$.
\end{definition}
\subsubsection{The Associator}\label{subsubsection-the-closed-symmetric-monoidal-category-of-relations-the-associator}
\begin{definition}{The Associator of $\sfRel$}{the-associator-of-rel}%
    The \index[set-theory]{category of relations!associator of}\textbf{associator of $\sfRel$} is the natural isomorphism%
    \[
        \alpha^{\sfRel}%
        \colon%
        {\times}\circ{({(\times)}\times{\sfid})}%
        \Longrightisoarrow%
        {\times}\circ{({\sfid}\times{(\times)})}\circ{\bfalpha^{\Cats}_{\sfRel,\sfRel,\sfRel}}\mrp{,}%
    \]%
    as in the diagram
    \[
        \begin{tikzcd}[row sep={0*\the\DL,between origins}, column sep={0*\the\DL,between origins}, background color=backgroundColor, ampersand replacement=\&]
            \&[0.30901699437\TwoCm]
            \&[0.5\TwoCm]
            \sfRel\times(\sfRel\times\sfRel)
            \&[0.5\TwoCm]
            \&[0.30901699437\TwoCm]
            \\[0.58778525229\TwoCm]
            (\sfRel\times\sfRel)\times\sfRel
            \&[0.30901699437\TwoCm]
            \&[0.5\TwoCm]
            \&[0.5\TwoCm]
            \&[0.30901699437\TwoCm]
            \sfRel\times\sfRel
            \\[0.95105651629\TwoCm]
            \&[0.30901699437\TwoCm]
            \sfRel\times\sfRel
            \&[0.5\TwoCm]
            \&[0.5\TwoCm]
            \sfRel\mrp{,}
            \&[0.30901699437\TwoCm]
            % 1-Arrows
            % Left Boundary
            \arrow[from=2-1,to=1-3,"\bfalpha^{\Cats}_{\sfRel,\sfRel,\sfRel}"{pos=0.1},isoarrowprime, mid vert]%
            \arrow[from=1-3,to=2-5,"{\sfid\times{(\times)}}"{pos=0.95},""{name=2}, mid vert]%
            \arrow[from=2-5,to=3-4,"\times"{pos=0.35}, mid vert]%
            % Right Boundary
            \arrow[from=2-1,to=3-2,"{{(\times)}\times\sfid}"'{pos=0.25}, mid vert]%
            \arrow[from=3-2,to=3-4,"\times"', mid vert]%
            % 2-Arrows
            \arrow[from=3-2,to=2,"\alpha^{\sfRel}"{description,pos=0.475},Rightarrow,shorten <= 0.5*\the\DL,shorten >= 1*\the\DL]%
        \end{tikzcd}
    \]%
    whose component
    \[
        \alpha^{\sfRel}_{A,B,C}%
        \colon%
        (A\times B)\times C%
        \rightproarrow%
        A\times(B\times C)%
    \]%
    at $A,B,C\in\Obj(\sfRel)$ is the relation defined by declaring
    \[
        ((a,b),c)
        \sim_{\alpha^{\sfRel}_{A,B,C}}
        (a',(b',c'))
    \]%
    \textiff $a=a'$, $b=b'$, and $c=c'$.
\end{definition}
\subsubsection{The Left Unitor}\label{subsubsection-the-closed-symmetric-monoidal-category-of-relations-the-left-unitor}
\begin{definition}{The Left Unitor of $\sfRel$}{the-left-unitor-of-rel}%
    The \index[set-theory]{category of relations!left unitor of}\textbf{left unitor of $\sfRel$} is the natural isomorphism%
    \begin{webcompile}
        \LUnitor^{\sfRel}
        \colon
        {\times}\circ{({\Unit^{\sfRel}}\times{\sfid})}
        \Longrightisoarrow
        \bfLUnitor^{\TwoCategoryOfCategories}_{\sfRel},%
        \quad
        \begin{tikzcd}[row sep={9.0*\the\DL,between origins}, column sep={9.0*\the\DL,between origins}, background color=backgroundColor, ampersand replacement=\&]
            \PunctualCategory\times\sfRel
            \arrow[r,  "\Unit^{\sfRel}\times\sfid"]
            \arrow[rd, dashed,"\bfLUnitor^{\TwoCategoryOfCategories}_{\sfRel}"'{name=1,pos=0.475},bend right=30]
            \&
            \sfRel\times\sfRel\mathrlap{,}
            \arrow[d, "\times"]
            \\
            {}
            \&
            \sfRel
            % 2-Arrows
            \arrow[Rightarrow,from=1-2,to=1,shorten >=1.0*\the\DL,shorten <=1.0*\the\DL,"\LUnitor^{\sfRel}"description]
        \end{tikzcd}
    \end{webcompile}%
    whose component
    \[
        \LUnitor^{\sfRel}_{A}
        \colon
        \Unit_{\sfRel}\times A
        \rightproarrow%
        A
    \]%
    at $A$ is defined by declaring
    \[
        (\point,a)
        \sim_{\LUnitor^{\sfRel}_{A}}
        b
    \]%
    \textiff $a=b$.
\end{definition}
\subsubsection{The Right Unitor}\label{subsubsection-the-closed-symmetric-monoidal-category-of-relations-the-right-unitor}
\begin{definition}{The Right Unitor of $\sfRel$}{the-right-unitor-of-rel}%
    The \index[set-theory]{category of relations!right unitor of}\textbf{right unitor of $\sfRel$} is the natural isomorphism%
    \begin{webcompile}
        \RUnitor^{\sfRel}
        \colon
        {\times}\circ{({\sfid}\times{\Unit^{\sfRel}})}
        \Longrightisoarrow%
        \bfRUnitor^{\TwoCategoryOfCategories}_{\sfRel},
        \quad
        \begin{tikzcd}[row sep={9.0*\the\DL,between origins}, column sep={9.0*\the\DL,between origins}, background color=backgroundColor, ampersand replacement=\&]
            \sfRel\times\PunctualCategory
            \arrow[r, "\sfid\times\Unit^{\sfRel}"]
            \arrow[rd, dashed,"\bfRUnitor^{\TwoCategoryOfCategories}_{\sfRel}"'{name=1,pos=0.475},bend right=30]
            \&
            \sfRel\times\sfRel\mathrlap{,}
            \arrow[d, "\times"]
            \\
            {}
            \&
            \sfRel
            % 2-Arrows
            \arrow[Rightarrow,from=1-2,to=1,shorten >=1.0*\the\DL,shorten <=1.0*\the\DL,"\RUnitor^{\sfRel}"description]
        \end{tikzcd}
    \end{webcompile}%
    whose component
    \[
        \RUnitor^{\sfRel}_{A}
        \colon
        A\times\Unit_{\sfRel}
        \rightproarrow
        A
    \]%
    at $A$ is defined by declaring
    \[
        (a,\point)
        \sim_{\RUnitor^{\sfRel}_{A}}
        b
    \]%
    \textiff $a=b$.
\end{definition}
\subsubsection{The Symmetry}\label{subsubsection-the-closed-symmetric-monoidal-category-of-relations-the-symmetry}
\begin{definition}{The Symmetry of $\sfRel$}{the-symmetry-of-rel}%
    The \index[set-theory]{category of relations!symmetry of}\textbf{symmetry of $\sfRel$} is the natural isomorphism%
    \begin{webcompile}
        \sigma^{\sfRel}
        \colon
        \times
        \Longrightarrow
        {\times}\circ{\bfsigma^{\TwoCategoryOfCategories}_{\sfRel,\sfRel}},
        \quad
        \begin{tikzcd}[row sep={5.0*\the\DL,between origins}, column sep={4.0*\the\DL,between origins}, background color=backgroundColor, ampersand replacement=\&]
            \sfRel\times\sfRel
            \arrow[rr,"\times"{name=1},pos=0.5]
            \arrow[rd,"\bfsigma^{\TwoCategoryOfCategories}_{\sfRel,\sfRel}"'{pos=0.25},bend right=15]
            \&
            {}
            \&
            \sfRel\mrp{,}
            \\
            \&
            \sfRel\times\sfRel
            \arrow[ru,"\times"'{pos=0.525},bend right=15]
            \&
            % 2-arrows
            \arrow[from=1-2,to=2-2,"\sigma^{\sfRel}"{description,pos=0.425},shorten <= 0.0*\the\DL,shorten >=0.25*\the\DL,Rightarrow]%
        \end{tikzcd}
    \end{webcompile}%
    whose component
    \[
        \sigma^{\sfRel}_{A,B}
        \colon
        A\times B
        \to
        B\times A
    \]%
    at $(A,B)$ is defined by declaring
    \[
        (a,b)
        \sim_{\sigma^{\sfRel}_{A,B}}
        (b',a')
    \]%
    \textiff $a=a'$ and $b=b'$.
\end{definition}
\subsubsection{The Internal Hom}\label{subsubsection-the-closed-symmetric-monoidal-category-of-relations-the-internal-hom}
\begin{definition}{The Internal Hom of $\sfRel$}{the-internal-hom-of-rel}%
    The \index[set-theory]{category of relations!internal Hom of}\textbf{internal Hom of $\sfRel$} is the functor%
    \[%
        \Rel%
        \colon%
        \sfRel^{\op}\times\sfRel%
        \to
        \sfRel%
    \]%
    defined
    \begin{itemize}
        \item On objects by sending $A,B\in\Obj(\sfRel)$ to the set $\Rel(A,B)$ of \cref{the-set-of-relations-between-two-sets-1} of \cref{the-set-of-relations-between-two-sets}.
        \item On morphisms by pre/post-composition defined as in \ChapterRef{\ChapterConstructionsWithRelations, \cref{constructions-with-relations:composition-of-relations}}{\cref{composition-of-relations}}.
    \end{itemize}
\end{definition}
\begin{proposition}{Properties of the Internal Hom of $\sfRel$}{properties-of-the-internal-hom-of-rel}%
    Let $A,B,C\in\Obj(\sfRel)$.
    \begin{enumerate}
        \item\label{properties-of-the-internal-hom-of-rel-adjointness}\SloganFont{Adjointness. }We have adjunctions
            \begin{webcompile}%TODO: I don't like the spacing here
                \begin{gathered}
                    \Adjunction#A\times -#{\Rel(A,-)}#\sfRel#\sfRel,#\\
                    \Adjunction#-\times B#{\Rel(B,-)}#\sfRel#\sfRel,#
                \end{gathered}
            \end{webcompile}%
            witnessed by bijections
            \begin{align*}
                \Rel(A\times B,C) &\cong \Rel(A,\Rel(B,C)),\\
                \Rel(A\times B,C) &\cong \Rel(B,\Rel(A,C)),
            \end{align*}
            natural in $A,B,C\in\Obj(\sfRel)$.
        %\item\label{properties-of-the-internal-hom-of-rel-}\SloganFont{. }
    \end{enumerate}
\end{proposition}
\begin{Proof}{Proof of \cref{properties-of-the-internal-hom-of-rel}}%
    \ProofBox{\cref{properties-of-the-internal-hom-of-rel-adjointness}: Adjointness}%
    Indeed, we have
    \begin{align*}
        \Rel(A\times B,C) &\defeq \Sets(A\times B\times C,\TV)\\
                          &\defeq \Rel(A,B\times C)\\
                          &\defeq \Rel(A,\Rel(B,C)),
    \end{align*}
    and similarly for the bijection $\Rel(A\times B,C)\cong\Rel(B,\Rel(A,C))$.
\end{Proof}
\subsubsection{The Closed Symmetric Monoidal Category of Relations}\label{subsubsection-the-closed-symmetric-monoidal-category-of-relations}
\begin{proposition}{The Closed Symmetric Monoidal Category of Relations}{the-closed-symmetric-monoidal-category-of-relations}%
    The category $\sfRel$ admits a closed symmetric monoidal category structure consisting of\index[set-theory]{relation!closed symmetric monoidal category of}%
    %--- Begin Footnote ---%
    \footnote{%
        \textdbend\SloganFont{Warning: }This is not a Cartesian monoidal structure, as the product on $\sfRel$ is in fact given by the disjoint union of sets; see \ChapterRef{\ChapterConstructionsWithRelations, \cref{TODO}}{\cref{TODO}}.
        \par\vspace*{\TCBBoxCorrection}
    }%
    %---  End Footnote  ---%
    \begin{itemize}
        \item\SloganFont{The Underlying Category. }The category $\sfRel$ of sets and relations of \cref{the-category-of-relations}.
        \item\SloganFont{The Monoidal Product. }The functor
            \[
                \times%
                \colon%
                \Rel\times\Rel%
                \to%
                \Rel
            \]%
            of \cref{the-monoidal-product-of-rel}.
        \item\SloganFont{The Internal Hom. }The internal Hom functor
            \[
                \eRel%
                \colon%
                \Rel^{\op}\times\Rel%
                \to%
                \Rel%
            \]%
            of \cref{the-internal-hom-of-rel}.
        \item\SloganFont{The Monoidal Unit. }The functor
            \[
                \Unit^{\Rel}
                \colon
                \PunctualCategory
                \to
                \Rel
            \]
            of \cref{the-monoidal-unit-of-rel}.
        \item\SloganFont{The Associators. }The natural isomorphism
            \[
                \alpha^{\Rel}
                \colon
                {\times}\circ{({\times}\times\id_{\Rel})}
                \Longrightisoarrow
                {\times}\circ{(\id_{\Rel}\times{\times})}\circ{\bfalpha^{\Cats}_{\Rel,\Rel,\Rel}}
            \]
            of \cref{the-associator-of-rel}.
        \item\SloganFont{The Left Unitors. }The natural isomorphism
            \[
                \LUnitor^{\Rel}%
                \colon%
                {\times}\circ{(\Unit^{\Rel}\times\id_{\Rel})}
                \Longrightisoarrow
                \bfLUnitor^{\TwoCategoryOfCategories}_{\Rel}
            \]
            of \cref{the-left-unitor-of-rel}.
        \item\SloganFont{The Right Unitors. }The natural isomorphism
            \[
                \RUnitor^{\Rel}%
                \colon%
                {\times}\circ{({\sfid}\times{\Unit^{\Rel}})}%
                \Longrightisoarrow%
                \bfRUnitor^{\TwoCategoryOfCategories}_{\Rel}%
            \]
            of \cref{the-right-unitor-of-rel}.
        \item\SloganFont{The Symmetry. }The natural isomorphism
            \[
                \sigma^{\Rel}
                \colon
                {\times}
                \Longrightisoarrow
                {\times}\circ{\bfsigma^{\TwoCategoryOfCategories}_{\Rel,\Rel}}
            \]
            of \cref{the-symmetry-of-rel}.
    \end{itemize}
\end{proposition}
\begin{Proof}{Proof of \cref{the-closed-symmetric-monoidal-category-of-relations}}%
    Omitted.
\end{Proof}
\subsection{The $2$-Category of Relations}\label{subsection-the-2-category-of-relations}
\begin{definition}{The $2$-Category of Relations}{the-2-category-of-relations}%
    The \index[set-theory]{relation!two-category of@$2$-category of}\textbf{$2$-category of relations} is the locally posetal $2$-category \index[notation]{Rel@$\sfbfRel$}$\sfbfRel$ where
    \begin{itemize}
        \item\SloganFont{Objects. }The objects of $\sfbfRel$ are sets.
        \item\SloganFont{$\eHom$-Objects. }For each $A,B\in\Obj(\Sets)$, we have
            \begin{align*}
                \Hom_{\sfbfRel}(A,B) &\defeq \eRel(A,B)\\%
                                     &\defeq (\Rel(A,B),\subset).%
            \end{align*}
        \item\SloganFont{Identities. }For each $A\in\Obj(\sfbfRel)$, the unit map
            \[
                \Unit^{\sfbfRel}_{A}
                \colon
                \pt
                \to
                \eRel(A,A)
            \]%
            of $\sfbfRel$ at $A$ is defined by
            \[
                \id^{\sfbfRel}_{A}
                \defeq
                \chi_{A}(-_{1},-_{2}),
            \]%
            where $\chi_{A}(-_{1},-_{2})$ is the characteristic relation of $A$ of \ChapterRef{\ChapterConstructionsWithSets, \cref{constructions-with-sets:characteristic-functions-characteristic-relation} of \cref{constructions-with-sets:characteristic-functions}}{\cref{characteristic-functions-characteristic-relation} of \cref{characteristic-functions}}.
        \item\SloganFont{Composition. }For each $A,B,C\in\Obj(\sfbfRel)$, the composition map%
            %--- Begin Footnote ---%
            \footnote{%
                Note that this is indeed a morphism of posets: given relations $R_{1},R_{2}\in\eRel(A,B)$ and $S_{1},S_{2}\in\eRel(B,C)$ such that
                \begin{align*}
                    R_{1} &\subset R_{2},\\
                    S_{1} &\subset S_{2},
                \end{align*}
                we have also $S_{1}\procirc R_{1}\subset S_{2}\procirc R_{2}$.
                \par\vspace*{\TCBBoxCorrection}
            }%
            %---  End Footnote  ---%
            \[
                \circ^{\sfbfRel}_{A,B,C}%
                \colon%
                \eRel(B,C)%
                \times%
                \eRel(A,B)%
                \to%
                \eRel(A,C)%
            \]%
            of $\sfbfRel$ at $(A,B,C)$ is defined by%
            \[
                S\mathbin{{\circ}^{\sfbfRel}_{A,B,C}}R
                \defeq
                S\procirc R
            \]%
            for each $(S,R)\in\sfbfRel(B,C)\times\sfbfRel(A,B)$, where $S\procirc R$ is the composition of $S$ and $R$ of \ChapterRef{\ChapterConstructionsWithRelations, \cref{constructions-with-relations:composition-of-relations}}{\cref{composition-of-relations}}.
    \end{itemize}
\end{definition}
\subsection{The Double Category of Relations}\label{subsection-the-double-category-of-relations}
\subsubsection{The Double Category of Relations}\label{subsubsection-the-double-category-of-relations}
\begin{definition}{The Double Category of Relations}{the-double-category-of-relations}%
    The \index[set-theory]{relation!double category of}\index[higher-categories]{double category!of relations}\textbf{double category of relations} is the locally posetal double category \index[notation]{Reldbl@$\dblRel$}$\smash{\dblRel}$ where
    \begin{itemize}
        \item\SloganFont{Objects. }The objects of $\dblRel$ are sets.
        \item\SloganFont{Vertical Morphisms. }The vertical morphisms of $\dblRel$ are maps of sets $f\colon A\to B$.
        \item\SloganFont{Horizontal Morphisms. }The horizontal morphisms of $\dblRel$ are relations $R\colon A\rightproarrow X$.
        \item\SloganFont{$2$-Morphisms. }A $2$-cell
            \[
                \begin{tikzcd}[row sep={5.0*\the\DL,between origins}, column sep={5.0*\the\DL,between origins}, background color=backgroundColor, ampersand replacement=\&]
                    A
                    \arrow[r,mid vert,"R"{name=1}]
                    \arrow[d,"f"']
                    \&
                    B
                    \arrow[d,"g"]
                    \\
                    X
                    \arrow[r,mid vert,"S"'{name=2}]
                    \&
                    Y
                    % 2-Arrows
                    \arrow[from=1,to=2,"\alpha"description,shorten=0.75*\the\DL,Rightarrow]%
                \end{tikzcd}
            \]%
            of $\dblRel$ is either non-existent or an inclusion of relations of the form
            \begin{webcompile}
                R%
                \subset%
                S\circ(f\times g),%
                \quad
                \begin{tikzcd}[row sep={5.0*\the\DL,between origins}, column sep={7.0*\the\DL,between origins}, background color=backgroundColor, ampersand replacement=\&]
                    A\times B
                    \arrow[r,"R"{name=1}]
                    \arrow[d,"f\times g"']
                    \&
                    \TV
                    \arrow[d,"\id_{\TV}"]
                    \\
                    X\times Y
                    \arrow[r,"S"'{name=2}]
                    \&
                    \TV\mrp{.}
                    % 2-Arrows
                    \arrow[from=1-2,to=2-1,"\scalebox{1.5}{$\subset$}"{sloped,description},phantom,shorten <= 0.5*\the\DL,shorten >= 0.625*\the\DL,Rightarrow,pos=0.5]%
                \end{tikzcd}
            \end{webcompile}%
        \item\SloganFont{Horizontal Identities. }The horizontal unit functor of $\dblRel$ is the functor of \cref{the-horizontal-identities-of-dblrel}.
        \item\SloganFont{Vertical Identities. }For each $A\in\Obj(\dblRel)$, we have
            \[
                \id^{\dblRel}_{A}
                \defeq
                \id_{A}.
            \]%
        \item\SloganFont{Identity $2$-Morphisms. }For each horizontal morphism $R\colon A\rightproarrow B$ of $\dblRel$, the identity $2$-morphism
            \[
                \begin{tikzcd}[row sep={5.0*\the\DL,between origins}, column sep={5.0*\the\DL,between origins}, background color=backgroundColor, ampersand replacement=\&]
                    A
                    \arrow[r,mid vert,"R"{name=1}]
                    \arrow[d,"\id_{A}"']
                    \&
                    B
                    \arrow[d,"\id_{B}"]
                    \\
                    A
                    \arrow[r,mid vert,"R"'{name=2}]
                    \&
                    B
                    % 2-arrows
                    \arrow[from=1,to=2,"\id_{R}"description,shorten=0.75*\the\DL,Rightarrow]%
                \end{tikzcd}
            \]%
            of $R$ is the identity inclusion
            \begin{webcompile}
                R%
                \subset%
                R,%
                \quad
                \begin{tikzcd}[row sep={5.0*\the\DL,between origins}, column sep={7.0*\the\DL,between origins}, background color=backgroundColor, ampersand replacement=\&]
                    B\times A
                    \arrow[r,"R"]
                    \arrow[d,"\id_{B}\times\id_{A}"']
                    \&
                    \TV
                    \arrow[d,"\id_{\TV}"]
                    \\
                    B\times A
                    \arrow[r,"R"']
                    \&
                    \TV\mrp{.}
                    % 2-Arrows
                    \arrow[from=1-2,to=2-1,"\scalebox{1.5}{$\subset$}"{sloped,description},phantom,shorten <= 0.5*\the\DL,shorten >= 0.625*\the\DL,Rightarrow,pos=0.5]%
                \end{tikzcd}
            \end{webcompile}%
        \item\SloganFont{Horizontal Composition. }The horizontal composition functor of $\dblRel$ is the functor of \cref{the-horizontal-composition-of-dblrel}.
        \item\SloganFont{Vertical Composition of $1$-Morphisms. }For each composable pair $A\smash{\xlongrightarrow{F}}B\smash{\xlongrightarrow{G}}C$ of vertical morphisms of $\dblRel$, i.e.\ maps of sets, we have
            \[
                g\mathbin{{\circ}^{\dblRel}}f
                \defeq
                g\circ f.
            \]%
        \item\SloganFont{Vertical Composition of $2$-Morphisms. }The vertical composition of $2$-morphisms in $\dblRel$ is defined as in \cref{the-vertical-composition-of-two-morphisms-in-dblrel}.
        \item\SloganFont{Associators. }The associators of $\dblRel$ is defined as in \cref{the-associators-of-dblrel}.
        \item\SloganFont{Left Unitors. }The left unitors of $\dblRel$ is defined as in \cref{the-left-unitors-of-dblrel}.
        \item\SloganFont{Right Unitors. }The right unitors of $\dblRel$ is defined as in \cref{the-right-unitors-of-dblrel}.
    \end{itemize}
\end{definition}
\subsubsection{Horizontal Identities}\label{subsubsection-the-double-category-of-relations-the-associator}
\begin{definition}{The Horizontal Identities of $\dblRel$}{the-horizontal-identities-of-dblrel}%
    The \textbf{horizontal unit functor} of $\dblRel$ is the functor
    \[
        \Unit^{\dblRel}
        \colon
        \dblRel_{0}
        \to
        \dblRel_{1}
    \]%
    of $\dblRel$ is the functor where
    \begin{itemize}
        \item\SloganFont{Action on Objects. }For each $A\in\Obj(\dblRel_{0})$, we have
            \[
                \Unit_{A}
                \defeq
                \chi_{A}(-_{1},-_{2}).
            \]%
        \item\SloganFont{Action on Morphisms. }For each vertical morphism $f\colon A\to B$ of $\dblRel$, i.e.\ each map of sets $f$ from $A$ to $B$, the identity $2$-morphism
            \[
                \begin{tikzcd}[row sep={5.0*\the\DL,between origins}, column sep={5.0*\the\DL,between origins}, background color=backgroundColor, ampersand replacement=\&]
                    A
                    \arrow[r,mid vert,"\Unit_{A}"{name=1}]
                    \arrow[d,"f"']
                    \&
                    A
                    \arrow[d,"f"]
                    \\
                    B
                    \arrow[r,mid vert,"\Unit_{B}"'{name=2}]
                    \&
                    B
                    % 2-arrows
                    \arrow[from=1,to=2,"\Unit_{f}"description,shorten=0.75*\the\DL,Rightarrow]%
                \end{tikzcd}
            \]%
            of $f$ is the inclusion
            \begin{webcompile}
                \chi_{B}\circ(f\times f)%
                \subset%
                \chi_{A},%
                \quad%
                \begin{tikzcd}[row sep={5.0*\the\DL,between origins}, column sep={10.0*\the\DL,between origins}, background color=backgroundColor, ampersand replacement=\&]
                    A\times A
                    \arrow[r,"{\chi_{A}(-_{1},-_{2})}"]
                    \arrow[d,"f\times f"']
                    \&
                    \TV
                    \arrow[d,"\id_{\TV}"]
                    \\
                    B\times B
                    \arrow[r,"{\chi_{B}(-_{1},-_{2})}"']
                    \&
                    \TV
                    % 2-Arrows
                    \arrow[from=1-2,to=2-1,"\scalebox{1.5}{$\subset$}"{sloped,description},phantom,shorten <= 0.5*\the\DL,shorten >= 0.625*\the\DL,Rightarrow,pos=0.5]%
                \end{tikzcd}
            \end{webcompile}%
            of \ChapterRef{\ChapterConstructionsWithSets, \cref{constructions-with-sets:properties-of-characteristic-relations-the-inclusion-of-characteristic-relations-associated-to-a-function} of \cref{constructions-with-sets:properties-of-characteristic-relations}}{\cref{properties-of-characteristic-relations-the-inclusion-of-characteristic-relations-associated-to-a-function} of \cref{properties-of-characteristic-relations}}.
    \end{itemize}
\end{definition}
\subsubsection{Horizontal Composition}\label{subsubsection-the-double-category-of-relations-the-horizontal-composition}
\begin{definition}{The Horizontal Composition of $\dblRel$}{the-horizontal-composition-of-dblrel}%
    The \textbf{horizontal composition functor} of $\dblRel$ is the functor
    \[
        \doublecirc^{\dblRel}
        \colon
        \dblRel_{1}\ttimes_{\dblRel_{0}}\dblRel_{1}
        \to
        \dblRel_{1}
    \]%
    of $\dblRel$ is the functor where
    \begin{itemize}
        \item\SloganFont{Action on Objects. }For each composable pair $\smash{A\xrightproarrow{R}B\xrightproarrow{S}C}$ of horizontal morphisms of $\dblRel$, we have
            \[%
                S\doublecirc R%
                \defeq%
                S\procirc R,%
            \]%
            where $S\procirc R$ is the composition of $R$ and $S$ of \ChapterRef{\ChapterConstructionsWithRelations, \cref{constructions-with-relations:composition-of-relations}}{\cref{composition-of-relations}}.
        \item\SloganFont{Action on Morphisms. }For each horizontally composable pair
            \begin{webcompile}
                \begin{tikzcd}[row sep={5.0*\the\DL,between origins}, column sep={5.0*\the\DL,between origins}, background color=backgroundColor, ampersand replacement=\&]
                    A
                    \arrow[r,mid vert,"R"{name=1}]
                    \arrow[d,"f"']
                    \&
                    B
                    \arrow[d,"g"]
                    \\
                    X
                    \arrow[r,mid vert,"T"'{name=2}]
                    \&
                    Y
                    % 2-arrows
                    \arrow[from=1,to=2,"\alpha"description,shorten=0.75*\the\DL,Rightarrow]%
                \end{tikzcd}
                \qquad
                \begin{tikzcd}[row sep={5.0*\the\DL,between origins}, column sep={5.0*\the\DL,between origins}, background color=backgroundColor, ampersand replacement=\&]
                    B
                    \arrow[r,mid vert,"S"{name=1}]
                    \arrow[d,"g"']
                    \&
                    C
                    \arrow[d,"h"]
                    \\
                    Y
                    \arrow[r,mid vert,"U"'{name=2}]
                    \&
                    Z
                    % 2-arrows
                    \arrow[from=1,to=2,"\beta"description,shorten=0.75*\the\DL,Rightarrow]%
                \end{tikzcd}
            \end{webcompile}%
            of $2$-morphisms of $\dblRel$, i.e.\ for each pair
            \begin{webcompile}
                \begin{tikzcd}[row sep={5.0*\the\DL,between origins}, column sep={7.0*\the\DL,between origins}, background color=backgroundColor, ampersand replacement=\&]
                    A\times B
                    \arrow[r,"R"]
                    \arrow[d,"f\times g"']
                    \&
                    \TV
                    \arrow[d,"\id_{\TV}"]
                    \\
                    X\times Y
                    \arrow[r,"T"']
                    \&
                    \TV
                    % 2-Arrows
                    \arrow[from=1-2,to=2-1,"\scalebox{1.5}{$\subset$}"{sloped,description},phantom,shorten <= 0.5*\the\DL,shorten >= 0.625*\the\DL,Rightarrow,pos=0.5]%
                \end{tikzcd}
                \qquad
                \begin{tikzcd}[row sep={5.0*\the\DL,between origins}, column sep={7.0*\the\DL,between origins}, background color=backgroundColor, ampersand replacement=\&]
                    B\times C
                    \arrow[r,"S"]
                    \arrow[d,"g\times h"']
                    \&
                    \TV
                    \arrow[d,"\id_{\TV}"]
                    \\
                    Y\times Z
                    \arrow[r,"U"']
                    \&
                    \TV
                    % 2-Arrows
                    \arrow[from=1-2,to=2-1,"\scalebox{1.5}{$\subset$}"{sloped,description},phantom,shorten <= 0.5*\the\DL,shorten >= 0.625*\the\DL,Rightarrow,pos=0.5]%
                \end{tikzcd}
            \end{webcompile}%
            of inclusions of relations, the horizontal composition
            \[
                \begin{tikzcd}[row sep={5.0*\the\DL,between origins}, column sep={5.0*\the\DL,between origins}, background color=backgroundColor, ampersand replacement=\&]
                    A
                    \arrow[r,mid vert,"S\doublecirc R"{name=1}]
                    \arrow[d,"f"']
                    \&
                    C
                    \arrow[d,"h"]
                    \\
                    X
                    \arrow[r,mid vert,"U\doublecirc T"'{name=2}]
                    \&
                    Z
                    % 2-arrows
                    \arrow[from=1,to=2,"\beta\doublecirc\alpha"description,shorten=0.75*\the\DL,Rightarrow]%
                \end{tikzcd}
            \]%
            of $\alpha$ and $\beta$ is the inclusion of relations%
            %--- Begin Footnote ---%
            \footnote{%
                This is justified by noting that, given $(a,c)\in A\times C$, the statement
                \begin{itemize}
                    \item We have $a\sim_{(U\procirc T)\circ(f\times h)}c$, i.e.\ $f(a)\sim_{U\procirc T}h(c)$, i.e.\ there exists some $y\in Y$ such that:
                        \begin{itemize}
                            \item We have $f(a)\sim_{T}y$.
                            \item We have $y\sim_{U}h(c)$.
                        \end{itemize}
                \end{itemize}
                is implied by the statement
                \begin{itemize}
                    \item We have $a\sim_{S\procirc R}c$, i.e.\ there exists some $b\in B$ such that:
                        \begin{itemize}
                            \item We have $a\sim_{R}b$.
                            \item We have $b\sim_{S}c$.
                        \end{itemize}
                \end{itemize}
                since:
                \begin{itemize}
                    \item If $a\sim_{R}b$, then $f(a)\sim_{T}g(b)$, as $T\circ(f\times g)\subset R$;
                    \item If $b\sim_{S}c$, then $g(b)\sim_{U}h(c)$, as $U\circ(g\times h)\subset S$.
                \end{itemize}
                \par\vspace*{\TCBBoxCorrection}
            }%
            %---  End Footnote  ---%
            \begin{webcompile}
                (U\procirc T)\circ(f\times h)%
                \subset%
                (S\procirc R)%
                \quad
                \begin{tikzcd}[row sep={5.0*\the\DL,between origins}, column sep={8.0*\the\DL,between origins}, background color=backgroundColor, ampersand replacement=\&]
                    A\times C
                    \arrow[r,"S\procirc R"]
                    \arrow[d,"f\times h"']
                    \&
                    \TV
                    \arrow[d,"\id_{\TV}"]
                    \\
                    X\times Z
                    \arrow[r,"U\procirc T"']
                    \&
                    \TV\mrp{.}
                    % 2-Arrows
                    \arrow[from=1-2,to=2-1,"\scalebox{1.5}{$\subset$}"{sloped,description},phantom,shorten <= 0.5*\the\DL,shorten >= 0.625*\the\DL,Rightarrow,pos=0.5]%
                \end{tikzcd}
            \end{webcompile}%
    \end{itemize}
\end{definition}
\subsubsection{Vertical Composition of 2-Morphisms}\label{subsubsection-the-double-category-of-relations-vertical-composition-of-2-morphisms}
\begin{definition}{The Vertical Composition of 2-Morphisms in $\dblRel$}{the-vertical-composition-of-two-morphisms-in-dblrel}%
    The \textbf{vertical composition} in $\dblRel$ is defined as follows: for each vertically composable pair
    \begin{webcompile}
        \begin{tikzcd}[row sep={5.0*\the\DL,between origins}, column sep={5.0*\the\DL,between origins}, background color=backgroundColor, ampersand replacement=\&]
            A
            \arrow[r,mid vert,"R"{name=1}]
            \arrow[d,"f"']
            \&
            X
            \arrow[d,"g"]
            \\
            B
            \arrow[r,mid vert,"S"'{name=2}]
            \&
            Y
            % 2-arrows
            \arrow[from=1,to=2,"\alpha"description,shorten=0.75*\the\DL,Rightarrow]%
        \end{tikzcd}
        \qquad
        \begin{tikzcd}[row sep={5.0*\the\DL,between origins}, column sep={5.0*\the\DL,between origins}, background color=backgroundColor, ampersand replacement=\&]
            B
            \arrow[r,mid vert,"S"{name=1}]
            \arrow[d,"h"']
            \&
            Y
            \arrow[d,"k"]
            \\
            C
            \arrow[r,mid vert,"T"'{name=2}]
            \&
            Z
            % 2-arrows
            \arrow[from=1,to=2,"\beta"description,shorten=0.75*\the\DL,Rightarrow]%
        \end{tikzcd}
    \end{webcompile}%
    of $2$-morphisms of $\dblRel$, i.e.\ for each each pair
    \begin{webcompile}
        \begin{tikzcd}[row sep={5.0*\the\DL,between origins}, column sep={7.0*\the\DL,between origins}, background color=backgroundColor, ampersand replacement=\&]
            A\times X
            \arrow[r,"R"]
            \arrow[d,"f\times g"']
            \&
            \TV
            \arrow[d,"\id_{\TV}"]
            \\
            B\times Y
            \arrow[r,"S"']
            \&
            \TV
            % 2-Arrows
            \arrow[from=1-2,to=2-1,"\scalebox{1.5}{$\subset$}"{sloped,description},phantom,shorten <= 0.5*\the\DL,shorten >= 0.625*\the\DL,Rightarrow,pos=0.5]%
        \end{tikzcd}
        \qquad
        \begin{tikzcd}[row sep={5.0*\the\DL,between origins}, column sep={7.0*\the\DL,between origins}, background color=backgroundColor, ampersand replacement=\&]
            B\times Y
            \arrow[r,"S"]
            \arrow[d,"h\times k"']
            \&
            \TV
            \arrow[d,"\id_{\TV}"]
            \\
            C\times Z
            \arrow[r,"T"']
            \&
            \TV
            % 2-Arrows
            \arrow[from=1-2,to=2-1,"\scalebox{1.5}{$\subset$}"{sloped,description},phantom,shorten <= 0.5*\the\DL,shorten >= 0.625*\the\DL,Rightarrow,pos=0.5]%
        \end{tikzcd}
    \end{webcompile}%
    of inclusions of relations, we define the vertical composition
    \[
        \begin{tikzcd}[row sep={5.0*\the\DL,between origins}, column sep={5.0*\the\DL,between origins}, background color=backgroundColor, ampersand replacement=\&]
            A
            \arrow[r,mid vert,"R"{name=1}]
            \arrow[d,"h\circ f"']
            \&
            X
            \arrow[d,"k\circ g"]
            \\
            C
            \arrow[r,mid vert,"T"'{name=2}]
            \&
            Z
            % 2-arrows
            \arrow[from=1,to=2,"\beta\circ\alpha"description,shorten=0.75*\the\DL,Rightarrow]%
        \end{tikzcd}
    \]%
    of $\alpha$ and $\beta$ as the inclusion of relations
    \begin{webcompile}
        T\circ[(h\circ f)\times(k\circ g)]%
        \subset%
        R,%
        \quad%
        \begin{tikzcd}[row sep={5.0*\the\DL,between origins}, column sep={7.0*\the\DL,between origins}, background color=backgroundColor, ampersand replacement=\&]
            A\times X
            \arrow[r,"R"]
            \arrow[d,"{(h\circ f)\times(k\circ g)}"']
            \&
            \TV
            \arrow[d,"\id_{\TV}"]
            \\
            C\times Z
            \arrow[r,"T"']
            \&
            \TV
            % 2-Arrows
            \arrow[from=1-2,to=2-1,"\scalebox{1.5}{$\subset$}"{sloped,description},phantom,shorten <= 0.5*\the\DL,shorten >= 0.625*\the\DL,Rightarrow,pos=0.5]%
        \end{tikzcd}
    \end{webcompile}%
    given by the pasting of inclusions%
    %--- Begin Footnote ---%
    \footnote{%
        This is justified by noting that, given $(a,x)\in A\times X$, the statement
        \begin{itemize}
            \item We have $h(f(a))\sim_{T}k(g(x))$;
        \end{itemize}
        is implied by the statement
        \begin{itemize}
            \item We have $a\sim_{R}x$;
        \end{itemize}
        since
        \begin{itemize}
            \item If $a\sim_{R}x$,       then $f(a)\sim_{S}g(x)$, as $S\circ(f\times g)\subset R$;
            \item If $b\sim_{S}y$,       then $h(b)\sim_{T}k(y)$, as $T\circ(h\times k)\subset S$, and thus, in particular:
                \begin{itemize}
                    \item If $f(a)\sim_{S}g(x)$, then $h(f(a))\sim_{T}k(g(x))$.
                \end{itemize}
        \end{itemize}
        \par\vspace*{\TCBBoxCorrection}
    }%
    %---  End Footnote  ---%
    \[
        \begin{tikzcd}[row sep={5.0*\the\DL,between origins}, column sep={7.5*\the\DL,between origins}, background color=backgroundColor, ampersand replacement=\&]
            A\times X
            \arrow[r,"R"]
            \arrow[d,"f\times g"']
            \&
            \TV
            \arrow[d,"\id_{\TV}"]
            \\
            B\times Y
            \arrow[r,"S"description]
            \arrow[d,"h\times k"']
            \&
            \TV
            \arrow[d,"\id_{\TV}"]
            \\
            C\times Z
            \arrow[r,"T"']
            \&
            \TV\mrp{.}
            % 2-Arrows
            \arrow[from=1-2,to=2-1,"\scalebox{1.5}{$\subset$}"{sloped,description},phantom,shorten <= 0.5*\the\DL,shorten >= 0.625*\the\DL,Rightarrow,pos=0.5]%
            \arrow[from=2-2,to=3-1,"\scalebox{1.5}{$\subset$}"{sloped,description},phantom,shorten <= 0.5*\the\DL,shorten >= 0.625*\the\DL,Rightarrow,pos=0.5]%
        \end{tikzcd}
    \]%
\end{definition}
\subsubsection{The Associators}\label{subsubsection-the-double-category-of-relations-the-associators}
\begin{definition}{The Associators of $\dblRel$}{the-associators-of-dblrel}%
    For each composable triple%
    \[%
        A\xrightproarrow{R}B\xrightproarrow{S}C\xrightproarrow{T}D%
    \]%
    of horizontal morphisms of $\dblRel$, the component
    \begin{webcompile}
        \alpha^{\dblRel}_{T,S,R}%
        \colon%
        \textcolor{OIvermillion}{(T\doublecirc S)}\doublecirc\textcolor{OIblue}{R}%
        \Longrightisoarrow%
        \textcolor{OIblue}{T}\doublecirc\textcolor{OIvermillion}{(S\doublecirc R)},%
        \quad%
        \begin{tikzcd}[row sep={5.0*\the\DL,between origins}, column sep={3.5*\the\DL,between origins}, background color=backgroundColor, ampersand replacement=\&]
            \textcolor{OIblue}{A}
            \arrow[r, mid vert,"R",OIblue]
            \arrow[d, "\id_{A}"']
            \&
            \textcolor{OIgreen}{B}
            \arrow[r, mid vert,"S"{name=1},OIvermillion]
            %\arrow[d, "g"description]
            \&
            \textcolor{OIvermillion}{C}
            \arrow[r, mid vert,"T",OIvermillion]
            %\arrow[d, "h"]
            \&
            \textcolor{OIvermillion}{D}
            \arrow[d, "\id_{D}"]
            \\
            \textcolor{OIvermillion}{A}
            \arrow[r, mid vert,"R"',OIvermillion]
            \&
            \textcolor{OIvermillion}{B}
            \arrow[r, mid vert,"S"'{name=2},OIvermillion]
            \&
            \textcolor{OIgreen}{C}
            \arrow[r, mid vert,"T"',OIblue]
            \&
            \textcolor{OIblue}{D}
            % 2-Arrows
            \arrow[from=1,to=2,"\alpha^{\dblRel}_{T,S,R}"',shorten=0.75*\the\DL,Rightarrow]
        \end{tikzcd}
    \end{webcompile}%
    of the associator of $\dblRel$ at $(R,S,T)$ is the identity inclusion%
    %--- Begin Footnote ---%
    \footnote{%
        This is justified by \ChapterRef{\ChapterConstructionsWithRelations, \cref{constructions-with-relations:properties-of-composition-of-relations-associativity} of \cref{constructions-with-relations:properties-of-composition-of-relations}}{\cref{properties-of-composition-of-relations-associativity} of \cref{properties-of-composition-of-relations}}.
        \par\vspace*{\TCBBoxCorrection}
    }%
    %---  End Footnote  ---%
    \begin{webcompile}
        (T\procirc S)\procirc R%
        =%
        T\procirc(S\procirc R)%
        \quad
        \begin{tikzcd}[row sep={5.0*\the\DL,between origins}, column sep={9.0*\the\DL,between origins}, background color=backgroundColor, ampersand replacement=\&]
            A\times B
            \arrow[r,"(T\procirc S)\procirc R"]
            \arrow[d,Equals]
            \&
            \TV
            \arrow[d,"\id_{\TV}"]
            \\
            A\times B
            \arrow[r,"T\procirc(S\procirc R)"']
            \&
            \TV\mrp{.}
            % 2-Arrows
            \arrow[from=1-2,to=2-1,"\scalebox{1.5}{$=$}"{sloped,description},phantom,shorten <= 0.5*\the\DL,shorten >= 0.625*\the\DL,Rightarrow,pos=0.5]%
        \end{tikzcd}
    \end{webcompile}%
\end{definition}
\subsubsection{The Left Unitors}\label{subsubsection-the-double-category-of-relations-the-left-unitors}
\begin{definition}{The Left Unitors of $\dblRel$}{the-left-unitors-of-dblrel}%
    For each horizontal morphism $R\colon A\rightproarrow B$ of $\dblRel$, the component
    \begin{webcompile}
        \LUnitor^{\dblRel}_{R}
        \colon
        \Unit_{B}\doublecirc R
        \Longrightisoarrow
        R,
        \qquad
        \begin{tikzcd}[row sep={5.0*\the\DL,between origins}, column sep={5.0*\the\DL,between origins}, background color=backgroundColor, ampersand replacement=\&]
            A
            \arrow[r, mid vert,"R"{name=1}]
            \arrow[d, "\id_{A}"']
            \&
            B
            \arrow[r, mid vert,"\Unit_{B}"]
            \&
            B
            \arrow[d, "\id_{B}"]
            \\
            A
            \arrow[rr, mid vert,"R"'{name=2}]
            \&
            \&
            B
            % 2-Arrows
            \arrow[from=1-2,to=2,"\LUnitor^{\dblRel}_{R}"'{pos=0.4},shorten <= 0.25*\the\DL,shorten >= 0.75*\the\DL,Rightarrow]
        \end{tikzcd}
    \end{webcompile}%
    of the left unitor of $\dblRel$ at $R$ is the identity inclusion%
    %--- Begin Footnote ---%
    \footnote{%
        This is justified by \ChapterRef{\ChapterConstructionsWithRelations, \cref{constructions-with-relations:properties-of-composition-of-relations-unitality} of \cref{constructions-with-relations:properties-of-composition-of-relations}}{\cref{properties-of-composition-of-relations-unitality} of \cref{properties-of-composition-of-relations}}.
        \par\vspace*{\TCBBoxCorrection}
    }%
    %---  End Footnote  ---%
    \begin{webcompile}
        R%
        =%
        \chi_{B}\procirc R,%
        \qquad
        \begin{tikzcd}[row sep={5.0*\the\DL,between origins}, column sep={8.0*\the\DL,between origins}, background color=backgroundColor, ampersand replacement=\&]
            A\times B
            \arrow[r,"\chi_{B}\procirc R"]
            \arrow[d,Equals]
            \&
            \TV
            \arrow[d,"\id_{\TV}"]
            \\
            A\times B
            \arrow[r,"R"']
            \&
            \TV\mrp{.}
            % 2-Arrows
            \arrow[from=1-2,to=2-1,"\scalebox{1.5}{$=$}"{sloped,description},phantom,shorten <= 0.5*\the\DL,shorten >= 0.625*\the\DL,Rightarrow,pos=0.5]%
        \end{tikzcd}
    \end{webcompile}%
\end{definition}
\subsubsection{The Right Unitors}\label{subsubsection-the-double-category-of-relations-the-right-unitors}
\begin{definition}{The Right Unitors of $\dblRel$}{the-right-unitors-of-dblrel}%
    For each horizontal morphism $R\colon A\rightproarrow B$ of $\dblRel$, the component
    \begin{webcompile}
        \RUnitor^{\dblRel}_{R}
        \colon
        R\doublecirc\Unit_{A}
        \Longrightisoarrow
        R,
        \qquad
        \begin{tikzcd}[row sep={5.0*\the\DL,between origins}, column sep={5.0*\the\DL,between origins}, background color=backgroundColor, ampersand replacement=\&]
            A
            \arrow[r, mid vert,"\Unit_{A}"{name=1}]
            \arrow[d, "\id_{A}"']
            \&
            A
            \arrow[r, mid vert,"R"]
            \&
            B
            \arrow[d, "\id_{B}"]
            \\
            A
            \arrow[rr, mid vert,"R"'{name=2}]
            \&
            \&
            B
            % 2-Arrows
            \arrow[from=1-2,to=2,"\RUnitor^{\dblRel}_{R}"'{pos=0.4},shorten <= 0.25*\the\DL,shorten >= 0.75*\the\DL,Rightarrow]
        \end{tikzcd}
    \end{webcompile}%
    of the right unitor of $\dblRel$ at $R$ is the identity inclusion%
    %--- Begin Footnote ---%
    \footnote{%
        This is justified by \ChapterRef{\ChapterConstructionsWithRelations, \cref{constructions-with-relations:properties-of-composition-of-relations-unitality} of \cref{constructions-with-relations:properties-of-composition-of-relations}}{\cref{properties-of-composition-of-relations-unitality} of \cref{properties-of-composition-of-relations}}.
        \par\vspace*{\TCBBoxCorrection}
    }%
    %---  End Footnote  ---%
    \begin{webcompile}
        R%
        =%
        R\procirc\chi_{A},%
        \qquad%
        \begin{tikzcd}[row sep={5.0*\the\DL,between origins}, column sep={8.0*\the\DL,between origins}, background color=backgroundColor, ampersand replacement=\&]
            A\times B
            \arrow[r,"{R\procirc\chi_{A}}"]
            \arrow[d,Equals]
            \&
            \TV
            \arrow[d,"\id_{\TV}"]
            \\
            A\times B
            \arrow[r,"R"']
            \&
            \TV\mrp{.}
            % 2-Arrows
            \arrow[from=1-2,to=2-1,"\scalebox{1.5}{$=$}"{sloped,description},phantom,shorten <= 0.5*\the\DL,shorten >= 0.625*\the\DL,Rightarrow,pos=0.5]%
        \end{tikzcd}
    \end{webcompile}%
\end{definition}
\section{Properties of the $2$-Category of Relations}\label{section-properties-of-the-2-category-of-relations}
\subsection{Self-Duality}\label{subsection-self-duality-of-rel}
\begin{proposition}{Self-Duality for the (2-)Category of Relations}{self-duality-for-the-2-category-of-relations}%
    The ($2$-)category of relations is self-dual:
    \begin{enumerate}
        \item\label{self-duality-for-the-2-category-of-relations-1}\SloganFont{Self-Duality \rmI. }We have an isomorphism
            \[
                \Rel^{\op}%
                \eqcong%
                \Rel%
            \]%
            of categories.
        \item\label{self-duality-for-the-2-category-of-relations-2}\SloganFont{Self-Duality \rmII. }We have a $2$-isomorphism
            \[
                \sfbfRel^{\op}%
                \eqcong%
                \sfbfRel%
            \]%
            of $2$-categories.
    \end{enumerate}
\end{proposition}
\begin{Proof}{Proof of \cref{self-duality-for-the-2-category-of-relations}}%
    \FirstProofBox{\cref{self-duality-for-the-2-category-of-relations-1}: Self-Duality \rmI}%
    We claim that the functor
    \[
        F%
        \colon%
        \sfRel^{\op}%
        \to%
        \sfRel%
    \]%
    given by the identity on objects and by $R\mapsto R^{\dagger}$ on morphisms is an isomorphism of categories.

    \indent By \ChapterRef{\ChapterCategories, \cref{categories:properties-of-isomorphisms-of-categories-characterisations} of \cref{categories:properties-of-isomorphisms-of-categories}}{\cref{properties-of-isomorphisms-of-categories-characterisations} of \cref{properties-of-isomorphisms-of-categories}}, it suffices to show that $F$ is bijective on objects (which is clear) and fully faithful. Indeed, the map
    \[
        (-)^{\dagger}%
        \colon%
        \Rel(A,B)%
        \to%
        \Rel(B,A)%
    \]%
    defined by the assignment $R\mapsto R^{\dagger}$ is a bijection by \ChapterRef{\ChapterConstructionsWithRelations, \cref{constructions-with-relations:properties-of-inverses-of-relations-invertibility} of \cref{constructions-with-relations:properties-of-inverses-of-relations}}{\cref{properties-of-inverses-of-relations-invertibility} of \cref{properties-of-inverses-of-relations}}, showing $F$ to be fully faithful.

    \ProofBox{\cref{self-duality-for-the-2-category-of-relations-2}: Self-Duality \rmII}%
    We claim that the $2$-functor
    \[
        F%
        \colon%
        \sfRel^{\op}%
        \to%
        \sfRel%
    \]%
    given by the identity on objects, by $R\mapsto R^{\dagger}$ on morphisms, and by preserving inclusions on $2$-morphisms via \ChapterRef{\ChapterConstructionsWithRelations, \cref{constructions-with-relations:properties-of-inverses-of-relations-functoriality} of \cref{constructions-with-relations:properties-of-inverses-of-relations}}{\cref{properties-of-inverses-of-relations-functoriality} of \cref{properties-of-inverses-of-relations}}, is an isomorphism of categories.

    \indent By \ChapterRef{\ChapterBicategories, \cref{bicategories:properties-of-2-isomorphisms-of-2-categories-characterisations} of \cref{bicategories:properties-of-2-isomorphisms-of-2-categories}}{\cref{properties-of-2-isomorphisms-of-2-categories-characterisations} of \cref{properties-of-2-isomorphisms-of-2-categories}}, it suffices to show that $F$ is:
    \begin{itemize}
        \item Bijective on objects, which is clear.
        \item Bijective on $1$-morphisms, which was shown in \cref{self-duality-for-the-2-category-of-relations-1}.
        \item Bijective on $2$-morphisms, which follows from \ChapterRef{\ChapterConstructionsWithRelations, \cref{constructions-with-relations:properties-of-inverses-of-relations-functoriality} of \cref{constructions-with-relations:properties-of-inverses-of-relations}}{\cref{properties-of-inverses-of-relations-functoriality} of \cref{properties-of-inverses-of-relations}}.
    \end{itemize}
    Thus $F$ is indeed a $2$-isomorphism of categories.
\end{Proof}
\subsection{Isomorphisms and Equivalences in $\sfbfRel$}\label{subsection-isomorphisms-and-equivalences-in-rel}
Let $R\colon A\rightproarrow B$ be a relation from $A$ to $B$.
\begin{proposition}{Isomorphisms and Equivalences in $\sfbfRel$}{isomorphisms-and-equivalences-in-rel}%
    The following conditions are equivalent:
    \begin{enumerate}
        \item\label{isomorphisms-and-equivalences-in-rel-1}The relation $R\colon A\rightproarrow B$ is an equivalence in $\sfbfRel$, i.e.:
            \begin{itemize}
                \itemstar There exists a relation $R^{-1}\colon B\rightproarrow A$ from $B$ to $A$ together with isomorphisms
                    \begin{align*}
                        R^{-1}\procirc R &\cong \chi_{A},\\
                        R\procirc R^{-1} &\cong \chi_{B}.
                    \end{align*}
            \end{itemize}
        \item\label{isomorphisms-and-equivalences-in-rel-2}The relation $R\colon A\rightproarrow B$ is an isomorphism in $\Rel$, i.e.:
            \begin{itemize}
                \itemstar There exists a relation $R^{-1}\colon B\rightproarrow A$ from $B$ to $A$ such that we have
                    \begin{align*}
                        R^{-1}\procirc R &= \chi_{A},\\
                        R\procirc R^{-1} &= \chi_{B}.
                    \end{align*}
            \end{itemize}
        \item\label{isomorphisms-and-equivalences-in-rel-3}There exists a bijection $f\colon A\isorightarrow B$ with $R=\Gr(f)$.
    \end{enumerate}
\end{proposition}
\begin{Proof}{Proof of \cref{isomorphisms-and-equivalences-in-rel}}%
    We claim that \cref{isomorphisms-and-equivalences-in-rel-1,isomorphisms-and-equivalences-in-rel-2,isomorphisms-and-equivalences-in-rel-3} are indeed equivalent:
    \begin{itemize}
        \item\SloganFont{\cref{isomorphisms-and-equivalences-in-rel-1}$\iff$\cref{isomorphisms-and-equivalences-in-rel-2}: }This follows from the fact that $\sfbfRel$ is locally posetal, so that natural isomorphisms and equalities of $1$-morphisms in $\sfbfRel$ coincide.
        \item\SloganFont{\cref{isomorphisms-and-equivalences-in-rel-2}$\implies$\cref{isomorphisms-and-equivalences-in-rel-3}: }The equalities in \cref{isomorphisms-and-equivalences-in-rel-2} imply $R\dashv R^{-1}$, and thus by \cref{adjunctions-in-rel}, there exists a function $f_{R}\colon A\to B$ associated to $R$, where, for each $a\in A$, the image $f_{R}(a)$ of $a$ by $f_{R}$ is the unique element of $R(a)$, which implies $R=\Gr(f_{R})$ in particular. Furthermore, we have $R^{-1}=f^{-1}_{R}$ (as in \ChapterRef{\ChapterConstructionsWithRelations, \cref{constructions-with-relations:the-inverse-of-a-function}}{\cref{the-inverse-of-a-function}}). The conditions from \cref{isomorphisms-and-equivalences-in-rel-2} then become the following:
            \begin{align*}
                f^{-1}_{R}\procirc f_{R} = \chi_{A},\\
                f_{R}\procirc f^{-1}_{R} = \chi_{B}.
            \end{align*}
            All that is left is to show then is that $f_{R}$ is a bijection:
            \begin{itemize}
                \item\SloganFont{The Function $f_{R}$ Is Injective. }Let $a,b\in A$ and suppose that $f_{R}(a)=f_{R}(b)$. Since $a\sim_{R}f_{R}(a)$ and $f_{R}(a)=f_{R}(b)\sim_{R^{-1}}b$, the condition $f^{-1}_{R}\procirc f_{R}=\chi_{A}$ implies that $a=b$, showing $f_{R}$ to be injective.
                \item\SloganFont{The Function $f_{R}$ Is Surjective. }Let $b\in B$. Applying the condition $f_{R}\procirc f^{-1}_{R}=\chi_{B}$ to $(b,b)$, it follows that there exists some $a\in A$ such that $f^{-1}_{R}(b)=a$ and $f_{R}(a)=b$. This shows $f_{R}$ to be surjective.
            \end{itemize}
        \item\SloganFont{\cref{isomorphisms-and-equivalences-in-rel-3}$\implies$\cref{isomorphisms-and-equivalences-in-rel-2}: }By \ChapterRef{\ChapterConstructionsWithRelations, \cref{constructions-with-relations:properties-of-graphs-of-functions-adjointness-inside-sfbfrel} of \cref{constructions-with-relations:properties-of-graphs-of-functions}}{\cref{properties-of-graphs-of-functions-adjointness-inside-sfbfrel} of \cref{properties-of-graphs-of-functions}}, we have an adjunction $\Gr(f)\dashv f^{-1}$, giving inclusions
            \begin{gather*}
                \chi_{A}              \subset f^{-1}\procirc\Gr(f),\\
                \Gr(f)\procirc f^{-1} \subset \chi_{B}.
            \end{gather*}
            We claim the reverse inclusions are also true:
            \begin{itemize}
                \item\SloganFont{$f^{-1}\procirc\Gr(f)\subset\chi_{A}$: }This is equivalent to the statement that if $f(a)=b$ and $f^{-1}(b)=a'$, then $a=a'$, which follows from the injectivity of $f$.
                \item\SloganFont{$\chi_{B}\subset\Gr(f)\procirc f^{-1}$: }This is equivalent to the statement that given $b\in B$ there exists some $a\in A$ such that $f^{-1}(b)=a$ and $f(a)=b$, which follows from the surjectivity of $f$.
            \end{itemize}
    \end{itemize}
    This finishes the proof.
\end{Proof}
\subsection{Adjunctions in $\sfbfRel$}\label{subsection-adjunctions-in-rel}
Let $A$ and $B$ be sets.
\begin{proposition}{Adjunctions in $\sfbfRel$}{adjunctions-in-rel}%
    We have a natural bijection
    \[
        \{
            \begin{gathered}
                \text{Adjunctions in $\sfbfRel$}\\
                \text{from $A$ to $B$}
            \end{gathered}
        \}%
        \cong
        \{
            \begin{gathered}
                \text{Functions}\\
                \text{from $A$ to $B$}
            \end{gathered}
        \},
    \]%
    with every adjunction in $\sfbfRel$ being of the form $\Gr(f)\dashv f^{-1}$ for some function $f$.
\end{proposition}
\begin{Proof}{Proof of \cref{adjunctions-in-rel}}%
    We proceed step by step:
    \begin{enumerate}
        \item\label{proof-of-adjunctions-in-rel-1}\SloganFont{From Adjunctions in $\sfbfRel$ to Functions. }An adjunction in $\sfbfRel$ from $A$ to $B$ consists of a pair of relations
            \begin{align*}
                R &\colon A\rightproarrow B,\\
                S &\colon B\rightproarrow A,
            \end{align*}
            together with inclusions
            \begin{align*}
                \chi_{A}    &\subset S\procirc R,\\
                R\procirc S &\subset \chi_{B}.
            \end{align*}
            We claim that these conditions imply that $R$ is total and functional, i.e.\ that $R(a)$ is a singleton for each $a\in A$:
            \begin{enumerate}
                \item\label{proof-of-adjunctions-in-rel-1a}\SloganFont{$R(a)$ Has an Element. }Given $a\in A$, since $\chi_{A}\subset S\procirc R$, we must have $\{a\}\subset S(R(a))$, implying that there exists some $b\in B$ such that $a\sim_{R}b$ and $b\sim_{S}a$, and thus $R(a)\neq\emptyset$, as $b\in R(a)$.
                \item\label{proof-of-adjunctions-in-rel-1b}\SloganFont{$R(a)$ Has No More Than One Element. }Suppose that we have $a\sim_{R}b$ and $a\sim_{R}b'$ for $b,b'\in B$. We claim that $b=b'$:
                    \begin{enumerate}
                        \item\label{proof-of-adjunctions-in-rel-1bi}Since $\chi_{A}\subset S\procirc R$, there exists some $k\in B$ such that $a\sim_{R}k$ and $k\sim_{S}a$.
                        \item\label{proof-of-adjunctions-in-rel-1bii}Since $R\procirc S\subset\chi_{B}$, if $b''\sim_{S}a'$ and $a'\sim_{R}b'''$, then $b''=b'''$.
                        \item\label{proof-of-adjunctions-in-rel-1biii}Applying the above to $b''=k$, $b'''=b$, and $a'=a$, since $k\sim_{S}a$ and $a\sim_{R}b'$, we have $k=b$.
                        \item\label{proof-of-adjunctions-in-rel-1biv}Similarly $k=b'$.
                        \item\label{proof-of-adjunctions-in-rel-1bv}Thus $b=b'$.
                    \end{enumerate}
            \end{enumerate}
            Together, the above two items show $R(a)$ to be a singleton, being thus given by $\Gr(f)$ for some function $f\colon A\to B$, which gives a map
            \[
                \{
                    \begin{gathered}
                        \text{Adjunctions in $\sfbfRel$}\\
                        \text{from $A$ to $B$}
                    \end{gathered}
                \}%
                \to
                \{
                    \begin{gathered}
                        \text{Functions}\\
                        \text{from $A$ to $B$}
                    \end{gathered}
                \}.
            \]%
            Moreover, by uniqueness of adjoints (\ChapterRef{\ChapterInternalAdjunctions, \cref{internal-adjunctions:properties-of-internal-adjunctions-uniqueness-of-adjoints} of \cref{internal-adjunctions:properties-of-internal-adjunctions}}{\cref{properties-of-internal-adjunctions-uniqueness-of-adjoints} of \cref{properties-of-internal-adjunctions}}), this implies also that $S=f^{-1}$.
        \item\label{proof-of-adjunctions-in-rel-2}\SloganFont{From Functions to Adjunctions in $\sfbfRel$. }By \ChapterRef{\ChapterConstructionsWithRelations, \cref{constructions-with-relations:properties-of-graphs-of-functions-adjointness-inside-sfbfrel} of \cref{constructions-with-relations:properties-of-graphs-of-functions}}{\cref{properties-of-graphs-of-functions-adjointness-inside-sfbfrel} of \cref{properties-of-graphs-of-functions}}, every function $f\colon A\to B$ gives rise to an adjunction $\Gr(f)\dashv f^{-1}$ in $\Rel$, giving a map
            \[
                \{
                    \begin{gathered}
                        \text{Functions}\\
                        \text{from $A$ to $B$}
                    \end{gathered}
                \}
                \to
                \{
                    \begin{gathered}
                        \text{Adjunctions in $\sfbfRel$}\\
                        \text{from $A$ to $B$}
                    \end{gathered}
                \}.%
            \]%
        \item\label{proof-of-adjunctions-in-rel-3}\SloganFont{Invertibility: From Functions to Adjunctions Back to Functions. }We need to show that starting with a function $f\colon A\to B$, passing to $\Gr(f)\dashv f^{-1}$, and then passing again to a function gives $f$ again. This is clear however, since we have $a\sim_{\Gr(f)}b$ iff $f(a)=b$.
        \item\label{proof-of-adjunctions-in-rel-4}\SloganFont{Invertibility: From Adjunctions to Functions Back to Adjunctions. }We need to show that, given an adjunction $R\dashv S$ in $\sfbfRel$ giving rise to a function $f_{R,S}\colon A\to B$, we have
            \begin{align*}
                \Gr(f_{R,S}) &= R,\\
                f^{-1}_{R,S} &= S.
            \end{align*}
            We check these explicitly:
            \begin{itemize}
                \item\SloganFont{$\Gr(f_{R,S})=R$. }We have
                    \begin{align*}
                        \Gr(f_{R,S}) &\defeq \{(a,f_{R,S}(a))\in A\times B\ \middle|\ a\in A\}\\
                                     &\defeq \{(a,R(a))\in A\times B\ \middle|\ a\in A\}\\
                                     &=      R.
                    \end{align*}
                \item\SloganFont{$f^{-1}_{R,S}=S$. }We first claim that, given $a\in A$ and $b\in B$, the following conditions are equivalent:
                    \begin{itemize}
                        \item We have $a\sim_{R}b$.
                        \item We have $b\sim_{S}a$.
                    \end{itemize}
                    Indeed:
                    \begin{itemize}
                        \item\SloganFont{If $a\sim_{R}b$, then $b\sim_{S}a$: }Since $\chi_{A}\subset S\procirc R$, there exists $k\in B$ such that $a\sim_{R}k$ and $k\sim_{S}a$, but since $a\sim_{R}b$ and $R$ is functional, we have $k=b$ and thus $b\sim_{S}a$.
                        \item\SloganFont{If $b\sim_{S}a$, then $a\sim_{R}b$: }First note that since $R$ is total we have $a\sim_{R}b'$ for some $b'\in B$. Now, since $R\procirc S\subset\chi_{B}$, $b\sim_{S}a$, and $a\sim_{R}b'$, we have $b=b'$, and thus $a\sim_{R}b$.
                    \end{itemize}
                    Having show this, we now have
                    \begin{align*}
                        f^{-1}_{R,S}(b) &\defeq \{a\in A\ \middle|\ f_{R,S}(a)=b\}\\
                                        &\defeq \{a\in A\ \middle|\ a\sim_{R}b\}\\
                                        &=      \{a\in A\ \middle|\ b\sim_{S}a\}\\
                                        &\defeq S(b).
                    \end{align*}
                    for each $b\in B$, showing $f^{-1}_{R,S}=S$.
            \end{itemize}
    \end{enumerate}
    This finishes the proof.
\end{Proof}
\subsection{Monads in $\sfbfRel$}\label{subsection-monads-in-rel}
Let $A$ be a set.
\begin{proposition}{Monads in $\sfbfRel$}{monads-in-rel}%
    We have a natural identification%
    %--- Begin Footnote ---%
    \footnote{%
        See also \cref{section-relative-preorders} for an extension of this correspondence to \say{relative monads in $\sfbfRel$}.
        \par\vspace*{\TCBBoxCorrection}
    }%
    %---  End Footnote  ---%
    \[
        \{
            \begin{gathered}
                \text{Monads in}\\
                \text{$\sfbfRel$ on $A$}
            \end{gathered}
        \}
        \cong
        \{\text{Preorders on $A$}\}.
    \]%
\end{proposition}
\begin{Proof}{Proof of \cref{monads-in-rel}}%
    A monad in $\sfbfRel$ on $A$ consists of a relation $R\colon A\rightproarrow A$ together with maps
    \begin{align*}
        \mu_{R}  &\colon R\procirc R \subset R,\\
        \eta_{R} &\colon \chi_{A}    \subset R
    \end{align*}
    making the diagrams
    \begin{webcompile}
        \begin{tikzcd}[row sep={5.0*\the\DL,between origins}, column sep={7.0*\the\DL,between origins}, background color=backgroundColor, ampersand replacement=\&,outer sep=0.2*\the\DL]
            \chi_{A}\procirc R
            \arrow[r,"\eta_{R}\procirc\id_{R}",mid vert]
            \arrow[rd,"\LUnitor^{\eRel(A,B)}_{R}"'{pos=0.55},Equals]
            \&
            R\procirc R
            \arrow[d,"\mu_{R}",mid vert]
            \\
            \&
            R
        \end{tikzcd}
        \begin{tikzcd}[row sep={0*\the\DL,between origins}, column sep={0*\the\DL,between origins}, background color=backgroundColor, ampersand replacement=\&]
            \&[0.30901699437\TwoCm]
            \&[0.5\TwoCm]
            R\procirc(R\procirc R)
            \&[0.5\TwoCm]
            \&[0.30901699437\TwoCm]
            \\[0.58778525229\TwoCm]
            (R\procirc R)\procirc R
            \&[0.30901699437\TwoCm]
            \&[0.5\TwoCm]
            \&[0.5\TwoCm]
            \&[0.30901699437\TwoCm]
            R\procirc R
            \\[0.95105651629\TwoCm]
            \&[0.30901699437\TwoCm]
            R\procirc R
            \&[0.5\TwoCm]
            \&[0.5\TwoCm]
            R
            \&[0.30901699437\TwoCm]
            % 1-Arrows
            % Left Boundary
            \arrow[from=2-1,to=1-3,"\alpha^{\eRel(A,B)}_{R,R,R}"{pos=0.35},Equals]%
            \arrow[from=1-3,to=2-5,"\id_{R}\procirc\mu_{R}"{pos=0.6},mid vert]%
            \arrow[from=2-5,to=3-4,"\mu_{R}"{pos=0.35},mid vert]%
            % Right Boundary
            \arrow[from=2-1,to=3-2,"\mu_{R}\procirc\id_{R}"'{pos=0.3},mid vert]%
            \arrow[from=3-2,to=3-4,"\mu_{R}"',mid vert]%
        \end{tikzcd}
        \begin{tikzcd}[row sep={5.0*\the\DL,between origins}, column sep={7.0*\the\DL,between origins}, background color=backgroundColor, ampersand replacement=\&,outer sep=0.2*\the\DL]
            R\procirc\chi_{A}
            \arrow[r,"\id_{R}\procirc\eta_{R}",mid vert]
            \arrow[rd,"\RUnitor^{\eRel(A,B)}_{R}"'{pos=0.55},Equals]
            \&
            R\procirc R
            \arrow[d,"\mu_{R}",mid vert]
            \\
            \&
            R
        \end{tikzcd}%
    \end{webcompile}%%
    commute. However, since all morphisms involved are inclusions, the commutativity of the above diagrams is automatic (\ChapterRef{\ChapterCategories, \cref{categories:properties-of-posetal-categories-automatic-commutativity-of-diagrams} of \cref{categories:properties-of-posetal-categories}}{\cref{categories:properties-of-posetal-categories-automatic-commutativity-of-diagrams} of \cref{categories:properties-of-posetal-categories}}), and hence all that is left is the data of the two maps $\mu_{R}$ and $\eta_{R}$, which correspond respectively to the following conditions:
    \begin{enumerate}
        \item\label{proof-of-monads-in-rel-1}For each $a,b,c\in A$, if $a\sim_{R}b$ and $b\sim_{R}c$, then $a\sim_{R}c$.
        \item\label{proof-of-monads-in-rel-2}For each $a\in A$, we have $a\sim_{R}a$.
    \end{enumerate}
    These are exactly the requirements for $R$ to be a preorder (\ChapterRef{\ChapterPreordersAndPartialOrders, \cref{preorders-and-partial-orders:preorders}}{\cref{preorders}}). Conversely, any preorder $\preceq$ gives rise to a pair of maps $\mu_{\preceq}$ and $\eta_{\preceq}$, forming a monad on $A$.
\end{Proof}
\subsection{Comonads in $\sfbfRel$}\label{subsection-comonads-in-rel}
Let $A$ be a set.
\begin{proposition}{Comonads in $\sfbfRel$}{comonads-in-rel}%
    We have a natural identification
    \[
        \{
            \begin{gathered}
                \text{Comonads in}\\
                \text{$\sfbfRel$ on $A$}
            \end{gathered}
        \}
        \cong
        \{\text{Subsets of $A$}\}.
    \]%
\end{proposition}
\begin{Proof}{Proof of \cref{comonads-in-rel}}%
    A comonad in $\sfbfRel$ on $A$ consists of a relation $R\colon A\rightproarrow A$ together with maps
    \begin{align*}
        \Delta_{R}   &\colon R \subset R\procirc R,\\
        \epsilon_{R} &\colon R \subset \chi_{A}
    \end{align*}
    making the diagrams
    \begingroup\footnotesize% PDF ONLY, so we use \begingroup\footnotesize instead of \begin{envfootnotesize}
    \begin{webcompile}
        \begin{tikzcd}[row sep={5.0*\the\DL,between origins}, column sep={7.0*\the\DL,between origins}, background color=backgroundColor, ampersand replacement=\&,outer sep=0.2*\the\DL]
            R
            \arrow[r,"\Delta_{R}",mid vert]
            \arrow[rd,"\LUnitor^{\eRel(A,B),-1}_{R}"'{pos=0.55},Equals]
            \&
            R\procirc R
            \arrow[d,"\epsilon_{R}\procirc\id_{R}",mid vert]
            \\
            \&
            \chi_{A}\procirc R
        \end{tikzcd}
        \begin{tikzcd}[row sep={0*\the\DL,between origins}, column sep={0*\the\DL,between origins}, background color=backgroundColor, ampersand replacement=\&]
            \&[0.30901699437\TwoCmPlusAQuarter]
            \&[0.5\TwoCmPlusAQuarter]
            R\procirc R
            \&[0.5\TwoCmPlusAQuarter]
            \&[0.30901699437\TwoCmPlusAQuarter]
            \\[0.58778525229\TwoCmPlusAQuarter]
            R
            \&[0.30901699437\TwoCmPlusAQuarter]
            \&[0.5\TwoCmPlusAQuarter]
            \&[0.5\TwoCmPlusAQuarter]
            \&[0.30901699437\TwoCmPlusAQuarter]
            R\procirc(R\procirc R)
            \\[0.95105651629\TwoCmPlusAQuarter]
            \&[0.30901699437\TwoCmPlusAQuarter]
            R\procirc R
            \&[0.5\TwoCmPlusAQuarter]
            \&[0.5\TwoCmPlusAQuarter]
            (R\procirc R)\procirc R
            \&[0.30901699437\TwoCmPlusAQuarter]
            % 1-Arrows
            % Left Boundary
            \arrow[from=2-1,to=1-3,"\Delta_{R}"{pos=0.5},mid vert]%
            \arrow[from=1-3,to=2-5,"\id_{R}\procirc\Delta_{R}"{pos=0.35},mid vert]%
            \arrow[from=2-5,to=3-4,"\alpha^{\eRel(A,B),-1}_{R,R,R}"{pos=0.35},Equals]%
            % Right Boundary
            \arrow[from=2-1,to=3-2,"\Delta_{R}"'{pos=0.375},mid vert]%
            \arrow[from=3-2,to=3-4,"\Delta_{R}\procirc\id_{R}"',mid vert]%
        \end{tikzcd}
        \begin{tikzcd}[row sep={5.0*\the\DL,between origins}, column sep={7.0*\the\DL,between origins}, background color=backgroundColor, ampersand replacement=\&,outer sep=0.2*\the\DL]
            R
            \arrow[r,"\Delta_{R}",mid vert]
            \arrow[rd,"\RUnitor^{\eRel(A,B),-1}_{R}"'{pos=0.55},Equals]
            \&
            R\procirc R
            \arrow[d,"\id_{R}\procirc\epsilon_{R}",mid vert]
            \\
            \&
            R\procirc\chi_{A}
        \end{tikzcd}%
    \end{webcompile}%
    \endgroup
    commute. However, since all morphisms involved are inclusions, the commutativity of the above diagrams is automatic (\ChapterRef{\ChapterCategories, \cref{categories:properties-of-posetal-categories-automatic-commutativity-of-diagrams} of \cref{categories:properties-of-posetal-categories}}{\cref{categories:properties-of-posetal-categories-automatic-commutativity-of-diagrams} of \cref{categories:properties-of-posetal-categories}}), and hence all that is left is the data of the two maps $\Delta_{R}$ and $\epsilon_{R}$, which correspond respectively to the following conditions:
    \begin{enumerate}
        \item\label{proof-of-comonads-in-rel-1}For each $a,b\in A$, if $a\sim_{R}b$, then there exists some $k\in A$ such that $a\sim_{R}k$ and $k\sim_{R}b$.
        \item\label{proof-of-comonads-in-rel-2}For each $a,b\in A$, if $a\sim_{R}b$, then $a=b$.
    \end{enumerate}
    Taking $k=b$ in the first condition above shows it to be trivially satisfied, while the second condition implies $R\subset\Delta_{A}$, i.e.\ $R$ must be a subset of $A$. Conversely, any subset $U$ of $A$ satisfies $U\subset\Delta_{A}$, defining a comonad as above.
\end{Proof}
\subsection{Co/Monoids in $\sfbfRel$}\label{subsection-co-monoids-in-rel}
\begin{remark}{Co/Monoids in $\sfbfRel$}{co-monoids-in-rel}%
    The monoids in $\sfbfRel$ with respect to the Cartesian monoidal structure of \cref{the-closed-symmetric-monoidal-category-of-relations} are called \emph{hypermonoids}, and their theory is explored in \ChapterHypermonoids. Similarly, the comonoids in $\sfbfRel$ are called \emph{hypercomonoids}, and they are defined and studied in \ChapterHypergroups.
\end{remark}
\subsection{Monomorphisms in $\sfRel$}\label{subsection-monomorphisms-in-rel}
In this section we characterise the epimorphisms in the category $\sfRel$, following \ChapterRef{\ChapterTypesOfMorphismsInCategories, \cref{types-of-morphisms-in-categories:section-monomorphisms}}{\cref{section-monomorphisms}}.
\begin{proposition}{Characterisations of Monomorphisms in $\sfRel$}{characterisations-of-monomorphisms-in-rel}%
    Let $R\colon A\rightproarrow B$ be a relation. The following conditions are equivalent:
    \begin{enumerate}
        \item\label{characterisations-of-monomorphisms-in-rel-1}The relation $R$ is a monomorphism in $\sfRel$.
        \item\label{characterisations-of-monomorphisms-in-rel-2}The direct image function%
            \[
                R_{*}%
                \colon%
                \mathcal{P}(A)%
                \to%
                \mathcal{P}(B)%
            \]%
            associated to $R$ is injective.
        \item\label{characterisations-of-monomorphisms-in-rel-3}The direct image with compact support function
            \[
                R_{!}%
                \colon%
                \mathcal{P}(A)%
                \to%
                \mathcal{P}(B)%
            \]%
            associated to $R$ is injective.
    \end{enumerate}
    Moreover, if $R$ is a monomorphism, then it satisfies the following condition, and the converse holds if $R$ is total:
    \begin{itemize}
        \itemstar For each $a,a'\in A$, if there exists some $b\in B$ such that
            \begin{align*}%
                a\sim_{R}b,\\%
                a'\sim_{R}b,%
            \end{align*}%
            then $a=a'$.
    \end{itemize}
\end{proposition}
\begin{Proof}{Proof of \cref{characterisations-of-monomorphisms-in-rel}}%
    Firstly note that \cref{characterisations-of-monomorphisms-in-rel-2,characterisations-of-monomorphisms-in-rel-3} are equivalent by \ChapterRef{\ChapterConstructionsWithRelations, \cref{constructions-with-relations:properties-of-direct-image-functions-associated-to-relations-relation-to-direct-images-with-compact-support} of \cref{constructions-with-relations:properties-of-direct-image-functions-associated-to-relations}}{\cref{properties-of-direct-image-functions-associated-to-relations-relation-to-direct-images-with-compact-support} of \cref{properties-of-direct-image-functions-associated-to-relations}}. We then claim that \cref{characterisations-of-monomorphisms-in-rel-1,characterisations-of-monomorphisms-in-rel-2} are also equivalent:
    \begin{itemize}
        \item\SloganFont{\cref{characterisations-of-monomorphisms-in-rel-1}$\implies$\cref{characterisations-of-monomorphisms-in-rel-2}: }Let $U,V\in\mathcal{P}(A)$ and consider the diagram
            \[
                \begin{tikzcd}[row sep={4.0*\the\DL,between origins}, column sep={4.0*\the\DL,between origins}, background color=backgroundColor, ampersand replacement=\&]
                    \pt
                    \arrow[r, "U", shift left=0.8, mid vert]
                    \arrow[r, "V"', shift right=0.8, mid vert]
                    \&
                    A
                    \arrow[r, "R",mid vert]
                    \&
                    B\mrp{.}
                \end{tikzcd}
            \]%
            By \ChapterRef{\ChapterConstructionsWithRelations, \cref{constructions-with-relations:unwinding-the-direct-image-function-associated-to-a-relation}}{\cref{unwinding-the-direct-image-function-associated-to-a-relation}}, we have
            \begin{align*}
                R_{*}(U) &= R\procirc U,\\
                R_{*}(V) &= R\procirc V.
            \end{align*}
            Now, if $R\procirc U=R\procirc V$, i.e.\ $R_{*}(U)=R_{*}(V)$, then $U=V$ since $R$ is assumed to be a monomorphism, showing $R_{*}$ to be injective.
        \item\SloganFont{\cref{characterisations-of-monomorphisms-in-rel-2}$\implies$\cref{characterisations-of-monomorphisms-in-rel-1}: }Conversely, suppose that $R_{*}$ is injective, consider the diagram
            \[
                \begin{tikzcd}[row sep={4.0*\the\DL,between origins}, column sep={4.0*\the\DL,between origins}, background color=backgroundColor, ampersand replacement=\&]
                    X
                    \arrow[r, "S", shift left=0.8, mid vert]
                    \arrow[r, "T"', shift right=0.8, mid vert]
                    \&
                    A
                    \arrow[r, "R",mid vert]
                    \&
                    B\mrp{,}
                \end{tikzcd}
            \]%
            and suppose that $R\procirc S=R\procirc T$. Note that, since $R_{*}$ is injective, given a diagram of the form
            \[
                \begin{tikzcd}[row sep={4.0*\the\DL,between origins}, column sep={4.0*\the\DL,between origins}, background color=backgroundColor, ampersand replacement=\&]
                    \pt
                    \arrow[r, "U", shift left=0.8, mid vert]
                    \arrow[r, "V"', shift right=0.8, mid vert]
                    \&
                    A
                    \arrow[r, "R",mid vert]
                    \&
                    B\mrp{,}
                \end{tikzcd}
            \]%
            if $R_{*}(U)=R\procirc U=R\procirc V=R_{*}(V)$, then $U=V$. In particular, for each $x\in X$, we may consider the diagram
            \[
                \begin{tikzcd}[row sep={4.0*\the\DL,between origins}, column sep={4.0*\the\DL,between origins}, background color=backgroundColor, ampersand replacement=\&]
                    \pt
                    \arrow[r, "{[x]}", shift left=0.8, mid vert]
                    \&
                    X
                    \arrow[r, "S", shift left=0.8, mid vert]
                    \arrow[r, "T"', shift right=0.8, mid vert]
                    \&
                    A
                    \arrow[r, "R",mid vert]
                    \&
                    B\mrp{,}
                \end{tikzcd}
            \]%
            for which we have $R\procirc S\procirc[x]=R\procirc T\procirc[x]$, implying that we have
            \[
                S(x)%
                =%
                S\procirc[x]%
                =%
                T\procirc[x]%
                =%
                T(x)
            \]%
            for each $x\in X$, implying $S=T$, and thus $R$ is a monomorphism.
    \end{itemize}
    We can also prove this in a more abstract way, following \cite{MSE350788}:
    \begin{itemize}
        \item\SloganFont{\cref{characterisations-of-monomorphisms-in-rel-1}$\implies$\cref{characterisations-of-monomorphisms-in-rel-2}: }Assume that $R$ is a monomorphism.
            \begin{itemize}
                \item We first notice that the functor $\Rel(\pt,-)\colon\Rel\to\Sets$ maps $R$ to $R_{*}$ by \ChapterRef{\ChapterConstructionsWithRelations, \cref{constructions-with-relations:unwinding-the-direct-image-function-associated-to-a-relation}}{\cref{unwinding-the-direct-image-function-associated-to-a-relation}}.
                \item Since $\Rel(\pt,-)$ preserves all limits by \ChapterRef{\ChapterLimitsAndColimits, \cref{limits-and-colimits:properties-of-co-limits-commutativity-with-homs} of \cref{limits-and-colimits:properties-of-co-limits}}{\cref{properties-of-co-limits-commutativity-with-homs} of \cref{properties-of-co-limits}}, it follows by \ChapterRef{\ChapterTypesOfMorphismsInCategories, \cref{types-of-morphisms-in-categories:properties-of-monomorphism-preserving-functors-interaction-with-limits} of \cref{types-of-morphisms-in-categories:properties-of-monomorphism-preserving-functors}}{\cref{properties-of-monomorphism-preserving-functors-interaction-with-limits} of \cref{properties-of-monomorphism-preserving-functors}} that $\Rel(\pt,-)$ also preserves monomorphisms.
                \item Since $R$ is a monomorphism and $\Rel(\pt,-)$ maps $R$ to $R_{*}$, it follows that $R_{*}$ is also a monomorphism.
                \item Since the monomorphisms in $\Sets$ are precisely the injections (\ChapterRef{\ChapterTypesOfMorphismsInCategories, \cref{types-of-morphisms-in-categories:examples-of-monomorphisms-monomorphisms-in-sets} of \cref{types-of-morphisms-in-categories:examples-of-monomorphisms}}{\cref{examples-of-monomorphisms-monomorphisms-in-sets} of \cref{examples-of-monomorphisms}}), it follows that $R_{*}$ is injective.
            \end{itemize}
        \item\SloganFont{\cref{characterisations-of-monomorphisms-in-rel-2}$\implies$\cref{characterisations-of-monomorphisms-in-rel-1}: }Assume that $R_{*}$ is injective.
            \begin{itemize}
                \item We first notice that the functor $\Rel(\pt,-)\colon\Rel\to\Sets$ maps $R$ to $R_{*}$ by \ChapterRef{\ChapterConstructionsWithRelations, \cref{constructions-with-relations:unwinding-the-direct-image-function-associated-to-a-relation}}{\cref{unwinding-the-direct-image-function-associated-to-a-relation}}.
                \item Since the monomorphisms in $\Sets$ are precisely the injections (\ChapterRef{\ChapterTypesOfMorphismsInCategories, \cref{types-of-morphisms-in-categories:examples-of-monomorphisms-monomorphisms-in-sets} of \cref{types-of-morphisms-in-categories:examples-of-monomorphisms}}{\cref{examples-of-monomorphisms-monomorphisms-in-sets} of \cref{examples-of-monomorphisms}}), it follows that $R_{*}$ is a monomorphism.
                \item Since $\Rel(\pt,-)$ is faithful, it follows by \ChapterRef{\ChapterTypesOfMorphismsInCategories, \cref{types-of-morphisms-in-categories:properties-of-monomorphism-reflecting-functors-interaction-with-faithfulness} of \cref{types-of-morphisms-in-categories:properties-of-monomorphism-reflecting-functors}}{\cref{properties-of-monomorphism-reflecting-functors-interaction-with-faithfulness} of \cref{properties-of-monomorphism-reflecting-functors}} that $\Rel(\pt,-)$ reflects monomorphisms.
                \item Since $R_{*}$ is a monomorphism and $\Rel(\pt,-)$ maps $R$ to $R_{*}$, it follows that $R$ is also a monomorphism.
            \end{itemize}
    \end{itemize}
    Finally, we prove the second part of the statement. Assume that $R$ is a monomorphism, let $a,a'\in A$ such that $a\sim_{R}b$ and $a'\sim_{R}b$ for some $b\in B$, and consider the diagram
    \[
        \begin{tikzcd}[row sep={4.0*\the\DL,between origins}, column sep={4.0*\the\DL,between origins}, background color=backgroundColor, ampersand replacement=\&]
            \pt
            \arrow[r, "{[a]}",   shift left=0.8, mid vert]
            \arrow[r, "{[a']}"', shift right=0.8, mid vert]
            \&
            A
            \arrow[r, "R",mid vert]
            \&
            B\mrp{.}
        \end{tikzcd}
    \]%
    Since $\star\sim_{[a]}a$ and $a\sim_{R}b$, we have $\star\sim_{R\procirc[a]}b$. Similarly, $\star\sim_{R\procirc[a']}b$. Thus $R\procirc[a]=R\procirc[a']$, and since $R$ is a monomorphism, we have $[a]=[a']$, i.e.\ $a=a'$.

    Conversely, assume the condition
    \begin{itemize}
        \itemstar For each $a,a'\in A$, if there exists some $b\in B$ such that
            \begin{align*}%
                a\sim_{R}b,\\%
                a'\sim_{R}b,%
            \end{align*}%
            then $a=a'$.
    \end{itemize}
    consider the diagram
    \[
        \begin{tikzcd}[row sep={4.0*\the\DL,between origins}, column sep={4.0*\the\DL,between origins}, background color=backgroundColor, ampersand replacement=\&]
            X
            \arrow[r, "S",  shift left=0.8,  mid vert]
            \arrow[r, "T"', shift right=0.8, mid vert]
            \&
            A
            \arrow[r, "R",mid vert]
            \&
            B\mrp{,}
        \end{tikzcd}
    \]%
    and let $(x,a)\in S$. Since $R$ is total and $a\in A$, there exists some $b\in B$ such that $a\sim_{R}b$. In this case, we have $x\sim_{R\procirc S}b$, and since $R\procirc S=R\procirc T$, we have also $x\sim_{R\procirc T}b$. Thus there must exist some $a'\in A$ such that $x\sim_{T}a'$ and $a'\sim_{R}b$. However, since $a,a'\sim_{R}b$, we must have $a=a'$, and thus $(x,a)\in T$ as well.

    A similar argument shows that if $(x,a)\in T$, then $(x,a)\in S$, and thus $S=T$ and it follows that $R$ is a monomorphism.
\end{Proof}
\subsection{2-Categorical Monomorphisms in $\sfbfRel$}\label{subsection-2-categorical-monomorphisms-in-rel}
In this section we characterise (for now, some of) the $2$-categorical monomorphisms in $\sfbfRel$, following \ChapterRef{\ChapterTypesOfMorphismsInBicategories, \cref{types-of-morphisms-in-bicategories:section-monomorphisms-in-bicategories}}{\cref{section-monomorphisms-in-bicategories}}.
\begin{proposition}{2-Categorical Monomorphisms in $\sfbfRel$}{2-categorical-monomorphisms-in-rel}%
    Let $R\colon A\rightproarrow B$ be a relation.
    \begin{enumerate}
        \item\label{2-categorical-monomorphisms-in-rel-representably-faithful-morphisms-in-rel}\SloganFont{Representably Faithful Morphisms in $\sfbfRel$. }Every morphism of $\sfbfRel$ is a representably faithful morphism.
        \item\label{2-categorical-monomorphisms-in-rel-representably-full-morphisms-in-rel}\SloganFont{Representably Full Morphisms in $\sfbfRel$. }The following conditions are equivalent:
            \begin{enumerate}
                \item\label{2-categorical-monomorphisms-in-rel-representably-full-morphisms-in-rel-1}The morphism $R\colon A\rightproarrow B$ is a representably full morphism.
                \item\label{2-categorical-monomorphisms-in-rel-representably-full-morphisms-in-rel-2}For each pair of relations $S,T\colon X\rightproarrows A$, the following condition is satisfied:
                    \begin{itemize}
                        \itemstar If $R\procirc S\subset R\procirc T$, then $S\subset T$.
                    \end{itemize}
                \item\label{2-categorical-monomorphisms-in-rel-representably-full-morphisms-in-rel-3}The functor
                    \[
                        R_{*}%
                        \colon%
                        (\mathcal{P}(A),\subset)%
                        \to%
                        (\mathcal{P}(B),\subset)%
                    \]%
                    is full.
                \item\label{2-categorical-monomorphisms-in-rel-representably-full-morphisms-in-rel-4}For each $U,V\in\mathcal{P}(A)$, if $R_{*}(U)\subset R_{*}(V)$, then $U\subset V$.
                \item\label{2-categorical-monomorphisms-in-rel-representably-full-morphisms-in-rel-5}The functor
                    \[
                        R_{!}%
                        \colon%
                        (\mathcal{P}(A),\subset)%
                        \to%
                        (\mathcal{P}(B),\subset)%
                    \]%
                    is full.
                \item\label{2-categorical-monomorphisms-in-rel-representably-full-morphisms-in-rel-6}For each $U,V\in\mathcal{P}(A)$, if $R_{!}(U)\subset R_{!}(V)$, then $U\subset V$.
            \end{enumerate}
        \item\label{2-categorical-monomorphisms-in-rel-representably-fully-faithful-morphisms-in-rel}\SloganFont{Representably Fully Faithful Morphisms in $\sfbfRel$. }Every representably full morphism in $\sfbfRel$ is a representably fully faithful morphism.
    \end{enumerate}
\end{proposition}
\begin{Proof}{Proof of \cref{2-categorical-monomorphisms-in-rel}}%
    \FirstProofBox{\cref{2-categorical-monomorphisms-in-rel-representably-faithful-morphisms-in-rel}: Representably Faithful Morphisms in $\sfbfRel$}%
    The relation $R$ is a representably faithful morphism in $\sfbfRel$ \textiff, for each $X\in\Obj(\sfbfRel)$, the functor
    \[
        R_{*}%
        \colon%
        \eRel(X,A)%
        \to%
        \eRel(X,B)%
    \]%
    is faithful, i.e.\ \textiff the morphism
    \[
        R_{*|S,T}%
        \colon%
        \Hom_{\eRel(X,A)}(S,T)%
        \to%
        \Hom_{\eRel(X,B)}(R\procirc S,R\procirc T)%
    \]%
    is injective for each $S,T\in\Obj(\eRel(X,A))$. However, $\Hom_{\eRel(X,A)}(S,T)$ is either empty or a singleton, in either case of which the map $R_{*|S,T}$ is necessarily injective.

    \ProofBox{\cref{2-categorical-monomorphisms-in-rel-representably-full-morphisms-in-rel}: Representably Full Morphisms in $\sfbfRel$}%
    We claim \cref{2-categorical-monomorphisms-in-rel-representably-full-morphisms-in-rel-1,2-categorical-monomorphisms-in-rel-representably-full-morphisms-in-rel-2,2-categorical-monomorphisms-in-rel-representably-full-morphisms-in-rel-3,2-categorical-monomorphisms-in-rel-representably-full-morphisms-in-rel-4,2-categorical-monomorphisms-in-rel-representably-full-morphisms-in-rel-5,2-categorical-monomorphisms-in-rel-representably-full-morphisms-in-rel-6} are indeed equivalent:
    \begin{itemize}
        \item\SloganFont{\cref{2-categorical-monomorphisms-in-rel-representably-full-morphisms-in-rel-1}$\iff$\cref{2-categorical-monomorphisms-in-rel-representably-full-morphisms-in-rel-2}: }This is simply a matter of unwinding definitions: The relation $R$ is a representably full morphism in $\sfbfRel$ \textiff, for each $X\in\Obj(\sfbfRel)$, the functor
            \[
                R_{*}%
                \colon%
                \eRel(X,A)%
                \to%
                \eRel(X,B)%
            \]%
            is full, i.e.\ \textiff the morphism
            \[
                R_{*|S,T}%
                \colon%
                \Hom_{\eRel(X,A)}(S,T)%
                \to%
                \Hom_{\eRel(X,B)}(R\procirc S,R\procirc T)%
            \]%
            is surjective for each $S,T\in\Obj(\eRel(X,A))$, i.e.\ \textiff, whenever $R\procirc S\subset R\procirc T$, we also have $S\subset T$.
        \item\SloganFont{\cref{2-categorical-monomorphisms-in-rel-representably-full-morphisms-in-rel-3}$\iff$\cref{2-categorical-monomorphisms-in-rel-representably-full-morphisms-in-rel-4}: }This is also simply a matter of unwinding definitions: The functor
            \[
                R_{*}%
                \colon%
                (\mathcal{P}(A),\subset)%
                \to%
                (\mathcal{P}(B),\subset)%
            \]%
            is full \textiff, for each $U,V\in\mathcal{P}(A)$, the morphism
            \[
                R_{*|U,V}%
                \colon%
                \Hom_{\mathcal{P}(A)}(U,V)%
                \to%
                \Hom_{\mathcal{P}(B)}(R_{*}(U),R_{*}(V))%
            \]%
            is surjective, i.e.\ \textiff whenever $R_{*}(U)\subset R_{*}(V)$, we also necessarily have $U\subset V$.
        \item\SloganFont{\cref{2-categorical-monomorphisms-in-rel-representably-full-morphisms-in-rel-5}$\iff$\cref{2-categorical-monomorphisms-in-rel-representably-full-morphisms-in-rel-6}: }This is once again simply a matter of unwinding definitions, and proceeds exactly in the same way as in the proof of the equivalence between \cref{2-categorical-monomorphisms-in-rel-representably-full-morphisms-in-rel-3,2-categorical-monomorphisms-in-rel-representably-full-morphisms-in-rel-4} given above.
        \item\SloganFont{\cref{2-categorical-monomorphisms-in-rel-representably-full-morphisms-in-rel-4}$\implies$\cref{2-categorical-monomorphisms-in-rel-representably-full-morphisms-in-rel-6}: }Suppose that the following condition is true:
            \begin{itemize}
                \itemstar For each $U,V\in\mathcal{P}(A)$, if $R_{*}(U)\subset R_{*}(V)$, then $U\subset V$.
            \end{itemize}
            We need to show that the condition
            \begin{itemize}
                \itemstar For each $U,V\in\mathcal{P}(A)$, if $R_{!}(U)\subset R_{!}(V)$, then $U\subset V$.
            \end{itemize}
            is also true. We proceed step by step:
            \begin{itemize}
                \item Suppose we have $U,V\in\mathcal{P}(A)$ with $R_{!}(U)\subset R_{!}(V)$.
                \item By \ChapterRef{\ChapterConstructionsWithRelations, \cref{constructions-with-relations:properties-of-direct-image-with-compact-support-functions-associated-to-relations-relation-to-direct-images} of \cref{constructions-with-relations:properties-of-direct-image-with-compact-support-functions-associated-to-relations}}{\cref{properties-of-direct-image-with-compact-support-functions-associated-to-relations-relation-to-direct-images} of \cref{properties-of-direct-image-with-compact-support-functions-associated-to-relations}}, we have
                    \begin{align*}
                        R_{!}(U) &= B\setminus R_{*}(A\setminus U),\\
                        R_{!}(V) &= B\setminus R_{*}(A\setminus V).
                    \end{align*}
                \item By \ChapterRef{\ChapterConstructionsWithSets, \cref{constructions-with-sets:properties-of-differences-functoriality} of \cref{constructions-with-sets:properties-of-differences}}{\cref{properties-of-differences-functoriality} of \cref{properties-of-differences}} we have $R_{*}(A\setminus V)\subset R_{*}(A\setminus U)$.
                \item By assumption, we then have $A\setminus V\subset A\setminus U$.
                \item By \ChapterRef{\ChapterConstructionsWithSets, \cref{constructions-with-sets:properties-of-differences-functoriality} of \cref{constructions-with-sets:properties-of-differences}}{\cref{properties-of-differences-functoriality} of \cref{properties-of-differences}} again, we have $U\subset V$.
            \end{itemize}
        \item\SloganFont{\cref{2-categorical-monomorphisms-in-rel-representably-full-morphisms-in-rel-6}$\implies$\cref{2-categorical-monomorphisms-in-rel-representably-full-morphisms-in-rel-4}: }Suppose that the following condition is true:
            \begin{itemize}
                \itemstar For each $U,V\in\mathcal{P}(A)$, if $R_{!}(U)\subset R_{!}(V)$, then $U\subset V$.
            \end{itemize}
            We need to show that the condition
            \begin{itemize}
                \itemstar For each $U,V\in\mathcal{P}(A)$, if $R_{*}(U)\subset R_{*}(V)$, then $U\subset V$.
            \end{itemize}
            is also true. We proceed step by step:
            \begin{itemize}
                \item Suppose we have $U,V\in\mathcal{P}(A)$ with $R_{*}(U)\subset R_{*}(V)$.
                \item By \ChapterRef{\ChapterConstructionsWithRelations, \cref{constructions-with-relations:properties-of-direct-image-functions-associated-to-relations-relation-to-direct-images-with-compact-support} of \cref{constructions-with-relations:properties-of-direct-image-functions-associated-to-relations}}{\cref{properties-of-direct-image-functions-associated-to-relations-relation-to-direct-images-with-compact-support} of \cref{properties-of-direct-image-functions-associated-to-relations}}, we have
                    \begin{align*}
                        R_{*}(U) &= B\setminus R_{!}(A\setminus U),\\
                        R_{*}(V) &= B\setminus R_{!}(A\setminus V).
                    \end{align*}
                \item By \ChapterRef{\ChapterConstructionsWithSets, \cref{constructions-with-sets:properties-of-differences-functoriality} of \cref{constructions-with-sets:properties-of-differences}}{\cref{properties-of-differences-functoriality} of \cref{properties-of-differences}} we have $R_{!}(A\setminus V)\subset R_{!}(A\setminus U)$.
                \item By assumption, we then have $A\setminus V\subset A\setminus U$.
                \item By \ChapterRef{\ChapterConstructionsWithSets, \cref{constructions-with-sets:properties-of-differences-functoriality} of \cref{constructions-with-sets:properties-of-differences}}{\cref{properties-of-differences-functoriality} of \cref{properties-of-differences}} again, we have $U\subset V$.
            \end{itemize}
        \item\SloganFont{\cref{2-categorical-monomorphisms-in-rel-representably-full-morphisms-in-rel-2}$\implies$\cref{2-categorical-monomorphisms-in-rel-representably-full-morphisms-in-rel-4}: }Consider the diagram
            \[
                \begin{tikzcd}[row sep={4.0*\the\DL,between origins}, column sep={4.0*\the\DL,between origins}, background color=backgroundColor, ampersand replacement=\&]
                    X
                    \arrow[r, "S", shift left=0.8, mid vert]
                    \arrow[r, "T"', shift right=0.8, mid vert]
                    \&
                    A
                    \arrow[r, "R",mid vert]
                    \&
                    B\mrp{,}
                \end{tikzcd}
            \]%
            and suppose that $R\procirc S\subset R\procirc T$. Note that, by assumption, given a diagram of the form
            \[
                \begin{tikzcd}[row sep={4.0*\the\DL,between origins}, column sep={4.0*\the\DL,between origins}, background color=backgroundColor, ampersand replacement=\&]
                    \pt
                    \arrow[r, "U", shift left=0.8, mid vert]
                    \arrow[r, "V"', shift right=0.8, mid vert]
                    \&
                    A
                    \arrow[r, "R",mid vert]
                    \&
                    B\mrp{,}
                \end{tikzcd}
            \]%
            if $R_{*}(U)=R\procirc U\subset R\procirc V=R_{*}(V)$, then $U\subset V$. In particular, for each $x\in X$, we may consider the diagram
            \[
                \begin{tikzcd}[row sep={4.0*\the\DL,between origins}, column sep={4.0*\the\DL,between origins}, background color=backgroundColor, ampersand replacement=\&]
                    \pt
                    \arrow[r, "{[x]}", shift left=0.8, mid vert]
                    \&
                    X
                    \arrow[r, "S", shift left=0.8, mid vert]
                    \arrow[r, "T"', shift right=0.8, mid vert]
                    \&
                    A
                    \arrow[r, "R",mid vert]
                    \&
                    B\mrp{,}
                \end{tikzcd}
            \]%
            for which we have $R\procirc S\procirc[x]\subset R\procirc T\procirc[x]$, implying that we have
            \[
                S(x)%
                =%
                S\procirc[x]%
                \subset%
                T\procirc[x]%
                =%
                T(x)
            \]%
            for each $x\in X$, implying $S\subset T$.
        \item\SloganFont{\cref{2-categorical-monomorphisms-in-rel-representably-full-morphisms-in-rel-4}$\implies$\cref{2-categorical-monomorphisms-in-rel-representably-full-morphisms-in-rel-2}: }Let $U,V\in\mathcal{P}(A)$ and consider the diagram
            \[
                \begin{tikzcd}[row sep={4.0*\the\DL,between origins}, column sep={4.0*\the\DL,between origins}, background color=backgroundColor, ampersand replacement=\&]
                    \pt
                    \arrow[r, "U",  shift left=0.8,  mid vert]
                    \arrow[r, "V"', shift right=0.8, mid vert]
                    \&
                    A
                    \arrow[r, "R",mid vert]
                    \&
                    B\mrp{.}
                \end{tikzcd}
            \]%
            By \cref{unwinding-the-direct-image-function-associated-to-a-relation}, we have
            \begin{align*}
                R_{*}(U) &= R\procirc U,\\
                R_{*}(V) &= R\procirc V.
            \end{align*}
            Now, if $R_{*}(U)\subset R_{*}(V)$, i.e.\ $R\procirc U\subset R\procirc V$, then $U\subset V$ by assumption.
    \end{itemize}

    \ProofBox{\cref{2-categorical-monomorphisms-in-rel-representably-fully-faithful-morphisms-in-rel}: Representably Fully Faithful Morphisms in $\sfbfRel$}%
    This follows from \cref{2-categorical-monomorphisms-in-rel-representably-faithful-morphisms-in-rel,2-categorical-monomorphisms-in-rel-representably-full-morphisms-in-rel}.
\end{Proof}
\begin{question}{Better Characterisations of Representably Full Morphisms in $\sfbfRel$}{better-characterisations-of-representably-full-morphisms-in-rel}%
    \cref{2-categorical-monomorphisms-in-rel-representably-full-morphisms-in-rel} of \cref{2-categorical-monomorphisms-in-rel} gives a characterisation of the representably full morphisms in $\sfbfRel$.

    Are there other nice characterisations of these?

    This question also appears as \cite{MO467527}.
\end{question}
\subsection{Epimorphisms in $\sfRel$}\label{subsection-epimorphisms-in-rel}
In this section we characterise the epimorphisms in the category $\sfRel$, following \ChapterRef{\ChapterTypesOfMorphismsInCategories, \cref{types-of-morphisms-in-categories:section-epimorphisms}}{\cref{section-epimorphisms}}.
\begin{proposition}{Characterisations of Epimorphisms in $\sfRel$}{characterisations-of-epimorphisms-in-rel}%
    Let $R\colon A\rightproarrow B$ be a relation. The following conditions are equivalent:
    \begin{enumerate}
        \item\label{characterisations-of-epimorphisms-in-rel-1}The relation $R$ is an epimorphism in $\sfRel$.
        \item\label{characterisations-of-epimorphisms-in-rel-2}The weak inverse image function
            \[
                R^{-1}%
                \colon%
                \mathcal{P}(B)%
                \to%
                \mathcal{P}(A)%
            \]%
            associated to $R$ is injective.
        \item\label{characterisations-of-epimorphisms-in-rel-3}The strong inverse image function
            \[
                R_{-1}%
                \colon%
                \mathcal{P}(B)%
                \to%
                \mathcal{P}(A)%
            \]%
            associated to $R$ is injective.
        \item\label{characterisations-of-epimorphisms-in-rel-4}The function $R\colon A\to\mathcal{P}(B)$ is \say{surjective on singletons}:
            \begin{itemize}
                \itemstar For each $b\in B$, there exists some $a\in A$ such that $R(a)=\{b\}$.
            \end{itemize}
    \end{enumerate}
    Moreover, if $R$ is total and an epimorphism, then it satisfies the following equivalent conditions:
    \begin{enumerate}
        \item\label{characterisations-of-epimorphisms-in-rel-5}For each $b\in B$, there exists some $a\in A$ such that $a\sim_{R}b$.
        \item\label{characterisations-of-epimorphisms-in-rel-6}We have $\Im(R)=B$.
    \end{enumerate}
\end{proposition}
\begin{Proof}{Proof of \cref{characterisations-of-epimorphisms-in-rel}}%
    Firstly note that \cref{characterisations-of-epimorphisms-in-rel-2,characterisations-of-epimorphisms-in-rel-3} are equivalent by \ChapterRef{\ChapterConstructionsWithRelations, \cref{constructions-with-relations:properties-of-strong-inverse-image-functions-associated-to-relations-interaction-with-weak-inverse-images-1} of \cref{constructions-with-relations:properties-of-strong-inverse-image-functions-associated-to-relations}}{\cref{properties-of-strong-inverse-image-functions-associated-to-relations-interaction-with-weak-inverse-images-1} of \cref{properties-of-strong-inverse-image-functions-associated-to-relations}}. We then claim that \cref{characterisations-of-epimorphisms-in-rel-1,characterisations-of-epimorphisms-in-rel-2} are also equivalent:
    \begin{itemize}
        \item\SloganFont{\cref{characterisations-of-epimorphisms-in-rel-1}$\implies$\cref{characterisations-of-epimorphisms-in-rel-2}: }Let $U,V\in\mathcal{P}(A)$ and consider the diagram
            \[
                \begin{tikzcd}[row sep={4.0*\the\DL,between origins}, column sep={4.0*\the\DL,between origins}, background color=backgroundColor, ampersand replacement=\&]
                    A
                    \arrow[r, "R",mid vert]
                    \&
                    B
                    \arrow[r, "U", shift left=0.8, mid vert]
                    \arrow[r, "V"', shift right=0.8, mid vert]
                    \&
                    \pt\mrp{.}
                \end{tikzcd}
            \]%
            By \ChapterRef{\ChapterConstructionsWithRelations, \cref{constructions-with-relations:unwinding-the-direct-image-function-associated-to-a-relation}}{\cref{unwinding-the-direct-image-function-associated-to-a-relation}}, we have
            \begin{align*}
                R^{-1}(U) &= U\procirc R,\\
                R^{-1}(V) &= V\procirc R.
            \end{align*}
            Now, if $U\procirc R=V\procirc R$, i.e.\ $R^{-1}(U)=R^{-1}(V)$, then $U=V$ since $R$ is assumed to be an epimorphism, showing $R^{-1}$ to be injective.
        \item\SloganFont{\cref{characterisations-of-epimorphisms-in-rel-2}$\implies$\cref{characterisations-of-epimorphisms-in-rel-1}: }Conversely, suppose that $R^{-1}$ is injective, consider the diagram
            \[
                \begin{tikzcd}[row sep={4.0*\the\DL,between origins}, column sep={4.0*\the\DL,between origins}, background color=backgroundColor, ampersand replacement=\&]
                    A
                    \arrow[r, "R",mid vert]
                    \&
                    B
                    \arrow[r, "S", shift left=0.8, mid vert]
                    \arrow[r, "T"', shift right=0.8, mid vert]
                    \&
                    X\mrp{,}
                \end{tikzcd}
            \]%
            and suppose that $S\procirc R=T\procirc R$. Note that, since $R^{-1}$ is injective, given a diagram of the form
            \[
                \begin{tikzcd}[row sep={4.0*\the\DL,between origins}, column sep={4.0*\the\DL,between origins}, background color=backgroundColor, ampersand replacement=\&]
                    A
                    \arrow[r, "R",mid vert]
                    \&
                    B
                    \arrow[r, "U", shift left=0.8, mid vert]
                    \arrow[r, "V"', shift right=0.8, mid vert]
                    \&
                    \pt\mrp{,}
                \end{tikzcd}
            \]%
            if $R^{-1}(U)=U\procirc R=V\procirc R=R^{-1}(V)$, then $U=V$. In particular, for each $x\in X$, we may consider the diagram
            \[
                \begin{tikzcd}[row sep={4.0*\the\DL,between origins}, column sep={4.0*\the\DL,between origins}, background color=backgroundColor, ampersand replacement=\&]
                    A
                    \arrow[r, "R",mid vert]
                    \&
                    B
                    \arrow[r, "S",  shift left=0.8,  mid vert]
                    \arrow[r, "T"', shift right=0.8, mid vert]
                    \&
                    X
                    \arrow[r, "{[x]}", shift left=0.8, mid vert]
                    \&
                    \pt\mrp{,}
                \end{tikzcd}
            \]%
            for which we have $[x]\procirc S\procirc R=[x]\procirc T\procirc R$, implying that we have
            \[
                S^{-1}(x)%
                =%
                [x]\procirc S%
                =%
                [x]\procirc T%
                =%
                T^{-1}(x)
            \]%
            for each $x\in X$, implying $S=T$, and thus $R$ is an epimorphism.
    \end{itemize}
    We can also prove this in a more abstract way, following \cite{MSE350788}:
    \begin{itemize}
        \item\SloganFont{\cref{characterisations-of-epimorphisms-in-rel-1}$\implies$\cref{characterisations-of-epimorphisms-in-rel-2}: }Assume that $R$ is an epimorphism.
            \begin{itemize}
                \item We first notice that the functor $\Rel(-,\pt)\colon\Rel^{\op}\to\Sets$ maps $R$ to $R^{-1}$ by \ChapterRef{\ChapterConstructionsWithRelations, \cref{constructions-with-relations:unwinding-the-weak-inverse-image-function-associated-to-a-relation}}{\cref{unwinding-the-weak-inverse-image-function-associated-to-a-relation}}.
                \item Since $\Rel(-,\pt)$ preserves limits by \ChapterRef{\ChapterLimitsAndColimits, \cref{limits-and-colimits:properties-of-co-limits-commutativity-with-homs} of \cref{limits-and-colimits:properties-of-co-limits}}{\cref{properties-of-co-limits-commutativity-with-homs} of \cref{properties-of-co-limits}}, it follows by \ChapterRef{\ChapterTypesOfMorphismsInCategories, \cref{types-of-morphisms-in-categories:properties-of-monomorphism-preserving-functors-interaction-with-limits} of \cref{types-of-morphisms-in-categories:properties-of-monomorphism-preserving-functors}}{\cref{properties-of-monomorphism-preserving-functors-interaction-with-limits} of \cref{properties-of-monomorphism-preserving-functors}} that $\Rel(-,\pt)$ also preserves monomorphisms.
                \item That is: $\Rel(-,\pt)$ sends monomorphisms in $\Rel^{\op}$ to monomorphisms in $\Sets$.
                \item The monomorphisms $\Rel^{\op}$ are precisely the epimorphisms in $\Rel$ by \ChapterRef{\ChapterTypesOfMorphismsInCategories, \cref{types-of-morphisms-in-categories:properties-of-monomorphisms-duality} of \cref{types-of-morphisms-in-categories:properties-of-monomorphisms}}{\cref{properties-of-monomorphisms-duality} of \cref{properties-of-monomorphisms}}.
                \item Since $R$ is an epimorphism and $\Rel(-,\pt)$ maps $R$ to $R^{-1}$, it follows that $R^{-1}$ is a monomorphism.
                \item Since the monomorphisms in $\Sets$ are precisely the injections (\ChapterRef{\ChapterTypesOfMorphismsInCategories, \cref{types-of-morphisms-in-categories:examples-of-monomorphisms-monomorphisms-in-sets} of \cref{types-of-morphisms-in-categories:examples-of-monomorphisms}}{\cref{examples-of-monomorphisms-monomorphisms-in-sets} of \cref{examples-of-monomorphisms}}), it follows that $R^{-1}$ is injective.
            \end{itemize}
        \item\SloganFont{\cref{characterisations-of-epimorphisms-in-rel-2}$\implies$\cref{characterisations-of-epimorphisms-in-rel-1}: }Assume that $R^{-1}$ is injective.
            \begin{itemize}
                \item We first notice that the functor $\Rel(-,\pt)\colon\Rel^{\op}\to\Sets$ maps $R$ to $R^{-1}$ by \ChapterRef{\ChapterConstructionsWithRelations, \cref{constructions-with-relations:unwinding-the-weak-inverse-image-function-associated-to-a-relation}}{\cref{unwinding-the-weak-inverse-image-function-associated-to-a-relation}}.
                \item Since the monomorphisms in $\Sets$ are precisely the injections (\ChapterRef{\ChapterTypesOfMorphismsInCategories, \cref{types-of-morphisms-in-categories:examples-of-monomorphisms-monomorphisms-in-sets} of \cref{types-of-morphisms-in-categories:examples-of-monomorphisms}}{\cref{examples-of-monomorphisms-monomorphisms-in-sets} of \cref{examples-of-monomorphisms}}), it follows that $R^{-1}$ is a monomorphism.
                \item Since $\Rel(-,\pt)$ is faithful, it follows by \ChapterRef{\ChapterTypesOfMorphismsInCategories, \cref{types-of-morphisms-in-categories:properties-of-monomorphism-reflecting-functors-interaction-with-faithfulness} of \cref{types-of-morphisms-in-categories:properties-of-monomorphism-reflecting-functors}}{\cref{properties-of-monomorphism-reflecting-functors-interaction-with-faithfulness} of \cref{properties-of-monomorphism-reflecting-functors}} that $\Rel(,\pt)$ reflects monomorphisms.
                \item That is: $\Rel(-,\pt)$ reflects monomorphisms in $\Sets$ to monomorphisms in $\Rel^{\op}$.
                \item The monomorphisms $\Rel^{\op}$ are precisely the epimorphisms in $\Rel$ by \ChapterRef{\ChapterTypesOfMorphismsInCategories, \cref{types-of-morphisms-in-categories:properties-of-monomorphisms-duality} of \cref{types-of-morphisms-in-categories:properties-of-monomorphisms}}{\cref{properties-of-monomorphisms-duality} of \cref{properties-of-monomorphisms}}.
                \item Since $R^{-1}$ is a monomorphism and $\Rel(-,\pt)$ maps $R$ to $R^{-1}$, it follows that $R$ is an epimorphism.
            \end{itemize}
    \end{itemize}

    We also claim that \cref{characterisations-of-epimorphisms-in-rel-2,characterisations-of-epimorphisms-in-rel-4} are equivalent, following \cite{MO455260}:
    \begin{itemize}
        \item\SloganFont{\cref{characterisations-of-epimorphisms-in-rel-2}$\implies$\cref{characterisations-of-epimorphisms-in-rel-4}: }Since $B\setminus\{b\}\subset B$ and $R^{-1}$ is injective, we have $R^{-1}(B\setminus\{b\})\subsetneq R^{-1}(B)$. So taking some $a\in R^{-1}(B)\setminus R^{-1}(B\setminus\{b\})$ we get an element of $A$ such that $R(a)=\{b\}$.
        \item\SloganFont{\cref{characterisations-of-epimorphisms-in-rel-4}$\implies$\cref{characterisations-of-epimorphisms-in-rel-2}: }Let $U,V\subset B$ with $U\neq V$. Without loss of generality, we can assume $U\setminus V\neq\emptyset$; otherwise just swap $U$ and $V$. Let then $b\in U\setminus V$. By assumption, there exists an $a\in A$ with $R(a)=\{b\}$. Then $a\in R^{-1}(U)$ but $a\nin R^{-1}(V)$, and thus $R^{-1}(U)\neq R^{-1}(V)$, showing $R^{-1}$ to be injective.
    \end{itemize}

    Finally, we prove the second part of the statement. So assume $R$ is a total epimorphism in $\sfRel$ and consider the diagram
    \[
        \begin{tikzcd}[row sep={4.0*\the\DL,between origins}, column sep={4.0*\the\DL,between origins}, background color=backgroundColor, ampersand replacement=\&]
            A
            \arrow[r, "R",mid vert]
            \&
            B
            \arrow[r, "S", shift left=0.8, mid vert]
            \arrow[r, "T"', shift right=0.8, mid vert]
            \&
            {\{0,1\}\mrp{,}}
        \end{tikzcd}
    \]%
    where $b\sim_{S}0$ for each $b\in B$ and where we have
    \[
        b%
        \sim_{T}%
        \begin{cases}
            0 &\text{if $b\in\Im(R)$,}\\
            1 &\text{otherwise}
        \end{cases}
    \]%
    for each $b\in B$. Since $R$ is total, we have $a\sim_{S\procirc R}0$ and $a\sim_{T\procirc R}0$ for all $a\in A$, and no element of $A$ is related to $1$ by $S\procirc R$ or $T\procirc R$. Thus $S\procirc R=T\procirc R$, and since $R$ is an epimorphism, we have $S=T$. But by the definition of $T$, this implies $\Im(R)=B$.
\end{Proof}
\subsection{2-Categorical Epimorphisms in $\sfbfRel$}\label{subsection-2-categorical-epimorphisms-in-rel}
In this section we characterise (for now, some of) the $2$-categorical epimorphisms in $\sfbfRel$, following \ChapterRef{\ChapterTypesOfMorphismsInBicategories, \cref{types-of-morphisms-in-bicategories:section-epimorphisms-in-bicategories}}{\cref{section-epimorphisms-in-bicategories}}.
\begin{proposition}{2-Categorical Epimorphisms in $\sfbfRel$}{2-categorical-epimorphisms-in-rel}%
    Let $R\colon A\rightproarrow B$ be a relation.
    \begin{enumerate}
        \item\label{2-categorical-epimorphisms-in-rel-corepresentably-faithful-morphisms-in-rel}\SloganFont{Corepresentably Faithful Morphisms in $\sfbfRel$. }Every morphism of $\sfbfRel$ is a corepresentably faithful morphism.
        \item\label{2-categorical-epimorphisms-in-rel-corepresentably-full-morphisms-in-rel}\SloganFont{Corepresentably Full Morphisms in $\sfbfRel$. }The following conditions are equivalent:
            \begin{enumerate}
                \item\label{2-categorical-epimorphisms-in-rel-corepresentably-full-morphisms-in-rel-1}The morphism $R\colon A\rightproarrow B$ is a corepresentably full morphism.
                \item\label{2-categorical-epimorphisms-in-rel-corepresentably-full-morphisms-in-rel-2}For each pair of relations $S,T\colon X\rightproarrows A$, the following condition is satisfied:
                    \begin{itemize}
                        \itemstar If $S\procirc R\subset T\procirc R$, then $S\subset T$.
                    \end{itemize}
                \item\label{2-categorical-epimorphisms-in-rel-corepresentably-full-morphisms-in-rel-3}The functor
                    \[
                        R^{-1}%
                        \colon%
                        (\mathcal{P}(B),\subset)%
                        \to%
                        (\mathcal{P}(A),\subset)%
                    \]%
                    is full.
                \item\label{2-categorical-epimorphisms-in-rel-corepresentably-full-morphisms-in-rel-4}For each $U,V\in\mathcal{P}(B)$, if $R^{-1}(U)\subset R^{-1}(V)$, then $U\subset V$.
                \item\label{2-categorical-epimorphisms-in-rel-corepresentably-full-morphisms-in-rel-5}The functor
                    \[
                        R_{-1}%
                        \colon%
                        (\mathcal{P}(B),\subset)%
                        \to%
                        (\mathcal{P}(A),\subset)%
                    \]%
                    is full.
                \item\label{2-categorical-epimorphisms-in-rel-corepresentably-full-morphisms-in-rel-6}For each $U,V\in\mathcal{P}(B)$, if $R_{-1}(U)\subset R_{-1}(V)$, then $U\subset V$.
            \end{enumerate}
        \item\label{2-categorical-epimorphisms-in-rel-corepresentably-fully-faithful-morphisms-in-rel}\SloganFont{Corepresentably Fully Faithful Morphisms in $\sfbfRel$. }Every corepresentably full morphism of $\sfbfRel$ is a corepresentably fully faithful morphism.
    \end{enumerate}
\end{proposition}
\begin{Proof}{Proof of \cref{2-categorical-epimorphisms-in-rel}}%
    \FirstProofBox{\cref{2-categorical-epimorphisms-in-rel-corepresentably-faithful-morphisms-in-rel}: Corepresentably Faithful Morphisms in $\sfbfRel$}%
    The relation $R$ is a corepresentably faithful morphism in $\sfbfRel$ \textiff, for each $X\in\Obj(\sfbfRel)$, the functor
    \[
        R^{*}%
        \colon%
        \eRel(B,X)%
        \to%
        \eRel(A,X)%
    \]%
    is faithful, i.e.\ \textiff the morphism
    \[
        R^{*}_{S,T}%
        \colon%
        \Hom_{\eRel(B,X)}(S,T)%
        \to%
        \Hom_{\eRel(A,X)}(S\procirc R,T\procirc R)%
    \]%
    is injective for each $S,T\in\Obj(\eRel(B,X))$. However, $\Hom_{\eRel(B,X)}(S,T)$ is either empty or a singleton, in either case of which the map $R^{*}_{S,T}$ is necessarily injective.

    \ProofBox{\cref{2-categorical-epimorphisms-in-rel-corepresentably-full-morphisms-in-rel}: Corepresentably Full Morphisms in $\sfbfRel$}%
    We claim \cref{2-categorical-epimorphisms-in-rel-corepresentably-full-morphisms-in-rel-1,2-categorical-epimorphisms-in-rel-corepresentably-full-morphisms-in-rel-2,2-categorical-epimorphisms-in-rel-corepresentably-full-morphisms-in-rel-3,2-categorical-epimorphisms-in-rel-corepresentably-full-morphisms-in-rel-4,2-categorical-epimorphisms-in-rel-corepresentably-full-morphisms-in-rel-5,2-categorical-epimorphisms-in-rel-corepresentably-full-morphisms-in-rel-6} are indeed equivalent:
    \begin{itemize}
        \item\SloganFont{\cref{2-categorical-epimorphisms-in-rel-corepresentably-full-morphisms-in-rel-1}$\iff$\cref{2-categorical-epimorphisms-in-rel-corepresentably-full-morphisms-in-rel-2}: }This is simply a matter of unwinding definitions: The relation $R$ is a corepresentably full morphism in $\sfbfRel$ \textiff, for each $X\in\Obj(\sfbfRel)$, the functor
            \[
                R^{*}%
                \colon%
                \eRel(B,X)%
                \to%
                \eRel(A,X)%
            \]%
            is full, i.e.\ \textiff the morphism
            \[
                R^{*}_{S,T}%
                \colon%
                \Hom_{\eRel(B,X)}(S,T)%
                \to%
                \Hom_{\eRel(A,X)}(S\procirc R,T\procirc R)%
            \]%
            is surjective for each $S,T\in\Obj(\eRel(B,X))$, i.e.\ \textiff, whenever $S\procirc R\subset T\procirc R$, we also have $S\subset T$.
        \item\SloganFont{\cref{2-categorical-epimorphisms-in-rel-corepresentably-full-morphisms-in-rel-3}$\iff$\cref{2-categorical-epimorphisms-in-rel-corepresentably-full-morphisms-in-rel-4}: }This is also simply a matter of unwinding definitions: The functor
            \[
                R^{-1}%
                \colon%
                (\mathcal{P}(B),\subset)%
                \to%
                (\mathcal{P}(A),\subset)%
            \]%
            is full \textiff, for each $U,V\in\mathcal{P}(A)$, the morphism
            \[
                R^{-1}_{U,V}%
                \colon%
                \Hom_{\mathcal{P}(B)}(U,V)%
                \to%
                \Hom_{\mathcal{P}(A)}(R^{-1}(U),R^{-1}(V))%
            \]%
            is surjective, i.e.\ \textiff whenever $R^{-1}(U)\subset R^{-1}(V)$, we also necessarily have $U\subset V$.
        \item\SloganFont{\cref{2-categorical-epimorphisms-in-rel-corepresentably-full-morphisms-in-rel-5}$\iff$\cref{2-categorical-epimorphisms-in-rel-corepresentably-full-morphisms-in-rel-6}: }This is once again simply a matter of unwinding definitions, and proceeds exactly in the same way as in the proof of the equivalence between \cref{2-categorical-epimorphisms-in-rel-corepresentably-full-morphisms-in-rel-3,2-categorical-epimorphisms-in-rel-corepresentably-full-morphisms-in-rel-4} given above.
        \item\SloganFont{\cref{2-categorical-epimorphisms-in-rel-corepresentably-full-morphisms-in-rel-4}$\implies$\cref{2-categorical-epimorphisms-in-rel-corepresentably-full-morphisms-in-rel-6}: }Suppose that the following condition is true:
            \begin{itemize}
                \itemstar For each $U,V\in\mathcal{P}(B)$, if $R^{-1}(U)\subset R^{-1}(V)$, then $U\subset V$.
            \end{itemize}
            We need to show that the condition
            \begin{itemize}
                \itemstar For each $U,V\in\mathcal{P}(B)$, if $R_{-1}(U)\subset R_{-1}(V)$, then $U\subset V$.
            \end{itemize}
            is also true. We proceed step by step:
            \begin{itemize}
                \item Suppose we have $U,V\in\mathcal{P}(B)$ with $R_{-1}(U)\subset R_{-1}(V)$.
                \item By \ChapterRef{\ChapterConstructionsWithRelations, \cref{constructions-with-relations:properties-of-strong-inverse-image-functions-associated-to-relations-interaction-with-weak-inverse-images-1} of \cref{constructions-with-relations:properties-of-strong-inverse-image-functions-associated-to-relations}}{\cref{properties-of-strong-inverse-image-functions-associated-to-relations-interaction-with-weak-inverse-images-1} of \cref{properties-of-strong-inverse-image-functions-associated-to-relations}}, we have
                    \begin{align*}
                        R_{-1}(U) &= B\setminus R^{-1}(A\setminus U),\\
                        R_{-1}(V) &= B\setminus R^{-1}(A\setminus V).
                    \end{align*}
                \item By \ChapterRef{\ChapterConstructionsWithSets, \cref{constructions-with-sets:properties-of-differences-functoriality} of \cref{constructions-with-sets:properties-of-differences}}{\cref{properties-of-differences-functoriality} of \cref{properties-of-differences}} we have $R^{-1}(A\setminus V)\subset R^{-1}(A\setminus U)$.
                \item By assumption, we then have $A\setminus V\subset A\setminus U$.
                \item By \ChapterRef{\ChapterConstructionsWithSets, \cref{constructions-with-sets:properties-of-differences-functoriality} of \cref{constructions-with-sets:properties-of-differences}}{\cref{properties-of-differences-functoriality} of \cref{properties-of-differences}} again, we have $U\subset V$.
            \end{itemize}
        \item\SloganFont{\cref{2-categorical-epimorphisms-in-rel-corepresentably-full-morphisms-in-rel-6}$\implies$\cref{2-categorical-epimorphisms-in-rel-corepresentably-full-morphisms-in-rel-4}: }Suppose that the following condition is true:
            \begin{itemize}
                \itemstar For each $U,V\in\mathcal{P}(B)$, if $R_{-1}(U)\subset R_{-1}(V)$, then $U\subset V$.
            \end{itemize}
            We need to show that the condition
            \begin{itemize}
                \itemstar For each $U,V\in\mathcal{P}(B)$, if $R^{-1}(U)\subset R^{-1}(V)$, then $U\subset V$.
            \end{itemize}
            is also true. We proceed step by step:
            \begin{itemize}
                \item Suppose we have $U,V\in\mathcal{P}(B)$ with $R^{-1}(U)\subset R^{-1}(V)$.
                \item By \ChapterRef{\ChapterConstructionsWithRelations, \cref{constructions-with-relations:properties-of-weak-inverse-image-functions-associated-to-relations-interaction-with-strong-inverse-images-1} of \cref{constructions-with-relations:properties-of-weak-inverse-image-functions-associated-to-relations}}{\cref{properties-of-weak-inverse-image-functions-associated-to-relations-interaction-with-strong-inverse-images-1} of \cref{properties-of-weak-inverse-image-functions-associated-to-relations}}, we have
                    \begin{align*}
                        R^{-1}(U) &= B\setminus R_{-1}(A\setminus U),\\
                        R^{-1}(V) &= B\setminus R_{-1}(A\setminus V).
                    \end{align*}
                \item By \ChapterRef{\ChapterConstructionsWithSets, \cref{constructions-with-sets:properties-of-differences-functoriality} of \cref{constructions-with-sets:properties-of-differences}}{\cref{properties-of-differences-functoriality} of \cref{properties-of-differences}} we have $R_{-1}(A\setminus V)\subset R_{-1}(A\setminus U)$.
                \item By assumption, we then have $A\setminus V\subset A\setminus U$.
                \item By \ChapterRef{\ChapterConstructionsWithSets, \cref{constructions-with-sets:properties-of-differences-functoriality} of \cref{constructions-with-sets:properties-of-differences}}{\cref{properties-of-differences-functoriality} of \cref{properties-of-differences}} again, we have $U\subset V$.
            \end{itemize}
        \item\SloganFont{\cref{2-categorical-epimorphisms-in-rel-corepresentably-full-morphisms-in-rel-2}$\implies$\cref{2-categorical-epimorphisms-in-rel-corepresentably-full-morphisms-in-rel-4}: }Consider the diagram
            \[
                \begin{tikzcd}[row sep={4.0*\the\DL,between origins}, column sep={4.0*\the\DL,between origins}, background color=backgroundColor, ampersand replacement=\&]
                    A
                    \arrow[r, "R",mid vert]
                    \&
                    B
                    \arrow[r, "S", shift left=0.8, mid vert]
                    \arrow[r, "T"', shift right=0.8, mid vert]
                    \&
                    X\mrp{,}
                \end{tikzcd}
            \]%
            and suppose that $S\procirc R\subset T\procirc R$. Note that, by assumption, given a diagram of the form
            \[
                \begin{tikzcd}[row sep={4.0*\the\DL,between origins}, column sep={4.0*\the\DL,between origins}, background color=backgroundColor, ampersand replacement=\&]
                    A
                    \arrow[r, "R",mid vert]
                    \&
                    B
                    \arrow[r, "U", shift left=0.8, mid vert]
                    \arrow[r, "V"', shift right=0.8, mid vert]
                    \&
                    \pt\mrp{,}
                \end{tikzcd}
            \]%
            if $R^{-1}(U)=R\procirc U\subset R\procirc V=R^{-1}(V)$, then $U\subset V$. In particular, for each $x\in X$, we may consider the diagram
            \[
                \begin{tikzcd}[row sep={4.0*\the\DL,between origins}, column sep={4.0*\the\DL,between origins}, background color=backgroundColor, ampersand replacement=\&]
                    A
                    \arrow[r, "R",mid vert]
                    \&
                    B
                    \arrow[r, "S",  shift left=0.8,  mid vert]
                    \arrow[r, "T"', shift right=0.8, mid vert]
                    \&
                    X
                    \arrow[r, "{[x]}", shift left=0.8, mid vert]
                    \&
                    \pt\mrp{,}
                \end{tikzcd}
            \]%
            for which we have $[x]\procirc S\procirc R\subset[x]\procirc T\procirc R$, implying that we have
            \[
                S^{-1}(x)%
                =%
                [x]\procirc S%
                \subset%
                [x]\procirc T%
                =%
                T^{-1}(x)
            \]%
            for each $x\in X$, implying $S\subset T$.
        \item\SloganFont{\cref{2-categorical-monomorphisms-in-rel-representably-full-morphisms-in-rel-4}$\implies$\cref{2-categorical-monomorphisms-in-rel-representably-full-morphisms-in-rel-2}: }Let $U,V\in\mathcal{P}(B)$ and consider the diagram
            \[
                \begin{tikzcd}[row sep={4.0*\the\DL,between origins}, column sep={4.0*\the\DL,between origins}, background color=backgroundColor, ampersand replacement=\&]
                    A
                    \arrow[r, "R",mid vert]
                    \&
                    B
                    \arrow[r, "U", shift left=0.8, mid vert]
                    \arrow[r, "V"', shift right=0.8, mid vert]
                    \&
                    \pt\mrp{.}
                \end{tikzcd}
            \]%
            By \cref{unwinding-the-direct-image-function-associated-to-a-relation}, we have
            \begin{align*}
                R^{-1}(U) &= U\procirc R,\\
                R^{-1}(V) &= V\procirc R.
            \end{align*}
            Now, if $R^{-1}(U)\subset R^{-1}(V)$, i.e.\ $U\procirc R\subset V\procirc R$, then $U\subset V$ by assumption.
    \end{itemize}

    \ProofBox{\cref{2-categorical-epimorphisms-in-rel-corepresentably-fully-faithful-morphisms-in-rel}: Corepresentably Fully Faithful Morphisms in $\sfbfRel$}%
    This follows from \cref{2-categorical-epimorphisms-in-rel-corepresentably-faithful-morphisms-in-rel,2-categorical-epimorphisms-in-rel-corepresentably-full-morphisms-in-rel}.
\end{Proof}
\begin{question}{Better Characterisations of Corepresentably Full Morphisms in $\sfbfRel$}{better-characterisations-of-corepresentably-full-morphisms-in-rel}%
    \cref{2-categorical-epimorphisms-in-rel-corepresentably-full-morphisms-in-rel} of \cref{2-categorical-epimorphisms-in-rel} gives a characterisation of the corepresentably full morphisms in $\sfbfRel$.

    Are there other nice characterisations of these?

    This question also appears as \cite{MO467527}.
\end{question}
\subsection{Co/Limits in $\sfRel$}\label{subsection-co-limits-in-rel}
\begin{proposition}{Co/Limits in $\sfRel$}{co-limits-in-rel}%
    This will be properly written later on.
    %The category $\sfRel$ admits some co/limits, but not all, as does its $2$-categorical counterpart:
    %\begin{enumerate}
    %    \item\SloganFont{Zero Objects. }The category $\Rel$ has a zero object, the empty set $\emptyset$.
    %    \item\SloganFont{Co/Products. }The category $\Rel$ has co/products, both given by disjoint union of sets.
    %    \item\SloganFont{Lack of Co/Equalisers. }The category $\Rel$ does not have co/equalisers.
    %    \item\SloganFont{Limits of Graphs of Functions. }The category $\Rel$ has limits whose arrows are all graphs of functions.
    %    \item\SloganFont{Colimits of Graphs of Functions. }The category $\Rel$ has colimits whose arrows are all graphs of functions, and these agree with the corresponding limits in $\Sets$.
    %\end{enumerate}
\end{proposition}
\begin{Proof}{Proof of \cref{co-limits-in-rel}}%
    Omitted.
\end{Proof}
\subsection{Kan Extensions and Kan Lifts in $\sfbfRel$}\label{subsection-kan-extensions-and-kan-lifts-in-rel}
\begin{remark}{Kan Extensions and Kan Lifts in $\sfbfRel$}{kan-extensions-and-kan-lifts-in-rel}%
    The $2$-category $\sfbfRel$ admits all right Kan extensions and right Kan lifts, though not all left Kan extensions and neither does it admit all left Kan lifts. See \ChapterRef{\ChapterConstructionsWithRelations, \cref{constructions-with-relations:section-kan-extensions-and-kan-lifts-in-the-2-category-of-relations}}{\cref{section-kan-extensions-and-kan-lifts-in-the-2-category-of-relations}} for a detailed discussion of this.
\end{remark}
\subsection{Closedness of $\sfbfRel$}\label{subsection-closedness-of-rel}
\begin{proposition}{Closedness of $\sfbfRel$}{closedness-of-rel}%
    The $2$-category $\sfbfRel$ is a closed bicategory, there being, for each $R\colon A\rightproarrow B$ and set $X$, a pair of adjunctions
    \begin{webcompile}
        \begin{gathered}
            \Adjunction#R^{*}#\Ran_{R}#\Rel(B,X)#\Rel(A,X),#\\
            \Adjunction#R_{*}#\Rift_{R}#\Rel(X,A)#\Rel(X,B),#
        \end{gathered}
    \end{webcompile}%
    witnessed by bijections
    \begin{align*}
        \eRel(S\procirc R,T) &\cong \eRel(S,\Ran_{R}(T)),\\
        \eRel(R\procirc U,V) &\cong \eRel(U,\Rift_{R}(V)),
    \end{align*}
    natural in $S\in\Rel(B,X)$, $T\in\Rel(A,X)$, $U\in\Rel(X,A)$, and $V\in\Rel(X,B)$.
\end{proposition}
\begin{Proof}{Proof of \cref{closedness-of-rel}}%
    This follows from \ChapterRef{\ChapterConstructionsWithRelations, \cref{constructions-with-relations:existence-of-right-kan-extensions-in-rel,constructions-with-relations:existence-of-right-kan-lifts-in-rel}}{\cref{existence-of-right-kan-extensions-in-rel,existence-of-right-kan-lifts-in-rel}}.
\end{Proof}
\subsection{$\sfbfRel$ as a Category of Free Algebras}\label{subsection-rel-as-a-category-of-free-algebras}
\begin{proposition}{$\sfbfRel$ as a Category of Free Algebras}{rel-as-a-category-of-free-algebras}%
    We have an isomorphism of categories
    \[
        \Rel%
        \cong%
        \FreeAlg_{\mathcal{P}_{*}}(\Sets),
    \]%
    where $\mathcal{P}_{*}$ is the powerset monad of \ChapterRef{\ChapterMonads, \cref{monads:the-powerset-monad}}{\cref{the-powerset-monad}}.
\end{proposition}
\begin{Proof}{Proof of \cref{rel-as-a-category-of-free-algebras}}%
    Omitted.
\end{Proof}
\section{The Left Skew Monoidal Structure on $\eRel(A,B)$}\label{section-the-left-skew-monoidal-structure-on-rel-a-b}
\subsection{The Left Skew Monoidal Product}\label{subsection-the-left-skew-monoidal-structure-on-rel-a-b-the-left-skew-monoidal-product}
Let $A$ and $B$ be sets and let $J\colon A\rightproarrow B$ be a relation.
\begin{definition}{The Left $J$-Skew Monoidal Product of $\eRel(A,B)$}{the-left-j-skew-monoidal-product-of-erelab}%
    The \textbf{left $J$-skew monoidal product of $\eRel(A,B)$} is the functor
    \[
        \lhd_{J}%
        \colon
        \eRel(A,B)\times\eRel(A,B)
        \to
        \eRel(A,B)
    \]%
    where
    \begin{itemize}
        \item\SloganFont{Action on Objects. }For each $R,S\in\Obj(\eRel(A,B))$, we have
            \begin{webcompile}
                S\lhd_{J}R%
                \defeq%
                S\procirc\Rift_{J}(R),
                \quad
                \begin{tikzcd}[row sep={5.0*\the\DL,between origins}, column sep={5.0*\the\DL,between origins}, background color=backgroundColor, ampersand replacement=\&]
                    \&
                    A
                    \arrow[r,mid vert,"S"]
                    \arrow[d,mid vert,"J"]
                    \&
                    B\mrp{.}
                    \\
                    A
                    \arrow[ru,mid vert,"{\Rift_{J}(R)}",densely dashed]
                    \arrow[r, mid vert,"R"',""'{pos=0.325,name=R}]
                    \&
                    B
                    \&
                    % 2-Arrows
                    \arrow[from=1-2,to=R,shorten <= 0.5em,shorten >= 0.75em,Rightarrow,start anchor={[xshift=0.2*\the\DL]}]%
                \end{tikzcd}
            \end{webcompile}%
        \item\SloganFont{Action on Morphisms. }For each $R,S,R',S'\in\Obj(\eRel(A,B))$, the action on $\Hom$-sets
            \begin{envscriptsize}
                \[
                    (\lhd_{J})_{(G,F),(G',F')}
                    \colon%
                    \Hom_{\eRel(A,B)}(S,S')\times\Hom_{\eRel(A,B)}(R,R')
                    \to
                    \Hom_{\eRel(A,B)}(S\lhd_{J}R,S'\lhd_{J}R')
                \]%
            \end{envscriptsize}
            of $\lhd_{J}$ at $((R,S),(R',S'))$ is defined by%
            %--- Begin Footnote ---%
            \footnote{%
                Since $\eRel(A,B)$ is posetal, this is to say that if $S\subset S'$ and $R\subset R'$, then $S\lhd_{J}R\subset S'\lhd_{J}R'$.
                \par\vspace*{\TCBBoxCorrection}
            }%
            %---  End Footnote  ---%
            \begin{webcompile}
                \beta\lhd_{J}\alpha%
                \defeq%
                \beta\procirc\Rift_{J}(\alpha),%
                \begin{tikzcd}[row sep={8.0*\the\DL,between origins}, column sep={10.0*\the\DL,between origins}, background color=backgroundColor, ampersand replacement=\&]
                    \&
                    A
                    \arrow[r,bend left  = 30,mid vert,"S"{name=S}]
                    \arrow[r,bend right = 30,mid vert,"S'"'{name=Sprime}]
                    \arrow[d,mid vert,"J"]
                    \&[-3.0*\the\DL]
                    B
                    \\
                    A
                    \arrow[ru,bend left  = 35, mid vert,"{\Rift_{J}(R)}"{sloped,name=RiftJR},densely dashed]
                    \arrow[ru,bend right = 0, mid vert,"{\Rift_{J}(R')}"'{sloped,name=RiftJRprime},densely dashed]
                    \arrow[r, mid vert=0.35,"R"{description,name=R}]
                    \arrow[r, mid vert,bend right=40,"R'"'{name=Rprime}]
                    \&
                    B
                    \&[-3.0*\the\DL]
                    % 2-Arrows
                    \arrow[from=R,to=Rprime,shorten <= -0.25em,shorten >= 0.45em,Rightarrow,"\alpha"'{pos=0.1}]%
                    \arrow[from=RiftJR,to=RiftJRprime,shorten <= 0.45em,shorten >= 0.45em,Rightarrow,"\Rift_{J}(\alpha)"'{pos=0.3,rotate=37.5}]%
                    \arrow[from=S,to=Sprime,shorten=0.45em,Rightarrow,"\beta"'{pos=0.475}]%
                    \arrow[from=1-2,to=R,shorten <= 0.5em,shorten >= -0.25em,Rightarrow,start anchor={[xshift=0.2*\the\DL]}]%
                \end{tikzcd}
            \end{webcompile}%
            for each $\beta\in\Hom_{\eRel(A,B)}(S,S')$ and each $\alpha\in\Hom_{\eRel(A,B)}(R,R')$.
    \end{itemize}
\end{definition}
\subsection{The Left Skew Monoidal Unit}\label{subsection-the-left-skew-monoidal-structure-on-rel-a-b-the-left-skew-monoidal-unit}
Let $A$ and $B$ be sets and let $J\colon A\rightproarrow B$ be a relation.
\begin{definition}{The Left $J$-Skew Monoidal Unit of $\eRel(A,B)$}{the-left-j-skew-monoidal-unit-of-erelab}%
    The \textbf{left $J$-skew monoidal unit of $\eRel(A,B)$} is the functor
    \[
        \Unit^{\eRel(A,B)}_{\lhd_{J}}
        \colon
        \PunctualCategory
        \to
        \eRel(A,B)
    \]
    picking the object
    \[
        \Unit^{\lhd_{J}}_{\eRel(A,B)}%
        \defeq%
        J
    \]%
    of $\eRel(A,B)$.
\end{definition}
\subsection{The Left Skew Associators}\label{subsection-the-left-skew-monoidal-structure-on-rel-a-b-the-skew-associators}
Let $A$ and $B$ be sets and let $J\colon A\rightproarrow B$ be a relation.
\begin{definition}{The Left $J$-Skew Associator of $\eRel(A,B)$}{the-left-j-skew-associator-of-erelab}%
    The \textbf{left $J$-skew associator of $\eRel(A,B)$} is the natural transformation
    \begin{envsmallsize}
        \[
            \alpha^{\eRel(A,B),\lhd_{J}}%
            \colon%
            {\lhd_{J}}\circ{({\lhd_{J}}\times\sfid)}%
            \Longrightarrow%
            {\lhd_{J}}\circ{(\sfid\times{\lhd_{J}})}\circ{\bfalpha^{\Cats}_{\eRel(A,B),\eRel(A,B),\eRel(A,B)}},%
        \]
    \end{envsmallsize}
    as in the diagram
    \[
        \begin{tikzcd}[row sep={0*\the\DL,between origins}, column sep={0*\the\DL,between origins}, background color=backgroundColor, ampersand replacement=\&]
            \&[0.30901699437\ThreeCm]
            \&[0.5\ThreeCm]
            {\eRel(A,B)\times(\eRel(A,B)\times\eRel(A,B))}
            \&[0.5\ThreeCm]
            \&[0.30901699437\ThreeCm]
            \\[0.58778525229\ThreeCm]
            {(\eRel(A,B)\times\eRel(A,B))\times\eRel(A,B)}
            \&[0.30901699437\ThreeCm]
            \&[0.5\ThreeCm]
            \&[0.5\ThreeCm]
            \&[0.30901699437\ThreeCm]
            {\eRel(A,B)\times\eRel(A,B)}
            \\[0.95105651629\ThreeCm]
            \&[0.30901699437\ThreeCm]
            {\eRel(A,B)\times\eRel(A,B)}
            \&[0.5\ThreeCm]
            \&[0.5\ThreeCm]
            {\eRel(A,B)\mrp{,}}
            \&[0.30901699437\ThreeCm]
            % 1-Arrows
            % Left Boundary
            \arrow[from=2-1,to=1-3,"{\bfalpha^{\Cats}_{\eRel(A,B),\eRel(A,B),\eRel(A,B)}}"{pos=0.35},isoarrowprime]%
            \arrow[from=1-3,to=2-5,"{\sfid\times{\lhd_{J}}}"{pos=0.575},""{name=2}]%
            \arrow[from=2-5,to=3-4,"\lhd_{J}"{pos=0.425}]%
            % Right Boundary
            \arrow[from=2-1,to=3-2,"{{\lhd_{J}}\times\sfid}"'{pos=0.425}]%
            \arrow[from=3-2,to=3-4,"\lhd_{J}"']%
            % 2-Arrows
            \arrow[from=3-2,to=2,"\alpha^{\eRel(A,B),\lhd_{J}}"{description,pos=0.475},Rightarrow,shorten <= 0.5*\the\DL,shorten >= 1*\the\DL]%
        \end{tikzcd}
    \]%
    whose component
    \[
        \alpha^{\eRel(A,B),\lhd_{J}}_{T,S,R}%
        \colon%
        \underbrace{(T\lhd_{J}S)\lhd_{J}R}_{\defeq T\procirc\Rift_{J}(S)\procirc\Rift_{J}(R)}%
        \hookrightarrow
        \underbrace{T\lhd_{J}(S\lhd_{J}R)}_{\defeq T\procirc\Rift_{J}(S\procirc\Rift_{J}(R))}%
    \]%
    at $(T,S,R)$ is given by
    \[
        \alpha^{\eRel(A,B),\lhd_{J}}_{T,S,R}%
        \defeq%
        \id_{T}\procirc\gamma,
    \]%
    where
    \[
        \gamma%
        \colon%
        \Rift_{J}(S)\procirc\Rift_{J}(R)
        \hookrightarrow
        \Rift_{J}(S\procirc\Rift_{J}(R))
    \]%
    is the inclusion adjunct to the inclusion
    \[
        \epsilon_{S}\twocirc\id_{\Rift_{J}(R)}
        \colon
        \underbrace{J\procirc\Rift_{J}(S)\procirc\Rift_{J}(R)}_{\defeq J_{*}(\Rift_{J}(S)\procirc\Rift_{J}(R))}
        \hookrightarrow
        S\procirc\Rift_{J}(R)
    \]%
    under the adjunction $J_{*}\dashv\Rift_{J}$, where $\epsilon\colon{J\procirc\Rift_{J}}\Longrightarrow\id_{\eRel(A,B)}$ is the counit of the adjunction $J_{*}\dashv\Rift_{J}$.
\end{definition}
\subsection{The Left Skew Left Unitors}\label{subsection-the-left-skew-monoidal-structure-on-rel-a-b-the-left-skew-left-unitors}
Let $A$ and $B$ be sets and let $J\colon A\rightproarrow B$ be a relation.
\begin{definition}{The Left $J$-Skew Left Unitor of $\eRel(A,B)$}{the-left-j-skew-left-unitor-of-erelab}%
    The \textbf{left $J$-skew left unitor of $\eRel(A,B)$} is the natural transformation
    \[
        \LUnitor^{\eRel(A,B),\lhd_{J}}
        \colon
        {\lhd_{J}}\circ{({\Unit^{\eRel(A,B)}_{\lhd_{J}}}\times{\sfid})}
        \Longrightarrow
        \bfLUnitor^{\TwoCategoryOfCategories}_{\eRel(A,B)}%
    \]%
    as in the diagram
    \[
        \begin{tikzcd}[row sep={9.0*\the\DL,between origins}, column sep={14.0*\the\DL,between origins}, background color=backgroundColor, ampersand replacement=\&]
            {\PunctualCategory\times\eRel(A,B)}
            \arrow[r,  "{\Unit^{\eRel(A,B)}_{\lhd_{J}}\times\sfid}"]
            \arrow[rd, dashed,"\bfLUnitor^{\TwoCategoryOfCategories}_{\eRel(A,B)}"'{name=1,pos=0.475},bend right=30]
            \&
            \eRel(A,B)\times\eRel(A,B)
            \arrow[d, "\lhd_{J}"]
            \\
            {}
            \&
            \eRel(A,B)\mathrlap{,}
            % 2-Arrows
            \arrow[Rightarrow,from=1-2,to=1,shorten >=1.0*\the\DL,shorten <=1.0*\the\DL,"\LUnitor^{\eRel(A,B),\lhd_{J}}"description]
        \end{tikzcd}
    \]%
    whose component
    \[
        \LUnitor^{\eRel(A,B),\lhd_{J}}_{R}
        \colon
        \underbrace{J\lhd_{J}R}_{\defeq J\procirc\Rift_{J}(R)}
        \hookrightarrow
        R
    \]%
    at $R$ is given by
    \[
        \LUnitor^{\eRel(A,B),\lhd_{J}}_{R}%
        \defeq%
        \epsilon_{R},
    \]%
    where $\epsilon\colon J_{*}\procirc\Rift_{J}\Longrightarrow\id_{\eRel(A,B)}$ is the counit of the adjunction $J_{*}\dashv\Rift_{J}$.
\end{definition}
\subsection{The Left Skew Right Unitors}\label{subsection-the-left-skew-monoidal-structure-on-rel-a-b-the-left-skew-right-unitors}
Let $A$ and $B$ be sets and let $J\colon A\rightproarrow B$ be a relation.
\begin{definition}{The Left $J$-Skew Right Unitor of $\eRel(A,B)$}{the-left-j-skew-right-unitor-of-erelab}%
    The \textbf{left $J$-skew right unitor of $\eRel(A,B)$} is the natural transformation
    \[
        \RUnitor^{\eRel(A,B),\lhd_{J}}
        \colon
        \bfRUnitor^{\TwoCategoryOfCategories}_{\eRel(A,B)}
        \Longrightarrow
        {\lhd_{J}}\circ{(\sfid\times\Unit^{\eRel(A,B)}_{\lhd_{J}})}
    \]
    as in the diagram
    \[
        \begin{tikzcd}[row sep={9.0*\the\DL,between origins}, column sep={14.0*\the\DL,between origins}, background color=backgroundColor, ampersand replacement=\&]
            {\eRel(A,B)\times\PunctualCategory}
            \arrow[r, "\sfid\times\Unit^{\eRel(A,B)}_{\lhd_{J}}"]
            \arrow[rd, dashed,"\bfRUnitor^{\TwoCategoryOfCategories}_{\eRel(A,B)}"'{name=1,pos=0.475},bend right=30]
            \&
            {\eRel(A,B)\times\eRel(A,B)\mrp{,}}
            \arrow[d, "\lhd_{J}"]
            \\
            {}
            \&
            {\eRel(A,B)}
            % 2-Arrows
            \arrow[Rightarrow,from=1,to=1-2,shorten >=1.0*\the\DL,shorten <=1.0*\the\DL,"\RUnitor^{\eRel(A,B),\lhd_{J}}"description]
        \end{tikzcd}
    \]%
    whose component
    \[
        \RUnitor^{\eRel(A,B),\lhd_{J}}_{R}%
        \colon%
        R%
        \hookrightarrow
        \underbrace{R\lhd_{J}J}_{\defeq R\procirc\Rift_{J}(J)}%
    \]%
    at $R$ is given by the composition
    \begin{align*}
        R &\mkern10mu\mrp{\Longrightisoarrow}\mkern50mu                               R\procirc\chi_{A}\\
          &\mkern10mu\mrp{\xLongrightarrow{\id_{R}\procirc\eta_{\chi_{A}}}}\mkern50mu R\procirc\Rift_{J}(J_{*}(\chi_{A}))\\
          &\mkern10mu\mrp{\defeq}\mkern50mu                                           R\procirc\Rift_{J}(J\procirc\chi_{A})\\
          &\mkern10mu\mrp{\Longrightisoarrow}\mkern50mu                               R\procirc\Rift_{J}(J)\\
          &\mkern10mu\mrp{\defeq}\mkern50mu                                           R\lhd_{J}J,
    \end{align*}
    where $\eta\colon\id_{\eRel(A,A)}\Longrightarrow\Rift_{J}\circ J_{*}$ is the unit of the adjunction $J_{*}\dashv\Rift_{J}$.
\end{definition}
\subsection{The Left Skew Monoidal Structure on $\eRel(A,B)$}\label{subsection-the-left-skew-monoidal-structure-on-rel-a-b}
\begin{proposition}{The Left $J$-Skew Monoidal Structure on $\eRel(A,B)$}{the-left-j-skew-monoidal-structure-on-erelab}%
    The category $\eRel(A,B)$ admits a left skew monoidal category structure consisting of%
    \begin{itemize}
        \item\SloganFont{The Underlying Category. }The posetal category associated to the poset $\eRel(A,B)$ of relations from $A$ to $B$ of \cref{the-set-of-relations-between-two-sets-2} of \cref{the-set-of-relations-between-two-sets}.
        \item\SloganFont{The Left Skew Monoidal Product. }The left $J$-skew monoidal product
            \[
                \lhd_{J}%
                \colon%
                \eRel(A,B)\times\eRel(A,B)%
                \to%
                \eRel(A,B)
            \]%
            of \cref{the-left-j-skew-monoidal-product-of-erelab}.
        \item\SloganFont{The Left Skew Monoidal Unit. }The functor
            \[
                \Unit^{\eRel(A,B),\lhd_{J}}
                \colon
                \PunctualCategory
                \to
                \eRel(A,B)
            \]
            of \cref{the-left-j-skew-monoidal-unit-of-erelab}.
        \item\SloganFont{The Left Skew Associators. }The natural transformation
            \begin{envsmallsize}
                \[
                    \alpha^{\eRel(A,B),\lhd_{J}}%
                    \colon%
                    {\lhd_{J}}\circ{({\lhd_{J}}\times\sfid)}%
                    \Longrightarrow%
                    {\lhd_{J}}\circ{(\sfid\times{\lhd_{J}})}\circ{\bfalpha^{\Cats}_{\eRel(A,B),\eRel(A,B),\eRel(A,B)}}%
                \]
            \end{envsmallsize}
            of \cref{the-left-j-skew-associator-of-erelab}.
        \item\SloganFont{The Left Skew Left Unitors. }The natural transformation
            \[
                \LUnitor^{\eRel(A,B),\lhd_{J}}
                \colon
                {\lhd_{J}}\circ{({\Unit^{\eRel(A,B)}_{\lhd_{J}}}\times{\sfid})}
                \Longrightarrow
                \bfLUnitor^{\TwoCategoryOfCategories}_{\eRel(A,B)}%
            \]
            of \cref{the-left-j-skew-left-unitor-of-erelab}.
        \item\SloganFont{The Left Skew Right Unitors. }The natural transformation
            \[
                \RUnitor^{\eRel(A,B),\lhd_{J}}
                \colon
                \bfRUnitor^{\TwoCategoryOfCategories}_{\eRel(A,B)}
                \Longrightarrow
                {\lhd_{J}}\circ{(\sfid\times\Unit^{\eRel(A,B)}_{\lhd_{J}})}
            \]
            of \cref{the-left-j-skew-right-unitor-of-erelab}.
    \end{itemize}
\end{proposition}
\begin{Proof}{Proof of \cref{the-left-j-skew-monoidal-structure-on-erelab}}%
    Since $\eRel(A,B)$ is posetal, the commutativity of the pentagon identity, the left skew left triangle identity, the left skew right triangle identity, the left skew middle triangle identity, and the zigzag identity is automatic (\ChapterRef{\ChapterCategories, \cref{categories:properties-of-posetal-categories-automatic-commutativity-of-diagrams} of \cref{categories:properties-of-posetal-categories}}{\cref{categories:properties-of-posetal-categories-automatic-commutativity-of-diagrams} of \cref{categories:properties-of-posetal-categories}}), and thus $\eRel(A,B)$ together with the data in the statement forms a left skew monoidal category.
\end{Proof}
\section{The Right Skew Monoidal Structure on $\eRel(A,B)$}\label{section-the-right-skew-monoidal-structure-on-rel-a-b}
Let $A$ and $B$ be sets and let $J\colon A\rightproarrow B$ be a relation.
\subsection{The Right Skew Monoidal Product}\label{subsection-the-right-skew-monoidal-structure-on-rel-a-b-the-right-skew-monoidal-product}
\begin{definition}{The Right $J$-Skew Monoidal Product of $\eRel(A,B)$}{the-right-j-skew-monoidal-product-of-erelab}%
    The \textbf{right $J$-skew monoidal product of $\eRel(A,B)$} is the functor
    \[
        \rhd_{J}%
        \colon
        \eRel(A,B)\times\eRel(A,B)
        \to
        \eRel(A,B)
    \]%
    where
    \begin{itemize}
        \item\SloganFont{Action on Objects. }For each $R,S\in\Obj(\eRel(A,B))$, we have
            \begin{webcompile}
                S\rhd_{J}R%
                \defeq%
                \Ran_{J}(S)\procirc R,
                \quad
                \begin{tikzcd}[row sep={5.0*\the\DL,between origins}, column sep={5.0*\the\DL,between origins}, background color=backgroundColor, ampersand replacement=\&]
                    A
                    \arrow[r,"R",mid vert]
                    \&
                    B
                    \arrow[r,"{\Ran_{J}(S)}",densely dashed,mid vert]
                    \&
                    B\mrp{.}
                    \\
                    \&
                    A
                    \arrow[u,"J",mid vert]
                    \arrow[ru,"S"'{name=S},mid vert]
                    \&
                    % 2-Arrows
                    \arrow[from=S,to=1-2,shorten <= 0.5em,shorten >= -0.125em,Rightarrow]%
                \end{tikzcd}
            \end{webcompile}%
        \item\SloganFont{Action on Morphisms. }For each $R,S,R',S'\in\Obj(\eRel(A,B))$, the action on $\Hom$-sets
            \begin{envscriptsize}
                \[
                    (\rhd_{J})_{(S,R),(S',R')}
                    \colon%
                    \Hom_{\eRel(A,B)}(S,S')\times\Hom_{\eRel(A,B)}(R,R')
                    \to
                    \Hom_{\eRel(A,B)}(S\rhd_{J}R,S'\rhd_{J}R')
                \]%
            \end{envscriptsize}
            of $\rhd_{J}$ at $((S,R),(S',R'))$ is defined by%
            %--- Begin Footnote ---%
            \footnote{%
                Since $\eRel(A,B)$ is posetal, this is to say that if $S\subset S'$ and $R\subset R'$, then $S\rhd_{J}R\subset S'\rhd_{J}R'$.
                \par\vspace*{\TCBBoxCorrection}
            }%
            %---  End Footnote  ---%
            \begin{webcompile}
                \beta\rhd_{J}\alpha%
                \defeq%
                \Ran_{J}(\beta)\procirc\alpha,%
                \begin{tikzcd}[row sep={8.0*\the\DL,between origins}, column sep={10.0*\the\DL,between origins}, background color=backgroundColor, ampersand replacement=\&]
                    A
                    \arrow[r,"R'",""{name=Rprime},  bend left  = 30,mid vert]
                    \arrow[r,"R"',""'{name=R},      bend right = 30,mid vert]
                    \&[-3.0*\the\DL]
                    B
                    \arrow[r,"{\Ran_{J}(S')}",          ,""{name=RanJSprime},bend left  = 25,densely dashed,mid vert]
                    \arrow[r,"{\Ran_{J}(S)}"'description,""'{name=RanJS},    bend right = 25,densely dashed,mid vert=0.25]
                    \&
                    B
                    \\
                    \&[-3.0*\the\DL]
                    A
                    \arrow[u,"J",mid vert]
                    \arrow[ru,bend left  = 0,"S'"'description,""'{name=Sprime},mid vert=0.35]
                    \arrow[ru,bend right = 40,"S"'{pos=0.525},""'{name=S,pos=0.535},mid vert=0.525]
                    \&
                    % 2-Arrows
                    \arrow[from=Sprime,to=1-2,shorten <= 0.75em,shorten >= -0.125em,Rightarrow]%
                    \arrow[from=R,to=Rprime,shorten=0.45em,Rightarrow,"\alpha"{pos=0.475}]%
                    \arrow[from=S,to=Sprime,shorten <= 0.5em,shorten >= -1.125em,Rightarrow,"\beta"{pos=1.2}]%
                    \arrow[from=RanJS,to=RanJSprime,shorten <= 0.5em,shorten >= 0.5em,Rightarrow,"\Ran_{J}(\beta)"{pos=0.475}]%
                \end{tikzcd}
            \end{webcompile}%
            for each $\beta\in\Hom_{\eRel(A,B)}(S,S')$ and each $\alpha\in\Hom_{\eRel(A,B)}(R,R')$.
    \end{itemize}
\end{definition}
\subsection{The Right Skew Monoidal Unit}\label{subsection-the-right-skew-monoidal-structure-on-rel-a-b-the-right-skew-monoidal-unit}
\begin{definition}{The Right $J$-Skew Monoidal Unit of $\eRel(A,B)$}{the-right-j-skew-monoidal-unit-of-erelab}%
    The \textbf{right $J$-skew monoidal unit of $\eRel(A,B)$} is the functor
    \[
        \Unit^{\eRel(A,B)}_{\rhd_{J}}
        \colon
        \PunctualCategory
        \to
        \eRel(A,B)
    \]
    picking the object
    \[
        \Unit^{\rhd_{J}}_{\eRel(A,B)}%
        \defeq%
        J
    \]%
    of $\eRel(A,B)$.
\end{definition}
\subsection{The Right Skew Associators}\label{subsection-the-right-skew-monoidal-structure-on-rel-a-b-the-right-skew-associators}
\begin{definition}{The Right $J$-Skew Associator of $\eRel(A,B)$}{the-right-j-skew-associator-of-erelab}%
    The \textbf{right $J$-skew associator of $\eRel(A,B)$} is the natural transformation
    \begin{envsmallsize}
        \[
            \alpha^{\eRel(A,B),\rhd_{J}}%
            \colon%
            {\rhd_{J}}\circ{(\sfid\times{\rhd_{J}})}%
            \Longrightarrow%
            {\rhd_{J}}\circ{({\rhd_{J}}\times\sfid)}\circ{\bfalpha^{\Cats,-1}_{\eRel(A,B),\eRel(A,B),\eRel(A,B)}},%
        \]
    \end{envsmallsize}
    as in the diagram
    \[
        \begin{tikzcd}[row sep={0*\the\DL,between origins}, column sep={0*\the\DL,between origins}, background color=backgroundColor, ampersand replacement=\&]
            \&[0.30901699437\ThreeCm]
            \&[0.5\ThreeCm]
            {(\eRel(A,B)\times\eRel(A,B))\times\eRel(A,B)}
            \&[0.5\ThreeCm]
            \&[0.30901699437\ThreeCm]
            \\[0.58778525229\ThreeCm]
            {\eRel(A,B)\times(\eRel(A,B)\times\eRel(A,B))}
            \&[0.30901699437\ThreeCm]
            \&[0.5\ThreeCm]
            \&[0.5\ThreeCm]
            \&[0.30901699437\ThreeCm]
            {\eRel(A,B)\times\eRel(A,B)}
            \\[0.95105651629\ThreeCm]
            \&[0.30901699437\ThreeCm]
            {\eRel(A,B)\times\eRel(A,B)}
            \&[0.5\ThreeCm]
            \&[0.5\ThreeCm]
            {\eRel(A,B)\mrp{,}}
            \&[0.30901699437\ThreeCm]
            % 1-Arrows
            % Left Boundary
            \arrow[from=2-1,to=1-3,"{\bfalpha^{\Cats,-1}_{\eRel(A,B),\eRel(A,B),\eRel(A,B)}}"{pos=0.35},isoarrowprime]%
            \arrow[from=1-3,to=2-5,"{{\rhd_{J}}\times\sfid}"{pos=0.575},""{name=2}]%
            \arrow[from=2-5,to=3-4,"\rhd_{J}"{pos=0.425}]%
            % Right Boundary
            \arrow[from=2-1,to=3-2,"{\sfid\times{\rhd_{J}}}"'{pos=0.425}]%
            \arrow[from=3-2,to=3-4,"\rhd_{J}"']%
            % 2-Arrows
            \arrow[from=3-2,to=2,"\alpha^{\eRel(A,B),\rhd_{J}}"{description,pos=0.475},Rightarrow,shorten <= 0.5*\the\DL,shorten >= 1*\the\DL]%
        \end{tikzcd}
    \]%
    whose component
    \[
        \alpha^{\eRel(A,B),\rhd_{J}}_{T,S,R}%
        \colon%
        \underbrace{T\rhd_{J}(S\rhd_{J}R)}_{\defeq\Ran_{J}(T)\procirc\Ran_{J}(S)\procirc R}%
        \hookrightarrow
        \underbrace{(T\rhd_{J}S)\rhd_{J}R}_{\defeq\Ran_{J}(\Ran_{J}(T)\procirc S)\procirc R}%
    \]%
    at $(T,S,R)$ is given by
    \[
        \alpha^{\eRel(A,B),\rhd}_{T,S,R}%
        \defeq%
        \gamma\procirc\id_{R},
    \]%
    where
    \[%
        \gamma%
        \colon%
        \Ran_{J}(T)\procirc\Ran_{J}(S)
        \hookrightarrow
        \Ran_{J}(\Ran_{J}(T)\procirc S)
    \]%
    is the inclusion adjunct to the inclusion
    \[
        \id_{\Ran_{J}(T)}\procirc\epsilon_{S}%
        \colon%
        \underbrace{\Ran_{J}(T)\procirc\Ran_{J}(S)\procirc J}_{\defeq J^{*}(\Ran_{J}(T)\procirc\Ran_{J}(S))}%
        \hookrightarrow
        \Ran_{J}(T)\procirc S%
    \]%
    under the adjunction $J^{*}\dashv\Ran_{J}$, where $\epsilon\colon{\Ran_{J}}\procirc{J}\Longrightarrow\id_{\eRel(A,B)}$ is the counit of the adjunction $J^{*}\dashv\Ran_{J}$.
\end{definition}
\subsection{The Right Skew Left Unitors}\label{subsection-the-right-skew-monoidal-structure-on-rel-a-b-the-right-skew-left-unitors}
\begin{definition}{The Right $J$-Skew Left Unitor of $\eRel(A,B)$}{the-right-j-skew-left-unitor-of-erelab}%
    The \textbf{right $J$-skew left unitor of $\eRel(A,B)$} is the natural transformation
    \[
        \LUnitor^{\eRel(A,B),\rhd_{J}}
        \colon
        \bfLUnitor^{\TwoCategoryOfCategories}_{\eRel(A,B)}
        \Longrightarrow
        {\rhd_{J}}\circ{(\Unit^{\eRel(A,B)}_{\rhd}\times\sfid)},
    \]
    as in the diagram
    \[
        \begin{tikzcd}[row sep={9.0*\the\DL,between origins}, column sep={14.0*\the\DL,between origins}, background color=backgroundColor, ampersand replacement=\&]
            {\PunctualCategory\times\eRel(A,B)}
            \arrow[r,  "{\Unit^{\eRel(A,B)}_{\rhd_{J}}\times\sfid}"]
            \arrow[rd, dashed,"\bfLUnitor^{\TwoCategoryOfCategories}_{\eRel(A,B)}"'{name=1,pos=0.475},bend right=30]
            \&
            \eRel(A,B)\times\eRel(A,B)
            \arrow[d, "\rhd_{J}"]
            \\
            {}
            \&
            \eRel(A,B)\mathrlap{,}
            % 2-Arrows
            \arrow[Rightarrow,from=1,to=1-2,shorten >=1.0*\the\DL,shorten <=1.0*\the\DL,"\LUnitor^{\eRel(A,B),\rhd_{J}}"description]
        \end{tikzcd}
    \]%
    whose component
    \[
        \LUnitor^{\eRel(A,B),\rhd_{J}}_{R}
        \colon
        R
        \hookrightarrow
        \underbrace{J\rhd_{J}R}_{\defeq\Ran_{J}(J)\procirc R}
    \]%
    at $R$ is given by the composition
    \begin{align*}
        R &\mkern10mu\mrp{\Longrightisoarrow}\mkern50mu                               \chi_{B}\procirc R\\
          &\mkern10mu\mrp{\xLongrightarrow{\eta_{\chi_{B}}}\procirc\id_{R}}\mkern50mu \Ran_{J}(J^{*}(\chi_{A}))\procirc R\\
          &\mkern10mu\mrp{\defeq}\mkern50mu                                           \Ran_{J}(J^{*}\procirc\chi_{A})\procirc R\\
          &\mkern10mu\mrp{\Longrightisoarrow}\mkern50mu                               \Ran_{J}(J)\procirc R\\
          &\mkern10mu\mrp{\defeq}\mkern50mu                                           R\rhd_{J}J,
    \end{align*}
    where $\eta\colon\id_{\eRel(B,B)}\Longrightarrow\Ran_{J}\circ J^{*}$ is the unit of the adjunction $J^{*}\dashv\Ran_{J}$.
\end{definition}
\subsection{The Right Skew Right Unitors}\label{subsection-the-right-skew-monoidal-structure-on-rel-a-b-the-right-skew-right-unitors}
\begin{definition}{The Right $J$-Skew Right Unitor of $\eRel(A,B)$}{the-right-j-skew-right-unitor-of-erelab}%
    The \textbf{right $J$-skew right unitor of $\eRel(A,B)$} is the natural transformation
    \[
        \RUnitor^{\eRel(A,B),\rhd_{J}}
        \colon
        {\rhd_{J}}\circ{(\sfid\times\Unit^{\eRel(A,B)}_{\rhd})}
        \Longrightarrow
        \bfRUnitor^{\TwoCategoryOfCategories}_{\eRel(A,B)},
    \]
    as in the diagram
    \[
        \begin{tikzcd}[row sep={9.0*\the\DL,between origins}, column sep={14.0*\the\DL,between origins}, background color=backgroundColor, ampersand replacement=\&]
            {\eRel(A,B)\times\PunctualCategory}
            \arrow[r, "\sfid\times\Unit^{\eRel(A,B)}_{\rhd_{J}}"]
            \arrow[rd, dashed,"\bfRUnitor^{\TwoCategoryOfCategories}_{\eRel(A,B)}"'{name=1,pos=0.475},bend right=30]
            \&
            {\eRel(A,B)\times\eRel(A,B)\mrp{,}}
            \arrow[d, "\rhd_{J}"]
            \\
            {}
            \&
            {\eRel(A,B)}
            % 2-Arrows
            \arrow[Rightarrow,from=1-2,to=1,shorten >=1.0*\the\DL,shorten <=1.0*\the\DL,"\RUnitor^{\eRel(A,B),\rhd_{J}}"description]
        \end{tikzcd}
    \]%
    whose component
    \[
        \RUnitor^{\eRel(A,B),\rhd_{J}}_{S}%
        \colon%
        \underbrace{S\rhd_{J}J}_{\defeq\Ran_{J}(S)\procirc J}%
        \hookrightarrow%
        S%
    \]%
    at $S$ is given by%
    \[
        \RUnitor^{\eRel(A,B),\rhd_{J}}_{S}%
        \defeq%
        \epsilon_{R},
    \]%
    where $\epsilon\colon J^{*}\circ\Ran_{J}\Longrightarrow\id_{\eRel(A,B)}$ is the counit of the adjunction $J^{*}\dashv\Ran_{J}$.
\end{definition}
\subsection{The Right Skew Monoidal Structure on $\eRel(A,B)$}\label{subsection-the-right-skew-monoidal-structure-on-rel-a-b}
\begin{proposition}{The Right $J$-Skew Monoidal Structure on $\eRel(A,B)$}{the-right-j-skew-monoidal-structure-on-erelab}%
    The category $\eRel(A,B)$ admits a right skew monoidal category structure consisting of%
    \begin{itemize}
        \item\SloganFont{The Underlying Category. }The posetal category associated to the poset $\eRel(A,B)$ of relations from $A$ to $B$ of \cref{the-set-of-relations-between-two-sets-2} of \cref{the-set-of-relations-between-two-sets}.
        \item\SloganFont{The Right Skew Monoidal Product. }The right $J$-skew monoidal product
            \[
                \lhd_{J}%
                \colon%
                \eRel(A,B)\times\eRel(A,B)%
                \to%
                \eRel(A,B)
            \]%
            of \cref{the-right-j-skew-monoidal-product-of-erelab}.
        \item\SloganFont{The Right Skew Monoidal Unit. }The functor
            \[
                \Unit^{\eRel(A,B),\lhd_{J}}
                \colon
                \PunctualCategory
                \to
                \eRel(A,B)
            \]
            of \cref{the-right-j-skew-monoidal-unit-of-erelab}.
        \item\SloganFont{The Right Skew Associators. }The natural transformation
            \begin{envsmallsize}
                \[
                    \alpha^{\eRel(A,B),\rhd_{J}}%
                    \colon%
                    {\rhd_{J}}\circ{(\sfid\times{\rhd_{J}})}%
                    \Longrightarrow%
                    {\rhd_{J}}\circ{({\rhd_{J}}\times\sfid)}\circ{\bfalpha^{\Cats,-1}_{\eRel(A,B),\eRel(A,B),\eRel(A,B)}}%
                \]
            \end{envsmallsize}
            of \cref{the-right-j-skew-associator-of-erelab}.
        \item\SloganFont{The Right Skew Left Unitors. }The natural transformation
            \[
                \LUnitor^{\eRel(A,B),\rhd_{J}}
                \colon
                \bfLUnitor^{\TwoCategoryOfCategories}_{\eRel(A,B)}
                \Longrightarrow
                {\rhd_{J}}\circ{(\Unit^{\eRel(A,B)}_{\rhd}\times\sfid)}
            \]
            of \cref{the-right-j-skew-left-unitor-of-erelab}.
        \item\SloganFont{The Right Skew Right Unitors. }The natural transformation
            \[
                \RUnitor^{\eRel(A,B),\rhd_{J}}
                \colon
                {\rhd_{J}}\circ{(\sfid\times\Unit^{\eRel(A,B)}_{\rhd})}
                \Longrightarrow
                \bfRUnitor^{\TwoCategoryOfCategories}_{\eRel(A,B)}
            \]
            of \cref{the-right-j-skew-right-unitor-of-erelab}.
    \end{itemize}
\end{proposition}
\begin{Proof}{Proof of \cref{the-right-j-skew-monoidal-structure-on-erelab}}%
    Since $\eRel(A,B)$ is posetal, the commutativity of the pentagon identity, the right skew left triangle identity, the right skew right triangle identity, the right skew middle triangle identity, and the zigzag identity is automatic (\ChapterRef{\ChapterCategories, \cref{categories:properties-of-posetal-categories-automatic-commutativity-of-diagrams} of \cref{categories:properties-of-posetal-categories}}{\cref{categories:properties-of-posetal-categories-automatic-commutativity-of-diagrams} of \cref{categories:properties-of-posetal-categories}}), and thus $\eRel(A,B)$ together with the data in the statement forms a right skew monoidal category.
\end{Proof}
\begin{appendices}
\begin{multicols}{2}[\section{Other Chapters}]
\noindent
\textbf{Preliminaries}
\begin{enumerate}
\item \hyperref[introduction:section-phantom]{Introduction}
\end{enumerate}
\textbf{Sets}
\begin{enumerate}
\setcounter{enumi}{2}
\item \hyperref[sets:section-phantom]{Sets}
\item \hyperref[constructions-with-sets:section-phantom]{Constructions With Sets}
\item \hyperref[monoidal-structures-on-the-category-of-sets:section-phantom]{Monoidal Structures on the Category of Sets}
\item \hyperref[pointed-sets:section-phantom]{Pointed Sets}
\item \hyperref[tensor-products-of-pointed-sets:section-phantom]{Tensor Products of Pointed Sets}
\end{enumerate}
\textbf{Relations}
\begin{enumerate}
\setcounter{enumi}{6}
\item \hyperref[relations:section-phantom]{Relations}
\item \hyperref[constructions-with-relations:section-phantom]{Constructions With Relations}
\item \hyperref[conditions-on-relations:section-phantom]{Conditions on Relations}
\end{enumerate}
\textbf{Category Theory}
\begin{enumerate}
\setcounter{enumi}{9}
\item \hyperref[categories:section-phantom]{Categories}
\end{enumerate}
\textbf{Monoidal Categories}
\begin{enumerate}
\setcounter{enumi}{10}
\item \hyperref[constructions-with-monoidal-categories:section-phantom]{Constructions With Monoidal Categories}
\end{enumerate}
\textbf{Bicategories}
\begin{enumerate}
\setcounter{enumi}{11}
\item \hyperref[types-of-morphisms-in-bicategories:section-phantom]{Types of Morphisms in Bicategories}
\end{enumerate}
\textbf{Extra Part}
\begin{enumerate}
\setcounter{enumi}{12}
\item \hyperref[notes:section-phantom]{Notes}
\end{enumerate}
\end{multicols}

\end{appendices}
\end{document}
