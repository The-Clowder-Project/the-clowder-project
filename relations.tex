\input{preamble}

% OK, start here.
%
\usepackage{fontspec}
\let\hyperwhite\relax
\let\hyperred\relax
\newcommand{\hyperwhite}{\hypersetup{citecolor=white,filecolor=white,linkcolor=white,urlcolor=white}}
\newcommand{\hyperred}{%
\hypersetup{%
    citecolor=TitlingRed,%
    filecolor=TitlingRed,%
    linkcolor=TitlingRed,%
     urlcolor=TitlingRed%
}}
\let\ChapterRef\relax
\newcommand{\ChapterRef}[2]{#1}
\setcounter{tocdepth}{2}
%▓▓▓▓▓▓▓▓▓▓▓▓▓▓▓▓▓▓▓▓▓▓▓▓▓▓▓▓▓▓▓▓▓
%▓▓ ╔╦╗╦╔╦╗╦  ╔═╗  ╔═╗╔═╗╔╗╔╔╦╗ ▓▓
%▓▓  ║ ║ ║ ║  ║╣   ╠╣ ║ ║║║║ ║  ▓▓
%▓▓  ╩ ╩ ╩ ╩═╝╚═╝  ╚  ╚═╝╝╚╝ ╩  ▓▓
%▓▓▓▓▓▓▓▓▓▓▓▓▓▓▓▓▓▓▓▓▓▓▓▓▓▓▓▓▓▓▓▓▓
%\usepackage{titlesec}
%▓▓▓▓▓▓▓▓▓▓▓▓▓▓▓▓▓▓▓▓▓▓▓▓▓▓▓▓▓▓▓▓▓▓▓▓▓▓▓▓▓▓▓▓▓▓▓▓▓▓▓▓▓▓▓
%▓▓ ╔╦╗╔═╗╔╗ ╦  ╔═╗  ╔═╗╔═╗  ╔═╗╔═╗╔╗╔╔╦╗╔═╗╔╗╔╔╦╗╔═╗ ▓▓
%▓▓  ║ ╠═╣╠╩╗║  ║╣   ║ ║╠╣   ║  ║ ║║║║ ║ ║╣ ║║║ ║ ╚═╗ ▓▓
%▓▓  ╩ ╩ ╩╚═╝╩═╝╚═╝  ╚═╝╚    ╚═╝╚═╝╝╚╝ ╩ ╚═╝╝╚╝ ╩ ╚═╝ ▓▓
%▓▓▓▓▓▓▓▓▓▓▓▓▓▓▓▓▓▓▓▓▓▓▓▓▓▓▓▓▓▓▓▓▓▓▓▓▓▓▓▓▓▓▓▓▓▓▓▓▓▓▓▓▓▓▓
\newcommand{\ChapterTableOfContents}{%
    \begingroup
    \addfontfeature{Numbers={Lining,Monospaced}}
    \hypersetup{hidelinks}\tableofcontents%
    \endgroup
}%

\makeatletter
\newcommand \DotFill {\leavevmode \cleaders \hb@xt@ .33em{\hss .\hss }\hfill \kern \z@}
\makeatother

\definecolor{ToCGrey}{rgb}{0.4,0.4,0.4}
\definecolor{mainColor}{rgb}{0.82745098,0.18431373,0.18431373}
\usepackage{titletoc}
\titlecontents{part}
[0.0em]
{\addvspace{1pc}\color{TitlingRed}\large\bfseries\text{Part }}
{\bfseries\textcolor{TitlingRed}{\contentslabel{0.0em}}\hspace*{1.35em}}
{}
{\textcolor{TitlingRed}{{\hfill\bfseries\contentspage\nobreak}}}
[]
\titlecontents{section}
[0.0em]
{\addvspace{1pc}}
{\color{black}\bfseries\textcolor{TitlingRed}{\contentslabel{0.0em}}\hspace*{1.35em}}
{}
{\textcolor{black}{\textbf{\DotFill}{\bfseries\contentspage\nobreak}}}
[]
\titlecontents{subsection}
[0.0em]
{}
{\hspace*{1.35em}\color{ToCGrey}{\contentslabel{0.0em}}\hspace*{2.1em}}
{}
{{\textcolor{ToCGrey}\DotFill}\textcolor{ToCGrey}{\contentspage}\nobreak}
[]
\usepackage{marginnote}
\renewcommand*{\marginfont}{\normalfont}
\usepackage{inconsolata}
\setmonofont{inconsolata}%
\let\ChapterRef\relax
\newcommand{\ChapterRef}[2]{#1}
\AtBeginEnvironment{subappendices}{%%
    \section*{\huge Appendices}%
}%

\begin{document}

\title{Relations}

\maketitle

\phantomsection
\label{section-phantom}

This chapter contains some material about relations. Notably, we discuss and explore:
\begin{enumerate}
    \item\label{relations-introduction-item-1}The definition of relations (\cref{subsection-relations-foundations}).
    \item\label{relations-introduction-item-2}How relations may be viewed as decategorification of profunctors (\cref{subsection-relations-as-decategorifications-of-profunctors}).
    \item\label{relations-introduction-item-3}The various kinds of categories that relations form, namely:
        \begin{enumerate}
            \item\label{relations-introduction-item-3a}A category (\cref{subsection-the-category-of-relations}).
            \item\label{relations-introduction-item-3b}A monoidal category (\cref{subsection-the-closed-symmetric-monoidal-category-of-relations}).
            \item\label{relations-introduction-item-3c}A 2-category (\cref{subsection-the-2-category-of-relations}).
            \item\label{relations-introduction-item-3d}A double category (\cref{subsection-the-double-category-of-relations}).
        \end{enumerate}
    \item\label{relations-introduction-item-4}The various categorical properties of the 2-category of relations, including:
        \begin{enumerate}
            \item\label{relations-introduction-item-4a}The self-duality of $\sfRel$ and $\sfbfRel$ (\cref{self-duality-for-the-2-category-of-relations}).
            \item\label{relations-introduction-item-4b}Identifications of equivalences and isomorphisms in $\sfbfRel$ with bijections (\cref{isomorphisms-and-equivalences-in-rel}).
            \item\label{relations-introduction-item-4c}Identifications of adjunctions in $\sfbfRel$ with functions (\cref{adjunctions-in-rel}).
            \item\label{relations-introduction-item-4d}Identifications of monads in $\sfbfRel$ with preorders (\cref{monads-in-rel}).
            \item\label{relations-introduction-item-4e}Identifications of comonads in $\sfbfRel$ with subsets (\cref{comonads-in-rel}).
            \item\label{relations-introduction-item-4f}A description of the monoids and comonoids in $\sfbfRel$ with respect to the Cartesian product (\cref{co-monoids-in-rel}).
            \item\label{relations-introduction-item-4g}Characterisations of monomorphisms in $\sfRel$ (\cref{characterisations-of-monomorphisms-in-rel}).
            \item\label{relations-introduction-item-4h}Characterisations of 2-categorical notions of monomorphisms in $\sfbfRel$ (\cref{2-categorical-monomorphisms-in-rel}).
            \item\label{relations-introduction-item-4i}Characterisations of epimorphisms in $\sfRel$ (\cref{characterisations-of-epimorphisms-in-rel}).
            \item\label{relations-introduction-item-4j}Characterisations of 2-categorical notions of epimorphisms in $\sfbfRel$ (\cref{2-categorical-epimorphisms-in-rel}).
            \item\label{relations-introduction-item-4k}The partial co/completeness of $\sfRel$ (\cref{co-limits-in-rel}).
            \item\label{relations-introduction-item-4l}The existence or non-existence of Kan extensions and Kan lifts in $\sfRel$ (\cref{kan-extensions-and-kan-lifts-in-rel}).
            \item\label{relations-introduction-item-4m}The closedness of $\sfbfRel$ (\cref{closedness-of-rel}).
            \item\label{relations-introduction-item-4n}The identification of $\sfbfRel$ with the category of free algebras of the powerset monad on $\Sets$ (\cref{rel-as-a-category-of-free-algebras}).
        \end{enumerate}
    \item\label{relations-introduction-item-3}The adjoint pairs% TODO: FIX LABEL NUMBERING
        \begin{align*}
            R_{!}  \dashv R_{-1} &\colon \mathcal{P}(A) \rightleftarrows \mathcal{P}(B),\\
            R^{-1} \dashv R_{*}  &\colon \mathcal{P}(B) \rightleftarrows \mathcal{P}(A)
        \end{align*}
        of functors (morphisms of posets) between $\mathcal{P}(A)$ and $\mathcal{P}(B)$ induced by a relation $R\colon A\rightproarrow B$, as well as the properties of $R_{!}$, $R_{-1}$, $R^{-1}$, and $R_{*}$ (\cref{section-the-adjoint-pairs-r-shriek-r-minus-one-and-r-minus-one-r-star}).

        Of particular note are the following points:
        \begin{enumerate}
            \item\label{relations-introduction-item-3a}These two pairs of adjoint functors are the counterpart for relations of the adjoint triple $f_{!}\dashv f^{-1}\dashv f_{*}$ induced by a function $f\colon A\to B$ studied in \ChapterRef{\ChapterConstructionsWithSets, \cref{constructions-with-sets:section-the-adjoint-triple-f-shriek-f-minus-one-f-star}}{\cref{section-the-adjoint-triple-f-shriek-f-minus-one-f-star}}.
            \item\label{relations-introduction-item-3b}We have $R_{-1}=R^{-1}$ \textiff $R$ is total and functional (\cref{properties-of-strong-inverse-image-functions-associated-to-relations-interaction-with-weak-inverse-images-2} of \cref{properties-of-strong-inverse-image-functions-associated-to-relations}).
            \item\label{relations-introduction-item-3c}As a consequence of the previous item, when $R$ comes from a function $f$, the pair of adjunctions
                \[
                    R_{!}\dashv R_{-1}=R^{-1}\dashv R_{*}%
                \]%
                reduces to the triple adjunction
                \[
                    f_{!}\dashv f^{-1}\dashv f_{*}%
                \]%
                from \ChapterRef{\ChapterConstructionsWithSets, \cref{constructions-with-sets:section-the-adjoint-triple-f-shriek-f-minus-one-f-star}}{\cref{section-the-adjoint-triple-f-shriek-f-minus-one-f-star}}.
            \item\label{relations-introduction-item-3d}The pairs $R_{!}\dashv R_{-1}$ and $R^{-1}\dashv R_{*}$ turn out to be rather important later on, as they appear in the definition and study of continuous, open, and closed relations between topological spaces (\ChapterRef{\ChapterTopologicalSpaces, \cref{topological-spaces:section-relations-between-topological-spaces}}{\cref{section-relations-between-topological-spaces}}).
        \end{enumerate}
    \item\label{relations-introduction-item-5}A description of two notions of \say{skew composition} on $\eRel(A,B)$, giving rise to left and right skew monoidal structures analogous to the left skew monoidal structure on $\Fun(\CatFont{C},\CatFont{D})$ appearing in the definition of a relative monad (\cref{section-the-left-skew-monoidal-structure-on-rel-a-b,section-the-right-skew-monoidal-structure-on-rel-a-b}).
\end{enumerate}
This chapter is under revision. TODO:
\begin{enumerate}
    \item Revise \cref{section-properties-of-the-2-category-of-relations}
        \begin{enumerate}
            \item Co/limits in $\sfbfRel$.
        \end{enumerate}
    \item Replicate \cref{section-properties-of-the-2-category-of-relations} for apartness composition
    \item Revise \cref{section-the-adjoint-pairs-r-shriek-r-minus-one-and-r-minus-one-r-star}
    \item Add subsection \say{A Six Functor Formalism for Sets, Part 2}, now with relations, building upon \cref{section-the-adjoint-pairs-r-shriek-r-minus-one-and-r-minus-one-r-star}.
    \item Replicate \cref{section-the-adjoint-pairs-r-shriek-r-minus-one-and-r-minus-one-r-star} for apartness composition
    \item Revise sections on skew monoidal structures on $\eRel(A,B)$
    \item Replicate the sections on skew monoidal structures on $\eRel(A,B)$ for apartness composition.
    \item Explore relative co/monads in $\sfbfRel$, defined to be co/monoids in $\eRel(A,B)$ with its left/right skew monoidal structures of \ChapterRef{\ChapterRelations, \cref{relations:section-the-left-skew-monoidal-structure-on-rel-a-b,relations:section-the-right-skew-monoidal-structure-on-rel-a-b}}{\cref{section-the-left-skew-monoidal-structure-on-rel-a-b,section-the-right-skew-monoidal-structure-on-rel-a-b}}
    \item functional total relations defined with \say{satisfying the following equivalent conditions:}
    \item Add table:
        \begingroup%
        \setlength\cellspacetoplimit{3pt}
        \setlength\cellspacebottomlimit{3pt}
        \renewcommand{\arraystretch}{1.2}
        \begin{center}
            \begin{tabular}{|Sc|Sc|}\\\hline\rowcolor{darkRed}
                \textcolor{white}{\textbf{\textsc{Condition}}} & \textcolor{white}{\textbf{\textsc{Inclusion}}}       \\\hline\rowcolor{backgroundColor}
                $R$ is functional                              & $R\procirc R^{\dagger}\subset \Delta_{B}$            \\\rowcolor{backgroundColor}
                $R$ is surjective                              & $\Delta_{A}           \subset R^{\dagger}\procirc R$ \\\rowcolor{black!05!backgroundColor}
                $R$ is total                                   & $\Delta_{B}           \subset R\procirc R^{\dagger}$ \\\rowcolor{backgroundColor}
                $R$ is injective                               & $R^{\dagger}\procirc R\subset \Delta_{A}$            \\\hline
            \end{tabular}
        \end{center}
        \endgroup
\end{enumerate}

\ChapterTableOfContents

\section{Relations}\label{section-relations}
\subsection{Foundations}\label{subsection-relations-foundations}
Let $A$ and $B$ be sets.
\begin{definition}{Relations}{relations}%
    A \index[set-theory]{relation}\textbf{relation $R\colon A\rightproarrow B$ from $A$ to $B$}%
    %--- Begin Footnote ---%
    \footnote{%
        \SloganFont{Further Terminology: }Also called a \index[set-theory]{multivalued function!see {relation}}\textbf{multivalued function from $A$ to $B$}.
    }%
    %---  End Footnote  ---%
    %--- Begin Footnote ---%
    \footnote{%
        \SloganFont{Further Terminology: }When $A=B$, we also call $R\subset A\times A$ a \textbf{relation on $A$}.
        \par\vspace*{\TCBBoxCorrection}
    } %
    %---  End Footnote  ---%
    is equivalently:
    \begin{enumerate}
        \item\label{relations-1}A subset $R$ of $A\times B$.%
        \item\label{relations-2}A function from $A\times B$                                       to $\TV$.
        \item\label{relations-3}A function from $A$                                               to $\mathcal{P}(B)$.
        \item\label{relations-4}A function from $B$                                               to $\mathcal{P}(A)$.
        \item\label{relations-5}A cocontinuous morphism of posets from $(\mathcal{P}(A),\subset)$ to $(\mathcal{P}(B),\subset)$.
        \item\label{relations-6}A continuous morphism of posets from $(\mathcal{P}(B),\supset)$ to $(\mathcal{P}(A),\supset)$.
    \end{enumerate}
\end{definition}
\begin{Proof}{Proof of the Equivalences in \cref{relations}}%
    (We will prove that \cref{relations-1,relations-2,relations-3,relations-4,relations-5,relations-6} are indeed equivalent in a bit.)
\end{Proof}
\begin{remark}{Unwinding \cref{relations-1}, \rmI}{unwinding-relations-1}%
    We may think of a relation $R\colon A\rightproarrow B$ as a function from $A$ to $B$ that is \emph{multivalued}, assigning to each element $a$ in $A$ a set $R(a)$ of elements of $B$, thought of as the \emph{set of values of $R$ at $a$}.

    \indent Note that this includes also the possibility of $R$ having no value at all on a given $a\in A$ when $R(a)=\emptyset$.
\end{remark}
\begin{remark}{Unwinding \cref{relations-2}, \rmII}{unwinding-relations-2}%
    Another way of stating the equivalence between \cref{relations-1,relations-2,relations-3,relations-4,relations-5} of \cref{relations} is by saying that we have bijections of sets
    \begin{align*}
        \{\text{relations from $A$ to $B$}\} &\cong \mathcal{P}(A\times B)\\
                                             &\cong \Sets(A\times B,\TV)\\
                                             &\cong \Sets(A,\mathcal{P}(B))\\
                                             &\cong \Sets(B,\mathcal{P}(A))\\
                                             &\cong \CoContPos(\mathcal{P}(A),\mathcal{P}(B))\\
                                             &\cong \ContPos(\mathcal{P}(B),\mathcal{P}(A))
    \end{align*}
    natural in $A,B\in\Obj(\Sets)$, where $\mathcal{P}(A)$ and $\mathcal{P}(B)$ are endowed with the poset structure given by inclusion.
\end{remark}
\begin{Proof}{Proof of the Equivalences in \cref{relations}}%
    We claim that \cref{relations-1,relations-2,relations-3,relations-4,relations-5} are indeed equivalent:
    \begin{itemize}
        \item\SloganFont{\cref{relations-1}$\iff$\cref{relations-2}: }This is a special case of \ChapterRef{\ChapterConstructionsWithSets, \cref{constructions-with-sets:properties-of-characteristic-functions-of-subsets-bijectivity,constructions-with-sets:properties-of-characteristic-functions-of-subsets-naturality} of \cref{constructions-with-sets:properties-of-characteristic-functions-of-subsets}}{\cref{properties-of-characteristic-functions-of-subsets-bijectivity,properties-of-characteristic-functions-of-subsets-naturality} of \cref{properties-of-characteristic-functions-of-subsets}}.
        \item\SloganFont{\cref{relations-2}$\iff$\cref{relations-3}: }This follows from the bijections
            \begin{align*}
                \Sets(A\times B,\TV) &\cong \Sets(A,\Sets(B,\TV))\\
                                     &\cong \Sets(A,\mathcal{P}(B)),
            \end{align*}
            where the last bijection is from \ChapterRef{\ChapterConstructionsWithSets, \cref{constructions-with-sets:properties-of-characteristic-functions-of-subsets-bijectivity,constructions-with-sets:properties-of-characteristic-functions-of-subsets-naturality} of \cref{constructions-with-sets:properties-of-characteristic-functions-of-subsets}}{\cref{properties-of-characteristic-functions-of-subsets-bijectivity,properties-of-characteristic-functions-of-subsets-naturality} of \cref{properties-of-characteristic-functions-of-subsets}}.
        \item\SloganFont{\cref{relations-2}$\iff$\cref{relations-4}: }This follows from the bijections
            \begin{align*}
                \Sets(A\times B,\TV) &\cong \Sets(B,\Sets(B,\TV))\\
                                     &\cong \Sets(B,\mathcal{P}(A)),
            \end{align*}
            where again the last bijection is from \ChapterRef{\ChapterConstructionsWithSets, \cref{constructions-with-sets:properties-of-characteristic-functions-of-subsets-bijectivity,constructions-with-sets:properties-of-characteristic-functions-of-subsets-naturality} of \cref{constructions-with-sets:properties-of-characteristic-functions-of-subsets}}{\cref{properties-of-characteristic-functions-of-subsets-bijectivity,properties-of-characteristic-functions-of-subsets-naturality} of \cref{properties-of-characteristic-functions-of-subsets}}.
        \item\SloganFont{\cref{relations-2}$\iff$\cref{relations-5}: }This follows from the universal property of the powerset $\mathcal{P}(X)$ of a set $X$ as the free cocompletion of $X$ via the characteristic embedding
            \[
                \chi_{X}
                \colon
                X
                \hookrightarrow
                \mathcal{P}(X)
            \]%
            of $X$ into $\mathcal{P}(X)$, as in \ChapterRef{\ChapterConstructionsWithSets, \cref{constructions-with-sets:powersets-as-free-cocompletions-universal-property}}{\cref{powersets-as-free-cocompletions-universal-property}}. In particular, the bijection
            \[
                \Sets(A,\mathcal{P}(B))%
                \cong
                \CoContPos(\mathcal{P}(A),\mathcal{P}(B))
            \]%
            is given by extending each $f\colon A\to\mathcal{P}(B)$ in $\Sets(A,\mathcal{P}(B))$ from $A$ to all of $\mathcal{P}(A)$ by taking its left Kan extension along $\chi_{X}$, recovering the direct image function $f_{!}\colon\mathcal{P}(A)\to\mathcal{P}(B)$ of $f$ of \ChapterRef{\ChapterConstructionsWithSets, \cref{constructions-with-sets:the-direct-image-function-associated-to-a-function}}{\cref{the-direct-image-function-associated-to-a-function}}.
        \item\SloganFont{\cref{relations-5}$\iff$\cref{relations-6}: }Omitted.
    \end{itemize}
    This finishes the proof.
\end{Proof}
\begin{notation}{Further Notation for Relations}{further-notation-for-relations}%
    Let $A$ and $B$ be sets and let $R\colon\rightproarrow B$ be a relation from $A$ to $B$.
    \begin{enumerate}
        \item\label{further-notation-for-relations-the-set-of-relations-between-two-sets}We write \index[notation]{RelAB@$\Rel(A,B)$}$\Rel(A,B)$ for the set of relations from $A$ to $B$.
        \item\label{further-notation-for-relations-the-poset-of-relations-between-two-sets}We write \index[notation]{RelAB@$\eRel(A,B)$}$\eRel(A,B)$ for the sub-poset of $(\mathcal{P}(A\times B),\subset)$ spanned by the relations from $A$ to $B$.
        \item\label{further-notation-for-relations-a-simr-b}Given $a\in A$ and $b\in B$, we write $a\sim_{R}b$ to mean $(a,b)\in R$.
        \item\label{further-notation-for-relations-r-b-a}When viewing $R$ as a function
            \[
                R%
                \colon%
                A\times B%
                \to%
                \TTV,%
            \]%
            we write $R^{b}_{a}$ for the value of $R$ at $(a,b)$.%
            %--- Begin Footnote ---%
            \footnote{%
                The choice to write $R^{b}_{a}$ in place of $R^{a}_{b}$ is to keep the notation consistent with the notation we will later employ for profunctors in \ChapterProfunctors.
                \par\vspace*{\TCBBoxCorrection}
            }%
            %---  End Footnote  ---%
    \end{enumerate}
\end{notation}
\begin{proposition}{Properties of Relations}{properties-of-relations}%
    Let $A$ and $B$ be sets and let $R,S\colon A\rightproarrow B$ be relations.
    \begin{enumerate}
        \item\label{properties-of-relations-end-formula-for-the-set-of-inclusions-of-relations}\SloganFont{End Formula for the Set of Inclusions of Relations. }We have
            \[
                \Hom_{\eRel(A,B)}(R,S)%
                \cong%
                \int_{a\in A}\int_{b\in B}\Hom_{\TTV}(R^{b}_{a},S^{b}_{a}).%
            \]%
        %\item\label{properties-of-relations-}\SloganFont{. }
    \end{enumerate}
\end{proposition}
\begin{Proof}{Proof of \cref{properties-of-relations}}%
    \FirstProofBox{\cref{properties-of-relations-end-formula-for-the-set-of-inclusions-of-relations}: End Formula for the Set of Inclusions of Relations}%
    Unwinding the expression inside the end on the right hand side, we have
    \[
        \int_{a\in A}\int_{b\in B}\Hom_{\TTV}(R^{b}_{a},S^{b}_{a})%
        \cong%
        \begin{cases}%
            \pt       &\text{if, for each $a\in A$ and each $b\in B$,}\\%
                      &\text{we have $\Hom_{\TTV}(R^{b}_{a},S^{b}_{a})\cong\pt$}\\%
            \emptyset &\text{otherwise.}%
        \end{cases}%
    \]%
    Since we have $\Hom_{\TTV}(R^{b}_{a},S^{b}_{a})=\{\true\}\cong\pt$ exactly when $R^{b}_{a}=\false$ or $R^{b}_{a}=S^{b}_{a}=\true$, we get
    \[
        \int_{a\in A}\int_{b\in B}\Hom_{\TTV}(R^{b}_{a},S^{b}_{a})%
        \cong%
        \begin{cases}%
            \pt       &\text{if, for each $a\in A$ and each $b\in B$,}\\%
                      &\text{if $a\sim_{R}b$, then $a\sim_{S}b$,}\\%
            \emptyset &\text{otherwise.}%
        \end{cases}%
    \]%
    On the left hand-side, we have
    \[
        \Hom_{\eRel(A,B)}(R,S)%
        \cong%
        \begin{cases}%
            \pt       &\text{if $R\subset S$,}\\%
            \emptyset &\text{otherwise.}%
        \end{cases}%
    \]%
    Since $(a\sim_{R}b\implies a\sim_{S}b)$ \textiff $R\subset S$, the two sets above are isomorphic. This finishes the proof.
\end{Proof}
\subsection{Relations as Decategorifications of Profunctors}\label{subsection-relations-as-decategorifications-of-profunctors}
\begin{remark}{Relations as Decategorifications of Profunctors \rmI}{relations-as-decategorifications-of-profunctors-1}%
    The notion of a relation is a decategorification of that of a profunctor:
    \begin{enumerate}
        \item\label{relations-as-decategorifications-of-profunctors-1-item-1}A profunctor from a category $\CatFont{C}$ to a category $\CatFont{D}$ is a functor
            \[%
                \mathfrak{p}%
                \colon%
                \CatFont{D}^{\op}\times\CatFont{C}%
                \to%
                \Sets.%
            \]%
        \item\label{relations-as-decategorifications-of-profunctors-1-item-2}A relation on sets $A$ and $B$ is a function%
            \[%
                R%
                \colon%
                A\times B%
                \to%
                \TV.%
            \]%
    \end{enumerate}
    Here we notice that:
    \begin{itemize}
        \item The opposite $X^{\op}$ of a set $X$ is itself, as $(-)^{\op}\colon\Cats\to\Cats$ restricts to the identity endofunctor on $\Sets$.
        \item The values that profunctors and relations take are analogous:
            \begin{itemize}
                \item A category is enriched over the category
                    \[
                        \Sets%
                        \defeq%
                        \ZeroCats%
                    \]%
                    of sets, with profunctors taking values on it.
                \item A set is enriched over the set
                    \[
                        \TV%
                        \defeq%
                        \MinusOneCats%
                    \]%
                    of classical truth values, with relations taking values on it.
            \end{itemize}
    \end{itemize}
\end{remark}
\begin{remark}{Relations as Decategorifications of Profunctors \rmII}{relations-as-decategorifications-of-profunctors-2}%
    Extending \cref{relations-as-decategorifications-of-profunctors-1}, the equivalent definitions of relations in \cref{relations} are also related to the corresponding ones for profunctors (\cref{TODO}), which state that a profunctor $\mathfrak{p}\colon\CatFont{C}\rightproarrow\CatFont{D}$ is equivalently:
    \begin{enumerate}
        \item\label{profunctors-equivalent-characterisations-item-1}A                    functor $\mathfrak{p}\colon\CatFont{D}^{\op}\times\CatFont{C}\to\Sets$.
        \item\label{profunctors-equivalent-characterisations-item-2}A                    functor $\mathfrak{p}\colon\CatFont{C}\to\PSh(\CatFont{D})$.
        \item\label{profunctors-equivalent-characterisations-item-3}A                    functor $\mathfrak{p}\colon\CatFont{D}^{\op}\to\CoPSh(\CatFont{C})$.
        \item\label{profunctors-equivalent-characterisations-item-4}A colimit-preserving functor $\mathfrak{p}\colon\PSh(\CatFont{C})\to\PSh(\CatFont{D})$.
        \item\label{profunctors-equivalent-characterisations-item-5}A   limit-preserving functor $\mathfrak{p}\colon\CoPSh(\CatFont{D})^{\op}\to\CoPSh(\CatFont{C})^{\op}$.
    \end{enumerate}
    Indeed:
    \begin{itemize}
        \item The equivalence between \cref{profunctors-equivalent-characterisations-item-1,profunctors-equivalent-characterisations-item-2} (and also that between \cref{profunctors-equivalent-characterisations-item-1,profunctors-equivalent-characterisations-item-3}, which is proved analogously) is an instance of currying, both for profunctors as well as for relations, using the isomorphisms
            \begin{align*}
                \Sets(A\times B,\TV) &\cong \Sets(A,\Sets(B,\TV))\\
                                     &\cong \Sets(A,\mathcal{P}(B)),
            \end{align*}
            and
            \begin{align*}
                \Fun(\CatFont{D}^{\op}\times\CatFont{D},\Sets) &\cong \Fun(\CatFont{C},\Fun(\CatFont{D}^{\op},\Sets))\\
                                                               &\cong \Fun(\CatFont{C},\PSh(\CatFont{D})).
            \end{align*}
        \item The equivalence between \cref{profunctors-equivalent-characterisations-item-2,profunctors-equivalent-characterisations-item-4} follows from the universal properties of:
            \begin{itemize}
                \item The powerset $\mathcal{P}(X)$ of a set $X$ as the free cocompletion of $X$ via the characteristic embedding
                    \[
                        \chi_{(-)}
                        \colon
                        X
                        \hookrightarrow
                        \mathcal{P}(X)
                    \]%
                    of $X$ into $\mathcal{P}(X)$, as stated and proved in \ChapterRef{\ChapterConstructionsWithSets, \cref{constructions-with-sets:powersets-as-free-cocompletions-universal-property}}{\cref{powersets-as-free-cocompletions-universal-property}}.
                \item The category $\PSh(\CatFont{C})$ of presheaves on a category $\CatFont{C}$ as the free cocompletion of $\CatFont{C}$ via the Yoneda embedding
                    \[
                        \yo
                        \colon
                        \CatFont{C}
                        \hookrightarrow
                        \PSh(\CatFont{C})
                    \]%
                    of $\CatFont{C}$ into $\PSh(\CatFont{C})$, as stated and proved in \ChapterRef{\ChapterPresheavesAndTheYonedaLemma, \cref{presheaves-and-the-yoneda-lemma:properties-of-the-yoneda-embedding-as-the-free-cocompletion} of \cref{presheaves-and-the-yoneda-lemma:properties-of-the-yoneda-embedding}}{\cref{properties-of-the-yoneda-embedding-as-the-free-cocompletion} of \cref{properties-of-the-yoneda-embedding}}.
            \end{itemize}
        \item The equivalence between \cref{profunctors-equivalent-characterisations-item-3,profunctors-equivalent-characterisations-item-5} follows from the universal properties of:
            \begin{itemize}
                \item The powerset $\mathcal{P}(X)$ of a set $X$ as the free completion of $X$ via the characteristic embedding
                    \[
                        \chi_{(-)}
                        \colon
                        X
                        \hookrightarrow
                        \mathcal{P}(X)
                    \]%
                    of $X$ into $\mathcal{P}(X)$, as stated and proved in \ChapterRef{\ChapterConstructionsWithSets, \cref{constructions-with-sets:powersets-as-free-completions-universal-property}}{\cref{powersets-as-free-completions-universal-property}}.
                \item The category $\CoPSh(\CatFont{D})^{\op}$ of copresheaves on a category $\CatFont{D}$ as the free completion of $\CatFont{D}$ via the dual Yoneda embedding
                    \[
                        \coyo
                        \colon
                        \CatFont{D}
                        \hookrightarrow
                        \CoPSh(\CatFont{D})^{\op}
                    \]%
                    of $\CatFont{D}$ into $\CoPSh(\CatFont{D})^{\op}$, as stated and proved in \ChapterRef{\ChapterPresheavesAndTheYonedaLemma, \cref{presheaves-and-the-yoneda-lemma:properties-of-the-yoneda-embedding-as-the-free-completion} of \cref{presheaves-and-the-yoneda-lemma:properties-of-the-yoneda-embedding}}{\cref{properties-of-the-yoneda-embedding-as-the-free-completion} of \cref{properties-of-the-yoneda-embedding}}.
            \end{itemize}
    \end{itemize}
\end{remark}
\subsection{Composition of Relations}\label{subsection-composition-of-relations}
Let $A$, $B$, and $C$ be sets and let $R\colon A\rightproarrow B$ and $S\colon B\rightproarrow C$ be relations.
\begin{definition}{Composition of Relations}{composition-of-relations}%
    The \index[set-theory]{relation!composition of}\textbf{composition of $R$ and $S$} is the relation \index[notation]{SafterR@$S\procirc R$}$S\procirc R$ defined as follows:%
    \begin{enumerate}
        \item\label{composition-of-relations-1}Viewing relations from $A$ to $C$ as subsets of $A\times C$, we define
            \[
                S\procirc R
                \defeq
                \{%
                    (a,c)\in A\times C%
                    \ \middle|\ %
                    \begin{aligned}
                        &\text{there exists some $b\in B$ such}\\
                        &\text{that $a\sim_{R}b$ and $b\sim_{S}c$}%
                    \end{aligned}
                \}.%
            \]%
        \item\label{composition-of-relations-2}Viewing relations as functions $A\times B\to\TV$, we define
            \begin{align*}
                (S\procirc R)^{-_{1}}_{-_{2}} &\defeq \int^{b\in B}S^{-_{1}}_{b}\times R^{b}_{-_{2}}\\%
                                              &=      \bigvee_{b\in B}S^{-_{1}}_{b}\times R^{b}_{-_{2}},%
            \end{align*}
            where the join $\bigvee$ is taken in the poset $(\TV,\preceq)$ of \ChapterRef{\ChapterSets, \cref{sets:the-poset-of-truth-values}}{\cref{the-poset-of-truth-values}}.
        \item\label{composition-of-relations-3}Viewing relations as functions $A\to\mathcal{P}(B)$, we define
            \begin{webcompile}
                S\procirc R%
                \defeq%
                \Lan_{\chi_{B}}(S)\circ R,%
                \quad%
                \begin{tikzcd}[row sep={5.0*\the\DL,between origins}, column sep={5.0*\the\DL,between origins}, background color=backgroundColor, ampersand replacement=\&]
                    \&%
                    B%
                    \arrow[r, "S",""{name=S,pos=0.7}]%
                    \arrow[d, "\chi_{B}"', hook]%
                    \&%
                    \mathcal{P}(C)\mrp{,}%
                    \\%
                    A%
                    \arrow[r, "R"']%
                    \&%
                    \mathcal{P}(B)%
                    \arrow[ru, "\Lan_{\chi_{B}}(S)"', bend right=10]%
                    \&
                    % 2-Arrows
                    \arrow[from=S,to=2-2,shorten=0.75*\the\DL,Rightarrow]%
                \end{tikzcd}%
            \end{webcompile}
            where $\Lan_{\chi_{B}}(S)$ is computed by the formula
            \begin{align*}
                [\Lan_{\chi_{B}}(S)](V) &\cong \int^{b\in B}\chi_{\mathcal{P}(B)}(\chi_{b},V)\odot S(b)\\
                                        &\cong \int^{b\in B}\chi_{V}(b)\odot S(b)\\
                                        &\cong \bigcup_{b\in B}\chi_{V}(b)\odot S(b)\\
                                        &\cong \bigcup_{b\in V}S(b)
            \end{align*}
            for each $V\in\mathcal{P}(B)$, so we have%
            %--- Begin Footnote ---%
            \footnote{%
                That is: the relation $R$ may send $a\in A$ to a number of elements $\{b_{i}\}_{i\in I}$ in $B$, and then the relation $S$ may send the image of each of the $b_{i}$'s to a number of elements $\{S(b_{i})\}_{i\in I}=\{\{c_{j_{i}}\}_{j_{i}\in J_{i}}\}_{i\in I}$ in $C$.
                \par\vspace*{\TCBBoxCorrection}
            }%
            %---  End Footnote  ---%
            \begin{align*}
                [S\procirc R](a) &\defeq S(R(a))\\
                                 &\defeq \bigcup_{b\in R(a)}S(b).
            \end{align*}
            for each $a\in A$.
    \end{enumerate}
\end{definition}
\begin{remark}{Composing Relations With Right Kan Extensions}{composing-relations-with-right-kan-extensions}%
    You might wonder what happens if we instead define an alternative composition of relations $\mathord{\procirc'}$ via right Kan extensions. In this case, we would take the right Kan extension of $S$ along the dual characteristic embedding $B\hookrightarrow\mathcal{P}(B)^{\op}$:
    \begin{webcompile}
        S\mathbin{\mathord{\procirc}'}R%
        \defeq%
        \Ran_{\chi_{B}}(S)\circ R,%
        \quad%
        \begin{tikzcd}[row sep={5.0*\the\DL,between origins}, column sep={5.0*\the\DL,between origins}, background color=backgroundColor, ampersand replacement=\&]
            \&%
            B%
            \arrow[r, "S",""{name=S,pos=0.7}]%
            \arrow[d, "\chi_{B}"', hook]%
            \&%
            \mathcal{P}(C)\mrp{.}%
            \\%
            A%
            \arrow[r, "R"']%
            \&%
            \mathcal{P}(B)^{\op}%
            \arrow[ru, "\Ran_{\chi'_{B}}(S)"', bend right=10]%
            \&
            % 2-Arrows
            \arrow[from=S,to=2-2,shorten=0.75*\the\DL,Leftarrow]%
        \end{tikzcd}%
    \end{webcompile}
    In this case, we would have%
    %--- Begin Footnote ---%
    \footnote{%
        If we replace $R(a)$ with $B\setminus R(a)$, defining
        \[
            S\aptcirc R%
            \defeq%
            \bigcap_{b\in B\setminus R(a)}S(b),%
        \]%
        we instead obtain the apartness composition of relations; see \cref{subsection-apartness-composition-of-relations}.
        \par\vspace*{\TCBBoxCorrection}
    }%
    %---  End Footnote  ---%
    \[
        [S\mathbin{\mathord{\procirc}'}R](a)%
        \defeq%
        \bigcap_{b\in R(a)}S(b).
    \]%
    This alternative composition turns out to actually be a different kind of structure: it's an internal right Kan extension in $\sfbfRel$, namely $\Ran_{R^{\dagger}}(S)$ --- see \cref{subsection-internal-right-kan-extensions-in-rel}.
\end{remark}
\begin{example}{Examples of Composition of Relations}{examples-of-composition-of-relations}%
    Here are some examples of composition of relations.
    \begin{enumerate}
        \item\label{examples-of-composition-of-relations-1}\SloganFont{Composing Less/Greater Than Equal With Greater/Less Than Equal Signs. }Let $A=B=C=\R$. We have
            \begin{align*}
                \mathord{\leq}\procirc\mathord{\geq} &= \sim_{\triv},\\
                \mathord{\geq}\procirc\mathord{\leq} &= \sim_{\triv}.
            \end{align*}
        \item\label{examples-of-composition-of-relations-2}\SloganFont{Composing Less/Greater Than Equal Signs With Less/Greater Than Equal Signs. }Let $A=B=C=\R$. We have
            \begin{align*}
                \mathord{\leq}\procirc\mathord{\leq} &= \mathord{\leq},\\
                \mathord{\geq}\procirc\mathord{\geq} &= \mathord{\geq}.
            \end{align*}
    \end{enumerate}
\end{example}
\begin{proposition}{Properties of Composition of Relations}{properties-of-composition-of-relations}%
    Let $R\colon A\rightproarrow B$, $S\colon B\rightproarrow C$, and $T\colon C\rightproarrow D$ be relations.
    \begin{enumerate}
        \item\label{properties-of-composition-of-relations-functoriality}\SloganFont{Functoriality. }The assignments $R,S,(R,S)\mapsto S\procirc R$ define functors
            \[
                \BifunctorialityPeriod{S\procirc-}{-\procirc R}{-_{1}\procirc-_{2}}{\eRel(A,B)}{\eRel(B,C)}{\eRel(B,C)\times\eRel(A,B)}{\eRel(A,C)}%
            \]%
            In particular, given relations
            \[
                \begin{tikzcd}[row sep={4.0*\the\DL,between origins}, column sep={4.0*\the\DL,between origins}, background color=backgroundColor, ampersand replacement=\&]
                    A
                    \arrow[r,"R_{1}", mid vert,shift left=1.0]%
                    \arrow[r,"R_{2}"',mid vert,shift right=1.0]%
                    \&
                    B
                    \arrow[r,"S_{1}", mid vert,shift left=0.8]%
                    \arrow[r,"S_{2}"',mid vert,shift right=0.8]%
                    \&
                    C\mrp{,}
                \end{tikzcd}
            \]%
            if $R_{1}\subset R_{2}$ and $S_{1}\subset S_{2}$, then $S_{1}\procirc R_{1}\subset S_{2}\procirc R_{2}$.
        \item\label{properties-of-composition-of-relations-associativity}\SloganFont{Associativity. }We have
            \[
                (T\procirc S)\procirc R
                =
                T\procirc(S\procirc R).
            \]%
            That is, we have
            \[
                \bigcup_{b\in R(a)}\bigcup_{c\in S(b)}T(c)%
                =%
                \bigcup_{c\in\bigcup_{b\in R(a)}S(b)}T(c)%
            \]%
            for each $a\in A$.
        \item\label{properties-of-composition-of-relations-unitality}\SloganFont{Unitality. }We have
            \begin{align*}
                \Delta_{B}\procirc R &= R,\\
                R\procirc\Delta_{A}  &= R.
            \end{align*}
            That is, we have
            \begin{align*}
                \bigcup_{b\in R(a)}\{b\} &= R(a),\\
                \bigcup_{a\in\{a\}}R(a)  &= R(a)
            \end{align*}
            for each $a\in A$.
        \item\label{properties-of-composition-of-relations-linear-distributivity}\SloganFont{Linear Distributivity. }We have inclusions of relations
            \begin{align*}
                T\procirc(S\aptcirc R)  &\subset (T\procirc S)\aptcirc R,\\
                (T\aptcirc S)\procirc R &\subset T\aptcirc(S\procirc R).
            \end{align*}
            That is, we have
            \begin{align*}
                T(\bigcap_{b\in B\setminus R(a)}S(b))  &\subset \bigcap_{b\in B\setminus R(a)}T(S(b))\\
                \bigcup_{b\in R(a)}\bigcap_{c\in C\setminus S(b)}T(c) &\subset \bigcap_{c\in C\setminus S(R(a))}T(c)
            \end{align*}
            or, unwinding the expression for $S(R(a))$, we have
            \begin{align*}
                \bigcup_{c\in\bigcap_{b\in B\setminus R(a)}S(b)}T(c)  &\subset \bigcap_{b\in B\setminus R(a)}\bigcup_{c\in S(b)}T(c)\\
                \bigcup_{b\in R(a)}\bigcap_{c\in C\setminus S(b)}T(c) &\subset \bigcap_{c\in C\setminus\bigcup_{b\in R(a)}S(b)}T(c)
            \end{align*}
            for each $a\in A$.
        \item\label{properties-of-composition-of-relations-interaction-with-converses}\SloganFont{Interaction With Converses. }We have
            \[
                (S\procirc R)^{\dagger}
                =
                R^{\dagger}\procirc S^{\dagger}.
            \]%
        \item\label{properties-of-composition-of-relations-interaction-with-ranges-and-domains}\SloganFont{Interaction With Ranges and Domains. }We have
            \begin{align*}
                \dom(S\procirc R)   &\subset \dom(R),\\
                \range(S\procirc R) &\subset \range(S).
            \end{align*}
        %\item\label{properties-of-composition-of-relations-}\SloganFont{. }
    \end{enumerate}
\end{proposition}
\begin{Proof}{Proof of \cref{properties-of-composition-of-relations}}%
    \FirstProofBox{\cref{properties-of-composition-of-relations-functoriality}: Functoriality}%
    We have
    \begin{align*}
        S_{1}\procirc R_{1} &\defeq  \{%
                                         (a,c)\in A\times C%
                                         \ \middle|\ %
                                         \begin{aligned}
                                             &\text{there exists some $b\in B$, such}\\%
                                             &\text{that $a\sim_{R_{1}}b$ or $b\sim_{S_{1}}c$}%
                                         \end{aligned}
                                     \}\\%
                            &\subset \{%
                                        (a,c)\in A\times C%
                                        \ \middle|\ %
                                        \begin{aligned}
                                             &\text{there exists some $b\in B$, such}\\%
                                             &\text{that $a\sim_{R_{2}}b$ or $b\sim_{S_{2}}c$}%
                                        \end{aligned}
                                    \}\\%
                            &\defeq S_{2}\procirc R_{2}.
    \end{align*}
    This finishes the proof.

    \ProofBox{\cref{properties-of-composition-of-relations-associativity}: Associativity, Proof \rmI}%
    Indeed, we have
    \begin{align*}
        (T\procirc S)\procirc R &\defeq (\int^{c\in C}T^{-_{1}}_{c}\times S^{c}_{-_{2}})\procirc R\\
                                &\defeq \int^{b\in B}(\int^{c\in C}T^{-_{1}}_{c}\times S^{c}_{b})\times R^{b}_{-_{2}}\\
                                &=      \int^{b\in B}\int^{c\in C}(T^{-_{1}}_{c}\times S^{c}_{b})\times R^{b}_{-_{2}}\\
                                &=      \int^{c\in C}\int^{b\in B}(T^{-_{1}}_{c}\times S^{c}_{b})\times R^{b}_{-_{2}}\\
                                &=      \int^{c\in C}\int^{b\in B}T^{-_{1}}_{c}\times(S^{c}_{b}\times R^{b}_{-_{2}})\\
                                &=      \int^{c\in C}T^{-_{1}}_{c}\times(\int^{b\in B}S^{c}_{b}\times R^{b}_{-_{2}})\\
                                &\defeq \int^{c\in C}T^{-_{1}}_{c}\times(S\procirc R)^{c}_{-_{2}}\\
                                &\defeq T\procirc(S\procirc R).
    \end{align*}
    In the language of relations, given $a\in A$ and $d\in D$, the stated equality witnesses the equivalence of the following two statements:
    \begin{enumerate}
        \item\label{proof-of-properties-of-composition-of-relations-associativity-1}We have $a\sim_{(T\procirc S)\procirc R}d$, i.e.\ there exists some $b\in B$ such that:
            \begin{itemize}
                \item We have $a\sim_{R}b$;
                \item We have $b\sim_{T\procirc S}d$, i.e.\ there exists some $c\in C$ such that:
                    \begin{itemize}
                        \item We have $b\sim_{S}c$;
                        \item We have $c\sim_{T}d$;
                    \end{itemize}
            \end{itemize}
        \item\label{proof-of-properties-of-composition-of-relations-associativity-2}We have $a\sim_{T\procirc(S\procirc R)}d$, i.e.\ there exists some $c\in C$ such that:
            \begin{itemize}
                \item We have $a\sim_{S\procirc R}c$, i.e.\ there exists some $b\in B$ such that:
                    \begin{itemize}
                        \item We have $a\sim_{R}b$;
                        \item We have $b\sim_{S}c$;
                    \end{itemize}
                \item We have $c\sim_{T}d$;
            \end{itemize}
    \end{enumerate}
    both of which are equivalent to the statement
    \begin{itemize}
        \itemstar There exist $b\in B$ and $c\in C$ such that $a\sim_{R}b\sim_{S}c\sim_{T}d$.
    \end{itemize}

    \ProofBox{\cref{properties-of-composition-of-relations-associativity}: Associativity, Proof \rmII}%
    Using \cref{composition-of-relations-3} of \cref{composition-of-relations}, we have
    \begin{align*}
        [(T\procirc S)\procirc R](a) &\defeq \bigcup_{b\in R(a)}(T\procirc S)(b)\\
                                     &\defeq \bigcup_{b\in R(a)}\bigcup_{c\in S(b)}T(c)
    \end{align*}
    on the one hand and
    \begin{align*}
        [T\procirc(S\procirc R)](a) &\defeq \bigcup_{c\in[S\procirc R](a)}T(c)\\
                                    &\defeq \bigcup_{c\in\bigcup_{b\in R(a)}S(b)}T(c)
    \end{align*}
    on the other, so we want to prove an equality of the form
    \[
        \bigcup_{b\in R(a)}\bigcup_{c\in S(b)}T(c)%
        =%
        \bigcup_{c\in\bigcup_{b\in R(a)}S(b)}T(c).%
    \]%
    This then follows from an application of \ChapterRef{\ChapterConstructionsWithSets, \cref{constructions-with-sets:properties-of-unions-of-families-of-subsets-associativity} of \cref{constructions-with-sets:properties-of-unions-of-families-of-subsets}}{\cref{properties-of-unions-of-families-of-subsets-associativity} of \cref{properties-of-unions-of-families-of-subsets}} in which we consider $X=D$, consider $\mathcal{P}(\mathcal{P}(\mathcal{P}(D)))$, take $U=U_{c}=T(c)$, take $A$ to be%
    \[
        A_{b}%
        \defeq%
        \{T(c)\in\mathcal{P}(D)\ \middle|\ c\in S(b)\},%
    \]%
    and then finally take
    \begin{align*}
        \mathcal{A} &\defeq \{A_{b}\in\mathcal{P}(\mathcal{P}(D))\ \middle|\ b\in R(a)\}\\
                    &\defeq \{\{T(c)\in\mathcal{P}(D)\ \middle|\ c\in S(b)\}\ \middle|\ b\in R(a)\}.
    \end{align*}
    Indeed, we have
    \begin{align*}
        \bigcup_{A\in\mathcal{A}}(\bigcup_{U\in A}U) &= \bigcup_{A_{b}\in\mathcal{A}}(\bigcup_{c\in S(b)}T(c))\\
                                                     &= \bigcup_{b\in R(a)}(\bigcup_{c\in S(b)}T(c))
    \end{align*}
    and
    \begin{align*}
        \bigcup_{U\in\bigcup_{A\in\mathcal{A}}A}U &= \bigcup_{U_{c}\in\bigcup_{b\in R(a)}A_{b}}U_{c}\\%
                                                  &= \bigcup_{T(c)\in\bigcup_{b\in R(a)}A_{b}}T(c)\\%
                                                  &= \bigcup_{c\in\bigcup_{b\in R(a)}S(b)}T(c).%
    \end{align*}
    This finishes the proof.

    \ProofBox{\cref{properties-of-composition-of-relations-unitality}: Unitality}%
    Indeed, we have
    \begin{align*}
        \Delta_{B}\procirc R &\defeq \int^{b\in B}(\Delta_{B})^{-_{1}}_{b}\times R^{b}_{-_{2}}\\
                             &=      \bigvee_{b\in B}(\Delta_{B})^{-_{1}}_{b}\times R^{b}_{-_{2}}\\
                             &=      \bigvee_{\substack{b\in B\\b=-_{1}}}R^{b}_{-_{2}}\\
                             &=      R^{-_{1}}_{-_{2}},
    \end{align*}
    and
    \begin{align*}
        R\procirc\Delta_{A} &\defeq \int^{a\in A}R^{-_{1}}_{a}\times(\Delta_{A})^{a}_{-_{2}}\\
                            &=      \bigvee_{a\in B}R^{-_{1}}_{a}\times(\Delta_{A})^{a}_{-_{2}}\\
                            &=      \bigvee_{\substack{a\in B\\a=-_{2}}}R^{-_{1}}_{a}\\
                            &=      R^{-_{1}}_{-_{2}}.
    \end{align*}
    In the language of relations, given $a\in A$ and $b\in B$:
    \begin{itemize}
        \item The equality
            \[
                \Delta_{B}\procirc R%
                =%
                R%
            \]%
            witnesses the equivalence of the following two statements:
            \begin{itemize}
                \item We have $a\sim_{b}B$.
                \item There exists some $b'\in B$ such that:
                    \begin{itemize}
                        \item We have $a\sim_{R}b'$
                        \item We have $b'\sim_{\Delta_{B}}b$, i.e.\ $b'=b$.
                    \end{itemize}
            \end{itemize}
        \item The equality
            \[
                R\procirc\Delta_{A}%
                =%
                R%
            \]%
            witnesses the equivalence of the following two statements:
            \begin{itemize}
                \item There exists some $a'\in A$ such that:
                    \begin{itemize}
                        \item We have $a\sim_{\Delta_{B}}a'$, i.e.\ $a=a'$.
                        \item We have $a'\sim_{R}b$
                    \end{itemize}
                \item We have $a\sim_{b}B$.
            \end{itemize}
    \end{itemize}

    \ProofBox{\cref{properties-of-composition-of-relations-linear-distributivity}: Linear Distributivity}%
    This is a repetition of \cref{properties-of-apartness-composition-of-relations-linear-distributivity} of \cref{properties-of-apartness-composition-of-relations} and is proved there.

    \ProofBox{\cref{properties-of-composition-of-relations-interaction-with-converses}: Interaction With Converses}%
    This is a repetition of \cref{properties-of-converses-of-relations-interaction-with-composition} of \cref{properties-of-converses-of-relations} and is proved there.

    \ProofBox{\cref{properties-of-composition-of-relations-interaction-with-ranges-and-domains}: Interaction With Ranges and Domains}%
    We have
    \begin{align*}
        \dom(S\procirc R) &\defeq  \{a\in A\ \middle|\ a\sim_{S\procirc R}c\text{ for some $c\in C$}\},\\
                          &=       \{%
                                       a\in A%
                                       \ \middle|\ %
                                       \begin{aligned}
                                           &\text{there exists some $b\in B$ and $c\in C$}\\
                                           &\text{such that $a\sim_{R}b$ and $b\sim_{R}c$}
                                       \end{aligned}
                                   \},\\
                          &\subset \{%
                                       a\in A%
                                       \ \middle|\ %
                                       \begin{aligned}
                                           &\text{there exists some $b\in B$}\\
                                           &\text{such that $a\sim_{R}b$}
                                       \end{aligned}
                                   \},\\
                          &\defeq  \dom(R)
    \end{align*}
    and
    \begin{align*}
        \range(S\procirc R) &\defeq  \{c\in C\ \middle|\ a\sim_{S\procirc R}c\text{ for some $a\in A$}\},\\
                            &=       \{%
                                         c\in C%
                                         \ \middle|\ %
                                         \begin{aligned}
                                             &\text{there exists some $a\in A$ and $b\in B$}\\
                                             &\text{such that $a\sim_{R}b$ and $b\sim_{R}c$}
                                         \end{aligned}
                                     \},\\
                            &\subset \{%
                                         c\in C%
                                         \ \middle|\ %
                                         \begin{aligned}
                                             &\text{there exists some $b\in B$}\\
                                             &\text{such that $b\sim_{S}c$}
                                         \end{aligned}
                                     \},\\
                            &\defeq  \range(S).
    \end{align*}
    This finishes the proof.
\end{Proof}
\subsection{Apartness Composition of Relations}\label{subsection-apartness-composition-of-relations}
Let $A$, $B$, and $C$ be sets and let $R\colon A\rightproarrow B$ and $S\colon B\rightproarrow C$ be relations.
\begin{definition}{Apartness Composition of Relations}{apartness-composition-of-relations}%
    The \index[set-theory]{relation!apartness composition of}\textbf{apartness composition of $R$ and $S$} is the relation \index[notation]{SafterR@$S\aptcirc R$}$S\aptcirc R$ defined as follows:%
    \begin{itemize}
        \item Viewing relations as subsets of $A\times C$, we define
            \[
                S\aptcirc R
                \defeq
                \{%
                    (a,c)\in A\times C%
                    \ \middle|\ %
                    \begin{aligned}
                        &\text{for each $b\in B$, we have}\\
                        &\text{$a\sim_{R}b$ or $b\sim_{S}c$}%
                    \end{aligned}
                \}.%
            \]%
        \item Viewing relations as functions $A\times C\to\TV$, we define
            \begin{align*}
                (S\aptcirc R)^{-_{1}}_{-_{2}} &\defeq \int_{b\in B}S^{-_{1}}_{b}\icoprod R^{b}_{-_{2}}\\%
                                              &=      \bigwedge_{b\in B}S^{-_{1}}_{b}\icoprod R^{b}_{-_{2}},%
            \end{align*}
            where the meet $\bigwedge$ is taken in the poset $(\TV,\preceq)$ of \ChapterRef{\ChapterSets, \cref{sets:the-poset-of-truth-values}}{\cref{the-poset-of-truth-values}}.
        \item Viewing relations as functions $A\to\mathcal{P}(C)$, we define
            \[
                [S\aptcirc R](a)%
                \defeq%
                \bigcap_{b\in B\setminus R(a)}S(b)%
            \]%
            for each $a\in A$.
    \end{itemize}
\end{definition}
\begin{example}{Examples of Apartness Composition of Relations}{examples-of-apartness-composition-of-relations}%
    Here are some examples of apartness composition of relations.
    \begin{enumerate}
        \item\label{examples-of-apartness-composition-of-relations-1}\SloganFont{Composing Less/Greater Than Equal With Greater/Less Than Equal Signs. }Let $A=B=C=\R$. We have
            \begin{align*}
                \mathord{\leq}\aptcirc\mathord{\geq} &= \emptyset,\\
                \mathord{\geq}\aptcirc\mathord{\leq} &= \emptyset.
            \end{align*}
        \item\label{examples-of-apartness-composition-of-relations-2}\SloganFont{Composing Less/Greater Than Equal Signs With Less/Greater Than Equal Signs. }Let $A=B=C=\R$. We have
            \begin{align*}
                \mathord{\leq}\aptcirc\mathord{\leq} &= \emptyset,\\
                \mathord{\geq}\aptcirc\mathord{\geq} &= \emptyset.
            \end{align*}
        \item\label{examples-of-apartness-composition-of-relations-3}\SloganFont{Equality and Inequality. }Let $A=B=C=\Z$. We have
            \begin{align*}
                \mathord{=}\aptcirc\mathord{\neq} &= \mathord{=},\\
                \mathord{\neq}\aptcirc\mathord{=} &= \mathord{=}.
            \end{align*}
        \item\label{examples-of-apartness-composition-of-relations-4}\SloganFont{Subset Inclusion. }Let $X$ be a set with at least three elements and consider the relations $\subset$ and $\supset$ in $\mathcal{P}(X)$. We have
            \[
                \mathord{\supset}\aptcirc\mathord{\subset}%
                =%
                \{%
                    (U,V)\in\mathcal{P}(X)%
                    \ \middle|\ %
                    \text{$U=\emptyset$ or $V=\emptyset$}%
                \}.%
            \]%
    \end{enumerate}
\end{example}
\begin{proposition}{Properties of Apartness Composition of Relations}{properties-of-apartness-composition-of-relations}%
    Let $R\colon A\rightproarrow B$, $S\colon B\rightproarrow C$, and $T\colon C\rightproarrow D$ be relations.
    \begin{enumerate}
        \item\label{properties-of-apartness-composition-of-relations-functoriality}\SloganFont{Functoriality. }The assignments $R,S,(R,S)\mapsto S\aptcirc R$ define functors
            \[
                \BifunctorialityPeriod{S\aptcirc-}{-\aptcirc R}{-_{1}\aptcirc-_{2}}{\eRel(A,B)}{\eRel(B,C)}{\eRel(B,C)\times\eRel(A,B)}{\eRel(A,C)}%
            \]%
            In particular, given relations
            \[
                \begin{tikzcd}[row sep={4.0*\the\DL,between origins}, column sep={4.0*\the\DL,between origins}, background color=backgroundColor, ampersand replacement=\&]
                    A
                    \arrow[r,"R_{1}", mid vert,shift left=1.0]%
                    \arrow[r,"R_{2}"',mid vert,shift right=1.0]%
                    \&
                    B
                    \arrow[r,"S_{1}", mid vert,shift left=0.8]%
                    \arrow[r,"S_{2}"',mid vert,shift right=0.8]%
                    \&
                    C\mrp{,}
                \end{tikzcd}
            \]%
            if $R_{1}\subset R_{2}$ and $S_{1}\subset S_{2}$, then $S_{1}\aptcirc R_{1}\subset S_{2}\aptcirc R_{2}$.
        \item\label{properties-of-apartness-composition-of-relations-associativity}\SloganFont{Associativity. }We have
            \[
                (T\aptcirc S)\aptcirc R
                =
                T\aptcirc(S\aptcirc R).
            \]%
        \item\label{properties-of-apartness-composition-of-relations-unitality}\SloganFont{Unitality. }We have
            \begin{align*}
                \nabla_{B}\aptcirc R &= R,\\
                R\aptcirc\nabla_{A}  &= R.
            \end{align*}
        \item\label{properties-of-apartness-composition-of-relations-linear-distributivity}\SloganFont{Linear Distributivity. }We have inclusions of relations
            \begin{align*}
                T\procirc(S\aptcirc R)  &\subset (T\procirc S)\aptcirc R,\\
                (T\aptcirc S)\procirc R &\subset T\aptcirc(S\procirc R).
            \end{align*}
        \item\label{properties-of-apartness-composition-of-relations-interaction-with-converses}\SloganFont{Interaction With Converses. }We have
            \[
                (S\aptcirc R)^{\dagger}
                =
                R^{\dagger}\aptcirc S^{\dagger}.
            \]%
        %\item\label{properties-of-apartness-composition-of-relations-}\SloganFont{. }
    \end{enumerate}
\end{proposition}
\begin{Proof}{Proof of \cref{properties-of-apartness-composition-of-relations}}%
    \FirstProofBox{\cref{properties-of-apartness-composition-of-relations-functoriality}: Functoriality}%
    We have
    \begin{align*}
        S_{1}\aptcirc R_{1} &\defeq  \{%
                                         (a,c)\in A\times C%
                                         \ \middle|\ %
                                         \begin{aligned}
                                             &\text{for each $b\in B$, we have}\\%
                                             &\text{$a\sim_{R_{1}}b$ or $b\sim_{S_{1}}c$}%
                                         \end{aligned}
                                     \}\\%
                            &\subset \{%
                                        (a,c)\in A\times C%
                                        \ \middle|\ %
                                        \begin{aligned}
                                            &\text{for each $b\in B$, we have}\\%
                                            &\text{$a\sim_{R_{2}}b$ or $b\sim_{S_{2}}c$}%
                                        \end{aligned}
                                    \}\\%
                            &\defeq S_{2}\aptcirc R_{2}.
    \end{align*}
    This finishes the proof.

    \ProofBox{\cref{properties-of-apartness-composition-of-relations-associativity}: Associativity}%
    Indeed, we have
    \begin{align*}
        (T\aptcirc S)\aptcirc R &\defeq (\int_{c\in C}T^{-_{1}}_{c}\icoprod S^{c}_{-_{2}})\aptcirc R\\
                                &\defeq \int_{b\in B}(\int_{c\in C}T^{-_{1}}_{c}\icoprod S^{c}_{b})\icoprod R^{b}_{-_{2}}\\
                                &=      \int_{b\in B}\int_{c\in C}(T^{-_{1}}_{c}\icoprod S^{c}_{b})\icoprod R^{b}_{-_{2}}\\
                                &=      \int_{c\in C}\int_{b\in B}(T^{-_{1}}_{c}\icoprod S^{c}_{b})\icoprod R^{b}_{-_{2}}\\
                                &=      \int_{c\in C}\int_{b\in B}T^{-_{1}}_{c}\icoprod(S^{c}_{b}\icoprod R^{b}_{-_{2}})\\
                                &=      \int_{c\in C}T^{-_{1}}_{c}\icoprod(\int_{b\in B}S^{c}_{b}\icoprod R^{b}_{-_{2}})\\
                                &\defeq \int_{c\in C}T^{-_{1}}_{c}\icoprod(S\aptcirc R)^{c}_{-_{2}}\\
                                &\defeq T\aptcirc(S\aptcirc R).
    \end{align*}
    In the language of relations, given $a\in A$ and $d\in D$, the stated equality witnesses the equivalence of the following two statements:
    \begin{itemize}
        \item We have $a\sim_{(T\aptcirc S)\aptcirc R}d$, i.e.\ there exists some $b\in B$ such that:
            \begin{itemize}
                \item We have $a\sim_{R}b$;
                \item We have $b\sim_{T\aptcirc S}d$, i.e.\ there exists some $c\in C$ such that:
                    \begin{itemize}
                        \item We have $b\sim_{S}c$;
                        \item We have $c\sim_{T}d$;
                    \end{itemize}
            \end{itemize}
        \item We have $a\sim_{T\aptcirc(S\aptcirc R)}d$, i.e.\ there exists some $c\in C$ such that:
            \begin{itemize}
                \item We have $a\sim_{S\aptcirc R}c$, i.e.\ there exists some $b\in B$ such that:
                    \begin{itemize}
                        \item We have $a\sim_{R}b$;
                        \item We have $b\sim_{S}c$;
                    \end{itemize}
                \item We have $c\sim_{T}d$;
            \end{itemize}
    \end{itemize}
    both of which are equivalent to the statement
    \begin{itemize}
        \item There exist $b\in B$ and $c\in C$ such that $a\sim_{R}b\sim_{S}c\sim_{T}d$.
    \end{itemize}

    \ProofBox{\cref{properties-of-apartness-composition-of-relations-unitality}: Unitality}%
    Indeed, we have
    \begin{align*}
        \nabla_{B}\aptcirc R &\defeq \int_{b\in B}(\nabla_{B})^{-_{1}}_{b}\icoprod R^{b}_{-_{2}}\\
                             &=      \bigwedge_{b\in B}(\nabla_{B})^{-_{1}}_{b}\icoprod R^{b}_{-_{2}}\\
                             &=      (\bigwedge_{\substack{b\in B\\b=-_{1}}}(\nabla_{B})^{-_{1}}_{b}\icoprod R^{b}_{-_{2}})\wedge(\bigwedge_{\substack{b\in B\\b\neq-_{1}}}(\nabla_{B})^{-_{1}}_{b}\icoprod R^{b}_{-_{2}})\\
                             &=      ((\nabla_{B})^{-_{1}}_{-_{1}}\icoprod R^{-_{1}}_{-_{2}})\wedge(\bigwedge_{\substack{b\in B\\b\neq-_{1}}}\ttrue\icoprod R^{b}_{-_{2}})\\
                             &=      (\tfalse\icoprod R^{-_{1}}_{-_{2}})\wedge(\bigwedge_{\substack{b\in B\\b\neq-_{1}}}\ttrue)\\
                             &=      R^{-_{1}}_{-_{2}}\wedge\ttrue\\
                             &=      R^{-_{1}}_{-_{2}},
    \end{align*}
    and
    \begin{align*}
        R\aptcirc\nabla_{A} &\defeq \int_{a\in A}R^{-_{1}}_{a}\icoprod(\nabla_{A})^{a}_{-_{2}}\\
                            &=      \bigwedge_{a\in A}R^{-_{1}}_{a}\icoprod(\nabla_{A})^{a}_{-_{2}}\\
                            &=      (\bigwedge_{\substack{a\in A\\a=-_{2}}}R^{-_{1}}_{a}\icoprod(\nabla_{A})^{a}_{-_{2}})\wedge(\bigwedge_{\substack{a\in A\\a\neq-_{2}}}R^{-_{1}}_{a}\icoprod(\nabla_{A})^{a}_{-_{2}})\\
                            &=      (R^{-_{1}}_{-_{2}}\icoprod(\nabla_{A})^{-_{2}}_{-_{2}})\wedge(\bigwedge_{\substack{a\in A\\a\neq-_{2}}}R^{-_{1}}_{a}\icoprod\ttrue)\\
                            &=      (R^{-_{1}}_{-_{2}}\icoprod\tfalse)\wedge(\bigwedge_{\substack{a\in A\\a\neq-_{2}}}\ttrue)\\
                            &=      R^{-_{1}}_{-_{2}}\wedge\ttrue\\
                            &=      R^{-_{1}}_{-_{2}},
    \end{align*}
    This finishes the proof.

    \ProofBox{\cref{properties-of-apartness-composition-of-relations-linear-distributivity}: Linear Distributivity}%
    We have
    \begin{envsmallsize}
        \begin{align*}
            T\procirc(S\aptcirc R) &\defeq  \{%
                                                (d,a)\in D\times A%
                                                \ \middle|\ %
                                                \begin{aligned}
                                                    &\text{there exists some $c\in C$ such}\\
                                                    &\text{that $a\sim_{S\aptcirc R}c$ and $c\sim_{T}d$}
                                                \end{aligned}
                                            \}\\
                                   &\defeq  \{%
                                                (d,a)\in D\times A%
                                                \ \middle|\ %
                                                \begin{aligned}
                                                    &\text{there exists some $c\in C$ such that}\\
                                                    &\text{$c\sim_{T}d$ and, for each $b\in B$,}\\
                                                    &\text{we have $a\sim_{R}b$ or $b\sim_{S}c$}%
                                                \end{aligned}
                                            \}\\
                                   &=       \{%
                                                (d,a)\in D\times A%
                                                \ \middle|\ %
                                                \begin{aligned}
                                                    &\text{the following conditions are satisfied:}\\
                                                    &\mkern20mu\text{1. For each $b\in B$, we have $a\sim_{R}b$ or $b\sim_{S}c$.}\\
                                                    &\mkern20mu\text{2. There exists $c\in C$ such that $c\sim_{T}d$.}
                                                \end{aligned}
                                            \}\\
                                   &\subset \{%
                                                (d,a)\in D\times A%
                                                \ \middle|\ %
                                                \begin{aligned}
                                                    &\text{for each $b\in B$, at least one of the}\\
                                                    &\text{following conditions is satisfied:}\\
                                                    &\mkern20mu\text{1. We have $a\sim_{R}b$.}\\
                                                    &\mkern20mu\text{2. There exists $c\in C$ such that $b\sim_{S}c$}\\
                                                    &\mkern20mu\phantom{\text{2. }}\text{and $c\sim_{T}d$.}
                                                \end{aligned}
                                            \}\\
                                   &\defeq  \{%
                                                (d,a)\in D\times A%
                                                \ \middle|\ %
                                                \begin{aligned}
                                                    &\text{for each $b\in B$, we have}\\
                                                    &\text{$a\sim_{R}b$ or there exists some $c\in C$}\\
                                                    &\text{such that $b\sim_{S}c$ and $c\sim_{T}d$}\\
                                                \end{aligned}
                                            \}\\
                                   &\defeq  \{%
                                                (d,a)\in D\times A%
                                                \ \middle|\ %
                                                \begin{aligned}
                                                    &\text{for each $b\in B$, we have}\\
                                                    &\text{$a\sim_{R}b$ or $b\sim_{T\procirc S}d$}\\
                                                \end{aligned}
                                            \}\\
                                   &\defeq  (T\procirc S)\aptcirc R%
        \end{align*}
    \end{envsmallsize}
    and
    \begin{envsmallsize}
        \begin{align*}
            (T\aptcirc S)\procirc R &\defeq  \{%
                                                 (d,a)\in D\times A%
                                                 \ \middle|\ %
                                                 \begin{aligned}
                                                     &\text{there exists some $b\in B$ such}\\
                                                     &\text{that $a\sim_{R}b$ and $b\sim_{T\aptcirc S}d$}
                                                 \end{aligned}
                                             \}\\
                                    &\defeq  \{%
                                                 (d,a)\in D\times A%
                                                 \ \middle|\ %
                                                 \begin{aligned}
                                                     &\text{there exists some $b\in B$ such}\\
                                                     &\text{that $a\sim_{R}b$ and, for each $c\in C$,}\\
                                                     &\text{we have $b\sim_{S}c$ or $c\sim_{T}d$}%
                                                 \end{aligned}
                                             \}\\
                                    &\defeq  \{%
                                                 (d,a)\in D\times A%
                                                 \ \middle|\ %
                                                 \begin{aligned}
                                                     &\text{there exists some $b\in B$ satisfying}\\
                                                     &\text{the following conditions:}\\
                                                     &\mkern20mu\text{1. We have $a\sim_{R}b$.}\\
                                                     &\mkern20mu\text{2. For each $c\in C$, we have $b\sim_{S}c$}\\
                                                     &\mkern20mu\phantom{\text{2. }}\text{or $c\sim_{T}d$.}
                                                 \end{aligned}
                                             \}\\
                                    &\subset  \{%
                                                 (d,a)\in D\times A%
                                                 \ \middle|\ %
                                                 \begin{aligned}
                                                     &\text{for each $c\in C$, at least one of the}\\
                                                     &\text{following conditions is satisfied:}\\
                                                     &\mkern20mu\text{1. We have $c\sim_{T}d$.}\\
                                                     &\mkern20mu\text{2. There exists some $b\in B$ such that}\\
                                                     &\mkern20mu\phantom{\text{2. }}\text{we have $a\sim_{R}b$ and $b\sim_{S}c$}
                                                 \end{aligned}
                                             \}\\
                                    &\defeq  \{%
                                                 (d,a)\in D\times A%
                                                 \ \middle|\ %
                                                 \begin{aligned}
                                                     &\text{for each $c\in C$, we have $c\sim_{T}d$}\\
                                                     &\text{or there exists some $b\in B$, such that}\\
                                                     &\text{we have $a\sim_{R}b$ and $b\sim_{S}c$}
                                                 \end{aligned}
                                             \}\\
                                    &\defeq  \{%
                                                 (d,a)\in D\times A%
                                                 \ \middle|\ %
                                                 \begin{aligned}
                                                     &\text{for each $c\in C$, we have}\\
                                                     &\text{$a\sim_{S\procirc R}c$ or $c\sim_{T}d$}%
                                                 \end{aligned}
                                             \}\\
                                    &\subset T\aptcirc(S\procirc R).
        \end{align*}
    \end{envsmallsize}
    This finishes the proof.

    \ProofBox{\cref{properties-of-apartness-composition-of-relations-interaction-with-converses}: Interaction With Converses}%
    This is a repetition of \cref{properties-of-converses-of-relations-interaction-with-apartness-composition} of \cref{properties-of-converses-of-relations} and is proved there.
\end{Proof}
\subsection{The Converse of a Relation}\label{subsection-the-converse-of-a-relation}
Let $A$, $B$, and $C$ be sets and let $R\subset A\times B$ be a relation.
\begin{definition}{The Converse of a Relation}{the-converse-of-a-relation}%
    The \index[set-theory]{relation!converse of}\textbf{converse of $R$}%
    %--- Begin Footnote ---%
    \footnote{%
        \SloganFont{Further Terminology: }Also called the \textbf{opposite of $R$} or the \textbf{transpose of $R$}.
    } %
    %---  End Footnote  ---%
    is the relation \index[notation]{Rdagger@$R^{\dagger}$}$\smash{R^{\dagger}}$ defined as follows:
    \begin{itemize}
        \item Viewing relations as subsets, we define
            \[
                R^{\dagger}
                \defeq
                \{%
                    (b,a)\in B\times A%
                    \ \middle|\ %
                    \text{we have $a\sim_{R}b$}%
                \}.%
            \]%
        \item Viewing relations as functions $A\times B\to\TV$, we define
            \[%
                {[R^{\dagger}]}{}^{a}_{b}
                \defeq%
                R^{b}_{a}
            \]%
            for each $(b,a)\in B\times A$.
        \item Viewing relations as functions $A\to\mathcal{P}(B)$, we define%
            %--- Begin Footnote ---%
            \footnote{%
                Note that $R^{\dagger}(b)=R^{-1}(\{b\})$.
                \par\vspace*{\TCBBoxCorrection}
            }%
            %---  End Footnote  ---%
            \[
                R^{\dagger}(b)%
                \defeq%
                \{a\in A\ \middle|\ b\in R(a)\}%
            \]%
            for each $b\in B$.
    \end{itemize}
\end{definition}
\begin{example}{Examples of Converses of Relations}{examples-of-converses-of-relations}%
    Here are some examples of converses of relations.
    \begin{enumerate}
        \item\label{examples-of-converses-of-relations-less-than-equal-signs}\SloganFont{Less Than Equal Signs. }We have $(\mathord{\leq})^{\dagger}=\mathord{\geq}$.
        \item\label{examples-of-converses-of-relations-greater-than-equal-signs}\SloganFont{Greater Than Equal Signs. }Dually to \cref{examples-of-converses-of-relations-less-than-equal-signs}, we have $(\mathord{\geq})^{\dagger}=\mathord{\leq}$.
        \item\label{examples-of-converses-of-relations-functions}\SloganFont{Functions. }Let $f\colon A\to B$ be a function. We have
            \begin{align*}
                \Gr(f)^{\dagger}   &= f^{-1},\\
                (f^{-1})^{\dagger} &= \Gr(f),
            \end{align*}
            where $\Gr(f)$ and $f^{-1}$ are the relations of \cref{subsection-the-graph-of-a-function,subsection-the-inverse-of-a-function}.
    \end{enumerate}
\end{example}
\begin{proposition}{Properties of Converses of Relations}{properties-of-converses-of-relations}%
    Let $R\colon A\rightproarrow B$ and $S\colon B\rightproarrow C$ be relations.
    \begin{enumerate}
        \item\label{properties-of-converses-of-relations-functoriality}\SloganFont{Functoriality. }The assignment $R\mapsto R^{\dagger}$ defines a functor (i.e.\ morphism of posets)
            \[
                (-)^{\dagger}%
                \colon%
                \eRel(A,B)%
                \to%
                \eRel(B,A).%
            \]%
            In other words, given relations $R,S\colon A\rightproarrows B$, we have:
            \begin{itemize}
                \itemstar If $R\subset S$, then $R^{\dagger}\subset S^{\dagger}$.
            \end{itemize}
        \item\label{properties-of-converses-of-relations-interaction-with-ranges-and-domains}\SloganFont{Interaction With Ranges and Domains. }We have
            \begin{align*}
                \dom(R^{\dagger})   &= \range(R),\\
                \range(R^{\dagger}) &= \dom(R).
            \end{align*}
        \item\label{properties-of-converses-of-relations-interaction-with-composition}\SloganFont{Interaction With Composition. }We have
            \[
                (S\procirc R)^{\dagger}
                =
                R^{\dagger}\procirc S^{\dagger}.
            \]%
        \item\label{properties-of-converses-of-relations-interaction-with-apartness-composition}\SloganFont{Interaction With Apartness Composition. }We have
            \[
                (S\aptcirc R)^{\dagger}
                =
                R^{\dagger}\aptcirc S^{\dagger}.
            \]%
        \item\label{properties-of-converses-of-relations-invertibility}\SloganFont{Invertibility. }We have
            \[
                (R^{\dagger})^{\dagger}
                =
                R.
            \]%
        \item\label{properties-of-converses-of-relations-identity-1}\SloganFont{Identity \rmI. }We have
            \[
                \Delta^{\dagger}_{A}
                =
                \Delta_{A}.
            \]%
        \item\label{properties-of-converses-of-relations-identity-2}\SloganFont{Identity \rmII. }We have
            \[
                \nabla^{\dagger}_{A}
                =
                \nabla_{A}.
            \]%
        %\item\label{properties-of-converses-of-relations-}\SloganFont{. }
    \end{enumerate}
\end{proposition}
\begin{Proof}{Proof of \cref{properties-of-converses-of-relations}}%
    \FirstProofBox{\cref{properties-of-converses-of-relations-functoriality}: Functoriality}%
    We have
    \begin{align*}
        R^{\dagger} &\defeq  \{a\in A\ \middle|\ b\in R(a)\}\\
                    &\subset \{a\in A\ \middle|\ b\in S(a)\}\\
                    &\defeq  S^{\dagger}.
    \end{align*}
    This finishes the proof.

    \ProofBox{\cref{properties-of-converses-of-relations-interaction-with-ranges-and-domains}: Interaction With Ranges and Domains}%
    We have
    \begin{align*}
        \dom(R^{\dagger}) &\defeq \{b\in B\ \middle|\ \text{$b\sim_{R^{\dagger}}a$ for some $a\in A$}\}\\
                          &=      \{b\in B\ \middle|\ \text{$a\sim_{R}b$ for some $a\in A$}\}\\
                          &\defeq \range(R)
    \end{align*}
    and
    \begin{align*}
        \range(R^{\dagger}) &\defeq \{a\in A\ \middle|\ \text{$b\sim_{R^{\dagger}}a$ for some $b\in B$}\}\\
                            &=      \{a\in A\ \middle|\ \text{$a\sim_{R}b$ for some $b\in B$}\}\\
                            &\defeq \dom(R).
    \end{align*}
    This finishes the proof.

    \ProofBox{\cref{properties-of-converses-of-relations-interaction-with-composition}: Interaction With Composition}%
    We have
    \begin{align*}
        (S\procirc R)^{\dagger} &\defeq \{(c,a)\in C\times A\ \middle|\ c\sim_{(S\procirc R)^{\dagger}}a\}\\
                                &=      \{(c,a)\in C\times A\ \middle|\ a\sim_{S\procirc R}c\}\\
                                &=      \{%
                                            (c,a)\in C\times A%
                                            \ \middle|\ %
                                            \begin{aligned}
                                                &\text{there exists some $b\in B$ such}\\
                                                &\text{that $a\sim_{R}b$ and $b\sim_{S}c$}\\
                                            \end{aligned}
                                        \}\\
                                &=      \{%
                                            (c,a)\in C\times A%
                                            \ \middle|\ %
                                            \begin{aligned}
                                                &\text{there exists some $b\in B$ such}\\
                                                &\text{that $b\sim_{R^{\dagger}}a$ and $c\sim_{S^{\dagger}}b$}\\
                                            \end{aligned}
                                        \}\\
                                &=      \{%
                                            (c,a)\in C\times A%
                                            \ \middle|\ %
                                            \begin{aligned}
                                                &\text{there exists some $b\in B$ such}\\
                                                &\text{that $c\sim_{S^{\dagger}}b$ and $b\sim_{R^{\dagger}}a$ }\\
                                            \end{aligned}
                                        \}\\
                                &\defeq R^{\dagger}\procirc S^{\dagger}.
    \end{align*}
    This finishes the proof.

    \ProofBox{\cref{properties-of-converses-of-relations-interaction-with-apartness-composition}: Interaction With Apartness Composition}%
    We have
    \begin{align*}
        (S\aptcirc R)^{\dagger} &\defeq \{(c,a)\in C\times A\ \middle|\ c\sim_{(S\aptcirc R)^{\dagger}}a\}\\
                                &=      \{(c,a)\in C\times A\ \middle|\ a\sim_{S\aptcirc R}c\}\\
                                &=      \{%
                                            (c,a)\in C\times A%
                                            \ \middle|\ %
                                            \begin{aligned}
                                                &\text{for each $b\in B$, we have}\\
                                                &\text{$a\sim_{R}b$ or $b\sim_{S}c$}\\
                                            \end{aligned}
                                        \}\\
                                &=      \{%
                                            (c,a)\in C\times A%
                                            \ \middle|\ %
                                            \begin{aligned}
                                                &\text{for each $b\in B$, we have}\\
                                                &\text{$b\sim_{R^{\dagger}}a$ or $c\sim_{S^{\dagger}}b$}\\
                                            \end{aligned}
                                        \}\\
                                &=      \{%
                                            (c,a)\in C\times A%
                                            \ \middle|\ %
                                            \begin{aligned}
                                                &\text{for each $b\in B$, we have}\\
                                                &\text{$c\sim_{S^{\dagger}}b$ or $b\sim_{R^{\dagger}}a$}\\
                                            \end{aligned}
                                        \}\\
                                &\defeq R^{\dagger}\aptcirc S^{\dagger}.
    \end{align*}
    This finishes the proof.

    \ProofBox{\cref{properties-of-converses-of-relations-invertibility}: Invertibility}%
    We have
    \begin{align*}
        (R^{\dagger})^{\dagger} &\defeq \{(a,b)\in A\times B\ \middle|\ b\sim_{R^{\dagger}}a\}\\
                                &=      \{(a,b)\in A\times B\ \middle|\ a\sim_{R}b\}\\
                                &\defeq R.
    \end{align*}
    This finishes the proof.

    \ProofBox{\cref{properties-of-converses-of-relations-identity-1}: Identity \rmI}%
    We have
    \begin{align*}
        \Delta^{\dagger}_{A} &\defeq \{(a,b)\in A\times A\ \middle|\ a\sim_{\Delta_{A}}b\}\\
                             &=      \{(a,b)\in A\times A\ \middle|\ a=b\}\\
                             &=      \Delta_{A}.
    \end{align*}
    This finishes the proof.

    \ProofBox{\cref{properties-of-converses-of-relations-identity-2}: Identity \rmII}%
    We have
    \begin{align*}
        \nabla^{\dagger}_{A} &\defeq \{(a,b)\in A\times A\ \middle|\ a\sim_{\nabla_{A}}b\}\\
                             &=      \{(a,b)\in A\times A\ \middle|\ a\neq b\}\\
                             &=      \nabla_{A}.
    \end{align*}
    This finishes the proof.
\end{Proof}
\section{Examples of Relations}\label{section-examples-of-relations}
\subsection{Elementary Examples of Relations}\label{subsection-elementary-examples-of-relations}
% TODO: Examples in TODO
\begin{example}{The Trivial Relation}{the-trivial-relation}%
    The \index[set-theory]{relation!trivial}\textbf{trivial relation on $A$ and $B$} is the relation \index[notation]{simtriv@$\unsim_{\triv}$}$\unsim_{\triv}$ defined equivalently as follows:
    \begin{enumerate}
        \item\label{the-trivial-relation-1}As a subset of $A\times B$, we have
            \[
                \unsim_{\triv}%
                \defeq%
                A\times B.%
            \]%
        \item\label{the-trivial-relation-2}As a function from $A\times B$ to $\TV$, the relation $\unsim_{\triv}$ is the constant function
            \[%
                \Delta_{\true}%
                \colon%
                A\times B%
                \to%
                \TV%
            \]%
            from $A\times B$ to $\TV$ taking the value $\true$.
        \item\label{the-trivial-relation-3}As a function from $A$ to $\mathcal{P}(B)$, the relation $\unsim_{\triv}$ is the function
            \[%
                \Delta_{\true}%
                \colon%
                A%
                \to%
                \mathcal{P}(B)%
            \]%
            defined by
            \[
                \Delta_{\true}(a)%
                \defeq%
                B%
            \]%
            for each $a\in A$.
        \item\label{the-trivial-relation-4}Lastly, it is the unique relation $R$ on $A$ and $B$ such that we have $a\sim_{R}b$ for each $a\in A$ and each $b\in B$.
    \end{enumerate}
\end{example}
\begin{example}{The Cotrivial Relation}{the-cotrivial-relation}%
    The \index[set-theory]{relation!cotrivial}\textbf{cotrivial relation on $A$ and $B$} is the relation \index[notation]{simcotriv@$\unsim_{\cotriv}$}$\unsim_{\cotriv}$ defined equivalently as follows:%
    \begin{enumerate}
        \item\label{the-cotrivial-relation-1}As a subset of $A\times B$, we have
            \[
                \unsim_{\cotriv}%
                \defeq%
                \emptyset.%
            \]%
        \item\label{the-cotrivial-relation-2}As a function from $A\times B$ to $\TV$, the relation $\unsim_{\cotriv}$ is the constant function
            \[%
                \Delta_{\false}%
                \colon%
                A\times B%
                \to%
                \TV%
            \]%
            from $A\times B$ to $\TV$ taking the value $\false$.
        \item\label{the-cotrivial-relation-3}As a function from $A$ to $\mathcal{P}(B)$, the relation $\unsim_{\cotriv}$ is the function
            \[%
                \Delta_{\false}%
                \colon%
                A%
                \to%
                \mathcal{P}(B)%
            \]%
            defined by
            \[
                \Delta_{\false}(a)%
                \defeq%
                \emptyset%
            \]%
            for each $a\in A$.
        \item\label{the-cotrivial-relation-4}Lastly, it is the unique relation $R$ on $A$ and $B$ such that we have $a\nsim_{R}b$ for each $a\in A$ and each $b\in B$.
    \end{enumerate}
\end{example}
\begin{example}{The Characteristic Relation of a Set}{the-characteristic-relation}%
    The characteristic relation $\chi_{X}$ on $X$ of \ChapterRef{\ChapterConstructionsWithSets, \cref{constructions-with-sets:the-characteristic-relation-of-a-set}}{\cref{the-characteristic-relation-of-a-set}}:
    \begin{enumerate}
        \item\label{the-characteristic-relation-1}As a subset of $X\times X$, we have
            \begin{align*}
                \unsim_{\chi_{X}} &\defeq \Delta_{X}\\%
                                  &\defeq \{(x,x)\in X\times X\}.%
            \end{align*}
        \item\label{the-characteristic-relation-2}As a function from $X\times X$ to $\TV$, we have
            \[
                \chi_{X}(x,y)
                \defeq
                \begin{cases}
                    \true  &\text{if $x=y$,}\\
                    \false &\text{if $x\neq y$}
                \end{cases}
            \]%
            for each $x,y\in X$.
        \item\label{the-characteristic-relation-3}As a function from $X$ to $\mathcal{P}(X)$, we have
            \[%
                \chi_{X}(x)%
                \defeq%
                \{x\}%
            \]%
            for each $x\in X$.
    \end{enumerate}
\end{example}
\begin{example}{The Antidiagonal Relation on a Set}{the-antidiagonal-relation}%
    The \index[set-theory]{relation!antidiagonal}\textbf{antidiagonal relation on $X$} is the relation \index[notation]{nablaX@$\nabla_{X}$}$\nabla_{X}$ defined equivalently as follows:%
    \begin{enumerate}
        \item\label{the-antidiagonal-relation-1}As a subset of $X\times X$, we have
            \begin{align*}
                \unsim_{\nabla_{X}} &\defeq \nabla_{X}\\%
                                    &\defeq X\setminus\Delta_{X}\\
                                    &=      \{(x,y)\in X\times X\ |\ x\neq y\}.%
            \end{align*}
        \item\label{the-antidiagonal-relation-2}As a function from $X\times X$ to $\TV$, we have
            \[
                \nabla_{X}(x,y)
                \defeq
                \begin{cases}
                    \true  &\text{if $a\neq b$,}\\
                    \false &\text{if $a=b$}
                \end{cases}
            \]%
            for each $x,y\in X$.
        \item\label{the-antidiagonal-relation-3}As a function from $X$ to $\mathcal{P}(X)$, we have
            \[%
                \nabla_{X}(x)%
                \defeq%
                X\setminus\{x\}%
            \]%
            for each $x\in X$.
    \end{enumerate}
\end{example}
\begin{example}{Partial Functions}{examples-of-relations-partial-functions}%
    Partial functions may be viewed (or defined) as being exactly those relations which are functional; see \ChapterRef{\ChapterConditionsOnRelations, \cref{conditions-on-relations:subsection-functional-relations}}{\cref{subsection-functional-relations}}.
\end{example}
\begin{example}{Square Roots}{square-roots}%
    Square roots are examples of relations:
    \begin{enumerate}
        \item\label{square-roots-1}\SloganFont{Square Roots in $\R$. }The assignment $x\mapsto\sqrt{x}$ defines a relation
            \[
                \sqrt{-}%
                \colon%
                \R%
                \to%
                \mathcal{P}(\R)%
            \]%
            from $\R$ to itself, being explicitly given by
            \[
                \sqrt{x}%
                \defeq%
                \begin{cases}
                    0                                  &\text{if $x=0$,}\\
                    \{-\sqrt{\abs{x}},\sqrt{\abs{x}}\} &\text{if $x\neq0$.}
                \end{cases}
            \]%
        \item\label{square-roots-2}\SloganFont{Square Roots in $\Q$. }Square roots in $\Q$ are similar to square roots in $\R$, though now additionally it may also occur that $\sqrt{-}\colon\Q\to\mathcal{P}(\Q)$ sends a rational number $x$ (e.g.\ $2$) to the empty set (since $\sqrt{2}\nin\Q$).
    \end{enumerate}
\end{example}
\begin{example}{Complex Logarithms}{complex-logarithms}%
    The complex logarithm defines a relation
    \[
        \log%
        \colon%
        \C%
        \to%
        \mathcal{P}(\C)%
    \]%
    from $\C$ to itself, where we have
    \[
        \log(a+bi)%
        \defeq%
        \{%
            \log(\sqrt{a^{2}+b^{2}})+i\arg(a+bi)+(2\pi i)k%
            \ \middle|\ %
            k\in\Z%
        \}%
    \]%
    for each $a+bi\in\C$.
\end{example}
\begin{example}{More Examples of Relations}{more-examples-of-relations}%
    See \cite{wikipedia:multivalued-function} for more examples of relations, such as antiderivation, inverse trigonometric functions, and inverse hyperbolic functions.
\end{example}
\subsection{The Graph of a Function}\label{subsection-the-graph-of-a-function}
Let $f\colon A\to B$ be a function.
\begin{definition}{The Graph of a Function}{the-graph-of-a-function}%
    The \index[set-theory]{function!graph of}\textbf{graph of $f$} is the relation $\Gr(f)\colon A\rightproarrow B$ defined as follows:%
    %--- Begin Footnote ---%
    \footnote{%
        \SloganFont{Further Terminology and Notation: }When $f=\id_{A}$, we write \index[notation]{GrA@$\Gr(A)$}$\Gr(A)$ for $\Gr(\id_{A})$, calling it the \textbf{graph of $A$}.
        \par\vspace*{\TCBBoxCorrection}
    }%
    %---  End Footnote  ---%
    \begin{itemize}
        \item Viewing relations from $A$ to $B$ as subsets of $A\times B$, we define
            \[
                \Gr(f)%
                \defeq%
                \{(a,f(a))\in A\times B\ \middle|\ a\in A\}.%
            \]%
        \item Viewing relations from $A$ to $B$ as functions $A\times B\to\TV$, we define
            \[
                \Gr(f)^{b}_{a}%
                \defeq%
                \begin{cases}
                    \true  &\text{if $b=f(a)$,}\\
                    \false &\text{otherwise}
                \end{cases}
            \]%
            for each $(a,b)\in A\times B$.
        \item Viewing relations from $A$ to $B$ as functions $A\to\mathcal{P}(B)$, we define
            \[
                [\Gr(f)](a)%
                \defeq%
                \{f(a)\}%
            \]%
            for each $a\in A$, i.e.\ we define $\Gr(f)$ as the composition
            \[
                A
                \xlongrightarrow{f}
                B
                \xlonghookrightarrow{\chi_{B}}
                \mathcal{P}(B).
            \]%
    \end{itemize}
\end{definition}
\begin{proposition}{Properties of Graphs of Functions}{properties-of-graphs-of-functions}%
    Let $f\colon A\to B$ be a function.
    \begin{enumerate}
        \item\label{properties-of-graphs-of-functions-functoriality}\SloganFont{Functoriality. }The assignment $A\mapsto\Gr(A)$ defines a functor
            \[
                \Gr%
                \colon%
                \Sets%
                \to%
                \Rel%
            \]%
            where
            \begin{itemize}
                \item\SloganFont{Action on Objects. }For each $A\in\Obj(\Sets)$, we have
                    \[
                        \Gr(A)%
                        \defeq
                        A.
                    \]%
                \item\SloganFont{Action on Morphisms. }For each $A,B\in\Obj(\Sets)$, the action on $\Hom$-sets
                    \[
                        \Gr_{A,B}%
                        \colon%
                        \Sets(A,B)
                        \to
                        \underbrace{\Rel(\Gr(A),\Gr(B))}_{\defeq\Rel(A,B)}%
                    \]%
                    of $\Gr$ at $(A,B)$ is defined by
                    \[
                        \Gr_{A,B}(f)
                        \defeq
                        \Gr(f),
                    \]%
                    where $\Gr(f)$ is the graph of $f$ as in \cref{the-graph-of-a-function}.
            \end{itemize}
            In particular, the following statements are true:
            \begin{itemize}
                \item\SloganFont{Preservation of Identities. }We have
                    \[
                        \Gr(\id_{A})%
                        =%
                        \chi_{A}%
                    \]%
                    for each $A\in\Obj(\Sets)$.
                \item\SloganFont{Preservation of Composition. }We have
                    \[
                        \Gr(g\circ f)%
                        =%
                        \Gr(g)\procirc\Gr(f)%
                    \]%
                    for each pair of functions $f\colon A\to B$ and $g\colon B\to C$.
            \end{itemize}
        \item\label{properties-of-graphs-of-functions-adjointness}\SloganFont{Adjointness. }We have an adjunction
            \begin{webcompile}
                \Adjunction#\Gr#\mathcal{P}_{!}#\Sets#\Rel,#
            \end{webcompile}%
            witnessed by a bijection of sets%
            \[
                \Rel(\Gr(A),B)
                \cong
                \Sets(A,\mathcal{P}(B))
            \]%
            natural in $A\in\Obj(\Sets)$ and $B\in\Obj(\Rel)$.
        \item\label{properties-of-graphs-of-functions-cocontinuity}\SloganFont{Cocontinuity. }The functor $\Gr\colon\Sets\to\Rel$ of \cref{properties-of-graphs-of-functions-functoriality} preserves colimits.
        \item\label{properties-of-graphs-of-functions-adjointness-inside-sfbfrel}\SloganFont{Adjointness Inside $\sfbfRel$. }We have an internal adjunction
            \begin{webcompile}
                \RelAdjunctionShortSize#4.0#\Gr(f)#f^{-1}#A#B#
            \end{webcompile}%
            in $\sfbfRel$, where $f^{-1}$ is the inverse of $f$ of \cref{the-inverse-of-a-function}.
        \item\label{properties-of-graphs-of-functions-interaction-with-converses}\SloganFont{Interaction With Converses. }We have
            \begin{align*}
                \Gr(f)^{\dagger}   &= f^{-1},\\
                (f^{-1})^{\dagger} &= \Gr(f).
            \end{align*}
        \item\label{properties-of-graphs-of-functions-characterisations}\SloganFont{Characterisations. }Let $R\colon A\rightproarrow B$ be a relation. The following conditions are equivalent:
            \begin{enumerate}
                \item\label{properties-of-graphs-of-functions-characterisations-1}There exists a function $f\colon A\to B$ such that $R=\Gr(f)$.
                \item\label{properties-of-graphs-of-functions-characterisations-2}The relation $R$ is total and functional.
                \item\label{properties-of-graphs-of-functions-characterisations-3}The weak and strong inverse images of $R$ agree, i.e.\ we have $R^{-1}=R_{-1}$.
                \item\label{properties-of-graphs-of-functions-characterisations-4}The relation $R$ has a right adjoint $R^{\dagger}$ in $\Rel$.
            \end{enumerate}
        %\item\label{properties-of-graphs-of-functions-}\SloganFont{. }
    \end{enumerate}
\end{proposition}
\begin{Proof}{Proof of \cref{properties-of-graphs-of-functions}}%
    \FirstProofBox{\cref{properties-of-graphs-of-functions-functoriality}: Functoriality}%
    Omitted.

    \ProofBox{\cref{properties-of-graphs-of-functions-adjointness}: Adjointness}%
    This is a repetition of \ChapterRef{\ChapterConstructionsWithSets, \cref{constructions-with-sets:adjointness-of-powersets-2}}{\cref{adjointness-of-powersets-2}}, and is proved there.

    \ProofBox{\cref{properties-of-graphs-of-functions-cocontinuity}: Cocontinuity}%
    This follows from \cref{properties-of-graphs-of-functions-adjointness} and \cref{TODO}.

    \ProofBox{\cref{properties-of-graphs-of-functions-adjointness-inside-sfbfrel}: Adjointness Inside $\sfbfRel$}%
    We need to check that there are inclusions
    \begin{gather*}
        \chi_{A}              \subset f^{-1}\procirc\Gr(f),\\
        \Gr(f)\procirc f^{-1} \subset \chi_{B}.
    \end{gather*}
    These correspond respectively to the following conditions:
    \begin{enumerate}
        \item\label{proof-of-properties-of-graphs-of-functions-adjointness-inside-sfbfrel-1}For each $a\in A$, there exists some $b\in B$ such that $a\sim_{\Gr(f)}b$ and $b\sim_{f^{-1}}a$.
        \item\label{proof-of-properties-of-graphs-of-functions-adjointness-inside-sfbfrel-2}For each $a,b\in A$, if $a\sim_{\Gr(f)}b$ and $b\sim_{f^{-1}}a$, then $a=b$.
    \end{enumerate}
    In other words, the first condition states that the image of any $a\in A$ by $f$ is nonempty, whereas the second condition states that $f$ is not multivalued. As $f$ is a function, both of these statements are true, and we are done.

    \ProofBox{\cref{properties-of-graphs-of-functions-interaction-with-converses}: Interaction With Converses}%
    Omitted.

    \ProofBox{\cref{properties-of-graphs-of-functions-characterisations}: Characterisations}%
    We claim that \cref{properties-of-graphs-of-functions-characterisations-1,properties-of-graphs-of-functions-characterisations-2,properties-of-graphs-of-functions-characterisations-3,properties-of-graphs-of-functions-characterisations-4} are indeed equivalent:
    \begin{itemize}
        \item\SloganFont{\cref{properties-of-graphs-of-functions-characterisations-1}$\iff$\cref{properties-of-graphs-of-functions-characterisations-2}. }This is shown in the proof of \cref{isomorphisms-and-equivalences-in-rel}.
        \item\SloganFont{\cref{properties-of-graphs-of-functions-characterisations-2}$\implies$\cref{properties-of-graphs-of-functions-characterisations-3}. }If $R$ is total and functional, then, for each $a\in A$, the set $R(a)$ is a singleton. Since the conditions
            \begin{itemize}
                \item $R(a)\cap V\neq\emptyset$;
                \item $R(a)\subset V$;
            \end{itemize}
            are equivalent when $R(a)$ is a singleton, it follows that the sets
            \begin{align*}
                R^{-1}(V) &\defeq \{a\in A \ \middle|\ R(a)\cap V\neq\emptyset\},\\
                R_{-1}(V) &\defeq \{a\in A\ \middle|\ R(a)\subset V\}%
            \end{align*}
            are equal for all $V\in\mathcal{P}(B)$.
        \item\SloganFont{\cref{properties-of-graphs-of-functions-characterisations-3}$\implies$\cref{properties-of-graphs-of-functions-characterisations-2}. }We claim that $R$ is indeed total and functional:
            \begin{itemize}
                \item\SloganFont{Totality. }We proceed in a few steps:
                    \begin{itemize}
                        \item If we had $R(a)=\emptyset$ for some $a\in A$, then we would have $a\in R_{-1}(\emptyset)$, so that $R_{-1}(\emptyset)\neq\emptyset$.
                        \item But since $R^{-1}(\emptyset)=\emptyset$, this would imply $R_{-1}(\emptyset)\neq R^{-1}(\emptyset)$, a contradiction.
                        \item Thus $R(a)\neq\emptyset$ for all $a\in A$ and $R$ is total.
                    \end{itemize}
                \item\SloganFont{Functionality. }If $R^{-1}=R_{-1}$, then we have
                    \begin{align*}
                        \{a\} &= R^{-1}(\{b\})\\
                              &= R_{-1}(\{b\})
                    \end{align*}
                    for each $b\in R(a)$ and each $a\in A$, and thus $R(a)\subset\{b\}$. But since $R$ is total, we must have $R(a)=\{b\}$, so $R$ is functional.
            \end{itemize}
        \item\SloganFont{\cref{properties-of-graphs-of-functions-characterisations-1}$\iff$\cref{properties-of-graphs-of-functions-characterisations-4}. }This follows from \ChapterRef{\ChapterRelations, \cref{relations:adjunctions-in-rel}}{\cref{adjunctions-in-rel}}.
    \end{itemize}
    This finishes the proof.
\end{Proof}
\subsection{The Inverse of a Function}\label{subsection-the-inverse-of-a-function}
Let $f\colon A\to B$ be a function.
\begin{definition}{The Inverse of a Function}{the-inverse-of-a-function}%
    The \index[set-theory]{function!inverse of}\textbf{inverse of $f$} is the relation $f^{-1}\colon B\rightproarrow A$ defined as follows:%
    \begin{itemize}
        \item Viewing relations from $B$ to $A$ as subsets of $B\times A$, we define
            \[
                f^{-1}%
                \defeq%
                \{(b,f^{-1}(b))\in B\times A\ \middle|\ a\in A\},%
            \]%
            where
            \[
                f^{-1}(b)%
                \defeq%
                \{%
                    a\in A%
                    \ \middle|\ %
                    f(a)=b%
                \}%
            \]%
            for each $b\in B$.
        \item Viewing relations from $B$ to $A$ as functions $B\times A\to\TV$, we define
            \[
                [f^{-1}]^{b}_{a}%
                \defeq%
                \begin{cases}
                    \true  &\text{if there exists $a\in A$ with $f(a)=b$,}\\
                    \false &\text{otherwise}
                \end{cases}
            \]%
            for each $(b,a)\in B\times A$.
        \item Viewing relations from $B$ to $A$ as functions $B\to\mathcal{P}(A)$, we define
            \[
                f^{-1}(b)%
                \defeq%
                \{%
                    a\in A%
                    \ \middle|\ %
                    f(a)=b%
                \}%
            \]%
            for each $b\in B$.
    \end{itemize}
\end{definition}
\begin{proposition}{Properties of Inverses of Functions}{properties-of-inverses-of-functions}%
    Let $f\colon A\to B$ be a function.
    \begin{enumerate}
        \item\label{properties-of-inverses-of-functions-functoriality}\SloganFont{Functoriality. }The assignment $A\mapsto A$, $f\mapsto f^{-1}$ defines a functor
            \[
                (-)^{-1}%
                \colon%
                \Sets%
                \to%
                \Rel%
            \]%
            where
            \begin{itemize}
                \item\SloganFont{Action on Objects. }For each $A\in\Obj(\Sets)$, we have
                    \[
                        \left[(-)^{-1}\right](A)%
                        \defeq
                        A.
                    \]%
                \item\SloganFont{Action on Morphisms. }For each $A,B\in\Obj(\Sets)$, the action on $\Hom$-sets
                    \[
                        (-)^{-1}_{A,B}%
                        \colon%
                        \Sets(A,B)
                        \to
                        \Rel(A,B)
                    \]%
                    of $(-)^{-1}$ at $(A,B)$ is defined by
                    \[
                        (-)^{-1}_{A,B}(f)
                        \defeq
                        \left[(-)^{-1}\right](f),
                    \]%
                    where $f^{-1}$ is the inverse of $f$ as in \cref{the-inverse-of-a-function}.
            \end{itemize}
            In particular, the following statements are true:
            \begin{itemize}
                \item\SloganFont{Preservation of Identities. }We have
                    \[
                        \id^{-1}_{A}%
                        =%
                        \chi_{A}%
                    \]%
                    for each $A\in\Obj(\Sets)$.
                \item\SloganFont{Preservation of Composition. }We have
                    \[
                        (g\circ f)^{-1}%
                        =%
                        g^{-1}\procirc f^{-1}
                    \]%
                    for pair of functions $f\colon A\to B$ and $g\colon B\to C$.
            \end{itemize}
        \item\label{properties-of-inverses-of-functions-adjointness-inside-sfbfrel}\SloganFont{Adjointness Inside $\sfbfRel$. }We have an adjunction
            \begin{webcompile}
                \RelAdjunctionShortSize#4.0#\Gr(f)#f^{-1}#A#B#
            \end{webcompile}%
            in $\sfbfRel$.
        \item\label{properties-of-inverses-of-functions-interaction-with-converses-of-relations}\SloganFont{Interaction With Converses of Relations. }We have
            \begin{align*}
                (f^{-1})^{\dagger} &= \Gr(f),\\
                \Gr(f)^{\dagger}   &= f^{-1}.
            \end{align*}
        %\item\label{properties-of-inverses-of-functions-}\SloganFont{. }
    \end{enumerate}
\end{proposition}
\begin{Proof}{Proof of \cref{properties-of-inverses-of-functions}}%
    \FirstProofBox{\cref{properties-of-inverses-of-functions-functoriality}: Functoriality}%
    Omitted.

    \ProofBox{\cref{properties-of-inverses-of-functions-adjointness-inside-sfbfrel}: Adjointness Inside $\sfbfRel$}%
    This is a repetition of \cref{properties-of-graphs-of-functions-adjointness-inside-sfbfrel} of \cref{properties-of-graphs-of-functions} and is proved there.

    \ProofBox{\cref{properties-of-inverses-of-functions-interaction-with-converses-of-relations}: Interaction With Converses of Relations}%
    This is a repetition of \cref{properties-of-graphs-of-functions-interaction-with-converses} of \cref{properties-of-graphs-of-functions} and is proved there.
\end{Proof}
\subsection{Representable Relations}\label{subsection-representable-relations}
Let $A$ and $B$ be sets.
\begin{definition}{Representable Relations}{representable-relations}%
    Let $f\colon A\to B$ and $g\colon B\to A$ be functions.%
    %--- Begin Footnote ---%
    \footnote{%
        More generally, given functions
        \begin{align*}
            f &\colon A \to C,\\
            g &\colon B \to D
        \end{align*}
        and a relation $B\rightproarrow D$, we may consider the composite relation
        \[
            A\times B%
            \xlongrightarrow{f\times g}%
            C\times D%
            \xlongrightarrow{R}%
            \TV,%
        \]%
        for which we have $a\sim_{R\circ(f\times g)}b$ \textiff $f(a)\sim_{R}g(b)$.
        \par\vspace*{\TCBBoxCorrection}
    }%
    %---  End Footnote  ---%
    \begin{enumerate}
        \item\label{representable-relations-representable-relations}The \index[set-theory]{relation!representable}\textbf{representable relation associated to $f$} is the relation $\chi_{f}\colon A\rightproarrow B$ defined as the composition%
            \[
                A\times B%
                \xlongrightarrow{f\times\id_{B}}%
                B\times B%
                \xlongrightarrow{\chi_{B}}%
                \TV,%
            \]%
            i.e.\ given by declaring $a\sim_{\chi_{f}}b$ \textiff $f(a)=b$.
        \item\label{representable-relations-corepresentable-relations}The \index[set-theory]{relation!corepresentable}\textbf{corepresentable relation associated to $g$} is the relation $\chi^{g}\colon B\rightproarrow A$ defined as the composition%
            \[
                B\times A%
                \xlongrightarrow{g\times\id_{A}}%
                A\times A%
                \xlongrightarrow{\chi_{A}}%
                \TV,%
            \]%
            i.e.\ given by declaring $b\sim_{\chi^{g}}a$ \textiff $g(b)=a$.
    \end{enumerate}
\end{definition}
\section{Categories of Relations}\label{subsection-categories-of-relations}
\subsection{The Category of Relations Between Two Sets}\label{subsection-the-category-of-relations-between-two-sets}
\begin{definition}{The Category of Relations Between Two Sets}{the-category-of-relations-between-two-sets}%
    The \index[set-theory]{relation!category of}\textbf{category of relations from $A$ to $B$} is the category \index[notation]{RelAB@$\eRel(A,B)$}$\eRel(A,B)$ defined by%
    %--- Begin Footnote ---%
    \footnote{%
        Here we choose to abuse notation by writing $\eRel(A,B)$ instead of $\eRel(A,B)_{\pos}$ for the posetal category of relations from $A$ to $B$, even though the same notation is used for the poset of relations from $A$ to $B$.
        \par\vspace*{\TCBBoxCorrection}
    } %
    %---  End Footnote  ---%
    \[
        \eRel(A,B)%
        \defeq
        \eRel(A,B)_{\pos},%
    \]%
    where $\eRel(A,B)_{\pos}$ is the posetal category associated to the poset $\eRel(A,B)$ of \cref{further-notation-for-relations-the-poset-of-relations-between-two-sets} of \cref{further-notation-for-relations} and \ChapterRef{\ChapterCategories, \cref{categories:posetal-categories}}{\cref{posetal-categories}}.%
\end{definition}
\subsection{The Category of Relations}\label{subsection-the-category-of-relations}
\begin{definition}{The Category of Relations}{the-category-of-relations}%
    The \index[set-theory]{category of relations}\index[set-theory]{relation!category of@category of}\textbf{category of relations} is the category \index[notation]{Rel@$\sfRel$}$\sfRel$ where
    \begin{itemize}
        \item\SloganFont{Objects. }The objects of $\sfRel$ are sets.
        \item\SloganFont{Morphisms. }For each $A,B\in\Obj(\Sets)$, we have
            \[
                \sfRel(A,B)
                \defeq
                \Rel(A,B).
            \]%
        \item\SloganFont{Identities. }For each $A\in\Obj(\sfRel)$, the unit map
            \[
                \Unit^{\sfRel}_{A}
                \colon
                \pt
                \to
                \Rel(A,A)
            \]%
            of $\sfRel$ at $A$ is defined by
            \[
                \id^{\sfRel}_{A}
                \defeq
                \chi_{A}(-_{1},-_{2}),
            \]%
            where $\chi_{A}(-_{1},-_{2})$ is the characteristic relation of $A$ of \cref{the-characteristic-relation}.
        \item\SloganFont{Composition. }For each $A,B,C\in\Obj(\sfRel)$, the composition map
            \[
                \circ^{\sfRel}_{A,B,C}%
                \colon%
                \Rel(B,C)%
                \times%
                \Rel(A,B)%
                \to%
                \Rel(A,C)%
            \]%
            of $\sfRel$ at $(A,B,C)$ is defined by%
            \[
                S\mathbin{{\circ}^{\sfRel}_{A,B,C}}R
                \defeq
                S\procirc R
            \]%
            for each $(S,R)\in\Rel(B,C)\times\Rel(A,B)$, where $S\procirc R$ is the composition of $S$ and $R$ of \cref{composition-of-relations}.
    \end{itemize}
\end{definition}
\subsection{The Closed Symmetric Monoidal Category of Relations}\label{subsection-the-closed-symmetric-monoidal-category-of-relations}
\subsubsection{The Monoidal Product}\label{subsubsection-the-closed-symmetric-monoidal-category-of-relations-the-monoidal-product}
\begin{definition}{The Monoidal Product of $\sfRel$}{the-monoidal-product-of-rel}%
    The \index[set-theory]{category of relations!monoidal product of}\textbf{monoidal product of $\sfRel$} is the functor
    \[
        \times%
        \colon%
        \sfRel\times\sfRel%
        \to%
        \sfRel%
    \]%
    where
    \begin{itemize}
        \item\SloganFont{Action on Objects. }For each $A,B\in\Obj(\sfRel)$, we have%
            \[%
                \mathord{\times}(A,B)%
                \defeq%
                A\times B,%
            \]%
            where $A\times B$ is the Cartesian product of sets of \ChapterRef{\ChapterConstructionsWithSets, \cref{constructions-with-sets:binary-products-of-sets}}{\cref{binary-products-of-sets}}.
        \item\SloganFont{Action on Morphisms. }For each $(A,C),(B,D)\in\Obj(\sfRel\times\sfRel)$, the action on morphisms
            \[
                \times_{(A,C),(B,D)}%
                \colon%
                \Rel(A,B)\times\Rel(C,D)%
                \to%
                \Rel(A\times C,B\times D)%
            \]%
            of $\times$ is given by sending a pair of morphisms $(R,S)$ of the form
            \begin{align*}
                R &\colon A\rightproarrow B,\\
                S &\colon C\rightproarrow D
            \end{align*}
            to the relation
            \[
                R\times S%
                \colon%
                A\times C%
                \rightproarrow%
                B\times D%
            \]%
            of \ChapterRef{\ChapterConstructionsWithRelations, \cref{constructions-with-relations:binary-products-of-relations}}{\cref{binary-products-of-relations}}.
    \end{itemize}
\end{definition}
\subsubsection{The Monoidal Unit}\label{subsubsection-the-closed-symmetric-monoidal-category-of-relations-the-monoidal-unit}
\begin{definition}{The Monoidal Unit of $\sfRel$}{the-monoidal-unit-of-rel}%
    The \index[set-theory]{category of relations!monoidal unit of}\textbf{monoidal unit of $\sfRel$} is the functor
    \[
        \Unit^{\sfRel}%
        \colon%
        \pt%
        \to%
        \sfRel%
    \]%
    picking the set
    \[
        \Unit_{\sfRel}%
        \defeq%
        \pt%
    \]%
    of $\sfRel$.
\end{definition}
\subsubsection{The Associator}\label{subsubsection-the-closed-symmetric-monoidal-category-of-relations-the-associator}
\begin{definition}{The Associator of $\sfRel$}{the-associator-of-rel}%
    The \index[set-theory]{category of relations!associator of}\textbf{associator of $\sfRel$} is the natural isomorphism%
    \[
        \alpha^{\sfRel}%
        \colon%
        {\times}\circ{({(\times)}\times{\sfid})}%
        \Longrightisoarrow%
        {\times}\circ{({\sfid}\times{(\times)})}\circ{\bfalpha^{\Cats}_{\sfRel,\sfRel,\sfRel}}\mrp{,}%
    \]%
    as in the diagram
    \[
        \begin{tikzcd}[row sep={0*\the\DL,between origins}, column sep={0*\the\DL,between origins}, background color=backgroundColor, ampersand replacement=\&]
            \&[0.30901699437\TwoCm]
            \&[0.5\TwoCm]
            \sfRel\times(\sfRel\times\sfRel)
            \&[0.5\TwoCm]
            \&[0.30901699437\TwoCm]
            \\[0.58778525229\TwoCm]
            (\sfRel\times\sfRel)\times\sfRel
            \&[0.30901699437\TwoCm]
            \&[0.5\TwoCm]
            \&[0.5\TwoCm]
            \&[0.30901699437\TwoCm]
            \sfRel\times\sfRel
            \\[0.95105651629\TwoCm]
            \&[0.30901699437\TwoCm]
            \sfRel\times\sfRel
            \&[0.5\TwoCm]
            \&[0.5\TwoCm]
            \sfRel\mrp{,}
            \&[0.30901699437\TwoCm]
            % 1-Arrows
            % Left Boundary
            \arrow[from=2-1,to=1-3,"\bfalpha^{\Cats}_{\sfRel,\sfRel,\sfRel}"{pos=0.1},isoarrowprime, mid vert]%
            \arrow[from=1-3,to=2-5,"{\sfid\times{(\times)}}"{pos=0.95},""{name=2}, mid vert]%
            \arrow[from=2-5,to=3-4,"\times"{pos=0.35}, mid vert]%
            % Right Boundary
            \arrow[from=2-1,to=3-2,"{{(\times)}\times\sfid}"'{pos=0.25}, mid vert]%
            \arrow[from=3-2,to=3-4,"\times"', mid vert]%
            % 2-Arrows
            \arrow[from=3-2,to=2,"\alpha^{\sfRel}"{description,pos=0.475},Rightarrow,shorten <= 0.5*\the\DL,shorten >= 1*\the\DL]%
        \end{tikzcd}
    \]%
    whose component
    \[
        \alpha^{\sfRel}_{A,B,C}%
        \colon%
        (A\times B)\times C%
        \rightproarrow%
        A\times(B\times C)%
    \]%
    at $A,B,C\in\Obj(\sfRel)$ is the relation defined by declaring
    \[
        ((a,b),c)
        \sim_{\alpha^{\sfRel}_{A,B,C}}
        (a',(b',c'))
    \]%
    \textiff $a=a'$, $b=b'$, and $c=c'$.
\end{definition}
\subsubsection{The Left Unitor}\label{subsubsection-the-closed-symmetric-monoidal-category-of-relations-the-left-unitor}
\begin{definition}{The Left Unitor of $\sfRel$}{the-left-unitor-of-rel}%
    The \index[set-theory]{category of relations!left unitor of}\textbf{left unitor of $\sfRel$} is the natural isomorphism%
    \begin{webcompile}
        \LUnitor^{\sfRel}
        \colon
        {\times}\circ{({\Unit^{\sfRel}}\times{\sfid})}
        \Longrightisoarrow
        \bfLUnitor^{\TwoCategoryOfCategories}_{\sfRel},%
        \quad
        \begin{tikzcd}[row sep={9.0*\the\DL,between origins}, column sep={9.0*\the\DL,between origins}, background color=backgroundColor, ampersand replacement=\&]
            \PunctualCategory\times\sfRel
            \arrow[r,  "\Unit^{\sfRel}\times\sfid"]
            \arrow[rd, dashed,"\bfLUnitor^{\TwoCategoryOfCategories}_{\sfRel}"'{name=1,pos=0.475},bend right=30]
            \&
            \sfRel\times\sfRel\mathrlap{,}
            \arrow[d, "\times"]
            \\
            {}
            \&
            \sfRel
            % 2-Arrows
            \arrow[Rightarrow,from=1-2,to=1,shorten >=1.0*\the\DL,shorten <=1.0*\the\DL,"\LUnitor^{\sfRel}"description]
        \end{tikzcd}
    \end{webcompile}%
    whose component
    \[
        \LUnitor^{\sfRel}_{A}
        \colon
        \Unit_{\sfRel}\times A
        \rightproarrow%
        A
    \]%
    at $A$ is defined by declaring
    \[
        (\point,a)
        \sim_{\LUnitor^{\sfRel}_{A}}
        b
    \]%
    \textiff $a=b$.
\end{definition}
\subsubsection{The Right Unitor}\label{subsubsection-the-closed-symmetric-monoidal-category-of-relations-the-right-unitor}
\begin{definition}{The Right Unitor of $\sfRel$}{the-right-unitor-of-rel}%
    The \index[set-theory]{category of relations!right unitor of}\textbf{right unitor of $\sfRel$} is the natural isomorphism%
    \begin{webcompile}
        \RUnitor^{\sfRel}
        \colon
        {\times}\circ{({\sfid}\times{\Unit^{\sfRel}})}
        \Longrightisoarrow%
        \bfRUnitor^{\TwoCategoryOfCategories}_{\sfRel},
        \quad
        \begin{tikzcd}[row sep={9.0*\the\DL,between origins}, column sep={9.0*\the\DL,between origins}, background color=backgroundColor, ampersand replacement=\&]
            \sfRel\times\PunctualCategory
            \arrow[r, "\sfid\times\Unit^{\sfRel}"]
            \arrow[rd, dashed,"\bfRUnitor^{\TwoCategoryOfCategories}_{\sfRel}"'{name=1,pos=0.475},bend right=30]
            \&
            \sfRel\times\sfRel\mathrlap{,}
            \arrow[d, "\times"]
            \\
            {}
            \&
            \sfRel
            % 2-Arrows
            \arrow[Rightarrow,from=1-2,to=1,shorten >=1.0*\the\DL,shorten <=1.0*\the\DL,"\RUnitor^{\sfRel}"description]
        \end{tikzcd}
    \end{webcompile}%
    whose component
    \[
        \RUnitor^{\sfRel}_{A}
        \colon
        A\times\Unit_{\sfRel}
        \rightproarrow
        A
    \]%
    at $A$ is defined by declaring
    \[
        (a,\point)
        \sim_{\RUnitor^{\sfRel}_{A}}
        b
    \]%
    \textiff $a=b$.
\end{definition}
\subsubsection{The Symmetry}\label{subsubsection-the-closed-symmetric-monoidal-category-of-relations-the-symmetry}
\begin{definition}{The Symmetry of $\sfRel$}{the-symmetry-of-rel}%
    The \index[set-theory]{category of relations!symmetry of}\textbf{symmetry of $\sfRel$} is the natural isomorphism%
    \begin{webcompile}
        \sigma^{\sfRel}
        \colon
        \times
        \Longrightarrow
        {\times}\circ{\bfsigma^{\TwoCategoryOfCategories}_{\sfRel,\sfRel}},
        \quad
        \begin{tikzcd}[row sep={5.0*\the\DL,between origins}, column sep={4.0*\the\DL,between origins}, background color=backgroundColor, ampersand replacement=\&]
            \sfRel\times\sfRel
            \arrow[rr,"\times"{name=1},pos=0.3425]
            \arrow[rd,"\bfsigma^{\TwoCategoryOfCategories}_{\sfRel,\sfRel}"'{pos=0.25},bend right=15]
            \&
            {}
            \&
            \sfRel\mrp{,}
            \\
            \&
            \sfRel\times\sfRel
            \arrow[ru,"\times"'{pos=0.525},bend right=15]
            \&
            % 2-arrows
            \arrow[from=1-2,to=2-2,"\sigma^{\sfRel}"{description,pos=0.425},shorten <= 0.0*\the\DL,shorten >=0.25*\the\DL,Rightarrow]%
        \end{tikzcd}
    \end{webcompile}%
    whose component
    \[
        \sigma^{\sfRel}_{A,B}
        \colon
        A\times B
        \to
        B\times A
    \]%
    at $(A,B)$ is defined by declaring
    \[
        (a,b)
        \sim_{\sigma^{\sfRel}_{A,B}}
        (b',a')
    \]%
    \textiff $a=a'$ and $b=b'$.
\end{definition}
\subsubsection{The Internal Hom}\label{subsubsection-the-closed-symmetric-monoidal-category-of-relations-the-internal-hom}
\begin{definition}{The Internal Hom of $\sfRel$}{the-internal-hom-of-rel}%
    The \index[set-theory]{category of relations!internal Hom of}\textbf{internal Hom of $\sfRel$} is the functor%
    \[%
        \Rel%
        \colon%
        \sfRel^{\op}\times\sfRel%
        \to
        \sfRel%
    \]%
    defined
    \begin{itemize}
        \item On objects by sending $A,B\in\Obj(\sfRel)$ to the set $\Rel(A,B)$ of \cref{the-set-of-relations-between-two-sets-1} of \cref{the-set-of-relations-between-two-sets}.
        \item On morphisms by pre/post-composition defined as in \cref{composition-of-relations}.
    \end{itemize}
\end{definition}
\begin{proposition}{Properties of the Internal Hom of $\sfRel$}{properties-of-the-internal-hom-of-rel}%
    Let $A,B,C\in\Obj(\sfRel)$.
    \begin{enumerate}
        \item\label{properties-of-the-internal-hom-of-rel-adjointness}\SloganFont{Adjointness. }We have adjunctions
            \begin{webcompile}%TODO: I don't like the spacing here
                \begin{gathered}
                    \Adjunction#A\times -#{\Rel(A,-)}#\sfRel#\sfRel,#\\
                    \Adjunction#-\times B#{\Rel(B,-)}#\sfRel#\sfRel,#
                \end{gathered}
            \end{webcompile}%
            witnessed by bijections
            \begin{align*}
                \Rel(A\times B,C) &\cong \Rel(A,\Rel(B,C)),\\
                \Rel(A\times B,C) &\cong \Rel(B,\Rel(A,C)),
            \end{align*}
            natural in $A,B,C\in\Obj(\sfRel)$.
        %\item\label{properties-of-the-internal-hom-of-rel-}\SloganFont{. }
    \end{enumerate}
\end{proposition}
\begin{Proof}{Proof of \cref{properties-of-the-internal-hom-of-rel}}%
    \ProofBox{\cref{properties-of-the-internal-hom-of-rel-adjointness}: Adjointness}%
    Indeed, we have
    \begin{align*}
        \Rel(A\times B,C) &\defeq \Sets(A\times B\times C,\TV)\\
                          &\defeq \Rel(A,B\times C)\\
                          &\defeq \Rel(A,\Rel(B,C)),
    \end{align*}
    and similarly for the bijection $\Rel(A\times B,C)\cong\Rel(B,\Rel(A,C))$.
\end{Proof}
\subsubsection{The Closed Symmetric Monoidal Category of Relations}\label{subsubsection-the-closed-symmetric-monoidal-category-of-relations}
\begin{proposition}{The Closed Symmetric Monoidal Category of Relations}{the-closed-symmetric-monoidal-category-of-relations}%
    The category $\sfRel$ admits a closed symmetric monoidal category structure consisting of\index[set-theory]{relation!closed symmetric monoidal category of}%
    %--- Begin Footnote ---%
    \footnote{%
        \textdbend\SloganFont{Warning: }This is not a Cartesian monoidal structure, as the product on $\sfRel$ is in fact given by the disjoint union of sets; see \ChapterRef{\ChapterConstructionsWithRelations, \cref{TODO}}{\cref{TODO}}.
        \par\vspace*{\TCBBoxCorrection}
    }%
    %---  End Footnote  ---%
    \begin{itemize}
        \item\SloganFont{The Underlying Category. }The category $\sfRel$ of sets and relations of \cref{the-category-of-relations}.
        \item\SloganFont{The Monoidal Product. }The functor
            \[
                \times%
                \colon%
                \Rel\times\Rel%
                \to%
                \Rel
            \]%
            of \cref{the-monoidal-product-of-rel}.
        \item\SloganFont{The Internal Hom. }The internal Hom functor
            \[
                \eRel%
                \colon%
                \Rel^{\op}\times\Rel%
                \to%
                \Rel%
            \]%
            of \cref{the-internal-hom-of-rel}.
        \item\SloganFont{The Monoidal Unit. }The functor
            \[
                \Unit^{\Rel}
                \colon
                \PunctualCategory
                \to
                \Rel
            \]
            of \cref{the-monoidal-unit-of-rel}.
        \item\SloganFont{The Associators. }The natural isomorphism
            \[
                \alpha^{\Rel}
                \colon
                {\times}\circ{({\times}\times\id_{\Rel})}
                \Longrightisoarrow
                {\times}\circ{(\id_{\Rel}\times{\times})}\circ{\bfalpha^{\Cats}_{\Rel,\Rel,\Rel}}
            \]
            of \cref{the-associator-of-rel}.
        \item\SloganFont{The Left Unitors. }The natural isomorphism
            \[
                \LUnitor^{\Rel}%
                \colon%
                {\times}\circ{(\Unit^{\Rel}\times\id_{\Rel})}
                \Longrightisoarrow
                \bfLUnitor^{\TwoCategoryOfCategories}_{\Rel}
            \]
            of \cref{the-left-unitor-of-rel}.
        \item\SloganFont{The Right Unitors. }The natural isomorphism
            \[
                \RUnitor^{\Rel}%
                \colon%
                {\times}\circ{({\sfid}\times{\Unit^{\Rel}})}%
                \Longrightisoarrow%
                \bfRUnitor^{\TwoCategoryOfCategories}_{\Rel}%
            \]
            of \cref{the-right-unitor-of-rel}.
        \item\SloganFont{The Symmetry. }The natural isomorphism
            \[
                \sigma^{\Rel}
                \colon
                {\times}
                \Longrightisoarrow
                {\times}\circ{\bfsigma^{\TwoCategoryOfCategories}_{\Rel,\Rel}}
            \]
            of \cref{the-symmetry-of-rel}.
    \end{itemize}
\end{proposition}
\begin{Proof}{Proof of \cref{the-closed-symmetric-monoidal-category-of-relations}}%
    Omitted.
\end{Proof}
\subsection{The 2-Category of Relations}\label{subsection-the-2-category-of-relations}
\begin{definition}{The 2-Category of Relations}{the-2-category-of-relations}%
    The \index[set-theory]{relation!two-category of@2-category of}\textbf{2-category of relations} is the locally posetal 2-category \index[notation]{Rel@$\sfbfRel$}$\sfbfRel$ where
    \begin{itemize}
        \item\SloganFont{Objects. }The objects of $\sfbfRel$ are sets.
        \item\SloganFont{$\eHom$-Objects. }For each $A,B\in\Obj(\Sets)$, we have
            \begin{align*}
                \Hom_{\sfbfRel}(A,B) &\defeq \eRel(A,B)\\%
                                     &\defeq (\Rel(A,B),\subset).%
            \end{align*}
        \item\SloganFont{Identities. }For each $A\in\Obj(\sfbfRel)$, the unit map
            \[
                \Unit^{\sfbfRel}_{A}
                \colon
                \pt
                \to
                \eRel(A,A)
            \]%
            of $\sfbfRel$ at $A$ is defined by
            \[
                \id^{\sfbfRel}_{A}
                \defeq
                \chi_{A}(-_{1},-_{2}),
            \]%
            where $\chi_{A}(-_{1},-_{2})$ is the characteristic relation of $A$ of \cref{the-characteristic-relation}.
        \item\SloganFont{Composition. }For each $A,B,C\in\Obj(\sfbfRel)$, the composition map%
            %--- Begin Footnote ---%
            \footnote{%
                That this is indeed a morphism of posets is proven in \cref{properties-of-composition-of-relations-interaction-with-inclusions} of \cref{properties-of-composition-of-relations}.
                \par\vspace*{\TCBBoxCorrection}
            }%
            %---  End Footnote  ---%
            \[
                \circ^{\sfbfRel}_{A,B,C}%
                \colon%
                \eRel(B,C)%
                \times%
                \eRel(A,B)%
                \to%
                \eRel(A,C)%
            \]%
            of $\sfbfRel$ at $(A,B,C)$ is defined by%
            \[
                S\mathbin{{\circ}^{\sfbfRel}_{A,B,C}}R
                \defeq
                S\procirc R
            \]%
            for each $(S,R)\in\sfbfRel(B,C)\times\sfbfRel(A,B)$, where $S\procirc R$ is the composition of $S$ and $R$ of \cref{composition-of-relations}.
    \end{itemize}
\end{definition}
\subsection{The Double Category of Relations}\label{subsection-the-double-category-of-relations}
\subsubsection{The Double Category of Relations}\label{subsubsection-the-double-category-of-relations}
\begin{definition}{The Double Category of Relations}{the-double-category-of-relations}%
    The \index[set-theory]{relation!double category of}\index[higher-categories]{double category!of relations}\textbf{double category of relations} is the locally posetal double category \index[notation]{Reldbl@$\dblRel$}$\smash{\dblRel}$ where
    \begin{itemize}
        \item\SloganFont{Objects. }The objects of $\dblRel$ are sets.
        \item\SloganFont{Vertical Morphisms. }The vertical morphisms of $\dblRel$ are maps of sets $f\colon A\to B$.
        \item\SloganFont{Horizontal Morphisms. }The horizontal morphisms of $\dblRel$ are relations $R\colon A\rightproarrow X$.
        \item\SloganFont{2-Morphisms. }A 2-cell
            \[
                \begin{tikzcd}[row sep={5.0*\the\DL,between origins}, column sep={5.0*\the\DL,between origins}, background color=backgroundColor, ampersand replacement=\&]
                    A
                    \arrow[r,mid vert,"R"{name=1}]
                    \arrow[d,"f"']
                    \&
                    B
                    \arrow[d,"g"]
                    \\
                    X
                    \arrow[r,mid vert,"S"'{name=2}]
                    \&
                    Y
                    % 2-Arrows
                    \arrow[from=1,to=2,"\alpha"description,shorten=0.75*\the\DL,Rightarrow]%
                \end{tikzcd}
            \]%
            of $\dblRel$ is either non-existent or an inclusion of relations of the form
            \begin{webcompile}
                R%
                \subset%
                S\circ(f\times g),%
                \quad
                \begin{tikzcd}[row sep={5.0*\the\DL,between origins}, column sep={7.0*\the\DL,between origins}, background color=backgroundColor, ampersand replacement=\&]
                    A\times B
                    \arrow[r,"R"{name=1}]
                    \arrow[d,"f\times g"']
                    \&
                    \TV
                    \arrow[d,"\id_{\TV}"]
                    \\
                    X\times Y
                    \arrow[r,"S"'{name=2}]
                    \&
                    \TV\mrp{.}
                    % 2-Arrows
                    \arrow[from=1-2,to=2-1,"\scalebox{1.5}{$\subset$}"{sloped,description},phantom,shorten <= 0.5*\the\DL,shorten >= 0.625*\the\DL,Rightarrow,pos=0.5]%
                \end{tikzcd}
            \end{webcompile}%
        \item\SloganFont{Horizontal Identities. }The horizontal unit functor of $\dblRel$ is the functor of \cref{the-horizontal-identities-of-dblrel}.
        \item\SloganFont{Vertical Identities. }For each $A\in\Obj(\dblRel)$, we have
            \[
                \id^{\dblRel}_{A}
                \defeq
                \id_{A}.
            \]%
        \item\SloganFont{Identity 2-Morphisms. }For each horizontal morphism $R\colon A\rightproarrow B$ of $\dblRel$, the identity 2-morphism
            \[
                \begin{tikzcd}[row sep={5.0*\the\DL,between origins}, column sep={5.0*\the\DL,between origins}, background color=backgroundColor, ampersand replacement=\&]
                    A
                    \arrow[r,mid vert,"R"{name=1}]
                    \arrow[d,"\id_{A}"']
                    \&
                    B
                    \arrow[d,"\id_{B}"]
                    \\
                    A
                    \arrow[r,mid vert,"R"'{name=2}]
                    \&
                    B
                    % 2-arrows
                    \arrow[from=1,to=2,"\id_{R}"description,shorten=0.75*\the\DL,Rightarrow]%
                \end{tikzcd}
            \]%
            of $R$ is the identity inclusion
            \begin{webcompile}
                R%
                \subset%
                R,%
                \quad
                \begin{tikzcd}[row sep={5.0*\the\DL,between origins}, column sep={7.0*\the\DL,between origins}, background color=backgroundColor, ampersand replacement=\&]
                    B\times A
                    \arrow[r,"R"]
                    \arrow[d,"\id_{B}\times\id_{A}"']
                    \&
                    \TV
                    \arrow[d,"\id_{\TV}"]
                    \\
                    B\times A
                    \arrow[r,"R"']
                    \&
                    \TV\mrp{.}
                    % 2-Arrows
                    \arrow[from=1-2,to=2-1,"\scalebox{1.5}{$\subset$}"{sloped,description},phantom,shorten <= 0.5*\the\DL,shorten >= 0.625*\the\DL,Rightarrow,pos=0.5]%
                \end{tikzcd}
            \end{webcompile}%
        \item\SloganFont{Horizontal Composition. }The horizontal composition functor of $\dblRel$ is the functor of \cref{the-horizontal-composition-of-dblrel}.
        \item\SloganFont{Vertical Composition of $1$-Morphisms. }For each composable pair $A\smash{\xlongrightarrow{F}}B\smash{\xlongrightarrow{G}}C$ of vertical morphisms of $\dblRel$, i.e.\ maps of sets, we have
            \[
                g\mathbin{{\circ}^{\dblRel}}f
                \defeq
                g\circ f.
            \]%
        \item\SloganFont{Vertical Composition of 2-Morphisms. }The vertical composition of 2-morphisms in $\dblRel$ is defined as in \cref{the-vertical-composition-of-two-morphisms-in-dblrel}.
        \item\SloganFont{Associators. }The associators of $\dblRel$ are defined as in \cref{the-associators-of-dblrel}.
        \item\SloganFont{Left Unitors. }The left unitors of $\dblRel$ are defined as in \cref{the-left-unitors-of-dblrel}.
        \item\SloganFont{Right Unitors. }The right unitors of $\dblRel$ are defined as in \cref{the-right-unitors-of-dblrel}.
    \end{itemize}
\end{definition}
\subsubsection{Horizontal Identities}\label{subsubsection-the-double-category-of-relations-the-associator}
\begin{definition}{The Horizontal Identities of $\dblRel$}{the-horizontal-identities-of-dblrel}%
    The \textbf{horizontal unit functor} of $\dblRel$ is the functor
    \[
        \Unit^{\dblRel}
        \colon
        \dblRel_{0}
        \to
        \dblRel_{1}
    \]%
    of $\dblRel$ is the functor where
    \begin{itemize}
        \item\SloganFont{Action on Objects. }For each $A\in\Obj(\dblRel_{0})$, we have
            \[
                \Unit_{A}
                \defeq
                \chi_{A}(-_{1},-_{2}).
            \]%
        \item\SloganFont{Action on Morphisms. }For each vertical morphism $f\colon A\to B$ of $\dblRel$, i.e.\ each map of sets $f$ from $A$ to $B$, the identity 2-morphism
            \[
                \begin{tikzcd}[row sep={5.0*\the\DL,between origins}, column sep={5.0*\the\DL,between origins}, background color=backgroundColor, ampersand replacement=\&]
                    A
                    \arrow[r,mid vert,"\Unit_{A}"{name=1}]
                    \arrow[d,"f"']
                    \&
                    A
                    \arrow[d,"f"]
                    \\
                    B
                    \arrow[r,mid vert,"\Unit_{B}"'{name=2}]
                    \&
                    B
                    % 2-arrows
                    \arrow[from=1,to=2,"\Unit_{f}"description,shorten=0.75*\the\DL,Rightarrow]%
                \end{tikzcd}
            \]%
            of $f$ is the inclusion
            \begin{webcompile}
                \chi_{B}\circ(f\times f)%
                \subset%
                \chi_{A},%
                \quad%
                \begin{tikzcd}[row sep={5.0*\the\DL,between origins}, column sep={10.0*\the\DL,between origins}, background color=backgroundColor, ampersand replacement=\&]
                    A\times A
                    \arrow[r,"{\chi_{A}(-_{1},-_{2})}"]
                    \arrow[d,"f\times f"']
                    \&
                    \TV
                    \arrow[d,"\id_{\TV}"]
                    \\
                    B\times B
                    \arrow[r,"{\chi_{B}(-_{1},-_{2})}"']
                    \&
                    \TV
                    % 2-Arrows
                    \arrow[from=1-2,to=2-1,"\scalebox{1.5}{$\subset$}"{sloped,description},phantom,shorten <= 0.5*\the\DL,shorten >= 0.625*\the\DL,Rightarrow,pos=0.5]%
                \end{tikzcd}
            \end{webcompile}%
            of \ChapterRef{\ChapterConstructionsWithSets, \cref{constructions-with-sets:properties-of-characteristic-relations-the-inclusion-of-characteristic-relations-associated-to-a-function} of \cref{constructions-with-sets:properties-of-characteristic-relations}}{\cref{properties-of-characteristic-relations-the-inclusion-of-characteristic-relations-associated-to-a-function} of \cref{properties-of-characteristic-relations}}.
    \end{itemize}
\end{definition}
\subsubsection{Horizontal Composition}\label{subsubsection-the-double-category-of-relations-the-horizontal-composition}
\begin{definition}{The Horizontal Composition of $\dblRel$}{the-horizontal-composition-of-dblrel}%
    The \textbf{horizontal composition functor} of $\dblRel$ is the functor
    \[
        \doublecirc^{\dblRel}
        \colon
        \dblRel_{1}\ttimes_{\dblRel_{0}}\dblRel_{1}
        \to
        \dblRel_{1}
    \]%
    of $\dblRel$ is the functor where
    \begin{itemize}
        \item\SloganFont{Action on Objects. }For each composable pair $\smash{A\xrightproarrow{R}B\xrightproarrow{S}C}$ of horizontal morphisms of $\dblRel$, we have
            \[%
                S\doublecirc R%
                \defeq%
                S\procirc R,%
            \]%
            where $S\procirc R$ is the composition of $R$ and $S$ of \cref{composition-of-relations}.
        \item\SloganFont{Action on Morphisms. }For each horizontally composable pair
            \begin{webcompile}
                \begin{tikzcd}[row sep={5.0*\the\DL,between origins}, column sep={5.0*\the\DL,between origins}, background color=backgroundColor, ampersand replacement=\&]
                    A
                    \arrow[r,mid vert,"R"{name=1}]
                    \arrow[d,"f"']
                    \&
                    B
                    \arrow[d,"g"]
                    \\
                    X
                    \arrow[r,mid vert,"T"'{name=2}]
                    \&
                    Y
                    % 2-arrows
                    \arrow[from=1,to=2,"\alpha"description,shorten=0.75*\the\DL,Rightarrow]%
                \end{tikzcd}
                \qquad
                \begin{tikzcd}[row sep={5.0*\the\DL,between origins}, column sep={5.0*\the\DL,between origins}, background color=backgroundColor, ampersand replacement=\&]
                    B
                    \arrow[r,mid vert,"S"{name=1}]
                    \arrow[d,"g"']
                    \&
                    C
                    \arrow[d,"h"]
                    \\
                    Y
                    \arrow[r,mid vert,"U"'{name=2}]
                    \&
                    Z
                    % 2-arrows
                    \arrow[from=1,to=2,"\beta"description,shorten=0.75*\the\DL,Rightarrow]%
                \end{tikzcd}
            \end{webcompile}%
            of 2-morphisms of $\dblRel$, i.e.\ for each pair
            \begin{webcompile}
                \begin{tikzcd}[row sep={5.0*\the\DL,between origins}, column sep={7.0*\the\DL,between origins}, background color=backgroundColor, ampersand replacement=\&]
                    A\times B
                    \arrow[r,"R"]
                    \arrow[d,"f\times g"']
                    \&
                    \TV
                    \arrow[d,"\id_{\TV}"]
                    \\
                    X\times Y
                    \arrow[r,"T"']
                    \&
                    \TV
                    % 2-Arrows
                    \arrow[from=1-2,to=2-1,"\scalebox{1.5}{$\subset$}"{sloped,description},phantom,shorten <= 0.5*\the\DL,shorten >= 0.625*\the\DL,Rightarrow,pos=0.5]%
                \end{tikzcd}
                \qquad
                \begin{tikzcd}[row sep={5.0*\the\DL,between origins}, column sep={7.0*\the\DL,between origins}, background color=backgroundColor, ampersand replacement=\&]
                    B\times C
                    \arrow[r,"S"]
                    \arrow[d,"g\times h"']
                    \&
                    \TV
                    \arrow[d,"\id_{\TV}"]
                    \\
                    Y\times Z
                    \arrow[r,"U"']
                    \&
                    \TV
                    % 2-Arrows
                    \arrow[from=1-2,to=2-1,"\scalebox{1.5}{$\subset$}"{sloped,description},phantom,shorten <= 0.5*\the\DL,shorten >= 0.625*\the\DL,Rightarrow,pos=0.5]%
                \end{tikzcd}
            \end{webcompile}%
            of inclusions of relations, the horizontal composition
            \[
                \begin{tikzcd}[row sep={5.0*\the\DL,between origins}, column sep={5.0*\the\DL,between origins}, background color=backgroundColor, ampersand replacement=\&]
                    A
                    \arrow[r,mid vert,"S\doublecirc R"{name=1}]
                    \arrow[d,"f"']
                    \&
                    C
                    \arrow[d,"h"]
                    \\
                    X
                    \arrow[r,mid vert,"U\doublecirc T"'{name=2}]
                    \&
                    Z
                    % 2-arrows
                    \arrow[from=1,to=2,"\beta\doublecirc\alpha"description,shorten=0.75*\the\DL,Rightarrow]%
                \end{tikzcd}
            \]%
            of $\alpha$ and $\beta$ is the inclusion of relations%
            \begin{webcompile}
                (U\procirc T)\circ(f\times h)%
                \subset%
                (S\procirc R)%
                \quad
                \begin{tikzcd}[row sep={5.0*\the\DL,between origins}, column sep={8.0*\the\DL,between origins}, background color=backgroundColor, ampersand replacement=\&]
                    A\times C
                    \arrow[r,"S\procirc R"]
                    \arrow[d,"f\times h"']
                    \&
                    \TV
                    \arrow[d,"\id_{\TV}"]
                    \\
                    X\times Z
                    \arrow[r,"U\procirc T"']
                    \&
                    \TV\mrp{.}
                    % 2-Arrows
                    \arrow[from=1-2,to=2-1,"\scalebox{1.5}{$\subset$}"{sloped,description},phantom,shorten <= 0.5*\the\DL,shorten >= 0.625*\the\DL,Rightarrow,pos=0.5]%
                \end{tikzcd}
            \end{webcompile}%
    \end{itemize}
\end{definition}
\begin{Proof}{Proof of the Inclusion in \cref{the-horizontal-composition-of-dblrel}}%
    The inclusion of relations
    \[
        (U\procirc T)\circ(f\times h)%
        \subset%
        (S\procirc R)%
    \]%
    follows from the fact that the statement
    \begin{itemize}
        \item We have $a\sim_{(U\procirc T)\circ(f\times h)}c$, i.e.\ $f(a)\sim_{U\procirc T}h(c)$, i.e.\ there exists some $y\in Y$ such that:
            \begin{itemize}
                \item We have $f(a)\sim_{T}y$.
                \item We have $y\sim_{U}h(c)$.
            \end{itemize}
    \end{itemize}
    is implied by the statement
    \begin{itemize}
        \item We have $a\sim_{S\procirc R}c$, i.e.\ there exists some $b\in B$ such that:
            \begin{itemize}
                \item We have $a\sim_{R}b$.
                \item We have $b\sim_{S}c$.
            \end{itemize}
    \end{itemize}
    since:
    \begin{itemize}
        \item If $a\sim_{R}b$, then $f(a)\sim_{T}g(b)$, as $T\circ(f\times g)\subset R$;
        \item If $b\sim_{S}c$, then $g(b)\sim_{U}h(c)$, as $U\circ(g\times h)\subset S$.
    \end{itemize}
    This finishes the proof.
\end{Proof}
\subsubsection{Vertical Composition of 2-Morphisms}\label{subsubsection-the-double-category-of-relations-vertical-composition-of-2-morphisms}
\begin{definition}{The Vertical Composition of 2-Morphisms in $\dblRel$}{the-vertical-composition-of-two-morphisms-in-dblrel}%
    The \textbf{vertical composition} in $\dblRel$ is defined as follows: for each vertically composable pair
    \begin{webcompile}
        \begin{tikzcd}[row sep={5.0*\the\DL,between origins}, column sep={5.0*\the\DL,between origins}, background color=backgroundColor, ampersand replacement=\&]
            A
            \arrow[r,mid vert,"R"{name=1}]
            \arrow[d,"f"']
            \&
            X
            \arrow[d,"g"]
            \\
            B
            \arrow[r,mid vert,"S"'{name=2}]
            \&
            Y
            % 2-arrows
            \arrow[from=1,to=2,"\alpha"description,shorten=0.75*\the\DL,Rightarrow]%
        \end{tikzcd}
        \qquad
        \begin{tikzcd}[row sep={5.0*\the\DL,between origins}, column sep={5.0*\the\DL,between origins}, background color=backgroundColor, ampersand replacement=\&]
            B
            \arrow[r,mid vert,"S"{name=1}]
            \arrow[d,"h"']
            \&
            Y
            \arrow[d,"k"]
            \\
            C
            \arrow[r,mid vert,"T"'{name=2}]
            \&
            Z
            % 2-arrows
            \arrow[from=1,to=2,"\beta"description,shorten=0.75*\the\DL,Rightarrow]%
        \end{tikzcd}
    \end{webcompile}%
    of 2-morphisms of $\dblRel$, i.e.\ for each each pair
    \begin{webcompile}
        \begin{tikzcd}[row sep={5.0*\the\DL,between origins}, column sep={7.0*\the\DL,between origins}, background color=backgroundColor, ampersand replacement=\&]
            A\times X
            \arrow[r,"R"]
            \arrow[d,"f\times g"']
            \&
            \TV
            \arrow[d,"\id_{\TV}"]
            \\
            B\times Y
            \arrow[r,"S"']
            \&
            \TV
            % 2-Arrows
            \arrow[from=1-2,to=2-1,"\scalebox{1.5}{$\subset$}"{sloped,description},phantom,shorten <= 0.5*\the\DL,shorten >= 0.625*\the\DL,Rightarrow,pos=0.5]%
        \end{tikzcd}
        \qquad
        \begin{tikzcd}[row sep={5.0*\the\DL,between origins}, column sep={7.0*\the\DL,between origins}, background color=backgroundColor, ampersand replacement=\&]
            B\times Y
            \arrow[r,"S"]
            \arrow[d,"h\times k"']
            \&
            \TV
            \arrow[d,"\id_{\TV}"]
            \\
            C\times Z
            \arrow[r,"T"']
            \&
            \TV
            % 2-Arrows
            \arrow[from=1-2,to=2-1,"\scalebox{1.5}{$\subset$}"{sloped,description},phantom,shorten <= 0.5*\the\DL,shorten >= 0.625*\the\DL,Rightarrow,pos=0.5]%
        \end{tikzcd}
    \end{webcompile}%
    of inclusions of relations, we define the vertical composition
    \[
        \begin{tikzcd}[row sep={5.0*\the\DL,between origins}, column sep={5.0*\the\DL,between origins}, background color=backgroundColor, ampersand replacement=\&]
            A
            \arrow[r,mid vert,"R"{name=1}]
            \arrow[d,"h\circ f"']
            \&
            X
            \arrow[d,"k\circ g"]
            \\
            C
            \arrow[r,mid vert,"T"'{name=2}]
            \&
            Z
            % 2-arrows
            \arrow[from=1,to=2,"\beta\circ\alpha"description,shorten=0.75*\the\DL,Rightarrow]%
        \end{tikzcd}
    \]%
    of $\alpha$ and $\beta$ as the inclusion of relations
    \begin{webcompile}
        T\circ[(h\circ f)\times(k\circ g)]%
        \subset%
        R,%
        \quad%
        \begin{tikzcd}[row sep={5.0*\the\DL,between origins}, column sep={7.0*\the\DL,between origins}, background color=backgroundColor, ampersand replacement=\&]
            A\times X
            \arrow[r,"R"]
            \arrow[d,"{(h\circ f)\times(k\circ g)}"']
            \&
            \TV
            \arrow[d,"\id_{\TV}"]
            \\
            C\times Z
            \arrow[r,"T"']
            \&
            \TV
            % 2-Arrows
            \arrow[from=1-2,to=2-1,"\scalebox{1.5}{$\subset$}"{sloped,description},phantom,shorten <= 0.5*\the\DL,shorten >= 0.625*\the\DL,Rightarrow,pos=0.5]%
        \end{tikzcd}
    \end{webcompile}%
    given by the pasting of inclusions%
    \[
        \begin{tikzcd}[row sep={5.0*\the\DL,between origins}, column sep={7.5*\the\DL,between origins}, background color=backgroundColor, ampersand replacement=\&]
            A\times X
            \arrow[r,"R"]
            \arrow[d,"f\times g"']
            \&
            \TV
            \arrow[d,"\id_{\TV}"]
            \\
            B\times Y
            \arrow[r,"S"description]
            \arrow[d,"h\times k"']
            \&
            \TV
            \arrow[d,"\id_{\TV}"]
            \\
            C\times Z
            \arrow[r,"T"']
            \&
            \TV\mrp{.}
            % 2-Arrows
            \arrow[from=1-2,to=2-1,"\scalebox{1.5}{$\subset$}"{sloped,description},phantom,shorten <= 0.5*\the\DL,shorten >= 0.625*\the\DL,Rightarrow,pos=0.5]%
            \arrow[from=2-2,to=3-1,"\scalebox{1.5}{$\subset$}"{sloped,description},phantom,shorten <= 0.5*\the\DL,shorten >= 0.625*\the\DL,Rightarrow,pos=0.5]%
        \end{tikzcd}
    \]%
\end{definition}
\begin{Proof}{Proof of the Inclusion in \cref{the-vertical-composition-of-two-morphisms-in-dblrel}}%
    The inclusion
    \[%
        T\circ[(h\circ f)\times(k\circ g)]%
        \subset%
        R%
    \]%
    follows from the fact that, given $(a,x)\in A\times X$, the statement
    \begin{itemize}
        \item We have $h(f(a))\sim_{T}k(g(x))$;
    \end{itemize}
    is implied by the statement
    \begin{itemize}
        \item We have $a\sim_{R}x$;
    \end{itemize}
    since
    \begin{itemize}
        \item If $a\sim_{R}x$,       then $f(a)\sim_{S}g(x)$, as $S\circ(f\times g)\subset R$;
        \item If $b\sim_{S}y$,       then $h(b)\sim_{T}k(y)$, as $T\circ(h\times k)\subset S$, and thus, in particular:
            \begin{itemize}
                \item If $f(a)\sim_{S}g(x)$, then $h(f(a))\sim_{T}k(g(x))$.
            \end{itemize}
    \end{itemize}
    This finishes the proof.
\end{Proof}
\subsubsection{The Associators}\label{subsubsection-the-double-category-of-relations-the-associators}
\begin{definition}{The Associators of $\dblRel$}{the-associators-of-dblrel}%
    For each composable triple%
    \[%
        A\xrightproarrow{R}B\xrightproarrow{S}C\xrightproarrow{T}D%
    \]%
    of horizontal morphisms of $\dblRel$, the component
    \begin{webcompile}
        \alpha^{\dblRel}_{T,S,R}%
        \colon%
        \textcolor{OIvermillion}{(T\doublecirc S)}\doublecirc\textcolor{OIblue}{R}%
        \Longrightisoarrow%
        \textcolor{OIblue}{T}\doublecirc\textcolor{OIvermillion}{(S\doublecirc R)},%
        \quad%
        \begin{tikzcd}[row sep={5.0*\the\DL,between origins}, column sep={3.5*\the\DL,between origins}, background color=backgroundColor, ampersand replacement=\&]
            \textcolor{OIblue}{A}
            \arrow[r, mid vert,"R",OIblue]
            \arrow[d, "\id_{A}"']
            \&
            \textcolor{OIgreen}{B}
            \arrow[r, mid vert,"S"{name=1},OIvermillion]
            %\arrow[d, "g"description]
            \&
            \textcolor{OIvermillion}{C}
            \arrow[r, mid vert,"T",OIvermillion]
            %\arrow[d, "h"]
            \&
            \textcolor{OIvermillion}{D}
            \arrow[d, "\id_{D}"]
            \\
            \textcolor{OIvermillion}{A}
            \arrow[r, mid vert,"R"',OIvermillion]
            \&
            \textcolor{OIvermillion}{B}
            \arrow[r, mid vert,"S"'{name=2},OIvermillion]
            \&
            \textcolor{OIgreen}{C}
            \arrow[r, mid vert,"T"',OIblue]
            \&
            \textcolor{OIblue}{D}
            % 2-Arrows
            \arrow[from=1,to=2,"\alpha^{\dblRel}_{T,S,R}"',shorten=0.75*\the\DL,Rightarrow]
        \end{tikzcd}
    \end{webcompile}%
    of the associator of $\dblRel$ at $(R,S,T)$ is the identity inclusion%
    %--- Begin Footnote ---%
    \footnote{%
        As proved in \cref{properties-of-composition-of-relations-associativity} of \cref{properties-of-composition-of-relations}.
        \par\vspace*{\TCBBoxCorrection}
    }%
    %---  End Footnote  ---%
    \begin{webcompile}
        (T\procirc S)\procirc R%
        =%
        T\procirc(S\procirc R)%
        \quad
        \begin{tikzcd}[row sep={5.0*\the\DL,between origins}, column sep={9.0*\the\DL,between origins}, background color=backgroundColor, ampersand replacement=\&]
            A\times B
            \arrow[r,"(T\procirc S)\procirc R"]
            \arrow[d,Equals]
            \&
            \TV
            \arrow[d,"\id_{\TV}"]
            \\
            A\times B
            \arrow[r,"T\procirc(S\procirc R)"']
            \&
            \TV\mrp{.}
            % 2-Arrows
            \arrow[from=1-2,to=2-1,"\scalebox{1.5}{$=$}"{sloped,description},phantom,shorten <= 0.5*\the\DL,shorten >= 0.625*\the\DL,Rightarrow,pos=0.5]%
        \end{tikzcd}
    \end{webcompile}%
\end{definition}
\subsubsection{The Left Unitors}\label{subsubsection-the-double-category-of-relations-the-left-unitors}
\begin{definition}{The Left Unitors of $\dblRel$}{the-left-unitors-of-dblrel}%
    For each horizontal morphism $R\colon A\rightproarrow B$ of $\dblRel$, the component
    \begin{webcompile}
        \LUnitor^{\dblRel}_{R}
        \colon
        \Unit_{B}\doublecirc R
        \Longrightisoarrow
        R,
        \qquad
        \begin{tikzcd}[row sep={5.0*\the\DL,between origins}, column sep={5.0*\the\DL,between origins}, background color=backgroundColor, ampersand replacement=\&]
            A
            \arrow[r, mid vert,"R"{name=1}]
            \arrow[d, "\id_{A}"']
            \&
            B
            \arrow[r, mid vert,"\Unit_{B}"]
            \&
            B
            \arrow[d, "\id_{B}"]
            \\
            A
            \arrow[rr, mid vert,"R"'{name=2}]
            \&
            \&
            B
            % 2-Arrows
            \arrow[from=1-2,to=2,"\LUnitor^{\dblRel}_{R}"'{pos=0.4},shorten <= 0.25*\the\DL,shorten >= 0.75*\the\DL,Rightarrow]
        \end{tikzcd}
    \end{webcompile}%
    of the left unitor of $\dblRel$ at $R$ is the identity inclusion%
    %--- Begin Footnote ---%
    \footnote{%
        As proved in \cref{properties-of-composition-of-relations-unitality} of \cref{properties-of-composition-of-relations}.
        \par\vspace*{\TCBBoxCorrection}
    }%
    %---  End Footnote  ---%
    \begin{webcompile}
        R%
        =%
        \chi_{B}\procirc R,%
        \qquad
        \begin{tikzcd}[row sep={5.0*\the\DL,between origins}, column sep={8.0*\the\DL,between origins}, background color=backgroundColor, ampersand replacement=\&]
            A\times B
            \arrow[r,"\chi_{B}\procirc R"]
            \arrow[d,Equals]
            \&
            \TV
            \arrow[d,"\id_{\TV}"]
            \\
            A\times B
            \arrow[r,"R"']
            \&
            \TV\mrp{.}
            % 2-Arrows
            \arrow[from=1-2,to=2-1,"\scalebox{1.5}{$=$}"{sloped,description},phantom,shorten <= 0.5*\the\DL,shorten >= 0.625*\the\DL,Rightarrow,pos=0.5]%
        \end{tikzcd}
    \end{webcompile}%
\end{definition}
\subsubsection{The Right Unitors}\label{subsubsection-the-double-category-of-relations-the-right-unitors}
\begin{definition}{The Right Unitors of $\dblRel$}{the-right-unitors-of-dblrel}%
    For each horizontal morphism $R\colon A\rightproarrow B$ of $\dblRel$, the component
    \begin{webcompile}
        \RUnitor^{\dblRel}_{R}
        \colon
        R\doublecirc\Unit_{A}
        \Longrightisoarrow
        R,
        \qquad
        \begin{tikzcd}[row sep={5.0*\the\DL,between origins}, column sep={5.0*\the\DL,between origins}, background color=backgroundColor, ampersand replacement=\&]
            A
            \arrow[r, mid vert,"\Unit_{A}"{name=1}]
            \arrow[d, "\id_{A}"']
            \&
            A
            \arrow[r, mid vert,"R"]
            \&
            B
            \arrow[d, "\id_{B}"]
            \\
            A
            \arrow[rr, mid vert,"R"'{name=2}]
            \&
            \&
            B
            % 2-Arrows
            \arrow[from=1-2,to=2,"\RUnitor^{\dblRel}_{R}"'{pos=0.4},shorten <= 0.25*\the\DL,shorten >= 0.75*\the\DL,Rightarrow]
        \end{tikzcd}
    \end{webcompile}%
    of the right unitor of $\dblRel$ at $R$ is the identity inclusion%
    %--- Begin Footnote ---%
    \footnote{%
        As proved in \cref{properties-of-composition-of-relations-unitality} of \cref{properties-of-composition-of-relations}.
        \par\vspace*{\TCBBoxCorrection}
    }%
    %---  End Footnote  ---%
    \begin{webcompile}
        R%
        =%
        R\procirc\chi_{A},%
        \qquad%
        \begin{tikzcd}[row sep={5.0*\the\DL,between origins}, column sep={8.0*\the\DL,between origins}, background color=backgroundColor, ampersand replacement=\&]
            A\times B
            \arrow[r,"{R\procirc\chi_{A}}"]
            \arrow[d,Equals]
            \&
            \TV
            \arrow[d,"\id_{\TV}"]
            \\
            A\times B
            \arrow[r,"R"']
            \&
            \TV\mrp{.}
            % 2-Arrows
            \arrow[from=1-2,to=2-1,"\scalebox{1.5}{$=$}"{sloped,description},phantom,shorten <= 0.5*\the\DL,shorten >= 0.625*\the\DL,Rightarrow,pos=0.5]%
        \end{tikzcd}
    \end{webcompile}%
\end{definition}
\section{Categories of Relations With Apartness Composition}\label{subsection-categories-of-relations-with-apartness-composition}
\subsection{The Category of Relations With Apartness Composition}\label{subsection-the-category-of-relations-with-apartness-composition}
\begin{definition}{The Category of Relations With Apartness Composition}{the-category-of-relations-with-apartness-composition}%
    The \index[set-theory]{category of relations!with apartness composition}\textbf{category of relations with apartness composition} is the category \index[notation]{Relbox@$\aptsfRel$}$\aptsfRel$ where
    \begin{itemize}
        \item\SloganFont{Objects. }The objects of $\aptsfRel$ are sets.
        \item\SloganFont{Morphisms. }For each $A,B\in\Obj(\Sets)$, we have
            \[
                \aptsfRel(A,B)
                \defeq
                \Rel(A,B).
            \]%
        \item\SloganFont{Identities. }For each $A\in\Obj(\aptsfRel)$, the unit map
            \[
                \Unit^{\aptsfRel}_{A}
                \colon
                \pt
                \to
                \Rel(A,A)
            \]%
            of $\aptsfRel$ at $A$ is defined by
            \[
                \id^{\aptsfRel}_{A}
                \defeq
                \nabla_{A}(-_{1},-_{2}),
            \]%
            where $\nabla_{A}(-_{1},-_{2})$ is the antidiagonal relation of $A$ of \cref{the-antidiagonal-relation}.
        \item\SloganFont{Composition. }For each $A,B,C\in\Obj(\aptsfRel)$, the composition map
            \[
                \circ^{\aptsfRel}_{A,B,C}%
                \colon%
                \Rel(B,C)%
                \times%
                \Rel(A,B)%
                \to%
                \Rel(A,C)%
            \]%
            of $\aptsfRel$ at $(A,B,C)$ is defined by%
            \[
                S\mathbin{{\circ}^{\aptsfRel}_{A,B,C}}R
                \defeq
                S\aptcirc R
            \]%
            for each $(S,R)\in\Rel(B,C)\times\Rel(A,B)$, where $S\procirc R$ is the composition of $S$ and $R$ of \cref{apartness-composition-of-relations}.
    \end{itemize}
\end{definition}
\subsection{The 2-Category of Relations With Apartness Composition}\label{subsection-the-2-category-of-relations-with-apartness-composition}
\begin{definition}{The 2-Category of Relations With Apartness Composition}{the-2-category-of-relations-with-apartness-composition}%
    The \index[set-theory]{relation!two-category of@2-category of!with apartness composition}\textbf{2-category of relations with apartness composition} is the locally posetal 2-category \index[notation]{Rel@$\sfbfRel$}$\sfbfRel$ where
    \begin{itemize}
        \item\SloganFont{Objects. }The objects of $\sfbfRel$ are sets.
        \item\SloganFont{$\eHom$-Objects. }For each $A,B\in\Obj(\Sets)$, we have
            \begin{align*}
                \Hom_{\sfbfRel}(A,B) &\defeq \eRel(A,B)\\%
                                     &\defeq (\Rel(A,B),\subset).%
            \end{align*}
        \item\SloganFont{Identities. }For each $A\in\Obj(\sfbfRel)$, the unit map
            \[
                \Unit^{\sfbfRel}_{A}
                \colon
                \pt
                \to
                \eRel(A,A)
            \]%
            of $\sfbfRel$ at $A$ is defined by
            \[
                \id^{\sfbfRel}_{A}
                \defeq
                \chi_{A}(-_{1},-_{2}),
            \]%
            where $\chi_{A}(-_{1},-_{2})$ is the characteristic relation of $A$ of \cref{the-characteristic-relation}.
        \item\SloganFont{Composition. }For each $A,B,C\in\Obj(\sfbfRel)$, the composition map%
            %--- Begin Footnote ---%
            \footnote{%
                That this is indeed a morphism of posets is proven in \cref{properties-of-apartness-composition-of-relations-interaction-with-inclusions} of \cref{properties-of-apartness-composition-of-relations}.
                \par\vspace*{\TCBBoxCorrection}
            }%
            %---  End Footnote  ---%
            \[
                \circ^{\sfbfRel}_{A,B,C}%
                \colon%
                \eRel(B,C)%
                \times%
                \eRel(A,B)%
                \to%
                \eRel(A,C)%
            \]%
            of $\sfbfRel$ at $(A,B,C)$ is defined by%
            \[
                S\mathbin{{\circ}^{\sfbfRel}_{A,B,C}}R
                \defeq
                S\procirc R
            \]%
            for each $(S,R)\in\sfbfRel(B,C)\times\sfbfRel(A,B)$, where $S\procirc R$ is the composition of $S$ and $R$ of \cref{composition-of-relations}.
    \end{itemize}
\end{definition}
\subsection{The Linear Bicategory of Relations}\label{subsection-the-linear-bicategory-of-relations}
\begin{definition}{The Linear Bicategory of Relations}{the-linear-bicategory-of-relations}%
    The \index[set-theory]{relation!linear bicategory of}\textbf{linear bicategory of relations} is the linear bicategory consisting of:
    \begin{itemize}
        \item\SloganFont{The Underlying Bicategory \rmI. }The bicategory $\sfRel$ of \cref{the-2-category-of-relations}.
        \item\SloganFont{The Underlying Bicategory \rmII. }The bicategory $\sfRel$ of \cref{the-2-category-of-relations-with-apartness-composition}.
        \item\SloganFont{Linear Distributors. }The inclusions
            \begin{align*}
                \delta^{\ell}_{R,S,T} &\colon T\procirc(S\aptcirc R)  \hookrightarrow (T\procirc S)\aptcirc R,\\
                \delta^{r}_{R,S,T}    &\colon (T\aptcirc S)\procirc R \hookrightarrow T\aptcirc(S\procirc R)
            \end{align*}
            of \cref{properties-of-apartness-composition-of-relations-linear-distributivity} of \cref{properties-of-apartness-composition-of-relations}.
    \end{itemize}
\end{definition}
\begin{Proof}{Proof of the Claims in \cref{the-linear-bicategory-of-relations}}%
    Since $\sfRel$ and $\aptsfRel$ are locally posetal, the commutativity of the coherence conditions for linear bicategories follows automatically (\ChapterRef{\ChapterCategories, \cref{categories:properties-of-posetal-categories-automatic-commutativity-of-diagrams} of \cref{categories:properties-of-posetal-categories}}{\cref{categories:properties-of-posetal-categories-automatic-commutativity-of-diagrams} of \cref{categories:properties-of-posetal-categories}}).
\end{Proof}
\subsection{Other Categorical Structures With Apartness Composition}\label{subsection-other-categorical-structures-with-apartness-composition}
\begin{remark}{Other Categorical Structures With Apartness Composition}{other-categorical-structures-with-apartness-composition}%
    It seems apartness composition fails to form the following categorical structures:
    \begin{itemize}
        \item\label{other-categorical-structures-with-apartness-composition-monoidal-category-with-products}\SloganFont{Monoidal Category With Products. }Products don't seem to endow $\aptsfRel$ with a monoidal structure.
        \item\label{other-categorical-structures-with-apartness-composition-monoidal-category-with-coproducts}\SloganFont{Monoidal Category With Coproducts. }Coproducts also don't seem to endow $\aptsfRel$ with a monoidal structure.
        \item\label{other-categorical-structures-with-apartness-composition-double-categorical-structure}\SloganFont{Double Categorical Structure. }It seems the apartness composition of relations doesn't form a double category in a natural way.
    \end{itemize}
\end{remark}
\section{Properties of the 2-Category of Relations}\label{section-properties-of-the-2-category-of-relations}
\subsection{Self-Duality}\label{subsection-self-duality-of-rel}
\begin{proposition}{Self-Duality for the (2-)Category of Relations}{self-duality-for-the-2-category-of-relations}%
    The 2-/category of relations is self-dual:
    \begin{enumerate}
        \item\label{self-duality-for-the-2-category-of-relations-1}\SloganFont{Self-Duality \rmI. }We have an isomorphism
            \[
                \sfRel^{\op}%
                \cong%
                \sfRel%
            \]%
            of categories.
        \item\label{self-duality-for-the-2-category-of-relations-2}\SloganFont{Self-Duality \rmII. }We have a 2-isomorphism
            \[
                \sfbfRel^{\op}%
                \cong%
                \sfbfRel%
            \]%
            of 2-categories.
    \end{enumerate}
\end{proposition}
\begin{Proof}{Proof of \cref{self-duality-for-the-2-category-of-relations}}%
    \FirstProofBox{\cref{self-duality-for-the-2-category-of-relations-1}: Self-Duality \rmI}%
    We claim that the functor
    \[
        (-)^{\dagger}%
        \colon%
        \sfRel^{\op}%
        \to%
        \sfRel%
    \]%
    given by the identity on objects and by $R\mapsto R^{\dagger}$ on morphisms is an isomorphism of categories. Note that this is indeed a functor by \cref{properties-of-converses-of-relations-interaction-with-composition,properties-of-converses-of-relations-identity-1} of \cref{properties-of-converses-of-relations}.

    \indent By \ChapterRef{\ChapterCategories, \cref{categories:properties-of-isomorphisms-of-categories-characterisations} of \cref{categories:properties-of-isomorphisms-of-categories}}{\cref{properties-of-isomorphisms-of-categories-characterisations} of \cref{properties-of-isomorphisms-of-categories}}, it suffices to show that $(-)^{\dagger}$ is bijective on objects (which follows by definition) and fully faithful. Indeed, the map
    \[
        (-)^{\dagger}%
        \colon%
        \Rel(A,B)%
        \to%
        \Rel(B,A)%
    \]%
    defined by the assignment $R\mapsto R^{\dagger}$ is a bijection by \cref{properties-of-converses-of-relations-invertibility} of \cref{properties-of-converses-of-relations}, showing $(-)^{\dagger}$ to be fully faithful.

    \ProofBox{\cref{self-duality-for-the-2-category-of-relations-2}: Self-Duality \rmII}%
    We claim that the 2-functor
    \[
        (-)^{\dagger}%
        \colon%
        \sfRel^{\op}%
        \to%
        \sfRel%
    \]%
    given by the identity on objects, by $R\mapsto R^{\dagger}$ on morphisms, and by preserving inclusions on 2-morphisms via \cref{properties-of-converses-of-relations-functoriality} of \cref{properties-of-converses-of-relations}, is an isomorphism of categories.

    \indent By \cref{TODO}, it suffices to show that $(-)^{\dagger}$ is:
    \begin{itemize}
        \item Bijective on objects, which follows by definition.
        \item Bijective on $1$-morphisms, which was shown in \cref{self-duality-for-the-2-category-of-relations-1}.
        \item Bijective on 2-morphisms, which follows from \cref{properties-of-converses-of-relations-functoriality} of \cref{properties-of-converses-of-relations}.
    \end{itemize}
    Thus $(-)^{\dagger}$ is indeed a 2-isomorphism of categories.
\end{Proof}
\subsection{Isomorphisms and Equivalences}\label{subsection-isomorphisms-and-equivalences-in-rel}
Let $R\colon A\rightproarrow B$ be a relation from $A$ to $B$.
\begin{lemma}{Conditions Involving a Relation and Its Converse}{conditions-involving-a-relation-and-its-converse}%
    The conditions below are row-wise equivalent:
    \begingroup%
    \setlength\cellspacetoplimit{3pt}
    \setlength\cellspacebottomlimit{3pt}
    \renewcommand{\arraystretch}{1.2}
    \begin{center}
        \begin{tabular}{|Sc|Sc|}\\\hline\rowcolor{darkRed}
            \textcolor{white}{\textbf{\textsc{Condition}}} & \textcolor{white}{\textbf{\textsc{Inclusion}}}       \\\hline\rowcolor{backgroundColor}
            $R$ is functional                              & $R\procirc R^{\dagger}\subset \Delta_{B}$            \\\rowcolor{backgroundColor}
            $R$ is surjective                              & $\Delta_{A}           \subset R^{\dagger}\procirc R$ \\\rowcolor{black!05!backgroundColor}
            $R$ is total                                   & $\Delta_{B}           \subset R\procirc R^{\dagger}$ \\\rowcolor{backgroundColor}
            $R$ is injective                               & $R^{\dagger}\procirc R\subset \Delta_{A}$            \\\hline
        \end{tabular}
    \end{center}
    \endgroup
\end{lemma}
\begin{Proof}{Proof of \cref{conditions-involving-a-relation-and-its-converse}}%
    \FirstProofBox{Functionality Is Equivalent to $R\procirc R^{\dagger}\subset\Delta_{B}$}%
    The condition $R\procirc R^{\dagger}\subset\Delta_{B}$ unwinds to
    \begin{itemize}
        \itemstar For each $b,b'\in B$, if there exists some $a\in A$ such that $b\sim_{R^{\dagger}}a$ and $a\sim_{R}b'$, then $b=b'$.
    \end{itemize}
    Since $b\sim_{R^{\dagger}}a$ is the same as $a\sim_{R}b$, the condition says that $a\sim_{R}b$ and $a\sim_{R}b'$ imply $b=b'$. This is precisely the condition for $R$ to be functional.

    \ProofBox{Totality Is Equivalent to $\Delta_{A}\subset R^{\dagger}\procirc R$}%
    The condition $\Delta_{A}\subset R^{\dagger}\procirc R$ unwinds to
    \begin{itemize}
        \itemstar For each $a,a'\in A$, if $a=a'$, then there exists some $b\in B$ such that $a\sim_{R}b$ and $b\sim_{R^{\dagger}}a'$.
    \end{itemize}
    Since $b\sim_{R^{\dagger}}a'$ is the same as $a'\sim_{R}b$, the condition says that for each $a\in A$, there is some $b\in B$ with $b\in R(a)$. This is precisely the condition for $R$ to be total.

    \ProofBox{Surjectivity Is Equivalent to $\Delta_{B}\subset R\procirc R^{\dagger}$}%
    The condition $\Delta_{B}\subset R\procirc R^{\dagger}$ unwinds to
    \begin{itemize}
        \itemstar For each $b,b'\in A$, if $b=b'$, then there exists some $a\in A$ such that $b\sim_{R^{\dagger}}a$ and $a\sim_{R}b'$.
    \end{itemize}
    Since $b\sim_{R^{\dagger}}a$ is the same as $a\sim_{R}b$, the condition says that for each $b\in B$, there is some $a\in A$ with $b\in R(a)$. This is precisely the condition for $R$ to be surjective.

    \ProofBox{Injectivity Is Equivalent to $R^{\dagger}\procirc R\subset\Delta_{A}$}%
    The condition $R^{\dagger}\procirc R\subset\Delta_{A}$ unwinds to
    \begin{itemize}
        \itemstar For each $a,a'\in A$, if there exists some $b\in B$ such that $a\sim_{R}b$ and $b\sim_{R^{\dagger}}a'$, then $a=a'$.
    \end{itemize}
    Since $b\sim_{R^{\dagger}}a'$ is the same as $a'\sim_{R}b$, the condition says that for each $b\in B$, if $a\sim_{R}b$ and $a'\sim_{R}b$, then $a=a'$. This is precisely the condition for $R$ to be injective.
\end{Proof}
\begin{proposition}{Isomorphisms and Equivalences in $\sfbfRel$}{isomorphisms-and-equivalences-in-rel}%
    The following conditions are equivalent:
    \begin{enumerate}
        \item\label{isomorphisms-and-equivalences-in-rel-1}The relation $R\colon A\rightproarrow B$ is an equivalence in $\sfbfRel$, i.e.:
            \begin{itemize}
                \itemstar There exists a relation $R^{-1}\colon B\rightproarrow A$ from $B$ to $A$ together with isomorphisms
                    \begin{align*}
                        R^{-1}\procirc R &\cong \Delta_{A},\\
                        R\procirc R^{-1} &\cong \Delta_{B}.
                    \end{align*}
            \end{itemize}
        \item\label{isomorphisms-and-equivalences-in-rel-2}The relation $R\colon A\rightproarrow B$ is an isomorphism in $\Rel$, i.e.:
            \begin{itemize}
                \itemstar There exists a relation $R^{-1}\colon B\rightproarrow A$ from $B$ to $A$ such that we have
                    \begin{align*}
                        R^{-1}\procirc R &= \Delta_{A},\\
                        R\procirc R^{-1} &= \Delta_{B}.
                    \end{align*}
            \end{itemize}
        \item\label{isomorphisms-and-equivalences-in-rel-3}There exists a bijection $f\colon A\isorightarrow B$ with $R=\Gr(f)$.
    \end{enumerate}
\end{proposition}
\begin{Proof}{Proof of \cref{isomorphisms-and-equivalences-in-rel}}%
    We claim that \cref{isomorphisms-and-equivalences-in-rel-1,isomorphisms-and-equivalences-in-rel-2,isomorphisms-and-equivalences-in-rel-3} are indeed equivalent:
    \begin{itemize}
        \item\SloganFont{\cref{isomorphisms-and-equivalences-in-rel-1}$\iff$\cref{isomorphisms-and-equivalences-in-rel-2}: }This follows from the fact that $\sfbfRel$ is locally posetal, so that natural isomorphisms and equalities of $1$-morphisms in $\sfbfRel$ coincide.
        \item\SloganFont{\cref{isomorphisms-and-equivalences-in-rel-2}$\implies$\cref{isomorphisms-and-equivalences-in-rel-3}: }We proceed in a few steps:
            \begin{itemize}
                \item First, note that the equalities in \cref{isomorphisms-and-equivalences-in-rel-2} imply $R\dashv R^{-1}$ and thus, by \cref{adjunctions-in-rel}, there exists a function $f_{R}\colon A\to B$ associated to $R$.
                \item By \cref{conditions-involving-a-relation-and-its-converse}, $f_{R}$ is a bijection.
            \end{itemize}
        \item\SloganFont{\cref{isomorphisms-and-equivalences-in-rel-3}$\implies$\cref{isomorphisms-and-equivalences-in-rel-2}: }By \cref{properties-of-graphs-of-functions-adjointness-inside-sfbfrel} of \cref{properties-of-graphs-of-functions}, we have an adjunction $\Gr(f)\dashv f^{-1}$, giving inclusions
            \begin{gather*}
                \Delta_{A}            \subset f^{-1}\procirc\Gr(f),\\
                \Gr(f)\procirc f^{-1} \subset \Delta_{B}.
            \end{gather*}
            If $f$ is bijective, then the reverse inclusions are also true by \cref{conditions-involving-a-relation-and-its-converse}.
    \end{itemize}
    This finishes the proof.
\end{Proof}
\subsection{Internal Adjunctions}\label{subsection-internal-adjunctions-in-rel}
Let $A$ and $B$ be sets.
\begin{proposition}{Adjunctions in $\sfbfRel$}{adjunctions-in-rel}%
    We have a natural bijection
    \[
        \{
            \begin{gathered}
                \text{Adjunctions in $\sfbfRel$}\\
                \text{from $A$ to $B$}
            \end{gathered}
        \}%
        \cong
        \{
            \begin{gathered}
                \text{Functions}\\
                \text{from $A$ to $B$}
            \end{gathered}
        \},
    \]%
    with every adjunction in $\sfbfRel$ being of the form $\Gr(f)\dashv f^{-1}$ for some function $f$.
\end{proposition}
\begin{Proof}{Proof of \cref{adjunctions-in-rel}}%
    We proceed step by step:
    \begin{enumerate}
        \item\label{proof-of-adjunctions-in-rel-1}\SloganFont{From Adjunctions in $\sfbfRel$ to Functions. }An adjunction in $\sfbfRel$ from $A$ to $B$ consists of a pair of relations
            \begin{align*}
                R &\colon A\rightproarrow B,\\
                S &\colon B\rightproarrow A,
            \end{align*}
            together with inclusions
            \begin{align*}
                \Delta_{A}  &\subset S\procirc R,\\
                R\procirc S &\subset \Delta_{B}.
            \end{align*}
            By \cref{conditions-involving-a-relation-and-its-converse}, $R$ is total and functional. In particular, $R(a)$ is a singleton for all $a\in A$. Defining $f_{R}$ such that $f_{R}(a)$ is the unique element of $R(a)$ then gives us our desired function, forming a map
            \[
                \{
                    \begin{gathered}
                        \text{Adjunctions in $\sfbfRel$}\\
                        \text{from $A$ to $B$}
                    \end{gathered}
                \}%
                \to
                \{
                    \begin{gathered}
                        \text{Functions}\\
                        \text{from $A$ to $B$}
                    \end{gathered}
                \}.
            \]%
            Moreover, by uniqueness of adjoints (\cref{TODO}), this implies also that $S=f^{-1}$.
        \item\label{proof-of-adjunctions-in-rel-2}\SloganFont{From Functions to Adjunctions in $\sfbfRel$. }By \cref{properties-of-graphs-of-functions-adjointness-inside-sfbfrel} of \cref{properties-of-graphs-of-functions}, every function $f\colon A\to B$ gives rise to an adjunction $\Gr(f)\dashv f^{-1}$ in $\Rel$, giving a map
            \[
                \{
                    \begin{gathered}
                        \text{Functions}\\
                        \text{from $A$ to $B$}
                    \end{gathered}
                \}
                \to
                \{
                    \begin{gathered}
                        \text{Adjunctions in $\sfbfRel$}\\
                        \text{from $A$ to $B$}
                    \end{gathered}
                \}.%
            \]%
        \item\label{proof-of-adjunctions-in-rel-3}\SloganFont{Invertibility: From Functions to Adjunctions Back to Functions. }We need to show that starting with a function $f\colon A\to B$, passing to $\Gr(f)\dashv f^{-1}$, and then passing again to a function gives $f$ again. This follows form the fact that we have $a\sim_{\Gr(f)}b$ iff $f(a)=b$.
        \item\label{proof-of-adjunctions-in-rel-4}\SloganFont{Invertibility: From Adjunctions to Functions Back to Adjunctions. }We need to show that, given an adjunction $R\dashv S$ in $\sfbfRel$ giving rise to a function $f_{R,S}\colon A\to B$, we have
            \begin{align*}
                \Gr(f_{R,S}) &= R,\\
                f^{-1}_{R,S} &= S.
            \end{align*}
            We check these explicitly:
            \begin{itemize}
                \item\SloganFont{$\Gr(f_{R,S})=R$. }We have
                    \begin{align*}
                        \Gr(f_{R,S}) &\defeq \{(a,f_{R,S}(a))\in A\times B\ \middle|\ a\in A\}\\
                                     &\defeq \{(a,R(a))\in A\times B\ \middle|\ a\in A\}\\
                                     &=      R.
                    \end{align*}
                \item\SloganFont{$f^{-1}_{R,S}=S$. }We first claim that, given $a\in A$ and $b\in B$, the following conditions are equivalent:
                    \begin{itemize}
                        \item We have $a\sim_{R}b$.
                        \item We have $b\sim_{S}a$.
                    \end{itemize}
                    Indeed:
                    \begin{itemize}
                        \item\SloganFont{If $a\sim_{R}b$, then $b\sim_{S}a$: }We proceed in a few steps.
                            \begin{itemize}
                                \item Since $\Delta_{A}\subset S\procirc R$, there exists $k\in B$ such that $a\sim_{R}k$ and $k\sim_{S}a$.
                                \item Since $a\sim_{R}b$ and $R$ is functional, we have $k=b$.
                                \item Thus $b\sim_{S}a$.
                            \end{itemize}
                        \item\SloganFont{If $b\sim_{S}a$, then $a\sim_{R}b$: }We proceed in a few steps.
                            \begin{itemize}
                                \item First note that, since $R$ is total, we have $a\sim_{R}b'$ for some $b'\in B$.
                                \item Since $R\procirc S\subset\Delta_{B}$, $b\sim_{S}a$, and $a\sim_{R}b'$, we have $b=b'$.
                                \item Thus $a\sim_{R}b$.
                            \end{itemize}
                    \end{itemize}
                    Having show this, we now have
                    \begin{align*}
                        f^{-1}_{R,S}(b) &\defeq \{a\in A\ \middle|\ f_{R,S}(a)=b\}\\
                                        &\defeq \{a\in A\ \middle|\ a\sim_{R}b\}\\
                                        &=      \{a\in A\ \middle|\ b\sim_{S}a\}\\
                                        &\defeq S(b).
                    \end{align*}
                    for each $b\in B$, and thus $f^{-1}_{R,S}=S$.
            \end{itemize}
    \end{enumerate}
    This finishes the proof.
\end{Proof}
\subsection{Internal Monads}\label{subsection-internal-monads-in-rel}
Let $A$ be a set.
\begin{proposition}{Internal Monads in $\sfbfRel$}{internal-monads-in-rel}%
    We have a natural identification%
    %--- Begin Footnote ---%
    \footnote{%
        See also \cref{section-relative-preorders} for an extension of this correspondence to \say{relative monads in $\sfbfRel$}.
        \par\vspace*{\TCBBoxCorrection}
    }%
    %---  End Footnote  ---%
    \[
        \{
            \begin{gathered}
                \text{Monads in}\\
                \text{$\sfbfRel$ on $A$}
            \end{gathered}
        \}
        \cong
        \{\text{Preorders on $A$}\}.
    \]%
\end{proposition}
\begin{Proof}{Proof of \cref{internal-monads-in-rel}}%
    A monad in $\sfbfRel$ on $A$ consists of a relation $R\colon A\rightproarrow A$ together with maps
    \begin{align*}
        \mu_{R}  &\colon R\procirc R \subset R,\\
        \eta_{R} &\colon \Delta_{A}    \subset R
    \end{align*}
    making the diagrams
    \begin{webcompile}
        \begin{tikzcd}[row sep={5.0*\the\DL,between origins}, column sep={7.0*\the\DL,between origins}, background color=backgroundColor, ampersand replacement=\&,outer sep=0.2*\the\DL]
            \Delta_{A}\procirc R
            \arrow[r,"\eta_{R}\procirc\id_{R}",mid vert]
            \arrow[rd,"\LUnitor^{\eRel(A,B)}_{R}"'{pos=0.55},Equals]
            \&
            R\procirc R
            \arrow[d,"\mu_{R}",mid vert]
            \\
            \&
            R
        \end{tikzcd}
        \begin{tikzcd}[row sep={0*\the\DL,between origins}, column sep={0*\the\DL,between origins}, background color=backgroundColor, ampersand replacement=\&]
            \&[0.30901699437\TwoCm]
            \&[0.5\TwoCm]
            R\procirc(R\procirc R)
            \&[0.5\TwoCm]
            \&[0.30901699437\TwoCm]
            \\[0.58778525229\TwoCm]
            (R\procirc R)\procirc R
            \&[0.30901699437\TwoCm]
            \&[0.5\TwoCm]
            \&[0.5\TwoCm]
            \&[0.30901699437\TwoCm]
            R\procirc R
            \\[0.95105651629\TwoCm]
            \&[0.30901699437\TwoCm]
            R\procirc R
            \&[0.5\TwoCm]
            \&[0.5\TwoCm]
            R
            \&[0.30901699437\TwoCm]
            % 1-Arrows
            % Left Boundary
            \arrow[from=2-1,to=1-3,"\alpha^{\eRel(A,B)}_{R,R,R}"{pos=0.35},Equals]%
            \arrow[from=1-3,to=2-5,"\id_{R}\procirc\mu_{R}"{pos=0.6},mid vert]%
            \arrow[from=2-5,to=3-4,"\mu_{R}"{pos=0.35},mid vert]%
            % Right Boundary
            \arrow[from=2-1,to=3-2,"\mu_{R}\procirc\id_{R}"'{pos=0.3},mid vert]%
            \arrow[from=3-2,to=3-4,"\mu_{R}"',mid vert]%
        \end{tikzcd}
        \begin{tikzcd}[row sep={5.0*\the\DL,between origins}, column sep={7.0*\the\DL,between origins}, background color=backgroundColor, ampersand replacement=\&,outer sep=0.2*\the\DL]
            R\procirc\Delta_{A}
            \arrow[r,"\id_{R}\procirc\eta_{R}",mid vert]
            \arrow[rd,"\RUnitor^{\eRel(A,B)}_{R}"'{pos=0.55},Equals]
            \&
            R\procirc R
            \arrow[d,"\mu_{R}",mid vert]
            \\
            \&
            R
        \end{tikzcd}%
    \end{webcompile}%%
    commute. However, since all morphisms involved are inclusions, the commutativity of the above diagrams is automatic (\ChapterRef{\ChapterCategories, \cref{categories:properties-of-posetal-categories-automatic-commutativity-of-diagrams} of \cref{categories:properties-of-posetal-categories}}{\cref{categories:properties-of-posetal-categories-automatic-commutativity-of-diagrams} of \cref{categories:properties-of-posetal-categories}}), and hence all that is left is the data of the two maps $\mu_{R}$ and $\eta_{R}$, which correspond respectively to the following conditions:
    \begin{enumerate}
        \item\label{proof-of-internal-monads-in-rel-1}For each $a,b,c\in A$, if $a\sim_{R}b$ and $b\sim_{R}c$, then $a\sim_{R}c$.
        \item\label{proof-of-internal-monads-in-rel-2}For each $a\in A$, we have $a\sim_{R}a$.
    \end{enumerate}
    These are exactly the requirements for $R$ to be a preorder (\cref{TODO}). Conversely, any preorder $\preceq$ gives rise to a pair of maps $\mu_{\preceq}$ and $\eta_{\preceq}$, forming a monad on $A$.
\end{Proof}
\begin{example}{Codensity Monads in $\sfbfRel$}{codensity-monads-in-rel}%
    Let $R\colon A\rightproarrow B$ be a relation.
    \begin{enumerate}
        \item\label{codensity-monads-in-rel-ran}The codensity monad $\Ran_{R}(R)\colon B\rightproarrow B$ is given by
            \begin{webcompile}
                [\Ran_{R}(R)](b)%
                =%
                \bigcap_{a\in R^{-1}(b)}R(b)%
                \quad
                \begin{tikzcd}[row sep={5.0*\the\DL,between origins}, column sep={5.0*\the\DL,between origins}, background color=backgroundColor, ampersand replacement=\&]
                    \&%
                    B%
                    \arrow[d, "\Ran_{R}(R)", densely dashed for mid vert, mid vert]%
                    \\%
                    A%
                    \arrow[ru, "R",mid vert]%
                    \arrow[r, "R"',""'{name=R,pos=0.425},mid vert]%
                    \&%
                    B%
                    % 2-Arrows
                    \arrow[from=1-2,to=R,shorten=0.75*\the\DL,Rightarrow,shift left=0.15*\the\DL]%
                \end{tikzcd}
            \end{webcompile}
            for each $b\in B$. Thus, it corresponds to the preorder
            \[
                \mathord{\preceq_{\Ran_{R}(R)}}%
                \colon%
                B\times B%
                \to%
                \TTV%
            \]%
            on $B$ obtained by declaring $b\preceq_{\Ran_{R}(R)}b'$ \textiff the following equivalent conditions are satisfied:
            \begin{enumerate}
                \item\label{codensity-monads-in-rel-1-a}For each $a\in A$, if $a\sim_{R}b$, then $a\sim_{R}b'$.
                \item\label{codensity-monads-in-rel-1-b}We have $R^{-1}(b)\subset R^{-1}(b')$.
            \end{enumerate}
        \item\label{codensity-monads-in-rel-rift}The dual codensity monad $\Rift_{R}(R)\colon A\rightproarrow A$ is given by
            \begin{webcompile}
                [\Rift_{R}(R)](a)%
                =%
                \{a'\in A\ \middle|\ R(a')\subset R(a)\}%
                \quad
                \begin{tikzcd}[row sep={5.0*\the\DL,between origins}, column sep={5.0*\the\DL,between origins}, background color=backgroundColor, ampersand replacement=\&]
                    \&%
                    A%
                    \arrow[d, "R", mid vert]%
                    \\%
                    A%
                    \arrow[ru, "\Rift_{R}(R)",densely dashed for mid vert, mid vert]%
                    \arrow[r, "R"',""'{name=R,pos=0.425},mid vert]%
                    \&%
                    B%
                    % 2-Arrows
                    \arrow[from=1-2,to=R,shorten=0.75*\the\DL,Rightarrow,shift left=0.15*\the\DL]%
                \end{tikzcd}
            \end{webcompile}
            for each $a\in A$. Thus, it corresponds to the preorder
            \[
                \mathord{\preceq_{\Rift_{R}(R)}}%
                \colon%
                A\times A%
                \to%
                \TTV%
            \]%
            on $A$ obtained by declaring $a\preceq_{\Rift_{R}(R)}a'$ \textiff the following equivalent conditions are satisfied:
            \begin{enumerate}
                \item\label{codensity-monads-in-rel-2-a}For each $a\in A$, if $a\sim_{R}b$, then $a'\sim_{R}b$.
                \item\label{codensity-monads-in-rel-2-b}We have $R(a')\subset R(a)$.
            \end{enumerate}
    \end{enumerate}
\end{example}
\subsection{Internal Comonads}\label{subsection-internal-comonads-in-rel}
Let $A$ be a set.
\begin{proposition}{Internal Comonads in $\sfbfRel$}{internal-comonads-in-rel}%
    We have a natural identification
    \[
        \{
            \begin{gathered}
                \text{Comonads in}\\
                \text{$\sfbfRel$ on $A$}
            \end{gathered}
        \}
        \cong
        \{\text{Subsets of $A$}\}.
    \]%
\end{proposition}
\begin{Proof}{Proof of \cref{internal-comonads-in-rel}}%
    A comonad in $\sfbfRel$ on $A$ consists of a relation $R\colon A\rightproarrow A$ together with maps
    \begin{align*}
        \Delta_{R}   &\colon R \subset R\procirc R,\\
        \epsilon_{R} &\colon R \subset \Delta_{A}
    \end{align*}
    making the diagrams
    \begingroup\footnotesize% PDF ONLY, so we use \begingroup\footnotesize instead of \begin{envfootnotesize}
    \begin{webcompile}
        \begin{tikzcd}[row sep={5.0*\the\DL,between origins}, column sep={7.0*\the\DL,between origins}, background color=backgroundColor, ampersand replacement=\&,outer sep=0.2*\the\DL]
            R
            \arrow[r,"\Delta_{R}",mid vert]
            \arrow[rd,"\LUnitor^{\eRel(A,B),-1}_{R}"'{pos=0.55},Equals]
            \&
            R\procirc R
            \arrow[d,"\epsilon_{R}\procirc\id_{R}",mid vert]
            \\
            \&
            \Delta_{A}\procirc R
        \end{tikzcd}
        \begin{tikzcd}[row sep={0*\the\DL,between origins}, column sep={0*\the\DL,between origins}, background color=backgroundColor, ampersand replacement=\&]
            \&[0.30901699437\TwoCmPlusAQuarter]
            \&[0.5\TwoCmPlusAQuarter]
            R\procirc R
            \&[0.5\TwoCmPlusAQuarter]
            \&[0.30901699437\TwoCmPlusAQuarter]
            \\[0.58778525229\TwoCmPlusAQuarter]
            R
            \&[0.30901699437\TwoCmPlusAQuarter]
            \&[0.5\TwoCmPlusAQuarter]
            \&[0.5\TwoCmPlusAQuarter]
            \&[0.30901699437\TwoCmPlusAQuarter]
            R\procirc(R\procirc R)
            \\[0.95105651629\TwoCmPlusAQuarter]
            \&[0.30901699437\TwoCmPlusAQuarter]
            R\procirc R
            \&[0.5\TwoCmPlusAQuarter]
            \&[0.5\TwoCmPlusAQuarter]
            (R\procirc R)\procirc R
            \&[0.30901699437\TwoCmPlusAQuarter]
            % 1-Arrows
            % Left Boundary
            \arrow[from=2-1,to=1-3,"\Delta_{R}"{pos=0.5},mid vert]%
            \arrow[from=1-3,to=2-5,"\id_{R}\procirc\Delta_{R}"{pos=0.35},mid vert]%
            \arrow[from=2-5,to=3-4,"\alpha^{\eRel(A,B),-1}_{R,R,R}"{pos=0.35},Equals]%
            % Right Boundary
            \arrow[from=2-1,to=3-2,"\Delta_{R}"'{pos=0.375},mid vert]%
            \arrow[from=3-2,to=3-4,"\Delta_{R}\procirc\id_{R}"',mid vert]%
        \end{tikzcd}
        \begin{tikzcd}[row sep={5.0*\the\DL,between origins}, column sep={7.0*\the\DL,between origins}, background color=backgroundColor, ampersand replacement=\&,outer sep=0.2*\the\DL]
            R
            \arrow[r,"\Delta_{R}",mid vert]
            \arrow[rd,"\RUnitor^{\eRel(A,B),-1}_{R}"'{pos=0.55},Equals]
            \&
            R\procirc R
            \arrow[d,"\id_{R}\procirc\epsilon_{R}",mid vert]
            \\
            \&
            R\procirc\Delta_{A}
        \end{tikzcd}%
    \end{webcompile}%
    \endgroup
    commute. However, since all morphisms involved are inclusions, the commutativity of the above diagrams is automatic (\ChapterRef{\ChapterCategories, \cref{categories:properties-of-posetal-categories-automatic-commutativity-of-diagrams} of \cref{categories:properties-of-posetal-categories}}{\cref{categories:properties-of-posetal-categories-automatic-commutativity-of-diagrams} of \cref{categories:properties-of-posetal-categories}}), and hence all that is left is the data of the two maps $\Delta_{R}$ and $\epsilon_{R}$, which correspond respectively to the following conditions:
    \begin{enumerate}
        \item\label{proof-of-internal-comonads-in-rel-1}For each $a,b\in A$, if $a\sim_{R}b$, then there exists some $k\in A$ such that $a\sim_{R}k$ and $k\sim_{R}b$.
        \item\label{proof-of-internal-comonads-in-rel-2}For each $a,b\in A$, if $a\sim_{R}b$, then $a=b$.
    \end{enumerate}
    Taking $k=b$ in the first condition above shows it to be trivially satisfied, while the second condition implies $R\subset\Delta_{A}$, i.e.\ $R$ must be a subset of $A$. Conversely, any subset $U$ of $A$ satisfies $U\subset\Delta_{A}$, defining a comonad as above.
\end{Proof}
\begin{example}{Density Comonads in $\sfbfRel$}{density-comonads-in-rel}%
    Let $f\colon A\to B$ be a function.
    \begin{enumerate}
        \item\label{density-comonads-in-rel-lan}The density comonad $\Lan_{f}(f)\colon B\rightproarrow B$ is given by
            \begin{webcompile}
                [\Lan_{f}(f)](b)%
                =%
                \bigcup_{a\in f^{-1}(b)}f(a)%
                \quad
                \begin{tikzcd}[row sep={5.0*\the\DL,between origins}, column sep={5.0*\the\DL,between origins}, background color=backgroundColor, ampersand replacement=\&]
                    \&%
                    B%
                    \arrow[d, "\Lan_{f}(f)", densely dashed for mid vert, mid vert]%
                    \\%
                    A%
                    \arrow[ru, "f",mid vert]%
                    \arrow[r, "f"',""'{name=R,pos=0.425},mid vert]%
                    \&%
                    B%
                    % 2-Arrows
                    \arrow[from=1-2,to=R,shorten=0.75*\the\DL,Leftarrow,shift left=0.15*\the\DL]%
                \end{tikzcd}
            \end{webcompile}
            for each $b\in B$. Thus, it corresponds to the image $\Im(f)$ of $f$ as a subset of $B$.
        \item\label{density-comonads-in-rel-rift}The dual density comonad $\Lift_{f^{\dagger}}(f^{\dagger})\colon A\rightproarrow A$ is given by
            \begin{webcompile}
                [\Lift_{f^{\dagger}}(f^{\dagger})](b)%
                =%
                \bigcup_{a\in f^{-1}(b)}f(a)
                \quad
                \begin{tikzcd}[row sep={5.0*\the\DL,between origins}, column sep={5.0*\the\DL,between origins}, background color=backgroundColor, ampersand replacement=\&]
                    \&%
                    B%
                    \arrow[d, "f^{\dagger}", mid vert]%
                    \\%
                    B%
                    \arrow[ru, "\Lift_{f^{\dagger}}(f^{\dagger})",densely dashed for mid vert, mid vert]%
                    \arrow[r, "f^{\dagger}"',""'{name=R,pos=0.425},mid vert]%
                    \&%
                    A%
                    % 2-Arrows
                    \arrow[from=1-2,to=R,shorten=0.75*\the\DL,Leftarrow,shift left=0.15*\the\DL]%
                \end{tikzcd}
            \end{webcompile}
            for each $b\in B$. Thus, it also corresponds to the image $\Im(f)$ of $f$ as a subset of $B$.
    \end{enumerate}
\end{example}
\subsection{Modules Over Internal Monads}\label{subsection-modules-over-internal-monads-in-rel}
Let $A$ be a set.
\begin{proposition}{Modules Over Internal Monads in $\sfbfRel$}{modules-over-internal-monads-in-rel}%
    Let $\preceq_{A}$ be a preorder on $A$, viewed also as an internal monad on $A$ via \cref{internal-monads-in-rel}.
    \begin{enumerate}
        \item\label{modules-over-internal-monads-in-rel-left-modules}\SloganFont{Left Modules. }We have a natural identification%
            \[
                \{\text{Left modules over $\mathord{\preceq_{A}}$}\}%
                \cong
                \{%
                    \begin{aligned}
                        &\text{Relations $R\colon B\rightproarrow A$ such that,}\\%
                        &\text{for each $b\in B$, the set $R(b)$ is}\\%
                        &\text{upward-closed in $A$}%
                    \end{aligned}
                \}.%
            \]%
        \item\label{modules-over-internal-monads-in-rel-right-modules}\SloganFont{Right Modules. }We have a natural identification%
            \[
                \{\text{Right modules over $\mathord{\preceq_{A}}$}\}%
                \cong
                \{%
                    \begin{aligned}
                        &\text{Relations $R\colon A\rightproarrow B$ such that,}\\%
                        &\text{for each $b\in B$, the set $R^{-1}(b)$ is}\\%
                        &\text{downward-closed in $A$}%
                    \end{aligned}
                \}.%
            \]%
        \item\label{modules-over-internal-monads-in-rel-bimodules}\SloganFont{Bimodules. }We have a natural identification
            \[
                \{\text{Bimodules over $\mathord{\preceq_{A}}$}\}%
                \cong
                \{%
                    \begin{gathered}
                        \text{Quadruples $(B,C,R,S)$ such that:}\\%
                        \begin{aligned}
                            &\text{1. For each $b\in B$, the set $R(b)$ is}\\
                            &\phantom{\text{1. }}\text{upward-closed in $A$.}\\%
                            &\text{2. For each $c\in C$, the set $S^{-1}(c)$ is}\\
                            &\phantom{\text{2. }}\text{downward-closed in $A$.}%
                        \end{aligned}
                    \end{gathered}
                \}.%
            \]%
    \end{enumerate}
\end{proposition}
\begin{Proof}{Proof of \cref{modules-over-internal-monads-in-rel}}%
    \FirstProofBox{\cref{modules-over-internal-monads-in-rel-left-modules}: Left Modules}%
    A left module over $\mathord{\preceq_{A}}$ in $\sfbfRel$ consists of a relation $R\colon B\rightproarrow A$ together with an inclusion
    \[
        \alpha_{B}%
        \colon%
        \mathord{\preceq_{A}}\procirc R%
        \subset%
        R
    \]%
    making appropriate diagrams commute. Since $\sfbfRel$ is locally posetal, however, the commutativity of the diagrams in question is automatic (\ChapterRef{\ChapterCategories, \cref{categories:properties-of-posetal-categories-automatic-commutativity-of-diagrams} of \cref{categories:properties-of-posetal-categories}}{\cref{categories:properties-of-posetal-categories-automatic-commutativity-of-diagrams} of \cref{categories:properties-of-posetal-categories}}), and hence all that is left is the data of the inclusion $\alpha_{B}$. This corresponds to the following condition:
    \begin{itemize}
        \itemstar For each $a,a'\in A$, if there exists some $b\in B$ such that $b\sim_{R}a$ and $a\preceq_{a}a'$, then $b\sim_{R}a'$.
    \end{itemize}
    This condition is equivalent to $R(b)$ being downward-closed for all $b\in B$.

    \ProofBox{\cref{modules-over-internal-monads-in-rel-right-modules}: Right Modules}%
    The proof is dual to \cref{modules-over-internal-monads-in-rel-left-modules}, and is therefore omitted.

    \ProofBox{\cref{modules-over-internal-monads-in-rel-bimodules}: Bimodules}%
    Since $\sfbfRel$ is locally posetal, the diagram encoding the compatibility conditions for a bimodule commutes automatically (\ChapterRef{\ChapterCategories, \cref{categories:properties-of-posetal-categories-automatic-commutativity-of-diagrams} of \cref{categories:properties-of-posetal-categories}}{\cref{categories:properties-of-posetal-categories-automatic-commutativity-of-diagrams} of \cref{categories:properties-of-posetal-categories}}), and hence a bimodule is just a left module along with a right module.
\end{Proof}
\subsection{Comodules Over Internal Comonads}\label{subsection-comodules-over-internal-comonads-in-rel}
Let $A$ be a set.
\begin{proposition}{Comodules Over Internal Comonads in $\sfbfRel$}{comodules-over-internal-comonads-in-rel}%
    Let $U$ be a subset of $A$, viewed also as an internal comonad on $A$ via \cref{internal-comonads-in-rel}.
    \begin{enumerate}
        \item\label{comodules-over-internal-comonads-in-rel-left-comodules}\SloganFont{Left Comodules. }We have a natural identification%
            \[
                \{\text{Left comodules over $U$}\}%
                \cong
                \{%
                    \begin{aligned}
                        &\text{Relations $R\colon B\rightproarrow A$ such that,}\\
                        &\text{for each $b\in B$, we have $R(b)\subset U$}
                    \end{aligned}
                \}.%
            \]%
        \item\label{comodules-over-internal-comonads-in-rel-right-comodules}\SloganFont{Right Comodules. }We have a natural identification%
            \[
                \{\text{Right comodules over $U$}\}%
                \cong
                \{%
                    \begin{aligned}
                        &\text{Relations $R\colon A\rightproarrow B$ such that,}\\%
                        &\text{for each $b\in B$, we have $R^{-1}(b)\subset U$}%
                    \end{aligned}
                \}.%
            \]%
        \item\label{comodules-over-internal-comonads-in-rel-bicomodules}\SloganFont{Bicomodules. }We have a natural identification
            \[
                \{\text{Bicomodules over $U$}\}%
                \cong
                \{%
                    \begin{gathered}
                        \text{Quadruples $(B,C,R,S)$ such that:}\\%
                        \begin{aligned}
                            &\text{1. For each $b\in B$, we have $R(b)\subset U$}\\
                            &\text{2. For each $c\in C$, we have $S^{-1}(c)\subset U$}
                        \end{aligned}
                    \end{gathered}
                \}.%
            \]%
    \end{enumerate}
\end{proposition}
\begin{Proof}{Proof of \cref{comodules-over-internal-comonads-in-rel}}%
    \FirstProofBox{\cref{comodules-over-internal-comonads-in-rel-left-comodules}: Left Comodules}%
    A left comodule over $U$ in $\sfbfRel$ consists of a relation $R\colon B\rightproarrow A$ together with an inclusion
    \[
        R
        \subset%
        U\procirc R%
    \]%
    making appropriate diagrams commute. Since $\sfbfRel$ is locally posetal, however, the commutativity of the diagrams in question is automatic (\ChapterRef{\ChapterCategories, \cref{categories:properties-of-posetal-categories-automatic-commutativity-of-diagrams} of \cref{categories:properties-of-posetal-categories}}{\cref{categories:properties-of-posetal-categories-automatic-commutativity-of-diagrams} of \cref{categories:properties-of-posetal-categories}}), and hence all that is left is the data of the inclusion. This corresponds to the following condition:
    \begin{itemize}
        \itemstar For each $b\in B$, if $b\sim_{R}a$, then there exists some $a'\in A$ such that $b\sim_{R}a'$ and $a'\sim_{U}a$.
    \end{itemize}
    Since $a'\sim_{U}a$ is true if $a=a'$ and $a\in U$, this condition ends up being equivalent to $R(b)\subset U$.

    \ProofBox{\cref{comodules-over-internal-comonads-in-rel-right-comodules}: Right Comodules}%
    A right comodule over $U$ in $\sfbfRel$ consists of a relation $R\colon A\rightproarrow B$ together with an inclusion
    \[
        R
        \subset%
        R\procirc U%
    \]%
    making appropriate diagrams commute. Since $\sfbfRel$ is locally posetal, however, the commutativity of the diagrams in question is automatic (\ChapterRef{\ChapterCategories, \cref{categories:properties-of-posetal-categories-automatic-commutativity-of-diagrams} of \cref{categories:properties-of-posetal-categories}}{\cref{categories:properties-of-posetal-categories-automatic-commutativity-of-diagrams} of \cref{categories:properties-of-posetal-categories}}), and hence all that is left is the data of the inclusion. This corresponds to the following condition:
    \begin{itemize}
        \itemstar For each $a\in A$, if $a\sim_{R}b$, then there exists some $x\in A$ such that $a\sim_{U}x$ and $x\sim_{R}b$.
    \end{itemize}
    Since $a\sim_{U}x$ is true if $a=x$ and $a\in U$, this condition ends up being equivalent to $R^{-1}(b)\subset U$.

    \ProofBox{\cref{comodules-over-internal-comonads-in-rel-bicomodules}: Bicomodules}%
    Since $\sfbfRel$ is locally posetal, the diagram encoding the compatibility conditions for a bimodule commutes automatically (\ChapterRef{\ChapterCategories, \cref{categories:properties-of-posetal-categories-automatic-commutativity-of-diagrams} of \cref{categories:properties-of-posetal-categories}}{\cref{categories:properties-of-posetal-categories-automatic-commutativity-of-diagrams} of \cref{categories:properties-of-posetal-categories}}), and hence a bicomodule is just a left comodule along with a right comodule.
\end{Proof}
\subsection{Eilenberg--Moore and Kleisli Objects}\label{subsection-eilenberg-moore-and-kleisli-objects-in-rel}
Let $X$ be a set.
\begin{proposition}{Eilenberg--Moore and Kleisli Objects in $\sfbfRel$}{eilenberg-moore-and-kleisli-objects-in-rel}%
    Let $R$ be a preorder on $X$, viewed as an internal monad on $X$ via \cref{internal-monads-in-rel}.
    \begin{enumerate}
        \item\label{eilenberg-moore-and-kleisli-objects-in-rel-eilenberg-moore-objects-in-rel}\SloganFont{Eilenberg--Moore Objects in $\sfbfRel$. }The Eilenberg--Moore object for $R$ exists \textiff it is an equivalence relation, in which case it is the quotient $X/\unsim_{R}$ of $X$ by $R$.
        \item\label{eilenberg-moore-and-kleisli-objects-in-rel-kleisli-objects-in-rel}\SloganFont{Kleisli Objects in $\sfbfRel$. }[\ldots]
    \end{enumerate}
\end{proposition}
\begin{Proof}{Proof of \cref{eilenberg-moore-and-kleisli-objects-in-rel}}%
    Omitted.
\end{Proof}
\subsection{Co/Monoids}\label{subsection-co-monoids-in-rel}
\begin{remark}{Co/Monoids in $\sfbfRel$}{co-monoids-in-rel}%
    The monoids in $\sfbfRel$ with respect to the Cartesian monoidal structure of \cref{the-closed-symmetric-monoidal-category-of-relations} are called \emph{hypermonoids}, and their theory is explored in \ChapterHypermonoids. Similarly, the comonoids in $\sfbfRel$ are called \emph{hypercomonoids}, and they are defined and studied in \ChapterHypergroups.
\end{remark}
\subsection{Monomorphisms}\label{subsection-monomorphisms-in-rel}
In this section we characterise the epimorphisms in the category $\sfRel$, following \ChapterRef{\ChapterTypesOfMorphismsInCategories, \cref{types-of-morphisms-in-categories:section-monomorphisms}}{\cref{section-monomorphisms}}.
\begin{proposition}{Characterisations of Monomorphisms in $\sfRel$}{characterisations-of-monomorphisms-in-rel}%
    Let $R\colon A\rightproarrow B$ be a relation. The following conditions are equivalent:
    \begin{enumerate}
        \item\label{characterisations-of-monomorphisms-in-rel-1}The relation $R$ is a monomorphism in $\sfRel$.
        \item\label{characterisations-of-monomorphisms-in-rel-2}The direct image function%
            \[
                R_{!}%
                \colon%
                \mathcal{P}(A)%
                \to%
                \mathcal{P}(B)%
            \]%
            associated to $R$ is injective.
        \item\label{characterisations-of-monomorphisms-in-rel-3}The codirect image function
            \[
                R_{*}%
                \colon%
                \mathcal{P}(A)%
                \to%
                \mathcal{P}(B)%
            \]%
            associated to $R$ is injective.
    \end{enumerate}
    Moreover, if $R$ is a monomorphism, then it satisfies the following condition, and the converse holds if $R$ is total:
    \begin{itemize}
        \itemstar For each $a,a'\in A$, if there exists some $b\in B$ such that
            \begin{align*}%
                a\sim_{R}b,\\%
                a'\sim_{R}b,%
            \end{align*}%
            then $a=a'$.
    \end{itemize}
\end{proposition}
\begin{Proof}{Proof of \cref{characterisations-of-monomorphisms-in-rel}}%
    Firstly note that \cref{characterisations-of-monomorphisms-in-rel-2,characterisations-of-monomorphisms-in-rel-3} are equivalent by \ChapterRef{\ChapterConstructionsWithRelations, \cref{constructions-with-relations:properties-of-direct-image-functions-associated-to-relations-relation-to-codirect-images} of \cref{constructions-with-relations:properties-of-direct-image-functions-associated-to-relations}}{\cref{properties-of-direct-image-functions-associated-to-relations-relation-to-codirect-images} of \cref{properties-of-direct-image-functions-associated-to-relations}}. We then claim that \cref{characterisations-of-monomorphisms-in-rel-1,characterisations-of-monomorphisms-in-rel-2} are also equivalent:
    \begin{itemize}
        \item\SloganFont{\cref{characterisations-of-monomorphisms-in-rel-1}$\implies$\cref{characterisations-of-monomorphisms-in-rel-2}: }Let $U,V\in\mathcal{P}(A)$ and consider the diagram
            \[
                \begin{tikzcd}[row sep={4.0*\the\DL,between origins}, column sep={4.0*\the\DL,between origins}, background color=backgroundColor, ampersand replacement=\&]
                    \pt
                    \arrow[r, "U", shift left=0.8, mid vert]
                    \arrow[r, "V"', shift right=0.8, mid vert]
                    \&
                    A
                    \arrow[r, "R",mid vert]
                    \&
                    B\mrp{.}
                \end{tikzcd}
            \]%
            By \ChapterRef{\ChapterConstructionsWithRelations, \cref{constructions-with-relations:unwinding-the-direct-image-function-associated-to-a-relation}}{\cref{unwinding-the-direct-image-function-associated-to-a-relation}}, we have
            \begin{align*}
                R_{!}(U) &= R\procirc U,\\
                R_{!}(V) &= R\procirc V.
            \end{align*}
            Now, if $R\procirc U=R\procirc V$, i.e.\ $R_{!}(U)=R_{!}(V)$, then $U=V$ since $R$ is assumed to be a monomorphism, showing $R_{!}$ to be injective.
        \item\SloganFont{\cref{characterisations-of-monomorphisms-in-rel-2}$\implies$\cref{characterisations-of-monomorphisms-in-rel-1}: }Conversely, suppose that $R_{!}$ is injective, consider the diagram
            \[
                \begin{tikzcd}[row sep={4.0*\the\DL,between origins}, column sep={4.0*\the\DL,between origins}, background color=backgroundColor, ampersand replacement=\&]
                    X
                    \arrow[r, "S", shift left=0.8, mid vert]
                    \arrow[r, "T"', shift right=0.8, mid vert]
                    \&
                    A
                    \arrow[r, "R",mid vert]
                    \&
                    B\mrp{,}
                \end{tikzcd}
            \]%
            and suppose that $R\procirc S=R\procirc T$. Note that, since $R_{!}$ is injective, given a diagram of the form
            \[
                \begin{tikzcd}[row sep={4.0*\the\DL,between origins}, column sep={4.0*\the\DL,between origins}, background color=backgroundColor, ampersand replacement=\&]
                    \pt
                    \arrow[r, "U", shift left=0.8, mid vert]
                    \arrow[r, "V"', shift right=0.8, mid vert]
                    \&
                    A
                    \arrow[r, "R",mid vert]
                    \&
                    B\mrp{,}
                \end{tikzcd}
            \]%
            if $R_{!}(U)=R\procirc U=R\procirc V=R_{!}(V)$, then $U=V$. In particular, for each $x\in X$, we may consider the diagram
            \[
                \begin{tikzcd}[row sep={4.0*\the\DL,between origins}, column sep={4.0*\the\DL,between origins}, background color=backgroundColor, ampersand replacement=\&]
                    \pt
                    \arrow[r, "{[x]}", shift left=0.8, mid vert]
                    \&
                    X
                    \arrow[r, "S", shift left=0.8, mid vert]
                    \arrow[r, "T"', shift right=0.8, mid vert]
                    \&
                    A
                    \arrow[r, "R",mid vert]
                    \&
                    B\mrp{,}
                \end{tikzcd}
            \]%
            for which we have $R\procirc S\procirc[x]=R\procirc T\procirc[x]$, implying that we have
            \[
                S(x)%
                =%
                S\procirc[x]%
                =%
                T\procirc[x]%
                =%
                T(x)
            \]%
            for each $x\in X$, implying $S=T$, and thus $R$ is a monomorphism.
    \end{itemize}
    We can also prove this in a more abstract way, following \cite{MSE350788}:
    \begin{itemize}
        \item\SloganFont{\cref{characterisations-of-monomorphisms-in-rel-1}$\implies$\cref{characterisations-of-monomorphisms-in-rel-2}: }Assume that $R$ is a monomorphism.
            \begin{itemize}
                \item We first notice that the functor $\Rel(\pt,-)\colon\Rel\to\Sets$ maps $R$ to $R_{!}$ by \ChapterRef{\ChapterConstructionsWithRelations, \cref{constructions-with-relations:unwinding-the-direct-image-function-associated-to-a-relation}}{\cref{unwinding-the-direct-image-function-associated-to-a-relation}}.
                \item Since $\Rel(\pt,-)$ preserves all limits by \ChapterRef{\ChapterLimitsAndColimits, \cref{limits-and-colimits:properties-of-co-limits-commutativity-with-homs} of \cref{limits-and-colimits:properties-of-co-limits}}{\cref{properties-of-co-limits-commutativity-with-homs} of \cref{properties-of-co-limits}}, it follows by \ChapterRef{\ChapterTypesOfMorphismsInCategories, \cref{types-of-morphisms-in-categories:properties-of-monomorphism-preserving-functors-interaction-with-limits} of \cref{types-of-morphisms-in-categories:properties-of-monomorphism-preserving-functors}}{\cref{properties-of-monomorphism-preserving-functors-interaction-with-limits} of \cref{properties-of-monomorphism-preserving-functors}} that $\Rel(\pt,-)$ also preserves monomorphisms.
                \item Since $R$ is a monomorphism and $\Rel(\pt,-)$ maps $R$ to $R_{!}$, it follows that $R_{!}$ is also a monomorphism.
                \item Since the monomorphisms in $\Sets$ are precisely the injections (\ChapterRef{\ChapterTypesOfMorphismsInCategories, \cref{types-of-morphisms-in-categories:examples-of-monomorphisms-monomorphisms-in-sets} of \cref{types-of-morphisms-in-categories:examples-of-monomorphisms}}{\cref{examples-of-monomorphisms-monomorphisms-in-sets} of \cref{examples-of-monomorphisms}}), it follows that $R_{!}$ is injective.
            \end{itemize}
        \item\SloganFont{\cref{characterisations-of-monomorphisms-in-rel-2}$\implies$\cref{characterisations-of-monomorphisms-in-rel-1}: }Assume that $R_{!}$ is injective.
            \begin{itemize}
                \item We first notice that the functor $\Rel(\pt,-)\colon\Rel\to\Sets$ maps $R$ to $R_{!}$ by \ChapterRef{\ChapterConstructionsWithRelations, \cref{constructions-with-relations:unwinding-the-direct-image-function-associated-to-a-relation}}{\cref{unwinding-the-direct-image-function-associated-to-a-relation}}.
                \item Since the monomorphisms in $\Sets$ are precisely the injections (\ChapterRef{\ChapterTypesOfMorphismsInCategories, \cref{types-of-morphisms-in-categories:examples-of-monomorphisms-monomorphisms-in-sets} of \cref{types-of-morphisms-in-categories:examples-of-monomorphisms}}{\cref{examples-of-monomorphisms-monomorphisms-in-sets} of \cref{examples-of-monomorphisms}}), it follows that $R_{!}$ is a monomorphism.
                \item Since $\Rel(\pt,-)$ is faithful, it follows by \ChapterRef{\ChapterTypesOfMorphismsInCategories, \cref{types-of-morphisms-in-categories:properties-of-monomorphism-reflecting-functors-interaction-with-faithfulness} of \cref{types-of-morphisms-in-categories:properties-of-monomorphism-reflecting-functors}}{\cref{properties-of-monomorphism-reflecting-functors-interaction-with-faithfulness} of \cref{properties-of-monomorphism-reflecting-functors}} that $\Rel(\pt,-)$ reflects monomorphisms.
                \item Since $R_{!}$ is a monomorphism and $\Rel(\pt,-)$ maps $R$ to $R_{!}$, it follows that $R$ is also a monomorphism.
            \end{itemize}
    \end{itemize}
    Finally, we prove the second part of the statement. Assume that $R$ is a monomorphism, let $a,a'\in A$ such that $a\sim_{R}b$ and $a'\sim_{R}b$ for some $b\in B$, and consider the diagram
    \[
        \begin{tikzcd}[row sep={4.0*\the\DL,between origins}, column sep={4.0*\the\DL,between origins}, background color=backgroundColor, ampersand replacement=\&]
            \pt
            \arrow[r, "{[a]}",   shift left=0.8, mid vert]
            \arrow[r, "{[a']}"', shift right=0.8, mid vert]
            \&
            A
            \arrow[r, "R",mid vert]
            \&
            B\mrp{.}
        \end{tikzcd}
    \]%
    Since $\star\sim_{[a]}a$ and $a\sim_{R}b$, we have $\star\sim_{R\procirc[a]}b$. Similarly, $\star\sim_{R\procirc[a']}b$. Thus $R\procirc[a]=R\procirc[a']$, and since $R$ is a monomorphism, we have $[a]=[a']$, i.e.\ $a=a'$.

    Conversely, assume the condition
    \begin{itemize}
        \itemstar For each $a,a'\in A$, if there exists some $b\in B$ such that
            \begin{align*}%
                a\sim_{R}b,\\%
                a'\sim_{R}b,%
            \end{align*}%
            then $a=a'$.
    \end{itemize}
    consider the diagram
    \[
        \begin{tikzcd}[row sep={4.0*\the\DL,between origins}, column sep={4.0*\the\DL,between origins}, background color=backgroundColor, ampersand replacement=\&]
            X
            \arrow[r, "S",  shift left=0.8,  mid vert]
            \arrow[r, "T"', shift right=0.8, mid vert]
            \&
            A
            \arrow[r, "R",mid vert]
            \&
            B\mrp{,}
        \end{tikzcd}
    \]%
    and let $(x,a)\in S$. Since $R$ is total and $a\in A$, there exists some $b\in B$ such that $a\sim_{R}b$. In this case, we have $x\sim_{R\procirc S}b$, and since $R\procirc S=R\procirc T$, we have also $x\sim_{R\procirc T}b$. Thus there must exist some $a'\in A$ such that $x\sim_{T}a'$ and $a'\sim_{R}b$. However, since $a,a'\sim_{R}b$, we must have $a=a'$, and thus $(x,a)\in T$ as well.

    A similar argument shows that if $(x,a)\in T$, then $(x,a)\in S$, and thus $S=T$ and it follows that $R$ is a monomorphism.
\end{Proof}
\subsection{2-Categorical Monomorphisms}\label{subsection-2-categorical-monomorphisms-in-rel}
In this section we characterise (for now, some of) the 2-categorical monomorphisms in $\sfbfRel$, following \ChapterRef{\ChapterTypesOfMorphismsInBicategories, \cref{types-of-morphisms-in-bicategories:section-monomorphisms-in-bicategories}}{\cref{section-monomorphisms-in-bicategories}}.
\begin{proposition}{2-Categorical Monomorphisms in $\sfbfRel$}{2-categorical-monomorphisms-in-rel}%
    Let $R\colon A\rightproarrow B$ be a relation.
    \begin{enumerate}
        \item\label{2-categorical-monomorphisms-in-rel-representably-faithful-morphisms-in-rel}\SloganFont{Representably Faithful Morphisms in $\sfbfRel$. }Every morphism of $\sfbfRel$ is a representably faithful morphism.
        \item\label{2-categorical-monomorphisms-in-rel-representably-full-morphisms-in-rel}\SloganFont{Representably Full Morphisms in $\sfbfRel$. }The following conditions are equivalent:
            \begin{enumerate}
                \item\label{2-categorical-monomorphisms-in-rel-representably-full-morphisms-in-rel-1}The morphism $R\colon A\rightproarrow B$ is a representably full morphism.
                \item\label{2-categorical-monomorphisms-in-rel-representably-full-morphisms-in-rel-2}For each pair of relations $S,T\colon X\rightproarrows A$, the following condition is satisfied:
                    \begin{itemize}
                        \itemstar If $R\procirc S\subset R\procirc T$, then $S\subset T$.
                    \end{itemize}
                \item\label{2-categorical-monomorphisms-in-rel-representably-full-morphisms-in-rel-3}The functor
                    \[
                        R_{!}%
                        \colon%
                        (\mathcal{P}(A),\subset)%
                        \to%
                        (\mathcal{P}(B),\subset)%
                    \]%
                    is full.
                \item\label{2-categorical-monomorphisms-in-rel-representably-full-morphisms-in-rel-4}For each $U,V\in\mathcal{P}(A)$, if $R_{!}(U)\subset R_{!}(V)$, then $U\subset V$.
                \item\label{2-categorical-monomorphisms-in-rel-representably-full-morphisms-in-rel-5}The functor
                    \[
                        R_{*}%
                        \colon%
                        (\mathcal{P}(A),\subset)%
                        \to%
                        (\mathcal{P}(B),\subset)%
                    \]%
                    is full.
                \item\label{2-categorical-monomorphisms-in-rel-representably-full-morphisms-in-rel-6}For each $U,V\in\mathcal{P}(A)$, if $R_{*}(U)\subset R_{*}(V)$, then $U\subset V$.
            \end{enumerate}
        \item\label{2-categorical-monomorphisms-in-rel-representably-fully-faithful-morphisms-in-rel}\SloganFont{Representably Fully Faithful Morphisms in $\sfbfRel$. }Every representably full morphism in $\sfbfRel$ is a representably fully faithful morphism.
    \end{enumerate}
\end{proposition}
\begin{Proof}{Proof of \cref{2-categorical-monomorphisms-in-rel}}%
    \FirstProofBox{\cref{2-categorical-monomorphisms-in-rel-representably-faithful-morphisms-in-rel}: Representably Faithful Morphisms in $\sfbfRel$}%
    The relation $R$ is a representably faithful morphism in $\sfbfRel$ \textiff, for each $X\in\Obj(\sfbfRel)$, the functor
    \[
        R_{!}%
        \colon%
        \eRel(X,A)%
        \to%
        \eRel(X,B)%
    \]%
    is faithful, i.e.\ \textiff the morphism
    \[
        R_{*|S,T}%
        \colon%
        \Hom_{\eRel(X,A)}(S,T)%
        \to%
        \Hom_{\eRel(X,B)}(R\procirc S,R\procirc T)%
    \]%
    is injective for each $S,T\in\Obj(\eRel(X,A))$. However, $\Hom_{\eRel(X,A)}(S,T)$ is either empty or a singleton, in either case of which the map $R_{*|S,T}$ is necessarily injective.

    \ProofBox{\cref{2-categorical-monomorphisms-in-rel-representably-full-morphisms-in-rel}: Representably Full Morphisms in $\sfbfRel$}%
    We claim \cref{2-categorical-monomorphisms-in-rel-representably-full-morphisms-in-rel-1,2-categorical-monomorphisms-in-rel-representably-full-morphisms-in-rel-2,2-categorical-monomorphisms-in-rel-representably-full-morphisms-in-rel-3,2-categorical-monomorphisms-in-rel-representably-full-morphisms-in-rel-4,2-categorical-monomorphisms-in-rel-representably-full-morphisms-in-rel-5,2-categorical-monomorphisms-in-rel-representably-full-morphisms-in-rel-6} are indeed equivalent:
    \begin{itemize}
        \item\SloganFont{\cref{2-categorical-monomorphisms-in-rel-representably-full-morphisms-in-rel-1}$\iff$\cref{2-categorical-monomorphisms-in-rel-representably-full-morphisms-in-rel-2}: }This is simply a matter of unwinding definitions: The relation $R$ is a representably full morphism in $\sfbfRel$ \textiff, for each $X\in\Obj(\sfbfRel)$, the functor
            \[
                R_{!}%
                \colon%
                \eRel(X,A)%
                \to%
                \eRel(X,B)%
            \]%
            is full, i.e.\ \textiff the morphism
            \[
                R_{*|S,T}%
                \colon%
                \Hom_{\eRel(X,A)}(S,T)%
                \to%
                \Hom_{\eRel(X,B)}(R\procirc S,R\procirc T)%
            \]%
            is surjective for each $S,T\in\Obj(\eRel(X,A))$, i.e.\ \textiff, whenever $R\procirc S\subset R\procirc T$, we also have $S\subset T$.
        \item\SloganFont{\cref{2-categorical-monomorphisms-in-rel-representably-full-morphisms-in-rel-3}$\iff$\cref{2-categorical-monomorphisms-in-rel-representably-full-morphisms-in-rel-4}: }This is also simply a matter of unwinding definitions: The functor
            \[
                R_{!}%
                \colon%
                (\mathcal{P}(A),\subset)%
                \to%
                (\mathcal{P}(B),\subset)%
            \]%
            is full \textiff, for each $U,V\in\mathcal{P}(A)$, the morphism
            \[
                R_{*|U,V}%
                \colon%
                \Hom_{\mathcal{P}(A)}(U,V)%
                \to%
                \Hom_{\mathcal{P}(B)}(R_{!}(U),R_{!}(V))%
            \]%
            is surjective, i.e.\ \textiff whenever $R_{!}(U)\subset R_{!}(V)$, we also necessarily have $U\subset V$.
        \item\SloganFont{\cref{2-categorical-monomorphisms-in-rel-representably-full-morphisms-in-rel-5}$\iff$\cref{2-categorical-monomorphisms-in-rel-representably-full-morphisms-in-rel-6}: }This is once again simply a matter of unwinding definitions, and proceeds exactly in the same way as in the proof of the equivalence between \cref{2-categorical-monomorphisms-in-rel-representably-full-morphisms-in-rel-3,2-categorical-monomorphisms-in-rel-representably-full-morphisms-in-rel-4} given above.
        \item\SloganFont{\cref{2-categorical-monomorphisms-in-rel-representably-full-morphisms-in-rel-4}$\implies$\cref{2-categorical-monomorphisms-in-rel-representably-full-morphisms-in-rel-6}: }Suppose that the following condition is true:
            \begin{itemize}
                \itemstar For each $U,V\in\mathcal{P}(A)$, if $R_{!}(U)\subset R_{!}(V)$, then $U\subset V$.
            \end{itemize}
            We need to show that the condition
            \begin{itemize}
                \itemstar For each $U,V\in\mathcal{P}(A)$, if $R_{*}(U)\subset R_{*}(V)$, then $U\subset V$.
            \end{itemize}
            is also true. We proceed step by step:
            \begin{itemize}
                \item Suppose we have $U,V\in\mathcal{P}(A)$ with $R_{*}(U)\subset R_{*}(V)$.
                \item By \ChapterRef{\ChapterConstructionsWithRelations, \cref{constructions-with-relations:properties-of-codirect-image-functions-associated-to-relations-relation-to-direct-images} of \cref{constructions-with-relations:properties-of-codirect-image-functions-associated-to-relations}}{\cref{properties-of-codirect-image-functions-associated-to-relations-relation-to-direct-images} of \cref{properties-of-codirect-image-functions-associated-to-relations}}, we have
                    \begin{align*}
                        R_{*}(U) &= B\setminus R_{!}(A\setminus U),\\
                        R_{*}(V) &= B\setminus R_{!}(A\setminus V).
                    \end{align*}
                \item By \ChapterRef{\ChapterConstructionsWithSets, \cref{constructions-with-sets:properties-of-differences-functoriality} of \cref{constructions-with-sets:properties-of-differences}}{\cref{properties-of-differences-functoriality} of \cref{properties-of-differences}} we have $R_{!}(A\setminus V)\subset R_{!}(A\setminus U)$.
                \item By assumption, we then have $A\setminus V\subset A\setminus U$.
                \item By \ChapterRef{\ChapterConstructionsWithSets, \cref{constructions-with-sets:properties-of-differences-functoriality} of \cref{constructions-with-sets:properties-of-differences}}{\cref{properties-of-differences-functoriality} of \cref{properties-of-differences}} again, we have $U\subset V$.
            \end{itemize}
        \item\SloganFont{\cref{2-categorical-monomorphisms-in-rel-representably-full-morphisms-in-rel-6}$\implies$\cref{2-categorical-monomorphisms-in-rel-representably-full-morphisms-in-rel-4}: }Suppose that the following condition is true:
            \begin{itemize}
                \itemstar For each $U,V\in\mathcal{P}(A)$, if $R_{*}(U)\subset R_{*}(V)$, then $U\subset V$.
            \end{itemize}
            We need to show that the condition
            \begin{itemize}
                \itemstar For each $U,V\in\mathcal{P}(A)$, if $R_{!}(U)\subset R_{!}(V)$, then $U\subset V$.
            \end{itemize}
            is also true. We proceed step by step:
            \begin{itemize}
                \item Suppose we have $U,V\in\mathcal{P}(A)$ with $R_{!}(U)\subset R_{!}(V)$.
                \item By \ChapterRef{\ChapterConstructionsWithRelations, \cref{constructions-with-relations:properties-of-direct-image-functions-associated-to-relations-relation-to-codirect-images} of \cref{constructions-with-relations:properties-of-direct-image-functions-associated-to-relations}}{\cref{properties-of-direct-image-functions-associated-to-relations-relation-to-codirect-images} of \cref{properties-of-direct-image-functions-associated-to-relations}}, we have
                    \begin{align*}
                        R_{!}(U) &= B\setminus R_{*}(A\setminus U),\\
                        R_{!}(V) &= B\setminus R_{*}(A\setminus V).
                    \end{align*}
                \item By \ChapterRef{\ChapterConstructionsWithSets, \cref{constructions-with-sets:properties-of-differences-functoriality} of \cref{constructions-with-sets:properties-of-differences}}{\cref{properties-of-differences-functoriality} of \cref{properties-of-differences}} we have $R_{*}(A\setminus V)\subset R_{*}(A\setminus U)$.
                \item By assumption, we then have $A\setminus V\subset A\setminus U$.
                \item By \ChapterRef{\ChapterConstructionsWithSets, \cref{constructions-with-sets:properties-of-differences-functoriality} of \cref{constructions-with-sets:properties-of-differences}}{\cref{properties-of-differences-functoriality} of \cref{properties-of-differences}} again, we have $U\subset V$.
            \end{itemize}
        \item\SloganFont{\cref{2-categorical-monomorphisms-in-rel-representably-full-morphisms-in-rel-2}$\implies$\cref{2-categorical-monomorphisms-in-rel-representably-full-morphisms-in-rel-4}: }Consider the diagram
            \[
                \begin{tikzcd}[row sep={4.0*\the\DL,between origins}, column sep={4.0*\the\DL,between origins}, background color=backgroundColor, ampersand replacement=\&]
                    X
                    \arrow[r, "S", shift left=0.8, mid vert]
                    \arrow[r, "T"', shift right=0.8, mid vert]
                    \&
                    A
                    \arrow[r, "R",mid vert]
                    \&
                    B\mrp{,}
                \end{tikzcd}
            \]%
            and suppose that $R\procirc S\subset R\procirc T$. Note that, by assumption, given a diagram of the form
            \[
                \begin{tikzcd}[row sep={4.0*\the\DL,between origins}, column sep={4.0*\the\DL,between origins}, background color=backgroundColor, ampersand replacement=\&]
                    \pt
                    \arrow[r, "U", shift left=0.8, mid vert]
                    \arrow[r, "V"', shift right=0.8, mid vert]
                    \&
                    A
                    \arrow[r, "R",mid vert]
                    \&
                    B\mrp{,}
                \end{tikzcd}
            \]%
            if $R_{!}(U)=R\procirc U\subset R\procirc V=R_{!}(V)$, then $U\subset V$. In particular, for each $x\in X$, we may consider the diagram
            \[
                \begin{tikzcd}[row sep={4.0*\the\DL,between origins}, column sep={4.0*\the\DL,between origins}, background color=backgroundColor, ampersand replacement=\&]
                    \pt
                    \arrow[r, "{[x]}", shift left=0.8, mid vert]
                    \&
                    X
                    \arrow[r, "S", shift left=0.8, mid vert]
                    \arrow[r, "T"', shift right=0.8, mid vert]
                    \&
                    A
                    \arrow[r, "R",mid vert]
                    \&
                    B\mrp{,}
                \end{tikzcd}
            \]%
            for which we have $R\procirc S\procirc[x]\subset R\procirc T\procirc[x]$, implying that we have
            \[
                S(x)%
                =%
                S\procirc[x]%
                \subset%
                T\procirc[x]%
                =%
                T(x)
            \]%
            for each $x\in X$, implying $S\subset T$.
        \item\SloganFont{\cref{2-categorical-monomorphisms-in-rel-representably-full-morphisms-in-rel-4}$\implies$\cref{2-categorical-monomorphisms-in-rel-representably-full-morphisms-in-rel-2}: }Let $U,V\in\mathcal{P}(A)$ and consider the diagram
            \[
                \begin{tikzcd}[row sep={4.0*\the\DL,between origins}, column sep={4.0*\the\DL,between origins}, background color=backgroundColor, ampersand replacement=\&]
                    \pt
                    \arrow[r, "U",  shift left=0.8,  mid vert]
                    \arrow[r, "V"', shift right=0.8, mid vert]
                    \&
                    A
                    \arrow[r, "R",mid vert]
                    \&
                    B\mrp{.}
                \end{tikzcd}
            \]%
            By \cref{unwinding-the-direct-image-function-associated-to-a-relation}, we have
            \begin{align*}
                R_{!}(U) &= R\procirc U,\\
                R_{!}(V) &= R\procirc V.
            \end{align*}
            Now, if $R_{!}(U)\subset R_{!}(V)$, i.e.\ $R\procirc U\subset R\procirc V$, then $U\subset V$ by assumption.
    \end{itemize}

    \ProofBox{\cref{2-categorical-monomorphisms-in-rel-representably-fully-faithful-morphisms-in-rel}: Representably Fully Faithful Morphisms in $\sfbfRel$}%
    This follows from \cref{2-categorical-monomorphisms-in-rel-representably-faithful-morphisms-in-rel,2-categorical-monomorphisms-in-rel-representably-full-morphisms-in-rel}.
\end{Proof}
\begin{question}{Better Characterisations of Representably Full Morphisms in $\sfbfRel$}{better-characterisations-of-representably-full-morphisms-in-rel}%
    \cref{2-categorical-monomorphisms-in-rel-representably-full-morphisms-in-rel} of \cref{2-categorical-monomorphisms-in-rel} gives a characterisation of the representably full morphisms in $\sfbfRel$.

    Are there other nice characterisations of these?

    This question also appears as \cite{MO467527}.
\end{question}
\subsection{Epimorphisms}\label{subsection-epimorphisms-in-rel}
In this section we characterise the epimorphisms in the category $\sfRel$, following \ChapterRef{\ChapterTypesOfMorphismsInCategories, \cref{types-of-morphisms-in-categories:section-epimorphisms}}{\cref{section-epimorphisms}}.
\begin{proposition}{Characterisations of Epimorphisms in $\sfRel$}{characterisations-of-epimorphisms-in-rel}%
    Let $R\colon A\rightproarrow B$ be a relation. The following conditions are equivalent:
    \begin{enumerate}
        \item\label{characterisations-of-epimorphisms-in-rel-1}The relation $R$ is an epimorphism in $\sfRel$.
        \item\label{characterisations-of-epimorphisms-in-rel-2}The weak inverse image function
            \[
                R^{-1}%
                \colon%
                \mathcal{P}(B)%
                \to%
                \mathcal{P}(A)%
            \]%
            associated to $R$ is injective.
        \item\label{characterisations-of-epimorphisms-in-rel-3}The strong inverse image function
            \[
                R_{-1}%
                \colon%
                \mathcal{P}(B)%
                \to%
                \mathcal{P}(A)%
            \]%
            associated to $R$ is injective.
        \item\label{characterisations-of-epimorphisms-in-rel-4}The function $R\colon A\to\mathcal{P}(B)$ is \say{surjective on singletons}:
            \begin{itemize}
                \itemstar For each $b\in B$, there exists some $a\in A$ such that $R(a)=\{b\}$.
            \end{itemize}
    \end{enumerate}
    Moreover, if $R$ is total and an epimorphism, then it satisfies the following equivalent conditions:
    \begin{enumerate}
        \item\label{characterisations-of-epimorphisms-in-rel-5}For each $b\in B$, there exists some $a\in A$ such that $a\sim_{R}b$.
        \item\label{characterisations-of-epimorphisms-in-rel-6}We have $\Im(R)=B$.
    \end{enumerate}
\end{proposition}
\begin{Proof}{Proof of \cref{characterisations-of-epimorphisms-in-rel}}%
    Firstly note that \cref{characterisations-of-epimorphisms-in-rel-2,characterisations-of-epimorphisms-in-rel-3} are equivalent by \ChapterRef{\ChapterConstructionsWithRelations, \cref{constructions-with-relations:properties-of-strong-inverse-image-functions-associated-to-relations-interaction-with-weak-inverse-images-1} of \cref{constructions-with-relations:properties-of-strong-inverse-image-functions-associated-to-relations}}{\cref{properties-of-strong-inverse-image-functions-associated-to-relations-interaction-with-weak-inverse-images-1} of \cref{properties-of-strong-inverse-image-functions-associated-to-relations}}. We then claim that \cref{characterisations-of-epimorphisms-in-rel-1,characterisations-of-epimorphisms-in-rel-2} are also equivalent:
    \begin{itemize}
        \item\SloganFont{\cref{characterisations-of-epimorphisms-in-rel-1}$\implies$\cref{characterisations-of-epimorphisms-in-rel-2}: }Let $U,V\in\mathcal{P}(A)$ and consider the diagram
            \[
                \begin{tikzcd}[row sep={4.0*\the\DL,between origins}, column sep={4.0*\the\DL,between origins}, background color=backgroundColor, ampersand replacement=\&]
                    A
                    \arrow[r, "R",mid vert]
                    \&
                    B
                    \arrow[r, "U", shift left=0.8, mid vert]
                    \arrow[r, "V"', shift right=0.8, mid vert]
                    \&
                    \pt\mrp{.}
                \end{tikzcd}
            \]%
            By \ChapterRef{\ChapterConstructionsWithRelations, \cref{constructions-with-relations:unwinding-the-direct-image-function-associated-to-a-relation}}{\cref{unwinding-the-direct-image-function-associated-to-a-relation}}, we have
            \begin{align*}
                R^{-1}(U) &= U\procirc R,\\
                R^{-1}(V) &= V\procirc R.
            \end{align*}
            Now, if $U\procirc R=V\procirc R$, i.e.\ $R^{-1}(U)=R^{-1}(V)$, then $U=V$ since $R$ is assumed to be an epimorphism, showing $R^{-1}$ to be injective.
        \item\SloganFont{\cref{characterisations-of-epimorphisms-in-rel-2}$\implies$\cref{characterisations-of-epimorphisms-in-rel-1}: }Conversely, suppose that $R^{-1}$ is injective, consider the diagram
            \[
                \begin{tikzcd}[row sep={4.0*\the\DL,between origins}, column sep={4.0*\the\DL,between origins}, background color=backgroundColor, ampersand replacement=\&]
                    A
                    \arrow[r, "R",mid vert]
                    \&
                    B
                    \arrow[r, "S", shift left=0.8, mid vert]
                    \arrow[r, "T"', shift right=0.8, mid vert]
                    \&
                    X\mrp{,}
                \end{tikzcd}
            \]%
            and suppose that $S\procirc R=T\procirc R$. Note that, since $R^{-1}$ is injective, given a diagram of the form
            \[
                \begin{tikzcd}[row sep={4.0*\the\DL,between origins}, column sep={4.0*\the\DL,between origins}, background color=backgroundColor, ampersand replacement=\&]
                    A
                    \arrow[r, "R",mid vert]
                    \&
                    B
                    \arrow[r, "U", shift left=0.8, mid vert]
                    \arrow[r, "V"', shift right=0.8, mid vert]
                    \&
                    \pt\mrp{,}
                \end{tikzcd}
            \]%
            if $R^{-1}(U)=U\procirc R=V\procirc R=R^{-1}(V)$, then $U=V$. In particular, for each $x\in X$, we may consider the diagram
            \[
                \begin{tikzcd}[row sep={4.0*\the\DL,between origins}, column sep={4.0*\the\DL,between origins}, background color=backgroundColor, ampersand replacement=\&]
                    A
                    \arrow[r, "R",mid vert]
                    \&
                    B
                    \arrow[r, "S",  shift left=0.8,  mid vert]
                    \arrow[r, "T"', shift right=0.8, mid vert]
                    \&
                    X
                    \arrow[r, "{[x]}", shift left=0.8, mid vert]
                    \&
                    \pt\mrp{,}
                \end{tikzcd}
            \]%
            for which we have $[x]\procirc S\procirc R=[x]\procirc T\procirc R$, implying that we have
            \[
                S^{-1}(x)%
                =%
                [x]\procirc S%
                =%
                [x]\procirc T%
                =%
                T^{-1}(x)
            \]%
            for each $x\in X$, implying $S=T$, and thus $R$ is an epimorphism.
    \end{itemize}
    We can also prove this in a more abstract way, following \cite{MSE350788}:
    \begin{itemize}
        \item\SloganFont{\cref{characterisations-of-epimorphisms-in-rel-1}$\implies$\cref{characterisations-of-epimorphisms-in-rel-2}: }Assume that $R$ is an epimorphism.
            \begin{itemize}
                \item We first notice that the functor $\Rel(-,\pt)\colon\Rel^{\op}\to\Sets$ maps $R$ to $R^{-1}$ by \ChapterRef{\ChapterConstructionsWithRelations, \cref{constructions-with-relations:unwinding-the-weak-inverse-image-function-associated-to-a-relation}}{\cref{unwinding-the-weak-inverse-image-function-associated-to-a-relation}}.
                \item Since $\Rel(-,\pt)$ preserves limits by \ChapterRef{\ChapterLimitsAndColimits, \cref{limits-and-colimits:properties-of-co-limits-commutativity-with-homs} of \cref{limits-and-colimits:properties-of-co-limits}}{\cref{properties-of-co-limits-commutativity-with-homs} of \cref{properties-of-co-limits}}, it follows by \ChapterRef{\ChapterTypesOfMorphismsInCategories, \cref{types-of-morphisms-in-categories:properties-of-monomorphism-preserving-functors-interaction-with-limits} of \cref{types-of-morphisms-in-categories:properties-of-monomorphism-preserving-functors}}{\cref{properties-of-monomorphism-preserving-functors-interaction-with-limits} of \cref{properties-of-monomorphism-preserving-functors}} that $\Rel(-,\pt)$ also preserves monomorphisms.
                \item That is: $\Rel(-,\pt)$ sends monomorphisms in $\Rel^{\op}$ to monomorphisms in $\Sets$.
                \item The monomorphisms $\Rel^{\op}$ are precisely the epimorphisms in $\Rel$ by \ChapterRef{\ChapterTypesOfMorphismsInCategories, \cref{types-of-morphisms-in-categories:properties-of-monomorphisms-duality} of \cref{types-of-morphisms-in-categories:properties-of-monomorphisms}}{\cref{properties-of-monomorphisms-duality} of \cref{properties-of-monomorphisms}}.
                \item Since $R$ is an epimorphism and $\Rel(-,\pt)$ maps $R$ to $R^{-1}$, it follows that $R^{-1}$ is a monomorphism.
                \item Since the monomorphisms in $\Sets$ are precisely the injections (\ChapterRef{\ChapterTypesOfMorphismsInCategories, \cref{types-of-morphisms-in-categories:examples-of-monomorphisms-monomorphisms-in-sets} of \cref{types-of-morphisms-in-categories:examples-of-monomorphisms}}{\cref{examples-of-monomorphisms-monomorphisms-in-sets} of \cref{examples-of-monomorphisms}}), it follows that $R^{-1}$ is injective.
            \end{itemize}
        \item\SloganFont{\cref{characterisations-of-epimorphisms-in-rel-2}$\implies$\cref{characterisations-of-epimorphisms-in-rel-1}: }Assume that $R^{-1}$ is injective.
            \begin{itemize}
                \item We first notice that the functor $\Rel(-,\pt)\colon\Rel^{\op}\to\Sets$ maps $R$ to $R^{-1}$ by \ChapterRef{\ChapterConstructionsWithRelations, \cref{constructions-with-relations:unwinding-the-weak-inverse-image-function-associated-to-a-relation}}{\cref{unwinding-the-weak-inverse-image-function-associated-to-a-relation}}.
                \item Since the monomorphisms in $\Sets$ are precisely the injections (\ChapterRef{\ChapterTypesOfMorphismsInCategories, \cref{types-of-morphisms-in-categories:examples-of-monomorphisms-monomorphisms-in-sets} of \cref{types-of-morphisms-in-categories:examples-of-monomorphisms}}{\cref{examples-of-monomorphisms-monomorphisms-in-sets} of \cref{examples-of-monomorphisms}}), it follows that $R^{-1}$ is a monomorphism.
                \item Since $\Rel(-,\pt)$ is faithful, it follows by \ChapterRef{\ChapterTypesOfMorphismsInCategories, \cref{types-of-morphisms-in-categories:properties-of-monomorphism-reflecting-functors-interaction-with-faithfulness} of \cref{types-of-morphisms-in-categories:properties-of-monomorphism-reflecting-functors}}{\cref{properties-of-monomorphism-reflecting-functors-interaction-with-faithfulness} of \cref{properties-of-monomorphism-reflecting-functors}} that $\Rel(,\pt)$ reflects monomorphisms.
                \item That is: $\Rel(-,\pt)$ reflects monomorphisms in $\Sets$ to monomorphisms in $\Rel^{\op}$.
                \item The monomorphisms $\Rel^{\op}$ are precisely the epimorphisms in $\Rel$ by \ChapterRef{\ChapterTypesOfMorphismsInCategories, \cref{types-of-morphisms-in-categories:properties-of-monomorphisms-duality} of \cref{types-of-morphisms-in-categories:properties-of-monomorphisms}}{\cref{properties-of-monomorphisms-duality} of \cref{properties-of-monomorphisms}}.
                \item Since $R^{-1}$ is a monomorphism and $\Rel(-,\pt)$ maps $R$ to $R^{-1}$, it follows that $R$ is an epimorphism.
            \end{itemize}
    \end{itemize}

    We also claim that \cref{characterisations-of-epimorphisms-in-rel-2,characterisations-of-epimorphisms-in-rel-4} are equivalent, following \cite{MO455260}:
    \begin{itemize}
        \item\SloganFont{\cref{characterisations-of-epimorphisms-in-rel-2}$\implies$\cref{characterisations-of-epimorphisms-in-rel-4}: }Since $B\setminus\{b\}\subset B$ and $R^{-1}$ is injective, we have $R^{-1}(B\setminus\{b\})\subsetneq R^{-1}(B)$. So taking some $a\in R^{-1}(B)\setminus R^{-1}(B\setminus\{b\})$ we get an element of $A$ such that $R(a)=\{b\}$.
        \item\SloganFont{\cref{characterisations-of-epimorphisms-in-rel-4}$\implies$\cref{characterisations-of-epimorphisms-in-rel-2}: }Let $U,V\subset B$ with $U\neq V$. Without loss of generality, we can assume $U\setminus V\neq\emptyset$; otherwise just swap $U$ and $V$. Let then $b\in U\setminus V$. By assumption, there exists an $a\in A$ with $R(a)=\{b\}$. Then $a\in R^{-1}(U)$ but $a\nin R^{-1}(V)$, and thus $R^{-1}(U)\neq R^{-1}(V)$, showing $R^{-1}$ to be injective.
    \end{itemize}

    Finally, we prove the second part of the statement. So assume $R$ is a total epimorphism in $\sfRel$ and consider the diagram
    \[
        \begin{tikzcd}[row sep={4.0*\the\DL,between origins}, column sep={4.0*\the\DL,between origins}, background color=backgroundColor, ampersand replacement=\&]
            A
            \arrow[r, "R",mid vert]
            \&
            B
            \arrow[r, "S", shift left=0.8, mid vert]
            \arrow[r, "T"', shift right=0.8, mid vert]
            \&
            {\{0,1\}\mrp{,}}
        \end{tikzcd}
    \]%
    where $b\sim_{S}0$ for each $b\in B$ and where we have
    \[
        b%
        \sim_{T}%
        \begin{cases}
            0 &\text{if $b\in\Im(R)$,}\\
            1 &\text{otherwise}
        \end{cases}
    \]%
    for each $b\in B$. Since $R$ is total, we have $a\sim_{S\procirc R}0$ and $a\sim_{T\procirc R}0$ for all $a\in A$, and no element of $A$ is related to $1$ by $S\procirc R$ or $T\procirc R$. Thus $S\procirc R=T\procirc R$, and since $R$ is an epimorphism, we have $S=T$. But by the definition of $T$, this implies $\Im(R)=B$.
\end{Proof}
\subsection{2-Categorical Epimorphisms}\label{subsection-2-categorical-epimorphisms-in-rel}
In this section we characterise (for now, some of) the 2-categorical epimorphisms in $\sfbfRel$, following \ChapterRef{\ChapterTypesOfMorphismsInBicategories, \cref{types-of-morphisms-in-bicategories:section-epimorphisms-in-bicategories}}{\cref{section-epimorphisms-in-bicategories}}.
\begin{proposition}{2-Categorical Epimorphisms in $\sfbfRel$}{2-categorical-epimorphisms-in-rel}%
    Let $R\colon A\rightproarrow B$ be a relation.
    \begin{enumerate}
        \item\label{2-categorical-epimorphisms-in-rel-corepresentably-faithful-morphisms-in-rel}\SloganFont{Corepresentably Faithful Morphisms in $\sfbfRel$. }Every morphism of $\sfbfRel$ is a corepresentably faithful morphism.
        \item\label{2-categorical-epimorphisms-in-rel-corepresentably-full-morphisms-in-rel}\SloganFont{Corepresentably Full Morphisms in $\sfbfRel$. }The following conditions are equivalent:
            \begin{enumerate}
                \item\label{2-categorical-epimorphisms-in-rel-corepresentably-full-morphisms-in-rel-1}The morphism $R\colon A\rightproarrow B$ is a corepresentably full morphism.
                \item\label{2-categorical-epimorphisms-in-rel-corepresentably-full-morphisms-in-rel-2}For each pair of relations $S,T\colon X\rightproarrows A$, the following condition is satisfied:
                    \begin{itemize}
                        \itemstar If $S\procirc R\subset T\procirc R$, then $S\subset T$.
                    \end{itemize}
                \item\label{2-categorical-epimorphisms-in-rel-corepresentably-full-morphisms-in-rel-3}The functor
                    \[
                        R^{-1}%
                        \colon%
                        (\mathcal{P}(B),\subset)%
                        \to%
                        (\mathcal{P}(A),\subset)%
                    \]%
                    is full.
                \item\label{2-categorical-epimorphisms-in-rel-corepresentably-full-morphisms-in-rel-4}For each $U,V\in\mathcal{P}(B)$, if $R^{-1}(U)\subset R^{-1}(V)$, then $U\subset V$.
                \item\label{2-categorical-epimorphisms-in-rel-corepresentably-full-morphisms-in-rel-5}The functor
                    \[
                        R_{-1}%
                        \colon%
                        (\mathcal{P}(B),\subset)%
                        \to%
                        (\mathcal{P}(A),\subset)%
                    \]%
                    is full.
                \item\label{2-categorical-epimorphisms-in-rel-corepresentably-full-morphisms-in-rel-6}For each $U,V\in\mathcal{P}(B)$, if $R_{-1}(U)\subset R_{-1}(V)$, then $U\subset V$.
            \end{enumerate}
        \item\label{2-categorical-epimorphisms-in-rel-corepresentably-fully-faithful-morphisms-in-rel}\SloganFont{Corepresentably Fully Faithful Morphisms in $\sfbfRel$. }Every corepresentably full morphism of $\sfbfRel$ is a corepresentably fully faithful morphism.
    \end{enumerate}
\end{proposition}
\begin{Proof}{Proof of \cref{2-categorical-epimorphisms-in-rel}}%
    \FirstProofBox{\cref{2-categorical-epimorphisms-in-rel-corepresentably-faithful-morphisms-in-rel}: Corepresentably Faithful Morphisms in $\sfbfRel$}%
    The relation $R$ is a corepresentably faithful morphism in $\sfbfRel$ \textiff, for each $X\in\Obj(\sfbfRel)$, the functor
    \[
        R^{*}%
        \colon%
        \eRel(B,X)%
        \to%
        \eRel(A,X)%
    \]%
    is faithful, i.e.\ \textiff the morphism
    \[
        R^{*}_{S,T}%
        \colon%
        \Hom_{\eRel(B,X)}(S,T)%
        \to%
        \Hom_{\eRel(A,X)}(S\procirc R,T\procirc R)%
    \]%
    is injective for each $S,T\in\Obj(\eRel(B,X))$. However, $\Hom_{\eRel(B,X)}(S,T)$ is either empty or a singleton, in either case of which the map $R^{*}_{S,T}$ is necessarily injective.

    \ProofBox{\cref{2-categorical-epimorphisms-in-rel-corepresentably-full-morphisms-in-rel}: Corepresentably Full Morphisms in $\sfbfRel$}%
    We claim \cref{2-categorical-epimorphisms-in-rel-corepresentably-full-morphisms-in-rel-1,2-categorical-epimorphisms-in-rel-corepresentably-full-morphisms-in-rel-2,2-categorical-epimorphisms-in-rel-corepresentably-full-morphisms-in-rel-3,2-categorical-epimorphisms-in-rel-corepresentably-full-morphisms-in-rel-4,2-categorical-epimorphisms-in-rel-corepresentably-full-morphisms-in-rel-5,2-categorical-epimorphisms-in-rel-corepresentably-full-morphisms-in-rel-6} are indeed equivalent:
    \begin{itemize}
        \item\SloganFont{\cref{2-categorical-epimorphisms-in-rel-corepresentably-full-morphisms-in-rel-1}$\iff$\cref{2-categorical-epimorphisms-in-rel-corepresentably-full-morphisms-in-rel-2}: }This is simply a matter of unwinding definitions: The relation $R$ is a corepresentably full morphism in $\sfbfRel$ \textiff, for each $X\in\Obj(\sfbfRel)$, the functor
            \[
                R^{*}%
                \colon%
                \eRel(B,X)%
                \to%
                \eRel(A,X)%
            \]%
            is full, i.e.\ \textiff the morphism
            \[
                R^{*}_{S,T}%
                \colon%
                \Hom_{\eRel(B,X)}(S,T)%
                \to%
                \Hom_{\eRel(A,X)}(S\procirc R,T\procirc R)%
            \]%
            is surjective for each $S,T\in\Obj(\eRel(B,X))$, i.e.\ \textiff, whenever $S\procirc R\subset T\procirc R$, we also have $S\subset T$.
        \item\SloganFont{\cref{2-categorical-epimorphisms-in-rel-corepresentably-full-morphisms-in-rel-3}$\iff$\cref{2-categorical-epimorphisms-in-rel-corepresentably-full-morphisms-in-rel-4}: }This is also simply a matter of unwinding definitions: The functor
            \[
                R^{-1}%
                \colon%
                (\mathcal{P}(B),\subset)%
                \to%
                (\mathcal{P}(A),\subset)%
            \]%
            is full \textiff, for each $U,V\in\mathcal{P}(A)$, the morphism
            \[
                R^{-1}_{U,V}%
                \colon%
                \Hom_{\mathcal{P}(B)}(U,V)%
                \to%
                \Hom_{\mathcal{P}(A)}(R^{-1}(U),R^{-1}(V))%
            \]%
            is surjective, i.e.\ \textiff whenever $R^{-1}(U)\subset R^{-1}(V)$, we also necessarily have $U\subset V$.
        \item\SloganFont{\cref{2-categorical-epimorphisms-in-rel-corepresentably-full-morphisms-in-rel-5}$\iff$\cref{2-categorical-epimorphisms-in-rel-corepresentably-full-morphisms-in-rel-6}: }This is once again simply a matter of unwinding definitions, and proceeds exactly in the same way as in the proof of the equivalence between \cref{2-categorical-epimorphisms-in-rel-corepresentably-full-morphisms-in-rel-3,2-categorical-epimorphisms-in-rel-corepresentably-full-morphisms-in-rel-4} given above.
        \item\SloganFont{\cref{2-categorical-epimorphisms-in-rel-corepresentably-full-morphisms-in-rel-4}$\implies$\cref{2-categorical-epimorphisms-in-rel-corepresentably-full-morphisms-in-rel-6}: }Suppose that the following condition is true:
            \begin{itemize}
                \itemstar For each $U,V\in\mathcal{P}(B)$, if $R^{-1}(U)\subset R^{-1}(V)$, then $U\subset V$.
            \end{itemize}
            We need to show that the condition
            \begin{itemize}
                \itemstar For each $U,V\in\mathcal{P}(B)$, if $R_{-1}(U)\subset R_{-1}(V)$, then $U\subset V$.
            \end{itemize}
            is also true. We proceed step by step:
            \begin{itemize}
                \item Suppose we have $U,V\in\mathcal{P}(B)$ with $R_{-1}(U)\subset R_{-1}(V)$.
                \item By \ChapterRef{\ChapterConstructionsWithRelations, \cref{constructions-with-relations:properties-of-strong-inverse-image-functions-associated-to-relations-interaction-with-weak-inverse-images-1} of \cref{constructions-with-relations:properties-of-strong-inverse-image-functions-associated-to-relations}}{\cref{properties-of-strong-inverse-image-functions-associated-to-relations-interaction-with-weak-inverse-images-1} of \cref{properties-of-strong-inverse-image-functions-associated-to-relations}}, we have
                    \begin{align*}
                        R_{-1}(U) &= B\setminus R^{-1}(A\setminus U),\\
                        R_{-1}(V) &= B\setminus R^{-1}(A\setminus V).
                    \end{align*}
                \item By \ChapterRef{\ChapterConstructionsWithSets, \cref{constructions-with-sets:properties-of-differences-functoriality} of \cref{constructions-with-sets:properties-of-differences}}{\cref{properties-of-differences-functoriality} of \cref{properties-of-differences}} we have $R^{-1}(A\setminus V)\subset R^{-1}(A\setminus U)$.
                \item By assumption, we then have $A\setminus V\subset A\setminus U$.
                \item By \ChapterRef{\ChapterConstructionsWithSets, \cref{constructions-with-sets:properties-of-differences-functoriality} of \cref{constructions-with-sets:properties-of-differences}}{\cref{properties-of-differences-functoriality} of \cref{properties-of-differences}} again, we have $U\subset V$.
            \end{itemize}
        \item\SloganFont{\cref{2-categorical-epimorphisms-in-rel-corepresentably-full-morphisms-in-rel-6}$\implies$\cref{2-categorical-epimorphisms-in-rel-corepresentably-full-morphisms-in-rel-4}: }Suppose that the following condition is true:
            \begin{itemize}
                \itemstar For each $U,V\in\mathcal{P}(B)$, if $R_{-1}(U)\subset R_{-1}(V)$, then $U\subset V$.
            \end{itemize}
            We need to show that the condition
            \begin{itemize}
                \itemstar For each $U,V\in\mathcal{P}(B)$, if $R^{-1}(U)\subset R^{-1}(V)$, then $U\subset V$.
            \end{itemize}
            is also true. We proceed step by step:
            \begin{itemize}
                \item Suppose we have $U,V\in\mathcal{P}(B)$ with $R^{-1}(U)\subset R^{-1}(V)$.
                \item By \ChapterRef{\ChapterConstructionsWithRelations, \cref{constructions-with-relations:properties-of-weak-inverse-image-functions-associated-to-relations-interaction-with-strong-inverse-images-1} of \cref{constructions-with-relations:properties-of-weak-inverse-image-functions-associated-to-relations}}{\cref{properties-of-weak-inverse-image-functions-associated-to-relations-interaction-with-strong-inverse-images-1} of \cref{properties-of-weak-inverse-image-functions-associated-to-relations}}, we have
                    \begin{align*}
                        R^{-1}(U) &= B\setminus R_{-1}(A\setminus U),\\
                        R^{-1}(V) &= B\setminus R_{-1}(A\setminus V).
                    \end{align*}
                \item By \ChapterRef{\ChapterConstructionsWithSets, \cref{constructions-with-sets:properties-of-differences-functoriality} of \cref{constructions-with-sets:properties-of-differences}}{\cref{properties-of-differences-functoriality} of \cref{properties-of-differences}} we have $R_{-1}(A\setminus V)\subset R_{-1}(A\setminus U)$.
                \item By assumption, we then have $A\setminus V\subset A\setminus U$.
                \item By \ChapterRef{\ChapterConstructionsWithSets, \cref{constructions-with-sets:properties-of-differences-functoriality} of \cref{constructions-with-sets:properties-of-differences}}{\cref{properties-of-differences-functoriality} of \cref{properties-of-differences}} again, we have $U\subset V$.
            \end{itemize}
        \item\SloganFont{\cref{2-categorical-epimorphisms-in-rel-corepresentably-full-morphisms-in-rel-2}$\implies$\cref{2-categorical-epimorphisms-in-rel-corepresentably-full-morphisms-in-rel-4}: }Consider the diagram
            \[
                \begin{tikzcd}[row sep={4.0*\the\DL,between origins}, column sep={4.0*\the\DL,between origins}, background color=backgroundColor, ampersand replacement=\&]
                    A
                    \arrow[r, "R",mid vert]
                    \&
                    B
                    \arrow[r, "S", shift left=0.8, mid vert]
                    \arrow[r, "T"', shift right=0.8, mid vert]
                    \&
                    X\mrp{,}
                \end{tikzcd}
            \]%
            and suppose that $S\procirc R\subset T\procirc R$. Note that, by assumption, given a diagram of the form
            \[
                \begin{tikzcd}[row sep={4.0*\the\DL,between origins}, column sep={4.0*\the\DL,between origins}, background color=backgroundColor, ampersand replacement=\&]
                    A
                    \arrow[r, "R",mid vert]
                    \&
                    B
                    \arrow[r, "U", shift left=0.8, mid vert]
                    \arrow[r, "V"', shift right=0.8, mid vert]
                    \&
                    \pt\mrp{,}
                \end{tikzcd}
            \]%
            if $R^{-1}(U)=R\procirc U\subset R\procirc V=R^{-1}(V)$, then $U\subset V$. In particular, for each $x\in X$, we may consider the diagram
            \[
                \begin{tikzcd}[row sep={4.0*\the\DL,between origins}, column sep={4.0*\the\DL,between origins}, background color=backgroundColor, ampersand replacement=\&]
                    A
                    \arrow[r, "R",mid vert]
                    \&
                    B
                    \arrow[r, "S",  shift left=0.8,  mid vert]
                    \arrow[r, "T"', shift right=0.8, mid vert]
                    \&
                    X
                    \arrow[r, "{[x]}", shift left=0.8, mid vert]
                    \&
                    \pt\mrp{,}
                \end{tikzcd}
            \]%
            for which we have $[x]\procirc S\procirc R\subset[x]\procirc T\procirc R$, implying that we have
            \[
                S^{-1}(x)%
                =%
                [x]\procirc S%
                \subset%
                [x]\procirc T%
                =%
                T^{-1}(x)
            \]%
            for each $x\in X$, implying $S\subset T$.
        \item\SloganFont{\cref{2-categorical-monomorphisms-in-rel-representably-full-morphisms-in-rel-4}$\implies$\cref{2-categorical-monomorphisms-in-rel-representably-full-morphisms-in-rel-2}: }Let $U,V\in\mathcal{P}(B)$ and consider the diagram
            \[
                \begin{tikzcd}[row sep={4.0*\the\DL,between origins}, column sep={4.0*\the\DL,between origins}, background color=backgroundColor, ampersand replacement=\&]
                    A
                    \arrow[r, "R",mid vert]
                    \&
                    B
                    \arrow[r, "U", shift left=0.8, mid vert]
                    \arrow[r, "V"', shift right=0.8, mid vert]
                    \&
                    \pt\mrp{.}
                \end{tikzcd}
            \]%
            By \cref{unwinding-the-direct-image-function-associated-to-a-relation}, we have
            \begin{align*}
                R^{-1}(U) &= U\procirc R,\\
                R^{-1}(V) &= V\procirc R.
            \end{align*}
            Now, if $R^{-1}(U)\subset R^{-1}(V)$, i.e.\ $U\procirc R\subset V\procirc R$, then $U\subset V$ by assumption.
    \end{itemize}

    \ProofBox{\cref{2-categorical-epimorphisms-in-rel-corepresentably-fully-faithful-morphisms-in-rel}: Corepresentably Fully Faithful Morphisms in $\sfbfRel$}%
    This follows from \cref{2-categorical-epimorphisms-in-rel-corepresentably-faithful-morphisms-in-rel,2-categorical-epimorphisms-in-rel-corepresentably-full-morphisms-in-rel}.
\end{Proof}
\begin{question}{Better Characterisations of Corepresentably Full Morphisms in $\sfbfRel$}{better-characterisations-of-corepresentably-full-morphisms-in-rel}%
    \cref{2-categorical-epimorphisms-in-rel-corepresentably-full-morphisms-in-rel} of \cref{2-categorical-epimorphisms-in-rel} gives a characterisation of the corepresentably full morphisms in $\sfbfRel$.

    Are there other nice characterisations of these?

    This question also appears as \cite{MO467527}.
\end{question}
\subsection{Co/Limits}\label{subsection-co-limits-in-rel}
\begin{proposition}{Co/Limits in $\sfRel$}{co-limits-in-rel}%
    This will be properly written later on.
    %The category $\sfRel$ admits some co/limits, but not all, as does its 2-categorical counterpart:
    %\begin{enumerate}
    %    \item\SloganFont{Zero Objects. }The category $\Rel$ has a zero object, the empty set $\emptyset$.
    %    \item\SloganFont{Co/Products. }The category $\Rel$ has co/products, both given by disjoint union of sets.
    %    \item\SloganFont{Lack of Co/Equalisers. }The category $\Rel$ does not have co/equalisers.
    %    \item\SloganFont{Limits of Graphs of Functions. }The category $\Rel$ has limits whose arrows are all graphs of functions.
    %    \item\SloganFont{Colimits of Graphs of Functions. }The category $\Rel$ has colimits whose arrows are all graphs of functions, and these agree with the corresponding limits in $\Sets$.
    %\end{enumerate}
\end{proposition}
\begin{Proof}{Proof of \cref{co-limits-in-rel}}%
    Omitted.
\end{Proof}
\subsection{Internal Kan Extensions and Lifts}\label{subsection-internal-kan-extensions-and-lifts-in-rel}
\begin{remark}{Kan Extensions and Kan Lifts in $\sfbfRel$}{kan-extensions-and-kan-lifts-in-rel}%
    The 2-category $\sfbfRel$ admits all right Kan extensions and right Kan lifts, though not all left Kan extensions and neither does it admit all left Kan lifts. See \ChapterRef{\ChapterConstructionsWithRelations, \cref{constructions-with-relations:section-kan-extensions-and-kan-lifts-in-the-2-category-of-relations}}{\cref{section-kan-extensions-and-kan-lifts-in-the-2-category-of-relations}} for a detailed discussion of this.
\end{remark}
\subsection{Internal Left Kan Extensions}\label{subsection-internal-left-kan-extensions-in-rel}
\begin{proposition}{Internal Left Kan Extensions in $\sfbfRel$}{internal-left-kan-extensions-in-rel}%
    Let $R\colon A\rightproarrow B$ be a relation.
    \begin{enumerate}
        \item\label{internal-left-kan-extensions-in-rel-non-existence-of-all-internal-left-kan-extensions-in-rel}\SloganFont{Non-Existence of All Internal Left Kan Extensions in $\sfbfRel$. }Not all relations in $\sfbfRel$ admit left Kan extensions.
        \item\label{internal-left-kan-extensions-in-rel-characterisation-of-relations-admitting-left-kan-extensions-along-them}\SloganFont{Characterisation of Relations Admitting Left Kan Extensions Along Them. }The following conditions are equivalent:
            \begin{enumerate}
                \item\label{internal-left-kan-extensions-in-rel-characterisation-of-relations-admitting-left-kan-extensions-along-them-1}The left Kan extension
                    \[
                        \Lan_{R}%
                        \colon%
                        \eRel(A,X)%
                        \to%
                        \eRel(B,X)%
                    \]%
                    along $R$ exists.
                \item\label{internal-left-kan-extensions-in-rel-characterisation-of-relations-admitting-left-kan-extensions-along-them-2}The relation $R$ admits a left adjoint in $\sfbfRel$.
                \item\label{internal-left-kan-extensions-in-rel-characterisation-of-relations-admitting-left-kan-extensions-along-them-3}The relation $R$ is of the form $\Gr(f)$ (as in \cref{the-graph-of-a-function}) for some function $f$.
            \end{enumerate}
        %\item\label{internal-left-kan-extensions-in-rel-}\SloganFont{. }
    \end{enumerate}
\end{proposition}
\begin{Proof}{Proof of \cref{internal-left-kan-extensions-in-rel}}%
    \FirstProofBox{\cref{internal-left-kan-extensions-in-rel-non-existence-of-all-internal-left-kan-extensions-in-rel}: Non-Existence of All Internal Left Kan Extensions in $\sfbfRel$}%
    By \cref{internal-left-kan-extensions-in-rel-characterisation-of-relations-admitting-left-kan-extensions-along-them}, it suffices to take a relation that doesn't have a left adjoint.

    \ProofBox{\cref{internal-left-kan-extensions-in-rel-characterisation-of-relations-admitting-left-kan-extensions-along-them}: Characterisation of Relations Admitting Left Kan Extensions Along Them}%
    This proof is mostly due to Tim Campion, via \cite{MO460693}.
    \begin{itemize}
        \item We may view precomposition
            \[
                -\procirc R%
                \colon%
                \Rel(B,C)%
                \to%
                \Rel(A,C)%
            \]%
            with $R\colon A\rightproarrow B$ as a cocontinuous functor from $\mathcal{P}(B\times C)$ to $\mathcal{P}(A\times C)$ (via \cref{relations-5} of \cref{relations}).
        \item By the adjoint functor theorem (\cref{TODO}), this map has a left adjoint \textiff it preserves limits.
        \item If $C=\emptyset$, this holds trivially.
        \item Otherwise, $C$ admits $\pt$ as a retract, and we reduce to the case $C=\pt$ via \cref{TODO}.
        \item For the case $C=\pt$, a relation $T\colon B\rightproarrow\pt$ is the same as a subset of $B$, and $-\procirc R$ becomes the weak inverse image functor $R^{-1}$ of \cref{subsection-weak-inverse-images}.
        \item Now, again by the adjoint functor theorem, $R^{-1}$ preserves limits exactly when it has a left adjoint.
        \item Finally $R^{-1}$ has a left adjoint precisely when $R=\Gr(f)$ for $f$ a function (\cref{properties-of-weak-inverse-image-functions-associated-to-relations-interaction-with-strong-inverse-images-2} of \cref{properties-of-weak-inverse-image-functions-associated-to-relations}).
    \end{itemize}
    This finishes the proof.
\end{Proof}
\begin{example}{Left Kan Extensions Along Functions}{left-kan-extensions-along-functions}%
    Given a function $f\colon A\to B$, the left Kan extension
    \[
        \Lan_{f}%
        \colon%
        \eRel(A,X)%
        \to%
        \eRel(B,X)%
    \]%
    along $f$ exists by \cref{internal-left-kan-extensions-in-rel-characterisation-of-relations-admitting-left-kan-extensions-along-them} of \cref{internal-left-kan-extensions-in-rel}. Explicitly, given a relation $R\colon A\rightproarrow X$, the left Kan extension
    \begin{webcompile}
        \Lan_{f}(R)%
        \colon%
        B%
        \rightproarrow%
        X,%
        \quad%
        \begin{tikzcd}[row sep={5.0*\the\DL,between origins}, column sep={5.0*\the\DL,between origins}, background color=backgroundColor, ampersand replacement=\&]
            \&%
            B%
            \arrow[d, "\Lan_{f}(R)", densely dashed for mid vert, mid vert]%
            \\%
            A%
            \arrow[ru, "f",mid vert]%
            \arrow[r, "R"',""'{name=F,pos=0.425},mid vert]%
            \&%
            X%
            % 2-Arrows
            \arrow[from=1-2,to=F,shorten=0.75*\the\DL,Leftarrow,shift left=0.15*\the\DL]%
        \end{tikzcd}
    \end{webcompile}
    may be described as follows:
    \begin{enumerate}
        \item\label{left-kan-extensions-along-functions-1}We declare $b\sim_{\Lan_{f}(R)}x$ \textiff there exists some $a\in R$ such that $b=f(a)$ and $a\sim_{R}x$.
        \item\label{left-kan-extensions-along-functions-2}We have%
            %--- Begin Footnote ---%
            \footnote{%
                Cf.\ \cref{internal-right-kan-extensions-in-rel-3} of \cref{internal-right-kan-extensions-in-rel}.
            }%
            %---  End Footnote  ---%
            \[
                [\Lan_{f}(R)](b)%
                =%
                \bigcup_{a\in f^{-1}(b)}R(a)%
            \]%
            for each $b\in B$.
    \end{enumerate}
\end{example}
\begin{remark}{Illustrating the Failure of Internal Left Kan Extensions in $\sfbfRel$ to Exist}{illustrating-the-failure-of-internal-left-kan-extensions-in-rel-to-exist}%
    Following \cref{left-kan-extensions-along-functions}, given a relation $R\colon A\rightproarrow B$ and a relation $F\colon A\rightproarrow X$, we could perhaps try to define an \say{honorary} left Kan extension
    \[
        \Lan'_{R}(F)%
        \colon%
        B%
        \rightproarrow%
        X%
    \]%
    by
    \[
        [\Lan'_{F}(F)](b)%
        \defeq%
        \bigcup_{a\in R^{-1}(b)}F(a)%
    \]%
    for each $b\in B$.

    \indent The failure of $\Lan'_{R}(F)$ to be a Kan extension can then be seen as follows. Let $G\colon B\rightproarrow X$ be a relation. If $\Lan'_{R}(F)$ were a left Kan extension, then the following conditions \demph{would be} equivalent:
    \begin{enumerate}
        \item\label{illustrating-the-failure-of-internal-left-kan-extensions-in-rel-to-exist-1}For each $b\in B$, we have $\bigcup_{a\in R^{-1}(b)}F(a)\subset G(b)$.
        \item\label{illustrating-the-failure-of-internal-left-kan-extensions-in-rel-to-exist-2}For each $a\in A$, we have $F(a)\subset\bigcup_{b\in R(a)}G(b)$.
    \end{enumerate}
    The issue is two-fold:
    \begin{itemize}
        \item\SloganFont{Totality. }If $R$ isn't total, then the implication \cref{illustrating-the-failure-of-internal-left-kan-extensions-in-rel-to-exist-1} $\Rightarrow$ \cref{illustrating-the-failure-of-internal-left-kan-extensions-in-rel-to-exist-2} fails.
        \item\SloganFont{Functionality. }If $R$ isn't functional, then the implication \cref{illustrating-the-failure-of-internal-left-kan-extensions-in-rel-to-exist-2} $\Rightarrow$ \cref{illustrating-the-failure-of-internal-left-kan-extensions-in-rel-to-exist-1} fails.
    \end{itemize}
\end{remark}
\subsection{Internal Left Kan Lifts}\label{subsection-internal-left-kan-lifts-in-rel}
\begin{proposition}{Internal Left Kan Lifts in $\sfbfRel$}{internal-left-kan-lifts-in-rel}%
    Let $R\colon A\rightproarrow B$ be a relation.
    \begin{enumerate}
        \item\label{internal-left-kan-lifts-in-rel-non-existence-of-all-internal-left-kan-lifts-in-rel}\SloganFont{Non-Existence of All Internal Left Kan Lifts in $\sfbfRel$. }Not all relations in $\sfbfRel$ admit left Kan lifts.
        \item\label{internal-left-kan-lifts-in-rel-characterisation-of-relations-admitting-left-kan-lifts-along-them}\SloganFont{Characterisation of Relations Admitting Left Kan Lifts Along Them. }The following conditions are equivalent:
            \begin{enumerate}
                \item\label{internal-left-kan-lifts-in-rel-characterisation-of-relations-admitting-left-kan-lifts-along-them-1}The left Kan lift
                    \[
                        \Lift_{R}%
                        \colon%
                        \eRel(X,B)%
                        \to%
                        \eRel(X,A)%
                    \]%
                    along $R$ exists.
                \item\label{internal-left-kan-lifts-in-rel-characterisation-of-relations-admitting-left-kan-lifts-along-them-2}The relation $R$ admits a right adjoint in $\sfbfRel$.
                \item\label{internal-left-kan-lifts-in-rel-characterisation-of-relations-admitting-left-kan-lifts-along-them-3}The relation $R$ is of the form $f^{-1}$ (as in \cref{the-inverse-of-a-function}) for some function $f$.
            \end{enumerate}
        %\item\label{internal-left-kan-lifts-in-rel-}\SloganFont{. }
    \end{enumerate}
\end{proposition}
\begin{Proof}{Proof of \cref{internal-left-kan-lifts-in-rel}}%
    \FirstProofBox{\cref{internal-left-kan-lifts-in-rel-non-existence-of-all-internal-left-kan-lifts-in-rel}: Non-Existence of All Internal Left Kan Lifts in $\sfbfRel$}%
    By \cref{internal-left-kan-lifts-in-rel-characterisation-of-relations-admitting-left-kan-lifts-along-them}, it suffices to take a relation that doesn't have a right adjoint.

    \ProofBox{\cref{internal-left-kan-lifts-in-rel-characterisation-of-relations-admitting-left-kan-lifts-along-them}: Characterisation of Relations Admitting Left Kan Lifts Along Them}%
    This proof is dual to that of \cref{internal-left-kan-extensions-in-rel-characterisation-of-relations-admitting-left-kan-extensions-along-them} of \cref{internal-left-kan-extensions-in-rel}, and is therefore omitted.
\end{Proof}
\begin{example}{Left Kan Lifts Along Functions}{left-kan-lifts-along-functions}%
    Given a function $f\colon A\to B$, the left Kan lift
    \[
        \Lift_{f^{\dagger}}%
        \colon%
        \eRel(X,A)%
        \to%
        \eRel(X,B)%
    \]%
    along $f^{\dagger}$ exists by \cref{internal-left-kan-lifts-in-rel-characterisation-of-relations-admitting-left-kan-lifts-along-them} of \cref{internal-left-kan-lifts-in-rel}. Explicitly, given a relation $R\colon X\rightproarrow A$, the left Kan lift
    \begin{webcompile}
        \Lift_{f^{\dagger}}(R)%
        \colon%
        X%
        \rightproarrow%
        B,%
        \quad%
        \begin{tikzcd}[row sep={5.0*\the\DL,between origins}, column sep={5.0*\the\DL,between origins}, background color=backgroundColor, ampersand replacement=\&]
            \&%
            B%
            \arrow[d, "f^{\dagger}", mid vert]%
            \\%
            X%
            \arrow[ru, "\Lift_{f^{\dagger}}(R)",densely dashed for mid vert, mid vert]%
            \arrow[r, "R"',""'{name=F,pos=0.425},mid vert]%
            \&%
            A\mrp{.}%
            % 2-Arrows
            \arrow[from=1-2,to=F,shorten=0.75*\the\DL,Leftarrow,shift left=0.15*\the\DL]%
        \end{tikzcd}
    \end{webcompile}
    is given by
    \begin{align*}
        [\Lift_{f}(R)](x) &= [\Gr(f)\procirc R](a)\\
                          &= \bigcup_{a\in R(x)}f(a)%
    \end{align*}
    for each $x\in X$.
\end{example}
\subsection{Internal Right Kan Extensions}\label{subsection-internal-right-kan-extensions-in-rel}
Let $A$, $B$, and $X$ be sets and let $R\colon A\rightproarrow B$ and $F\colon A\rightproarrow X$ be relations.
\begin{motivation}{Setting for Internal Right Kan Extensions in $\sfbfRel$}{setting-for-internal-right-kan-extensions-in-rel}%
    We want to understand internal right Kan extensions in $\sfbfRel$, which look like this:
    \[
        \begin{tikzcd}[row sep={5.0*\the\DL,between origins}, column sep={5.0*\the\DL,between origins}, background color=backgroundColor, ampersand replacement=\&]
            \&%
            B%
            \arrow[d, "\Ran_{R}(F)", densely dashed for mid vert, mid vert]%
            \\%
            A%
            \arrow[ru, "R",mid vert]%
            \arrow[r, "F"',""'{name=F,pos=0.425},mid vert]%
            \&%
            X\mrp{.}%
            % 2-Arrows
            \arrow[from=1-2,to=F,shorten=0.75*\the\DL,Rightarrow,shift left=0.15*\the\DL]%
        \end{tikzcd}
    \]%
    Note in particular here that $F\colon A\rightproarrow X$ is a relation from $A$ to $X$. These will form a functor
    \[
        \Ran_{R}%
        \colon%
        \eRel(A,X)%
        \to%
        \eRel(B,X)%
    \]%
    that is right adjoint to the precomposition by $R$ functor
    \[
        R^{*}%
        \colon%
        \eRel(B,X)%
        \to%
        \eRel(A,X).%
    \]%
\end{motivation}
\begin{proposition}{Internal Right Kan Extensions in $\sfbfRel$}{internal-right-kan-extensions-in-rel}%
    The internal right Kan extension of $F$ along $R$ is the relation \index[notation]{RanRF@$\Ran_{R}(F)$}$\Ran_{R}(F)$ described as follows:
    \begin{enumerate}
        \item\label{internal-right-kan-extensions-in-rel-1}Viewing relations from $B$ to $X$ as subsets of $B\times X$, we have
            \[
                \Ran_{R}(F)%
                =%
                \{%
                    (b,x)\in B\times X%
                    \ \middle|\ %
                    \begin{aligned}
                        &\text{for each $a\in A$, if $a\sim_{R}b$,}\\
                        &\text{then we have $a\sim_{F}x$}%
                    \end{aligned}
                \}.%
            \]%
        \item\label{internal-right-kan-extensions-in-rel-2}Viewing relations as functions $B\times X\to\TV$, we have
            \begin{align*}
                (\Ran_{R}(F))^{-_{1}}_{-_{2}} &=      \int_{a\in A}\eHom_{\TTV}(R^{-_{2}}_{a},F^{-_{1}}_{a})\\
                                              &=      \bigwedge_{a\in A}\eHom_{\TTV}(R^{-_{2}}_{a},F^{-_{1}}_{a}),%
            \end{align*}
            where the meet $\bigwedge$ is taken in the poset $(\TV,\preceq)$ of \ChapterRef{\ChapterSets, \cref{sets:the-poset-of-truth-values}}{\cref{the-poset-of-truth-values}}.
        \item\label{internal-right-kan-extensions-in-rel-3}Viewing relations as functions $B\to\mathcal{P}(X)$, we have
            \begin{webcompile}
                \Ran_{R}(F)%
                =%
                \Ran_{\chi'_{A}}(F)\circ R^{-1},%
                \quad%
                \begin{tikzcd}[row sep={5.0*\the\DL,between origins}, column sep={5.0*\the\DL,between origins}, background color=backgroundColor, ampersand replacement=\&]
                    \&%
                    A%
                    \arrow[r, "F",""{name=F,pos=0.7}]%
                    \arrow[d, "\chi_{A}"', hook]%
                    \&%
                    \mathcal{P}(X)\mrp{,}%
                    \\%
                    B%
                    \arrow[r, "R^{-1}"']%
                    \&%
                    \mathcal{P}(A)^{\op}%
                    \arrow[ru, "\Ran_{\chi_{A}}(F)"', bend right=10]%
                    \&
                    % 2-Arrows
                    \arrow[from=F,to=2-2,shorten=0.75*\the\DL,Rightarrow]%
                \end{tikzcd}%
            \end{webcompile}
            where $\Ran_{\chi'_{B}}(F)$ is computed by the formula
            \begin{align*}
                [\Ran_{\chi'_{A}}(F)](V) &\cong \int_{a\in A}\chi_{\mathcal{P}(A)^{\op}}(V,\chi_{a})\pitchfork F(a)\\
                                         &\cong \int_{a\in A}\chi_{\mathcal{P}(A)}(\chi_{a},V)\pitchfork F(a)\\
                                         &\cong \int_{a\in A}\chi_{V}(a)\pitchfork F(a)\\
                                         &\cong \bigcap_{a\in A}\chi_{V}(a)\pitchfork F(a)\\
                                         &\cong \bigcap_{a\in V}F(a)
            \end{align*}
            for each $V\in\mathcal{P}(B)$, so we have%
            \[
                [\Ran_{R}(F)](b)%
                =%
                \bigcap_{a\in R^{-1}(b)}F(a)%
            \]%
            for each $b\in B$.
    \end{enumerate}
\end{proposition}
\begin{Proof}{Proof of \cref{internal-right-kan-extensions-in-rel}}%
    We have
    \begin{envsmallsize}
        \begin{align*}
            \Hom_{\eRel(A,X)}(F\procirc R,T) &\cong  \int_{a\in A}\int_{x\in X}\eHom_{\TTV}((F\procirc R)^{x}_{a},T^{x}_{a})\\
                                             &\cong  \int_{a\in A}\int_{x\in X}\eHom_{\TTV}((\int^{b\in B}F^{x}_{b}\times R^{b}_{a}),T^{x}_{a})\\
                                             &\cong  \int_{a\in A}\int_{x\in X}\int_{b\in B}\eHom_{\TTV}(F^{x}_{b}\times R^{b}_{a},T^{x}_{a})\\
                                             &\cong  \int_{a\in A}\int_{x\in X}\int_{b\in B}\eHom_{\TTV}(F^{x}_{b},\eHom_{\TTV}(R^{b}_{a},T^{x}_{a}))\\
                                             &\cong  \int_{b\in B}\int_{x\in X}\int_{a\in A}\eHom_{\TTV}(F^{x}_{b},\eHom_{\TTV}(R^{b}_{a},T^{x}_{a}))\\
                                             &\cong  \int_{b\in B}\int_{x\in X}\eHom_{\TTV}(F^{x}_{b},\int_{a\in A}\eHom_{\TTV}(R^{b}_{a},T^{x}_{a}))\\
                                             &\cong  \Hom_{\eRel(B,X)}(F,\int_{a\in A}\eHom_{\TTV}(R^{-_{2}}_{a},T^{-_{1}}_{a}))%
        \end{align*}
    \end{envsmallsize}
    naturally in each $F\in\eRel(B,X)$ and each $T\in\eRel(A,X)$, showing that
    \[
        \int_{a\in A}\eHom_{\TTV}(R^{-_{2}}_{a},T^{-_{1}}_{a})%
    \]%
    is right adjoint to the precomposition functor $-\procirc R$, being thus the right Kan extension along $R$. Here we have used the following results, respectively (i.e.\ for each $\cong$ sign):
    \begin{enumerate}
        \item\label{proof-of-existence-of-right-kan-extensions-in-rel-1}\ChapterRef{\ChapterRelations, \cref{relations:properties-of-relations-end-formula-for-the-set-of-inclusions-of-relations} of \cref{relations:properties-of-relations}}{\cref{properties-of-relations-end-formula-for-the-set-of-inclusions-of-relations} of \cref{properties-of-relations}}.
        \item\label{proof-of-existence-of-right-kan-extensions-in-rel-2}\cref{composition-of-relations}.
        \item\label{proof-of-existence-of-right-kan-extensions-in-rel-3}\ChapterRef{\ChapterEndsAndCoends, \cref{ends-and-coends:properties-of-co-ends-commutativity-with-homs} of \cref{ends-and-coends:properties-of-co-ends}}{\cref{properties-of-co-ends-commutativity-with-homs} of \cref{properties-of-co-ends}}.
        \item\label{proof-of-existence-of-right-kan-extensions-in-rel-4}\ChapterRef{\ChapterSets, \cref{sets:cartesian-closedness-of-the-poset-of-truth-values}}{\cref{cartesian-closedness-of-the-poset-of-truth-values}}.
        \item\label{proof-of-existence-of-right-kan-extensions-in-rel-5}\ChapterRef{\ChapterEndsAndCoends, \cref{ends-and-coends:properties-of-co-ends-the-fubini-rule} of \cref{ends-and-coends:properties-of-co-ends}}{\cref{properties-of-co-ends-the-fubini-rule} of \cref{properties-of-co-ends}}.
        \item\label{proof-of-existence-of-right-kan-extensions-in-rel-6}\ChapterRef{\ChapterEndsAndCoends, \cref{ends-and-coends:properties-of-co-ends-commutativity-with-homs} of \cref{ends-and-coends:properties-of-co-ends}}{\cref{properties-of-co-ends-commutativity-with-homs} of \cref{properties-of-co-ends}}.
        \item\label{proof-of-existence-of-right-kan-extensions-in-rel-7}\ChapterRef{\ChapterRelations, \cref{relations:properties-of-relations-end-formula-for-the-set-of-inclusions-of-relations} of \cref{relations:properties-of-relations}}{\cref{properties-of-relations-end-formula-for-the-set-of-inclusions-of-relations} of \cref{properties-of-relations}}.
    \end{enumerate}
    This finishes the proof.
\end{Proof}
\begin{example}{Examples of Internal Right Kan Extensions of Relations}{examples-of-internal-right-kan-extensions-of-relations}%
    Here are some examples of internal right Kan extensions of relations.
    \begin{enumerate}
        \item\label{examples-of-internal-right-kan-extensions-of-relations-orthogonal-complements}\SloganFont{Orthogonal Complements. }Let $A=B=X=\EuScript{V}$ be an inner product space, and let $R=F=\mathord{\perp}$ be the orthogonality relation, so that we have
            \begin{align*}
                R(v) &= v^{\perp}\\
                F(u) &= u^{\perp},
            \end{align*}
            for each $u,v\in\EuScript{V}$, where
            \[
                v^{\perp}%
                \defeq%
                \{u\in V\ \middle|\ v\perp u\}%
            \]%
            is the orthogonal complement of $v$. The right Kan extension $\Ran_{R}(F)$ is then given by
            \begin{align*}
                [\Ran_{R}(F)](v) &=      \bigcap_{u\in R^{-1}(v)}F(u)\\%
                                 &=      \bigcap_{\substack{u\in V\\u\perp v}}u^{\perp}\\%
                                 &=      (v^{\perp})^{\perp},%
            \end{align*}
            the double orthogonal complement. In particular:
            \begin{itemize}
                \item If $\EuScript{V}$ is finite-dimensional, then $[\Ran_{R}(F)](v)=\rmSpan(v)$.
                \item If $\EuScript{V}$ is a Hilbert space,    then $[\Ran_{R}(F)](v)=\widebar{\rmSpan(v)}$.
            \end{itemize}
        \item\label{examples-of-internal-right-kan-extensions-of-relations-galois-connections-and-closure-operators}\SloganFont{Galois Connections and Closure Operators. }Let:
            \begin{itemize}
                \item $B=X=(P,\preceq_{P})$ and $A=(Q,\preceq_{Q})$ be posets;
                \item $(f,g)$ be a Galois connection (adjunction) between $P$ and $Q$;
                \item $R,F\colon Q\rightrightproarrows P$ be the relations defined by
                    \begin{align*}
                        R(q) &\defeq \{p\in P\ \middle|\ q\preceq_{Q}f(p)\},\\
                        F(q) &\defeq \{p\in P\ \middle|\ p\preceq_{P}g(q)\}
                    \end{align*}
                    for each $q\in Q$.
            \end{itemize}
            We have
            \begin{align*}
                [\Ran_{R}(F)](p) &= \bigcap_{q\in R^{-1}(p)}F(q)\\%
                                 &= \bigcap_{\substack{q\in Q\\q\preceq_{Q}f(p)}}\{%
                                                                                     p\in P%
                                                                                     \ \middle|\ %
                                                                                     p\preceq_{P}g(q)%
                                                                                 \}\\%
                                 &= \{%
                                        p\in P%
                                        \ \middle|\ %
                                        p\preceq_{P}g(f(q))%
                                    \}\\%
                                 &= \downset g(f(p)),
            \end{align*}
            the down set of $g(f(p))$. In other words, $\Ran_{R}(F)$ is the closure operator on $P$ associated with the Galois connection $(f,g)$.
    \end{enumerate}
\end{example}
\begin{proposition}{Properties of Internal Right Kan Extensions in $\sfbfRel$}{properties-of-internal-right-kan-extensions-in-rel}%
    Let $A$, $B$, $C$ and $X$ be sets and let $R\colon A\rightproarrow B$, $S\colon B\rightproarrow C$, and $F\colon A\rightproarrow X$ be relations.
    \begin{enumerate}
        \item\label{properties-of-internal-right-kan-extensions-in-rel-functoriality}\SloganFont{Functoriality. }The assignments $R,F,(R,F)\mapsto\Ran_{R}(F)$ define functors
            \[
                \BifunctorialityPeriod{\Ran_{(-)}(F)}{\Ran_{R}}{\Ran_{(-_{1})}(-_{2})}{\eRel(A,B)^{\op}}{\eRel(A,X)}{\eRel(A,X)\times\eRel(A,B)^{\op}}{\eRel(B,X)}%
            \]%
            In other words, given relations
            \begin{webcompile}
                \begin{tikzcd}[row sep={4.0*\the\DL,between origins}, column sep={4.0*\the\DL,between origins}, background color=backgroundColor, ampersand replacement=\&]
                    A
                    \arrow[r,"R_{1}", mid vert,shift left=1.0]%
                    \arrow[r,"R_{2}"',mid vert,shift right=1.0]%
                    \&
                    B
                \end{tikzcd}
                \qquad
                \begin{tikzcd}[row sep={4.0*\the\DL,between origins}, column sep={4.0*\the\DL,between origins}, background color=backgroundColor, ampersand replacement=\&]
                    A
                    \arrow[r,"F_{1}", mid vert,shift left=0.8]%
                    \arrow[r,"F_{2}"',mid vert,shift right=0.8]%
                    \&
                    X\mrp{,}
                \end{tikzcd}
            \end{webcompile}
            if $R_{1}\subset R_{2}$ and $F_{1}\subset F_{2}$, then $\Ran_{R_{2}}(F_{1})\subset\Ran_{R_{1}}(F_{2})$.
        \item\label{properties-of-internal-right-kan-extensions-in-rel-interaction-with-composition}\SloganFont{Interaction With Composition. }We have
            \[
                \Ran_{S\procirc R}(F)%
                =
                \Ran_{S}(\Ran_{R}(F))%
            \]%
            and an equality
            \begin{webcompile}
                \begin{tikzcd}[row sep={4.0*\the\DL,between origins}, column sep={4.0*\the\DL,between origins}, background color=backgroundColor, ampersand replacement=\&]
                    \&
                    \&
                    C
                    \arrow[dd,"{\Ran_{S}(\Ran_{R}(F))}",densely dashed for mid vert,mid vert]
                    \\
                    \&
                    B
                    \arrow[ru,"S",mid vert]
                    \arrow[rd,"{\scriptscriptstyle\Ran_{R}(F)}"'{sloped,outer sep=0.1*\the\DL},""'{name=Ran,pos=0.625},densely dashed for mid vert,mid vert]
                    \&
                    \\
                    A
                    \arrow[rr,"F"',""'{name=F},mid vert]
                    \arrow[ru,"R",mid vert]
                    \&
                    \&
                    X
                    % 2-Arrows
                    \arrow[from=2-2,to=F,  shorten <= 0.25em, shorten >= 0.625em,Rightarrow]%
                    \arrow[from=1-3,to=Ran,shorten <= 0.75em, shorten >= 1.0em,  Rightarrow]%
                \end{tikzcd}
                \pastingdiagramequality%
                \begin{tikzcd}[row sep={4.0*\the\DL,between origins}, column sep={4.0*\the\DL,between origins}, background color=backgroundColor, ampersand replacement=\&]
                    \&
                    \&
                    C
                    \arrow[dd,"{\Ran_{S\procirc R}(F)}",densely dashed for mid vert,mid vert]
                    \\
                    \&
                    B
                    \arrow[ru,"S",mid vert]
                    \&
                    \\
                    A
                    \arrow[rr,"F"',""'{name=F,pos=0.425},mid vert]
                    \arrow[ru,"R",mid vert]
                    \&
                    \&
                    X
                    % 2-Arrows
                    %\arrow[from=1-3,to=F,shorten=0.5em,Rightarrow,shift left=1.0*\the\DL]%
                    \arrow[from=1-3,to=F,shorten <= 0.75em, shorten >= 0.75em, Rightarrow,shift left=0.15*\the\DL]%
                \end{tikzcd}
            \end{webcompile}
            of pasting diagrams in $\sfbfRel$.
        \item\label{properties-of-internal-right-kan-extensions-in-rel-interaction-with-converses}\SloganFont{Interaction With Converses. }We have
            \[
                \Ran_{R}(F)^{\dagger}
                =
                \Rift_{R^{\dagger}}(F^{\dagger}).%
            \]%
        \item\label{properties-of-internal-right-kan-extensions-in-rel-interaction-with-weak-inverse-images}\SloganFont{Interaction With Weak Inverse Images. }We have
            \[
                [\Ran_{R}(F)]^{-1}(x)%
                =%
                \{%
                    b\in B%
                    \ \middle|\ %
                    R^{-1}(b)\subset F^{-1}(x)%
                \}%
            \]%
            for each $x\in X$.
        %\item\label{properties-of-internal-right-kan-extensions-in-rel-}\SloganFont{. }
    \end{enumerate}
\end{proposition}
\begin{Proof}{Proof of \cref{properties-of-internal-right-kan-extensions-in-rel}}%
    \FirstProofBox{\cref{properties-of-internal-right-kan-extensions-in-rel-functoriality}: Functoriality}%
    We have
    \begin{align*}
        [\Ran_{R_{2}}(F_{1})](b) &=       \bigcap_{a\in R^{-1}_{2}(b)}F_{1}(a)\\
                                 &\subset \bigcap_{a\in R^{-1}_{1}(b)}F_{1}(a)\\
                                 &\subset \bigcap_{a\in R^{-1}_{1}(b)}F_{2}(a)\\
                                 &=       [\Ran_{R_{1}}(F_{2})](b)
    \end{align*}
    for each $b\in B$, so we therefore have $\Ran_{R_{2}}(F_{1})\subset\Ran_{R_{1}}(F_{2})$.

    \ProofBox{\cref{properties-of-internal-right-kan-extensions-in-rel-interaction-with-composition}: Interaction With Composition}%
    This holds in a general bicategory with the necessary right Kan extensions, being therefore a special case of \cref{TODO}.

    \ProofBox{\cref{properties-of-internal-right-kan-extensions-in-rel-interaction-with-converses}: Interaction With Converses}%
    We have
    \begin{align*}
        [\Rift_{R^{\dagger}}(F^{\dagger})](x) &= \{b\in B\ \middle|\ R^{\dagger}(b)\subset F^{\dagger}(x)\}\\%
                                              &= \{b\in B\ \middle|\ R^{-1}(b)\subset F^{-1}(x)\}\\%
                                              &= \Ran_{R}(F)^{-1}(x)\\%
                                              &= \Ran_{R}(F)^{\dagger}(x)%
    \end{align*}
    where we have used \cref{internal-right-kan-lifts-in-rel} and \cref{properties-of-internal-right-kan-extensions-in-rel-interaction-with-weak-inverse-images}.

    \ProofBox{\cref{properties-of-internal-right-kan-extensions-in-rel-interaction-with-weak-inverse-images}: Interaction With Weak Inverse Images}%
    We proceed in a few steps.
    \begin{itemize}
        \item We have $b\in[\Ran_{R}(F)]^{-1}(x)$ \textiff, for each $a\in R^{-1}(b)$, we have $b\in F(a)$.
        \item This holds \textiff, for each $a\in R^{-1}(b)$, we have $a\in F^{-1}(b)$.
        \item This holds \textiff $R^{-1}(b)\subset F^{-1}(b)$.
    \end{itemize}
    This finishes the proof.
\end{Proof}
\subsection{Internal Right Kan Lifts}\label{subsection-internal-right-kan-lifts-in-rel}
Let $A$, $B$, and $X$ be sets and let $R\colon A\rightproarrow B$ and $F\colon X\rightproarrow B$ be relations.
\begin{motivation}{Setting for Internal Right Kan Lifts in $\sfbfRel$}{setting-for-internal-right-kan-lifts-in-rel}%
    We want to understand internal right Kan lifts in $\sfbfRel$, which look like this:
    \[
        \begin{tikzcd}[row sep={5.0*\the\DL,between origins}, column sep={5.0*\the\DL,between origins}, background color=backgroundColor, ampersand replacement=\&]
            \&%
            A%
            \arrow[d, "R", mid vert]%
            \\%
            X%
            \arrow[ru, "\Rift_{R}(F)",densely dashed for mid vert, mid vert]%
            \arrow[r, "F"',""'{name=F,pos=0.425},mid vert]%
            \&%
            B\mrp{.}%
            % 2-Arrows
            \arrow[from=1-2,to=F,shorten=0.75*\the\DL,Rightarrow,shift left=0.15*\the\DL]%
        \end{tikzcd}
    \]%
    Note in particular here that $F\colon B\rightproarrow X$ is a relation from $B$ to $X$. These will form a functor
    \[
        \Rift_{R}%
        \colon%
        \eRel(X,B)%
        \to%
        \eRel(X,A)%
    \]%
    that is right adjoint to the postcomposition by $R$ functor
    \[
        R_{*}%
        \colon%
        \eRel(X,A)%
        \to%
        \eRel(X,B).%
    \]%
\end{motivation}
\begin{proposition}{Internal Right Kan Lifts in $\sfbfRel$}{internal-right-kan-lifts-in-rel}%
    The internal right Kan lift of $F$ along $R$ is the relation \index[notation]{RiftRF@$\Rift_{R}(F)$}$\Rift_{R}(F)$ described as follows:
    \begin{enumerate}
        \item\label{internal-right-kan-lifts-in-rel-1}Viewing relations from $X$ to $A$ as subsets of $X\times A$, we have
            \[
                \Rift_{R}(F)%
                =%
                \{%
                    (x,a)\in X\times A%
                    \ \middle|\ %
                    \begin{aligned}
                        &\text{for each $b\in B$, if $a\sim_{R}b$,}\\
                        &\text{then we have $x\sim_{F}b$}%
                    \end{aligned}
                \}.%
            \]%
        \item\label{internal-right-kan-lifts-in-rel-2}Viewing relations as functions $X\times A\to\TV$, we have
            \begin{align*}
                (\Rift_{R}(F))^{-_{1}}_{-_{2}} &= \int_{b\in B}\eHom_{\TTV}(R^{b}_{-_{1}},F^{b}_{-_{2}})\\
                                               &= \bigwedge_{b\in B}\eHom_{\TTV}(R^{b}_{-_{1}},F^{b}_{-_{2}}),%
            \end{align*}
            where the meet $\bigwedge$ is taken in the poset $(\TV,\preceq)$ of \ChapterRef{\ChapterSets, \cref{sets:the-poset-of-truth-values}}{\cref{the-poset-of-truth-values}}.
        \item\label{internal-right-kan-lifts-in-rel-3}Viewing relations as functions $X\to\mathcal{P}(A)$, we have
            \[
                [\Rift_{R}(F)](x)%
                =%
                \{%
                    a\in A%
                    \ \middle|\ %
                    R(a)\subset F(x)%
                \}%
            \]%
            for each $a\in A$.
    \end{enumerate}
\end{proposition}
\begin{Proof}{Proof of \cref{internal-right-kan-lifts-in-rel}}%
    We have
    \begin{envsmallsize}
        \begin{align*}
            \Hom_{\eRel(X,B)}(R\procirc F,T) &\cong  \int_{x\in X}\int_{b\in B}\eHom_{\TTV}((R\procirc F)^{b}_{x},T^{b}_{x})\\
                                             &\cong  \int_{x\in X}\int_{b\in B}\eHom_{\TTV}((\int^{a\in A}R^{b}_{a}\times F^{a}_{x}),T^{b}_{x})\\
                                             &\cong  \int_{x\in X}\int_{b\in B}\int_{a\in A}\eHom_{\TTV}(R^{b}_{a}\times F^{a}_{x},T^{b}_{x})\\
                                             &\cong  \int_{x\in X}\int_{b\in B}\int_{a\in A}\eHom_{\TTV}(F^{a}_{x},\eHom_{\TTV}(R^{b}_{a},T^{b}_{x}))\\
                                             &\cong  \int_{x\in X}\int_{a\in A}\int_{b\in B}\eHom_{\TTV}(F^{a}_{x},\eHom_{\TTV}(R^{b}_{a},T^{b}_{x}))\\
                                             &\cong  \int_{x\in X}\int_{a\in A}\eHom_{\TTV}(F^{a}_{x},\int_{b\in B}\eHom_{\TTV}(R^{b}_{a},T^{b}_{x}))\\
                                             &\cong  \Hom_{\eRel(X,A)}(F,\int_{b\in B}\eHom_{\TTV}(R^{b}_{-_{1}},T^{b}_{-_{2}}))%
        \end{align*}
    \end{envsmallsize}
    naturally in each $F\in\eRel(X,A)$ and each $T\in\eRel(X,B)$, showing that
    \[
        \int_{b\in B}\eHom_{\TTV}(R^{b}_{-_{1}},F^{b}_{-_{2}})%
    \]%
    is right adjoint to the postcomposition functor $R\procirc-$, being thus the right Kan lift along $R$. Here we have used the following results, respectively (i.e.\ for each $\cong$ sign):
    \begin{enumerate}
        \item\label{proof-of-existence-of-right-kan-lifts-in-rel-1}\ChapterRef{\ChapterRelations, \cref{relations:properties-of-relations-end-formula-for-the-set-of-inclusions-of-relations} of \cref{relations:properties-of-relations}}{\cref{properties-of-relations-end-formula-for-the-set-of-inclusions-of-relations} of \cref{properties-of-relations}}.
        \item\label{proof-of-existence-of-right-kan-lifts-in-rel-2}\cref{composition-of-relations}.
        \item\label{proof-of-existence-of-right-kan-lifts-in-rel-3}\ChapterRef{\ChapterEndsAndCoends, \cref{ends-and-coends:properties-of-co-ends-commutativity-with-homs} of \cref{ends-and-coends:properties-of-co-ends}}{\cref{properties-of-co-ends-commutativity-with-homs} of \cref{properties-of-co-ends}}.
        \item\label{proof-of-existence-of-right-kan-lifts-in-rel-4}\ChapterRef{\ChapterSets, \cref{sets:cartesian-closedness-of-the-poset-of-truth-values}}{\cref{cartesian-closedness-of-the-poset-of-truth-values}}.
        \item\label{proof-of-existence-of-right-kan-lifts-in-rel-5}\ChapterRef{\ChapterEndsAndCoends, \cref{ends-and-coends:properties-of-co-ends-the-fubini-rule} of \cref{ends-and-coends:properties-of-co-ends}}{\cref{properties-of-co-ends-the-fubini-rule} of \cref{properties-of-co-ends}}.
        \item\label{proof-of-existence-of-right-kan-lifts-in-rel-6}\ChapterRef{\ChapterEndsAndCoends, \cref{ends-and-coends:properties-of-co-ends-commutativity-with-homs} of \cref{ends-and-coends:properties-of-co-ends}}{\cref{properties-of-co-ends-commutativity-with-homs} of \cref{properties-of-co-ends}}.
        \item\label{proof-of-existence-of-right-kan-lifts-in-rel-7}\ChapterRef{\ChapterRelations, \cref{relations:properties-of-relations-end-formula-for-the-set-of-inclusions-of-relations} of \cref{relations:properties-of-relations}}{\cref{properties-of-relations-end-formula-for-the-set-of-inclusions-of-relations} of \cref{properties-of-relations}}.
    \end{enumerate}
    This finishes the proof.
\end{Proof}
\begin{example}{Examples of Internal Right Kan Extensions of Relations}{examples-of-internal-right-kan-lifts-of-relations}%
    Here are some examples of internal right Kan lifts of relations.
    \begin{enumerate}
        \item\label{examples-of-internal-right-kan-lifts-of-relations-pullbacks}\SloganFont{Pullbacks. }Let $p\colon A\to B$ and $f\colon X\to B$ be functions. We have
            \begin{align*}
                [\Rift_{\Gr(p)}(\Gr(f))](x) &= \{a\in A\ \middle|\ [\Gr(p)](a)\subset[\Gr(f)](x)\}\\%
                                            &= \{a\in A\ \middle|\ p(a)=f(x)\}.%
            \end{align*}
            Thus, as a subset of $X\times A$, the right Kan lift $\Rift_{\Gr(p)}(\Gr(f))$ corresponds precisely to the pullback $X\times_{B}A$ of $X$ and $A$ along $p$ and $f$ of \ChapterRef{\ChapterConstructionsWithSets, \cref{constructions-with-sets:subsection-limits-of-sets-pullbacks}}{\cref{subsection-limits-of-sets-pullbacks}}.
    \end{enumerate}
\end{example}
\begin{proposition}{Properties of Internal Right Kan Lifts in $\sfbfRel$}{properties-of-internal-right-kan-lifts-in-rel}%
    Let $A$, $B$, $C$ and $X$ be sets and let $R\colon A\rightproarrow B$, $S\colon B\rightproarrow C$, and $F\colon X\rightproarrow B$ be relations.
    \begin{enumerate}
        \item\label{properties-of-internal-right-kan-lifts-in-rel-functoriality}\SloganFont{Functoriality. }The assignments $R,F,(R,F)\mapsto\Rift_{R}(F)$ define functors
            \[
                \BifunctorialityPeriod{\Rift_{(-)}(F)}{\Rift_{R}}{\Rift_{(-_{1})}(-_{2})}{\eRel(A,B)^{\op}}{\eRel(A,X)}{\eRel(A,X)\times\eRel(A,B)^{\op}}{\eRel(B,X)}%
            \]%
            In other words, given relations
            \begin{webcompile}
                \begin{tikzcd}[row sep={4.0*\the\DL,between origins}, column sep={4.0*\the\DL,between origins}, background color=backgroundColor, ampersand replacement=\&]
                    A
                    \arrow[r,"R_{1}", mid vert,shift left=1.0]%
                    \arrow[r,"R_{2}"',mid vert,shift right=1.0]%
                    \&
                    B
                \end{tikzcd}
                \qquad
                \begin{tikzcd}[row sep={4.0*\the\DL,between origins}, column sep={4.0*\the\DL,between origins}, background color=backgroundColor, ampersand replacement=\&]
                    A
                    \arrow[r,"F_{1}", mid vert,shift left=0.8]%
                    \arrow[r,"F_{2}"',mid vert,shift right=0.8]%
                    \&
                    X\mrp{,}
                \end{tikzcd}
            \end{webcompile}
            if $R_{1}\subset R_{2}$ and $F_{1}\subset F_{2}$, then $\Rift_{R_{2}}(F_{1})\subset\Rift_{R_{1}}(F_{2})$.
        \item\label{properties-of-internal-right-kan-lifts-in-rel-interaction-with-composition}\SloganFont{Interaction With Composition. }We have
            \[
                \Rift_{S\procirc R}(F)%
                =
                \Rift_{R}(\Ran_{S}(F))%
            \]%
            and an equality
            \begin{webcompile}
                \begin{tikzcd}[row sep={4.0*\the\DL,between origins}, column sep={7.0*\the\DL,between origins}, background color=backgroundColor, ampersand replacement=\&]
                    \&
                    A
                    \arrow[d,"R",mid vert]
                    \\
                    \&
                    B
                    \arrow[d,"S",mid vert]
                    \\
                    X
                    \arrow[r,"F"',""'{name=F,pos=0.35},mid vert]
                    \arrow[ru, "{\scriptscriptstyle\Rift_{S}(F)}"'{sloped,pos=0.275,outer sep=0.0*\the\DL},""'{name=Rift,pos=0.1},mid vert,densely dashed for mid vert]
                    \arrow[ruu,"{\Rift_{R}(\Rift_{S}(F))}",    mid vert,densely dashed for mid vert]
                    \&
                    C
                    % 2-Arrows
                    \arrow[from=1-2,to=Rift,shorten <= 0.75em, shorten >= 2.5em,Rightarrow,shift left=0.35*\the\DL]%
                    \arrow[from=2-2,to=F,   shorten <= 0.0em,  shorten >= 0.75em,  Rightarrow, shift left=0.25*\the\DL]%
                \end{tikzcd}
                \pastingdiagramequality%
                \begin{tikzcd}[row sep={4.0*\the\DL,between origins}, column sep={7.0*\the\DL,between origins}, background color=backgroundColor, ampersand replacement=\&]
                    \&
                    A
                    \arrow[d,"R",mid vert]
                    \\
                    \&
                    B
                    \arrow[d,"S",mid vert]
                    \\
                    X
                    \arrow[r,"F"',""'{name=F,pos=0.35},mid vert]
                    \arrow[ruu,"{\Rift_{S\procirc R}(F)}",    mid vert,densely dashed for mid vert]
                    \&
                    C
                    % 2-Arrows
                    \arrow[from=1-2,to=F,   shorten <= 1.0em,  shorten >= 1.0em,  Rightarrow, shift left=0.175*\the\DL]%
                \end{tikzcd}
            \end{webcompile}
            of pasting diagrams in $\sfbfRel$.
        \item\label{properties-of-internal-right-kan-lifts-in-rel-interaction-with-converses}\SloganFont{Interaction With Converses. }We have
            \[
                \Rift_{R}(F)^{\dagger}
                =
                \Ran_{R^{\dagger}}(F^{\dagger}).%
            \]%
        \item\label{properties-of-internal-right-kan-lifts-in-rel-interaction-with-weak-inverse-images}\SloganFont{Interaction With Weak Inverse Images. }We have
            \begin{webcompile}
                \Rift_{R}(F)^{\dagger}%
                =%
                \Ran_{\chi'_{B}}\pig(F^{\dagger}\pig)\circ R,%
                \quad%
                \begin{tikzcd}[row sep={5.0*\the\DL,between origins}, column sep={5.0*\the\DL,between origins}, background color=backgroundColor, ampersand replacement=\&]
                    \&%
                    B%
                    \arrow[r, "F^{\dagger}",""{name=F,pos=0.7}]%
                    \arrow[d, "\chi_{B}"', hook]%
                    \&%
                    \mathcal{P}(X)\mrp{,}%
                    \\%
                    A%
                    \arrow[r, "R"']%
                    \&%
                    \mathcal{P}(B)^{\op}%
                    \arrow[ru, "{\Ran_{\chi_{A}}\pig(F^{-1}\pig)}"', bend right=10]%
                    \&
                    % 2-Arrows
                    \arrow[from=F,to=2-2,shorten=0.75*\the\DL,Rightarrow]%
                \end{tikzcd}%
            \end{webcompile}
            where $\Ran_{\chi_{A}}\pig(F^{\dagger}\pig)$ is computed by the formula
            \begin{align*}
                [\Ran_{\chi_{A}}\pig(F^{\dagger}\pig)](U) &\cong \int_{a\in A}\chi_{\mathcal{P}(B)^{\op}}(U,\chi_{a})\pitchfork F^{\dagger}(a)\\
                                                          &\cong \int_{a\in A}\chi_{\mathcal{P}(B)}(\chi_{a},U)\pitchfork F^{-1}(a)\\
                                                          &\cong \int_{a\in A}\chi_{U}(a)\pitchfork F(a)\\
                                                          &\cong \bigcap_{a\in A}\chi_{U}(a)\pitchfork F(a)\\
                                                          &\cong \bigcap_{a\in U}F(a)
            \end{align*}
            for each $U\in\mathcal{P}(A)$, so we have%
            \[
                [\Rift_{R}(F)]^{-1}(a)%
                =%
                \bigcap_{b\in R(a)}F^{-1}(b)%
            \]%
            for each $a\in A$.
        %\item\label{properties-of-internal-right-kan-lifts-in-rel-}\SloganFont{. }
    \end{enumerate}
\end{proposition}
\begin{Proof}{Proof of \cref{properties-of-internal-right-kan-lifts-in-rel}}%
    \FirstProofBox{\cref{properties-of-internal-right-kan-lifts-in-rel-functoriality}: Functoriality}%
    We have
    \begin{align*}
        [\Rift_{R_{2}}(F_{1})](x) &= \{a\in A\ \middle|\ R_{2}(a)\subset F_{1}(x)\}\\
                                  &\subset \{a\in A\ \middle|\ R_{1}(a)\subset F_{1}(x)\}\\
                                  &\subset \{a\in A\ \middle|\ R_{1}(a)\subset F_{2}(x)\}\\
                                  &=       \Rift_{R_{1}}(F_{2})
    \end{align*}
    for each $x\in X$, so we therefore have $\Rift_{R_{2}}(F_{1})\subset\Rift_{R_{1}}(F_{2})$.

    \ProofBox{\cref{properties-of-internal-right-kan-lifts-in-rel-interaction-with-composition}: Interaction With Composition}%
    This holds in a general bicategory with the necessary right Kan lifts, being therefore a special case of \cref{TODO}.

    \ProofBox{\cref{properties-of-internal-right-kan-lifts-in-rel-interaction-with-converses}: Interaction With Converses}%
    This follows from \cref{properties-of-internal-right-kan-extensions-in-rel-interaction-with-converses} of \cref{properties-of-internal-right-kan-extensions-in-rel} by duality.

    \ProofBox{\cref{properties-of-internal-right-kan-lifts-in-rel-interaction-with-weak-inverse-images}: Interaction With Weak Inverse Images}%
    We proceed in a few steps.
    \begin{itemize}
        \item We have $x\in\Rift_{R}(F)^{\dagger}(a)$ \textiff $a\in\Rift_{R}(F)(x)$.
        \item This holds \textiff $R(a)\subset F(x)$.
        \item This holds \textiff, for each $b\in R(a)$, we have $b\in F(x)$.
        \item This holds \textiff, for each $b\in R(a)$, we have $x\in F^{-1}(b)$.
        \item This holds \textiff $x\in\bigcap_{b\in R(a)}F^{-1}(b)$.
    \end{itemize}
    This finishes the proof.
\end{Proof}
\subsection{Closedness}\label{subsection-closedness-of-rel}
\begin{proposition}{Closedness of $\sfbfRel$}{closedness-of-rel}%
    The 2-category $\sfbfRel$ is a closed bicategory, there being, for each $R\colon A\rightproarrow B$ and set $X$, a pair of adjunctions
    \begin{webcompile}
        \begin{gathered}
            \Adjunction#R^{*}#\Ran_{R}#\Rel(B,X)#\Rel(A,X),#\\
            \Adjunction#R_{!}#\Rift_{R}#\Rel(X,A)#\Rel(X,B),#
        \end{gathered}
    \end{webcompile}%
    witnessed by bijections
    \begin{align*}
        \eRel(S\procirc R,T) &\cong \eRel(S,\Ran_{R}(T)),\\
        \eRel(R\procirc U,V) &\cong \eRel(U,\Rift_{R}(V)),
    \end{align*}
    natural in $S\in\Rel(B,X)$, $T\in\Rel(A,X)$, $U\in\Rel(X,A)$, and $V\in\Rel(X,B)$.
\end{proposition}
\begin{Proof}{Proof of \cref{closedness-of-rel}}%
    This follows from \ChapterRef{\ChapterConstructionsWithRelations, \cref{constructions-with-relations:existence-of-right-kan-extensions-in-rel,constructions-with-relations:existence-of-right-kan-lifts-in-rel}}{\cref{existence-of-right-kan-extensions-in-rel,existence-of-right-kan-lifts-in-rel}}.
\end{Proof}
\subsection{$\sfRel$ as a Category of Free Algebras}\label{subsection-rel-as-a-category-of-free-algebras}
\begin{proposition}{$\sfRel$ as a Category of Free Algebras}{rel-as-a-category-of-free-algebras}%
    We have an isomorphism of categories
    \[
        \Rel%
        \cong%
        \FreeAlg_{\mathcal{P}_{!}}(\Sets),
    \]%
    where $\mathcal{P}_{!}$ is the powerset monad of \ChapterRef{\ChapterMonads, \cref{monads:the-powerset-monad}}{\cref{the-powerset-monad}}.
\end{proposition}
\begin{Proof}{Proof of \cref{rel-as-a-category-of-free-algebras}}%
    Omitted.
\end{Proof}
\section{Properties of the 2-Category of Relations With Apartness Composition}\label{section-properties-of-the-2-category-of-relations-with-apartness-composition}
\subsection{Self-Duality}\label{subsection-self-duality-of-rel-apartness}
\begin{proposition}{Self-Duality for the (2-)Category of Relations With Apartness Composition}{self-duality-for-the-2-category-of-relations-with-apartness-composition}%
    The 2-/category of relations with apartness-composition-is self-dual:
    \begin{enumerate}
        \item\label{self-duality-for-the-2-category-of-relations-with-apartness-composition-1}\SloganFont{Self-Duality \rmI. }We have an isomorphism
            \[
                (\aptsfRel)^{\op}%
                \cong%
                \aptsfRel%
            \]%
            of categories.
        \item\label{self-duality-for-the-2-category-of-relations-with-apartness-composition-2}\SloganFont{Self-Duality \rmII. }We have a 2-isomorphism
            \[
                (\aptsfbfRel)^{\op}%
                \cong%
                \aptsfbfRel%
            \]%
            of 2-categories.
    \end{enumerate}
\end{proposition}
\begin{Proof}{Proof of \cref{self-duality-for-the-2-category-of-relations-with-apartness-composition}}%
    \FirstProofBox{\cref{self-duality-for-the-2-category-of-relations-with-apartness-composition-1}: Self-Duality \rmI}%
    We claim that the functor
    \[
        (-)^{\dagger}%
        \colon%
        (\aptsfRel)^{\op}%
        \to%
        \aptsfRel%
    \]%
    given by the identity on objects and by $R\mapsto R^{\dagger}$ on morphisms is an isomorphism of categories. Note that this is indeed a functor by \cref{properties-of-converses-of-relations-interaction-with-apartness-composition,properties-of-converses-of-relations-identity-2} of \cref{properties-of-converses-of-relations}.

    \indent By \ChapterRef{\ChapterCategories, \cref{categories:properties-of-isomorphisms-of-categories-characterisations} of \cref{categories:properties-of-isomorphisms-of-categories}}{\cref{properties-of-isomorphisms-of-categories-characterisations} of \cref{properties-of-isomorphisms-of-categories}}, it suffices to show that $(-)^{\dagger}$ is bijective on objects (which follows by definition) and fully faithful. Indeed, the map
    \[
        (-)^{\dagger}%
        \colon%
        \Rel(A,B)%
        \to%
        \Rel(B,A)%
    \]%
    defined by the assignment $R\mapsto R^{\dagger}$ is a bijection by \cref{properties-of-converses-of-relations-invertibility} of \cref{properties-of-converses-of-relations}, showing $(-)^{\dagger}$ to be fully faithful.

    \ProofBox{\cref{self-duality-for-the-2-category-of-relations-with-apartness-composition-2}: Self-Duality \rmII}%
    We claim that the 2-functor
    \[
        (-)^{\dagger}%
        \colon%
        \sfRel^{\op}%
        \to%
        \sfRel%
    \]%
    given by the identity on objects, by $R\mapsto R^{\dagger}$ on morphisms, and by preserving inclusions on 2-morphisms via \cref{properties-of-converses-of-relations-functoriality} of \cref{properties-of-converses-of-relations}, is an isomorphism of categories.

    \indent By \cref{TODO}, it suffices to show that $(-)^{\dagger}$ is:
    \begin{itemize}
        \item Bijective on objects, which follows by definition.
        \item Bijective on $1$-morphisms, which was shown in \cref{self-duality-for-the-2-category-of-relations-1}.
        \item Bijective on 2-morphisms, which follows from \cref{properties-of-converses-of-relations-functoriality} of \cref{properties-of-converses-of-relations}.
    \end{itemize}
    Thus $(-)^{\dagger}$ is indeed a 2-isomorphism of categories.
\end{Proof}
\subsection{Isomorphisms and Equivalences}\label{subsection-isomorphisms-and-equivalences-in-rel-apartness}
\begin{itemize}
    \item $R\aptcirc R^{\dagger}\subset\nabla_{B}$:  For each $b_{1},b_{2}\in B$, if, for each $x\in A$, either $x\sim_{R}b_{1}$ or $x\sim_{R}b_{2}$, then $b_{1}\neq b_{2}$.
    \item $R^{\dagger}\aptcirc R\subset\nabla_{A}$:  For each $a_{1},a_{2}\in A$, if, for each $y\in B$, either $a_{1}\sim_{R}y$ or $a_{2}\sim_{R}y$, then $a_{1}\neq a_{2}$.
    \item $\nabla_{A}\subset R\aptcirc R^{\dagger}$: For each $a_{1},a_{2}\in A$, if $a_{1}\neq a_{2}$, then, for each $x\in A$, either $x\sim_{R}b_{1}$ or $x\sim_{R}b_{2}$.
    \item $\nabla_{B}\subset R\aptcirc R^{\dagger}$: For each $b_{1},b_{2}\in B$, if $b_{1}\neq b_{2}$, then, for each $y\in B$, either $a_{1}\sim_{R}y$ or $a_{2}\sim_{R}y$.
\end{itemize}
A
\begin{itemize}
    \item $R\aptcirc R^{\dagger}\subset\nabla_{B}$:  For each $b_{1},b_{2}\in B$, if $R^{-1}(b_{1})\cup R^{-1}(b_{2})=A$, then $b_{1}\neq b_{2}$.
    \item $R^{\dagger}\aptcirc R\subset\nabla_{A}$:  For each $a_{1},a_{2}\in A$, if $R(a_{1})\cup R(a_{2})=B$, then $a_{1}\neq a_{2}$.
    \item $\nabla_{A}\subset R\aptcirc R^{\dagger}$: For each $a_{1},a_{2}\in A$, if $a_{1}\neq a_{2}$, then $R(a_{1})\cup R(a_{2})=B$.
    \item $\nabla_{B}\subset R\aptcirc R^{\dagger}$: For each $b_{1},b_{2}\in B$, if $b_{1}\neq b_{2}$, then $R^{-1}(b_{1})\cup R^{-1}(b_{2})=A$.
\end{itemize}
$R'=A\times B\setminus R$
\begin{itemize}
    \item $R\aptcirc R^{\dagger}\subset\nabla_{B}$:  For each $b_{1},b_{2}\in B$, if $R^{\prime,-1}(b_{1})\cap R^{\prime,-1}(b_{2})=\emptyset$, then $b_{1}\neq b_{2}$.
    \item $R^{\dagger}\aptcirc R\subset\nabla_{A}$:  For each $a_{1},a_{2}\in A$, if $R'(a_{1})\cup R'(a_{2})=\emptyset$,                       then $a_{1}\neq a_{2}$.
    \item $\nabla_{A}\subset R\aptcirc R^{\dagger}$: For each $a_{1},a_{2}\in A$, if $a_{1}\neq a_{2}$,                                         then $R'(a_{1})\cap R'(a_{2})=\emptyset$.
    \item $\nabla_{B}\subset R\aptcirc R^{\dagger}$: For each $b_{1},b_{2}\in B$, if $b_{1}\neq b_{2}$,                                         then $R^{\prime,-1}(b_{1})\cap R^{\prime,-1}(b_{2})=\emptyset$.
\end{itemize}
so:
\begin{itemize}
    \item $R\aptcirc R^{\dagger}\subset\nabla_{B}$:  $R'$ is surjective. (Via contrapositive)
    \item $R^{\dagger}\aptcirc R\subset\nabla_{A}$:  $R'$ is total. (Via contrapositive)
    \item $\nabla_{A}\subset R\aptcirc R^{\dagger}$: $R'$ is injective.
    \item $\nabla_{B}\subset R\aptcirc R^{\dagger}$: $R'$ is functional.
\end{itemize}
Let $R\colon A\rightproarrow B$ be a relation from $A$ to $B$.
\begin{proposition}{Isomorphisms and Equivalences in $\sfbfRel$}{isomorphisms-and-equivalences-in-rel-apartness}%
    The following conditions are equivalent:
    \begin{enumerate}
        \item\label{isomorphisms-and-equivalences-in-rel-apartness-1}The relation $R\colon A\rightproarrow B$ is an equivalence in $\sfbfRel$, i.e.:
            \begin{itemize}
                \itemstar There exists a relation $R^{-1}\colon B\rightproarrow A$ from $B$ to $A$ together with isomorphisms
                    \begin{align*}
                        R^{-1}\aptcirc R &\cong \nabla_{A},\\
                        R\aptcirc R^{-1} &\cong \nabla_{B}.
                    \end{align*}
            \end{itemize}
        \item\label{isomorphisms-and-equivalences-in-rel-apartness-2}The relation $R\colon A\rightproarrow B$ is an isomorphism in $\Rel$, i.e.:
            \begin{itemize}
                \itemstar There exists a relation $R^{-1}\colon B\rightproarrow A$ from $B$ to $A$ such that we have
                    \begin{align*}
                        R^{-1}\aptcirc R &= \nabla_{A},\\
                        R\aptcirc R^{-1} &= \nabla_{B}.
                    \end{align*}
            \end{itemize}
        \item\label{isomorphisms-and-equivalences-in-rel-apartness-3}There exists a bijection $f\colon A\isorightarrow B$ with $R=\Gr(f)$.
    \end{enumerate}
\end{proposition}
\begin{Proof}{Proof of \cref{isomorphisms-and-equivalences-in-rel-apartness}}%
    We claim that \cref{isomorphisms-and-equivalences-in-rel-apartness-1,isomorphisms-and-equivalences-in-rel-apartness-2,isomorphisms-and-equivalences-in-rel-apartness-3} are indeed equivalent:
    \begin{itemize}
        \item\SloganFont{\cref{isomorphisms-and-equivalences-in-rel-apartness-1}$\iff$\cref{isomorphisms-and-equivalences-in-rel-apartness-2}: }This follows from the fact that $\sfbfRel$ is locally posetal, so that natural isomorphisms and equalities of $1$-morphisms in $\sfbfRel$ coincide.
        \item\SloganFont{\cref{isomorphisms-and-equivalences-in-rel-apartness-2}$\implies$\cref{isomorphisms-and-equivalences-in-rel-apartness-3}: }
        \item\SloganFont{\cref{isomorphisms-and-equivalences-in-rel-apartness-3}$\implies$\cref{isomorphisms-and-equivalences-in-rel-apartness-2}: }
    \end{itemize}
    This finishes the proof.
\end{Proof}
\subsection{Internal Adjunctions}\label{subsection-internal-adjunctions-in-rel-apartness}
\subsection{Internal Monads}\label{subsection-internal-monads-in-rel-apartness}
\subsection{Internal Comonads}\label{subsection-internal-comonads-in-rel-apartness}
\subsection{Co/Monoids}\label{subsection-co-monoids-in-rel-apartness}
\subsection{Monomorphisms}\label{subsection-monomorphisms-in-rel-apartness}
\subsection{2-Categorical Monomorphisms}\label{subsection-2-categorical-monomorphisms-in-rel-apartness}
\subsection{Epimorphisms}\label{subsection-epimorphisms-in-rel-apartness}
\subsection{2-Categorical Epimorphisms}\label{subsection-2-categorical-epimorphisms-in-rel-apartness}
\subsection{Co/Limits}\label{subsection-co-limits-in-rel-apartness}
\subsection{Internal Kan Extensions and Lifts}\label{subsection-internal-kan-extensions-and-lifts-in-rel-apartness}
\subsection{Closedness}\label{subsection-closedness-of-rel-apartness}
\subsection{$\aptsfRel$ as a Category of Free Algebras}\label{subsection-rel-apartness-as-a-category-of-free-algebras}
\section{The Adjoint Pairs $R_{!}\dashv R_{-1}$ and $R^{-1}\dashv R_{*}$}\label{section-the-adjoint-pairs-r-shriek-r-minus-one-and-r-minus-one-r-star}
\subsection{Direct Images}\label{subsection-direct-images-relations}
Let $X$ and $Y$ be sets and let $R\colon X\rightproarrow Y$ be a relation.
\begin{definition}{Direct Images}{the-direct-image-function-associated-to-a-relation}%
    The \index[set-theory]{relation!associated direct image function}\textbf{direct image function associated to $R$} is the function\index[notation]{Rshriek@$R^{"!}$}%
    %--- Begin Footnote ---%
    \footnote{%
        \SloganFont{Further Notation: }Also written simply $R\colon\mathcal{P}(X)\to\mathcal{P}(Y)$.
    }%
    %---  End Footnote  ---%
    \[%
        R_{!}%
        \colon%
        \mathcal{P}(X)%
        \to%
        \mathcal{P}(Y)%
    \]%
    defined by\index[notation]{RU@$R(U)$}%
    %--- Begin Footnote ---%
    \footnote{%
        \SloganFont{Further Terminology: }The set $R(U)$ is called the \textbf{direct image of $U$ by $R$}.
    }%
    %---  End Footnote  ---%
    \begin{align*}
        R_{!}(U) &\defeq \bigcup_{a\in U}R(a)\\
                 &=      \{%
                             b\in Y%
                             \ \middle|\ %
                             \begin{aligned}
                                 &\text{there exists some $a\in U$}\\
                                 &\text{such that $b\in R(a)$}
                             \end{aligned}
                         \}%
    \end{align*}
    for each $U\in\mathcal{P}(X)$.
\end{definition}
\begin{warning}{Notation for Direct Images Is Confusing}{notation-for-direct-images-is-confusing-relations}%
    Notation for direct images between powersets is tricky; see \ChapterRef{\ChapterConstructionsWithSets, \cref{constructions-with-sets:notation-for-direct-images-is-confusing}}{\cref{notation-for-direct-images-is-confusing}}. Here we'll try to align our notation for relations with that for functions.
\end{warning}
\begin{remark}{Unwinding \cref{the-direct-image-function-associated-to-a-relation}}{unwinding-the-direct-image-function-associated-to-a-relation}%
    Identifying subsets of $X$ with relations from $\pt$ to $X$ via \ChapterRef{\ChapterConstructionsWithSets, \cref{constructions-with-sets:elementary-properties-of-powersets-powersets-as-sets-of-relations} of \cref{constructions-with-sets:elementary-properties-of-powersets}}{\cref{elementary-properties-of-powersets-powersets-as-sets-of-relations} of \cref{elementary-properties-of-powersets}}, we see that the direct image function associated to $R$ is equivalently the function
    \[
        R_{!}%
        \colon%
        \underbrace{\mathcal{P}(X)}_{\cong\Rel(\pt,X)}
        \to%
        \underbrace{\mathcal{P}(Y)}_{\cong\Rel(\pt,Y)}%
    \]%
    defined by
    \[
        R_{!}(U)%
        \defeq%
        R\procirc U%
    \]%
    for each $U\in\mathcal{P}(X)$, where $R\procirc U$ is the composition
    \[
        \pt%
        \xrightproarrow{U}%
        X%
        \xrightproarrow{R}%
        Y.
    \]%
\end{remark}
\begin{proposition}{Properties of Direct Image Functions}{properties-of-direct-image-functions-associated-to-relations}%
    Let $R\colon X\rightproarrow Y$ be a relation.
    \begin{enumerate}
        \item\label{properties-of-direct-image-functions-associated-to-relations-functoriality}\SloganFont{Functoriality. }The assignment $U\mapsto R_{!}(U)$ defines a functor
            \[
                R_{!}%
                \colon%
                (\mathcal{P}(X),\subset)%
                \to%
                (\mathcal{P}(Y),\subset)%
            \]%
            where
            \begin{itemize}
                \item\SloganFont{Action on Objects. }For each $U\in\mathcal{P}(X)$, we have
                    \[
                        [R_{!}](U)%
                        \defeq%
                        R_{!}(U).
                    \]%
                \item\SloganFont{Action on Morphisms. }For each $U,V\in\mathcal{P}(X)$:
                    \begin{itemize}
                        \item If $U\subset V$, then $R_{!}(U)\subset R_{!}(V)$.
                    \end{itemize}
            \end{itemize}
        \item\label{properties-of-direct-image-functions-associated-to-relations-adjointness}\SloganFont{Adjointness. }We have an adjunction
            \begin{webcompile}
                \Adjunction#R_{!}#R_{-1}#\mathcal{P}(X)#\mathcal{P}(Y),#
            \end{webcompile}
            witnessed by:
            \begin{enumerate}
                \item\label{properties-of-direct-image-functions-associated-to-relations-adjointness-1}Units and counits of the form
                    \begin{align*}
                        \id_{\mathcal{P}(X)} \hookrightarrow R_{-1}\circ R_{!},\\
                        R_{!}\circ R_{-1}    \hookrightarrow \id_{\mathcal{P}(Y)},
                    \end{align*}
                    having components of the form
                    \begin{gather*}
                        U\subset R_{-1}(R_{!}(U)),\\
                        R_{!}(R_{-1}(V))\subset V
                    \end{gather*}
                    indexed by $U\in\mathcal{P}(X)$ and $V\in\mathcal{P}(Y)$
                \item\label{properties-of-direct-image-functions-associated-to-relations-adjointness-2}A bijections of sets
                    \[%
                        \Hom_{\mathcal{P}(X)}(R_{!}(U),V)%
                        \cong%
                        \Hom_{\mathcal{P}(X)}(U,R_{-1}(V)),%
                    \]%
                    natural in $U\in\mathcal{P}(X)$ and $V\in\mathcal{P}(Y)$. In particular:
                    \begin{itemize}
                        \itemstar The following conditions are equivalent:
                            \begin{itemize}
                                \item We have $R_{!}(U)\subset V$.
                                \item We have $U\subset R_{-1}(V)$.
                            \end{itemize}
                    \end{itemize}
            \end{enumerate}
        \item\label{properties-of-direct-image-functions-associated-to-relations-preservation-of-colimits}\SloganFont{Preservation of Colimits. }We have an equality of sets
            \[
                R_{!}(\bigcup_{i\in I}U_{i})%
                =%
                \bigcup_{i\in I}R_{!}(U_{i}),%
            \]%
            natural in $\{U_{i}\}_{i\in I}\in\mathcal{P}(X)^{\times I}$. In particular, we have equalities%
            \[
                \begin{gathered}
                    R_{!}(U)\cup R_{!}(V)                  = R_{!}(U\cup V),\\
                    R_{!}(\emptyset)                       = \emptyset,
                \end{gathered}
            \]%
            natural in $U,V\in\mathcal{P}(X)$.
        \item\label{properties-of-direct-image-functions-associated-to-relations-oplax-preservation-of-limits}\SloganFont{Oplax Preservation of Limits. }We have an inclusion of sets
            \[
                R_{!}(\bigcap_{i\in I}U_{i})%
                \subset%
                \bigcap_{i\in I}R_{!}(U_{i}),%
            \]%
            natural in $\{U_{i}\}_{i\in I}\in\mathcal{P}(X)^{\times I}$. In particular, we have inclusions%
            \[
                \begin{gathered}
                    R_{!}(U\cap V) \subset R_{!}(U)\cap R_{!}(V),\\
                    R_{!}(X)       \subset Y,
                \end{gathered}
            \]%
            natural in $U,V\in\mathcal{P}(X)$.
        \item\label{properties-of-direct-image-functions-associated-to-relations-symmetric-strict-monoidality-with-respect-to-unions}\SloganFont{Symmetric Strict Monoidality With Respect to Unions. }The direct image function of \cref{properties-of-direct-image-functions-associated-to-relations-functoriality} has a symmetric strict monoidal structure
            \[
                (R_{!},R^{\otimes}_{!},R^{\otimes}_{*|\Unit})
                \colon
                (\mathcal{P}(X),\cup,\emptyset)
                \to
                (\mathcal{P}(Y),\cup,\emptyset),
            \]%
            being equipped with equalities%
            \[
                \begin{gathered}
                    R^{\otimes}_{*|U,V}   \colon R_{!}(U)\cup R_{!}(V) \rightequalsarrow R_{!}(U\cup V),\\
                    R^{\otimes}_{*|\Unit} \colon \emptyset             \rightequalsarrow \emptyset,
                \end{gathered}
            \]%
            natural in $U,V\in\mathcal{P}(X)$.
        \item\label{properties-of-direct-image-functions-associated-to-relations-symmetric-oplax-monoidality-with-respect-to-intersections}\SloganFont{Symmetric Oplax Monoidality With Respect to Intersections. }The direct image function of \cref{properties-of-direct-image-functions-associated-to-relations-functoriality} has a symmetric oplax monoidal structure
            \[
                (R_{!},R^{\otimes}_{!},R^{\otimes}_{*|\Unit})
                \colon
                (\mathcal{P}(X),\cap,X)
                \to
                (\mathcal{P}(Y),\cap,Y),
            \]%
            being equipped with inclusions%
            \[
                \begin{gathered}
                    R^{\otimes}_{*|U,V}   \colon R_{!}(U\cap V) \subset R_{!}(U)\cap R_{!}(V),\\
                    R^{\otimes}_{*|\Unit} \colon R_{!}(X)       \subset Y,
                \end{gathered}
            \]%
            natural in $U,V\in\mathcal{P}(X)$.
        \item\label{properties-of-direct-image-functions-associated-to-relations-relation-to-codirect-images}\SloganFont{Relation to Codirect Images. }We have
            \[
                R_{!}(U)%
                =%
                Y\setminus R_{*}(X\setminus U)%
            \]%
            for each $U\in\mathcal{P}(X)$.
        %\item\label{properties-of-direct-image-functions-associated-to-relations-}\SloganFont{. }
    \end{enumerate}
\end{proposition}
\begin{Proof}{Proof of \cref{properties-of-direct-image-functions-associated-to-relations}}%
    \FirstProofBox{\cref{properties-of-direct-image-functions-associated-to-relations-functoriality}: Functoriality}%
    Clear.

    \ProofBox{\cref{properties-of-direct-image-functions-associated-to-relations-adjointness}: Adjointness}%
    This follows from \ChapterRef{\ChapterKanExtensions, \cref{kan-extensions:properties-of-kan-extensions-triple-adjointness} of \cref{kan-extensions:properties-of-kan-extensions}}{\cref{properties-of-kan-extensions-triple-adjointness} of \cref{properties-of-kan-extensions}}.

    \ProofBox{\cref{properties-of-direct-image-functions-associated-to-relations-preservation-of-colimits}: Preservation of Colimits}%
    This follows from \cref{properties-of-direct-image-functions-associated-to-relations-adjointness} and \ChapterRef{\ChapterAdjunctionsAndTheYonedaLemma, \cref{adjunctions-and-the-yoneda-lemma:properties-of-adjunctions-interaction-of-co-limits} of \cref{adjunctions-and-the-yoneda-lemma:properties-of-adjunctions}}{\cref{properties-of-adjunctions-interaction-of-co-limits} of \cref{properties-of-adjunctions}}.

    \ProofBox{\cref{properties-of-direct-image-functions-associated-to-relations-oplax-preservation-of-limits}: Oplax Preservation of Limits}%
    Omitted.

    \ProofBox{\cref{properties-of-direct-image-functions-associated-to-relations-symmetric-strict-monoidality-with-respect-to-unions}: Symmetric Strict Monoidality With Respect to Unions}%
    This follows from \cref{properties-of-direct-image-functions-associated-to-relations-preservation-of-colimits}.

    \ProofBox{\cref{properties-of-direct-image-functions-associated-to-relations-symmetric-oplax-monoidality-with-respect-to-intersections}: Symmetric Oplax Monoidality With Respect to Intersections}%
    This follows from \cref{properties-of-direct-image-functions-associated-to-relations-oplax-preservation-of-limits}.

    \ProofBox{\cref{properties-of-direct-image-functions-associated-to-relations-relation-to-codirect-images}: Relation to Codirect Images}%
    The proof proceeds in the same way as in the case of functions (\ChapterRef{\ChapterConstructionsWithSets, \cref{constructions-with-sets:properties-of-direct-images-i-relation-to-codirect-images} of \cref{constructions-with-sets:properties-of-direct-images-i}}{\cref{properties-of-direct-images-properties-of-direct-images-i-relation-to-codirect-images} of \cref{properties-of-direct-images-i}}): applying \cref{properties-of-codirect-image-functions-associated-to-relations-relation-to-direct-images} of \cref{properties-of-codirect-image-functions-associated-to-relations} to $A\setminus U$, we have
    \begin{align*}
        R_{*}(X\setminus U) &= Y\setminus R_{!}(X\setminus(X\setminus U))\\
                            &= Y\setminus R_{!}(U).
    \end{align*}
    Taking complements, we then obtain
    \begin{align*}
        R_{!}(U) &= Y\setminus(Y\setminus R_{!}(U)),\\
                 &= Y\setminus R_{*}(X\setminus U),
    \end{align*}
    which finishes the proof.
\end{Proof}
\begin{proposition}{Properties of the Direct Image Function Operation}{properties-of-the-direct-image-function-operation-for-relations}%
    Let $R\colon X\rightproarrow Y$ be a relation.
    \begin{enumerate}
        \item\label{properties-of-the-direct-image-function-operation-for-relations-functionality-1}\SloganFont{Functionality \rmI. }The assignment $R\mapsto R_{!}$ defines a function
            \[
                (-)_{!}%
                \colon%
                \Rel(X,Y)
                \to%
                \Sets(\mathcal{P}(X),\mathcal{P}(Y)).
            \]%
        \item\label{properties-of-the-direct-image-function-operation-for-relations-functionality-2}\SloganFont{Functionality \rmII. }The assignment $R\mapsto R_{!}$ defines a function
            \[
                (-)_{!}%
                \colon%
                \Rel(X,Y)
                \to%
                \Pos((\mathcal{P}(X),\subset),(\mathcal{P}(Y),\subset)).
            \]%
        \item\label{properties-of-the-direct-image-function-operation-for-relations-interaction-with-identities}\SloganFont{Interaction With Identities. }For each $X\in\Obj(\Sets)$, we have%
            %--- Begin Footnote ---%
            \footnote{%
                That is, the postcomposition function
                \[
                    (\chi_{X})_{!}%
                    \colon%
                    \Rel(\pt,X)%
                    \to%
                    \Rel(\pt,X)%
                \]%
                is equal to $\id_{\Rel(\pt,X)}$.
            }%
            %---  End Footnote  ---%
            \[
                (\chi_{X})_{!}%
                =%
                \id_{\mathcal{P}(X)}.%
            \]%
        \item\label{properties-of-the-direct-image-function-operation-for-relations-interaction-with-composition}\SloganFont{Interaction With Composition. }For each pair of composable relations $R\colon X\rightproarrow Y$ and $S\colon Y\rightproarrow C$, we have%
            %--- Begin Footnote ---%
            \footnote{%
                That is, we have
                \begin{webcompile}
                    (S\procirc R)_{!}%
                    =%
                    S_{!}\circ R_{!},%
                    \quad%
                    \begin{tikzcd}[row sep={5.0*\the\DL,between origins}, column sep={6.5*\the\DL,between origins}, background color=backgroundColor, ampersand replacement=\&]
                        {\Rel(\pt,X)}
                        \arrow[r,"R_{!}"]
                        \arrow[rd,"{(S\procirc R)_{!}}"']
                        \&
                        {\Rel(\pt,Y)}
                        \arrow[d,"S_{!}"]
                        \\
                        \&
                        {\Rel(\pt,C)}\mrp{.}
                    \end{tikzcd}
                \end{webcompile}
                \par\vspace*{\TCBBoxCorrection}
            }%
            %---  End Footnote  ---%
            \begin{webcompile}
                (S\procirc R)_{!}%
                =%
                S_{!}\circ R_{!},%
                \quad%
                \begin{tikzcd}[row sep={5.0*\the\DL,between origins}, column sep={5.0*\the\DL,between origins}, background color=backgroundColor, ampersand replacement=\&]
                    \mathcal{P}(X)
                    \arrow[r,"R_{!}"]
                    \arrow[rd,"{(S\procirc R)_{!}}"']
                    \&
                    \mathcal{P}(Y)
                    \arrow[d,"S_{!}"]
                    \\
                    \&
                    \mathcal{P}(C)\mrp{.}
                \end{tikzcd}
            \end{webcompile}
        %\item\label{properties-of-the-direct-image-function-operation-for-relations-}\SloganFont{. }
    \end{enumerate}
\end{proposition}
\begin{Proof}{Proof of \cref{properties-of-the-direct-image-function-operation-for-relations}}%
    \FirstProofBox{\cref{properties-of-the-direct-image-function-operation-for-relations-functionality-1}: Functionality \rmI}%
    Clear.

    \ProofBox{\cref{properties-of-the-direct-image-function-operation-for-relations-functionality-2}: Functionality \rmII}%
    Clear.

    \ProofBox{\cref{properties-of-the-direct-image-function-operation-for-relations-interaction-with-identities}: Interaction With Identities}%
    Indeed, we have
    \begin{align*}
        (\chi_{X})_{!}(U) &\defeq \bigcup_{a\in U}\chi_{X}(a)\\
                          &\defeq \bigcup_{a\in U}\{a\}\\
                          &=      U\\
                          &\defeq \id_{\mathcal{P}(X)}(U)
    \end{align*}
    for each $U\in\mathcal{P}(X)$. Thus $(\chi_{X})_{!}=\id_{\mathcal{P}(X)}$.

    \ProofBox{\cref{properties-of-the-direct-image-function-operation-for-relations-interaction-with-composition}: Interaction With Composition}%
    Indeed, we have
    \begin{align*}
        (S\procirc R)_{!}(U) &\defeq \bigcup_{a\in U}[S\procirc R](a)\\
                             &\defeq \bigcup_{a\in U}S(R(a))\\
                             &\defeq \bigcup_{a\in U}S_{!}(R(a))\\
                             %&\defeq \bigcup_{a\in U}\bigcup_{x\in R(a)}S(x)\\
                             %&\defeq  \bigcup_{a\in U}S_{!}(R(a))\\
                             &=      S_{!}(\bigcup_{a\in U}R(a))\\
                             &\defeq S_{!}(R_{!}(U))\\
                             &\defeq [S_{!}\circ R_{!}](U)
    \end{align*}
    for each $U\in\mathcal{P}(X)$, where we used \cref{properties-of-direct-image-functions-associated-to-relations-preservation-of-colimits} of \cref{properties-of-direct-image-functions-associated-to-relations}. Thus $(S\procirc R)_{!}=S_{!}\circ R_{!}$.
\end{Proof}
\subsection{Strong Inverse Images}\label{subsection-strong-inverse-images}
Let $X$ and $Y$ be sets and let $R\colon X\rightproarrow Y$ be a relation.
\begin{definition}{Strong Inverse Images}{the-strong-inverse-image-function-associated-to-a-relation}%
    The \index[set-theory]{relation!associated strong inverse image function}\textbf{strong inverse image function associated to $R$} is the function\index[notation]{Rminusone@$R_{-1}$}%
    \[%
        R_{-1}%
        \colon%
        \mathcal{P}(Y)%
        \to%
        \mathcal{P}(X)%
    \]%
    defined by\index[notation]{RminusoneV@$R_{-1}(V)$}%
    %--- Begin Footnote ---%
    \footnote{%
        \SloganFont{Further Terminology: }The set $R_{-1}(V)$ is called the \textbf{strong inverse image of $V$ by $R$}.
        \par\vspace*{\TCBBoxCorrection}
    }%
    %---  End Footnote  ---%
    \[%
        R_{-1}(V)
        \defeq%
        \{%
            a\in X%
            \ \middle|\ %
            R(a)\subset V%
        \}%
    \]%
    for each $V\in\mathcal{P}(Y)$.
\end{definition}
\begin{remark}{Unwinding \cref{the-strong-inverse-image-function-associated-to-a-relation}}{unwinding-the-strong-inverse-image-function-associated-to-a-relation}%
    Identifying subsets of $Y$ with relations from $\pt$ to $Y$ via \ChapterRef{\ChapterConstructionsWithSets, \cref{constructions-with-sets:elementary-properties-of-powersets-powersets-as-sets-of-relations} of \cref{constructions-with-sets:elementary-properties-of-powersets}}{\cref{elementary-properties-of-powersets-powersets-as-sets-of-relations} of \cref{elementary-properties-of-powersets}}, we see that the inverse image function associated to $R$ is equivalently the function
    \[
        R_{-1}%
        \colon%
        \underbrace{\mathcal{P}(Y)}_{\cong\Rel(\pt,Y)}%
        \to%
        \underbrace{\mathcal{P}(X)}_{\cong\Rel(\pt,X)}%
    \]%
    defined by
    \begin{webcompile}
        R_{-1}(V)%
        \defeq%
        \Rift_{R}(V),%
        \quad
        \begin{tikzcd}[row sep={5.0*\the\DL,between origins}, column sep={5.0*\the\DL,between origins}, background color=backgroundColor, ampersand replacement=\&]
            \&
            X
            \arrow[d,mid vert, "R"]
            \\
            \pt%
            \arrow[ru,"\Rift_{R}(V)",dashed, mid vert]
            \arrow[r,mid vert, "V"'{name=F}]
            \&
            Y\mrp{,}
            % 2-Xrrows
            \arrow[from=1-2,to=F,Rightarrow,shorten=0.5em,pos=0.4]
        \end{tikzcd}
    \end{webcompile}
    and being explicitly computed by
    \begin{align*}
        R_{-1}(V) &\defeq \Rift_{R}(V)\\%
                  &\cong  \int_{b\in Y}\Hom_{\TTV}(R^{b}_{-_{1}},V^{b}_{-_{2}}),%
    \end{align*}
    where we have used \cref{existence-of-right-kan-lifts-in-rel}.
\end{remark}
\begin{Proof}{Proof of \cref{unwinding-the-strong-inverse-image-function-associated-to-a-relation}}%
    We have
    \begin{envsmallsize}
        \begin{align*}
            \Rift_{R}(V)%
            &\cong%
            \int_{b\in Y}\Hom_{\TTV}(R^{b}_{-_{1}},V^{b}_{-_{2}})\\%
            &=%
            \{%
                a\in X%
                \ \middle|\ %
                \int_{b\in Y}\Hom_{\TTV}(R^{b}_{a},V^{b}_{\point})=\true%
            \}%
            \\
            &=
            \{%
                a\in X%
                \ \middle|\ %
                \begin{aligned}
                    &\text{for each $b\in Y$, at least one of the}\\[-2.5pt]
                    &\text{following conditions hold:}\\[7.5pt]
                    &\mspace{25mu}\rlap{\text{1.}}\mspace{22.5mu}\text{We have $R^{b}_{a}=\false$}\\%
                    &\mspace{25mu}\rlap{\text{2.}}\mspace{22.5mu}\text{The following conditions hold:}\\[7.5pt]%
                    &\mspace{50mu}\rlap{\text{\norg(a\norg)}}\mspace{30mu}\text{We have $R^{b}_{a}=\true$}\\
                    &\mspace{50mu}\rlap{\text{\norg(b\norg)}}\mspace{30mu}\text{We have $V^{b}_{\point}=\true$}\\[10pt]
                \end{aligned}
            \}\\%
            &=
            \{%
                a\in X%
                \ \middle|\ %
                \begin{aligned}
                    &\text{for each $b\in Y$, at least one of the}\\[-2.5pt]
                    &\text{following conditions hold:}\\[7.5pt]
                    &\mspace{25mu}\rlap{\text{1.}}\mspace{22.5mu}\text{We have $b\nin R(a)$}\\%
                    &\mspace{25mu}\rlap{\text{2.}}\mspace{22.5mu}\text{The following conditions hold:}\\[7.5pt]%
                    &\mspace{50mu}\rlap{\text{\norg(a\norg)}}\mspace{30mu}\text{We have $b\in R(a)$}\\
                    &\mspace{50mu}\rlap{\text{\norg(b\norg)}}\mspace{30mu}\text{We have $b\in V$}\\[10pt]
                \end{aligned}
            \}\\%
            &=
            \{%
                a\in X%
                \ \middle|\ %
                \text{for each $b\in R(a)$, we have $b\in V$}%
            \}\\%
            &=
            \{%
                a\in X%
                \ \middle|\ %
                R(a)\subset V%
            \}\\%
            &\defeq%
            R_{-1}(V).%
        \end{align*}
    \end{envsmallsize}
    This finishes the proof.
\end{Proof}
\begin{proposition}{Properties of Strong Inverse Images}{properties-of-strong-inverse-image-functions-associated-to-relations}%
    Let $R\colon X\rightproarrow Y$ be a relation.
    \begin{enumerate}
        \item\label{properties-of-strong-inverse-image-functions-associated-to-relations-functoriality}\SloganFont{Functoriality. }The assignment $V\mapsto R_{-1}(V)$ defines a functor
            \[
                R_{-1}%
                \colon%
                (\mathcal{P}(Y),\subset)%
                \to%
                (\mathcal{P}(X),\subset)%
            \]%
            where
            \begin{itemize}
                \item\SloganFont{Action on Objects. }For each $V\in\mathcal{P}(Y)$, we have
                    \[
                        [R_{-1}](V)%
                        \defeq%
                        R_{-1}(V).
                    \]%
                \item\SloganFont{Action on Morphisms. }For each $U,V\in\mathcal{P}(Y)$:
                    \begin{itemize}
                        \item If $U\subset V$, then $R_{-1}(U)\subset R_{-1}(V)$.
                    \end{itemize}
            \end{itemize}
        \item\label{properties-of-strong-inverse-image-functions-associated-to-relations-adjointness}\SloganFont{Adjointness. }We have an adjunction
            \begin{webcompile}
                \Adjunction#R_{!}#R_{-1}#\mathcal{P}(X)#\mathcal{P}(Y),#
            \end{webcompile}
            witnessed by a bijections of sets
            \[%
                \Hom_{\mathcal{P}(X)}(R_{!}(U),V)%
                \cong%
                \Hom_{\mathcal{P}(X)}(U,R_{-1}(V)),%
            \]%
            natural in $U\in\mathcal{P}(X)$ and $V\in\mathcal{P}(Y)$, i.e.\ such that:
            \begin{itemize}
                \itemstar The following conditions are equivalent:
                    \begin{itemize}
                        \item We have $R_{!}(U)\subset V$.
                        \item We have $U\subset R_{-1}(V)$.
                    \end{itemize}
            \end{itemize}
        \item\label{properties-of-strong-inverse-image-functions-associated-to-relations-lax-preservation-of-colimits}\SloganFont{Lax Preservation of Colimits. }We have an inclusion of sets
            \[
                \bigcup_{i\in I}R_{-1}(U_{i})%
                \subset%
                R_{-1}(\bigcup_{i\in I}U_{i}),%
            \]%
            natural in $\{U_{i}\}_{i\in I}\in\mathcal{P}(Y)^{\times I}$. In particular, we have inclusions%
            \[
                \begin{gathered}
                    R_{-1}(U)\cup R_{-1}(V) \subset R_{-1}(U\cup V),\\
                    \emptyset               \subset R_{-1}(\emptyset),
                \end{gathered}
            \]%
            natural in $U,V\in\mathcal{P}(Y)$.
        \item\label{properties-of-strong-inverse-image-functions-associated-to-relations-preservation-of-limits}\SloganFont{Preservation of Limits. }We have an equality of sets
            \[
                    R_{-1}(\bigcap_{i\in I}U_{i})%
                    =%
                    \bigcap_{i\in I}R_{-1}(U_{i}),%
            \]%
            natural in $\{U_{i}\}_{i\in I}\in\mathcal{P}(Y)^{\times I}$. In particular, we have equalities%
            \[
                \begin{gathered}
                    R_{-1}(U\cap V) = R_{-1}(U)\cap R_{-1}(V),\\
                    R_{-1}(Y)       = Y,
                \end{gathered}
            \]%
            natural in $U,V\in\mathcal{P}(Y)$.
        \item\label{properties-of-strong-inverse-image-functions-associated-to-relations-symmetric-lax-monoidality-with-respect-to-unions}\SloganFont{Symmetric Lax Monoidality With Respect to Unions. }The codirect image function of \cref{properties-of-strong-inverse-image-functions-associated-to-relations-functoriality} has a symmetric lax monoidal structure
            \[
                (R_{-1},R^{\otimes}_{-1},R^{\otimes}_{-1|\Unit})
                \colon
                (\mathcal{P}(X),\cup,\emptyset)
                \to
                (\mathcal{P}(Y),\cup,\emptyset),
            \]%
            being equipped with inclusions%
            \[
                \begin{gathered}
                    R^{\otimes}_{-1|U,V}   \colon R_{-1}(U)\cup R_{-1}(V) \subset R_{-1}(U\cup V),\\
                    R^{\otimes}_{-1|\Unit} \colon \emptyset               \subset R_{-1}(\emptyset),
                \end{gathered}
            \]%
            natural in $U,V\in\mathcal{P}(Y)$.
        \item\label{properties-of-strong-inverse-image-functions-associated-to-relations-symmetric-strict-monoidality-with-respect-to-intersections}\SloganFont{Symmetric Strict Monoidality With Respect to Intersections. }The direct image function of \cref{properties-of-strong-inverse-image-functions-associated-to-relations-functoriality} has a symmetric strict monoidal structure
            \[
                (R_{-1},R^{\otimes}_{-1},R^{\otimes}_{-1|\Unit})
                \colon
                (\mathcal{P}(X),\cap,X)
                \to
                (\mathcal{P}(Y),\cap,Y),
            \]%
            being equipped with equalities%
            \[
                \begin{gathered}
                    R^{\otimes}_{-1|U,V}   \colon R_{-1}(U\cap V) \rightequalsarrow R_{-1}(U)\cap R_{-1}(V),\\
                    R^{\otimes}_{-1|\Unit} \colon R_{-1}(X)       \rightequalsarrow Y,
                \end{gathered}
            \]%
            natural in $U,V\in\mathcal{P}(Y)$.
        \item\label{properties-of-strong-inverse-image-functions-associated-to-relations-interaction-with-weak-inverse-images-1}\SloganFont{Interaction With Weak Inverse Images \rmI. }We have
            \[
                R_{-1}(V)%
                =%
                X\setminus R^{-1}(Y\setminus V)
            \]%
            for each $V\in\mathcal{P}(Y)$.
        \item\label{properties-of-strong-inverse-image-functions-associated-to-relations-interaction-with-weak-inverse-images-2}\SloganFont{Interaction With Weak Inverse Images \rmII. }Let $R\colon X\rightproarrow Y$ be a relation from $X$ to $Y$.
            \begin{enumerate}
                \item\label{properties-of-strong-inverse-image-functions-associated-to-relations-interaction-with-weak-inverse-images-2-a}If $R$ is a total relation, then we have an inclusion of sets
                    \[
                        R_{-1}(V)
                        \subset
                        R^{-1}(V)
                    \]%
                    natural in $V\in\mathcal{P}(Y)$.
                \item\label{properties-of-strong-inverse-image-functions-associated-to-relations-interaction-with-weak-inverse-images-2-b}If $R$ is total and functional, then the above inclusion is in fact an equality.
                \item\label{properties-of-strong-inverse-image-functions-associated-to-relations-interaction-with-weak-inverse-images-2-c}Conversely, if we have $R_{-1}=R^{-1}$, then $R$ is total and functional.
            \end{enumerate}
        %\item\label{properties-of-strong-inverse-image-functions-associated-to-relations-}\SloganFont{. }
    \end{enumerate}
\end{proposition}
\begin{Proof}{Proof of \cref{properties-of-strong-inverse-image-functions-associated-to-relations}}%
    \FirstProofBox{\cref{properties-of-strong-inverse-image-functions-associated-to-relations-functoriality}: Functoriality}%
    Clear.

    \ProofBox{\cref{properties-of-strong-inverse-image-functions-associated-to-relations-adjointness}: Adjointness}%
    This follows from \ChapterRef{\ChapterKanExtensions, \cref{kan-extensions:properties-of-kan-extensions-triple-adjointness} of \cref{kan-extensions:properties-of-kan-extensions}}{\cref{properties-of-kan-extensions-triple-adjointness} of \cref{properties-of-kan-extensions}}.

    \ProofBox{\cref{properties-of-strong-inverse-image-functions-associated-to-relations-lax-preservation-of-colimits}: Lax Preservation of Colimits}%
    Omitted.

    \ProofBox{\cref{properties-of-strong-inverse-image-functions-associated-to-relations-preservation-of-limits}: Preservation of Limits}%
    This follows from \cref{properties-of-strong-inverse-image-functions-associated-to-relations-adjointness} and \ChapterRef{\ChapterAdjunctionsAndTheYonedaLemma, \cref{adjunctions-and-the-yoneda-lemma:properties-of-adjunctions-interaction-of-co-limits} of \cref{adjunctions-and-the-yoneda-lemma:properties-of-adjunctions}}{\cref{properties-of-adjunctions-interaction-of-co-limits} of \cref{properties-of-adjunctions}}.

    \ProofBox{\cref{properties-of-strong-inverse-image-functions-associated-to-relations-symmetric-lax-monoidality-with-respect-to-unions}: Symmetric Lax Monoidality With Respect to Unions}%
    This follows from \cref{properties-of-strong-inverse-image-functions-associated-to-relations-lax-preservation-of-colimits}.

    \ProofBox{\cref{properties-of-strong-inverse-image-functions-associated-to-relations-symmetric-strict-monoidality-with-respect-to-intersections}: Symmetric Strict Monoidality With Respect to Intersections}%
    This follows from \cref{properties-of-strong-inverse-image-functions-associated-to-relations-preservation-of-limits}.

    \ProofBox{\cref{properties-of-strong-inverse-image-functions-associated-to-relations-interaction-with-weak-inverse-images-1}: Interaction With Weak Inverse Images \rmI}%
    We claim we have an equality
    \[
        R_{-1}(Y\setminus V)%
        =%
        X\setminus R^{-1}(V).%
    \]%
    Indeed, we have
    \begin{align*}
        R_{-1}(Y\setminus V) &= \{a\in X\ |\ R(a)\subset Y\setminus V\},\\%
        X\setminus R^{-1}(V) &= \{a\in X\ |\ R(a)\cap V=\emptyset\}.%
    \end{align*}
    Taking $V=Y\setminus V$ then implies the original statement.

    \ProofBox{\cref{properties-of-strong-inverse-image-functions-associated-to-relations-interaction-with-weak-inverse-images-2}: Interaction With Weak Inverse Images \rmII}%
    \cref{properties-of-strong-inverse-image-functions-associated-to-relations-interaction-with-weak-inverse-images-2-a} is clear, while \cref{properties-of-strong-inverse-image-functions-associated-to-relations-interaction-with-weak-inverse-images-2-b,properties-of-strong-inverse-image-functions-associated-to-relations-interaction-with-weak-inverse-images-2-c} follow from \cref{properties-of-graphs-of-functions-characterisations} of \cref{properties-of-graphs-of-functions}.
\end{Proof}
\begin{proposition}{Properties of the Strong Inverse Image Function Operation}{properties-of-the-strong-inverse-image-function-operation-for-relations}%
    Let $R\colon X\rightproarrow Y$ be a relation.
    \begin{enumerate}
        \item\label{properties-of-the-strong-inverse-image-function-operation-for-relations-functionality-1}\SloganFont{Functionality \rmI. }The assignment $R\mapsto R_{-1}$ defines a function
            \[
                (-)_{-1}%
                \colon%
                \Sets(X,Y)
                \to%
                \Sets(\mathcal{P}(X),\mathcal{P}(Y)).
            \]%
        \item\label{properties-of-the-strong-inverse-image-function-operation-for-relations-functionality-2}\SloganFont{Functionality \rmII. }The assignment $R\mapsto R_{-1}$ defines a function
            \[
                (-)_{-1}%
                \colon%
                \Sets(X,Y)
                \to%
                \Pos((\mathcal{P}(X),\subset),(\mathcal{P}(Y),\subset)).
            \]%
        \item\label{properties-of-the-strong-inverse-image-function-operation-for-relations-interaction-with-identities}\SloganFont{Interaction With Identities. }For each $X\in\Obj(\Sets)$, we have
            \[
                (\id_{X})_{-1}%
                =%
                \id_{\mathcal{P}(X)}.%
            \]%
        \item\label{properties-of-the-strong-inverse-image-function-operation-for-relations-interaction-with-composition}\SloganFont{Interaction With Composition. }For each pair of composable relations $R\colon X\rightproarrow Y$ and $S\colon Y\rightproarrow C$, we have%
            \begin{webcompile}
                (S\procirc R)_{-1}%
                =%
                R_{-1}\circ S_{-1},%
                \quad
                \begin{tikzcd}[row sep={5.0*\the\DL,between origins}, column sep={5.0*\the\DL,between origins}, background color=backgroundColor, ampersand replacement=\&]
                    \mathcal{P}(C)
                    \arrow[r,"S_{-1}"]
                    \arrow[rd,"{(S\procirc R)_{-1}}"']
                    \&
                    \mathcal{P}(Y)
                    \arrow[d,"R_{-1}"]
                    \\
                    \&
                    \mathcal{P}(X)\mrp{.}
                \end{tikzcd}
            \end{webcompile}
        %\item\label{properties-of-the-strong-inverse-image-function-operation-for-relations-}\SloganFont{. }
    \end{enumerate}
\end{proposition}
\begin{Proof}{Proof of \cref{properties-of-the-strong-inverse-image-function-operation-for-relations}}%
    \FirstProofBox{\cref{properties-of-the-strong-inverse-image-function-operation-for-relations-functionality-1}: Functionality \rmI}%
    Clear.

    \ProofBox{\cref{properties-of-the-strong-inverse-image-function-operation-for-relations-functionality-2}: Functionality \rmII}%
    Clear.

    \ProofBox{\cref{properties-of-the-strong-inverse-image-function-operation-for-relations-interaction-with-identities}: Interaction With Identities}%
    Indeed, we have
    \begin{align*}
        (\chi_{X})_{-1}(U) &\defeq \{a\in X\ \middle|\ \chi_{X}(a)\subset U\}\\
                           &\defeq \{a\in X\ \middle|\ \{a\}\subset U\}\\
                           &=      U
    \end{align*}
    for each $U\in\mathcal{P}(X)$. Thus $(\chi_{X})_{-1}=\id_{\mathcal{P}(X)}$.

    \ProofBox{\cref{properties-of-the-strong-inverse-image-function-operation-for-relations-interaction-with-composition}: Interaction With Composition}%
    Indeed, we have
    \begin{align*}
        (S\procirc R)_{-1}(U) &\defeq \{a\in X\ \middle|\ [S\procirc R](a)\subset U\}\\
                              &\defeq \{a\in X\ \middle|\ S(R(a))\subset U\}\\
                              &\defeq \{a\in X\ \middle|\ S_{!}(R(a))\subset U\}\\
                              &=      \{a\in X\ \middle|\ R(a)\subset S_{-1}(U)\}\\
                              &\defeq R_{-1}(S_{-1}(U))\\
                              &\defeq [R_{-1}\circ S_{-1}](U)
    \end{align*}
    for each $U\in\mathcal{P}(C)$, where we used \cref{properties-of-strong-inverse-image-functions-associated-to-relations-adjointness} of \cref{properties-of-strong-inverse-image-functions-associated-to-relations}, which implies that the conditions
    \begin{itemize}
        \item We have $S_{!}(R(a))\subset U$.
        \item We have $R(a)\subset S_{-1}(U)$.
    \end{itemize}
    are equivalent. Thus $(S\procirc R)_{-1}=R_{-1}\circ S_{-1}$.
\end{Proof}
\subsection{Weak Inverse Images}\label{subsection-weak-inverse-images}
Let $X$ and $Y$ be sets and let $R\colon X\rightproarrow Y$ be a relation.
\begin{definition}{Weak Inverse Images}{the-weak-inverse-image-function-associated-to-a-relation}%
    The \index[set-theory]{relation!associated weak inverse image function}\textbf{weak inverse image function associated to $R$}%
    %--- Begin Footnote ---%
    \footnote{%
        \SloganFont{Further Terminology: }Also called simply the \textbf{inverse image function associated to $R$}.
    } %
    %---  End Footnote  ---%
    is the function\index[notation]{Rminusone@$R^{-1}$}%
    \[%
        R^{-1}%
        \colon%
        \mathcal{P}(Y)%
        \to%
        \mathcal{P}(X)%
    \]%
    defined by\index[notation]{RminusoneV@$R^{-1}(V)$}%
    %--- Begin Footnote ---%
    \footnote{%
        \SloganFont{Further Terminology: }The set $R^{-1}(V)$ is called the \textbf{weak inverse image of $V$ by $R$} or simply the \textbf{inverse image of $V$ by $R$}.
        \par\vspace*{\TCBBoxCorrection}
    }%
    %---  End Footnote  ---%
    \[%
        R^{-1}(V)
        \defeq%
        \{%
            a\in X%
            \ \middle|\ %
            R(a)\cap V\neq\emptyset%
        \}%
    \]%
    for each $V\in\mathcal{P}(Y)$.
\end{definition}
\begin{remark}{Unwinding \cref{the-weak-inverse-image-function-associated-to-a-relation}}{unwinding-the-weak-inverse-image-function-associated-to-a-relation}%
    Identifying subsets of $Y$ with relations from $Y$ to $\pt$ via \ChapterRef{\ChapterConstructionsWithSets, \cref{constructions-with-sets:elementary-properties-of-powersets-powersets-as-sets-of-relations} of \cref{constructions-with-sets:elementary-properties-of-powersets}}{\cref{elementary-properties-of-powersets-powersets-as-sets-of-relations} of \cref{elementary-properties-of-powersets}}, we see that the weak inverse image function associated to $R$ is equivalently the function
    \[
        R^{-1}%
        \colon%
        \underbrace{\mathcal{P}(Y)}_{\cong\Rel(Y,\pt)}%
        \to%
        \underbrace{\mathcal{P}(X)}_{\cong\Rel(X,\pt)}
    \]%
    defined by
    \[
        R^{-1}(V)%
        \defeq%
        V\procirc R%
    \]%
    for each $V\in\mathcal{P}(X)$, where $R\procirc V$ is the composition
    \[
        X%
        \xrightproarrow{R}%
        Y
        \xrightproarrow{V}%
        \pt.%
    \]%
    Explicitly, we have
    \begin{align*}
        R^{-1}(V) &\defeq V\procirc R\\
                  &\defeq \int^{b\in Y}V^{-_{1}}_{b}\times R^{b}_{-_{2}}.
    \end{align*}
\end{remark}
\begin{Proof}{Proof of \cref{unwinding-the-weak-inverse-image-function-associated-to-a-relation}}%
    We have
    \begin{align*}
        V\procirc R%
        &\defeq%
        \int^{b\in Y}V^{-_{1}}_{b}\times R^{b}_{-_{2}}\\
        &=%
        \{%
            a\in X%
            \ \middle|\ %
            \int^{b\in Y}V^{\point}_{b}\times R^{b}_{a}=\true%
        \}%
        \\
        &=
        \{%
            a\in X%
            \ \middle|\ %
            \begin{aligned}
                &\text{there exists $b\in Y$ such that the}\\[-2.5pt]
                &\text{following conditions hold:}\\[7.5pt]
                &\mspace{25mu}\rlap{\text{1.}}\mspace{22.5mu}\text{We have $V^{\point}_{b}=\true$}\\%
                &\mspace{25mu}\rlap{\text{2.}}\mspace{22.5mu}\text{We have $R^{b}_{a}=\true$}\\[10pt]%
            \end{aligned}
        \}\\%
        &=
        \{%
            a\in X%
            \ \middle|\ %
            \begin{aligned}
                &\text{there exists $b\in Y$ such that the}\\[-2.5pt]
                &\text{following conditions hold:}\\[7.5pt]
                &\mspace{25mu}\rlap{\text{1.}}\mspace{22.5mu}\text{We have $b\in V$}\\%
                &\mspace{25mu}\rlap{\text{2.}}\mspace{22.5mu}\text{We have $b\in R(a)$}\\[10pt]%
            \end{aligned}
        \}\\%
        &=
        \{%
            a\in X%
            \ \middle|\ %
            \text{there exists $b\in V$ such that $b\in R(a)$}%
        \}\\%
        &=
        \{%
            a\in X%
            \ \middle|\ %
            R(a)\cap V\neq\emptyset%
        \}\\%
        &\defeq R^{-1}(V)%
    \end{align*}
    This finishes the proof.
\end{Proof}
\begin{proposition}{Properties of Weak Inverse Image Functions}{properties-of-weak-inverse-image-functions-associated-to-relations}%
    Let $R\colon X\rightproarrow Y$ be a relation.
    \begin{enumerate}
        \item\label{properties-of-weak-inverse-image-functions-associated-to-relations-functoriality}\SloganFont{Functoriality. }The assignment $V\mapsto R^{-1}(V)$ defines a functor
            \[
                R^{-1}%
                \colon%
                (\mathcal{P}(Y),\subset)%
                \to%
                (\mathcal{P}(X),\subset)%
            \]%
            where
            \begin{itemize}
                \item\SloganFont{Action on Objects. }For each $V\in\mathcal{P}(Y)$, we have
                    \[
                        [R^{-1}](V)%
                        \defeq%
                        R^{-1}(V).
                    \]%
                \item\SloganFont{Action on Morphisms. }For each $U,V\in\mathcal{P}(Y)$:
                    \begin{itemize}
                        \item If $U\subset V$, then $R^{-1}(U)\subset R^{-1}(V)$.
                    \end{itemize}
            \end{itemize}
        \item\label{properties-of-weak-inverse-image-functions-associated-to-relations-adjointness}\SloganFont{Adjointness. }We have an adjunction
            \begin{webcompile}
                \Adjunction#R^{-1}#R_{*}#\mathcal{P}(Y)#\mathcal{P}(X),#
            \end{webcompile}
            witnessed by a bijections of sets
            \[%
                \Hom_{\mathcal{P}(X)}(R^{-1}(U),V)%
                \cong%
                \Hom_{\mathcal{P}(X)}(U,R_{*}(V)),%
            \]%
            natural in $U\in\mathcal{P}(X)$ and $V\in\mathcal{P}(Y)$, i.e.\ such that:
            \begin{itemize}
                \itemstar The following conditions are equivalent:
                    \begin{itemize}
                        \item We have $R^{-1}(U)\subset V$.
                        \item We have $U\subset R_{*}(V)$.
                    \end{itemize}
            \end{itemize}
        \item\label{properties-of-weak-inverse-image-functions-associated-to-relations-preservation-of-colimits}\SloganFont{Preservation of Colimits. }We have an equality of sets
            \[
                R^{-1}(\bigcup_{i\in I}U_{i})%
                =%
                \bigcup_{i\in I}R^{-1}(U_{i}),%
            \]%
            natural in $\{U_{i}\}_{i\in I}\in\mathcal{P}(Y)^{\times I}$. In particular, we have equalities%
            \[
                \begin{gathered}
                    R^{-1}(U)\cup R^{-1}(V) = R^{-1}(U\cup V),\\
                    R^{-1}(\emptyset)       = \emptyset,
                \end{gathered}
            \]%
            natural in $U,V\in\mathcal{P}(Y)$.
        \item\label{properties-of-weak-inverse-image-functions-associated-to-relations-oplax-preservation-of-limits}\SloganFont{Oplax Preservation of Limits. }We have an inclusion of sets
            \[
                R^{-1}(\bigcap_{i\in I}U_{i})%
                \subset%
                \bigcap_{i\in I}R^{-1}(U_{i}),%
            \]%
            natural in $\{U_{i}\}_{i\in I}\in\mathcal{P}(Y)^{\times I}$. In particular, we have inclusions%
            \[
                \begin{gathered}
                    R^{-1}(U\cap V) \subset R^{-1}(U)\cap R^{-1}(V),\\
                    R^{-1}(X)       \subset Y,
                \end{gathered}
            \]%
            natural in $U,V\in\mathcal{P}(Y)$.
        \item\label{properties-of-weak-inverse-image-functions-associated-to-relations-symmetric-strict-monoidality-with-respect-to-unions}\SloganFont{Symmetric Strict Monoidality With Respect to Unions. }The direct image function of \cref{properties-of-weak-inverse-image-functions-associated-to-relations-functoriality} has a symmetric strict monoidal structure
            \[
                (R^{-1},R^{-1,\otimes},R^{-1,\otimes}_{\Unit})
                \colon
                (\mathcal{P}(X),\cup,\emptyset)
                \to
                (\mathcal{P}(Y),\cup,\emptyset),
            \]%
            being equipped with equalities%
            \[
                \begin{gathered}
                    R^{-1,\otimes}_{U,V}   \colon R^{-1}(U)\cup R^{-1}(V) \rightequalsarrow R^{-1}(U\cup V),\\
                    R^{-1,\otimes}_{\Unit} \colon \emptyset               \rightequalsarrow \emptyset,
                \end{gathered}
            \]%
            natural in $U,V\in\mathcal{P}(Y)$.
        \item\label{properties-of-weak-inverse-image-functions-associated-to-relations-symmetric-oplax-monoidality-with-respect-to-intersections}\SloganFont{Symmetric Oplax Monoidality With Respect to Intersections. }The direct image function of \cref{properties-of-weak-inverse-image-functions-associated-to-relations-functoriality} has a symmetric oplax monoidal structure
            \[
                (R^{-1},R^{-1,\otimes},R^{-1,\otimes}_{\Unit})
                \colon
                (\mathcal{P}(X),\cap,X)
                \to
                (\mathcal{P}(Y),\cap,Y),
            \]%
            being equipped with inclusions%
            \[
                \begin{gathered}
                    R^{-1,\otimes}_{U,V}   \colon R^{-1}(U\cap V) \subset R^{-1}(U)\cap R^{-1}(V),\\
                    R^{-1,\otimes}_{\Unit} \colon R^{-1}(X)       \subset Y,
                \end{gathered}
            \]%
            natural in $U,V\in\mathcal{P}(Y)$.
        \item\label{properties-of-weak-inverse-image-functions-associated-to-relations-interaction-with-strong-inverse-images-1}\SloganFont{Interaction With Strong Inverse Images \rmI. }We have
            \[
                R^{-1}(V)%
                =%
                X\setminus R_{-1}(Y\setminus V)
            \]%
            for each $V\in\mathcal{P}(Y)$.
        \item\label{properties-of-weak-inverse-image-functions-associated-to-relations-interaction-with-strong-inverse-images-2}\SloganFont{Interaction With Strong Inverse Images \rmII. }Let $R\colon X\rightproarrow Y$ be a relation from $X$ to $Y$.
            \begin{enumerate}
                \item\label{properties-of-weak-inverse-image-functions-associated-to-relations-interaction-with-strong-inverse-images-2-a}If $R$ is a total relation, then we have an inclusion of sets
                    \[
                        R_{-1}(V)
                        \subset
                        R^{-1}(V)
                    \]%
                    natural in $V\in\mathcal{P}(Y)$.
                \item\label{properties-of-weak-inverse-image-functions-associated-to-relations-interaction-with-strong-inverse-images-2-b}If $R$ is total and functional, then the above inclusion is in fact an equality.
                \item\label{properties-of-weak-inverse-image-functions-associated-to-relations-interaction-with-strong-inverse-images-2-c}Conversely, if we have $R_{-1}=R^{-1}$, then $R$ is total and functional.
            \end{enumerate}
    \end{enumerate}
\end{proposition}
\begin{Proof}{Proof of \cref{properties-of-weak-inverse-image-functions-associated-to-relations}}%
    \FirstProofBox{\cref{properties-of-weak-inverse-image-functions-associated-to-relations-functoriality}: Functoriality}%
    Clear.

    \ProofBox{\cref{properties-of-weak-inverse-image-functions-associated-to-relations-adjointness}: Adjointness}%
    This follows from \ChapterRef{\ChapterKanExtensions, \cref{kan-extensions:properties-of-kan-extensions-triple-adjointness} of \cref{kan-extensions:properties-of-kan-extensions}}{\cref{properties-of-kan-extensions-triple-adjointness} of \cref{properties-of-kan-extensions}}.

    \ProofBox{\cref{properties-of-weak-inverse-image-functions-associated-to-relations-preservation-of-colimits}: Preservation of Colimits}%
    This follows from \cref{properties-of-weak-inverse-image-functions-associated-to-relations-adjointness} and \ChapterRef{\ChapterAdjunctionsAndTheYonedaLemma, \cref{adjunctions-and-the-yoneda-lemma:properties-of-adjunctions-interaction-of-co-limits} of \cref{adjunctions-and-the-yoneda-lemma:properties-of-adjunctions}}{\cref{properties-of-adjunctions-interaction-of-co-limits} of \cref{properties-of-adjunctions}}.

    \ProofBox{\cref{properties-of-weak-inverse-image-functions-associated-to-relations-oplax-preservation-of-limits}: Oplax Preservation of Limits}%
    Omitted.

    \ProofBox{\cref{properties-of-weak-inverse-image-functions-associated-to-relations-symmetric-strict-monoidality-with-respect-to-unions}: Symmetric Strict Monoidality With Respect to Unions}%
    This follows from \cref{properties-of-weak-inverse-image-functions-associated-to-relations-preservation-of-colimits}.

    \ProofBox{\cref{properties-of-weak-inverse-image-functions-associated-to-relations-symmetric-oplax-monoidality-with-respect-to-intersections}: Symmetric Oplax Monoidality With Respect to Intersections}%
    This follows from \cref{properties-of-weak-inverse-image-functions-associated-to-relations-oplax-preservation-of-limits}.

    \ProofBox{\cref{properties-of-weak-inverse-image-functions-associated-to-relations-interaction-with-strong-inverse-images-1}: Interaction With Strong Inverse Images \rmI}%
    This follows from \cref{properties-of-strong-inverse-image-functions-associated-to-relations-interaction-with-weak-inverse-images-1} of \cref{properties-of-strong-inverse-image-functions-associated-to-relations}.

    \ProofBox{\cref{properties-of-weak-inverse-image-functions-associated-to-relations-interaction-with-strong-inverse-images-2}: Interaction With Strong Inverse Images \rmII}%
    This was proved in \cref{properties-of-strong-inverse-image-functions-associated-to-relations-interaction-with-weak-inverse-images-2} of \cref{properties-of-strong-inverse-image-functions-associated-to-relations}.
\end{Proof}
\begin{proposition}{Properties of the Weak Inverse Image Function Operation}{properties-of-the-weak-inverse-image-function-operation-for-relations}%
    Let $R\colon X\rightproarrow Y$ be a relation.
    \begin{enumerate}
        \item\label{properties-of-the-weak-inverse-image-function-operation-for-relations-functionality-1}\SloganFont{Functionality \rmI. }The assignment $R\mapsto R^{-1}$ defines a function
            \[
                (-)^{-1}%
                \colon%
                \Rel(X,Y)
                \to%
                \Sets(\mathcal{P}(X),\mathcal{P}(Y)).
            \]%
        \item\label{properties-of-the-weak-inverse-image-function-operation-for-relations-functionality-2}\SloganFont{Functionality \rmII. }The assignment $R\mapsto R^{-1}$ defines a function
            \[
                (-)^{-1}%
                \colon%
                \Rel(X,Y)
                \to%
                \Pos((\mathcal{P}(X),\subset),(\mathcal{P}(Y),\subset)).
            \]%
        \item\label{properties-of-the-weak-inverse-image-function-operation-for-relations-interaction-with-identities}\SloganFont{Interaction With Identities. }For each $X\in\Obj(\Sets)$, we have%
            %--- Begin Footnote ---%
            \footnote{%
                That is, the postcomposition
                \[
                    (\chi_{X})^{-1}%
                    \colon%
                    \Rel(\pt,X)%
                    \to%
                    \Rel(\pt,X)%
                \]%
                is equal to $\id_{\Rel(\pt,X)}$.
            }%
            %---  End Footnote  ---%
            \[
                (\chi_{X})^{-1}%
                =%
                \id_{\mathcal{P}(X)}.%
            \]%
        \item\label{properties-of-the-weak-inverse-image-function-operation-for-relations-interaction-with-composition}\SloganFont{Interaction With Composition. }For each pair of composable relations $R\colon X\rightproarrow Y$ and $S\colon Y\rightproarrow C$, we have%
            %--- Begin Footnote ---%
            \footnote{%
                That is, we have
                \begin{webcompile}
                    (S\procirc R)^{-1}%
                    =%
                    R^{-1}\circ S^{-1},%
                    \quad
                    \begin{tikzcd}[row sep={5.0*\the\DL,between origins}, column sep={6.5*\the\DL,between origins}, background color=backgroundColor, ampersand replacement=\&]
                        {\Rel(\pt,C)}
                        \arrow[r,"R^{-1}"]
                        \arrow[rd,"{(S\procirc R)^{-1}}"']
                        \&
                        {\Rel(\pt,Y)}
                        \arrow[d,"S^{-1}"]
                        \\
                        \&
                        {\Rel(\pt,X)}\mrp{.}
                    \end{tikzcd}
                \end{webcompile}
                \par\vspace*{\TCBBoxCorrection}
            }%
            %---  End Footnote  ---%
            \begin{webcompile}
                (S\procirc R)^{-1}%
                =%
                R^{-1}\circ S^{-1},%
                \quad
                \begin{tikzcd}[row sep={5.0*\the\DL,between origins}, column sep={5.0*\the\DL,between origins}, background color=backgroundColor, ampersand replacement=\&]
                    \mathcal{P}(C)
                    \arrow[r,"S^{-1}"]
                    \arrow[rd,"{(S\procirc R)^{-1}}"']
                    \&
                    \mathcal{P}(Y)
                    \arrow[d,"R^{-1}"]
                    \\
                    \&
                    \mathcal{P}(X)\mrp{.}
                \end{tikzcd}
            \end{webcompile}
        %\item\label{properties-of-the-weak-inverse-image-function-operation-for-relations-}\SloganFont{. }
    \end{enumerate}
\end{proposition}
\begin{Proof}{Proof of \cref{properties-of-the-weak-inverse-image-function-operation-for-relations}}%
    \FirstProofBox{\cref{properties-of-the-weak-inverse-image-function-operation-for-relations-functionality-1}: Functionality \rmI}%
    Clear.

    \ProofBox{\cref{properties-of-the-weak-inverse-image-function-operation-for-relations-functionality-2}: Functionality \rmII}%
    Clear.

    \ProofBox{\cref{properties-of-the-weak-inverse-image-function-operation-for-relations-interaction-with-identities}: Interaction With Identities}%
    This follows from \ChapterRef{\ChapterCategories, \cref{categories:properties-of-pre-postcomposition-interaction-with-identities} of \cref{categories:properties-of-pre-postcomposition}}{\cref{properties-of-pre-postcomposition-interaction-with-identities} of \cref{properties-of-pre-postcomposition}}.
    %Indeed, we have
    %\begin{align*}
    %    (\chi_{X})^{-1}(U) &\defeq \{a\in X\ \middle|\ \{a\}\cap U\neq\emptyset\}\\
    %                       &=      U\\
    %                       &\defeq \id_{\mathcal{P}(X)}(U)
    %\end{align*}
    %for each $U\in\mathcal{P}(X)$. Thus $(\chi_{X})^{-1}=\id_{\mathcal{P}(X)}$.

    \ProofBox{\cref{properties-of-the-weak-inverse-image-function-operation-for-relations-interaction-with-composition}: Interaction With Composition}%
    This follows from \ChapterRef{\ChapterCategories, \cref{categories:properties-of-pre-postcomposition-interaction-with-composition-1} of \cref{categories:properties-of-pre-postcomposition}}{\cref{properties-of-pre-postcomposition-interaction-with-composition-1} of \cref{properties-of-pre-postcomposition}}.
    %Indeed, we have
    %\begin{align*}
    %    (S\procirc R)^{-1}(U) &\defeq \{a\in X\ \middle|\ S(R(a))\cap U\neq\emptyset\}\\
    %                          &=      ?\\
    %                          &=      \{a\in X\ \middle|\ R(a)\cap\{b\in Y\ \middle|\ S(b)\cap U\neq\emptyset\}\neq\emptyset\}\\
    %                          &=      R^{-1}(\{b\in Y\ \middle|\ S(b)\cap U\neq\emptyset\})\\
    %                          &\defeq R^{-1}(S^{-1}(U))\\
    %                          &\defeq [R^{-1}\circ S^{-1}](U)
    %\end{align*}
    %for each $U\in\mathcal{P}(X)$. Thus $(S\procirc R)^{-1}=S^{-1}\circ R^{-1}$.
\end{Proof}
\subsection{Codirect Images}\label{subsection-codirect-images-relations}
Let $X$ and $Y$ be sets and let $R\colon X\rightproarrow Y$ be a relation.
\begin{definition}{Codirect Images}{the-codirect-image-function-associated-to-a-relation}%
    The \index[set-theory]{relation!associated codirect image function}\textbf{codirect image function associated to $R$} is the function\index[notation]{Rstar@$R_{*}$}%
    \[%
        R_{*}%
        \colon%
        \mathcal{P}(X)%
        \to%
        \mathcal{P}(Y)%
    \]%
    defined by\index[notation]{RstarU@$R_{*}(U)$}%
    %--- Begin Footnote ---%
    \footnote{%
        \SloganFont{Further Terminology: }The set $R_{*}(U)$ is called the \textbf{codirect image of $U$ by $R$}.
    }%
    %---  End Footnote  ---%
    %--- Begin Footnote ---%
    \footnote{%
        We also have
        \[
            R_{*}(U)%
            =%
            Y\setminus R_{!}(X\setminus U);
        \]%
        see \cref{properties-of-codirect-image-functions-associated-to-relations-relation-to-direct-images} of \cref{properties-of-codirect-image-functions-associated-to-relations}.
        \par\vspace*{\TCBBoxCorrection}
    }%
    %---  End Footnote  ---%
    \begin{align*}
        R_{*}(U) &\defeq \{%
                             b\in Y%
                             \ \middle|\ %
                             \begin{aligned}
                                 &\text{for each $a\in X$, if we have}\\%
                                 &\text{$b\in R(a)$, then $a\in U$}%
                             \end{aligned}
                         \}\\%
                 &=      \{%
                             b\in Y%
                             \ \middle|\ %
                             R^{-1}(b)\subset U%
                         \}%
    \end{align*}
    for each $U\in\mathcal{P}(X)$.
\end{definition}
\begin{remark}{Unwinding \cref{the-codirect-image-function-associated-to-a-relation}}{unwinding-the-codirect-image-function-associated-to-a-relation}%
    Identifying subsets of $Y$ with relations from $\pt$ to $Y$ via \ChapterRef{\ChapterConstructionsWithSets, \cref{constructions-with-sets:elementary-properties-of-powersets-powersets-as-sets-of-relations} of \cref{constructions-with-sets:elementary-properties-of-powersets}}{\cref{elementary-properties-of-powersets-powersets-as-sets-of-relations} of \cref{elementary-properties-of-powersets}}, we see that the codirect image function associated to $R$ is equivalently the function
    \[
        R_{*}%
        \colon%
        \underbrace{\mathcal{P}(X)}_{\cong\Rel(X,\pt)}%
        \to%
        \underbrace{\mathcal{P}(Y)}_{\cong\Rel(Y,\pt)}
    \]%
    defined by
    \begin{webcompile}
        R_{*}(U)%
        \defeq%
        \Ran_{R}(U),%
        \quad
        \begin{tikzcd}[row sep={5.0*\the\DL,between origins}, column sep={5.0*\the\DL,between origins}, background color=backgroundColor, ampersand replacement=\&]
            \&
            Y
            \arrow[d, "\Ran_{R}(U)",dashed,mid vert]
            \\
            X
            \arrow[ru,mid vert, "R"]
            \arrow[r,mid vert,"U"'{name=F}]
            \&
            \pt\mrp{,}%
            % 2-Arrows
            \arrow[from=1-2,to=F,Rightarrow,shorten=0.5em,pos=0.5]
        \end{tikzcd}
    \end{webcompile}
    being explicitly computed by
    \begin{align*}
        R^{*}(U) &\defeq \Ran_{R}(U)\\%
                 &\cong  \int_{a\in X}\Hom_{\TTV}(R^{-_{2}}_{a},U^{-_{1}}_{a}),
    \end{align*}
    where we have used \cref{existence-of-right-kan-extensions-in-rel}.
\end{remark}
\begin{Proof}{Proof of \cref{unwinding-the-codirect-image-function-associated-to-a-relation}}%
    We have
    \begin{envsmallsize}
        \begin{align*}
            \Ran_{R}(V)%
            &\cong%
            \int_{a\in X}\Hom_{\TTV}(R^{-_{2}}_{a},U^{-_{1}}_{a})\\%
            &=%
            \{%
                b\in Y%
                \ \middle|\ %
                \int_{a\in X}\Hom_{\TTV}(R^{b}_{a},U^{\point}_{a})=\true%
            \}%
            \\
            &=
            \{%
                b\in Y%
                \ \middle|\ %
                \begin{aligned}
                    &\text{for each $a\in X$, at least one of the}\\[-2.5pt]
                    &\text{following conditions hold:}\\[7.5pt]
                    &\mspace{25mu}\rlap{\text{1.}}\mspace{22.5mu}\text{We have $R^{b}_{a}=\false$}\\%
                    &\mspace{25mu}\rlap{\text{2.}}\mspace{22.5mu}\text{The following conditions hold:}\\[7.5pt]%
                    &\mspace{50mu}\rlap{\text{\norg(a\norg)}}\mspace{30mu}\text{We have $R^{b}_{a}=\true$}\\
                    &\mspace{50mu}\rlap{\text{\norg(b\norg)}}\mspace{30mu}\text{We have $U^{\point}_{a}=\true$}\\[10pt]
                \end{aligned}
            \}\\%
            &=
            \{%
                b\in Y%
                \ \middle|\ %
                \begin{aligned}
                    &\text{for each $a\in X$, at least one of the}\\[-2.5pt]
                    &\text{following conditions hold:}\\[7.5pt]
                    &\mspace{25mu}\rlap{\text{1.}}\mspace{22.5mu}\text{We have $b\nin R(X)$}\\%
                    &\mspace{25mu}\rlap{\text{2.}}\mspace{22.5mu}\text{The following conditions hold:}\\[7.5pt]%
                    &\mspace{50mu}\rlap{\text{\norg(a\norg)}}\mspace{30mu}\text{We have $b\in R(a)$}\\
                    &\mspace{50mu}\rlap{\text{\norg(b\norg)}}\mspace{30mu}\text{We have $a\in U$}\\[10pt]
                \end{aligned}
            \}\\%
            &=
            \{%
                b\in Y%
                \ \middle|\ %
                 \begin{aligned}
                     &\text{for each $a\in X$, if we have}\\%
                     &\text{$b\in R(a)$, then $a\in U$}%
                 \end{aligned}
            \}\\%
            &=
            \{%
                b\in Y%
                \ \middle|\ %
                R^{-1}(b)\subset U%
            \}\\%
            &\defeq%
            R^{-1}(U).%
        \end{align*}
    \end{envsmallsize}
    This finishes the proof.
\end{Proof}
\begin{proposition}{Properties of Codirect Images}{properties-of-codirect-image-functions-associated-to-relations}%
    Let $R\colon X\rightproarrow Y$ be a relation.
    \begin{enumerate}
        \item\label{properties-of-codirect-image-functions-associated-to-relations-functoriality}\SloganFont{Functoriality. }The assignment $U\mapsto R_{*}(U)$ defines a functor
            \[
                R_{*}%
                \colon%
                (\mathcal{P}(X),\subset)%
                \to%
                (\mathcal{P}(Y),\subset)%
            \]%
            where
            \begin{itemize}
                \item\SloganFont{Action on Objects. }For each $U\in\mathcal{P}(X)$, we have
                    \[
                        [R_{*}](U)%
                        \defeq%
                        R_{*}(U).
                    \]%
                \item\SloganFont{Action on Morphisms. }For each $U,V\in\mathcal{P}(X)$:
                    \begin{itemize}
                        \item If $U\subset V$, then $R_{*}(U)\subset R_{*}(V)$.
                    \end{itemize}
            \end{itemize}
        \item\label{properties-of-codirect-image-functions-associated-to-relations-adjointness}\SloganFont{Adjointness. }We have an adjunction
            \begin{webcompile}
                \Adjunction#R^{-1}#R_{*}#\mathcal{P}(Y)#\mathcal{P}(X),#
            \end{webcompile}
            witnessed by a bijections of sets
            \[%
                \Hom_{\mathcal{P}(X)}(R^{-1}(U),V)%
                \cong%
                \Hom_{\mathcal{P}(X)}(U,R_{*}(V)),%
            \]%
            natural in $U\in\mathcal{P}(X)$ and $V\in\mathcal{P}(Y)$, i.e.\ such that:
            \begin{itemize}
                \itemstar The following conditions are equivalent:
                    \begin{itemize}
                        \item We have $R^{-1}(U)\subset V$.
                        \item We have $U\subset R_{*}(V)$.
                    \end{itemize}
            \end{itemize}
        \item\label{properties-of-codirect-image-functions-associated-to-relations-lax-preservation-of-colimits}\SloganFont{Lax Preservation of Colimits. }We have an inclusion of sets
            \[
                \bigcup_{i\in I}R_{*}(U_{i})%
                \subset%
                R_{*}(\bigcup_{i\in I}U_{i}),%
            \]%
            natural in $\{U_{i}\}_{i\in I}\in\mathcal{P}(X)^{\times I}$. In particular, we have inclusions%
            \[
                \begin{gathered}
                    R_{*}(U)\cup R_{*}(V) \subset R_{*}(U\cup V),\\
                    \emptyset             \subset R_{*}(\emptyset),
                \end{gathered}
            \]%
            natural in $U,V\in\mathcal{P}(X)$.
        \item\label{properties-of-codirect-image-functions-associated-to-relations-preservation-of-limits}\SloganFont{Preservation of Limits. }We have an equality of sets
            \[
                    R_{*}(\bigcap_{i\in I}U_{i})%
                    =%
                    \bigcap_{i\in I}R_{*}(U_{i}),%
            \]%
            natural in $\{U_{i}\}_{i\in I}\in\mathcal{P}(X)^{\times I}$. In particular, we have equalities%
            \[
                \begin{gathered}
                    R_{*}(U\cap V) = R_{*}(U)\cap R_{*}(V),\\
                    R_{*}(X)       = Y,
                \end{gathered}
            \]%
            natural in $U,V\in\mathcal{P}(X)$.
        \item\label{properties-of-codirect-image-functions-associated-to-relations-symmetric-lax-monoidality-with-respect-to-unions}\SloganFont{Symmetric Lax Monoidality With Respect to Unions. }The codirect image function of \cref{properties-of-codirect-image-functions-associated-to-relations-functoriality} has a symmetric lax monoidal structure
            \[
                (R_{*},R^{\otimes}_{*},R^{\otimes}_{!|\Unit})
                \colon
                (\mathcal{P}(X),\cup,\emptyset)
                \to
                (\mathcal{P}(Y),\cup,\emptyset),
            \]%
            being equipped with inclusions%
            \[
                \begin{gathered}
                    R^{\otimes}_{!|U,V}   \colon R_{*}(U)\cup R_{*}(V) \subset R_{*}(U\cup V),\\
                    R^{\otimes}_{!|\Unit} \colon \emptyset               \subset R_{*}(\emptyset),
                \end{gathered}
            \]%
            natural in $U,V\in\mathcal{P}(X)$.
        \item\label{properties-of-codirect-image-functions-associated-to-relations-symmetric-strict-monoidality-with-respect-to-intersections}\SloganFont{Symmetric Strict Monoidality With Respect to Intersections. }The direct image function of \cref{properties-of-codirect-image-functions-associated-to-relations-functoriality} has a symmetric strict monoidal structure
            \[
                (R_{*},R^{\otimes}_{*},R^{\otimes}_{!|\Unit})
                \colon
                (\mathcal{P}(X),\cap,X)
                \to
                (\mathcal{P}(Y),\cap,Y),
            \]%
            being equipped with equalities%
            \[
                \begin{gathered}
                    R^{\otimes}_{!|U,V}   \colon R_{*}(U\cap V) \rightequalsarrow R_{*}(U)\cap R_{*}(V),\\
                    R^{\otimes}_{!|\Unit} \colon R_{*}(X)       \rightequalsarrow Y,
                \end{gathered}
            \]%
            natural in $U,V\in\mathcal{P}(X)$.
        \item\label{properties-of-codirect-image-functions-associated-to-relations-relation-to-direct-images}\SloganFont{Relation to Direct Images. }We have
            \[
                R_{*}(U)%
                =%
                Y\setminus R_{!}(X\setminus U)
            \]%
            for each $U\in\mathcal{P}(X)$.
        %\item\label{properties-of-codirect-image-functions-associated-to-relations-}\SloganFont{. }
    \end{enumerate}
\end{proposition}
\begin{Proof}{Proof of \cref{properties-of-codirect-image-functions-associated-to-relations}}%
    \FirstProofBox{\cref{properties-of-codirect-image-functions-associated-to-relations-functoriality}: Functoriality}%
    Clear.

    \ProofBox{\cref{properties-of-codirect-image-functions-associated-to-relations-adjointness}: Adjointness}%
    This follows from \ChapterRef{\ChapterKanExtensions, \cref{kan-extensions:properties-of-kan-extensions-triple-adjointness} of \cref{kan-extensions:properties-of-kan-extensions}}{\cref{properties-of-kan-extensions-triple-adjointness} of \cref{properties-of-kan-extensions}}.

    \ProofBox{\cref{properties-of-codirect-image-functions-associated-to-relations-lax-preservation-of-colimits}: Lax Preservation of Colimits}%
    Omitted.

    \ProofBox{\cref{properties-of-codirect-image-functions-associated-to-relations-preservation-of-limits}: Preservation of Limits}%
    This follows from \cref{properties-of-codirect-image-functions-associated-to-relations-adjointness} and \ChapterRef{\ChapterAdjunctionsAndTheYonedaLemma, \cref{adjunctions-and-the-yoneda-lemma:properties-of-adjunctions-interaction-of-co-limits} of \cref{adjunctions-and-the-yoneda-lemma:properties-of-adjunctions}}{\cref{properties-of-adjunctions-interaction-of-co-limits} of \cref{properties-of-adjunctions}}.

    \ProofBox{\cref{properties-of-codirect-image-functions-associated-to-relations-symmetric-lax-monoidality-with-respect-to-unions}: Symmetric Lax Monoidality With Respect to Unions}%
    This follows from \cref{properties-of-codirect-image-functions-associated-to-relations-lax-preservation-of-colimits}.

    \ProofBox{\cref{properties-of-codirect-image-functions-associated-to-relations-symmetric-strict-monoidality-with-respect-to-intersections}: Symmetric Strict Monoidality With Respect to Intersections}%
    This follows from \cref{properties-of-codirect-image-functions-associated-to-relations-preservation-of-limits}.

    \ProofBox{\cref{properties-of-codirect-image-functions-associated-to-relations-relation-to-direct-images}: Relation to Direct Images}%
    This follows from \cref{properties-of-direct-image-functions-associated-to-relations-relation-to-codirect-images} of \cref{properties-of-direct-image-functions-associated-to-relations}. Alternatively, we may prove it directly as follows, with the proof proceeding in the same way as in the case of functions (\ChapterRef{\ChapterConstructionsWithSets, \cref{constructions-with-sets:properties-of-codirect-images-i-relation-to-direct-images} of \cref{constructions-with-sets:properties-of-codirect-images-i}}{\cref{properties-of-codirect-images-i-relation-to-direct-images} of \cref{properties-of-codirect-images-i}}).

    We claim that $R_{*}(U)=Y\setminus R_{!}(X\setminus U)$:
    \begin{itemize}
        \item\SloganFont{The First Implication. }We claim that
            \[
                R_{*}(U)%
                \subset%
                Y\setminus R_{!}(X\setminus U).%
            \]%
            Let $b\in R_{*}(U)$. We need to show that $b\nin R_{!}(X\setminus U)$, i.e.\ that there is no $a\in X\setminus U$ such that $b\in R(a)$.

            This is indeed the case, as otherwise we would have $a\in R^{-1}(b)$ and $a\nin U$, contradicting $R^{-1}(b)\subset U$ (which holds since $b\in R_{*}(U)$).

            Thus $b\in Y\setminus R_{!}(X\setminus U)$.
        \item\SloganFont{The Second Implication. }We claim that
            \[
                Y\setminus R_{!}(X\setminus U)%
                \subset%
                R_{*}(U).%
            \]%
            Let $b\in Y\setminus R_{!}(X\setminus U)$. We need to show that $b\in R_{*}(U)$, i.e.\ that $R^{-1}(b)\subset U$.

            Since $b\nin R_{!}(X\setminus U)$, there exists no $a\in X\setminus U$ such that $b\in R(a)$, and hence $R^{-1}(b)\subset U$.

            Thus $b\in R_{*}(U)$.
    \end{itemize}
    This finishes the proof.
\end{Proof}
\begin{proposition}{Properties of the Codirect Image Function Operation}{properties-of-the-codirect-image-function-operation-for-relations}%
    Let $R\colon X\rightproarrow Y$ be a relation.
    \begin{enumerate}
        \item\label{properties-of-the-codirect-image-function-operation-for-relations-functionality-1}\SloganFont{Functionality \rmI. }The assignment $R\mapsto R_{*}$ defines a function
            \[
                (-)_{*}%
                \colon%
                \Sets(X,Y)
                \to%
                \Sets(\mathcal{P}(X),\mathcal{P}(Y)).
            \]%
        \item\label{properties-of-the-codirect-image-function-operation-for-relations-functionality-2}\SloganFont{Functionality \rmII. }The assignment $R\mapsto R_{*}$ defines a function
            \[
                (-)_{*}%
                \colon%
                \Sets(X,Y)
                \to%
                \Hom_{\Pos}((\mathcal{P}(X),\subset),(\mathcal{P}(Y),\subset)).
            \]%
        \item\label{properties-of-the-codirect-image-function-operation-for-relations-interaction-with-identities}\SloganFont{Interaction With Identities. }For each $X\in\Obj(\Sets)$, we have
            \[
                (\id_{X})_{*}%
                =%
                \id_{\mathcal{P}(X)}.%
            \]%
        \item\label{properties-of-the-codirect-image-function-operation-for-relations-interaction-with-composition}\SloganFont{Interaction With Composition. }For each pair of composable relations $R\colon X\rightproarrow Y$ and $S\colon Y\rightproarrow C$, we have%
            \begin{webcompile}
                (S\procirc R)_{*}%
                =%
                S_{*}\circ R_{*},%
                \quad
                \begin{tikzcd}[row sep={5.0*\the\DL,between origins}, column sep={5.0*\the\DL,between origins}, background color=backgroundColor, ampersand replacement=\&]
                    \mathcal{P}(X)
                    \arrow[r,"R_{*}"]
                    \arrow[rd,"{(S\procirc R)_{*}}"']
                    \&
                    \mathcal{P}(Y)
                    \arrow[d,"S_{*}"]
                    \\
                    \&
                    \mathcal{P}(C)\mrp{.}
                \end{tikzcd}
            \end{webcompile}
        %\item\label{properties-of-the-codirect-image-function-operation-for-relations-}\SloganFont{. }
    \end{enumerate}
\end{proposition}
\begin{Proof}{Proof of \cref{properties-of-the-codirect-image-function-operation-for-relations}}%
    \FirstProofBox{\cref{properties-of-the-codirect-image-function-operation-for-relations-functionality-1}: Functionality \rmI}%
    Clear.

    \ProofBox{\cref{properties-of-the-codirect-image-function-operation-for-relations-functionality-2}: Functionality \rmII}%
    Clear.

    \ProofBox{\cref{properties-of-the-codirect-image-function-operation-for-relations-interaction-with-identities}: Interaction With Identities}%
    Indeed, we have
    \begin{align*}
        (\chi_{X})_{*}(U) &\defeq \{a\in X\ \middle|\ \chi^{-1}_{X}(a)\subset U\}\\
                          &\defeq \{a\in X\ \middle|\ \{a\}\subset U\}\\
                          &=      U
    \end{align*}
    for each $U\in\mathcal{P}(X)$. Thus $(\chi_{X})_{*}=\id_{\mathcal{P}(X)}$.

    \ProofBox{\cref{properties-of-the-codirect-image-function-operation-for-relations-interaction-with-composition}: Interaction With Composition}%
    Indeed, we have
    \begin{align*}
        (S\procirc R)_{*}(U) &\defeq \{c\in C\ \middle|\ [S\procirc R]^{-1}(c)\subset U\}\\
                             &\defeq \{c\in C\ \middle|\ S^{-1}(R^{-1}(c))\subset U\}\\
                             &=      \{c\in C\ \middle|\ R^{-1}(c)\subset S_{*}(U)\}\\
                             &\defeq R_{*}(S_{*}(U))\\
                             &\defeq [R_{*}\circ S_{*}](U)
    \end{align*}
    for each $U\in\mathcal{P}(C)$, where we used \cref{properties-of-codirect-image-functions-associated-to-relations-adjointness} of \cref{properties-of-codirect-image-functions-associated-to-relations}, which implies that the conditions
    \begin{itemize}
        \item We have $S^{-1}(R^{-1}(c))\subset U$.
        \item We have $R^{-1}(c)\subset S_{*}(U)$.
    \end{itemize}
    are equivalent. Thus $(S\procirc R)_{*}=S_{*}\circ R_{*}$.
\end{Proof}
\subsection{Functoriality of Powersets}\label{subsection-functoriality-of-powersets}
\begin{proposition}{Functoriality of Powersets \rmI}{functoriality-of-powersets-1}%
    The assignment $X\mapsto\mathcal{P}(X)$ defines functors\index[notation]{Pshriek@$\mathcal{P}_{"!}$}\index[notation]{Pminusone@$\mathcal{P}^{-1}$}\index[notation]{Pstar@$\mathcal{P}_{*}$}%
    %--- Begin Footnote ---%
    \footnote{%
        The functor $\mathcal{P}_{!}\colon\Rel\to\Sets$ admits a left adjoint; see \cref{properties-of-graphs-of-functions-adjointness} of \cref{properties-of-graphs-of-functions}.
        \par\vspace*{\TCBBoxCorrection}
    }%
    %---  End Footnote  ---%
    \begin{align*}
        \mathcal{P}_{!}  &\colon \Rel       \to \Sets,\\%
        \mathcal{P}_{-1} &\colon \Rel^{\op} \to \Sets,\\%
        \mathcal{P}^{-1} &\colon \Rel^{\op} \to \Sets,\\%
        \mathcal{P}_{*}  &\colon \Rel       \to \Sets%
    \end{align*}
    where
    \begin{itemize}
        \item\SloganFont{Action on Objects. }For each $X\in\Obj(\Rel)$, we have
            \begin{align*}
                \mathcal{P}_{!}(X)  &\defeq \mathcal{P}(X),\\%
                \mathcal{P}_{-1}(X) &\defeq \mathcal{P}(X),\\%
                \mathcal{P}^{-1}(X) &\defeq \mathcal{P}(X),\\%
                \mathcal{P}_{*}(X)  &\defeq \mathcal{P}(X).%
            \end{align*}
        \item\SloganFont{Action on Morphisms. }For each morphism $R\colon X\rightproarrow Y$ of $\Rel$, the images
            \begin{align*}
                \mathcal{P}_{!}(R)  &\colon \mathcal{P}(X) \to \mathcal{P}(Y),\\%
                \mathcal{P}_{-1}(R) &\colon \mathcal{P}(Y) \to \mathcal{P}(X),\\%
                \mathcal{P}^{-1}(R) &\colon \mathcal{P}(Y) \to \mathcal{P}(X),\\%
                \mathcal{P}_{*}(R)  &\colon \mathcal{P}(X) \to \mathcal{P}(Y)%
            \end{align*}
            of $R$ by $\mathcal{P}_{!}$, $\mathcal{P}_{-1}$, $\mathcal{P}^{-1}$, and $\mathcal{P}_{*}$ are defined by
            \begin{align*}
                \mathcal{P}_{!}(R)  &\defeq R_{!},\\%
                \mathcal{P}_{-1}(R) &\defeq R_{-1},\\%
                \mathcal{P}^{-1}(R) &\defeq R^{-1},\\%
                \mathcal{P}_{*}(R)  &\defeq R_{*},%
            \end{align*}
            as in \cref{the-direct-image-function-associated-to-a-relation,the-strong-inverse-image-function-associated-to-a-relation,the-weak-inverse-image-function-associated-to-a-relation,the-codirect-image-function-associated-to-a-relation}.
    \end{itemize}
\end{proposition}
\begin{Proof}{Proof of \cref{functoriality-of-powersets-1}}%
    This follows from \cref{properties-of-the-direct-image-function-operation-for-relations-interaction-with-identities,properties-of-the-direct-image-function-operation-for-relations-interaction-with-composition} of \cref{properties-of-the-direct-image-function-operation-for-relations}, \cref{properties-of-the-strong-inverse-image-function-operation-for-relations-interaction-with-identities,properties-of-the-strong-inverse-image-function-operation-for-relations-interaction-with-composition} of \cref{properties-of-the-strong-inverse-image-function-operation-for-relations}, \cref{properties-of-the-weak-inverse-image-function-operation-for-relations-interaction-with-identities,properties-of-the-weak-inverse-image-function-operation-for-relations-interaction-with-composition} of \cref{properties-of-the-weak-inverse-image-function-operation-for-relations}, and \cref{properties-of-the-codirect-image-function-operation-for-relations-interaction-with-identities,properties-of-the-codirect-image-function-operation-for-relations-interaction-with-composition} of \cref{properties-of-the-codirect-image-function-operation-for-relations}.
\end{Proof}
\subsection{Functoriality of Powersets: Relations on Powersets}\label{functoriality-of-powersets-relations-on-powersets}
Let $X$ and $Y$ be sets and let $R\colon X\rightproarrow Y$ be a relation.
\begin{definition}{The Relation on Powersets Associated to a Relation}{the-relation-on-powersets-associated-to-a-relation}%
    The \index[set-theory]{relation!on powersets associated to a relation}\textbf{relation on powersets associated to $R$} is the relation
    \[
        \mathcal{P}(R)%
        \colon%
        \mathcal{P}(X)%
        \rightproarrow
        \mathcal{P}(Y)%
    \]%
    defined by%
    %--- Begin Footnote ---%
    \footnote{%
        Illustration:%
        \[
            \begin{tikzcd}[row sep={4.0*\the\DL,between origins}, column sep={4.0*\the\DL,between origins}, background color=backgroundColor, ampersand replacement=\&]
                \pt%
                \arrow[rrr,mid vert,bend left=15,"\chi_{\pt}"]
                \arrow[r,mid vert,"U"']
                \&
                X
                \arrow[r,mid vert,"R"']
                \&
                Y
                \arrow[r,mid vert,"V"']
                \&
                \pt\mrp{.}%
            \end{tikzcd}
        \]%
        \par\vspace*{\TCBBoxCorrection}
    }%
    %---  End Footnote  ---%
    \[
        \mathcal{P}(R)^{V}_{U}%
        \defeq%
        \eRel(\chi_{\pt},V\procirc R\procirc U)%
    \]%
    for each $U\in\mathcal{P}(X)$ and each $V\in\mathcal{P}(Y)$.
\end{definition}
\begin{remark}{Unwinding \cref{the-relation-on-powersets-associated-to-a-relation}}{unwinding-the-relation-on-powersets-associated-to-a-relation}%
    In detail, we have $U\sim_{\mathcal{P}(R)}V$ \textiff the following equivalent conditions hold:
    \begin{itemize}
        \item We have $\chi_{\pt}\subset V\procirc R\procirc U$.
        \item We have $(V\procirc R\procirc U)^{\point}_{\point}=\true$, i.e.\ we have
            \[
                \int^{a\in X}\int^{b\in Y}V^{\point}_{b}\times R^{b}_{a}\times U^{a}_{\point}%
                =%
                \true.%
            \]%
        \item There exists some $a\in X$ and some $b\in Y$ such that:
            \begin{itemize}
                \item We have $U^{a}_{\point}=\true$.
                \item We have $R^{b}_{a}=\true$.
                \item We have $V^{\point}_{b}=\true$.
            \end{itemize}
        \item There exists some $a\in X$ and some $b\in Y$ such that:
            \begin{itemize}
                \item We have $a\in U$.
                \item We have $a\sim_{R}b$.
                \item We have $b\in V$.
            \end{itemize}
    \end{itemize}
\end{remark}
\begin{proposition}{Functoriality of Powersets \rmII}{functoriality-of-powersets-2}%
    The assignment $R\mapsto\mathcal{P}(R)$ defines a functor
    \[
        \mathcal{P}%
        \colon%
        \Rel%
        \to%
        \Rel.
    \]%
\end{proposition}
\begin{Proof}{Proof of \cref{functoriality-of-powersets-2}}%
    Omitted.
\end{Proof}
\section{The Left Skew Monoidal Structure on $\eRel(A,B)$}\label{section-the-left-skew-monoidal-structure-on-rel-a-b}
\subsection{The Left Skew Monoidal Product}\label{subsection-the-left-skew-monoidal-structure-on-rel-a-b-the-left-skew-monoidal-product}
Let $A$ and $B$ be sets and let $J\colon A\rightproarrow B$ be a relation.
\begin{definition}{The Left $J$-Skew Monoidal Product of $\eRel(A,B)$}{the-left-j-skew-monoidal-product-of-erelab}%
    The \textbf{left $J$-skew monoidal product of $\eRel(A,B)$} is the functor
    \[
        \lhd_{J}%
        \colon
        \eRel(A,B)\times\eRel(A,B)
        \to
        \eRel(A,B)
    \]%
    where
    \begin{itemize}
        \item\SloganFont{Action on Objects. }For each $R,S\in\Obj(\eRel(A,B))$, we have
            \begin{webcompile}
                S\lhd_{J}R%
                \defeq%
                S\procirc\Rift_{J}(R),
                \quad
                \begin{tikzcd}[row sep={5.0*\the\DL,between origins}, column sep={5.0*\the\DL,between origins}, background color=backgroundColor, ampersand replacement=\&]
                    \&
                    A
                    \arrow[r,mid vert,"S"]
                    \arrow[d,mid vert,"J"]
                    \&
                    B\mrp{.}
                    \\
                    A
                    \arrow[ru,mid vert,"{\Rift_{J}(R)}",densely dashed]
                    \arrow[r, mid vert,"R"',""'{pos=0.325,name=R}]
                    \&
                    B
                    \&
                    % 2-Arrows
                    \arrow[from=1-2,to=R,shorten <= 0.5em,shorten >= 0.75em,Rightarrow,start anchor={[xshift=0.2*\the\DL]}]%
                \end{tikzcd}
            \end{webcompile}%
        \item\SloganFont{Action on Morphisms. }For each $R,S,R',S'\in\Obj(\eRel(A,B))$, the action on $\Hom$-sets
            \begin{envscriptsize}
                \[
                    (\lhd_{J})_{(G,F),(G',F')}
                    \colon%
                    \Hom_{\eRel(A,B)}(S,S')\times\Hom_{\eRel(A,B)}(R,R')
                    \to
                    \Hom_{\eRel(A,B)}(S\lhd_{J}R,S'\lhd_{J}R')
                \]%
            \end{envscriptsize}
            of $\lhd_{J}$ at $((R,S),(R',S'))$ is defined by%
            %--- Begin Footnote ---%
            \footnote{%
                Since $\eRel(A,B)$ is posetal, this is to say that if $S\subset S'$ and $R\subset R'$, then $S\lhd_{J}R\subset S'\lhd_{J}R'$.
                \par\vspace*{\TCBBoxCorrection}
            }%
            %---  End Footnote  ---%
            \begin{webcompile}
                \beta\lhd_{J}\alpha%
                \defeq%
                \beta\procirc\Rift_{J}(\alpha),%
                \begin{tikzcd}[row sep={8.0*\the\DL,between origins}, column sep={10.0*\the\DL,between origins}, background color=backgroundColor, ampersand replacement=\&]
                    \&
                    A
                    \arrow[r,bend left  = 30,mid vert,"S"{name=S}]
                    \arrow[r,bend right = 30,mid vert,"S'"'{name=Sprime}]
                    \arrow[d,mid vert,"J"]
                    \&[-3.0*\the\DL]
                    B
                    \\
                    A
                    \arrow[ru,bend left  = 35, mid vert,"{\Rift_{J}(R)}"{sloped,name=RiftJR},densely dashed]
                    \arrow[ru,bend right = 0, mid vert,"{\Rift_{J}(R')}"'{sloped,name=RiftJRprime},densely dashed]
                    \arrow[r, mid vert=0.35,"R"{description,name=R}]
                    \arrow[r, mid vert,bend right=40,"R'"'{name=Rprime}]
                    \&
                    B
                    \&[-3.0*\the\DL]
                    % 2-Arrows
                    \arrow[from=R,to=Rprime,shorten <= -0.25em,shorten >= 0.45em,Rightarrow,"\alpha"'{pos=0.1}]%
                    \arrow[from=RiftJR,to=RiftJRprime,shorten <= 0.45em,shorten >= 0.45em,Rightarrow,"\Rift_{J}(\alpha)"'{pos=0.3,rotate=37.5}]%
                    \arrow[from=S,to=Sprime,shorten=0.45em,Rightarrow,"\beta"'{pos=0.475}]%
                    \arrow[from=1-2,to=R,shorten <= 0.5em,shorten >= -0.25em,Rightarrow,start anchor={[xshift=0.2*\the\DL]}]%
                \end{tikzcd}
            \end{webcompile}%
            for each $\beta\in\Hom_{\eRel(A,B)}(S,S')$ and each $\alpha\in\Hom_{\eRel(A,B)}(R,R')$.
    \end{itemize}
\end{definition}
\subsection{The Left Skew Monoidal Unit}\label{subsection-the-left-skew-monoidal-structure-on-rel-a-b-the-left-skew-monoidal-unit}
Let $A$ and $B$ be sets and let $J\colon A\rightproarrow B$ be a relation.
\begin{definition}{The Left $J$-Skew Monoidal Unit of $\eRel(A,B)$}{the-left-j-skew-monoidal-unit-of-erelab}%
    The \textbf{left $J$-skew monoidal unit of $\eRel(A,B)$} is the functor
    \[
        \Unit^{\eRel(A,B)}_{\lhd_{J}}
        \colon
        \PunctualCategory
        \to
        \eRel(A,B)
    \]
    picking the object
    \[
        \Unit^{\lhd_{J}}_{\eRel(A,B)}%
        \defeq%
        J
    \]%
    of $\eRel(A,B)$.
\end{definition}
\subsection{The Left Skew Associators}\label{subsection-the-left-skew-monoidal-structure-on-rel-a-b-the-skew-associators}
Let $A$ and $B$ be sets and let $J\colon A\rightproarrow B$ be a relation.
\begin{definition}{The Left $J$-Skew Associator of $\eRel(A,B)$}{the-left-j-skew-associator-of-erelab}%
    The \textbf{left $J$-skew associator of $\eRel(A,B)$} is the natural transformation
    \begin{envsmallsize}
        \[
            \alpha^{\eRel(A,B),\lhd_{J}}%
            \colon%
            {\lhd_{J}}\circ{({\lhd_{J}}\times\sfid)}%
            \Longrightarrow%
            {\lhd_{J}}\circ{(\sfid\times{\lhd_{J}})}\circ{\bfalpha^{\Cats}_{\eRel(A,B),\eRel(A,B),\eRel(A,B)}},%
        \]
    \end{envsmallsize}
    as in the diagram
    \[
        \begin{tikzcd}[row sep={0*\the\DL,between origins}, column sep={0*\the\DL,between origins}, background color=backgroundColor, ampersand replacement=\&]
            \&[0.30901699437\ThreeCm]
            \&[0.5\ThreeCm]
            {\eRel(A,B)\times(\eRel(A,B)\times\eRel(A,B))}
            \&[0.5\ThreeCm]
            \&[0.30901699437\ThreeCm]
            \\[0.58778525229\ThreeCm]
            {(\eRel(A,B)\times\eRel(A,B))\times\eRel(A,B)}
            \&[0.30901699437\ThreeCm]
            \&[0.5\ThreeCm]
            \&[0.5\ThreeCm]
            \&[0.30901699437\ThreeCm]
            {\eRel(A,B)\times\eRel(A,B)}
            \\[0.95105651629\ThreeCm]
            \&[0.30901699437\ThreeCm]
            {\eRel(A,B)\times\eRel(A,B)}
            \&[0.5\ThreeCm]
            \&[0.5\ThreeCm]
            {\eRel(A,B)\mrp{,}}
            \&[0.30901699437\ThreeCm]
            % 1-Arrows
            % Left Boundary
            \arrow[from=2-1,to=1-3,"{\bfalpha^{\Cats}_{\eRel(A,B),\eRel(A,B),\eRel(A,B)}}"{pos=0.35},isoarrowprime]%
            \arrow[from=1-3,to=2-5,"{\sfid\times{\lhd_{J}}}"{pos=0.575},""{name=2}]%
            \arrow[from=2-5,to=3-4,"\lhd_{J}"{pos=0.425}]%
            % Right Boundary
            \arrow[from=2-1,to=3-2,"{{\lhd_{J}}\times\sfid}"'{pos=0.425}]%
            \arrow[from=3-2,to=3-4,"\lhd_{J}"']%
            % 2-Arrows
            \arrow[from=3-2,to=2,"\alpha^{\eRel(A,B),\lhd_{J}}"{description,pos=0.475},Rightarrow,shorten <= 0.5*\the\DL,shorten >= 1*\the\DL]%
        \end{tikzcd}
    \]%
    whose component
    \[
        \alpha^{\eRel(A,B),\lhd_{J}}_{T,S,R}%
        \colon%
        \underbrace{(T\lhd_{J}S)\lhd_{J}R}_{\defeq T\procirc\Rift_{J}(S)\procirc\Rift_{J}(R)}%
        \hookrightarrow
        \underbrace{T\lhd_{J}(S\lhd_{J}R)}_{\defeq T\procirc\Rift_{J}(S\procirc\Rift_{J}(R))}%
    \]%
    at $(T,S,R)$ is given by
    \[
        \alpha^{\eRel(A,B),\lhd_{J}}_{T,S,R}%
        \defeq%
        \id_{T}\procirc\gamma,
    \]%
    where
    \[
        \gamma%
        \colon%
        \Rift_{J}(S)\procirc\Rift_{J}(R)
        \hookrightarrow
        \Rift_{J}(S\procirc\Rift_{J}(R))
    \]%
    is the inclusion adjunct to the inclusion
    \[
        \epsilon_{S}\twocirc\id_{\Rift_{J}(R)}
        \colon
        \underbrace{J\procirc\Rift_{J}(S)\procirc\Rift_{J}(R)}_{\defeq J_{!}(\Rift_{J}(S)\procirc\Rift_{J}(R))}
        \hookrightarrow
        S\procirc\Rift_{J}(R)
    \]%
    under the adjunction $J_{!}\dashv\Rift_{J}$, where $\epsilon\colon{J\procirc\Rift_{J}}\Longrightarrow\id_{\eRel(A,B)}$ is the counit of the adjunction $J_{!}\dashv\Rift_{J}$.
\end{definition}
\subsection{The Left Skew Left Unitors}\label{subsection-the-left-skew-monoidal-structure-on-rel-a-b-the-left-skew-left-unitors}
Let $A$ and $B$ be sets and let $J\colon A\rightproarrow B$ be a relation.
\begin{definition}{The Left $J$-Skew Left Unitor of $\eRel(A,B)$}{the-left-j-skew-left-unitor-of-erelab}%
    The \textbf{left $J$-skew left unitor of $\eRel(A,B)$} is the natural transformation
    \[
        \LUnitor^{\eRel(A,B),\lhd_{J}}
        \colon
        {\lhd_{J}}\circ{({\Unit^{\eRel(A,B)}_{\lhd_{J}}}\times{\sfid})}
        \Longrightarrow
        \bfLUnitor^{\TwoCategoryOfCategories}_{\eRel(A,B)}%
    \]%
    as in the diagram
    \[
        \begin{tikzcd}[row sep={9.0*\the\DL,between origins}, column sep={14.0*\the\DL,between origins}, background color=backgroundColor, ampersand replacement=\&]
            {\PunctualCategory\times\eRel(A,B)}
            \arrow[r,  "{\Unit^{\eRel(A,B)}_{\lhd_{J}}\times\sfid}"]
            \arrow[rd, dashed,"\bfLUnitor^{\TwoCategoryOfCategories}_{\eRel(A,B)}"'{name=1,pos=0.475},bend right=30]
            \&
            \eRel(A,B)\times\eRel(A,B)
            \arrow[d, "\lhd_{J}"]
            \\
            {}
            \&
            \eRel(A,B)\mathrlap{,}
            % 2-Arrows
            \arrow[Rightarrow,from=1-2,to=1,shorten >=1.0*\the\DL,shorten <=1.0*\the\DL,"\LUnitor^{\eRel(A,B),\lhd_{J}}"description]
        \end{tikzcd}
    \]%
    whose component
    \[
        \LUnitor^{\eRel(A,B),\lhd_{J}}_{R}
        \colon
        \underbrace{J\lhd_{J}R}_{\defeq J\procirc\Rift_{J}(R)}
        \hookrightarrow
        R
    \]%
    at $R$ is given by
    \[
        \LUnitor^{\eRel(A,B),\lhd_{J}}_{R}%
        \defeq%
        \epsilon_{R},
    \]%
    where $\epsilon\colon J_{!}\procirc\Rift_{J}\Longrightarrow\id_{\eRel(A,B)}$ is the counit of the adjunction $J_{!}\dashv\Rift_{J}$.
\end{definition}
\subsection{The Left Skew Right Unitors}\label{subsection-the-left-skew-monoidal-structure-on-rel-a-b-the-left-skew-right-unitors}
Let $A$ and $B$ be sets and let $J\colon A\rightproarrow B$ be a relation.
\begin{definition}{The Left $J$-Skew Right Unitor of $\eRel(A,B)$}{the-left-j-skew-right-unitor-of-erelab}%
    The \textbf{left $J$-skew right unitor of $\eRel(A,B)$} is the natural transformation
    \[
        \RUnitor^{\eRel(A,B),\lhd_{J}}
        \colon
        \bfRUnitor^{\TwoCategoryOfCategories}_{\eRel(A,B)}
        \Longrightarrow
        {\lhd_{J}}\circ{(\sfid\times\Unit^{\eRel(A,B)}_{\lhd_{J}})}
    \]
    as in the diagram
    \[
        \begin{tikzcd}[row sep={9.0*\the\DL,between origins}, column sep={14.0*\the\DL,between origins}, background color=backgroundColor, ampersand replacement=\&]
            {\eRel(A,B)\times\PunctualCategory}
            \arrow[r, "\sfid\times\Unit^{\eRel(A,B)}_{\lhd_{J}}"]
            \arrow[rd, dashed,"\bfRUnitor^{\TwoCategoryOfCategories}_{\eRel(A,B)}"'{name=1,pos=0.475},bend right=30]
            \&
            {\eRel(A,B)\times\eRel(A,B)\mrp{,}}
            \arrow[d, "\lhd_{J}"]
            \\
            {}
            \&
            {\eRel(A,B)}
            % 2-Arrows
            \arrow[Rightarrow,from=1,to=1-2,shorten >=1.0*\the\DL,shorten <=1.0*\the\DL,"\RUnitor^{\eRel(A,B),\lhd_{J}}"description]
        \end{tikzcd}
    \]%
    whose component
    \[
        \RUnitor^{\eRel(A,B),\lhd_{J}}_{R}%
        \colon%
        R%
        \hookrightarrow
        \underbrace{R\lhd_{J}J}_{\defeq R\procirc\Rift_{J}(J)}%
    \]%
    at $R$ is given by the composition
    \begin{align*}
        R &\mkern10mu\mrp{\Longrightisoarrow}\mkern50mu                               R\procirc\chi_{A}\\
          &\mkern10mu\mrp{\xLongrightarrow{\id_{R}\procirc\eta_{\chi_{A}}}}\mkern50mu R\procirc\Rift_{J}(J_{!}(\chi_{A}))\\
          &\mkern10mu\mrp{\defeq}\mkern50mu                                           R\procirc\Rift_{J}(J\procirc\chi_{A})\\
          &\mkern10mu\mrp{\Longrightisoarrow}\mkern50mu                               R\procirc\Rift_{J}(J)\\
          &\mkern10mu\mrp{\defeq}\mkern50mu                                           R\lhd_{J}J,
    \end{align*}
    where $\eta\colon\id_{\eRel(A,A)}\Longrightarrow\Rift_{J}\circ J_{!}$ is the unit of the adjunction $J_{!}\dashv\Rift_{J}$.
\end{definition}
\subsection{The Left Skew Monoidal Structure on $\eRel(A,B)$}\label{subsection-the-left-skew-monoidal-structure-on-rel-a-b}
\begin{proposition}{The Left $J$-Skew Monoidal Structure on $\eRel(A,B)$}{the-left-j-skew-monoidal-structure-on-erelab}%
    The category $\eRel(A,B)$ admits a left skew monoidal category structure consisting of%
    \begin{itemize}
        \item\SloganFont{The Underlying Category. }The posetal category associated to the poset $\eRel(A,B)$ of relations from $A$ to $B$ of \cref{the-set-of-relations-between-two-sets-2} of \cref{the-set-of-relations-between-two-sets}.
        \item\SloganFont{The Left Skew Monoidal Product. }The left $J$-skew monoidal product
            \[
                \lhd_{J}%
                \colon%
                \eRel(A,B)\times\eRel(A,B)%
                \to%
                \eRel(A,B)
            \]%
            of \cref{the-left-j-skew-monoidal-product-of-erelab}.
        \item\SloganFont{The Left Skew Monoidal Unit. }The functor
            \[
                \Unit^{\eRel(A,B),\lhd_{J}}
                \colon
                \PunctualCategory
                \to
                \eRel(A,B)
            \]
            of \cref{the-left-j-skew-monoidal-unit-of-erelab}.
        \item\SloganFont{The Left Skew Associators. }The natural transformation
            \begin{envsmallsize}
                \[
                    \alpha^{\eRel(A,B),\lhd_{J}}%
                    \colon%
                    {\lhd_{J}}\circ{({\lhd_{J}}\times\sfid)}%
                    \Longrightarrow%
                    {\lhd_{J}}\circ{(\sfid\times{\lhd_{J}})}\circ{\bfalpha^{\Cats}_{\eRel(A,B),\eRel(A,B),\eRel(A,B)}}%
                \]
            \end{envsmallsize}
            of \cref{the-left-j-skew-associator-of-erelab}.
        \item\SloganFont{The Left Skew Left Unitors. }The natural transformation
            \[
                \LUnitor^{\eRel(A,B),\lhd_{J}}
                \colon
                {\lhd_{J}}\circ{({\Unit^{\eRel(A,B)}_{\lhd_{J}}}\times{\sfid})}
                \Longrightarrow
                \bfLUnitor^{\TwoCategoryOfCategories}_{\eRel(A,B)}%
            \]
            of \cref{the-left-j-skew-left-unitor-of-erelab}.
        \item\SloganFont{The Left Skew Right Unitors. }The natural transformation
            \[
                \RUnitor^{\eRel(A,B),\lhd_{J}}
                \colon
                \bfRUnitor^{\TwoCategoryOfCategories}_{\eRel(A,B)}
                \Longrightarrow
                {\lhd_{J}}\circ{(\sfid\times\Unit^{\eRel(A,B)}_{\lhd_{J}})}
            \]
            of \cref{the-left-j-skew-right-unitor-of-erelab}.
    \end{itemize}
\end{proposition}
\begin{Proof}{Proof of \cref{the-left-j-skew-monoidal-structure-on-erelab}}%
    Since $\eRel(A,B)$ is posetal, the commutativity of the pentagon identity, the left skew left triangle identity, the left skew right triangle identity, the left skew middle triangle identity, and the zigzag identity is automatic (\ChapterRef{\ChapterCategories, \cref{categories:properties-of-posetal-categories-automatic-commutativity-of-diagrams} of \cref{categories:properties-of-posetal-categories}}{\cref{categories:properties-of-posetal-categories-automatic-commutativity-of-diagrams} of \cref{categories:properties-of-posetal-categories}}), and thus $\eRel(A,B)$ together with the data in the statement forms a left skew monoidal category.
\end{Proof}
\section{The Right Skew Monoidal Structure on $\eRel(A,B)$}\label{section-the-right-skew-monoidal-structure-on-rel-a-b}
Let $A$ and $B$ be sets and let $J\colon A\rightproarrow B$ be a relation.
\subsection{The Right Skew Monoidal Product}\label{subsection-the-right-skew-monoidal-structure-on-rel-a-b-the-right-skew-monoidal-product}
\begin{definition}{The Right $J$-Skew Monoidal Product of $\eRel(A,B)$}{the-right-j-skew-monoidal-product-of-erelab}%
    The \textbf{right $J$-skew monoidal product of $\eRel(A,B)$} is the functor
    \[
        \rhd_{J}%
        \colon
        \eRel(A,B)\times\eRel(A,B)
        \to
        \eRel(A,B)
    \]%
    where
    \begin{itemize}
        \item\SloganFont{Action on Objects. }For each $R,S\in\Obj(\eRel(A,B))$, we have
            \begin{webcompile}
                S\rhd_{J}R%
                \defeq%
                \Ran_{J}(S)\procirc R,
                \quad
                \begin{tikzcd}[row sep={5.0*\the\DL,between origins}, column sep={5.0*\the\DL,between origins}, background color=backgroundColor, ampersand replacement=\&]
                    A
                    \arrow[r,"R",mid vert]
                    \&
                    B
                    \arrow[r,"{\Ran_{J}(S)}",densely dashed,mid vert]
                    \&
                    B\mrp{.}
                    \\
                    \&
                    A
                    \arrow[u,"J",mid vert]
                    \arrow[ru,"S"'{name=S},mid vert]
                    \&
                    % 2-Arrows
                    \arrow[from=S,to=1-2,shorten <= 0.5em,shorten >= -0.125em,Rightarrow]%
                \end{tikzcd}
            \end{webcompile}%
        \item\SloganFont{Action on Morphisms. }For each $R,S,R',S'\in\Obj(\eRel(A,B))$, the action on $\Hom$-sets
            \begin{envscriptsize}
                \[
                    (\rhd_{J})_{(S,R),(S',R')}
                    \colon%
                    \Hom_{\eRel(A,B)}(S,S')\times\Hom_{\eRel(A,B)}(R,R')
                    \to
                    \Hom_{\eRel(A,B)}(S\rhd_{J}R,S'\rhd_{J}R')
                \]%
            \end{envscriptsize}
            of $\rhd_{J}$ at $((S,R),(S',R'))$ is defined by%
            %--- Begin Footnote ---%
            \footnote{%
                Since $\eRel(A,B)$ is posetal, this is to say that if $S\subset S'$ and $R\subset R'$, then $S\rhd_{J}R\subset S'\rhd_{J}R'$.
                \par\vspace*{\TCBBoxCorrection}
            }%
            %---  End Footnote  ---%
            \begin{webcompile}
                \beta\rhd_{J}\alpha%
                \defeq%
                \Ran_{J}(\beta)\procirc\alpha,%
                \begin{tikzcd}[row sep={8.0*\the\DL,between origins}, column sep={10.0*\the\DL,between origins}, background color=backgroundColor, ampersand replacement=\&]
                    A
                    \arrow[r,"R'",""{name=Rprime},  bend left  = 30,mid vert]
                    \arrow[r,"R"',""'{name=R},      bend right = 30,mid vert]
                    \&[-3.0*\the\DL]
                    B
                    \arrow[r,"{\Ran_{J}(S')}",          ,""{name=RanJSprime},bend left  = 25,densely dashed,mid vert]
                    \arrow[r,"{\Ran_{J}(S)}"'description,""'{name=RanJS},    bend right = 25,densely dashed,mid vert=0.25]
                    \&
                    B
                    \\
                    \&[-3.0*\the\DL]
                    A
                    \arrow[u,"J",mid vert]
                    \arrow[ru,bend left  = 0,"S'"'description,""'{name=Sprime},mid vert=0.35]
                    \arrow[ru,bend right = 40,"S"'{pos=0.525},""'{name=S,pos=0.535},mid vert=0.525]
                    \&
                    % 2-Arrows
                    \arrow[from=Sprime,to=1-2,shorten <= 0.75em,shorten >= -0.125em,Rightarrow]%
                    \arrow[from=R,to=Rprime,shorten=0.45em,Rightarrow,"\alpha"{pos=0.475}]%
                    \arrow[from=S,to=Sprime,shorten <= 0.5em,shorten >= -1.125em,Rightarrow,"\beta"{pos=1.2}]%
                    \arrow[from=RanJS,to=RanJSprime,shorten <= 0.5em,shorten >= 0.5em,Rightarrow,"\Ran_{J}(\beta)"{pos=0.475}]%
                \end{tikzcd}
            \end{webcompile}%
            for each $\beta\in\Hom_{\eRel(A,B)}(S,S')$ and each $\alpha\in\Hom_{\eRel(A,B)}(R,R')$.
    \end{itemize}
\end{definition}
\subsection{The Right Skew Monoidal Unit}\label{subsection-the-right-skew-monoidal-structure-on-rel-a-b-the-right-skew-monoidal-unit}
\begin{definition}{The Right $J$-Skew Monoidal Unit of $\eRel(A,B)$}{the-right-j-skew-monoidal-unit-of-erelab}%
    The \textbf{right $J$-skew monoidal unit of $\eRel(A,B)$} is the functor
    \[
        \Unit^{\eRel(A,B)}_{\rhd_{J}}
        \colon
        \PunctualCategory
        \to
        \eRel(A,B)
    \]
    picking the object
    \[
        \Unit^{\rhd_{J}}_{\eRel(A,B)}%
        \defeq%
        J
    \]%
    of $\eRel(A,B)$.
\end{definition}
\subsection{The Right Skew Associators}\label{subsection-the-right-skew-monoidal-structure-on-rel-a-b-the-right-skew-associators}
\begin{definition}{The Right $J$-Skew Associator of $\eRel(A,B)$}{the-right-j-skew-associator-of-erelab}%
    The \textbf{right $J$-skew associator of $\eRel(A,B)$} is the natural transformation
    \begin{envsmallsize}
        \[
            \alpha^{\eRel(A,B),\rhd_{J}}%
            \colon%
            {\rhd_{J}}\circ{(\sfid\times{\rhd_{J}})}%
            \Longrightarrow%
            {\rhd_{J}}\circ{({\rhd_{J}}\times\sfid)}\circ{\bfalpha^{\Cats,-1}_{\eRel(A,B),\eRel(A,B),\eRel(A,B)}},%
        \]
    \end{envsmallsize}
    as in the diagram
    \[
        \begin{tikzcd}[row sep={0*\the\DL,between origins}, column sep={0*\the\DL,between origins}, background color=backgroundColor, ampersand replacement=\&]
            \&[0.30901699437\ThreeCm]
            \&[0.5\ThreeCm]
            {(\eRel(A,B)\times\eRel(A,B))\times\eRel(A,B)}
            \&[0.5\ThreeCm]
            \&[0.30901699437\ThreeCm]
            \\[0.58778525229\ThreeCm]
            {\eRel(A,B)\times(\eRel(A,B)\times\eRel(A,B))}
            \&[0.30901699437\ThreeCm]
            \&[0.5\ThreeCm]
            \&[0.5\ThreeCm]
            \&[0.30901699437\ThreeCm]
            {\eRel(A,B)\times\eRel(A,B)}
            \\[0.95105651629\ThreeCm]
            \&[0.30901699437\ThreeCm]
            {\eRel(A,B)\times\eRel(A,B)}
            \&[0.5\ThreeCm]
            \&[0.5\ThreeCm]
            {\eRel(A,B)\mrp{,}}
            \&[0.30901699437\ThreeCm]
            % 1-Arrows
            % Left Boundary
            \arrow[from=2-1,to=1-3,"{\bfalpha^{\Cats,-1}_{\eRel(A,B),\eRel(A,B),\eRel(A,B)}}"{pos=0.35},isoarrowprime]%
            \arrow[from=1-3,to=2-5,"{{\rhd_{J}}\times\sfid}"{pos=0.575},""{name=2}]%
            \arrow[from=2-5,to=3-4,"\rhd_{J}"{pos=0.425}]%
            % Right Boundary
            \arrow[from=2-1,to=3-2,"{\sfid\times{\rhd_{J}}}"'{pos=0.425}]%
            \arrow[from=3-2,to=3-4,"\rhd_{J}"']%
            % 2-Arrows
            \arrow[from=3-2,to=2,"\alpha^{\eRel(A,B),\rhd_{J}}"{description,pos=0.475},Rightarrow,shorten <= 0.5*\the\DL,shorten >= 1*\the\DL]%
        \end{tikzcd}
    \]%
    whose component
    \[
        \alpha^{\eRel(A,B),\rhd_{J}}_{T,S,R}%
        \colon%
        \underbrace{T\rhd_{J}(S\rhd_{J}R)}_{\defeq\Ran_{J}(T)\procirc\Ran_{J}(S)\procirc R}%
        \hookrightarrow
        \underbrace{(T\rhd_{J}S)\rhd_{J}R}_{\defeq\Ran_{J}(\Ran_{J}(T)\procirc S)\procirc R}%
    \]%
    at $(T,S,R)$ is given by
    \[
        \alpha^{\eRel(A,B),\rhd}_{T,S,R}%
        \defeq%
        \gamma\procirc\id_{R},
    \]%
    where
    \[%
        \gamma%
        \colon%
        \Ran_{J}(T)\procirc\Ran_{J}(S)
        \hookrightarrow
        \Ran_{J}(\Ran_{J}(T)\procirc S)
    \]%
    is the inclusion adjunct to the inclusion
    \[
        \id_{\Ran_{J}(T)}\procirc\epsilon_{S}%
        \colon%
        \underbrace{\Ran_{J}(T)\procirc\Ran_{J}(S)\procirc J}_{\defeq J^{*}(\Ran_{J}(T)\procirc\Ran_{J}(S))}%
        \hookrightarrow
        \Ran_{J}(T)\procirc S%
    \]%
    under the adjunction $J^{*}\dashv\Ran_{J}$, where $\epsilon\colon{\Ran_{J}}\procirc{J}\Longrightarrow\id_{\eRel(A,B)}$ is the counit of the adjunction $J^{*}\dashv\Ran_{J}$.
\end{definition}
\subsection{The Right Skew Left Unitors}\label{subsection-the-right-skew-monoidal-structure-on-rel-a-b-the-right-skew-left-unitors}
\begin{definition}{The Right $J$-Skew Left Unitor of $\eRel(A,B)$}{the-right-j-skew-left-unitor-of-erelab}%
    The \textbf{right $J$-skew left unitor of $\eRel(A,B)$} is the natural transformation
    \[
        \LUnitor^{\eRel(A,B),\rhd_{J}}
        \colon
        \bfLUnitor^{\TwoCategoryOfCategories}_{\eRel(A,B)}
        \Longrightarrow
        {\rhd_{J}}\circ{(\Unit^{\eRel(A,B)}_{\rhd}\times\sfid)},
    \]
    as in the diagram
    \[
        \begin{tikzcd}[row sep={9.0*\the\DL,between origins}, column sep={14.0*\the\DL,between origins}, background color=backgroundColor, ampersand replacement=\&]
            {\PunctualCategory\times\eRel(A,B)}
            \arrow[r,  "{\Unit^{\eRel(A,B)}_{\rhd_{J}}\times\sfid}"]
            \arrow[rd, dashed,"\bfLUnitor^{\TwoCategoryOfCategories}_{\eRel(A,B)}"'{name=1,pos=0.475},bend right=30]
            \&
            \eRel(A,B)\times\eRel(A,B)
            \arrow[d, "\rhd_{J}"]
            \\
            {}
            \&
            \eRel(A,B)\mathrlap{,}
            % 2-Arrows
            \arrow[Rightarrow,from=1,to=1-2,shorten >=1.0*\the\DL,shorten <=1.0*\the\DL,"\LUnitor^{\eRel(A,B),\rhd_{J}}"description]
        \end{tikzcd}
    \]%
    whose component
    \[
        \LUnitor^{\eRel(A,B),\rhd_{J}}_{R}
        \colon
        R
        \hookrightarrow
        \underbrace{J\rhd_{J}R}_{\defeq\Ran_{J}(J)\procirc R}
    \]%
    at $R$ is given by the composition
    \begin{align*}
        R &\mkern10mu\mrp{\Longrightisoarrow}\mkern50mu                               \chi_{B}\procirc R\\
          &\mkern10mu\mrp{\xLongrightarrow{\eta_{\chi_{B}}}\procirc\id_{R}}\mkern50mu \Ran_{J}(J^{*}(\chi_{A}))\procirc R\\
          &\mkern10mu\mrp{\defeq}\mkern50mu                                           \Ran_{J}(J^{*}\procirc\chi_{A})\procirc R\\
          &\mkern10mu\mrp{\Longrightisoarrow}\mkern50mu                               \Ran_{J}(J)\procirc R\\
          &\mkern10mu\mrp{\defeq}\mkern50mu                                           R\rhd_{J}J,
    \end{align*}
    where $\eta\colon\id_{\eRel(B,B)}\Longrightarrow\Ran_{J}\circ J^{*}$ is the unit of the adjunction $J^{*}\dashv\Ran_{J}$.
\end{definition}
\subsection{The Right Skew Right Unitors}\label{subsection-the-right-skew-monoidal-structure-on-rel-a-b-the-right-skew-right-unitors}
\begin{definition}{The Right $J$-Skew Right Unitor of $\eRel(A,B)$}{the-right-j-skew-right-unitor-of-erelab}%
    The \textbf{right $J$-skew right unitor of $\eRel(A,B)$} is the natural transformation
    \[
        \RUnitor^{\eRel(A,B),\rhd_{J}}
        \colon
        {\rhd_{J}}\circ{(\sfid\times\Unit^{\eRel(A,B)}_{\rhd})}
        \Longrightarrow
        \bfRUnitor^{\TwoCategoryOfCategories}_{\eRel(A,B)},
    \]
    as in the diagram
    \[
        \begin{tikzcd}[row sep={9.0*\the\DL,between origins}, column sep={14.0*\the\DL,between origins}, background color=backgroundColor, ampersand replacement=\&]
            {\eRel(A,B)\times\PunctualCategory}
            \arrow[r, "\sfid\times\Unit^{\eRel(A,B)}_{\rhd_{J}}"]
            \arrow[rd, dashed,"\bfRUnitor^{\TwoCategoryOfCategories}_{\eRel(A,B)}"'{name=1,pos=0.475},bend right=30]
            \&
            {\eRel(A,B)\times\eRel(A,B)\mrp{,}}
            \arrow[d, "\rhd_{J}"]
            \\
            {}
            \&
            {\eRel(A,B)}
            % 2-Arrows
            \arrow[Rightarrow,from=1-2,to=1,shorten >=1.0*\the\DL,shorten <=1.0*\the\DL,"\RUnitor^{\eRel(A,B),\rhd_{J}}"description]
        \end{tikzcd}
    \]%
    whose component
    \[
        \RUnitor^{\eRel(A,B),\rhd_{J}}_{S}%
        \colon%
        \underbrace{S\rhd_{J}J}_{\defeq\Ran_{J}(S)\procirc J}%
        \hookrightarrow%
        S%
    \]%
    at $S$ is given by%
    \[
        \RUnitor^{\eRel(A,B),\rhd_{J}}_{S}%
        \defeq%
        \epsilon_{R},
    \]%
    where $\epsilon\colon J^{*}\circ\Ran_{J}\Longrightarrow\id_{\eRel(A,B)}$ is the counit of the adjunction $J^{*}\dashv\Ran_{J}$.
\end{definition}
\subsection{The Right Skew Monoidal Structure on $\eRel(A,B)$}\label{subsection-the-right-skew-monoidal-structure-on-rel-a-b}
\begin{proposition}{The Right $J$-Skew Monoidal Structure on $\eRel(A,B)$}{the-right-j-skew-monoidal-structure-on-erelab}%
    The category $\eRel(A,B)$ admits a right skew monoidal category structure consisting of%
    \begin{itemize}
        \item\SloganFont{The Underlying Category. }The posetal category associated to the poset $\eRel(A,B)$ of relations from $A$ to $B$ of \cref{the-set-of-relations-between-two-sets-2} of \cref{the-set-of-relations-between-two-sets}.
        \item\SloganFont{The Right Skew Monoidal Product. }The right $J$-skew monoidal product
            \[
                \lhd_{J}%
                \colon%
                \eRel(A,B)\times\eRel(A,B)%
                \to%
                \eRel(A,B)
            \]%
            of \cref{the-right-j-skew-monoidal-product-of-erelab}.
        \item\SloganFont{The Right Skew Monoidal Unit. }The functor
            \[
                \Unit^{\eRel(A,B),\lhd_{J}}
                \colon
                \PunctualCategory
                \to
                \eRel(A,B)
            \]
            of \cref{the-right-j-skew-monoidal-unit-of-erelab}.
        \item\SloganFont{The Right Skew Associators. }The natural transformation
            \begin{envsmallsize}
                \[
                    \alpha^{\eRel(A,B),\rhd_{J}}%
                    \colon%
                    {\rhd_{J}}\circ{(\sfid\times{\rhd_{J}})}%
                    \Longrightarrow%
                    {\rhd_{J}}\circ{({\rhd_{J}}\times\sfid)}\circ{\bfalpha^{\Cats,-1}_{\eRel(A,B),\eRel(A,B),\eRel(A,B)}}%
                \]
            \end{envsmallsize}
            of \cref{the-right-j-skew-associator-of-erelab}.
        \item\SloganFont{The Right Skew Left Unitors. }The natural transformation
            \[
                \LUnitor^{\eRel(A,B),\rhd_{J}}
                \colon
                \bfLUnitor^{\TwoCategoryOfCategories}_{\eRel(A,B)}
                \Longrightarrow
                {\rhd_{J}}\circ{(\Unit^{\eRel(A,B)}_{\rhd}\times\sfid)}
            \]
            of \cref{the-right-j-skew-left-unitor-of-erelab}.
        \item\SloganFont{The Right Skew Right Unitors. }The natural transformation
            \[
                \RUnitor^{\eRel(A,B),\rhd_{J}}
                \colon
                {\rhd_{J}}\circ{(\sfid\times\Unit^{\eRel(A,B)}_{\rhd})}
                \Longrightarrow
                \bfRUnitor^{\TwoCategoryOfCategories}_{\eRel(A,B)}
            \]
            of \cref{the-right-j-skew-right-unitor-of-erelab}.
    \end{itemize}
\end{proposition}
\begin{Proof}{Proof of \cref{the-right-j-skew-monoidal-structure-on-erelab}}%
    Since $\eRel(A,B)$ is posetal, the commutativity of the pentagon identity, the right skew left triangle identity, the right skew right triangle identity, the right skew middle triangle identity, and the zigzag identity is automatic (\ChapterRef{\ChapterCategories, \cref{categories:properties-of-posetal-categories-automatic-commutativity-of-diagrams} of \cref{categories:properties-of-posetal-categories}}{\cref{categories:properties-of-posetal-categories-automatic-commutativity-of-diagrams} of \cref{categories:properties-of-posetal-categories}}), and thus $\eRel(A,B)$ together with the data in the statement forms a right skew monoidal category.
\end{Proof}
\begin{appendices}
\begin{multicols}{2}[\section{Other Chapters}]
\noindent
\textbf{Preliminaries}
\begin{enumerate}
\item \hyperref[introduction:section-phantom]{Introduction}
\end{enumerate}
\textbf{Sets}
\begin{enumerate}
\setcounter{enumi}{2}
\item \hyperref[sets:section-phantom]{Sets}
\item \hyperref[constructions-with-sets:section-phantom]{Constructions With Sets}
\item \hyperref[monoidal-structures-on-the-category-of-sets:section-phantom]{Monoidal Structures on the Category of Sets}
\item \hyperref[pointed-sets:section-phantom]{Pointed Sets}
\item \hyperref[tensor-products-of-pointed-sets:section-phantom]{Tensor Products of Pointed Sets}
\end{enumerate}
\textbf{Relations}
\begin{enumerate}
\setcounter{enumi}{6}
\item \hyperref[relations:section-phantom]{Relations}
\item \hyperref[constructions-with-relations:section-phantom]{Constructions With Relations}
\item \hyperref[conditions-on-relations:section-phantom]{Conditions on Relations}
\end{enumerate}
\textbf{Category Theory}
\begin{enumerate}
\setcounter{enumi}{9}
\item \hyperref[categories:section-phantom]{Categories}
\end{enumerate}
\textbf{Monoidal Categories}
\begin{enumerate}
\setcounter{enumi}{10}
\item \hyperref[constructions-with-monoidal-categories:section-phantom]{Constructions With Monoidal Categories}
\end{enumerate}
\textbf{Bicategories}
\begin{enumerate}
\setcounter{enumi}{11}
\item \hyperref[types-of-morphisms-in-bicategories:section-phantom]{Types of Morphisms in Bicategories}
\end{enumerate}
\textbf{Extra Part}
\begin{enumerate}
\setcounter{enumi}{12}
\item \hyperref[notes:section-phantom]{Notes}
\end{enumerate}
\end{multicols}

\end{appendices}
\end{document}
