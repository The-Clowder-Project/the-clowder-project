\input{preamble}

% OK, start here.
%
\usepackage{fontspec}
\let\hyperwhite\relax
\let\hyperred\relax
\newcommand{\hyperwhite}{\hypersetup{citecolor=white,filecolor=white,linkcolor=white,urlcolor=white}}
\newcommand{\hyperred}{%
\hypersetup{%
    citecolor=TitlingRed,%
    filecolor=TitlingRed,%
    linkcolor=TitlingRed,%
     urlcolor=TitlingRed%
}}
\let\ChapterRef\relax
\newcommand{\ChapterRef}[2]{#1}
\setcounter{tocdepth}{2}
%▓▓▓▓▓▓▓▓▓▓▓▓▓▓▓▓▓▓▓▓▓▓▓▓▓▓▓▓▓▓▓▓▓
%▓▓ ╔╦╗╦╔╦╗╦  ╔═╗  ╔═╗╔═╗╔╗╔╔╦╗ ▓▓
%▓▓  ║ ║ ║ ║  ║╣   ╠╣ ║ ║║║║ ║  ▓▓
%▓▓  ╩ ╩ ╩ ╩═╝╚═╝  ╚  ╚═╝╝╚╝ ╩  ▓▓
%▓▓▓▓▓▓▓▓▓▓▓▓▓▓▓▓▓▓▓▓▓▓▓▓▓▓▓▓▓▓▓▓▓
%\usepackage{titlesec}
%▓▓▓▓▓▓▓▓▓▓▓▓▓▓▓▓▓▓▓▓▓▓▓▓▓▓▓▓▓▓▓▓▓▓▓▓▓▓▓▓▓▓▓▓▓▓▓▓▓▓▓▓▓▓▓
%▓▓ ╔╦╗╔═╗╔╗ ╦  ╔═╗  ╔═╗╔═╗  ╔═╗╔═╗╔╗╔╔╦╗╔═╗╔╗╔╔╦╗╔═╗ ▓▓
%▓▓  ║ ╠═╣╠╩╗║  ║╣   ║ ║╠╣   ║  ║ ║║║║ ║ ║╣ ║║║ ║ ╚═╗ ▓▓
%▓▓  ╩ ╩ ╩╚═╝╩═╝╚═╝  ╚═╝╚    ╚═╝╚═╝╝╚╝ ╩ ╚═╝╝╚╝ ╩ ╚═╝ ▓▓
%▓▓▓▓▓▓▓▓▓▓▓▓▓▓▓▓▓▓▓▓▓▓▓▓▓▓▓▓▓▓▓▓▓▓▓▓▓▓▓▓▓▓▓▓▓▓▓▓▓▓▓▓▓▓▓
\newcommand{\ChapterTableOfContents}{%
    \begingroup
    \addfontfeature{Numbers={Lining,Monospaced}}
    \hypersetup{hidelinks}\tableofcontents%
    \endgroup
}%

\let\DotFill\relax
\makeatletter
\newcommand \DotFill {\leavevmode \cleaders \hb@xt@ .33em{\hss .\hss }\hfill \kern \z@}
\makeatother

\definecolor{ToCGrey}{rgb}{0.4,0.4,0.4}
\definecolor{mainColor}{rgb}{0.82745098,0.18431373,0.18431373}
\usepackage{titletoc}
\titlecontents{part}
[0.0em]
{\addvspace{1pc}\color{TitlingRed}\large\bfseries\text{Part }}
{\bfseries\textcolor{TitlingRed}{\contentslabel{0.0em}}\hspace*{1.35em}}
{}
{\textcolor{TitlingRed}{{\hfill\bfseries\contentspage\nobreak}}}
[]
\titlecontents{section}
[0.0em]
{\addvspace{1pc}}
{\color{black}\bfseries\textcolor{TitlingRed}{\contentslabel{0.0em}}\hspace*{1.65em}}
{}
{\textcolor{black}{\textbf{\DotFill}{\bfseries\contentspage\nobreak}}}
[]
\titlecontents{subsection}
[0.0em]
{}
{\hspace*{1.65em}\color{ToCGrey}{\contentslabel{0.0em}}\hspace*{2.5em}}
{}
{{\textcolor{ToCGrey}\DotFill}\textcolor{ToCGrey}{\contentspage}\nobreak}
[]
\usepackage{marginnote}
\renewcommand*{\marginfont}{\normalfont}
\usepackage{inconsolata}
\setmonofont{inconsolata}%
\let\ChapterRef\relax
\newcommand{\ChapterRef}[2]{#1}
\AtBeginEnvironment{subappendices}{%%
    \section*{\huge Appendices}%
}%

\begin{document}

\title{Miscellaneous Notes}

\maketitle

\phantomsection
\label{section-phantom}

\ChapterTableOfContents

\section{To Do List}\label{section-to-do-list}
\subsection{Omitted Proofs To Add}\label{subsection-proofs-to-add}
% BEGIN RAW HTML %
<div class="epigraph-container">
  <div class="epigraph">

  <p>Не так благотворна истина, как зловредна ее видимость.<audio id="audioPlayer" src="/static/sounds/Данковский.ogg" type="audio/ogg"></audio><span class="dark-svg"><button id="playButton"></button></span></p>
    <div class="line"></div>
    <p class="author-left">Даниил Данковский</p>
  </div>

  <div class="epigraph">
    <p>Truth does not do as much good in the world as the appearance of truth does evil.<audio id="audioPlayer2" src="/static/sounds/Dankovsky.ogg" type="audio/ogg"></audio><span class="dark-svg"><button id="playButton2"></button></span></p>
    <div class="line"></div>
    <p class="author-right">Daniil Dankovsky</p>
  </div>
</div>
% BEGIN LATEX HTML %
\doubleepigraph{\Russian{Не так благотворна истина, как зловредна ее видимость.}}{\Russian{\textit{Даниил Данковский}}}{Truth does not do as much good in the world as the appearance of truth does evil.}{\textit{Daniil Dankovsky}}
% END RAW HTML %
There's a very large number of omitted proofs throughout these notes. Here I list them in decreasing order of how nice it would be to add them.
\begin{remark}{Omitted Proofs To Add}{omitted-proofs-to-add}%
    Proofs that \emph{need} to be added at some point:
    \begin{enumerate}
        \item \ChapterRef{\ChapterTensorProductsOfPointedSets, \cref{tensor-products-of-pointed-sets:universal-properties-of-the-smash-product-of-pointed-sets-1}}{\cref{universal-properties-of-the-smash-product-of-pointed-sets-1}}.
        \item \ChapterRef{\ChapterTensorProductsOfPointedSets, \cref{tensor-products-of-pointed-sets:universal-properties-of-the-smash-product-of-pointed-sets-2}}{\cref{universal-properties-of-the-smash-product-of-pointed-sets-2}}.
        \item Horizontal composition of natural transformations is associative: \ChapterRef{\ChapterCategories, \cref{categories:properties-of-horizontal-composition-of-natural-transformations-associativity} of \cref{categories:properties-of-horizontal-composition-of-natural-transformations}}{\cref{properties-of-horizontal-composition-of-natural-transformations-associativity} of \cref{properties-of-horizontal-composition-of-natural-transformations}}.
        \item Fully faithful functors are essentially injective: \ChapterRef{\ChapterCategories, \cref{categories:properties-of-fully-faithful-functors-essential-injectivity} of \cref{categories:properties-of-fully-faithful-functors}}{\cref{properties-of-fully-faithful-functors-essential-injectivity} of \cref{properties-of-fully-faithful-functors}}.
    \end{enumerate}
    Proofs that \emph{would be very nice} to be added at some point:
    \begin{enumerate}
        \item Properties of pseudomonic functors: \ChapterRef{\ChapterCategories, \cref{categories:properties-of-pseudomonic-functors}}{\cref{properties-of-pseudomonic-functors}}.
        \item Characterisation of fully faithful functors: \ChapterRef{\ChapterCategories, \cref{categories:properties-of-fully-faithful-functors-characterisations} of \cref{categories:properties-of-fully-faithful-functors}}{\cref{properties-of-fully-faithful-functors-characterisations} of \cref{properties-of-fully-faithful-functors}}.
    \end{enumerate}

    Proofs that \emph{would be nice} to be added at some point:
    \begin{enumerate}
        \item Properties of posetal categories: \ChapterRef{\ChapterCategories, \cref{categories:properties-of-posetal-categories}}{\cref{properties-of-posetal-categories}}.
        \item The quadruple adjunction between categories and sets: \ChapterRef{\ChapterCategories, \cref{categories:the-quadruple-adjunction-between-sets-and-cats}}{\cref{the-quadruple-adjunction-between-sets-and-cats}}.
        \item Properties of groupoid completions: \ChapterRef{\ChapterCategories, \cref{categories:properties-of-groupoid-completion}}{\cref{properties-of-groupoid-completion}}.
        \item Properties of cores: \ChapterRef{\ChapterCategories, \cref{categories:properties-of-the-core-of-a-category}}{\cref{properties-of-the-core-of-a-category}}.
        \item $F_{*}$ faithful iff $F$ faithful: \ChapterRef{\ChapterCategories, \cref{categories:properties-of-faithful-functors-interaction-with-postcomposition} of \cref{categories:properties-of-faithful-functors}}{\cref{properties-of-faithful-functors-interaction-with-postcomposition} of \cref{properties-of-faithful-functors}}.
        \item $F_{*}$ full iff $F$ full: \ChapterRef{\ChapterCategories, \cref{categories:properties-of-full-functors-interaction-with-postcomposition} of \cref{categories:properties-of-full-functors}}{\cref{properties-of-full-functors-interaction-with-postcomposition} of \cref{properties-of-full-functors}}.
        \item Injective on objects functors are precisely the isocofibrations in $\TwoCategoryOfCategories$: \ChapterRef{\ChapterCategories, \cref{categories:properties-of-injective-on-objects-functors-characterisations} of \cref{categories:properties-of-injective-on-objects-functors}}{\cref{properties-of-injective-on-objects-functors-characterisations} of \cref{properties-of-injective-on-objects-functors}}.
        \item Characterisations of monomorphisms of categories: \ChapterRef{\ChapterCategories, \cref{categories:properties-of-monomorphisms-of-categories-characterisations} of \cref{categories:properties-of-monomorphisms-of-categories}}{\cref{properties-of-monomorphisms-of-categories-characterisations} of \cref{properties-of-monomorphisms-of-categories}}.
        \item Epimorphisms of categories are surjective on objects: \ChapterRef{\ChapterCategories, \cref{categories:properties-of-epimorphisms-of-categories-surjectivity-on-objects} of \cref{categories:properties-of-epimorphisms-of-categories}}{\cref{properties-of-epimorphisms-of-categories-surjectivity-on-objects} of \cref{properties-of-epimorphisms-of-categories}}.
        \item Properties of pseudoepic functors: \ChapterRef{\ChapterCategories, \cref{categories:properties-of-pseudoepic-functors}}{\cref{properties-of-pseudoepic-functors}}.
    \end{enumerate}
\end{remark}
\subsection{Things To Explore/Add}\label{subsection-things-to-explore}
Here we list things to be explored/added to this work in the future.
\begin{remark}{Things To Explore/Add}{things-to-explore-add}%
    Set theory through a category theory lens:
    \begin{enumerate}
        \item Isbell duality for sets.
        \item Density comonads and codensity monads for sets.
    \end{enumerate}
    Relations:
    \begin{enumerate}
        \item 2-Categorical monomorphisms and epimorphisms in $\sfbfRel$.
        \item Co/limits in $\sfbfRel$.
        \item Apartness composition, categorical properties of $\sfbfRel$ with apartness, and apartness relations.
        \item Apartness defines a composition for relations, but its analogue
            \[
                \mathfrak{q}\mathbin{\square}\mathfrak{p}%
                \defeq%
                \int_{A\in\CatFont{C}}\mathfrak{p}^{-_{1}}_{A}\icoprod\mathfrak{q}^{A}_{-_{2}}%
            \]%
            fails to be unital for profunctors. Is there a less obvious analogue of apartness composition for profunctors?
        \item Codensity monad $\Ran_{J}(J)$ of a relation (What about $\Rift_{J}(J)$?)
        \item Relative comonads in the $2$-category of relations
        \item Discrete fibrations and Street fibrations in $\sfbfRel$.
        \item Consider adding the sections
            \begin{itemize}
                \item The Monoidal Bicategory of Relations
                \item The Monoidal Double Category of Relations
            \end{itemize}
            to \ChapterRelations.
    \end{enumerate}
    Spans:
    \begin{enumerate}
        \item Universal property of the bicategory of spans, \url{https://ncatlab.org/nlab/show/span}
        \item Write about cospans.
    \end{enumerate}
    Un/Straightening:
    \begin{enumerate}
        \item Write proper sections on straightening for lax functors from sets to Rel or Span (displayed sets)
    \end{enumerate}
    Categories:
    \begin{enumerate}
        \item Expand \cref{construction-of-the-groupoid-completion-of-a-category} and add a proof to it.
        \item Sections and retractions; retracts, \url{https://ncatlab.org/nlab/show/retract}.
        \item Regular categories: \url{https://arxiv.org/pdf/2004.08964.pdf}.
        \item Are pseudoepic functors those functors whose restricted Yoneda embedding is pseudomonic and Yoneda preserves absolute colimits?
        \item Absolutely dense functors enriched over $\R^{+}$ apparently reduce to topological density
    \end{enumerate}
    Types of Morphisms in Categories:
    \begin{enumerate}
        \item Behaviour in $\Fun(\CatFont{C},\CatFont{D})$, e.g.\ pointwise sections vs.\ sections in $\Fun(\CatFont{C},\CatFont{D})$.
        \item A faithful functor from balanced category is conservative
    \end{enumerate}
    Yoneda stuff:
    \begin{enumerate}
        \item Properties of restricted Yoneda embedding, e.g.\ if the restricted Yoneda embedding is full, then what can we conclude? Related: \url{https://qchu.wordpress.com/2015/05/17/generators/}
    \end{enumerate}
    Adjunctions:
    \begin{enumerate}
        \item Adjunctions, units, counits, and fully faithfulness as in \url{https://mathoverflow.net/questions/100808/properties-of-functors-and-their-adjoints}.
        \item Morphisms between adjunctions and bicategory $\Adj(\CatFont{C})$.
        \item \url{https://ncatlab.org/nlab/show/transformation+of+adjoints}
    \end{enumerate}
    Constructions With Categories:
    \begin{enumerate}
        \item Comparison between pseudopullbacks and isocomma categories: the \say{evident} functor $\CatFont{C}\times^{\sfps}_{\CatFont{E}}\CatFont{D}\to\CatFont{C}\isocomma_{\CatFont{E}}\CatFont{D}$ is essentially surjective and full, but not faithful in general.
    \end{enumerate}
    Co/limits:
    \begin{enumerate}
        \item Add the characterisations of absolutely dense functors given in \cref{https://ncatlab.org/nlab/show/absolutely+dense+functor} to \cref{properties-of-fully-faithful-functors-interaction-with-precomposition-4}.
        \item Absolutely dense functors, \url{https://ncatlab.org/nlab/show/absolutely+dense+functor}. Also theorem 1.1 here: \url{http://www.tac.mta.ca/tac/volumes/8/n20/n20.pdf}.
        \item Dense functors, codense functors, and absolutely codense functors.
    \end{enumerate}
    Co/ends:
    \begin{enumerate}
        \item Examples of co/ends: \url{https://mathoverflow.net/a/461814}
        \item Cofinality for co/ends, \url{https://mathoverflow.net/questions/353876}
    \end{enumerate}
    Fibred category theory:
    \begin{enumerate}
        \item Internal $\eHom$ in categories of co/Cartesian fibrations.
        \item \textit{Tensor structures on fibered categories} by Luca Terenzi: \url{https://arxiv.org/abs/2401.13491}. Check also the other papers by Luca Terenzi.
        \item \url{https://ncatlab.org/nlab/show/cartesian+natural+transformation} (this is a cartesian morphism in $\Fun(\CatFont{C},\CatFont{D})$ apparently)
        \item CoCartesian fibration classifying $\Fun(F,G)$, \url{https://mathoverflow.net/questions/457533/cocartesian-fibration-classifying-mathrmfunf-g}
    \end{enumerate}
    Monoidal categories:
    \begin{enumerate}
        \item Free braided monoidal category with a braided monoid: \url{https://ncatlab.org/nlab/show/vine}
    \end{enumerate}
    Skew monoidal categories:
    \begin{enumerate}
        \item Does the $\E_{1}$ tensor product of monoids admit a skew monoidal category structure?
        \item Is there a (right?) skew monoidal category structure on $\Fun(\CatFont{C},\CatFont{D})$ using right Kan extensions instead of left Kan extensions?
        \item Similarly, are there skew monoidal category structures on the subcategory of $\eRel(A,B)$ spanned by the functions using left Kan extensions and left Kan lifts?
    \end{enumerate}
    Higher categories:
    \begin{enumerate}
        \item Internal adjunctions in $\Mod$ as in \cite[Section 6.3]{2-categories-book}; see \cite[Example 6.2.6]{2-categories-book}.
        \item Comonads in the bicategory of profunctors.
    \end{enumerate}
    Monoids:
    \begin{enumerate}
        \item Isbell's zigzag theorem for semigroups: the following conditions are equivalent:
            \begin{enumerate}
                \item A morphism $f\colon A\to B$ of semigroups is an epimorphism.
                \item For each $b\in B$, one of the following conditions is satisfied:
                    \begin{itemize}
                        \item We have $f(a)=b$.
                        \item There exist some $m\in\N_{\geq1}$ and two factorisations
                            \begin{align*}
                                b &= a_{0}y_{1},\\
                                b &= x_{m}a_{2m}
                            \end{align*}
                            connected by relations
                            \begin{align*}
                                a_{0}         = x_{1}a_{1},\\
                                a_{1}y_{1}    = a_{2}y_{2},\\
                                x_{1}a_{2}    = x_{2}a_{3},\\
                                a_{2m-1}y_{m} = a_{2m}
                            \end{align*}
                            such that, for each $1\leq i\leq m$, we have $a_{i}\in\Im(f)$.
                    \end{itemize}
            \end{enumerate}
            Wikipedia says in \url{https://en.wikipedia.org/wiki/Isbell\%27s\_zigzag\_theorem}:
            \begin{quote}
                For monoids, this theorem can be written more concisely:
            \end{quote}
    \end{enumerate}
    Types of morphisms in bicategories:
    \begin{enumerate}
        \item Behaviour in 2-categories of pseudofunctors (or lax functors, etc.), e.g.\ pointwise pseudoepic morphisms in vs.\ pseudoepic morphisms in 2-categories of pseudofunctors.
        \item Statements like \say{coequifiers are lax epimorphisms}, Item 2 of Examples 2.4 of \url{https://arxiv.org/abs/2109.09836}, along with most of the other statements/examples there.
        \item Dense, absolutely dense, etc.\ morphisms in bicategories
    \end{enumerate}
    Other:
    \begin{enumerate}
        \item \url{https://qchu.wordpress.com/}
        \item \url{https://aroundtoposes.com/}
        \item \url{https://ncatlab.org/nlab/show/essentially+surjective+and+full+functor}
        \item \url{https://mathoverflow.net/questions/415363/objects-whose-representable-presheaf-is-a-fibration}
        \item \url{https://mathoverflow.net/questions/460146/universal-property-of-isbell-duality}
        \item \url{http://www.tac.mta.ca/tac/volumes/36/12/36-12abs.html} ( Isbell conjugacy and the reflexive completion )
        \item \url{https://ncatlab.org/nlab/show/enrichment+versus+internalisation}
        \item The works of Philip Saville, \url{https://philipsaville.co.uk/}
        \item \url{https://golem.ph.utexas.edu/category/2024/02/from\_cartesian\_to\_symmetric\_mo.html}
        \item \url{https://mathoverflow.net/q/463855} (One-object lax transformations)
        \item \url{https://ncatlab.org/nlab/show/analytic+completion+of+a+ring}
        \item \url{https://en.wikipedia.org/wiki/Quaternionic\_analysis}
        \item \url{https://arxiv.org/abs/2401.15051} (The Norm Functor over Schemes)
        \item \url{https://mathoverflow.net/questions/407291/} (Adjunctions with respect to profunctors)
        \item \url{https://mathoverflow.net/a/462726} ($\Prof$ is free completion of $\Cats$ under right extensions)
        \item there's some cool stuff in \url{https://arxiv.org/abs/2312.00990} (Polynomial Functors: A Mathematical Theory of Interaction), e.g.\ on cofunctors.
        \item \url{https://ncatlab.org/nlab/show/adjoint+lifting+theorem}
        \item \url{https://ncatlab.org/nlab/show/Gabriel\%E2\%80\%93Ulmer+duality}
    \end{enumerate}
\end{remark}
\begin{appendices}
\begin{multicols}{2}[\section{Other Chapters}]
\noindent
\textbf{Sets}
\begin{enumerate}
\item \hyperref[sets:section-phantom]{Sets}
\item \hyperref[constructions-with-sets:section-phantom]{Constructions With Sets}
\item \hyperref[pointed-sets:section-phantom]{Pointed Sets}
\item \hyperref[tensor-products-of-pointed-sets:section-phantom]{Tensor Products of Pointed Sets}
\end{enumerate}
\textbf{Relations}
\begin{enumerate}
\setcounter{enumi}{4}
\item \hyperref[relations:section-phantom]{Relations}
\item \hyperref[constructions-with-relations:section-phantom]{Constructions With Relations}
\item \hyperref[equivalence-relations-and-apartness-relations:section-phantom]{Equivalence Relations and Apartness Relations}
\end{enumerate}
\textbf{Category Theory}
\begin{enumerate}
\setcounter{enumi}{7}
\item \hyperref[categories:section-phantom]{Categories}
\end{enumerate}
\textbf{Bicategories}
\begin{enumerate}
\setcounter{enumi}{8}
\item \hyperref[types-of-morphisms-in-bicategories:section-phantom]{Types of Morphisms in Bicategories}
\end{enumerate}
\end{multicols}

\end{appendices}
\end{document}
